\subsection{ARM}
\label{sec:hw_ARM}

\index{\idevices}
\index{Raspberry Pi}
\index{Xcode}
\index{LLVM}
\index{Keil}
\IFRU{Для экспериментов с процессором ARM, я выбрал два компилятора}
{For my experiments with ARM processors I chose two compilers}: 
\IFRU{популярный в embedded-среде}{popular in the embedded area} Keil Release 6/2013 
\IFRU{и среду разработки}{and} Apple Xcode 4.6.3 \IFRU{}{IDE} (\IFRU{с компилятором}{with} LLVM-GCC 4.2 \IFRU{}{compiler}), 
\IFRU{генерирующую код для ARM-совместимых процессоров и}{which produces code for ARM-compatible processors and} \IFRU{SoC}{SoCs}\footnote{system on chip} \In \idevices, 
\IFRU{планшетных компьютеров для Windows 8 и Windows RT}{Windows 8 and Window RT tables}\footnote{\url{http://en.wikipedia.org/wiki/List_of_Windows_8_and_RT_tablet_devices}} 
\IFRU{и таких устройствах как}{and also such devices as} Raspberry Pi.

\subsubsection{\NonOptimizingKeil + \ARMMode}

\IFRU{Для начала, скомпилируем наш пример в Keil}{Let's start by compiling our example in Keil}:

\begin{lstlisting}
armcc.exe --arm --c90 -O0 1.c 
\end{lstlisting}

\index{\IntelSyntax}
\IFRU{Компилятор \IT{armcc} генерирует листинг на ассемблере в формате Intel,}
{The \IT{armcc} compiler produces assembly listings in Intel-syntax} 
\IFRU{но он содержит некоторые высокоуровневые макросы связанные с ARM}
{but it has some high-level ARM-processor related macros}\footnote{
\IFRU{например, он показывает инструкции \PUSH/\POP отсутствующие в режиме ARM}{for example, ARM mode lacks 
\PUSH/\POP instructions}}, 
\IFRU{а нам важнее увидеть инструкции ``как есть'', так что посмотрим скомпилированный результат в \IDA}
{but it is more important for us to see instructions ``as is'' so let's see the compiled results in \IDA}.

\begin{lstlisting}[caption=\NonOptimizingKeil + \ARMMode + \IDA]
.text:00000000             main
.text:00000000 10 40 2D E9                 STMFD   SP!, {R4,LR}
.text:00000004 1E 0E 8F E2                 ADR     R0, aHelloWorld ; "hello, world"
.text:00000008 15 19 00 EB                 BL      __2printf
.text:0000000C 00 00 A0 E3                 MOV     R0, #0
.text:00000010 10 80 BD E8                 LDMFD   SP!, {R4,PC}

.text:000001EC 68 65 6C 6C+aHelloWorld     DCB "hello, world",0    ; DATA XREF: main+4
\end{lstlisting}

\index{ARM!\ARMMode}
\index{ARM!\ThumbMode}
\index{ARM!\ThumbTwoMode}
\IFRU{Вот чуть-чуть фактов о процессоре ARM, которые желательно знать}
{Here are a couple of ARM-related facts we should know in order to proceed}.
\IFRU{Процессор ARM имеет по крайней мере два основных режима: режим ARM и thumb}
{An ARM processor has at least two major modes: ARM mode and thumb}. 
\IFRU{В первом (ARM) режиме доступны все инструкции и каждая имеет размер 32 бита (или 4 байта)}
{In the first (ARM) mode all instructions are enabled and each has 32-bit (4 bytes) size}. 
\IFRU{Во втором режиме (thumb) каждая инструкция имеет размер 16 бит (или 2 байта)}
{In the second (thumb) mode each instruction has 16-bit (or 2 bytes) size}
\footnote{\IFRU{Кстати, инструкции фиксированного размера удобны тем, что всегда можно легко узнать адрес 
следующей (или предыдущей) инструкции. Эта особенность будет рассмотрена в секции о 
switch()~\ref{sec:SwitchARMLot}.}
{By the way, fixed-length instructions are handy in that way that one can calculate the next (or previous) 
instruction's address without effort. This feature will be discussed in switch()~\ref{sec:SwitchARMLot} section.}
}.
\IFRU{Режим thumb может выглядеть привлекательнее тем, что программа на нем может быть 1) компактнее; 2) эффективнее исполняться на микроконтроллере с 16-битной шиной данных}
{Thumb mode may look attractive because programs that use it may 1) be compact and
2) execute faster on microcontrollers that have a 16-bit memory datapath}. 
\IFRU{Но за всё нужно платить: в режиме thumb куда меньше возможностей процессора, например, возможен доступ только к 8-и регистрам процессора, и чтобы совершить некоторые действия, выполнимые в режиме ARM одной инструкцией, нужны несколько thumb-инструкций}
{Nothing comes free of charge, so, in thumb mode there is a reduced instruction set, 
only 8 registers are accessible and one needs several thumb instructions for doing some operations when in ARM mode you only need one}.
\IFRU{Начиная с ARMv7, имеется также поддержка инструкций thumb-2, это thumb расширенный до поддержки куда большего числа инструкций}
{Starting at ARMv7 the thumb-2 instruction set is also present. 
This is an extended thumb that supports a much larger instruction set}.
\IFRU{Распространено заблуждение что thumb-2 это смесь ARM и thumb. Это не верно. Просто thumb-2 был дополен до
более полной поддержки возможностей процессора, что теперь может легко конкурировать с режимом ARM.}
{There is a common misconception that thumb-2 is a mix of ARM and thumb. This is not correct. 
Rather, thumb-2 was extended to fully support processor features so that it could
compete with ARM mode.}
\IFRU{Программа для процессора ARM может представлять смесь процедур скомпилированных для обоих режимов}
{A program for ARM processor may be a mix of procedures compiled for both modes}.
\IFRU{Основное количество приложений для \idevices скомпилировано для набора инструкций thumb-2, потому что Xcode
делает так по умолчанию}
{The majority of \idevices applications are compiled for thumb-2 instructions set because Xcode does this by default}.

\IFRU{В вышеприведененном примере можно легко увидеть что каждая инструкция имеет размер 4 байта}
{In the example we can easily see that each instruction has a size of 4 bytes}.
\IFRU{Действительно, ведь мы же компилировали наш код для режима ARM а не thumb}
{Indeed, we compiled our code for ARM mode, not for thumb}.

\index{ARM!\Instructions!STMFD}
\index{ARM!\Instructions!POP}
\IFRU{Самая первая инструкция}{The very first instruction}, \TT{''STMFD SP!, \{R4,LR\}''}\footnote{\STMFDdesc}, 
\IFRU{работает как инструкция}{works as an x86} \PUSH \IFRU{в x86}{instruction},
\IFRU{записывает значения двух регистров}{writing the values of two registers}
(\Rfour \AndENRU \LR) \IFRU{в стек}{into the stack}.
\IFRU{Действительно, в выдаваемом листинге на ассемблере, компилятор \IT{armcc}, для упрощения, указывает здесь инструкцию}
{Indeed, in the output listing from the \IT{armcc} compiler, for the sake of simplification, 
actually shows the} \TT{''PUSH \{r4,lr\}''}\IFRU{}{ instruction}.
\IFRU{Но это не совсем точно, инструкция \PUSH доступна только в режиме thumb, поэтому,
во избежания путанницы, я предложил работать в \IDA}
{But it is not quite correct. \PUSH instruction are only available in thumb mode.
So, to make things less messy, I offered to work in \IDA}.

\IFRU{Итак, эта инструкция записывает значения регистров \Rfour и \LR по адресу в памяти, на который указывает регистр \SPwithfootnote, затем уменьшает \TT{SP}, 
чтобы он указывал на место в стеке, доступное для новых записей}
{This instruction writes the values of the \Rfour and \LR registers at the address in memory to which 
\SPwithfootnote is pointing, then it decrements \SP so it will point to a place in the stack that 
is free for new entries}.

\IFRU{Эта инструкция, как и инструкция \PUSH в режиме thumb, может сохранить в стеке одновременно несколько значений регистров, что может быть очень удобно}
{This instruction (like the \PUSH instruction in thumb mode) is able to save several register values at once and this may be useful}. 
\IFRU{Кстати, такого в x86 нет}{By the way, there is no such thing in x86}.
\IFRU{Так же следует заметить, что \TT{STMFD} ~--- генерализация инструкции \PUSH (то есть, расширяет её возможности), потому что может работать с любым регистром а не только с \SP, это тоже может быть очень удобно}
{It can also be noted that \TT{STMFD} ~--- generalization of \PUSH instruction (extending its features), since it can work with any register, not just with \SP and this can be very useful}.

\index{\PICcode}
\index{ARM!\Instructions!ADR}
\IFRU{Инструкция}{The} \TT{''ADR R0, aHelloWorld''}
\IFRU{прибавляет значение регистра \PC к смещению, где хранится строка}
{instruction adds the value in the \PC register to the offset where the}
\IT{``hello, world''} \IFRU{}{string is located}.
\IFRU{Причем здесь \PC, можно спросить}{How \TT{PC} register used here, one might ask}?
\IFRU{Притом, что это так называемый ``\PICcode''}{This is so called ``\PICcode''.}
\footnote{
	\IFRU{Читайте больше об этом в соответствующем разделе}
	{Read more about it in relevant section}~\ref{sec:PIC}
	},
\IFRU{он предназначен для исполнения будучи не привязанным к каким-либо адресам в памяти}
{It is intended to be executed at an unfixed addresses in memory}.
\IFRU{В опкоде инструкции \TT{ADR} указывается разница между адресом этой инструкции и местом, где хранится строка}
{In the opcode of the \TT{ADR} instruction is encoded the difference between the address of this instruction and the place where the string is located}.
\IFRU{Эта разница всегда будет постоянной, вне зависимости от того, куда был загружен \ac{OS}
наш код}
{The difference will always be the same, independant of the address where that code is loaded by 
\ac{OS}}.
\IFRU{Поэтому всё что нужно это прибавить адрес текущей инструкции (из \PC) чтобы получить текущий абсолютный адрес нашей Си-строки}
{That's why all we need is to add the address of the current instruction (from \PC) in order to get the absolute address of our C-string in memory}.

\index{ARM!\Registers!Link Register}
\index{ARM!\Instructions!BL}
\IFRU{Инструкция}{} \TT{''BL \_\_2printf''}\footnote{Branch with Link}
\IFRU{вызывает функцию \printf}{instruction calls the \printf function}. 
\IFRU{Работа этой инструкции состоит из двух фаз}{That's how this instruction works}: 
\begin{itemize}
\item
\IFRU{записать адрес после инструкции \TT{BL} (\TT{0xC}) в регистр \LR}
{write the address following the \TT{BL} instruction (\TT{0xC}) into the \LR};
\item
\IFRU{затем собственно передать управление в \printf, записав адрес этой функции в регистр \PCwithfootnote}
{then pass control flow into \printf by writing its address into the \PCwithfootnote register}.
\end{itemize}

\IFRU{Ведь, когда функция \printf закончит работу, нужно знать, куда вернуть управление, поэтому закончив работу, всякая функция передает управление по адресу записанному в регистре \LR}
{When \printf finishes its work it should have information about where it should return control.
That's why each function passes control to the address stored in the \LR register}.

\IFRU{В этом разница между ``чистыми'' RISC-процессорами вроде ARM и CISC-процессорами как x86,
где адрес возврата записывается в стек}
{That is the difference between ``pure'' RISC-processors like ARM and CISC-processors like x86,
where address of return is stored in stack}\footnote{\IFRU{Подробнее об этом будет описано в следующей главе}{Read more about this in next section}~\ref{sec:stack}}.

\IFRU{Кстати, 32-битный абсолютный адрес, либо же смещение, невозможно закодировать в 32-битной инструкции \TT{BL}, в ней есть место только для 24-х бит}
{By the way, an absolute 32-bit address or offset cannot be encoded in the 32-bit \TT{BL} instruction because
it only has space for 24 bits}.
\IFRU{Так же следует отметить, что из-за того что все инструкции в режиме ARM имеют длину 4 байта (32 бита), и инструкции могут находится только по адресам кратным 4, то последние 2 бита (всегда нулевых) можно не кодировать.}
{It is also worth noting that all ARM mode instructions have a size of 4 bytes (32 bits).
Hence they can only be located on 4-byte boundary addresses.
This means that the last 2 bits of the instruction address (always zero bits) may be omitted.}
\IFRU{В итоге имеем 26 бит, при помощи которых можно закодировать смещение}
{In summary, we have 26 bit for offset encoding. This is enough to represent offset} $\pm{}\approx{}32M$.

\index{ARM!\Instructions!MOV}
\IFRU{Следующая инструкция}{Next, the} \TT{''MOV R0, \#0''}\footnote{MOVe}
\IFRU{просто записывает $0$ в регистр \Rzero}{instruction just writes $0$ into the \Rzero register}.
\IFRU{Ведь наша Си-функция возвращает $0$ а возвращаемое значение всякая функция оставляет в \Rzero}
{That's because our C-function returns $0$ and the returned value is to be placed in the \Rzero register}.

\index{ARM!\Registers!Link Register}
\index{ARM!\Instructions!LDMFD}
\index{ARM!\Instructions!POP}
\IFRU{Последняя инструкция}{The last instruction} \TT{''LDMFD SP!, {R4,PC}''}\footnote{\LDMFDDESC} \IFRU{это инструкция обратная от}{is an inversive instruction of} \TT{STMFD}. 
\IFRU{Она загружает из стека значения для сохранения их в \Rfour и \PC, увеличивая указатель стека \SP}
{It loads values from the stack in order to save them into \Rfour and \PC, and incremets stack pointer \SP}.
\IFRU{Это, в каком-то смысле, аналог \POP}{It can be said it is similar to \POP}. 
\IFRU{Обратите внимание: самая первая инструкция \TT{STMFD} сохранила в стеке \Rfour и \LR, а \IT{восстанавливаются} во время исполнения \TT{LDMFD} регистры \Rfour и \PC}
{Note: the very first instruction \TT{STMFD} saves the \Rfour and \LR registers pair on the stack, but \Rfour and \PC are \IT{restored} during execution of \TT{LDMFD}}.
\IFRU{Как я уже описывал, в регистре \LR обычно сохраняется адрес места, куда нужно всякой функции вернуть управление}
{As I wrote before, address of place to where each function should return control is usually saved in the \LR
register}.
\IFRU{Самая первая инструкция сохраняет это значение в стеке, потому что наша функция \main позже будет сама пользоваться этим регистром, в момент вызова \printf}
{The very first function saves its value in the stack because our \main function will use that register in order to call \printf}.
\IFRU{А затем, в конце функции, это значение можно сразу записать в \PC, таким образом, передав управление туда, откуда была вызвана наша функция}
{In the function end this value can be written to the \PC register, thus passing control to where our function was called}.
\IFRU{Так как функция \main обычно самая главная в \CCpp, управление будет возвращено в загрузчик \ac{OS}, либо куда-то в runtime функции Си, или что-то в этом роде}
{Since our \main function is usually the primary function in \CCpp, control will be returned to \ac{OS} loader or to some point in runtime C functions, or something like that}.

\index{ARM!DCB}
\TT{DCB} ~--- \IFRU{директива ассемблера, описывающая массивы байт или ASCII-строк, аналог директивы DB в 
x86-ассемблере}
{assembly language directive defining array of bytes or ASCII-strings, similar to the DB directive 
in x86-assembly language}.

\subsubsection{\NonOptimizingKeil: \ThumbMode}

\IFRU{Скомпилируем тот же пример в Keil для режима thumb}{Let's compile the same example using Keil in thumb mode}:

\begin{lstlisting}
armcc.exe --thumb --c90 -O0 1.c 
\end{lstlisting}

\IFRU{Получим (в \IDA)}{We will get (in \IDA)}:

\begin{lstlisting}[caption=\NonOptimizingKeil + \ThumbMode + \IDA]
.text:00000000             main
.text:00000000 10 B5                       PUSH    {R4,LR}
.text:00000002 C0 A0                       ADR     R0, aHelloWorld ; "hello, world"
.text:00000004 06 F0 2E F9                 BL      __2printf
.text:00000008 00 20                       MOVS    R0, #0
.text:0000000A 10 BD                       POP     {R4,PC}

.text:00000304 68 65 6C 6C+aHelloWorld     DCB "hello, world",0    ; DATA XREF: main+2
\end{lstlisting}

\IFRU{Сразу бросаются в глаза двухбайтные (16-битные) опкоды, это, как я уже упоминал, thumb}
{We can easily spot the 2-byte (16-bit) opcodes. This is, as I mentioned, thumb}.
\index{ARM!\Instructions!BL}
\IFRU{Кроме инструкции \TT{BL}.}{Except for the \TT{BL} instruction}
\IFRU{Но на самом деле, она состоит из двух 16-битных инструкций}
{which consists of two 16-bit instructions}.
\IFRU{Это потому что загрузить в \PC смещение, по которому находится функция \printf, используя так мало места в одном 16-битном опкоде, нельзя}
{This is because it is not possible to load an offset for the \printf function into \PC while using so small of a space as one 16-bit opcode}.
\IFRU{Поэтому первая 16-битная инструкция загружает старшие 10 бит смещения, а вторая ~--- младшие 11 бит смещения}{That's why the first 16-bit instruction loads the higher 10 bits of the offset and the second instruction loads 
the lower 11 bits of the offset}.
\IFRU{Как я уже упоминал, все инструкции в thumb-режиме имеют длину 2 байта (или 16 бит)}
{As I mentioned, all instructions in thumb mode have a size of 2 bytes (or 16 bits)}.
\IFRU{Поэтому невозможна такая ситуация, когда thumb-инструкция начинается по нечетному адресу}
{This mean it is not possible for a thumb-instruction to be on an odd address whatsoever}.
\IFRU{Учитывая сказанное, последний бит адреса можно не кодировать}
{Given the above, the last address bit may be omitted while encoding instructions}.
\IFRU{Таким образом, в итоге, в thumb-инструкции \TT{BL} кодируется смещение}
{Summarizing, in the \TT{BL} thumb-instruction} $\pm{}\approx{}2M$ 
\IFRU{от текущего адреса}{can be encoded as the offset from the current address}.

\IFRU{Остальные инструкции в функции: \PUSH и \POP работают почти так же как и описанные \TT{STMFD}/\TT{LDMFD}, только регистр \SP здесь не указывается явно}
{Other instructions in functions are: \PUSH and \POP work just like the described \TT{STMFD}/\TT{LDMFD} but the \SP register is not mentioned explicitely here}.
\TT{ADR} \IFRU{работает также как и в предыдущем примере}{works just like in previous example}.
\TT{MOVS} \IFRU{записывает $0$ в регистр \Rzero для возврата нуля}
{writes $0$ into the \Rzero register in order to return zero value}.

\subsubsection{\OptimizingXcode + \ARMMode}

Xcode 4.6.3 \IFRU{без включенной оптимизации выдает слишком много лишнего кода, поэтому остановимся на той версии, где как можно меньше инструкций}
{without optimization turned on produces a lot of redundant code so we'll study the version where the instruction count is as small as possible}: \Othree.

\begin{lstlisting}[caption=\OptimizingXcode + \ARMMode]
__text:000028C4             _hello_world
__text:000028C4 80 40 2D E9                 STMFD           SP!, {R7,LR}
__text:000028C8 86 06 01 E3                 MOV             R0, #0x1686
__text:000028CC 0D 70 A0 E1                 MOV             R7, SP
__text:000028D0 00 00 40 E3                 MOVT            R0, #0
__text:000028D4 00 00 8F E0                 ADD             R0, PC, R0
__text:000028D8 C3 05 00 EB                 BL              _puts
__text:000028DC 00 00 A0 E3                 MOV             R0, #0
__text:000028E0 80 80 BD E8                 LDMFD           SP!, {R7,PC}

__cstring:00003F62 48 65 6C 6C+aHelloWorld_0   DCB "Hello world!",0
\end{lstlisting}

\IFRU{Инструкции}{The instructions} \TT{STMFD} \AndENRU \TT{LDMFD} \IFRU{нам уже знакомы}{are familiar to us}.

\IFRU{Инструкция \MOV просто записывает число \TT{0x1686} в регистр \Rzero, это смещение указывающее на строку ``Hello world!''}
{The \MOV instruction just writes the number \TT{0x1686} into the \Rzero register.
This is the offset pointing to the ``Hello world!'' string}.

\IFRU{Регистр \TT{R7}, по стандарту принятому в}{The \TT{R7} register as it is standardized in}\cite{IOSABI}
\IFRU{это}{is a} frame pointer\IFRU{ , о нем будет рассказано позже.}{. More on it below.}

\index{ARM!\Instructions!MOVT}
\IFRU{Инструкция}{The} \TT{MOVT R0, \#0} \IFRU{записывает $0$ в старшие 16 бит регистра}
{instruction writes $0$ into higher 16 bits of the register}.
\IFRU{Дело в том, что обычная инструкция \MOV в режиме ARM может записывать какое-либо значение только в младшие 16 бит регистра, ведь, больше нельзя закодировать в ней}
{The issue here is that the generic \MOV instruction in ARM mode may write only the lower 16 bits of the register}.
\IFRU{Помните, что в режиме ARM опкоды всех инструкций ограничены длиной в 32 бита. Конечно, это ограничение не касается перемещений между регистрами.}
{Remember, all instructions' opcodes in ARM mode are limited in size to 32 bits. Of course, this limitation is not related to moving between registers.}
\IFRU{Поэтому для записи в старшие биты (от 16-го по 31-го включительно) существует дополнительная команда \TT{MOVT}}
{That's why an additional instruction \TT{MOVT} exists for writing into the higher bits (from 16 to 31 inclusive)}.
\IFRU{Впрочем, здесь её использование избыточно, потому что инструкция \TT{''MOV R0, \#0x1686''} выше итак обнулила старшую часть регистра}
{However, its usage here is redundant because the \TT{''MOV R0, \#0x1686''} instruction above cleared the higher part of the register}.
\IFRU{Возможно, это недочет компилятора}{It is probably the compiler's shortcoming}.

\index{ARM!\Instructions!ADD}
\IFRU{Инструкция}{The} \TT{''ADD R0, PC, R0''} \IFRU{прибавляет \PC к \Rzero, для вычисления действительного адреса строки ``Hello world!'', как нам уже известно, это ``\PICcode'', поэтому такая корректива необходима}
{instruction adds value in the \PC to the value in the \Rzero, to calculate absolute address of the ``Hello world!'' string. 
And as we already know, it is ``\PICcode'' so this corrective is essential here}.

\IFRU{Инструкция \TT{BL} вызывает \puts вместо \printf}
{The \TT{BL} instruction calls the \puts function instead of \printf}.

\label{puts}
\index{puts() \IFRU{вместо}{instead of} printf()}
\IFRU{Компилятор заменил вызов \printf на \puts. 
Действительно, \printf с одним агрументом это почти аналог \puts.}
{GCC replaced the first \printf call with \puts.
Indeed: \printf with a sole argument is almost analogous to \puts.} 

\IFRU{\IT{Почти}, если принять условие что в строке не будет управляющих символов \printf 
начинающихся со знака процента. Тогда эффект от работы этих двух функций будет разным}
{\IT{Almost} because we need to be sure that this string will not contain printf-control 
statements starting with \IT{\%}: then the effect of these two functions would be different}
\footnote{
\IFRU{Также нужно заметить, что \puts не требует символа перевода строки '\textbackslash{}n' в конце строки,
поэтому его здесь нет.}
{It should also be noted that \puts does not require a '\textbackslash{}n' new line symbol 
at the end of a string, so we do not see it here.}}.

\IFRU{Зачем компилятор заменил один вызов на другой? Потому что \puts() работает быстрее}
{Why did the compiler replace the \printf with \puts? Because \puts is faster}
\footnote{\url{http://www.ciselant.de/projects/gcc_printf/gcc_printf.html}}. 

\IFRU{Видимо потому, что \puts проталкивает символы в stdout не сравнивая каждый со знаком процента.}
{\puts works faster because it just passes characters to stdout without comparing each to the \IT{\%} symbol.}

\IFRU{Далее уже знакомая инструкция}{Next, we see the familiar} 
\TT{''MOV R0, \#0''}\IFRU{, служащая для установки в $0$ возвращаемого значения функции}
{instruction intended to set $0$ to the \Rzero register}.

\subsubsection{\OptimizingXcode + \ThumbTwoMode}

\IFRU{По умолчанию,}{By default} Xcode 4.6.3 
\IFRU{генерирует код для режима thumb-2, примерно в такой манере}
{generates code for thumb-2 in this manner}:

\begin{lstlisting}[caption=\OptimizingXcode + \ThumbTwoMode]
__text:00002B6C                   _hello_world
__text:00002B6C 80 B5                             PUSH            {R7,LR}
__text:00002B6E 41 F2 D8 30                       MOVW            R0, #0x13D8
__text:00002B72 6F 46                             MOV             R7, SP
__text:00002B74 C0 F2 00 00                       MOVT.W          R0, #0
__text:00002B78 78 44                             ADD             R0, PC
__text:00002B7A 01 F0 38 EA                       BLX             _puts
__text:00002B7E 00 20                             MOVS            R0, #0
__text:00002B80 80 BD                             POP             {R7,PC}

...

__cstring:00003E70 48 65 6C 6C 6F 20+aHelloWorld     DCB "Hello world!",0xA,0
\end{lstlisting}

\index{ThumbTwoMode}
\IFRU{Инструкции \TT{BL} и \TT{BLX} в thumb, как мы помним, кодируются как пара 16-битных инструкций, 
а в thumb-2 эти \IT{суррогатные} опкоды расширены так, что новые инструкции кодируются здесь как 
32-битные инструкции}
{The \TT{BL} and \TT{BLX} instructions in thumb mode, as we remember, are encoded as pair
of 16-bit instructions. In thumb-2 these \IT{surrogate} opcodes are extended in such way so that new instructions
may be encoded here as 32-bit instructions}.
\IFRU{Это можно заметить по тому что опкоды thumb-2 инструкций всегда начинаются с \TT{0xFx} либо с \TT{0xEx}}
{That's
easily observable ~--- opcodes of thumb-2 instructions also begin with \TT{0xFx} or \TT{0xEx}}.
\IFRU{Но в листинге \IDA байты опкода переставлены местами,
это из-за того что в процессоре ARM инструкции кодируются так:
в начале последний байт, потом первый (для thumb и thumb-2 режима), либо, 
(для инструкций в режиме ARM) в начале четвертый байт, затем третий, второй и первый (т.е., другой endianness
\footnote{порядок байт})}
{But in the \IDA listings
two opcode bytes are swapped (for thumb and thumb-2 modes), or fourth byte, then third,
then second and first for instructions in ARM mode (different endianness)}.
\index{ARM!\Instructions!MOVW}
\index{ARM!\Instructions!MOVT.W}
\index{ARM!\Instructions!BLX}
\IFRU{Так что мы видим здесь что инструкции \TT{MOVW}, \TT{MOVT.W} и \TT{BLX} начинаются с}
{So as we see, \TT{MOVW}, \TT{MOVT.W} and \TT{BLX} instructions begin with} \TT{0xFx}.

\IFRU{Одна из thumb-2 инструкций это}{One of the thumb-2 instructions is}
\TT{``MOVW R0, \#0x13D8''} ~--- 
\IFRU{она записывает 16-битное число в младшую часть регистра \Rzero}
{it writes a 16-bit value into the lower part of the \Rzero register}.

\IFRU{Еще}{Also,} \TT{``MOVT.W R0, \#0''} ~--- 
\IFRU{эта инструкция работает так же как и}{this instruction works just like} 
\TT{MOVT} \IFRU{из предыдущего примера, но она работает в}{from previous example but it works in} thumb-2.

\index{ARM!\IFRU{переключение режимов}{mode switching}}
\index{ARM!\Instructions!BLX}
\IFRU{Помимо прочих отличий, здесь используется инструкция}{Among other differences, here} \TT{BLX} \IFRU{вместо}{instruction is used instead of} \TT{BL}.
\IFRU{Отличие в том, что помимо сохранения адреса возврата в регистре \LR и передаче управления в функцию \puts, происходит смена режима процессора с thumb на ARM, либо наоборот}
{The difference in that way that beside saving of \ac{RA} in the \LR register and passing control to the \puts function, processor is also switching from thumb mode to ARM (or back)}.
\IFRU{Здесь это нужно потому что инструкция, куда ведет переход, выглядит так (она закодирована в режиме ARM)}{This instruction in place here since the instruction to which control is passed looks like (it is encoded in ARM mode)}:

\begin{lstlisting}
__symbolstub1:00003FEC _puts           ; CODE XREF: _hello_world+E
__symbolstub1:00003FEC 44 F0 9F E5     LDR  PC, =__imp__puts
\end{lstlisting}

\IFRU{Итак, внимательный читатель может задать справделивый вопрос: почему бы не вызывать \puts сразу в 
том же месте кода, где он нужен?}
{So, observant reader may ask: why not to call \puts right at the point of code where it needed?}

\IFRU{Но это не очень выгодно (в плане экономия места) и вот почему}{Because that is not very space-efficient, that is why}.

\index{\IFRU{Динамически подгружаемые библиотеки}{Dynamically loaded libraries}}
\IFRU{Практически любая программа использует внешние динамические библиотеки (будь то DLL в Windows, .so в *NIX 
либо .dylib в Mac OS X)}{Almost any program uses external dynamic libraries (like DLL in Windows, .so in *NIX or .dylib in Mac OS X)}.
\IFRU{В динамических библиотеках находятся часто используемые библиотечные функции, в том числе стандартная функция Си \puts}
{Often used library functions are stored in dynamic libraries, including the standard C-function \puts}.

\index{Relocation}
\IFRU{В исполняемом бинарном файле}{In an executable binary file} 
(Windows PE .exe, ELF \IFRU{либо}{or} Mach-O) \IFRU{имеется секция импортов, список символов (функций либо глобальных переменных) импортируемых из внешних модулей, а также названия самих модулей}
{a section of imports is present. That is a list of symbols (functions or global variables) being imported from external modules along with the names of these modules}.

\IFRU{Загрузчик \ac{OS} загружает необходимые модули и, перебирая импортируемые символы в основном модуле, проставляет правильные адреса каждого символа}
{The \ac{OS} loader loads all modules it needs and, while enumerating importing symbols in the primary module, sets the correct addresses of each symbol}.

\IFRU{В нашем случае,}{In our case} \IT{\_\_imp\_\_puts} 
\IFRU{это 32-битная переменная, куда загрузчик \ac{OS} запишет правильный адрес этой же функции во внешней библиотеке}
{is a 32-bit variable where the \ac{OS} loader will write the correct address of that function in an external library}. 
\IFRU{Так что инструкция \TT{LDR} просто берет 32-битное значение из этой переменной и, записывая его в регистр \PC, просто передает туда управление}
{Then the \TT{LDR} instruction just takes the 32-bit value from this variable and writes it into the \PC register, passing control to it}.

\IFRU{Чтобы уменьшить время работы загрузчика \ac{OS}, нужно чтобы ему пришлось записать адрес каждого символа только один раз, в соответствующее выделенное для них место}
{So to reduce the time an \ac{OS} loader needs for doing this procedure it is good idea for it to write the address of each symbol only once to a specially allocated place just for it}.

\index{thunk-\IFRU{функции}{functions}}
\IFRU{К тому же, как мы уже убедились, нельзя одной инструкцией загрузить в регистр 32-битное число без обращений к памяти}
{Besides, as we have already figured out, it is not possible to load a 32-bit value into a register 
while using only one instruction without memory access}.
\IFRU{Так что, наиболее оптимально, выделить отдельную функцию, работающую в режиме ARM, 
чья единственная цель ~--- передавать управление дальше, в динамическую библиотеку.}
{So, it is optimal to allocate a separate function working in ARM mode with only one goal ~--- 
to pass control to the dynamic library}
\IFRU{И затем ссылаться на эту короткую функцию из одной инструкции (так называемую thunk-функцию) из thumb-кода}
{and then to jump to this short one-instruction function (the so called thunk-function) from thumb-code}.

\index{ARM!\Instructions!BL}
\IFRU{Кстати, в предыдущем примере (скомпилированном для режима ARM), переход при помощи инструкции \TT{BL} ведет 
на такую же thunk-функцию, однако режим процессора не переключается (отсюда, отсутствие ``X'' в мнемонике инструкции)}
{By the way, in the previous example (compiled for ARM mode) control passed by the \TT{BL} instruction goes to the same thunk-function.
However thr processor mode is not switched (hence the absence of an ``X'' in the instruction mnemonic)}.

