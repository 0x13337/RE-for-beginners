\subsection{x86}

\subsubsection{MSVC}

\IFRU{Компилируем в}{Let's compile it in} MSVC 2010:

\begin{lstlisting}
cl 1.cpp /Fa1.asm
\end{lstlisting}

\IFRU
{(Ключ /Fa означает сгенерировать листинг на ассемблере)}
{(/Fa option means generate assembly listing file)}

\begin{lstlisting}[caption=MSVC 2010]
CONST	SEGMENT
$SG3830	DB	'hello, world', 00H
CONST	ENDS
PUBLIC	_main
EXTRN	_printf:PROC
; Function compile flags: /Odtp
_TEXT	SEGMENT
_main	PROC
	push	ebp
	mov	ebp, esp
	push	OFFSET $SG3830
	call	_printf
	add	esp, 4
	xor	eax, eax
	pop	ebp
	ret	0
_main	ENDP
_TEXT	ENDS
\end{lstlisting}

\IFRU{MSVC выдает листинки в Intel-овском синтаксисе.}{MSVC produces assembly listings in Intel-syntax.} 
\IFRU{Разница между Intel-синтаксисом и AT\&T будет рассмотрена немного позже.}
{The difference between Intel-syntax and AT\&T-syntax will be discussed below.}

\IFRU{Компилятор сгенерировал файл \TT{1.obj}, который впоследствии будет слинкован линкером в \TT{1.exe}.} 
{The compiler generated \TT{1.obj} file will be linked into \TT{1.exe}.}

\IFRU{В нашем случае, этот файл состоит из двух сегментов: \TT{CONST} (для данных-констант) и \TT{\_TEXT} (для кода).}
{In our case, the file contain two segments: \TT{CONST} (for data constants) and \TT{\_TEXT} (for code).} 

\index{\CLanguageElements!const}
\IFRU{Строка \TT{``hello, world''} в \CCpp имеет тип \TT{const char*}, однако не имеет имени.}
{The string \TT{``hello, world''} in \CCpp has type \TT{const char*}, however it does not have
its own name.}

\IFRU{Но компилятору нужно как-то с ней работать, так что он дает ей внутреннее имя \TT{\$SG3830}.}
{The compiler needs to work with the string somehow so it defines the internal name \TT{\$SG3830} for it.}

\IFRU{Как видно, строка заканчивается нулевым байтом ~--- это требования стандарта \CCpp для строк.}
{As we can see, the string is terminated by a zero byte which is standard for \CCpp strings.}

\IFRU{В сегменте кода \TT{\_TEXT}, находится пока только одна функция}
{In the code segment, \TT{\_TEXT}, there is only one function so far}: \main.

\IFRU{Функция \main, как и практически все функции, начинается с пролога и заканчивается эпилогом}
{The function \main starts with prologue code and ends with epilogue code (like almost any function)}
\footnote{\IFRU{Об этом смотрите подробнее в разделе о прологе и эпилоге функции}
{Read more about it in section about function prolog and epilog}
~(\ref{sec:prologepilog}).}.

\index{x86!\Instructions!CALL}
\IFRU{Далее следует вызов функции \printf}
{After the function prologue we see the call to the \printf function}: \TT{CALL \_printf}. 

\index{x86!\Instructions!PUSH}
\IFRU
{Перед этим вызовом, адрес строки (или указатель на нее) с нашим приветствием при помощи инструкции \PUSH помещается в стек.}
{Before the call the string address (or a pointer to it) containing our greeting is placed on the stack with the help of the \PUSH instruction.}

\IFRU{После того как функция \printf возвращает управление в функцию \main, адрес строки (или указатель на нее) все еще лежит в стеке.}
{When the \printf function returns flow control to the \main function, string address (or pointer to it) is still in stack.}

\IFRU{Так как он больше не нужен, то \glslink{stack pointer}{указатель стека} (регистр \ESP) корректируется.} 
{Since we do not need it anymore the \gls{stack pointer} (the \ESP register) needs to be corrected.}

\index{x86!\Instructions!ADD}
\TT{ADD ESP, 4} \IFRU{означает прибавить 4 к значению в регистре \ESP.}
{means add 4 to the value in the \ESP register.}

\IFRU
{Почему 4? Так как, это 32-битный код, для передачи адреса нужно аккурат 4 байта. В x64-коде это 8 байт.}
{Why 4? Since it is 32-bit code we need exactly 4 bytes for address passing through the stack. 
It is 8 bytes in x64-code.}

\TT{``ADD ESP, 4''} \IFRU{эквивалентно \TT{``POP регистр''}, но без использования какого-либо регистра\footnote{Флаги
процессора, впрочем, модифицируются}.}
{is effectively equivalent to \TT{``POP register''} but without using any register\footnote{CPU flags, however, are modified}.}

\index{Intel C++}
\index{Oracle RDBMS}
\index{x86!\Instructions!POP}
\IFRU{Некоторые компиляторы, например Intel C++ Compiler, в этой же ситуации, могут вместо 
\ADD сгенерировать \TT{POP ECX} (подобное можно встретить например в коде \oracle{}, им скомпилированном),
что почти то же самое, только портится значение в регистре \ECX.}
{Some compilers (like Intel C++ Compiler) in the same situation may emit \TT{POP ECX} 
instead of \ADD (e.g. such a pattern can be observed in the \oracle{} code as it is compiled by Intel C++ compiler).
This instruction has almost the same effect but the \ECX register contents will be rewritten.}

\IFRU
{Возможно, компилятор применяет \TT{POP ECX} потому что эта инструкция короче (1 байт против 3).}
{The Intel C++ compiler probably uses \TT{POP ECX} since this instruction's opcode is shorter then 
\TT{ADD ESP, x} (1 byte against 3).}

\IFRU{О стеке можно прочитать в соответствующем разделе}
{Read more about the stack in section}~(\ref{sec:stack}).

\index{\CLanguageElements!return}
\IFRU{После вызова \printf, в оригинальном коде на \CCpp указано \TT{return 0} ~--- вернуть $0$ 
в качестве результата функции \main.} 
{After the call to \printf, in the original \CCpp code was \TT{return 0} ~--- 
return $0$ as the result of the \main function.}

\index{x86!\Instructions!XOR}
\IFRU{В сгенерированном коде это обеспечивается инструкцией}
{In the generated code this is implemented by instruction} \TT{XOR EAX, EAX} 

\index{x86!\Instructions!MOV}
\IFRU{\XOR, на самом деле, как легко догадаться, ``исключающее ИЛИ''}
{\XOR is in fact, just ``eXclusive OR''}
\footnote{\url{http://en.wikipedia.org/wiki/Exclusive_or}}
\IFRU{, но компиляторы часто используют его вместо простого}
{but compilers often use it instead of}
\TT{MOV EAX, 0} ~--- 
\IFRU
{снова потому что опкод короче (2 байта против 5).}
{again because it is a slightly shorter opcode (2 bytes against 5).}

\index{x86!\Instructions!SUB}
\IFRU{Бывает так, что некоторые компиляторы генерируют}{Some compilers emit}
\TT{SUB EAX, EAX}, 
\IFRU
{что значит, \IT{отнять значение в \EAX от значения в \EAX}, что в любом случае даст 0 в результате.}
{which means \IT{SUBtract the value in the \EAX from the value in \EAX}, which in any case will result zero.}

\index{x86!\Instructions!RET}
\IFRU{Самая последняя инструкция \RET возвращает управление в вызывающую функцию.
Обычно, это код \CCpp \ac{CRT}, который, в свою очередь, 
вернет управление операционной системе.}
{The last instruction \RET returns control flow to the \gls{caller}.
Usually, it is \CCpp \ac{CRT} code which in turn returns control to the \ac{OS}.}

\subsubsection{GCC}

\IFRU{Теперь скомпилируем то же самое компилятором GCC 4.4.1 в Linux}
{Now let's try to compile the same \CCpp code in the GCC 4.4.1 compiler in Linux}: \TT{gcc 1.c -o 1}

\IFRU{Затем при помощи \IDA. посмотрим как создалась функция \main.}
{After, with the assistance of the \IDA disassembler, let's see how the \main function was created.} 

(\IDA, \IFRU{как и MSVC, показывает код в Intel-синтаксисе}{as MSVC, showing code in Intel-syntax}).

N.B. \IFRU{Мы также можем заставить GCC генерировать листинги в этом формате при помощи ключей}
{We could also have GCC produce assembly listings in Intel-syntax by applying the options} 
\TT{-S -masm=intel}

\begin{lstlisting}[caption=GCC]
main            proc near

var_10          = dword ptr -10h

                push    ebp
                mov     ebp, esp
                and     esp, 0FFFFFFF0h
                sub     esp, 10h
                mov     eax, offset aHelloWorld ; "hello, world"
                mov     [esp+10h+var_10], eax
                call    _printf
                mov     eax, 0
                leave
                retn
main            endp
\end{lstlisting}

\index{Function prologue}
\index{x86!\Instructions!AND}
\IFRU{Почти то же самое. 
Адрес строки ``hello, world'' лежащей в сегменте данных, в начале сохраняется в \EAX, затем записывается в стек.
А еще в прологе функции мы видим \TT{AND ESP, 0FFFFFFF0h} ~--- 
эта инструкция выравнивает значение в \ESP по 16-байтной границе, делая все значения 
в стеке также выровненными по этой границе (процессор более эффективно работает с переменными расположенными
в памяти по адресам кратным 4 или 16)\footnote{\URLWPDA}.}
{The result is almost the same.
The address of the ``hello world'' string (stored in the data segment) is saved in the \EAX register first then it is stored on the stack.
Also in the function prologue we see \TT{AND ESP, 0FFFFFFF0h} ~--- 
this instruction aligns the value in the \ESP register on a 16-byte border.
This results in all values in the stack being aligned.
(CPU performs better if the values it is working with are located in memory at addresses aligned 
on a 4 or 16 byte border)\footnote{\URLWPDA}.}

\index{x86!\Instructions!SUB}
\TT{SUB ESP, 10h} \IFRU{выделяет в стеке 16 байт. Хотя, как будет видно далее, здесь достаточно только 4.}
{allocates 16 bytes on the stack. Although, as we can see below, only 4 are necessary here.} 

\IFRU{Это происходит потому что количество выделяемого места в локальном стеке тоже выровнено по 
16-байтной границе.}
{This is because the size of the allocated stack is also aligned on a 16-byte border.}

% TODO: rewrite.
\index{x86!\Instructions!PUSH}
\IFRU{Адрес строки (или указатель на строку) затем записывается прямо в стек без помощи инструкции \PUSH.
\IT{var\_10} по совместительству ~--- и локальная переменная и одновременно аргумент для \printf{}. Подробнее об этом будет ниже.}
{The string address (or a pointer to the string) is then written directly on the stack space 
without using the \PUSH instruction.
\IT{var\_10} ~--- is a local variable but is also an argument for \printf{}.
Read about it below.}

\IFRU{Затем вызывается \printf.}{Then the \printf function is called.}

\IFRU{В отличие от MSVC, GCC в компиляции без включенной оптимизации генерирует \TT{MOV EAX, 0} вместо 
более короткого опкода.}{Unlike MSVC, when GCC is compiling without optimization turned on,
it emits \TT{MOV EAX, 0} instead of a shorter opcode.}

\index{x86!\Instructions!LEAVE}
\IFRU{Последняя инструкция \LEAVE ~--- это аналог команд \TT{MOV ESP, EBP} и \TT{POP EBP} ~--- 
то есть возврат \glslink{stack pointer}{указателя стека} и регистра \EBP в первоначальное состояние.} 
{The last instruction, \LEAVE ~--- is the equivalent of the \TT{MOV ESP, EBP} and \TT{POP EBP} instruction pair ~--- 
in other words, this instruction sets the \gls{stack pointer} (\ESP) back and restores 
the \EBP register to its initial state.}

\IFRU{Это необходимо, т.к., в начале функции мы модифицировали регистры \ESP и \EBP (при помощи}
{This is necessary since we modified these register values (\ESP and \EBP) at the 
beginning of the function (executing}
\TT{\MOV EBP, ESP} / \TT{AND ESP, ...}).

\subsubsection{GCC: \ATTSyntax}

\IFRU{Попробуем посмотреть, как выглядит то же самое в AT\&T-синтаксисе языка ассемблера.}
{Let's see how this can be represented in the AT\&T syntax of assembly language.}
\IFRU{Этот синтаксис больше распространен в UNIX-мире.}
{This syntax is much more popular in the UNIX-world.}

\begin{lstlisting}[caption=\IFRU{компилируем в}{let's compile in} GCC 4.7.3]
gcc -S 1_1.c
\end{lstlisting}

\IFRU{Получим такой файл:}{We get this:}

\lstinputlisting[caption=GCC 4.7.3]{01_helloworld/1_1.s}

\IFRU{Здесь много макросов (начинающихся с точки). Они нам пока не интересны.}
{There are a lot of macros (beginning with dot). These are not very interesting to us so far.}
\IFRU{Пока что, ради упрощения, мы можем
их игнорировать и впредь (кроме макроса \IT{.string}, при помощи которого кодируется последовательность символов 
оканчивающихся нулем, такие же строки как в Си). И тогда получится следующее}
{For now, for the sake of simplification, we can ignore them (except the \IT{.string} macro which
encodes a null-terminated character sequence just like a C-string). Then we'll see this}
\footnote{\IFRU{Кстати, для уменьшения генерации ``лишних'' макросов, можно использовать такой ключ GCC}
{This GCC option can be used to eliminate ``unnecessary'' macros}: 
\IT{-fno-asynchronous-unwind-tables}}:

\lstinputlisting[caption=GCC 4.7.3]{01_helloworld/1_1_refined.s}

\index{\ATTSyntax}
\index{\IntelSyntax}
\IFRU{Основные отличия синтаксиса Intel и AT\&T следующие:}
{Some of the major differences between Intel and AT\&T syntax are:}

\begin{itemize}

\item
\IFRU{Операнды записываются наоборот.}{Operands are written backwards.}

\IFRU{В Intel-синтаксисе: <инструкция> <операнд назначения> <операнд-источник>.}
{In Intel-syntax: <instruction> <destination operand> <source operand>.}

\IFRU{В AT\&T-синтаксисе: <инструкция> <операнд-источник> <операнд назначения>.}
{In AT\&T syntax: <instruction> <source operand> <destination operand>.}

\IFRU{Чтобы легче понимать разницу, можно запомнить следующее}
{Here is a way to think about them}: \IFRU{когда вы работаете с Intel-синтаксисом, можете в уме ставить знак равенства ($=$) между операндами,}
{when you work with Intel-syntax, you can put in equality sign ($=$) in your mind between operands}
\IFRU{а когда с AT\&T-синтаксисом, мысленно ставьте стрелку направо}
{and when you work with AT\&T-syntax put in a right arrow} 
($\rightarrow$)
\footnote{
\index{\CStandardLibrary!memcpy()}
\index{\CStandardLibrary!strcpy()}
\IFRU{Кстати, в некоторые стандартных функциях библиотеки Си (например, memcpy(), strcpy()) также применяется 
расстановка аргументов как в Intel-синтаксисе: в начале указатель в памяти на блок назначения, 
затем указатель на блок-источник.}{By the way, in some C standard functions (e.g., memcpy(), strcpy()) arguments
are listed in the same way as in Intel-syntax: pointer to destination memory block at the beginning and then
pointer to source memory block.}}.

\item
AT\&T: \IFRU{Перед именами регистров ставится знак процента (\%), а перед числами знак доллара (\$).}
{Before register names a percent sign must be written (\%) and before numbers a dollar sign (\$).}
\IFRU{Вместо квадратных скобок применяются круглые.}{Parentheses are used instead of brackets.}

\item
AT\&T: \IFRU{К каждой инструкции добавляется специальный символ, определяющий тип данных:}
{A special symbol is to be added to each instruction defining the type of data:}

\begin{itemize}
\item l --- long (32 \IFRU{бита}{bits})
\item w --- word (16 \IFRU{бит}{bits})
\item b --- byte (8 \IFRU{бит}{bits})
\end{itemize}

\end{itemize}

\IFRU{Возвращаясь к результату компиляции: он идентичен тому, который мы посмотрели в \IDA.}
{Let's go back to the compilation result: it is identical to which we saw in \IDA.}
\IFRU{Одна мелочь}{With one subtle difference}: \TT{0FFFFFFF0h} \IFRU{записывается как}{is written as} \TT{\$-16}.
\IFRU{Это тоже самое}{It is the same}: \TT{16} \IFRU{в десятичной системе это}{in the decimal system is} \TT{0x10} 
\IFRU{в шестнадцатеричной}{in hexadecimal}. 
\TT{-0x10} \IFRU{будет как раз}{is equal to} \TT{0xFFFFFFF0} 
(\IFRU{в рамках 32-битных чисел}{for a 32-bit data type}).


