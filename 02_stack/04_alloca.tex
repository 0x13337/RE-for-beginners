\subsection{x86: \IFRU{Функция alloca()}{alloca() function}}
\label{alloca}
\index{\CStandardLibrary!alloca()}
\IFRU{Интересен случай с функцией \TT{alloca()}}
{It is worth noting \TT{alloca()} function.}\footnote{
\IFRU
{В MSVC, реализацию функции можно посмотреть в файлах}
{As of MSVC, function implementation can be found in} 
  \TT{alloca16.asm} 
  \IFRU{и}{and} 
  \TT{chkstk.asm} 
  \IFRU{в}{in} 
  \TT{C:\textbackslash{}Program Files (x86)\textbackslash{}Microsoft Visual Studio 10.0\textbackslash{}VC\textbackslash{}crt\textbackslash{}src\textbackslash{}intel}}. 

\IFRU{Эта функция работает как \TT{malloc()}, но выделяет память прямо в стеке.} 
{This function works like \TT{malloc()} but allocates memory just in stack.}

\IFRU{Память освобождать через \TT{free()} не нужно, так как эпилог функции~\ref{sec:prologepilog} 
вернет \ESP назад в изначальное состояние и выделенная память просто аyнулируется.}
{Allocated memory chunk is not needed to be freed via \TT{free()} function call since 
function epilogue~\ref{sec:prologepilog} shall return value of the \ESP back to initial state and 
allocated memory will be just annuled.} 

\IFRU{Интересна реализация функции \TT{alloca()}.}
{It is worth noting how \TT{alloca()} implemented.}

\IFRU{Эта функция, если упрощенно, просто сдвигает \ESP вглубь стека 
на столько байт сколько вам нужно и возвращает \ESP в качестве указателя на выделенный блок.}
{This function, if to simplify, just shifting \ESP deeply to stack bottom so much bytes you 
need and set \ESP as a pointer to that \IT{allocated} block.}
\IFRU{Попробуем:}{Let's try:}

\lstinputlisting{02_stack/2_1.c}

\IFRU{(Функция \TT{\_snprintf()} работает так же как и \printf, только вместо выдачи результата в 
stdout (т.е., на терминал или в консоль),
записывает его в буфер \TT{buf}. \puts выдает содержимое буфера \TT{buf} в stdout. Конечно, можно было бы
заменить оба этих вызова на один \printf, но мне нужно проиллюстрировать использование небольшого буфера.)}
{(\TT{\_snprintf()} function works just like \printf, but instead dumping result into stdout (e.g., to terminal or 
console), write it to the \TT{buf} buffer. \puts copies \TT{buf} contents to stdout. Of course, these two
function calls might be replaced by one \printf call, but I would like to illustrate small buffer usage.)}

\subsubsection{MSVC}

\IFRU{Компилируем}{Let's compile} (MSVC 2010):

\lstinputlisting[caption=MSVC 2010]{02_stack/2_2_msvc.asm}

\index{Compiler intrinsic}
\IFRU {Единственный параметр в \TT{alloca()} передается через \EAX, а не как обычно через стек}
{The sole \TT{alloca()} argument passed via \EAX (instead of pushing into stack)}
\footnote{\IFRU{Это потому что alloca() это не сколько функция, сколько т.е. compiler intrinsic}{It's because
alloca() is rather compiler intrinsic than usual function}}.
\IFRU{После вызова \TT{alloca()}, \ESP теперь указывает на блок в 600 байт который 
мы можем использовать под \TT{buf}.}
{After \TT{alloca()} call, \ESP is now pointing to the block of 600 bytes and we can 
use it as memory for \TT{buf} array.}

\subsubsection{GCC + \IntelSyntax}

\IFRU{А GCC 4.4.1 обходится без вызова других функций:}
{GCC 4.4.1 can do the same without calling external functions:}

\lstinputlisting[caption=GCC 4.7.3]{\IFRU{02_stack/2_1_gcc_intel_O3_ru.asm}{02_stack/2_1_gcc_intel_O3_en.asm}}

\subsubsection{GCC + \ATTSyntax}

\IFRU{Посмотрим на тот же код, только в синтаксисе AT\&T}{Let's see the same code, but in AT\&T syntax}:

\lstinputlisting[caption=GCC 4.7.3]{02_stack/2_1_gcc_ATT_O3.s}

\index{\ATTSyntax}
\IFRU{Всё то же самое что и в прошлом листинге.}{The same code as in previos listing.}

\IFRU{Обратите внимание что, например}{Please note that, for example}, \TT{movl \$3, 20(\%esp)} 
\IFRU{это аналог}{is analogous to} \TT{mov DWORD PTR [esp+20], 3} \IFRU{в Intel-синтаксисе}{in Intel-syntax} ~--- 
\IFRU{при адресации памяти в виде}{when addressing memory in form} \IT{\IFRU{регистр+смещение}{register+offset}}, 
\IFRU{это записывается в AT\&T синтаксисе как}{it's written in AT\&T syntax as} 
\TT{\IFRU{смещение}{offset}(\%\IFRU{регистр}{register})}.

