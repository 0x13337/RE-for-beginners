\section{\Stack}
\label{sec:stack}
\index{\Stack}

\IFRU{Стек в компьютерных науках ~--- это одна из наиболее фундаментальных вещей}
{Stack~---is one of the most fundamental things in computer science}
\footnote{\url{http://en.wikipedia.org/wiki/Call_stack}}.

\IFRU{Технически, это просто блок памяти в памяти процесса + регистр \ESP или \RSP в x86, либо \SP в ARM, который указывает где-то в пределах этого блока.}
{Technically it is just a memory block in process memory + the an \ESP or the \RSP register in x86 or the \SP register in ARM as a pointer within the block.}

\index{ARM!\Instructions!PUSH}
\index{ARM!\Instructions!POP}
\index{x86!\Instructions!PUSH}
\index{x86!\Instructions!POP}
\IFRU{Часто используемые инструкции для работы со стеком это \PUSH и \POP (в x86 и thumb-режиме ARM). 
\PUSH уменьшает \ESP/\RSP/\SP на $4$ в 32-битном режиме (или на $8$ в 64-битном),
затем записывает по адресу на который указывает \ESP/\RSP/\SP содержимое своего единственного операнда.}
{The most frequently used stack access instructions are \PUSH and \POP (in both x86 and ARM thumb-mode). 
\PUSH subtracts $4$ in 32-bit mode (or $8$ in 64-bit mode) from \ESP/\RSP/\SP and then writes the contents of its sole operand to the memory address pointed to by \ESP/\RSP/\SP.} 

\IFRU{\POP это обратная операция ~--- сначала достает из \glslink{stack pointer}{указателя стека} значение и помещает его в операнд 
(который очень часто является регистром) и затем увеличивает указатель стека на $4$ (или $8$).}
{\POP is the reverse operation: get the data from memory pointed to by \SP, 
put it in the operand (often a register) and then add $4$ (or $8$) to the \gls{stack pointer}.}

\IFRU{В самом начале, \glslink{stack pointer}{регистр-указатель} указывает на конец стека.}
{After stack allocation the \gls{stack pointer} points to the end of stack.}
\IFRU{\PUSH уменьшает \glslink{stack pointer}{регистр-указатель}, а \POP ~--- увеличивает.}
{\PUSH increases the \gls{stack pointer} and \POP decreases it.}
\IFRU{Конец стека находится в начале блока памяти выделенного под стек. Это странно, но это так.}
{The end of the stack is actually at the beginning of the memory allocated for the stack block. 
It seems strange, but it is so.}

\IFRU{В процессоре ARM, тем не менее, есть поддержка стеков растущих как в сторону уменьшения, так и в
сторону увеличения}
{Nevertheless ARM has not only instructions supporting ascending stacks but also descending stacks}. \\
\\
\index{ARM!\Instructions!STMFD}
\index{ARM!\Instructions!LDMFD}
\index{ARM!\Instructions!STMED}
\index{ARM!\Instructions!LDMED}
\index{ARM!\Instructions!STMFA}
\index{ARM!\Instructions!LDMFA}
\index{ARM!\Instructions!STMEA}
\index{ARM!\Instructions!LDMEA}
\IFRU{Например, инструкции}{For example the} 
STMFD\footnote{\STMFDdesc}/LDMFD\footnote{\LDMFDDESC}, 
STMED\footnote{\STMEDdesc}/LDMED\footnote{\LDMEDdesc} 
\IFRU{предназначены для descending-стека, т.е., уменьшающегося}{instructions are intended to work with 
a descending stack}.
\IFRU{Инструкции}{The}
STMFA\footnote{\STMFAdesc}/LMDFA\footnote{\LDMFAdesc}, 
STMEA\footnote{\STMEAdesc}/LDMEA\footnote{\LDMEAdesc} 
\IFRU{предназначены для ascending-стека, т.е., увеличивающегося}{instructions are intended to work with 
an ascending stack}.

\subsection{\IFRU{Для чего используется стек?}{What is the stack used for?}}

\subsubsection{\IFRU{Сохранение адреса куда должно вернуться управление после вызова функции}
{Save the return address where a function must return control after execution}}

\paragraph{x86}

\index{x86!\Instructions!CALL}
\IFRU{При вызове другой функции через \CALL, сначала в стек записывается адрес указывающий на место аккурат после 
инструкции \CALL, затем делается безусловный переход (почти как \TT{JMP}) на адрес указанный в операнде.} 
{While calling another function with a \CALL instruction the address of the point exactly after the \CALL instruction is saved 
to the stack and then an unconditional jump to the address in the CALL operand is executed.} 

\index{x86!\Instructions!PUSH}
\index{x86!\Instructions!JMP}
\IFRU{\CALL это аналог пары инструкций \TT{PUSH address\_after\_call / JMP}.}
{The \CALL instruction is equivalent to a \TT{PUSH address\_after\_call / JMP operand} instruction pair}.

\index{x86!\Instructions!RET}
\index{x86!\Instructions!POP}
\IFRU{\RET вытаскивает из стека значение и передает управление по этому адресу ~--- 
это аналог пары инструкций \TT{POP tmp / JMP tmp}.}
{\RET fetches a value from the stack and jumps to it ~--- it is equivalent to a \TT{POP tmp / JMP tmp} instruction pair.}

\index{\Stack!\IFRU{Переполнение стека}{Stack overflow}}
\index{\Recursion}
\IFRU{Крайне легко устроить переполнение стека запустив бесконечную рекурсию:}
{Overflowing the stack is straightforward. Just run eternal recursion:}

\begin{lstlisting}
void f()
{
	f();
};
\end{lstlisting}

\IFRU{MSVC 2008 предупреждает о проблеме:}{MSVC 2008 reports the problem:}

\begin{lstlisting}
c:\tmp6>cl ss.cpp /Fass.asm
Microsoft (R) 32-bit C/C++ Optimizing Compiler Version 15.00.21022.08 for 80x86
Copyright (C) Microsoft Corporation.  All rights reserved.

ss.cpp
c:\tmp6\ss.cpp(4) : warning C4717: 'f' : recursive on all control paths, function will cause runtime stack overflow
\end{lstlisting}

\dots \IFRU{но тем не менее создает нужный код}{but generates the right code anyway}:

\begin{lstlisting}
?f@@YAXXZ PROC						; f
; File c:\tmp6\ss.cpp
; Line 2
	push	ebp
	mov	ebp, esp
; Line 3
	call	?f@@YAXXZ				; f
; Line 4
	pop	ebp
	ret	0
?f@@YAXXZ ENDP						; f
\end{lstlisting}

\dots \IFRU
{причем, если включить оптимизацию (\Ox), то будет даже интереснее, без переполнения стека, 
но работать будет \IT{корректно}\footnote{здесь ирония}:}
{Also if we turn on optimization (\Ox option) the optimized code will not overflow the stack 
but will work \IT{correctly}\footnote{irony here}:}

\begin{lstlisting}
?f@@YAXXZ PROC						; f
; File c:\tmp6\ss.cpp
; Line 2
$LL3@f:
; Line 3
	jmp	SHORT $LL3@f
?f@@YAXXZ ENDP						; f
\end{lstlisting}

\IFRU{GCC 4.4.1 генерирует точно такой же код в обоих случаях, хотя и не предупреждает о проблеме.}
{GCC 4.4.1 generating the likewise code in both cases, although not warning about problem.}

\paragraph{ARM}

\index{ARM!\Registers!Link Register}
\IFRU{Программы для ARM также используют стек для сохранения \ac{RA}, куда нужно вернуться, но несколько иначе}{ARM
programs also use the stack for saving return addresses, but differently}.
\IFRU{Как уже упоминалось в секции}{As it was mentioned in} ``\HelloWorldSectionName''~(\ref{sec:hw_ARM}),
\IFRU{\ac{RA} записывается в регистр}{the \ac{RA} is saved to the} \LR (\IT{link register}).
\IFRU{Но если есть необходимость вызывать какую-то другую функцию, и использовать регистр \LR еще
раз, его значение желательно сохранить}
{However, if one needs to call another function and use the \LR register
one more time its value should be saved}.
\index{Function prologue}
\IFRU{Обычно, это происходит в прологе функции, часто мы видим там инструкцию вроде}
{Usually it is saved in the function prologue. Often, we see instructions like}
\index{ARM!\Instructions!PUSH}
\index{ARM!\Instructions!POP}
\TT{``PUSH {R4-R7,LR}''} \IFRU{, а в эпилоге}{along with this instruction in epilogue} \TT{``POP {R4-R7,PC}''} ~--- 
\IFRU{так сохраняются регистры, которые будут использоваться в текущей функции, в том числе}
{thus register values
to be used in the function are saved in the stack, including} \LR.

\index{ARM!Leaf function}
\IFRU{Тем не менее, если некая функция не вызывает никаких более функций, в терминологии ARM она называется}
{Nevertheless, if a function never calls any other function, in ARM terminology it is called}
\IT{\gls{leaf function}}\footnote{\url{http://infocenter.arm.com/help/index.jsp?topic=/com.arm.doc.faqs/ka13785.html}}. 
\IFRU{Как следствие, ``leaf''-функция не использует регистр \LR}
{As a consequence ``leaf'' functions do not use the \LR register}.
\IFRU{А если эта функция небольшая, использует мало регистров, она может не использовать стек вообще}
{And if this function is small and it uses a small number of registers it may not use stack at all}.
\IFRU{Таким образом, в ARM возможен вызов небольших ``leaf'' функций не используя стек}
{Thus, it is possible to call ``leaf'' functions without using stack}.
\IFRU{Это может быть быстрее чем в x86, ведь внешняя память для стека не используется}
{This can be faster than on x86 because external RAM is not used for the stack}
\footnote{\IFRU{Когда-то очень давно, на PDP-11 и VAX, на инструкцию CALL (вызов других функций) могло тратиться
вплоть до 50\% времени, возможно из-за работы с памятью,
поэтому считалось что много небольших функций это \glslink{anti-pattern}{анти-паттерн}}
{Some time ago, on PDP-11 and VAX, CALL instruction (calling other functions) was expensive, up to 50\%
of execution time might be spent on it, so it was common sense that big number of small function is \gls{anti-pattern}}\cite[Chapter 4, Part II]{Raymond:2003:AUP:829549}.}.
\IFRU{Либо, это может быть полезным для тех ситуаций, когда память для стека еще не выделена либо недоступна}
{It can be useful for such situations when memory for the stack is not yet allocated or not available}.

\subsection{\IFRU{Передача параметров для функции}{Function arguments passing}}

\begin{lstlisting}
push arg3
push arg2
push arg1
call f
add esp, 4*3
\end{lstlisting}

\IFRU{Вызываемая функция получает свои параметры также через указатель стека.}
{Callee{\footnote{Function being called}} function get its arguments via stack ponter.}

\IFRU{См.также в соответствующем разделе о способах передачи аргументов через стек}
{See also section about calling conventions}~\ref{sec:callingconventions}.

\IFRU{Важно отметить, что, в общем, никто не заставляет программистов передавать параметры именно через стек,
это не является требованием к исполняемому коду.}
{It is important to note that no one oblige programmers to pass arguments through stack, it is not prerequisite.}

\IFRU{Вы можете делать это совершенно иначе, не используя стек.}
{One could implement any other method not using stack.}

\IFRU{К примеру, можно выделять в куче\footnote{heap в англоязычной литературе} место для аргументов, 
заполнять их и передавать в функцию указатель на это место через \EAX. И это вполне будет работать}
{For example, it is possible to allocate a place for arguments in heap, fill it and pass to a function 
via pointer to this pack in \EAX register. And this will work}
\footnote{\IFRU{Например, в книге Дональда Кнута ``Искусство программирования'', в разделе 1.4.1 
посвященном подпрограммам\cite[раздел 1.4.1]{Knuth:1998:ACP:521463}, 
мы можем прочитать о возможности располагать параметры для вызываемой подпрограммы после инструкции \JMP
передающей управление подпрограмме. Кнут описывает что это было особенно удобно для компьютеров System/360.}
{For example, in ``The Art of Computer Programming'' book by Donald Knuth, 
in section 1.4.1 dedicated to subroutines\cite[section 1.4.1]{Knuth:1998:ACP:521463},
we can read about one way to supply arguments to subroutine is simply to list them after the \JMP instruction
passing control to subroutine. Knuth writes that this method was particularly convenient on System/360.}}.

\IFRU{Однако, так традиционно сложилось, что в x86 и ARM передача аргументов происходит именно через стек.}
{However, it is convenient tradition in x86 and ARM to use stack for this.}
\\
\\
\IFRU{Кстати, вызываемая ф-ция не имеет информации, сколько аргументов было ей было передано.}
{By the way, callee function hasn't any information, how many arguments were passed.}
\IFRU{Ф-ции Си с переменным количеством аргументов (как \printf) определяют их количество по 
спецификатором строки формата (начинающиеся со знака \%).}
{Functions with variable arguments count (like \printf) determines its count by specifiers
in format string (which begun with \% sign).}
\IFRU{Если написать что-то вроде}{If to write something like} 

\begin{lstlisting}
printf("%d %d %d", 1234);
\end{lstlisting}

\printf \IFRU{выведет 1234, затем еще два случайных числа, которые волею случая оказались в стеке рядом.}
{will dump 1234, and then also two random numbers, which were laying near it in stack, by chance.}
\\
\IFRU{Вот почему не так уж и важно, как объявлять ф-цию \main}
{That's why it's not very important how to declare \main function}: \IFRU{как}{as} \main, 
\TT{main(int argc, char *argv[])} 
\IFRU{либо}{or} \TT{main(int argc, char *argv[], char *envp[])}.

\IFRU{В реальности, т.н. startup-код вызывает \main примерно так:}
{In fact, so called startup-code is calling \main roughly as:}

\begin{lstlisting}
push envp
push argv
push argc
call main
...
\end{lstlisting}

\IFRU{Если вы объявляете \main как \main без аргументов, они, тем не менее, присутствуют в стеке, но не используются.}
{If you'll declare \main as \main without arguments, they are, nevertheless, are still present in stack, but
not used.}
\IFRU{Если вы объявите \main как}{If you declare \main as} \TT{main(int argc, char *argv[])}, 
\IFRU{вы будете использовать два аргумента, а третий останется для вашей ф-ции ``невидимым''.}
{you will use two arguments, and third will remain ``invisible'' for your function.}
\IFRU{Более того, можно даже объявить}{Even more than that, it's possible to declare} \TT{main(int argc)}, 
\IFRU{и это будет работать}{and it will work}.


\subsubsection{\IFRU{Хранение локальных переменных}{Local variable storage}}

\IFRU{Функция может выделить для себя некоторое место в стеке для локальных переменных просто отодвинув 
указатель стека глубже к концу стека.}
{A function could allocate some space in the stack for its local variables just by shifting 
the stack pointer towards stack bottom.}

\IFRU{Это снова не является необходимым требованием. Вы можете хранить локальные переменные где угодно. 
Но по традиции всё сложилось так.}
{It is also not a requirement. You could store local variables wherever you like. 
But traditionally it is so.}

\subsection{x86: \IFRU{Функция alloca()}{alloca() function}}
\label{alloca}
\index{\CStandardLibrary!alloca()}
\IFRU{Интересен случай с функцией \TT{alloca()}}
{It is worth noting \TT{alloca()} function.}\footnote{
\IFRU
{В MSVC, реализацию функции можно посмотреть в файлах}
{As of MSVC, function implementation can be found in} 
  \TT{alloca16.asm} 
  \IFRU{и}{and} 
  \TT{chkstk.asm} 
  \IFRU{в}{in} 
  \TT{C:\textbackslash{}Program Files (x86)\textbackslash{}Microsoft Visual Studio 10.0\textbackslash{}VC\textbackslash{}crt\textbackslash{}src\textbackslash{}intel}}. 

\IFRU{Эта функция работает как \TT{malloc()}, но выделяет память прямо в стеке.} 
{This function works like \TT{malloc()} but allocates memory just in stack.}

\IFRU{Память освобождать через \TT{free()} не нужно, так как эпилог функции~\ref{sec:prologepilog} 
вернет \ESP назад в изначальное состояние и выделенная память просто аyнулируется.}
{Allocated memory chunk is not needed to be freed via \TT{free()} function call since 
function epilogue~\ref{sec:prologepilog} shall return value of the \ESP back to initial state and 
allocated memory will be just annuled.} 

\IFRU{Интересна реализация функции \TT{alloca()}.}
{It is worth noting how \TT{alloca()} implemented.}

\IFRU{Эта функция, если упрощенно, просто сдвигает \ESP вглубь стека 
на столько байт сколько вам нужно и возвращает \ESP в качестве указателя на выделенный блок.}
{This function, if to simplify, just shifting \ESP deeply to stack bottom so much bytes you 
need and set \ESP as a pointer to that \IT{allocated} block.}
\IFRU{Попробуем:}{Let's try:}

\lstinputlisting{02_stack/2_1.c}

\IFRU{(Функция \TT{\_snprintf()} работает так же как и \printf, только вместо выдачи результата в 
stdout (т.е., на терминал или в консоль),
записывает его в буфер \TT{buf}. \puts выдает содержимое буфера \TT{buf} в stdout. Конечно, можно было бы
заменить оба этих вызова на один \printf, но мне нужно проиллюстрировать использование небольшого буфера.)}
{(\TT{\_snprintf()} function works just like \printf, but instead dumping result into stdout (e.g., to terminal or 
console), write it to the \TT{buf} buffer. \puts copies \TT{buf} contents to stdout. Of course, these two
function calls might be replaced by one \printf call, but I would like to illustrate small buffer usage.)}

\subsubsection{MSVC}

\IFRU{Компилируем}{Let's compile} (MSVC 2010):

\lstinputlisting[caption=MSVC 2010]{02_stack/2_2_msvc.asm}

\index{Compiler intrinsic}
\IFRU {Единственный параметр в \TT{alloca()} передается через \EAX, а не как обычно через стек}
{The sole \TT{alloca()} argument passed via \EAX (instead of pushing into stack)}
\footnote{\IFRU{Это потому что alloca() это не сколько функция, сколько т.е. compiler intrinsic}{It's because
alloca() is rather compiler intrinsic than usual function}}.
\IFRU{После вызова \TT{alloca()}, \ESP теперь указывает на блок в 600 байт который 
мы можем использовать под \TT{buf}.}
{After \TT{alloca()} call, \ESP is now pointing to the block of 600 bytes and we can 
use it as memory for \TT{buf} array.}

\subsubsection{GCC + \IntelSyntax}

\IFRU{А GCC 4.4.1 обходится без вызова других функций:}
{GCC 4.4.1 can do the same without calling external functions:}

\lstinputlisting[caption=GCC 4.7.3]{\IFRU{02_stack/2_1_gcc_intel_O3_ru.asm}{02_stack/2_1_gcc_intel_O3_en.asm}}

\subsubsection{GCC + \ATTSyntax}

\IFRU{Посмотрим на тот же код, только в синтаксисе AT\&T}{Let's see the same code, but in AT\&T syntax}:

\lstinputlisting[caption=GCC 4.7.3]{02_stack/2_1_gcc_ATT_O3.s}

\index{\ATTSyntax}
\IFRU{Всё то же самое что и в прошлом листинге.}{The same code as in previos listing.}

\IFRU{Обратите внимание что, например}{Please note that, for example}, \TT{movl \$3, 20(\%esp)} 
\IFRU{это аналог}{is analogous to} \TT{mov DWORD PTR [esp+20], 3} \IFRU{в Intel-синтаксисе}{in Intel-syntax} ~--- 
\IFRU{при адресации памяти в виде}{when addressing memory in form} \IT{\IFRU{регистр+смещение}{register+offset}}, 
\IFRU{это записывается в AT\&T синтаксисе как}{it's written in AT\&T syntax as} 
\TT{\IFRU{смещение}{offset}(\%\IFRU{регистр}{register})}.


\subsubsection{(Windows) SEH}
\index{Windows!Structured Exception Handling}

\IFRU{В стеке хранятся записи SEH (\IT{Structured Exception Handling}) для функции (если имеются)}
{SEH (\IT{Structured Exception Handling}) records are also stored on the stack (if needed).}
\footnote{
\IFRU{О SEH: классическая статья Мэтта Питрека}{Classic Matt Pietrek article about SEH}: 
\url{http://www.microsoft.com/msj/0197/Exception/Exception.aspx}}.

\subsection{\IFRU{Защита от переполнений буфера}{Buffer overflow protection}}

\IFRU{Здесь больше об этом}{More about it here}~\ref{subsec:bufferoverflow}.

