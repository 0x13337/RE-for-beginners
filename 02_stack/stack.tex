\section{\IFRU{Стек}{Stack}}
\label{sec:stack}

\IFRU{Стек в компьютерных науках ~--- это одна из наиболее фундаментальных вещей}
{Stack ~--- is one of the most fundamental things in computer science.}\footnote{\url{http://en.wikipedia.org/wiki/Call_stack}}.

\IFRU{Технически, это просто блок памяти в памяти процесса + регистр \ESP или \RSP в x86, либо \SP в ARM, который указывает где-то в пределах этого блока.}{Technically, this is just memory block in process memory + \ESP or \RSP register in x86, or \SP register in ARM, as a pointer within this block.}

\IFRU{Часто используемые инструкции для работы со стеком это \PUSH и \POP (в x86 и thumb-режиме ARM). 
\PUSH уменьшает \ESP/\RSP/\SP на 4, затем записывает по адресу на который указывает \ESP/\RSP/\SP содержимое своего единственного операнда.}
{Most frequently used stack access instructions are \PUSH and \POP (both in x86 and ARM thumb-mode). 
\PUSH subtracting \ESP/\RSP/\SP by 4 and then writing contents of its sole operand to the memory address pointing by \ESP/\RSP/\SP.} 

\IFRU{\POP это обратная операция ~--- сначала достает из \ESP/\RSP/\SP значение и помещает его в операнд 
(который очень часто является регистром) и затем увеличивает \ESP/\RSP/\SP на 4. 
Конечно, это для 32-битной среды. В x64-среде это будет 8 а не 4.}
{\POP is reverse operation: get a data from memory pointing by \ESP/\RSP/\SP and then add 4 to \ESP/\RSP/\SP. 
Of course, this is for 32-bit environment. 8 will be here instead of 4 in x64 environment.}

\IFRU{В самом начале, регистр-указатель указывает на конец стека.}{After stack allocation, stack pointer pointing to the end of stack.}
\IFRU{\PUSH уменьшает регистр-указатель, а \POP ~--- увеличивает.}{\PUSH increasing stack pointer, and \POP decreasing.}
\IFRU{Конец стека находится в начале блока памяти выделенного под стек. Это странно, но это так.}
{The end of stack is actually at the beginning of allocated for stack memory block. 
It seems strange, but it is so.}

\IFRU{В процессоре ARM, тем не менее, есть поддержка стеков растущих как в сторону уменьшения, так и в
сторону увеличения}{Nevertheless, ARM has instructions supporting ascending stacks, but also descending stacks}. 

\IFRU{Например, инструкции}{For example,} 
STMFD\footnote{Store Multiple Full Descending}/LDMFD\footnote{\LDMFDDESC}, 
STMED\footnote{Store Multiple Empty Descending}/LDMED\footnote{Load Multiple Empty Descending} 
\IFRU{предназначены для descending-стека, т.е., уменьшающегося}{instructions are intended for work with 
descending stack}.
\IFRU{Инструкции}{}
STMFA\footnote{Store Multiple Full Ascending}/LMDFAA\footnote{Load Multiple Full Ascending}, 
STMEA\footnote{Store Multiple Empty Ascending}/LDMEA\footnote{Load Multiple Empty Ascending} 
\IFRU{предназначены для ascending-стека, т.е., увеличивающегося}{instructions are intended for work with 
ascending stack}.

\IFRU{Для чего используется стек?}{What stack is used for?}

\subsection{\IFRU{Сохранение адреса куда должно вернуться управление после вызова функции}
{Save return address where function should return control after execution}}

\subsubsection{x86}

\IFRU{При вызове другой функции через \CALL, сначала в стек записывается адрес указывающий на место аккурат после 
инструкции \CALL, затем делается безусловный переход (\TT{JMP}) на адрес указанный в операнде.} 
{While calling another function by \CALL instruction, the address of point exactly after \CALL instruction is saved 
to stack, and then unconditional jump to the address from CALL operand is executed.} 

\IFRU{\CALL это аналог пары инструкций \TT{PUSH address\_after\_call / JMP}.}
{\CALL is \TT{PUSH address\_after\_call / JMP operand} instructions pair equivalent}.

\IFRU{\RET вытаскивает из стека значение и передает управление по этому адресу ~--- 
это аналог пары инструкций \TT{POP tmp / JMP tmp}.}
{\RET is fetching value from stack and jump to it ~--- it is \TT{POP tmp / JMP tmp} instructions pair equivalent.}

\IFRU{Крайне легко устроить переполнение стека запустив бесконечную рекурсию:}
{Stack overflow is simple, just run eternal recursion:}

\begin{lstlisting}
void f()
{
	f();
};
\end{lstlisting}

\IFRU{MSVC 2008 предупреждает о проблеме:}{MSVC 2008 reporting about problem:}

\begin{lstlisting}
c:\tmp6>cl ss.cpp /Fass.asm
Microsoft (R) 32-bit C/C++ Optimizing Compiler Version 15.00.21022.08 for 80x86
Copyright (C) Microsoft Corporation.  All rights reserved.

ss.cpp
c:\tmp6\ss.cpp(4) : warning C4717: 'f' : recursive on all control paths, function will cause runtime stack overflow
\end{lstlisting}

\dots \IFRU{но тем не менее создает нужный код}{but generates right code anyway}:

\begin{lstlisting}
?f@@YAXXZ PROC						; f
; File c:\tmp6\ss.cpp
; Line 2
	push	ebp
	mov	ebp, esp
; Line 3
	call	?f@@YAXXZ				; f
; Line 4
	pop	ebp
	ret	0
?f@@YAXXZ ENDP						; f
\end{lstlisting}

\dots \IFRU
{причем, если включить оптимизацию (\Ox), то будет даже интереснее, без переполнения стека, 
но работать будет \IT{корректно}\footnote{здесь ирония}:}
{Also, if we turn on optimization (\Ox option), the optimized code will not overflow stack, 
but will work \IT{correctly}\footnote{irony here}:}

\begin{lstlisting}
?f@@YAXXZ PROC						; f
; File c:\tmp6\ss.cpp
; Line 2
$LL3@f:
; Line 3
	jmp	SHORT $LL3@f
?f@@YAXXZ ENDP						; f
\end{lstlisting}

\IFRU{GCC 4.4.1 генерирует точно такой же код в обоих случаях, хотя и не предупреждает о проблеме.}
{GCC 4.4.1 generating the same code in both cases, although not warning about problem.}

\subsubsection{ARM}

\IFRU{Программы для ARM также используют стек для сохранения адреса, куда нужно вернуться, но несколько иначе}{ARM
programs also use stack for return address savings, but differently}.
\IFRU{Как уже упоминалось в секции}{As it was mentioned in} ``\HelloWorldSectionName''~\ref{sec:hw_ARM}, 
\IFRU{адрес возврата записывается в регистр}{return address is saved to} \LR (\IT{link register}).
\IFRU{Но если есть необходимость вызывать какую-то другую функцию, и использовать регистр \LR еще
раз, его значение желательно сохранить}{However, if one need to call some another function and use \LR register
one more time, its value should be saved}.
\IFRU{Обычно, это происходит в прологе функции, часто мы видим там инструкцию вроде}{Usually, it is saved in
function prologue, often, we see there instruction like}
\TT{``PUSH {R4-R7,LR}''}, \IFRU{а в эпилоге}{and also this instruction in epilogue} \TT{``POP {R4-R7,PC}''} ~--- 
\IFRU{так сохраняются регистры, которые будут использоваться в текущей функции, в том числе}{thus register values
to be used in function are saved in stack, including} \LR.

\IFRU{Тем не менее, если некая функция не вызывает никаких более функций, в терминологии ARM она называется}
{Nevertheless, if some function never call any other function, in ARM terminology it's called}
\IT{leaf function}\footnote{\url{http://infocenter.arm.com/help/index.jsp?topic=/com.arm.doc.faqs/ka13785.html}}. 
\IFRU{Как следствие, ``leaf''-функция не использует регистр \LR}{As a consequence, ``leaf'' function 
don't use \LR register}.
\IFRU{А если эта функция небольшая, использует мало регистров, она может не использовать стек вообще}
{And if this function is small, if it use small number of registers, it may not use stack at all}.
\IFRU{Таким образом, в ARM возможен вызов небольших ``leaf'' функций не используя стек}{Thus, it is possible
to call ``leaf'' function without stack usage}.
\IFRU{Это может быть быстрее чем в x86, ведь внешняя память для стека не используется}{This could be faster
than in x86, because external RAM is not used for stack}.
\IFRU{Либо, это может быть полезным для тех ситуаций, когда память для стека еще не выделена либо недоступна}
{Or, it can be useful for such situations, when memory for stack is not yet allocated or not available}.

\subsection{\IFRU{Передача параметров для функции}{Function arguments passing}}

\begin{lstlisting}
push arg3
push arg2
push arg1
call f
add esp, 4*3
\end{lstlisting}

\IFRU{Вызываемая функция получает свои параметры также через указатель стека.}
{Callee{\footnote{Function being called}} function get its arguments via stack ponter.}

\IFRU{См.также в соответствующем разделе о способах передачи аргументов через стек}
{See also section about calling conventions}~\ref{sec:callingconventions}.

\IFRU{Важно отметить, что, в общем, никто не заставляет программистов передавать параметры именно через стек,
это не является требованием к исполняемому коду.}
{It is important to note that no one oblige programmers to pass arguments through stack, it is not prerequisite.}

\IFRU{Вы можете делать это совершенно иначе, не используя стек.}
{One could implement any other method not using stack.}

\IFRU{К примеру, можно выделять в куче\footnote{heap в англоязычной литературе} место для аргументов, 
заполнять их и передавать в функцию указатель на это место через \EAX. И это вполне будет работать}
{For example, it is possible to allocate a place for arguments in heap, fill it and pass to a function 
via pointer to this pack in \EAX register. And this will work}
\footnote{\IFRU{Например, в книге Дональда Кнута ``Искусство программирования'', в разделе 1.4.1 посвященном подпрограммам\cite[раздел 1.4.1]{Knuth}, мы можем прочитать о возможности располагать параметры для вызываемой подпрограммы после инструкции \JMP
передающей управление подпрограмме. Кнут описывает что это было особенно удобно для компьютеров System/360.}
{For example, in ``The Art of Computer Programming'' book by Donald Knuth, 
in section 1.4.1 dedicated to subroutines\cite[section 1.4.1]{Knuth},
we can read about one way to supply arguments to subroutine is simply to list them after the \JMP instruction
passing control to subroutine. Knuth writes that this method was particularly convenient on System/360.}}.

\IFRU{Однако, так традиционно сложилось, что в x86 и ARM передача аргументов происходит именно через стек.}
{However, it is convenient tradition in x86 and ARM to use stack for this.}

\subsection{\IFRU{Хранение локальных переменных}{Local variable storage}}

\IFRU{Функция может выделить для себя некоторое место в стеке для локальных переменных просто отодвинув 
указатель стека глубже к концу стека.}
{A function could allocate some space in stack for its local variables just shifting 
stack pointer deeply enough to stack bottom.}

\IFRU{Это снова не является необходимым требованием. Вы можете хранить локальные переменные где угодно. 
Но по традиции всё сложилось так.}
{It is also not prerequisite. You could store local variables wherever you like. 
But traditionally it is so.}

\subsubsection{x86: \IFRU{Функция alloca()}{alloca() function}}
\label{alloca}
\IFRU{Интересен случай с функцией \TT{alloca()}}
{It is worth noting \TT{alloca()} function.}\footnote{
\IFRU
{В MSVC, реализацию функции можно посмотреть в файлах}
{As of MSVC, function implementation can be found in} 
  \TT{alloca16.asm} 
  \IFRU{и}{and} 
  \TT{chkstk.asm} 
  \IFRU{в}{in} 
  \TT{C:\textbackslash{}Program Files (x86)\textbackslash{}Microsoft Visual Studio 10.0\textbackslash{}VC\textbackslash{}crt\textbackslash{}src\textbackslash{}intel}}. 

\IFRU{Эта функция работает как \TT{malloc()}, но выделяет память прямо в стеке.} 
{This function works like \TT{malloc()}, but allocate memory just in stack.}

\IFRU{Память освобождать через \TT{free()} не нужно, так как эпилог функции~\ref{sec:prologepilog} 
вернет \ESP назад в изначальное состояние и выделенная память просто аyнулируется.}
{Allocated memory chunk is not needed to be freed via \TT{free()} function call because 
function epilogue~\ref{sec:prologepilog} will return \ESP back to initial state and 
allocated memory will be just annuled.} 

\IFRU{Интересна реализация функции \TT{alloca()}.}
{It is worth noting how \TT{alloca()} implemented.}

\IFRU{Эта функция, если упрощенно, просто сдвигает \ESP вглубь стека 
на столько байт сколько вам нужно и возвращает \ESP в качестве указателя на выделенный блок.}
{This function, if to simplify, just shifting \ESP deeply to stack bottom so much bytes you 
need and set \ESP as a pointer to that \IT{allocated} block.}
\IFRU{Попробуем:}{Let's try:}

\lstinputlisting{02_stack/2_1.c}

\IFRU{(Функция \TT{\_snprintf()} работает так же как и \printf, только вместо выдачи результата в 
stdout (т.е., на терминал или в консоль),
записывает его в буфер \TT{buf}. \puts выдает содержимое буфера \TT{buf} в stdout. Конечно, можно было бы
заменить оба этих вызова на один \printf, но мне нужно проиллюстрировать использование небольшого буфера.)}
{(\TT{\_snprintf()} function works just like \printf, but instead dumping result into stdout (e.g., to terminal or 
console), write it to \TT{buf} buffer. \puts copies \TT{buf} contents to stdout. Of course, these two
function calls might be replaced by one \printf call, but I would like to illustrate small buffer usage.)}

\IFRU{Компилируем}{Let's compile} (MSVC 2010):

\lstinputlisting{02_stack/2_2_msvc.asm}

\IFRU {Единственный параметр в \TT{alloca()} передается через \EAX, а не как обычно через стек.}
{The sole \TT{alloca()} argument passed via \EAX (instead of pushing into stack).}

\IFRU{После вызова \TT{alloca()}, \ESP теперь указывает на блок в 600 байт который 
мы можем использовать под \TT{buf}.}
{After \TT{alloca()} call, \ESP is now pointing to the block of 600 bytes and we can 
use it as memory for \TT{buf} array.}

\IFRU{А GCC 4.4.1 обходится без вызова других функций:}
{GCC 4.4.1 can do the same without calling external functions:}

\lstinputlisting{\IFRU{02_stack/2_2_gcc_ru.asm}{02_stack/2_2_gcc_en.asm}}

\subsection{(Windows) SEH}

\IFRU{В стеке хранятся записи SEH (\IT{Structured Exception Handling}) для функции (если имеются)}
{SEH (\IT{Structured Exception Handling}) records are also stored in stack (if needed).}
\footnote{
\IFRU{О SEH: классическая статья Мэтта Питрека}{Classic Matt Pietrek article about SEH}: 
\url{http://www.microsoft.com/msj/0197/Exception/Exception.aspx}}.

\subsection{\IFRU{Защита от переполнений буфера}{Buffer overflow protection}}

\IFRU{Здесь больше об этом}{More about it here}~\ref{subsec:bufferoverflow}.

