\subsection{ARM: \IFRU{3 аргумента в \printf}{3 \printf arguments}}

\IFRU{В ARM традиционно принята такая схема передачи аргументов в функцию: 
4 первых аргумента через регистры \Rzero-\Rthree,
а остальные ~--- через стек}
{Traditionally, ARM arguments passing scheme (calling convention) is as follows:
the 4 first arguments are passed in the \Rzero-\Rthree registers and remaining arguments ~--- via stack}.
\IFRU{Это немного похоже на то как аргументы передаются в}{This resembling arguments passing scheme in} 
fastcall~\ref{fastcall} \IFRU{или}{or} win64~\ref{sec:callingconventions_win64}.

\subsubsection{\NonOptimizingKeil + \ARMMode}

\begin{lstlisting}[caption=\NonOptimizingKeil + \ARMMode]
.text:00000014             printf_main1
.text:00000014 10 40 2D E9                 STMFD   SP!, {R4,LR}
.text:00000018 03 30 A0 E3                 MOV     R3, #3
.text:0000001C 02 20 A0 E3                 MOV     R2, #2
.text:00000020 01 10 A0 E3                 MOV     R1, #1
.text:00000024 1D 0E 8F E2                 ADR     R0, aADBDCD     ; "a=%d; b=%d; c=%d\n"
.text:00000028 0D 19 00 EB                 BL      __2printf
.text:0000002C 10 80 BD E8                 LDMFD   SP!, {R4,PC}
\end{lstlisting}

\IFRU{Итак, первые 4 аргумента передаются через регистры \Rzero-\Rthree, по порядку: 
указатель на формат-строку для \printf
в \Rzero, затем $1$ в R1, $2$ в R2 и $3$ в R3}
{So, the first 4 arguments are passing via \Rzero-\Rzero arguments in this order:
pointer to the \printf format string in the \Rzero, then $1$ in R1, $2$ in R2 and $3$ in R3}.

\IFRU{Пока что, здесь нет ничего необычного}{Nothing unusual so far}.

\subsubsection{\OptimizingKeil + \ARMMode}
\label{ARM_B_to_printf}

\begin{lstlisting}[caption=\OptimizingKeil + \ARMMode]
.text:00000014                             EXPORT printf_main1
.text:00000014             printf_main1
.text:00000014 03 30 A0 E3                 MOV     R3, #3
.text:00000018 02 20 A0 E3                 MOV     R2, #2
.text:0000001C 01 10 A0 E3                 MOV     R1, #1
.text:00000020 1E 0E 8F E2                 ADR     R0, aADBDCD     ; "a=%d; b=%d; c=%d\n"
.text:00000024 CB 18 00 EA                 B       __2printf
\end{lstlisting}

\index{ARM!\Registers!Link Register}
\index{ARM!\Instructions!B}
\index{Function epilogue}
\IFRU{Это соптимизированная версия (\Othree) для режима ARM, и здесь мы видим последнюю инструкцию: 
\TT{B} вместо привычной нам \TT{BL}}{This is optimized (\Othree) version for ARM mode and here we see \TT{B} as
the last instruction instead of familiar \TT{BL}}.
\IFRU{Отличия между этой соптимзированной версией и предыдущей, скомпилированной без оптимизации, 
еще и в том, 
что здесь нет пролога и эпилога функции (инструкций, сохранающих состояние регистров \TT{\Rzero} и \LR)}
{Another difference between this optimized version and previous one, compiled without optimization, 
is also in the
fact that there are no function prologue and epilogue (instructions which are saving \TT{\Rzero} and \LR registers values)}.
\index{x86!\Instructions!JMP}
\IFRU{Инструкция \TT{B} просто переходит на другой адрес, без манипуляций с регистром \LR, то есть,
это аналог \JMP в x86}
{\TT{B} instruction just jumping to another address, without any manipulation of the \LR register,
that is, it's \JMP analogue in x86}.
\IFRU{Почему это работает нормально? Потому что этот код эквивалентен предыдущему.}
{Why it works fine? Because this code is in fact effectively equivalent to the previous.}
\IFRU{Основных причин две: 1) стек не модифицируется, как и указатель стека \SP; 2) вызов функции \printf последний, 
после него ничего не происходит}{There are two main reasons: 1) stack is not modified, as well as \SP stack pointer; 
2) \printf call is the last one, there are nothing going on after it}.
\IFRU{Функция \printf, отработав, просто вернет управление по адресу, записанному в \LR}{After finishing, \printf
function will just return control to the address stored in the \LR}.
\IFRU{Но в \LR находится адрес места, откуда была вызвана наша функция}
{But the address of the point from where our function
was called is now in the \LR}!
\IFRU{А следовательно, управление из \printf вернется сразу туда}
{And consequently, control from \printf will returned to that point}.
\IFRU{Следовательно, нет нужды сохранять \LR, потому что нет нужны модифицировать \LR}
{As a consequent, we don't need to save \LR since we don't need to modify \LR}.
\IFRU{А нет нужды модифицировать \LR, потому что нет иных вызовов функций, кроме \printf, к тому же, после этого вызова не нужно ничего здесь больше делать}{And we don't need to modify \LR since there are no other functions calls except \printf, furthermore,
after this call we are not planning to do anything}!
\IFRU{Поэтому такая оптимизация возможна}{That's why this optimization is possible}.

\IFRU{Еще один похожий пример описан в секции}{Another similar example was described in} 
``\SwitchCaseDefaultSectionName'' 
\IFRU{, здесь}{section, here}~\ref{jump_to_last_printf}.

\subsubsection{\OptimizingKeil + \ThumbMode}

\begin{lstlisting}[caption=\OptimizingKeil + \ThumbMode]
.text:0000000C             printf_main1
.text:0000000C 10 B5                       PUSH    {R4,LR}
.text:0000000E 03 23                       MOVS    R3, #3
.text:00000010 02 22                       MOVS    R2, #2
.text:00000012 01 21                       MOVS    R1, #1
.text:00000014 A4 A0                       ADR     R0, aADBDCD     ; "a=%d; b=%d; c=%d\n"
.text:00000016 06 F0 EB F8                 BL      __2printf
.text:0000001A 10 BD                       POP     {R4,PC}
\end{lstlisting}

\IFRU{Здесь нет особых отличий от неоптимизированного варианта для режима ARM}
{There are no significant difference from non-optimized code for ARM mode}.


