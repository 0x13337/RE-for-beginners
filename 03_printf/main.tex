\section{\PrintfSeveralArgumentsSectionName}

\IFRU{Попробуем теперь немного расширить пример \IT{\HelloWorldSectionName}~(\ref{sec:helloworld}),
написав в теле функции \main:}
{Now let's extend \IT{\HelloWorldSectionName}~(\ref{sec:helloworld}) example, replacing \printf in
the \main function body by this:}

\begin{lstlisting}
printf("a=%d; b=%d; c=%d", 1, 2, 3);
\end{lstlisting}

\subsection{x86}

\IFRU{Компилируем при помощи MSVC 2010 Express, и в итоге получим:}
{Let's compile it by MSVC 2010 and we got:}

\begin{lstlisting}
$SG3830	DB	'a=%d; b=%d; c=%d', 00H

...

	push	3
	push	2
	push	1
	push	OFFSET $SG3830
	call	_printf
	add	esp, 16					; 00000010H
\end{lstlisting}

\IFRU{Все почти то же, за исключением того, что теперь видно, что аргументы для \printf заталкиваются в стек в обратном порядке: самый первый аргумент заталкивается последним.}
{Almost the same, but now we can see the \printf arguments are pushing into stack in reverse order: and the first argument is pushing in as the last one.}

\IFRU{Кстати, вспомним что переменные типа \Tint в 32-битной системе, как известно, имеет ширину 32 бита, это 4 байта.}
{By the way, variables of \Tint type in 32-bit environment has 32-bit width, that is 4 bytes.}

\IFRU{Итак, у нас всего 4 аргумента. $4*4 = 16$ ~--- именно 16 байт занимают в стеке указатель на строку плюс еще 3 числа типа \Tint.}
{So, we got here 4 arguments. $4*4 = 16$ ~--- they occupy exactly 16 bytes in the stack: 32-bit pointer to string and 3 number of \Tint type.}

\index{x86!\Instructions!ADD}
\index{x86!\Registers!ESP}
\index{cdecl}
\IFRU{Когда при помощи инструкции \TT{``ADD ESP, X''} корректируется указатель стека \ESP 
после вызова какой-либо функции, зачастую можно сделать вывод о том, сколько аргументов 
у вызываемой функции было, разделив X на 4.}
{When stack pointer (the \ESP register) is corrected by \TT{``ADD ESP, X''}
instruction after a function 
call, often, the number of function arguments could be deduced here: just divide X by 4.}

\IFRU{Конечно, это относится только к cdecl-методу передачи аргументов через стек.}
{Of course, this is related only to \IT{cdecl} calling convention.}

\IFRU{См.также в соответствующем разделе о способах передачи аргументов через стек}
{See also section about calling conventions}~\ref{sec:callingconventions}.

\IFRU{Иногда бывает так, что подряд идут несколько вызовов разных функций, 
но стек корректируется только один раз, после последнего вызова:}
{It is also possible for compiler to merge several \TT{``ADD ESP, X''} instructions into one, after last call:}

\begin{lstlisting}
push a1
push a2
call ...
...
push a1
call ...
...
push a1
push a2
push a3
call ...
add esp, 24
\end{lstlisting}

\IFRU{Скомпилируем то же самое в Linux при помощи GCC 4.4.1 и посмотрим в \IDA что вышло:}
{Now let's compile the same in Linux by GCC 4.4.1 and take a look in \IDA what we got:}

\begin{lstlisting}
main            proc near

var_10          = dword ptr -10h
var_C           = dword ptr -0Ch
var_8           = dword ptr -8
var_4           = dword ptr -4

                push    ebp
                mov     ebp, esp
                and     esp, 0FFFFFFF0h
                sub     esp, 10h
                mov     eax, offset aADBDCD ; "a=%d; b=%d; c=%d"
                mov     [esp+10h+var_4], 3
                mov     [esp+10h+var_8], 2
                mov     [esp+10h+var_C], 1
                mov     [esp+10h+var_10], eax
                call    _printf
                mov     eax, 0
                leave
                retn
main            endp
\end{lstlisting}

\IFRU{Можно сказать, что этот короткий код созданный GCC отличается от кода MSVC только способом помещения 
значений в стек.
Здесь GCC снова работает со стеком напрямую без \PUSH/\POP.}
{It can be said, the difference between code by MSVC and GCC is only in method of placing arguments on the stack.
Here GCC working directly with stack without \PUSH/\POP.}

\subsection{ARM: \IFRU{3 аргумента в \printf}{3 \printf arguments}}

\IFRU{В ARM традиционно принята такая схема передачи аргументов в функцию: 
4 первых аргумента через регистры \Rzero-\Rthree,
а остальные ~--- через стек}
{Traditionally, ARM arguments passing scheme (calling convention) is as follows:
the 4 first arguments are passed in \Rzero-\Rthree registers and remaining arguments ~--- via stack}.
\IFRU{Это немного похоже на то как аргументы передаются в}{This resembling arguments passing scheme in} 
fastcall~\ref{fastcall} \IFRU{или}{or} win64~\ref{sec:callingconventions_win64}.

\subsubsection{\NonOptimizingKeil + \ARMMode}

\begin{lstlisting}[caption=\NonOptimizingKeil + \ARMMode]
.text:00000014             printf_main1
.text:00000014 10 40 2D E9                 STMFD   SP!, {R4,LR}
.text:00000018 03 30 A0 E3                 MOV     R3, #3
.text:0000001C 02 20 A0 E3                 MOV     R2, #2
.text:00000020 01 10 A0 E3                 MOV     R1, #1
.text:00000024 1D 0E 8F E2                 ADR     R0, aADBDCD     ; "a=%d; b=%d; c=%d\n"
.text:00000028 0D 19 00 EB                 BL      __2printf
.text:0000002C 10 80 BD E8                 LDMFD   SP!, {R4,PC}
\end{lstlisting}

\IFRU{Итак, первые 4 аргумента передаются через регистры \Rzero-\Rthree, по порядку: 
указатель на формат-строку для \printf
в \Rzero, затем $1$ в R1, $2$ в R2 и $3$ в R3}
{So, the first 4 arguments are passing via \Rzero-\Rzero arguments in this order:
pointer to \printf format string in \Rzero, then $1$ in R1, $2$ in R2 and $3$ in R3}.

\IFRU{Пока что, здесь нет ничего необычного}{Nothing unusual so far}.

\subsubsection{\OptimizingKeil + \ARMMode}
\label{ARM_B_to_printf}

\begin{lstlisting}[caption=\OptimizingKeil + \ARMMode]
.text:00000014                             EXPORT printf_main1
.text:00000014             printf_main1
.text:00000014 03 30 A0 E3                 MOV     R3, #3
.text:00000018 02 20 A0 E3                 MOV     R2, #2
.text:0000001C 01 10 A0 E3                 MOV     R1, #1
.text:00000020 1E 0E 8F E2                 ADR     R0, aADBDCD     ; "a=%d; b=%d; c=%d\n"
.text:00000024 CB 18 00 EA                 B       __2printf
\end{lstlisting}

\index{ARM!\Registers!Link Register}
\index{ARM!\Instructions!B}
\index{Function epilogue}
\IFRU{Это соптимизированная версия (\Othree) для режима ARM, и здесь мы видим последнюю инструкцию: 
\TT{B} вместо привычной нам \TT{BL}}{This is optimized (\Othree) version for ARM mode and here we see \TT{B} as
the last instruction instead of familiar \TT{BL}}.
\IFRU{Отличия между этой соптимзированной версией и предыдущей, скомпилированной без оптимизации, 
еще и в том, 
что здесь нет пролога и эпилога функции (инструкций, сохранающих состояние регистров \TT{\Rzero} и \LR)}
{Another difference between this optimized version and previous one, compiled without optimization, 
is also in the
fact that there are no function prologue and epilogue (instructions saving \TT{\Rzero} and \LR registers values)}.
\index{x86!\Instructions!JMP}
\IFRU{Инструкция \TT{B} просто переходит на другой адрес, без манипуляций с регистром \LR, то есть,
это аналог \JMP в x86}
{\TT{B} instruction just jumping to another address, without any \LR register manipulation, 
that is, it's \JMP analogue in x86}.
\IFRU{Почему это работает нормально? Потому что этот код эквивалентен предыдущему.}
{Why it works fine? Because this code is in fact equivalent to the previous.}
\IFRU{Основных причин две: 1) стек не модифицируется, как и указатель стека \SP; 2) вызов функции \printf последний, 
после него ничего не происходит}{There are two main reasons: 1) stack is not modified, as well as \SP stack pointer; 
2) \printf call is the last one, there are nothing going on after it}.
\IFRU{Функция \printf, отработав, просто вернет управление по адресу, записанному в \LR}{After finishing, \printf
function will just return control to the address stored in \LR}. 
\IFRU{Но в \LR находится адрес места, откуда была вызвана наша функция}{But the address of the place from where our function
was called is now in \LR}!
\IFRU{А следовательно, управление из \printf вернется сразу туда}
{And consequently, control from \printf will returned to that place}.
\IFRU{Следовательно, нет нужды сохранять \LR, потому что нет нужны модифицировать \LR}
{As a consequent, we don't need to save \LR, because we don't need to modify \LR}.
\IFRU{А нет нужды модифицировать \LR, потому что нет иных вызовов функций, кроме \printf, к тому же, после этого вызова не нужно ничего здесь больше делать}{And we don't need to modify \LR because there are no other functions calls except \printf, furthermore,
after this call we are not planning to do anything}!
\IFRU{Поэтому такая оптимизация возможна}{That's why this optimization is possible}.

\IFRU{Еще один похожий пример описан в секции}{Another similar example was described in} 
``\SwitchCaseDefaultSectionName'' 
\IFRU{, здесь}{section, here}~\ref{jump_to_last_printf}.

\subsubsection{\OptimizingKeil + \ThumbMode}

\begin{lstlisting}[caption=\OptimizingKeil + \ThumbMode]
.text:0000000C             printf_main1
.text:0000000C 10 B5                       PUSH    {R4,LR}
.text:0000000E 03 23                       MOVS    R3, #3
.text:00000010 02 22                       MOVS    R2, #2
.text:00000012 01 21                       MOVS    R1, #1
.text:00000014 A4 A0                       ADR     R0, aADBDCD     ; "a=%d; b=%d; c=%d\n"
.text:00000016 06 F0 EB F8                 BL      __2printf
.text:0000001A 10 BD                       POP     {R4,PC}
\end{lstlisting}

\IFRU{Здесь нет особых отличий от неоптимизированного варианта для режима ARM}
{There are no significant difference from non-optimized code for ARM mode}.



\subsection{ARM: \IFRU{8 аргументов в \printf}{8 \printf arguments}}

\IFRU{Для того, чтобы посмотреть, как остальные аргументы будут передаваться через стек, 
изменим пример еще раз, 
увеличив количество передаваемых аргументов до 9 (строка формата \printf и 8 переменных типа \Tint)}
{To see,
how other arguments will be passed via stack, let's change our example again by increasing number of arguments
to be passed to 9 (\printf format string + 8 \Tint variables)}:

\begin{lstlisting}
void printf_main2()
{
	printf("a=%d; b=%d; c=%d; d=%d; e=%d; f=%d; g=%d; h=%d\n", 1, 2, 3, 4, 5, 6, 7, 8);
};
\end{lstlisting}

\subsubsection{\OptimizingKeil: \ARMMode}

\begin{lstlisting}
.text:00000028             printf_main2
.text:00000028
.text:00000028             var_18          = -0x18
.text:00000028             var_14          = -0x14
.text:00000028             var_4           = -4
.text:00000028
.text:00000028 04 E0 2D E5                 STR     LR, [SP,#var_4]!
.text:0000002C 14 D0 4D E2                 SUB     SP, SP, #0x14
.text:00000030 08 30 A0 E3                 MOV     R3, #8
.text:00000034 07 20 A0 E3                 MOV     R2, #7
.text:00000038 06 10 A0 E3                 MOV     R1, #6
.text:0000003C 05 00 A0 E3                 MOV     R0, #5
.text:00000040 04 C0 8D E2                 ADD     R12, SP, #0x18+var_14
.text:00000044 0F 00 8C E8                 STMIA   R12, {R0-R3}
.text:00000048 04 00 A0 E3                 MOV     R0, #4
.text:0000004C 00 00 8D E5                 STR     R0, [SP,#0x18+var_18]
.text:00000050 03 30 A0 E3                 MOV     R3, #3
.text:00000054 02 20 A0 E3                 MOV     R2, #2
.text:00000058 01 10 A0 E3                 MOV     R1, #1
.text:0000005C 6E 0F 8F E2                 ADR     R0, aADBDCDDDEDFDGD ; "a=%d; b=%d; c=%d; d=%d; e=%d; f=%d; g=%"...
.text:00000060 BC 18 00 EB                 BL      __2printf
.text:00000064 14 D0 8D E2                 ADD     SP, SP, #0x14
.text:00000068 04 F0 9D E4                 LDR     PC, [SP+4+var_4],#4
\end{lstlisting}

\IFRU{Этот код можно условно разделить на несколько частей}{This code can be divided into several parts}:

\begin{itemize}
\index{Function prologue}
\item \IFRU{Пролог функции}{Function prologue}:

\index{ARM!\Instructions!STR}
\IFRU{Самая первая инструкция}{The very first} \TT{``STR LR, [SP,\#var\_4]!''} 
\IFRU{сохраняет в стеке \LR, ведь, нам придется использовать этот регистр для вызова \printf}
{instruction saves the \LR in stack, because we will use this register for \printf call}.

\index{ARM!\Instructions!SUB}
\IFRU{Вторая инструкция}{The second} \TT{``SUB SP, SP, \#0x14''} \IFRU{уменьшает указатель стека \SP, но на самом деле, эта процедура нужна для выделения в локальном стеке места размером \TT{0x14} ($20$) байт}
{instruction decreasing
\SP stack pointer, but in fact, 
this procedure is needed for allocating a space of size \TT{0x14} ($20$) bytes in the stack}.
\IFRU{Действительно, нам нужно передать 5 32-битных значений через стек в \printf, каждое значение занимает 4 байта, а $5*4=20$ ~--- как раз}{Indeed, we need to pass 5 32-bit values via stack to the \printf function, and each one occupy 4 bytes, that is $5*4=20$ ~--- exactly}.
\IFRU{Остальные 4 32-битных значения будут переданы через регистры}{Other 4 32-bit values will be passed in
registers}.

\item \IFRU{Передача 5, 6, 7 и 8 через стек}{Passing 5, 6, 7 and 8 via stack}:

\IFRU{Затем значения 5, 6, 7 и 8 записываются в регистры \Rzero, \Rone, \Rtwo и \Rthree соответственно}
{Then values 5, 6, 7 and 8
are written to the \Rzero, \Rone, \Rtwo and \Rthree registers respectively}.
\IFRU{Затем инструкция}{Then} \TT{``ADD R12, SP, \#0x18+var\_14''} 
\IFRU{записывает в регистр \TT{R12} адрес места в стеке, куда будут помещены эти 4 значения}
{instruction writes an address of the point in the stack, where these 4 variables will be written, into the \TT{R12} register}.
\index{IDA!var\_?}
\IT{var\_14} \IFRU{это макрос ассемблера}{is an assembly macro}, \IFRU{равный}{equal to} $-0x14$, 
\IFRU{такие макросы создает \IDA, чтобы удобнее было показывать, как код обращается к стеку}{such macros are created by \IDA in order to show simply how code accessing stack}.
\IFRU{Макросы \IT{var\_?}, создаваемые \IDA, отражают локальные переменные в стеке}{\IT{var\_?} macros created
by \IDA reflecting local variables in stack}.
\IFRU{Так что, в \TT{R12} будет записано $SP+4$}{So, $SP+4$ will be written into the \TT{R12} register}.
\index{ARM!\Instructions!STMIA}
\IFRU{Следующая инструкция}{The next} \TT{``STMIA R12, {R0-R3}''} 
\IFRU{записывает содержимое регистров \Rzero-\Rthree по адресу в памяти, на который указывает \TT{R12}}
{instruction
writes \Rzero-\Rthree registers contents at the point in memory to which \TT{R12} pointing}.
\IFRU{Инструкция }\TT{STMIA} \IFRU{означает}{instruction meaning} \IT{Store Multiple Increment After}. 
\IT{Increment After} \IFRU{означает что \TT{R12} будет увеличиваться на 4 после записи каждого значения регистра}
{meaning that \TT{R12} will be increasing by $4$ after each register value write}.

\item \IFRU{Передача $4$ через стек}{Passing $4$ via stack}:
\IFRU{$4$ записывается в \Rzero, затем, это значение, при помощи инструкции}{$4$ is stored in the \Rzero and then,
this value, with the help of} \TT{``STR R0, [SP,\#0x18+var\_18]''} \IFRU{попадает в стек}{instruction, is saved
in stack}.
\IT{var\_18} \IFRU{равен}{is} $-0x18$, \IFRU{смещение будет $0$}{offset will be $0$}, 
\IFRU{так что, значение из регистра \Rzero ($4$) запишется туда, куда указывает \SP}
{so, value from the \Rzero register ($4$) will be written to a point the \SP pointing to}.

\item \IFRU{Передача 1, 2 и 3 через регистры}{Passing 1, 2 and 3 via registers}:

\IFRU{Значения для первых трех чисел (a, b, c) (1, 2, 3 соответственно) передаются в регистрах R1, R2 и R3 перед самим вызововм \printf}
{Values of first 3 numbers (a, b, c) (1, 2, 3 respectively) are passing in R1, R2 and R3
registers right before \printf call}, \IFRU{а остальные 5 значений передаются через стек, и вот как}{and other
5 values are passed via stack and this is how}:

\item \IFRU{Вызов \printf}{\printf call}:

\index{Function epilogue}
\item \IFRU{Эпилог функции}{Function epilogue}:

\IFRU{Инструкция }{}\TT{``ADD SP, SP, \#0x14''} \IFRU{возвращает \SP на прежнее место, 
аннулируя таким образом, всё что было записано в стеке}
{instruction returns the \SP pointer back to former point,
thus annulling what was written to stack}.
\IFRU{Конечно, то что было записано в стек, там пока и останется, но всё это будет многократно 
перезаписано во время исполнения последующих функций}
{Of course, what was written in stack will stay there, but it all will be
rewritten while execution of following functions}.

\index{ARM!\Instructions!LDR}
\IFRU{Инструкция }{}\TT{``LDR PC, [SP+4+var\_4],\#4''} \IFRU{загружает в \PC сохраненное значение \LR из стека, 
таким образом, обеспечивая выход из функции}{instruction loads saved \LR value in stack into the \PC register, providing
exit from the function}.

\end{itemize}

\subsubsection{\OptimizingKeil: \ThumbMode}

\begin{lstlisting}
.text:0000001C             printf_main2
.text:0000001C
.text:0000001C             var_18          = -0x18
.text:0000001C             var_14          = -0x14
.text:0000001C             var_8           = -8
.text:0000001C
.text:0000001C 00 B5                       PUSH    {LR}
.text:0000001E 08 23                       MOVS    R3, #8
.text:00000020 85 B0                       SUB     SP, SP, #0x14
.text:00000022 04 93                       STR     R3, [SP,#0x18+var_8]
.text:00000024 07 22                       MOVS    R2, #7
.text:00000026 06 21                       MOVS    R1, #6
.text:00000028 05 20                       MOVS    R0, #5
.text:0000002A 01 AB                       ADD     R3, SP, #0x18+var_14
.text:0000002C 07 C3                       STMIA   R3!, {R0-R2}
.text:0000002E 04 20                       MOVS    R0, #4
.text:00000030 00 90                       STR     R0, [SP,#0x18+var_18]
.text:00000032 03 23                       MOVS    R3, #3
.text:00000034 02 22                       MOVS    R2, #2
.text:00000036 01 21                       MOVS    R1, #1
.text:00000038 A0 A0                       ADR     R0, aADBDCDDDEDFDGD ; "a=%d; b=%d; c=%d; d=%d; e=%d; f=%d; g=%"...
.text:0000003A 06 F0 D9 F8                 BL      __2printf
.text:0000003E
.text:0000003E             loc_3E                                  ; CODE XREF: example13_f+16
.text:0000003E 05 B0                       ADD     SP, SP, #0x14
.text:00000040 00 BD                       POP     {PC}
\end{lstlisting}

\IFRU{Это почти то же самое что и в предыдущем примере, только код для thumb и значения помещаются в 
стек немного иначе: в начале $8$ за первый раз, затем $5$, $6$, $7$ за второй раз и $4$ за третий раз}{Almost 
same as
in previous example, however, this is thumb code and values are packed into stack differently: 
$8$ for the first time, then $5$, $6$, $7$ for the second and $4$ for the third}.

\subsubsection{\OptimizingXcode: \ARMMode}

\begin{lstlisting}
__text:0000290C             _printf_main2
__text:0000290C
__text:0000290C             var_1C          = -0x1C
__text:0000290C             var_C           = -0xC
__text:0000290C
__text:0000290C 80 40 2D E9                 STMFD           SP!, {R7,LR}
__text:00002910 0D 70 A0 E1                 MOV             R7, SP
__text:00002914 14 D0 4D E2                 SUB             SP, SP, #0x14
__text:00002918 70 05 01 E3                 MOV             R0, #0x1570
__text:0000291C 07 C0 A0 E3                 MOV             R12, #7
__text:00002920 00 00 40 E3                 MOVT            R0, #0
__text:00002924 04 20 A0 E3                 MOV             R2, #4
__text:00002928 00 00 8F E0                 ADD             R0, PC, R0
__text:0000292C 06 30 A0 E3                 MOV             R3, #6
__text:00002930 05 10 A0 E3                 MOV             R1, #5
__text:00002934 00 20 8D E5                 STR             R2, [SP,#0x1C+var_1C]
__text:00002938 0A 10 8D E9                 STMFA           SP, {R1,R3,R12}
__text:0000293C 08 90 A0 E3                 MOV             R9, #8
__text:00002940 01 10 A0 E3                 MOV             R1, #1
__text:00002944 02 20 A0 E3                 MOV             R2, #2
__text:00002948 03 30 A0 E3                 MOV             R3, #3
__text:0000294C 10 90 8D E5                 STR             R9, [SP,#0x1C+var_C]
__text:00002950 A4 05 00 EB                 BL              _printf
__text:00002954 07 D0 A0 E1                 MOV             SP, R7
__text:00002958 80 80 BD E8                 LDMFD           SP!, {R7,PC}
\end{lstlisting}

\index{ARM!\Instructions!STMFA}
\index{ARM!\Instructions!STMIB}
\IFRU{Почти то же самое что мы уже видели, за исключением того что}
{Almost the same what we already figured out, with the
exception of} \TT{STMFA} (Store Multiple Full Ascending) 
\IFRU{это синоним инструкции}{instruction, it is synonym to} 
\TT{STMIB} (Store Multiple Increment Before) \IFRU{}{instruction}. 
\IFRU{Эта инструкция увеличивает \SP и только затем записывает в память значение очередного регистра, 
но не наоборот}{This
instruction increasing value in the \SP register and only then writing next register value into memory, but not vice versa}.

\index{\IFRU{Внеочередное исполнение (OoOE)}{Out-of-order execution}}
\IFRU{Второе что бросается в глаза, это то что инструкции как будто бы расположены случайно}{Another thing
we easily spot is that instructions ostensibly located randomly}.
\IFRU{Например, значение в регистре \Rzero подготавливается в трех местах, по адресам \TT{0x2918}, \TT{0x2920} 
и \TT{0x2928}, 
когда это можно было бы сделать в одном месте}{For instance, value in the \Rzero register is prepared in three
places, at addresses \TT{0x2918}, \TT{0x2920} and \TT{0x2928}, when it would be possible to do it in one single point}.
\IFRU{Однако, у оптимизирующего компилятора могут быть свои доводы о том, как лучше составлять инструкции 
друг с другом для лучшей эффективности исполнения}
{However, optimizing compiler has its own reasons about how
to place instructions better}.
\IFRU{Процессор обычно пытается исполнять одновременно идущие друг за другом инструкции}{Usually,
processor attempts to execute instructions located side-by-side}.
\IFRU{К примеру, инструкции}{For example, instructions like} \TT{``MOVT R0, \#0''} \IFRU{и}{and} 
\TT{``ADD R0, PC, R0''} \IFRU{не могут быть исполнены одновременно, потому что обе инструкции модифицируют 
регистр \Rzero}{cannot be executed simultaneously since they both modifying the \Rzero register}. 
\IFRU{А вот инструкции}{On the other hand,} \TT{``MOVT R0, \#0''} \IFRU{и}{and} \TT{``MOV R2, \#4''} 
\IFRU{легко можно исполнить одновременно, 
потому что эффекты от их исполнения никак не конфликтуют друг с другом}{instructions can be executed
simultaneously since effects of their execution are not conflicting with each other}.
\IFRU{Вероятно, компилятор старается генерировать код именно таким образом, конечно, там где это возможно}
{Presumably, compiler tries to generate code in such way, where it's possible, of course}.
 
\subsubsection{\OptimizingXcode: \ThumbTwoMode}

\begin{lstlisting}
__text:00002BA0                   _printf_main2
__text:00002BA0
__text:00002BA0                   var_1C          = -0x1C
__text:00002BA0                   var_18          = -0x18
__text:00002BA0                   var_C           = -0xC
__text:00002BA0
__text:00002BA0 80 B5                             PUSH            {R7,LR}
__text:00002BA2 6F 46                             MOV             R7, SP
__text:00002BA4 85 B0                             SUB             SP, SP, #0x14
__text:00002BA6 41 F2 D8 20                       MOVW            R0, #0x12D8
__text:00002BAA 4F F0 07 0C                       MOV.W           R12, #7
__text:00002BAE C0 F2 00 00                       MOVT.W          R0, #0
__text:00002BB2 04 22                             MOVS            R2, #4
__text:00002BB4 78 44                             ADD             R0, PC  ; char *
__text:00002BB6 06 23                             MOVS            R3, #6
__text:00002BB8 05 21                             MOVS            R1, #5
__text:00002BBA 0D F1 04 0E                       ADD.W           LR, SP, #0x1C+var_18
__text:00002BBE 00 92                             STR             R2, [SP,#0x1C+var_1C]
__text:00002BC0 4F F0 08 09                       MOV.W           R9, #8
__text:00002BC4 8E E8 0A 10                       STMIA.W         LR, {R1,R3,R12}
__text:00002BC8 01 21                             MOVS            R1, #1
__text:00002BCA 02 22                             MOVS            R2, #2
__text:00002BCC 03 23                             MOVS            R3, #3
__text:00002BCE CD F8 10 90                       STR.W           R9, [SP,#0x1C+var_C]
__text:00002BD2 01 F0 0A EA                       BLX             _printf
__text:00002BD6 05 B0                             ADD             SP, SP, #0x14
__text:00002BD8 80 BD                             POP             {R7,PC}
\end{lstlisting}

\IFRU{Почти то же самое что и впредыдущем примере, лишь за тем исключением что здесь используются thumb-инструкции}
{Almost the same as in previous example, with the exception that thumb-instructions are used there instead}.



\subsection{\IFRU{Кстати}{By the way}}

\IFRU{Кстати, разница между способом передачи параметров принятая в x86 и ARM, неплохо иллюстрирует тот важный момент, что процессору, в общем, все равно как будут 
передаваться параметры функций. Можно создать гипотетический компилятор, который будет передавать их при 
помощи указателя на структуру с параметрами, не пользуясь стеком вообще.}
{By the way, this difference between passing arguments in x86 and ARM is a good illustration the CPU is not aware of how arguments is passed to functions. 
It is also possible to create hypothetical compiler which is able to pass arguments 
via a special structure not using stack at all.}

