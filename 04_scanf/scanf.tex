\section{scanf()}
\index{\CLanguageElements!scanf}
\label{label_scanf}

\IFRU{Теперь попробуем использовать scanf().}{Now let's use scanf().}

\begin{lstlisting}
int main() 
{
	int x;
	printf ("Enter X:\n");

	scanf ("%d", &x);

	printf ("You entered %d...\n", x);

	return 0;
};
\end{lstlisting}

\IFRU
{Да, согласен, использовать \scanf в наши времена для того чтобы спросить у пользователя что-то: 
не самая хорошая идея.
Но я хотел проиллюстрировать передачу указателя на \Tint.}
{OK, I agree, it is not clever to use \scanf today. But I wanted to illustrate passing pointer to \Tint.}

\subsection{\IFRU{Об указателях}{About pointers}}
\index{\CLanguageElements!\Pointers}

\IFRU{Это одна из фундаментальных вещей в компьютерных науках.}{It's one of the most fundamental things in computer
science.}
\IFRU{Часто, большой массив, структуру или объект, передавать в другую функцию никак не выгодно, 
а передать её адрес куда проще.}
{Often, large array, structure or object, it's too costly to pass to another function, 
while passing its address is much easier.}
\IFRU{К тому же, если вызываемая функция должна изменить что-то в этом большом массиве или структуре,
то возвращать её полностью это так же абсурдно.}
{More than that: if calling function should modify something in that large array or structure,
to return it as a whole is absurdical as well.}
\IFRU{Так что самое простое что можно сделать, это передать в функцию адрес массива или структуры,
и пусть она что-то там изменит.}
{So the most simple thing to do is to pass an address of array or structure to function,
and let it change it what need.}

\IFRU{Указатель в}{In} \CCpp \IFRU{это просто адрес какого-либо места в памяти.}{it's just an address of some place
in memory.}

\index{x86-64}
\IFRU{В x86 адрес представляется в виде 32-битного числа (т.е., занимает 4 байта), а в x86-64 как 64-битное число 
(занимает 8 байт).}
{In x86, address is represented as 32-bit number (i.e., occupying 4 bytes), while in x86-64 it's 64-bit number
(occupying 8 bytes).}
\IFRU{Кстати, отсюда негодование некоторых людей связанное с переходом на x86-64 ~--- на этой архитектуре все указатели
будут занимать места в 2 раза больше.}
{By the way, that's a reson of some people's indignation related to switching to x86-64 ~--- all pointers
on that architecture will require twice as more space.}

\index{\CLanguageElements!memcpy()}
\IFRU{При некотором упорстве, можно работать только с бестиповыми указателями (\TT{void*})}{With some effort,
it's possible to work only with untyped pointers}, \IFRU{например}{for example}, 
\IFRU{стандартная функция}{standard function} \TT{memcpy()},
\IFRU{копирующая блок из одного места памяти в другое}{copying a block from one place in memory to another}, 
\IFRU{принимает на вход 2 указателя типа}{takes 2 pointers of} \TT{void*}\IFRU{}{ type on input}, 
\IFRU{потому что, нельзя
зараннее предугадать, какого типа блок вы собираетесь копировать, да в общем это и не важно, важно только знать размер
блока.}
{because it's not possible to predict block type you would like to copy, and it's not even important to know, 
only block size is important.}

\IFRU{Также, указатели широко используются когда функции нужно вернуть более одного значения}
{Also, pointers are widely used when function need to return more than one value}
(\IFRU{мы еще вернемся к этому в будущем}{we will back to this in future}~\ref{label_pointers}).
\IT{scanf()} \IFRU{это как раз такой случай}{is just that case}. 
\IFRU{Помимо того, что этой функции нужно показать, сколько значений
было прочитано успешно, ей еще и нужно вернуть сами значения.}
{In addition to the function's need to show, how many values were read successfully, it also should
to return all these values.}

\IFRU{Тип указателя в}{In} \CCpp \IFRU{нужен для проверки типов на стадии компиляции.}
{pointer type is needed only for type checking on compiling stage.}
\IFRU{Внутри, в скомпилированном коде, никакой информации о типах указателей нет.}
{Internally, in compiled code, there are no information about pointers types.}

\subsection{x86}

\IFRU{Что получаем на ассемблере компилируя MSVC 2010:}
{What we got after compiling in MSVC 2010:}

\lstinputlisting{04_scanf/4_1_msvc.asm}

\IFRU{Переменная \TT{x} является локальной.}{Variable \TT{x} is local.} 

\IFRU{По стандарту \CCpp она доступна только из этой же функции и ниоткуда более. 
Так получилось, что локальные переменные располагаются в стеке. 
Может быть, можно было бы использовать и другие варианты, но в x86 это традиционно так.}
{\CCpp standard tell us it must be visible only in this function and not from any other place. 
Traditionally, local variables are placed in the stack. 
Probably, there could be other ways, but in x86 it is so.}

\index{x86!\Instructions!PUSH}
\IFRU{Следующая после пролога инструкция \TT{PUSH ECX} не ставит своей целью сохранить 
значение регистра \ECX. 
(Заметьте отсутствие сооветствующей инструкции \TT{POP ECX} в конце функции)}
{Next after function prologue instruction \TT{PUSH ECX} is not for saving \ECX state 
(notice absence of corresponding \TT{POP ECX} at the function end).}

\IFRU{Она на самом деле выделяет в стеке 4 байта для хранения \TT{x} в будущем.} 
{In fact, this instruction just allocate 4 bytes in stack for \TT{x} variable storage.} 

\index{\Stack!\IFRU{Стековый фрейм}{Stack frame}}
\index{x86!\Registers!EBP}
\IFRU{Доступ к \TT{x} будет осуществляться при помощи объявленного макроса \TT{\_x\$} 
(он равен -4) и регистра \EBP указывающего на текущий фрейм.}
{\TT{x} will be accessed with the assistance of \TT{\_x\$} macro 
(it equals to -4) and \EBP register pointing to current frame.}

\IFRU{Вообще, во все время исполнения функции, \EBP указывает на текущий фрейм и через \TT{EBP+смещение}
можно иметь доступ как к локальным переменным функции, так и аргументам функции.} 
{Over a span of function execution, \EBP is pointing to current stack frame and it is possible 
to have an access to local variables and function arguments via \TT{EBP+offset}.}

\index{x86!\Registers!ESP}
\IFRU
{Можно было бы использовать \ESP, но он во время исполнения функции постоянно меняется. 
Так что можно сказать что \EBP это \IT{замороженное состояние} \ESP на момент начала исполнения функции.}
{It is also possible to use \ESP, but it's often changing and not very convenient.
So it can be said, \EBP is \IT{frozen state} of \ESP at the moment of function execution start.}

\IFRU
{У функции \scanf в нашем примере два аргумента.}{Function \scanf in our example has two arguments.}

\IFRU
{Первый ~--- указатель на строку содержащую \TT{``\%d''} и второй ~--- адрес переменной \TT{x}.} 
{First is pointer to the string containing \TT{``\%d''} and second ~--- address of variable \TT{x}.} 

\index{x86!\Instructions!LEA}
\IFRU{Вначале адрес \TT{x} помещается в регистр \EAX при помощи инструкции \TT{lea eax, DWORD PTR \_x\$[ebp]}.}
{First of all, address of \TT{x} is placed into \EAX register by \TT{lea eax, DWORD PTR \_x\$[ebp]} instruction}

\IFRU{Инструкция \LEA означает \IT{load effective address}, но со временем она изменила свою функцию}
{\LEA meaning \IT{load effective address}, but over a time it changed its primary application}
~\ref{sec:LEA}.

\IFRU{Можно сказать что в данном случае \LEA просто помещает в \EAX результат суммы значения в регистре 
\EBP и макроса \TT{\_x\$}.}
{It can be said, \LEA here just placing to \EAX sum of \EBP value and \TT{\_x\$} macro.}

\IFRU{Это тоже что и}{It is the same as} \TT{lea eax, [ebp-4]}.

\IFRU{Итак, от значения \EBP отнимается 4 и помещается в \EAX.
Далее значение \EAX заталкивается в стек и вызывается \scanf.}
{So, 4 subtracting from \EBP value and result is placed to \EAX. 
And then value in \EAX is pushing into stack and \scanf is called.}

\IFRU{После этого вызывается \printf. Первый аргумент вызова которого, строка:} 
{After that, \printf is called. First argument is pointer to string:} \TT{``You entered \%d...\textbackslash{}n''}.

\IFRU{Второй аргумент: \TT{mov ecx, [ebp-4]}, эта инструкция помещает в \ECX не адрес переменной \TT{x}, 
а его значение, что там сейчас находится.}
{Second argument is prepared as: \TT{mov ecx, [ebp-4]},
this instruction placing to \ECX not address of \TT{x} variable but its contents.}

\IFRU{Далее значение \ECX заталкивается в стек и вызывается последний \printf.}
{After, \ECX value is placing into stack and last \printf called.}

\IFRU{Попробуем тоже самое скомпилировать в Linux при помощи GCC 4.4.1:}
{Let's try to compile this code in GCC 4.4.1 under Linux:}

\lstinputlisting{04_scanf/4_1_gcc.asm}

\index{puts() \IFRU{вместо}{instead of} printf()}
\IFRU{GCC заменил первый вызов \printf на \puts, почему это было сделано, 
уже было описано раннее~\ref{puts}.}
{GCC replaced first \printf call to \puts, it was already described~\ref{puts} 
why it was done.}

% TODO: rewrite
%\IFRU
%{Почему \scanf переименовали в \TT{\_\_\_isoc99\_scanf}, я честно говоря, пока не знаю.}
%{Why \scanf is renamed to \TT{\_\_\_isoc99\_scanf}, I do not know yet.}

\IFRU{Далее все как и прежде ~--- параметры заталкиваются через стек при помощи \MOV.}
{As before ~--- arguments are placed into stack by \MOV instruction.}


\subsection{ARM}

\subsubsection{\OptimizingKeil + \ThumbMode}

\begin{lstlisting}
.text:00000042             scanf_main
.text:00000042
.text:00000042             var_8           = -8
.text:00000042
.text:00000042 08 B5                       PUSH    {R3,LR}
.text:00000044 A9 A0                       ADR     R0, aEnterX     ; "Enter X:\n"
.text:00000046 06 F0 D3 F8                 BL      __2printf
.text:0000004A 69 46                       MOV     R1, SP
.text:0000004C AA A0                       ADR     R0, aD          ; "%d"
.text:0000004E 06 F0 CD F8                 BL      __0scanf
.text:00000052 00 99                       LDR     R1, [SP,#8+var_8]
.text:00000054 A9 A0                       ADR     R0, aYouEnteredD___ ; "You entered %d...\n"
.text:00000056 06 F0 CB F8                 BL      __2printf
.text:0000005A 00 20                       MOVS    R0, #0
.text:0000005C 08 BD                       POP     {R3,PC}
\end{lstlisting}

\index{\CLanguageElements!\Pointers}
\IFRU{Чтобы \scanf мог вернуть значение, ему нужно передать указатель на переменную типа \Tint.}
{A pointer
to \Tint-typed variable shoud be passed to \scanf so it can return value via it.}
\Tint \IFRU{~--- 32-битное значение, для его хранения нужно только 4 байта и оно помещается в 
32-битный регистр.}
{is 32-bit value, so we need 4 bytes for storing it somewhere in memory, and it fits exactly 
in 32-bit regsiter.}
\index{IDA!var\_?}
\IFRU{Место для локальной переменной \TT{x} выделяется в стеке, \IDA наименовала её \IT{var\_8}, 
впрочем, место для нее выделять не обязательно, т.к., указатель стека \SP уже указывает на место, 
свободное для использования.}{A place for local variable \TT{x} is allocated in stack and \IDA
named it \IT{var\_8}, however, it's not necessary to allocate it, because, \SP stack pointer
is already pointing to the place which may be used.}
\IFRU{Так что значение указателя \SP копируется в регистр \Rone, и вместе с format-строкой, 
передается в \scanf.}
{So, \SP stack pointer value is copied to \Rone register and, together with format-string, passed
into \scanf.}
\index{ARM!\Instructions!LDR}
\IFRU{Позже, при помощи инструкции \TT{LDR}, это значение перемещается из стека в регистр \Rone, 
чтобы быть переданным в \printf.}{Later, with the help of \TT{LDR} instruction, this value is moved
from stack into \Rone register in order to be passed into \printf.}

\IFRU{Варианты скомпилированные для ARM-режима процессора, а также варианты скомпилированные при 
помощи Xcode LLVM, не очень отличаются от этого, так что, мы можем пропустить их здесь.}
{Examples compiled for ARM-mode and also examples compiled with Xcode LLVM are not differ significally
from what we saw here, so they are omitted.}



\subsection{\IFRU{Глобальные переменные}{Global variables}}

\subsubsection{x86}

\IFRU
{А что если переменная \TT{x} из предыдущего примера будет глобальной переменной а не локальной? 
Тогда к ней смогут обращаться из любого другого места, а не только из тела функции. 
Это снова не очень хорошая практика программирования, но ради примера мы можем себе это позволить.}
{What if \TT{x} variable from previous example will not be local but global variable? 
Then it will be accessible from any place but not only from function body. 
It is not very good programming practice, but for the sake of experiment we could do this.}

\lstinputlisting{04_scanf/4_2_msvc.asm}

\IFRU
{Ничего особенного, в целом. Теперь \TT{x} объявлена в сегменте \TT{\_DATA}. 
Память для нее в стеке более не выделяется. Все обращения к ней происходит не через стек, а уже напрямую. 
Её значение неопределено. 
Это означает, что память под нее будет выделена, но ни компилятор, ни ОС не будет заботиться о том, 
что там будет лежать на момент старта функции \main.
В качестве домашнего задания, попробуйте объявить большой неопределенный массив и посмотреть 
что там будет лежать после загрузки.}
{Now \TT{x} variable is defined in \TT{\_DATA} segment. 
Memory in local stack is not allocated anymore. 
All accesses to it are not via stack but directly to process memory. 
Its value is not defined. 
This mean that memory will be allocated by operation system, but not compiler, 
neither operation system will not take care about its initial value at the moment of 
\main function start.
As experiment, try to declare large array and see what will it contain after 
program loading.}

\IFRU{Попробуем изменить объявление этой переменной:}{Now let's assign value to variable explicitly:}

\begin{lstlisting}
int x=10; // default value
\end{lstlisting}

\IFRU{Выйдет в итоге:}{We got:}

\begin{lstlisting}
_DATA	SEGMENT
_x	DD	0aH

...
\end{lstlisting}

\IFRU{Здесь уже по месту этой переменной записано \TT{0xA} с типом DD (dword = 32 бита).}
{Here we see value 0xA of DWORD type (DD meaning DWORD = 32 bit).}

\IFRU{Если вы откроете скомпилированный .exe-файл в \IDA, то увидите что \IT{x} 
находится аккурат в начале сегмента \TT{\_DATA}, после этой переменной будут текстовые строки.}
{If you will open compiled .exe in \IDA, you will see \IT{x} placed at the beginning of 
\TT{\_DATA} segment, and after you'll see text strings.}

\IFRU{А вот если вы откроете в \IDA, .exe скомплированный в прошлом примере, 
где значение \IT{x} неопределено, то в IDA вы увидите:}
{If you will open compiled .exe in \IDA from previous example where \IT{x} value is not defined, 
you'll see something like this:}

\begin{lstlisting}
.data:0040FA80 _x              dd ?                    ; DATA XREF: _main+10
.data:0040FA80                                         ; _main+22
.data:0040FA84 dword_40FA84    dd ?                    ; DATA XREF: _memset+1E
.data:0040FA84                                         ; unknown_libname_1+28
.data:0040FA88 dword_40FA88    dd ?                    ; DATA XREF: ___sbh_find_block+5
.data:0040FA88                                         ; ___sbh_free_block+2BC
.data:0040FA8C ; LPVOID lpMem
.data:0040FA8C lpMem           dd ?                    ; DATA XREF: ___sbh_find_block+B
.data:0040FA8C                                         ; ___sbh_free_block+2CA
.data:0040FA90 dword_40FA90    dd ?                    ; DATA XREF: _V6_HeapAlloc+13
.data:0040FA90                                         ; __calloc_impl+72
.data:0040FA94 dword_40FA94    dd ?                    ; DATA XREF: ___sbh_free_block+2FE
\end{lstlisting}

\IFRU{\TT{\_x} обозначен как \TT{?}, наряду с другими переменными не требующими инициализции. 
Это означает, что при загрузке .exe в память, место под все это выделено будет. 
Но в самом .exe ничего этого нет. Неинициализированные переменные не занимают места в исполняемых файлах. Удобно для больших массивов, например.}
{\TT{\_x} marked as \TT{?} among another variables not required to be initialized. 
This mean that after loading .exe to memory, place for all these variables will be 
allocated and some random garbage will be here. 
But in .exe file these not initialized variables are not occupy anything. 
It is suitable for large arrays, for example.}

\IFRU{В Linux все также почти. За исключением того что если значение \TT{x} не определено, 
то эта переменная будет находится в сегменте \TT{\_bss}. В ELF\footnote{Формат исполняемых файлов, использующийся в Linux и некоторых других *NIX} этот сегмент имеет такие аттрибуты:}
{It is almost the same in Linux, except segment names and properties: 
not initialized variables are located in \TT{\_bss} segment. 
In ELF\footnote{Executable file format widely used in *NIX system including Linux} 
file format this segment has such attributes:}

\begin{lstlisting}
; Segment type: Uninitialized
; Segment permissions: Read/Write
\end{lstlisting}

\IFRU{Ну а если сделать присвоение этой переменной значения 10, то она будет находится в сегменте \TT{\_data},
это сегмент с такими аттрибутами:}
{If to assign some value to variable, it will be placed in \TT{\_data} segment, 
this is segment with such attributes:}

\begin{lstlisting}
; Segment type: Pure data
; Segment permissions: Read/Write
\end{lstlisting}

\subsubsection{ARM: \OptimizingKeil + \ThumbMode}

\begin{lstlisting}
.text:00000000 ; Segment type: Pure code
.text:00000000                 AREA .text, CODE
...
.text:00000000 main
.text:00000000                 PUSH    {R4,LR}
.text:00000002                 ADR     R0, aEnterX     ; "Enter X:\n"
.text:00000004                 BL      __2printf
.text:00000008                 LDR     R1, =x
.text:0000000A                 ADR     R0, aD          ; "%d"
.text:0000000C                 BL      __0scanf
.text:00000010                 LDR     R0, =x
.text:00000012                 LDR     R1, [R0]
.text:00000014                 ADR     R0, aYouEnteredD___ ; "You entered %d...\n"
.text:00000016                 BL      __2printf
.text:0000001A                 MOVS    R0, #0
.text:0000001C                 POP     {R4,PC}
...
.text:00000020 aEnterX         DCB "Enter X:",0xA,0    ; DATA XREF: main+2
.text:0000002A                 DCB    0
.text:0000002B                 DCB    0
.text:0000002C off_2C          DCD x                   ; DATA XREF: main+8
.text:0000002C                                         ; main+10
.text:00000030 aD              DCB "%d",0              ; DATA XREF: main+A
.text:00000033                 DCB    0
.text:00000034 aYouEnteredD___ DCB "You entered %d...",0xA,0 ; DATA XREF: main+14
.text:00000047                 DCB 0
.text:00000047 ; .text         ends
.text:00000047
...
.data:00000048 ; Segment type: Pure data
.data:00000048                 AREA .data, DATA
.data:00000048                 ; ORG 0x48
.data:00000048                 EXPORT x
.data:00000048 x               DCD 0xA                 ; DATA XREF: main+8
.data:00000048                                         ; main+10
.data:00000048 ; .data         ends
\end{lstlisting}

\IFRU{Итак, переменная \TT{x} теперь глобальная, и она расположена, почему-то, в другом сегменте, 
а именно сегменте данных}{So, \TT{x} variable is now global and it's localted, and, somehow,
it's now located in other segment, namely data segment} (\IT{.data}).
\IFRU{Можно спросить, почему текстовые строки расположены в сегменте кода (\IT{.text}) 
а \TT{x} нельзя было разместить тут же?}{One could ask, why text strings are located in code segment
(\IT{.text}) and \TT{x} can be located right here?}
\IFRU{Потому что эта переменная, и как следует из определения, она может меняться. 
И может даже быть, меняться часто.}
{Since this is variable, and by its definition, it can be changed. And probably, can be changed 
very often.}
\index{\RAM}
\index{\ROM}
\IFRU{Сегмент кода нередко может быть расположен в ПЗУ микроконтроллера (не забывайте, 
мы сейчас имеем дело с embedded-микроэлектроникой, где дефицит памяти это обычное дело),
а изменяемые переменные ~--- в ОЗУ.}
{Segment of code not infrequently can be located in microcontroller ROM (remember, we now deal
with embedded microelectronics, and memory scarcity is common here), and changeable variables ~--- 
in RAM.}
\IFRU{Хранить в ОЗУ неизменяемые данные, когда в наличии есть ПЗУ, не экономно.}
{It's not very economically to store constant variables in RAM when one have ROM.}
\IFRU{К тому же, сегмент данных в ОЗУ с константами нужно было бы инициализировать перед работой,
ведь, после включения ОЗУ, очевидно, она содержит в себе случайную информацию.}
{Furthermore, data segment with constants in RAM should be initialized before, 
since after RAM turning on, obviously, it contain random information.}

\index{\IFRU{Компоновщик}{Linker}}
\IFRU{Далее, мы видим, в сегменте кода, хранится указатель на переменную \TT{x} (\TT{off\_2C}) и вообще, 
все операции с переменной, происходят через этот указатель.}
{Onwards, we see, in code segment, a pointer to the \TT{x} (\TT{off\_2C}) variable, and all
operations with variable occured via this pointer.}
\IFRU{Это связано с тем что переменная \TT{x} может быть расположена где-то довольно далеко от 
данного участка кода, так что её адрес нужно сохранить в непосредственной близости к этому коду.}
{This is because \TT{x} variable can be located somewhere far from this code fragment, so its address
should be saved somewhere in close proximity to the code.}
\IFRU{Инструкция \TT{LDR} в thumb-режиме может адресовать только переменные в пределах вплоть 
до 1020 байт от места где она находится.}
{\TT{LDR} instruction in thumb mode can address only variable in range of 1020 bytes from the point
it is located.}
\IFRU{Эта же инструкция в ARM-режиме ~--- переменные в пределах $\pm{}4095$ байт, таким образом,
адрес глобальной переменной \TT{x} нужно расположить в непосредственной близости, ведь нет никакой гарантии, 
что компоновщик\footnote{linker в англоязычной литературе} сможет разместить саму переменную где-то рядом, 
она может быть даже в другом чипе памяти!}
{Same instruction in ARM-mode ~--- variables in range $\pm{}4095$ bytes, this, address of the \TT{x} variable
should be located somewhere in close proximity, because, there are no guarantee that linker will able to place
this variable near the code, it could be even in other memory chip!}

\index{\CLanguageElements!const}
\index{\ROM}
\IFRU{Еще одна вещь: если переменную объявить как \IT{const}, то компилятор Keil разместит её в 
сегменте \TT{.constdata}.}
{One more thing: if variable will be declared as \IT{const}, Keil compiler shall allocate it in 
the \TT{.constdata} segment.}
\IFRU{Должно быть, впоследствии, компоновщик и этот сегмент сможет разместить в ПЗУ, вместе
с сегментом кода.}
{Perhaps, thereafter, linker will able to place this segment in ROM too, along with code segment.}





\subsection{\IFRU{Проверка результата scanf()}{scanf() result checking}}

\subsubsection{x86}

\IFRU {Как я уже упоминал, использовать \scanf в наше время это слегка старомодно. 
Но если уж жизнь заставила этим заниматься, нужно хотя бы проверять, сработал ли \scanf 
правильно или пользователь ввел вместо числа что-то другое, что \scanf не смог трактовать как число.}
{As I noticed before, it is slightly old-fashioned to use \scanf today. 
But if we have to, we need at least check if \scanf finished correctly without error.}

\lstinputlisting{04_scanf/retval_check.c}

\IFRU{По стандарту}{By standard}, \scanf\footnote{\href{http://msdn.microsoft.com/en-us/library/9y6s16x1(VS.71).aspx}{MSDN: scanf, wscanf}} 
\IFRU{возвращает количество успешно полученных значений.}{function returning number of fields it successfully read.}

\IFRU{В нашем случае, если все успешно и пользователь ввел таки некое число, \scanf вернет 1. 
А если нет, то 0 или EOF.} 
{In our case, if everything went fine and user entered some number, 
\scanf will return 1 or 0 or EOF in case of error.}

\IFRU{Я добавили код, который проверяет результат \scanf и в случае ошибки, говорит пользователю что-то другое.}
{I added C code for \scanf result checking and printing error message in case of error.}

\IFRU{Вот, что выходит на ассемблере}{What we got in assembly language} (MSVC 2010):

\lstinputlisting{04_scanf/retval_check_MSVC.asm}

\index{x86!\Registers!EAX}
\IFRU{Для того чтобы вызывающая функция имела доступ к результату вызываемой функции, 
вызываемая функция (в нашем случае \scanf) оставляет это значение в регистре \EAX.}
{Caller function (\main) should have access to result of callee function (\scanf), 
so callee leave this value in \EAX register.}

\index{x86!\Instructions!CMP}
\IFRU{Мы проверяем его инструкцией \TT{CMP EAX, 1} (\IT{CoMPare}), то есть, 
сравниваем значение в \EAX с 1.}
{After, we check it using instruction \TT{CMP EAX, 1} (\IT{CoMPare}), 
in other words, we compare value in \EAX with 1.} 

\index{x86!\Instructions!JNE}
\IFRU{Следующий за инструкцией \CMP: условный переход \JNE. 
Это означает \IT{Jump if Not Equal}, то есть, условный переход \IT{если не равно}.}
{\JNE conditional jump follows \CMP instruction. \JNE mean \IT{Jump if Not Equal}.}

\IFRU{Итак, если \EAX не равен 1, то \JNE заставит перейти процессор 
по адресу указанном в операнде \JNE, у нас это \TT{\$LN2@main}.}
{So, if \EAX value not equals to 1, then the processor will pass execution to the 
address mentioned in operand of \JNE, in our case it is \TT{\$LN2@main}.}
\IFRU
{Передав управление по этому адресу, процессор как раз начнет исполнять вызов \printf с 
аргументом \TT{``What you entered? Huh?''}.}
{Passing control to this address, microprocesor will execute function \printf 
with argument \TT{``What you entered? Huh?''}.}
\IFRU
{Но если все нормально, перехода не случится, и исполнится другой \printf с двумя аргументами: 
\TT{'You entered \%d...'} и значением переменной \TT{x}.}
{But if everything is fine, conditional jump will not be taken, and another \printf call 
will be executed, with two arguments: \TT{'You entered \%d...'} and value of variable \TT{x}. }

\index{x86!\Instructions!XOR}
\index{\CLanguageElements!return}
\IFRU {А для того чтобы после этого вызова не исполнился сразу второй вызов \printf, 
после него имеется инструкция \JMP, безусловный переход, он отправит процессор на место аккурат 
после второго \printf и перед инструкцией \TT{XOR EAX, EAX}, которая собственно \TT{return 0}.}
{Because second subsequent \printf not needed to be executed, there are \JMP after (unconditional jump) 
it will pass control to the place after second \printf and before \TT{XOR EAX, EAX} instruction, 
which implement \TT{return 0}.}

\index{x86!\Registers!\Flags}
\IFRU{Итак, можно сказать, что в подавляющем случае сравнение какой либо переменной с чем-то другим 
происходит при помощи пары инструкций \CMP и \Jcc, где \IT{cc} это \IT{condition code}.}
{So, it can be said that most often, comparing some value with another is implemented 
by \CMP/\Jcc instructions pair, where \IT{cc} is \IT{condition code}.}
\IFRU{\CMP сравнивает два значения и выставляет 
флаги процессора\footnote{См.также о флагах x86-процессора: \url{http://en.wikipedia.org/wiki/FLAGS_register_(computing)}.}.}
{\CMP comparing two values and set 
processor flags\footnote{About x86 flags, see also: \url{http://en.wikipedia.org/wiki/FLAGS_register_(computing)}.}.}
\IFRU
{\Jcc проверяет нужные ему флаги и выполняет переход по указанному адресу (или не выполняет).}
{\Jcc check flags needed to be checked and pass control to mentioned address (or not pass).}

\index{x86!\Instructions!CMP}
\index{x86!\Instructions!SUB}
\label{CMPandSUB}
\IFRU{Но на самом деле, как это не парадоксально поначалу звучит, \CMP это почти то же самое что и 
инструкция \SUB, которая отнимает числа одно от другого.}
{But in fact, this could be perceived paradoxial, but \CMP instruction is in fact \SUB (subtract).}
\IFRU{Все арифметические инструкции также выставляют флаги в соответствии с результатом, не только \CMP.}
{All arithmetic instructions set processor flags too, not only \CMP.}
\IFRU{Если мы сравним 1 и 1, от единицы отнимется единица, получится ноль, и выставится флаг 
\ZF (\IT{zero flag}), означающий что последний полученный результат является нулем.}
{If we compare 1 and 1, $1-1$ will be zero in result, \ZF flag will be set (meaning that last result was zero).}
\IFRU{Ни при каких других значениях \EAX, флаг \ZF выставлен не будет, кроме тех, когда операнды равны друг другу.}
{There are no any other circumstances when it's possible except when operands are equal.}
\index{x86!\Instructions!JNE}
\index{x86!\Registers!ZF}
\IFRU{Инструкция \JNE проверяет только флаг \ZF, и совершает переход только если флаг не поднят. 
Фактически, \JNE это синоним инструкции \JNZ (\IT{Jump if Not Zero}).}
{\JNE checks only \ZF flag and jumping only if it is not set. 
\JNE is in fact a synonym of \JNZ (\IT{Jump if Not Zero}) instruction.}
\IFRU{Ассемблер транслирует обе инструкции в один и тот же опкод.}
{Assembler translating both \JNE and \JNZ instructions into one single opcode.}
\IFRU
{Таким образом, можно \CMP заменить на \SUB и все будет работать также, но разница в том что \SUB 
все-таки испортит значение в первом операнде. \CMP это \IT{SUB без сохранения результата}.}
{So, \CMP can be replaced to \SUB and almost everything will be fine, but the difference is in 
that \SUB alter value at first operand. \CMP is \IT{``SUB without saving result''}.}

\IFRU
{Код созданный при помощи GCC 4.4.1 в Linux практически такой же, если не считать мелких отличий, 
которые мы уже рассмотрели раннее.}
{Code generated by GCC 4.4.1 in Linux is almost the same, except differences we already considered.}

\subsubsection{ARM: \OptimizingKeil + \ThumbMode}

\lstinputlisting[caption=\OptimizingKeil + \ThumbMode]{04_scanf/checking_retval_ARM_Keil_thumb_O3.asm}

\index{ARM!\Instructions!CMP}
\index{ARM!\Instructions!BEQ}
\IFRU{Новые инструкции здесь для нас: \CMP и \TT{BEQ}.}
{New instructions here are \CMP and \TT{BEQ}.}

\CMP \IFRU{аналогична той что в x86, она отнимает один аргумент от второго и сохраняет флаги.}
{is similar to the x86 instruction, it subtracts one argument from another and save flags.}
% TODO: в мануале ARM $op1 + NOT(op2) + 1$ вместо вычитания

\index{ARM!\Registers!Z}
\index{x86!\Instructions!JZ}
\TT{BEQ} (\IT{Branch Equal}) \IFRU{совершает переход по другому адресу, 
если операнды при сравнении были равны, 
либо если результат последнего вычисления был ноль, либо если флаг Z равен $1$.}
{is jumping to another address if operands while comparing were equal to each other, or,
if result of last computation was zero, or if Z flag is $1$.}
\IFRU{То же что и \JZ в}{Same thing as \JZ in} x86.

\IFRU{Всё остальное просто: исполнение разветвляется на две ветки, затем они сходятся там, 
где в \Rzero записывается $0$ как возвращаемое из функции значение и происходит выход из функции.}
{Everything else is simple: execution flow is forking into two branches, then the branches are 
converging at the place
where $0$ is written into \Rzero, as a value returned from the function, and then function finishing.}



