\subsection{x86}

\IFRU{Что получаем на ассемблере компилируя MSVC 2010:}
{What we got after compiling in MSVC 2010:}

\lstinputlisting{04_scanf/4_1_msvc.asm}

\IFRU{Переменная \TT{x} является локальной.}{Variable \TT{x} is local.} 

\IFRU{По стандарту \CCpp она доступна только из этой же функции и ниоткуда более. 
Так получилось, что локальные переменные располагаются в стеке. 
Может быть, можно было бы использовать и другие варианты, но в x86 это традиционно так.}
{\CCpp standard tell us it must be visible only in this function and not from any other point. 
Traditionally, local variables are placed in the stack. 
Probably, there could be other ways, but in x86 it is so.}

\index{x86!\Instructions!PUSH}
\IFRU{Следующая после пролога инструкция \TT{PUSH ECX} не ставит своей целью сохранить 
значение регистра \ECX. 
(Заметьте отсутствие сооветствующей инструкции \TT{POP ECX} в конце функции)}
{Next instruction after function prologue, \TT{PUSH ECX}, has not a goal to save \ECX state 
(notice absence of corresponding \TT{POP ECX} at the function end).}

\IFRU{Она на самом деле выделяет в стеке 4 байта для хранения \TT{x} в будущем.} 
{In fact, this instruction just allocates 4 bytes on the stack for \TT{x} variable storage.} 

\label{stack_frame}
\index{\Stack!\IFRU{Стековый фрейм}{Stack frame}}
\index{x86!\Registers!EBP}
\IFRU{Доступ к \TT{x} будет осуществляться при помощи объявленного макроса \TT{\_x\$} 
(он равен -4) и регистра \EBP указывающего на текущий фрейм.}
{\TT{x} will be accessed with the assistance of the \TT{\_x\$} macro 
(it equals to -4) and the \EBP register pointing to current frame.}

\IFRU{Вообще, во все время исполнения функции, \EBP указывает на текущий фрейм и через \TT{EBP+смещение}
можно иметь доступ как к локальным переменным функции, так и аргументам функции.} 
{Over a span of function execution, \EBP is pointing to current stack frame and it is possible 
to have an access to local variables and function arguments via \TT{EBP+offset}.}

\index{x86!\Registers!ESP}
\IFRU
{Можно было бы использовать \ESP, но он во время исполнения функции постоянно меняется. 
Так что можно сказать что \EBP это \IT{замороженное состояние} \ESP на момент начала исполнения функции.}
{It is also possible to use \ESP, but it is often changing and not very convenient.
So it can be said, the value of the \EBP is \IT{frozen state} of the value of the \ESP at the moment of function execution start.}

\IFRU
{У функции \scanf в нашем примере два аргумента.}{Function \scanf in our example has two arguments.}

\IFRU
{Первый ~--- указатель на строку содержащую \TT{``\%d''} и второй ~--- адрес переменной \TT{x}.} 
{First is pointer to the string containing \TT{``\%d''} and second~---address of variable \TT{x}.} 

\index{x86!\Instructions!LEA}
\IFRU{Вначале адрес \TT{x} помещается в регистр \EAX при помощи инструкции \TT{lea eax, DWORD PTR \_x\$[ebp]}.}
{First of all, address of the \TT{x} variable is placed into the \EAX register by \TT{lea eax, DWORD PTR \_x\$[ebp]} instruction}

\IFRU{Инструкция \LEA означает \IT{load effective address}, но со временем она изменила свою функцию}
{\LEA meaning \IT{load effective address} but over a time it changed its primary application}
~(\ref{sec:LEA}).

\IFRU{Можно сказать что в данном случае \LEA просто помещает в \EAX результат суммы значения в регистре 
\EBP и макроса \TT{\_x\$}.}
{It can be said, \LEA here just stores sum of the value in the \EBP register and \TT{\_x\$} macro to the \EAX register.}

\IFRU{Это тоже что и}{It is the same as} \TT{lea eax, [ebp-4]}.

\IFRU{Итак, от значения \EBP отнимается $4$ и помещается в \EAX.
Далее значение \EAX заталкивается в стек и вызывается \scanf.}
{So, $4$ subtracting from value in the \EBP register and result is placed to the \EAX register.
And then value in the \EAX register is pushing into stack and \scanf is called.}

\IFRU{После этого вызывается \printf. Первый аргумент вызова которого, строка:} 
{After that, \printf is called. First argument is pointer to string:} \TT{``You entered \%d...\textbackslash{}n''}.

\IFRU{Второй аргумент: \TT{mov ecx, [ebp-4]}, эта инструкция помещает в \ECX не адрес переменной \TT{x}, 
а его значение, что там сейчас находится.}
{Second argument is prepared as: \TT{mov ecx, [ebp-4]},
this instruction places to the \ECX not address of the \TT{x} variable, but its contents.}

\IFRU{Далее значение \ECX заталкивается в стек и вызывается последний \printf.}
{After, value in the \ECX is placed on the stack and the last \printf called.}

\IFRU{Попробуем тоже самое скомпилировать в Linux при помощи GCC 4.4.1:}
{Let's try to compile this code in GCC 4.4.1 under Linux:}

\lstinputlisting{04_scanf/4_1_gcc.asm}

\index{puts() \IFRU{вместо}{instead of} printf()}
\IFRU{GCC заменил первый вызов \printf на \puts, почему это было сделано, 
уже было описано раннее~(\ref{puts}).}
{GCC replaced first the \printf call to the \puts, it was already described~(\ref{puts})
why it was done.}

% TODO: rewrite
%\IFRU
%{Почему \scanf переименовали в \TT{\_\_\_isoc99\_scanf}, я честно говоря, пока не знаю.}
%{Why \scanf is renamed to \TT{\_\_\_isoc99\_scanf}, I do not know yet.}

\IFRU{Далее все как и прежде ~--- параметры заталкиваются через стек при помощи \MOV.}
{As before~---arguments are placed on the stack by \MOV instruction.}
