\section{\IFRU{Указатели}{Pointers}}
\index{\CLanguageElements!\Pointers}
\label{label_pointers}

\IFRU{Указатели также часто используются для возврата значений из функции (вспомните случай
со scanf()~\ref{label_scanf}).}
{Pointers are often used to return values from function (recall scanf() case~\ref{label_scanf}).}
\IFRU{Например, когда функции нужно вернуть сразу два значения:}
{For example, when function should return two values:}

\begin{lstlisting}
void f1 (int x, int y, int *sum, int *product)
{
	*sum=x+y;
	*product=x*y;
};

void main()
{
	int sum, product;

	f1(123, 456, &sum, &product);
	printf ("sum=%d, product=%d\n", sum, product);
};
\end{lstlisting}

\IFRU{Это компилируется в:}
{This compiling into:}

\begin{lstlisting}[caption=\Optimizing MSVC 2010]
CONST	SEGMENT
$SG3863	DB	'sum=%d, product=%d', 0aH, 00H
$SG3864	DB	'sum=%d, product=%d', 0aH, 00H
CONST	ENDS
_TEXT	SEGMENT
_x$ = 8							; size = 4
_y$ = 12						; size = 4
_sum$ = 16						; size = 4
_product$ = 20						; size = 4
f1 PROC					; f1
	mov	ecx, DWORD PTR _y$[esp-4]
	mov	eax, DWORD PTR _x$[esp-4]
	lea	edx, DWORD PTR [eax+ecx]
	imul	eax, ecx
	mov	ecx, DWORD PTR _product$[esp-4]
	push	esi
	mov	esi, DWORD PTR _sum$[esp]
	mov	DWORD PTR [esi], edx
	mov	DWORD PTR [ecx], eax
	pop	esi
	ret	0
f1 ENDP					; f1

_product$ = -8						; size = 4
_sum$ = -4						; size = 4
_main	PROC
	sub	esp, 8
	lea	eax, DWORD PTR _product$[esp+8]
	push	eax
	lea	ecx, DWORD PTR _sum$[esp+12]
	push	ecx
	push	456					; 000001c8H
	push	123					; 0000007bH
	call	f1			; f1
	mov	edx, DWORD PTR _product$[esp+24]
	mov	eax, DWORD PTR _sum$[esp+24]
	push	edx
	push	eax
	push	OFFSET $SG3863
	call	_printf

...
\end{lstlisting}

\subsection{C++ references}
\index{C++!References}

\IFRU{References в Си++ это тоже указатели, но их называют \IT{безопасными} (safe), потому что работая с ними, 
труднее сделать
ошибку}
{In C++, references are pointers as well, but they are called \IT{safe}, because it is harder to make a mistake while
working with them}\cite[8.3.2]{CPP11}.
\IFRU{Например, reference всегда должен указывать объект того же типа и не может быть NULL}
{For example, reference should be always be pointing to the object of corresponding type and cannot be NULL}
\cite[8.6]{ParashiftCPPFAQ}.
\IFRU{Более того, reference нельзя менять, нельзя его заставить указывать на другой объект (reseat)}
{Even more than that, reference cannot be changed, it is not possible to point it to another object (reseat)}
\cite[8.5]{ParashiftCPPFAQ}.

\IFRU{Если мы попробуем изменить наш пример чтобы он использовал reference вместо указателей:}
{If we will try to change are example to use references instead of pointers:}

\begin{lstlisting}
void f2 (int x, int y, int & sum, int & product)
{
	sum=x+y;
	product=x*y;
};
\end{lstlisting}

\IFRU{То выяснится что скомпилированный код абсолютно такой же:}
{Then we'll figure out the compiled code is just the same:}

\begin{lstlisting}[caption=\Optimizing MSVC 2010]
_x$ = 8							; size = 4
_y$ = 12						; size = 4
_sum$ = 16						; size = 4
_product$ = 20						; size = 4
?f2@@YAXHHAAH0@Z PROC					; f2
	mov	ecx, DWORD PTR _y$[esp-4]
	mov	eax, DWORD PTR _x$[esp-4]
	lea	edx, DWORD PTR [eax+ecx]
	imul eax, ecx
	mov ecx, DWORD PTR _product$[esp-4]
	push esi
	mov	esi, DWORD PTR _sum$[esp]
	mov	DWORD PTR [esi], edx
	mov	DWORD PTR [ecx], eax
	pop	esi
	ret	0
?f2@@YAXHHAAH0@Z ENDP					; f2
\end{lstlisting}

(\IFRU{Почему у С++ функций такие странные имена, будет описано позже}{A reason why C++ functions has such strange
names, will be described later}~\ref{namemangling}.)

