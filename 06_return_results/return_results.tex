\section{\IFRU{И еще немного о возвращаемых результатах}{One more word about results returning.}}

\newcommand{\MSDNURL}{\href{http://msdn.microsoft.com/en-us/library/7572ztz4.aspx}{MSDN: Return Values (C++)}}

\IFRU{Резльутат выполнения функции в x86 обычно возвращается\footnote{См.также: \MSDNURL} через регистр \EAX, 
а если результат имеет тип байт или символ (\IT{char}), 
то в самой младшей части \EAX ~--- \AL. Если функция возвращает число с плавающей запятой, 
то регистр FPU \STZERO будет использован.
В ARM обычно результат возвращается в регистре R0.}
{As of x86, function execution result is usually returned\footnote{See also: \MSDNURL} in 
\EAX register. 
If it's byte type or character (\IT{char}) ~--- then in lowest register \EAX part ~--- \AL. 
If function returning \Tfloat number, FPU register 
\STZERO will be used instead.
In ARM, result is usually returned in R0 register.}

\IFRU{Вот почему старые компиляторы Си не способны создавать функции возвращающие нечто большее нежели помещается 
в один регистр (обычно, тип \Tint), а когда нужно, приходится возвращать через указатели, указываемые 
в аргументах.}
{That is why old C compilers can't create functions capable of returning something not fitting in one 
register (usually type \Tint), but if one need it, one should return information via pointers passed 
in function arguments.}
\IFRU{Хотя, позже и стало возможным, вернуть, скажем, целую структуру, но этот метод до сих пор не очень популярен. 
Если функция должна вернуть структуру, вызывающая функция должна сама, скрыто и прозрачно для программиста, 
выделить место и передать указатель на него в качестве первого аргумента. Это почти то же самое 
что и сделать это вручную, но компилятор прячет это.

Небольшой пример:}
{Now it is possible, to return, let's say, whole structure, but its still not very popular. 
If function should return a large structure, caller must allocate it and pass pointer to it via first argument, 
hiddenly and transparently for programmer. 
That is almost the same as to pass pointer in first argument manually, but compiler hide this.

Small example:}

\lstinputlisting{06_return_results/6_1.c}

\dots \IFRU{получим}{what we got} (MSVC 2010 \Ox):

\lstinputlisting{06_return_results/6_1.asm}

\IFRU{Имя внутреннего макроса для передачи указателя на структуру здесь это \TT{\$T3853}.}
{Macro name for internal variable passing pointer to structure is \TT{\$T3853} here.}

