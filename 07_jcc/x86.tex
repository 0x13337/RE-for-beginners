\subsection{x86}

\subsubsection{x86 + MSVC}

\IFRU{Имеем в итоге функцию \TT{f\_signed()}:}{What we have in the \TT{f\_signed()} function:}

\lstinputlisting[caption=MSVC]{07_jcc/7_2.asm}

\index{x86!\Instructions!JLE}
\IFRU
{Первая инструкция \JLE значит \IT{Jump if Larger or Equal}. То есть, если второй операнд больше первого или 
равен ему, произойдет переход туда, где будет следующая проверка.}
{First instruction \JLE means \IT{Jump if Larger or Equal}. In other words, if second operand is 
larger than first or equal, control flow will be passed to address or label mentioned in instruction.}
\IFRU
{А если это условие не срабатывает, то есть второй операнд меньше первого, то перехода не будет, 
и сработает первый \printf.}
{But if this condition will not trigger (second operand less than first), control flow will 
not be altered and first \printf will be called.}
\index{x86!\Instructions!JNE}
\IFRU {Вторая проверка это \JNE: \IT{Jump if Not Equal}. Переход не произойдет, если операнды равны.}
{The second check is \JNE: \IT{Jump if Not Equal}. Control flow will not altered if operands are 
equals to each other.}
\index{x86!\Instructions!JGE}
\IFRU {Третья проверка \JGE: \IT{Jump if Greater or Equal} ~--- переход если второй операнд больше 
первого или равен ему.}
{The third check is \JGE: \IT{Jump if Greater or Equal} ~--- jump if second operand is larger 
than first or they are equals to each other.}
\IFRU {Кстати, если все три условных перехода сработают, ни один \printf не вызовется. 
Но, без внешнего вмешательства, это, пожалуй, невозможно.}
{By the way, if all three conditional jumps are triggered, no \printf will be called whatsoever. 
But, without special intervention, it is nearly impossible.}

\IFRU{Функция }{}\TT{f\_unsigned()} \IFRU{точно такая же, за тем исключением, что используются инструкции 
\JBE и \JAE вместо \JLE и \JGE, об этом читайте ниже:}
{function is the same, with the exception that \JBE and \JAE instructions are used here instead of 
\JLE and \JGE, see below about it:}

\subsubsection{GCC}

\index{puts() \IFRU{вместо}{instead of} printf()}
GCC 4.4.1 \IFRU{производит почти такой же код, за исключением}
{produce almost the same code, but with} \puts~\ref{puts} \IFRU{вместо}{instead of} \printf.

\IFRU{Далее функция \TT{f\_unsigned()} скомпилированная GCC:}
{Now let's take a look to the \TT{f\_unsigned()} function produced by GCC:}

\lstinputlisting[caption=GCC]{07_jcc/7_3_gcc.asm}

\index{x86!\Instructions!JBE}
\index{x86!\Instructions!JAE}
\IFRU{Здесь все то же самое, только инструкции условных переходов немного другие: 
\JBE ~--- \IT{Jump if Below or Equal} и \JAE ~--- \IT{Jump if Above or Equal}.}
{Almost the same, with exception of instructions: \JBE ~--- \IT{Jump if Below or Equal} 
and \JAE ~--- \IT{Jump if Above or Equal}.}
\IFRU{Эти инструкции (\JA/\JAE/\JB/\JBE) отличаются от \JG/\JGE/\JL/\JLE тем, 
что работают с беззнаковыми переменными.}
{These instructions (\JA/\JAE/\JB/\JBE) are distinct from \JG/\JGE/\JL/\JLE in that way, 
they works with unsigned numbers.}

\index{x86!\Instructions!JA}
\index{x86!\Instructions!JB}
\index{x86!\Instructions!JG}
\index{x86!\Instructions!JL}
\index{Signed numbers}
\IFRU{Отступление: смотрите также секцию о представлении знака в числах}
{See also section about signed number representations}~\ref{sec:signednumbers}.
\IFRU{Таким образом, увидев где используется \JG/\JL вместо \JA/\JB и наоборот, 
можно сказать почти уверенно насчет того, 
является ли тип переменной знаковым (signed) или беззнаковым (unsigned).}
{So, where we see usage of \JG/\JL instead of \JA/\JB or otherwise, 
we can almost be sure about signed or unsigned type of variable.}

\IFRU{Далее функция \main, где ничего нового для нас нет:}
{Here is also \main function, where nothing new to us:}

\lstinputlisting[caption=\main]{07_jcc/7_4.asm}
