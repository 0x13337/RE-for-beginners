\subsubsection{x86}

\IFRU{Это дает в итоге}{Result} (MSVC 2010):

\lstinputlisting[caption=MSVC 2010]{08_switch/8_2_msvc.asm}

\IFRU{Наша функция со switch()-ем, с небольшим количеством вариантов, 
это практически аналог подобной конструкции:}
{Out function with a few cases in switch(), in fact, is analogous to this construction:}

\lstinputlisting{08_switch/8_1_analogue.c}

\index{\CLanguageElements!switch}
\index{\CLanguageElements!if}
\IFRU{Когда вариантов немного и мы видим подобный код, невозможно сказать с уверенностью, был ли
в оригинальном исходном коде switch(), либо просто набор if()-ов.}
{When few cases in switch(), and we see such code, it is impossible to say with certainty, was it
switch() in source code, or just pack of if().}
\index{\SyntacticSugar}
\IFRU{То есть, switch() это синтаксический сахар для большого количества вложенных проверок 
при помощи if().}
{This means, switch() is syntactic sugar for large number of nested checks constructed using if().}

\IFRU{В самом выходном коде, в принципе, ничего особо нового для нас здесь, 
за исключением того, что компилятор зачем-то 
перекладывает входящую переменную (\TT{a}) во временную в локальном стеке \TT{v64}.}
{Nothing especially new to us in generated code,
with the exception the compiler moving 
input variable 
\TT{a} to temporary local variable \TT{tv64}.}

\IFRU{Если скомпилировать это при помощи GCC 4.4.1, то будет почти то же самое, даже с максимальной оптимизацией 
(ключ \Othree).}
{If to compile the same in GCC 4.4.1, we'll get almost the same, even with maximal optimization 
turned on (\Othree option).}

\IFRU{Попробуем, включить оптимизацию кодегенератора}
{Now let's turn on optimization in} MSVC (\Ox): \TT{cl 1.c /Fa1.asm /Ox}

\lstinputlisting[caption=MSVC]{08_switch/8_3_msvc.asm}

\IFRU{Вот здесь уже все немного по-другому, причем не без грязных хаков.}
{Here we can see even dirty hacks.}

\index{x86!\Instructions!JZ}
\index{x86!\Instructions!JE}
\index{x86!\Instructions!SUB}
\IFRU
{Первое: \TT{а} помещается в \EAX и от него отнимается 0. Звучит абсурдно, но нужно это для того, чтобы проверить, 
0 ли в \EAX был до этого? Если да, то выставится флаг \ZF (что означает что результат отнимания $0$ от числа 
стал $0$) и первый условный переход \JE (\IT{Jump if Equal} или его синоним \JZ ~--- \IT{Jump if Zero}) 
сработает на метку \TT{\$LN4@f}, где выводится сообщение \TT{'zero'}.
Если первый переход не сработал, от значения отнимается по единице, 
и если на какой-то стадии образуется в результате $0$, то сработает соответствующий переход.}
{First: the value of the \TT{a} variable is placed into \EAX and $0$ subtracted from it. Sounds absurdly, but it may needs to check if 
$0$ was in the \EAX register before? If yes, flag \ZF will be set (this also means that subtracting from $0$ is $0$) 
and first conditional jump \JE (\IT{Jump if Equal} or synonym \JZ ~--- \IT{Jump if Zero}) will be triggered 
and control flow passed to the \TT{\$LN4@f} label, where \TT{'zero'} message is begin printed. 
If first jump was not triggered, $1$ subtracted from the input value and if at some stage $0$ will be resulted, 
corresponding jump will be triggered.}

\IFRU{И в конце концов, если ни один из условных переходов не сработал, управление передается \printf
с агрументом \TT{'something unknown'}.}
{And if no jump triggered at all, control flow passed to the \printf with argument \TT{'something unknown'}.}

\label{jump_to_last_printf}
\index{\Stack}
\IFRU
{Второе: мы видим две, мягко говоря, необычные вещи: указатель на сообщение помещается в переменную \TT{a}, 
и затем \printf вызывается не через \CALL, а через \JMP. Объяснение этому простое. 
Вызывающая функция заталкивает в стек некоторое значение и через \CALL вызывает нашу функцию. 
\CALL в свою очередь затакливает в стек адрес возврата и делает безусловный переход на адрес нашей функции. 
Наша функция в самом начале (да и в любом её месте, потому что в теле функции нет ни одной инструкции, 
которая меняет что-то в стеке или в \ESP) имеет следующую разметку стека:}
{Second: we see unusual thing for us: string pointer is placed into the \TT{a} variable, and 
then \printf is called not via \CALL, but via \JMP. This could be explained simply. 
\Gls{caller} pushing to stack a value and calling our function via \CALL. 
\CALL itself pushing returning address to stack and do unconditional jump to our function address. 
Our function at any point of execution (since it do not contain any instruction moving stack 
pointer) has the following stack layout:}

\begin{itemize}
\item\ESP ~--- \IFRU{хранится}{pointing to} \ac{RA}
\item\TT{ESP+4} ~--- \IFRU{хранится значение \TT{a}}{pointing to the \TT{a} variable} 
\end{itemize}

\IFRU{С другой стороны, чтобы вызвать \printf нам нужна почти такая же разметка стека, 
только в первом аргументе нужен указатель на строку. Что, собственно, этот код и делает.}
{On the other side, when we need to call \printf here, we need exactly the same stack 
layout, except of first \printf argument pointing to string. 
And that is what our code does.}

\IFRU{Он заменяет свой первый аргумент на другой и затем передает управление \printf, как если бы вызвали не 
нашу функцию \TT{f()}, а сразу \printf. 
\printf выводит некую строку на \TT{stdout}, затем исполняет инструкцию \RET, 
которая из стека достает \ac{RA} и управление передается в ту функцию, 
которая вызывала \TT{f()}, минуя при этом саму \TT{f()}.}
{It replaces function's first argument to different and 
jumping to the \printf, as if not our function \TT{f()} was called firstly, but immediately \printf.
\printf printing a string to \TT{stdout} and then execute \RET instruction, which POPping 
\ac{RA} from stack and control flow is returned not to \TT{f()} but to the \TT{f()}'s \gls{callee}, 
escaping \TT{f()}.}

\index{\CStandardLibrary!longjmp()}
\newcommand{\URLSJ}{\url{http://en.wikipedia.org/wiki/Setjmp.h}}
\IFRU{Все это возможно потому что \printf вызывается в \TT{f()} в самом конце. 
Все это чем-то даже похоже на \TT{longjmp()}\footnote{\URLSJ}.
И все это, разумеется, сделано для экономии времени исполнения.}
{All this is possible since \printf is called right at the end of the \TT{f()} function in any case. 
In some way, it is all similar to the \TT{longjmp()}\footnote{\URLSJ} function.
And of course, it is all done for the sake of speed.}

\IFRU{Похожая ситуация с компилятором для ARM описана в секции}
{Similar case with ARM compiler described in} ``\PrintfSeveralArgumentsSectionName'', 
\IFRU{здесь}{section, here}~(\ref{ARM_B_to_printf}).
