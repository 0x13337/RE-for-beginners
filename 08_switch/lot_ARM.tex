\subsubsection{ARM: \OptimizingKeil + \ARMMode}
\label{sec:SwitchARMLot}

\lstinputlisting{08_switch/lot_ARM_ARM_O3.asm}

\IFRU{В этом коде используется та особенность режима ARM, что все инструкции в этом режиме имеют длину 4 байта.}
{This code makes use of the ARM feature in which all instructions in the ARM mode has size of 4 bytes.}

\IFRU{Итак, не будем забывать, что максимальное значение для $a$ это $4$, всё что выше, должно вызвать
вывод строки}
{Let's keep in mind the maximum value for $a$ is $4$ and any greater value must cause}
\IT{<<something unknown\textbackslash{}n>>}
\IFRU{.}
{string printing.}

\index{ARM!\Instructions!CMP}
\index{ARM!\Instructions!ADDCC}
\IFRU{Самая первая инструкция}{The very first} \TT{``CMP R0, \#5''} 
\IFRU{сравнивает входное значение в $a$ c $5$.}
{instruction compares $a$ input value with $5$.}

\IFRU{Следующая инструкция}{The next} \TT{``ADDCC PC, PC, R0,LSL\#2''}
\footnote{ADD ~--- \IFRU{складывание чисел}{addition}}
\IFRU{сработает только в случае если}{instruction will execute only if} $R0 < 5$ (\IT{CC=Carry clear / Less than}). 
\IFRU{Следвательно, если}{Consequently, if} \TT{ADDCC} \IFRU{не сработает}{will not trigger} 
(\IFRU{это случай с}{it is a} $R0 \geq 5$\IFRU{}{ case}), 
\IFRU{выполнится переход на метку}{a jump to} 
\IT{default\_case}\IFRU{.}{label will be occurred.}

\IFRU{Но если}{But if} $R0 < 5$ \AndENRU \TT{ADDCC} \IFRU{сработает, то произойдет следующее:}
{will trigger, following events will happen:}

\IFRU{Значение в \Rzero умножается на $4$}{Value in the \Rzero is multiplied by $4$}.
\IFRU{Фактически}{In fact}, \TT{LSL\#2} \IFRU{в конце инструкции означает ``сдвиг влево на 2 бита''.}
{at the instruction's ending means ``shift left by 2 bits''.}
\IFRU{Но как будет видно позже}{But as we will see later}~\ref{division_by_shifting} \IFRU{в секции}{in} 
``\ShiftsSectionName'' \IFRU{}{section}, 
\IFRU{сдвиг влево на 2 бита это как раз эквивалентно его умножению на $4$.}
{shift left by 2 bits is just equivalently to multiplying by $4$.}

\IFRU{Затем полученное}{Then,} $R0*4$ \IFRU{прибавляется к текущему значению \PC}{value we got, is added to
current value in the \PC}, 
\IFRU{совершая, таким образом, переход на одну из расположенных ниже инструкций \TT{B} (\IT{Branch}).}
{thus jumping to one of \TT{B} (\IT{Branch}) instructions located below.}

\IFRU{На момент исполнения}{At the moment of} \TT{ADDCC} \IFRU{}{execution}, 
\IFRU{содержимое \PC на 8 байт больше}{value in the \PC is 8 bytes ahead} (\TT{0x180}) 
\IFRU{чем адрес по которому расположена сама инструкция} 
{than address at which} \TT{ADDCC} \IFRU{}{instruction is located} (\TT{0x178}), 
\IFRU{либо, говоря иным языком, на 2 инструкции больше.}
{or, in other words, 2 instructions ahead.}

\index{ARM!\IFRU{Конвеер}{Pipeline}}
\IFRU{Это связано с работой конвеера процессора ARM:
пока исполняется инструкция \TT{ADDCC}, процессор уже начинает обрабатывать инструкцию после следующей, 
поэтому \PC указывает туда.}
{This is how ARM processor pipeline works: when \TT{ADDCC} instruction is executed,
the processor at the moment
is beginning to process instruction after the next one,
so that is why \PC pointing there.}

\IFRU{В случае, если $a=0$, тогда к \PC ничего не будет прибавлено, 
в \PC запишется актуальный на тот момент \PC (который больше на 8) 
и произойдет переход на метку \IT{loc\_180}, 
это на 8 байт дальше от места где находится инструкция \TT{ADDCC}.}
{If $a=0$, then nothing will be added to the value in the \PC,
and actual value in the \PC is to be written into the \PC (which is 8 bytes ahead)
and jump to the label \IT{loc\_180} will happen,
this is 8 bytes ahead of the point where \TT{ADDCC} instruction is.}

\IFRU{В случае, если}{In case of} $a=1$, \IFRU{тогда в \PC запишется}{then} 
$PC+8+a*4 = PC+8+1*4 = PC+16 = 0x184$\IFRU{, это адрес метки \IT{loc\_184}}{will be written to the \PC,
this is the address of the \IT{loc\_184} label}.

\IFRU{При каждой добавленной к $a$ единице, итоговый \PC увеличивается на $4$.}
{With every $1$ added to $a$, resulting \PC increasing by $4$.}
\IFRU{$4$ это как раз длина инструкции  в режиме ARM и одновременно с этим, 
длина каждой инструкции \TT{B}, их здесь следует 5 в ряд.}
{$4$ is also instruction length in ARM mode and also, length of each \TT{B} instruction length,
there are 5 of them in row.}

\IFRU{Каждая из этих пяти инструкций \TT{B}, передает управление дальше, где собственно и происходит то, 
что запрограммировано в}
{Each of these five \TT{B} instructions passing control further, where something is going on, 
what was programmed in}
\IT{switch()}.
\IFRU{Там происходит загрузка указателя на свою строку, итд.}
{Pointer loading to corresponding string occuring there, etc.}

\subsubsection{ARM: \OptimizingKeil + \ThumbMode}

\lstinputlisting{08_switch/lot_ARM_thumb_O3.asm}

\index{ARM!\ThumbMode}
\index{ARM!\ThumbTwoMode}
\IFRU{В режимах thumb и thumb-2, уже нельзя надеятся на то что все инструкции будут иметь одну длину.}
{One cannot be sure all instructions in thumb and thumb-2 modes will have same size.}
\IFRU{Можно даже сказать что в этих режимах инструкции переменной длины, как в x86.}
{It is even can be said that in these modes instructions has variable length, just like in x86.}

\index{jumptable}
\IFRU{Так что здесь добавляется специальная таблица, содержащая информацию о том, как много вариантов здесь,
не включая default-варианта, и смещения, для каждого варианта, каждое кодирует метку, куда нужно передать
управление в соответствующем случае.}
{So there is a special table added, containing information about how much cases are there, not including 
default-case, and offset, for each, each encoding a label, to which control must be passed in 
corresponding case.}

\index{ARM!\IFRU{Переключение режимов}{Mode switching}}
\index{ARM!\Instructions!BX}
\IFRU{Для того чтобы работать с таблицей и совершить переход, вызывается служебная функция}
{A special function here present in order to work with the table and pass control, named} \\
\IT{\_\_ARM\_common\_switch8\_thumb}. 
\IFRU{Она начинается с инструкции}{It is beginning with} \TT{``BX PC''}
\IFRU{, чья функция ~--- переключить процессор в ARM-режим.}
{instruction, which function is to switch processor to ARM-mode.}
\IFRU{Далее функция работающая с таблицей.}{Then you may see the function for table processing.} 
\IFRU{Она слишком сложная для 
рассмотрения в данном месте, так что я пропущу объяснения.}
{It is too complex for describing it here now, so I will omit elaborations.}
%TODO дописать когда-то?

\index{ARM!\Registers!Link Register}
\IFRU{Но можно отметить, что эта функция использует регистр \LR как указатель на таблицу.}
{But it is interesting to note the function uses \LR reigster as a pointer to the table.}
\IFRU{Действительно, после вызова этой функции, в \LR был записан
адрес после инструкции}{Indeed, after this function calling, \LR will contain address after} \\ 
\TT{``BL \_\_ARM\_common\_switch8\_thumb''}\IFRU{, а там как раз и начинается таблица.}
{ instruction, and the table is beginning right there.}

\IFRU{Еще можно отметить что код для этого выделен в отдельную функцию для того, чтобы и в других местах,
в похожих случаях, обрабатывались \IT{switch()}-и и не нужно было каждый раз генерировать во всех этих
местах такой фрагмент кода.}
{It is also worth noting the code is generated as a separate function in order to reuse it, in similar places,
in similar cases, for \IT{switch()} processing, so compiler will not generate same code at each point.}
% код выделен -> en?

\IDA 
\IFRU{распознала эту служебную функцию и таблицу автоматически, дописав комментарии к меткам вроде}
{successfully perceived it as a service function and table, automatically, and added commentaries to labels
like} \TT{jumptable 000000FA case 0}.

