\subsubsection{x86}

\IFRU{Имеем в итоге}{We got} (MSVC 2010):

\lstinputlisting[caption=MSVC 2010]{08_switch/8_5_msvc.asm}

\index{jumptable}
\IFRU{Здесь происходит следующее: в теле функции есть набор вызовов \printf с разными аргументами. 
Все они имеют, конечно же, адреса, а также внутренние символические метки, которые присвоил им компилятор.
Помимо всего прочего, все эти метки складываются во внутреннюю таблицу \TT{\$LN11@f}.}
{OK, what we see here is: there is a set of the \printf calls with various arguments. 
All them has not only addresses in process memory, but also internal symbolic labels assigned 
by compiler. 
Besides, all these labels are also places into \TT{\$LN11@f} internal table.}

\IFRU{В начале функции, если \TT{a} больше 4, то сразу происходит переход на метку \TT{\$LN1@f}, 
где вызывается \printf с аргументом \TT{'something unknown'}.}
{At the function beginning, if \TT{a} is greater than 4, control flow is passed to label 
\TT{\$LN1@f}, where \printf with argument \TT{'something unknown'} is called.}

\IFRU{А если \TT{a} меньше или равно 4, то это значение умножается на 4 и прибавляется адрес таблицы 
с переходами. 
Таким образом, получается адрес внутри таблицы, где лежит нужный адрес внутри тела функции. 
Например, возьмем \TT{a} равным 2. $2*4 = 8$ (ведь все элементы таблицы это адреса внутри 32-битного процесса, 
таким образом, каждый элемент занимает 4 байта). 8 прибавить к \TT{\$LN11@f} ~--- это будет элемент таблицы,
где лежит \TT{\$LN4@f}. \JMP вытаскивает из таблицы адрес \TT{\$LN4@f} и делает безусловный переход туда.}
{And if \TT{a} value is less or equals to 4, let's multiply it by 4 and add \TT{\$LN1@f} 
table address. That is how address inside of table is constructed, pointing exactly to the 
element we need. For example, let's say \TT{a} is equal to 2. $2*4 = 8$ (all table elements 
are addresses within 32-bit process, that is why all elements contain 4 bytes). 
Address of the \TT{\$LN11@f} table + 8 ~--- it will be table element where \TT{\$LN4@f} label is stored.
\JMP fetches \TT{\$LN4@f} address from the table and jump to it.}

\IFRU{Эта таблица иногда называется}{This table called sometimes} \IT{jumptable}.

\IFRU{А там вызывается \printf с агрументом \TT{'two'}. 
Дословно, инструкция \TT{jmp DWORD PTR \$LN11@f[ecx*4]} 
означает \IT{перейти по DWORD, который лежит по адресу} \TT{\$LN11@f + ecx * 4}.}
{Then corresponding \printf is called with argument \TT{'two'}. 
Literally, \TT{jmp DWORD PTR \$LN11@f[ecx*4]} instruction means
\IT{jump to DWORD, which is stored at address} \TT{\$LN11@f + ecx * 4}.}

\TT{npad}~\ref{sec:npad} 
\IFRU{это макрос ассемблера, немного выровнять начало таблицы, 
дабы она располагалась по 
адресу кратному 4 (или 16). Это нужно для того чтобы процессор мог эффективнее загружать 32-битное 
значения из памяти, через шину с памятью, кеш-память, итд.}
{is assembly language macro, aligning next label so that it will be stored at address aligned on a 4 byte
(or 16 byte) border.
This is very suitable for processor since it is able to fetch 32-bit values from memory through memory bus,
cache memory, etc, in much effective way if it is aligned.}

\IFRU{Посмотрим что сгенерирует GCC 4.4.1}{Let's see what GCC 4.4.1 generates}:

\lstinputlisting[caption=GCC 4.4.1]{08_switch/8_6_gcc.asm}

\index{x86!\Registers!JMP}
\IFRU{Практически то же самое, за исключением мелкого нюанса: аргумент из \TT{arg\_0} умножается на 4 
при помощи сдвига влево на 2 бита (это почти то же самое что и умножение на 4)~\ref{SHR}. 
Затем адрес метки внутри функции берется из массива \TT{off\_804855C} и адресуется при помощи 
вычисленного индекса.}
{It is almost the same, except little nuance: argument \TT{arg\_0} is multiplied by 4 with by
shifting it to left by 2 bits (it is almost the same as multiplication by 4)~\ref{SHR}. 
Then label address is taken from \TT{off\_804855C} array, address calculated and stored into 
\EAX, then \TT{``JMP EAX''} do actual jump.}

