\subsection{x86}

\IFRU{Итак, компилируем:}{Let's compile:}

\lstinputlisting{\IFRU{10_strlen/10_1_msvc_ru.asm}{10_strlen/10_1_msvc_en.asm}}

\IFRU{Здесь две новых инструкции: \MOVSX и \TEST.}
{Two new instructions here: \MOVSX and \TEST.}

\label{MOVSX}
\IFRU{О первой: \MOVSX предназначен для того чтобы взять байт из какого-либо места в памяти и положить его, 
в нашем случае, в регистр \EDX. 
Но регистр \EDX ~--- 32-битный. \MOVSX означает \IT{MOV with Sign-Extent}. 
Оставшиеся биты с 8-го по 31-й \MOVSX сделает единицей, если исходный байт в памяти имеет знак \IT{минус}, 
или заполнит нулями, если знак \IT{плюс}.}
{About first: \MOVSX is intended to take byte from some place and store value in 32-bit register. 
\MOVSX meaning \IT{MOV with Sign-Extent}. 
Rest bits starting at 8th till 31th \MOVSX will set to $1$ if source byte in memory has \IT{minus} 
sign or to 0 if \IT{plus}.}

\IFRU{И вот зачем все это.}{And here is why all this.}

\IFRU{По стандарту \CCpp, тип \Tchar ~--- знаковый. Если у нас есть две переменные, одна \Tchar, а другая \Tint 
(\Tint тоже знаковый), и если в первой переменной лежит $-2$ (что кодируется как $0xFE$) и мы просто 
переложим это в \Tint, 
то там будет $0x000000FE$, а это, с точки зрения \Tint, даже знакового, будет $254$, но никак не $-2$. 
$-2$ в переменной \Tint кодируется как $0xFFFFFFFE$. И для того чтобы значение $0xFE$ из переменной типа 
\Tchar переложить 
в знаковый \Tint с сохранением всего, нужно узнать его знак, и затем заполнить остальные биты. 
Это делает \MOVSX.}
{\CCpp standard defines \Tchar type as signed. If we have two values, one is \Tchar 
and another is \Tint, (\Tint is signed too), and if first value contain $-2$ (it is coded as $0xFE$) 
and we just copying this byte into \Tint container, there will be $0x000000FE$, and this, 
from the point of signed \Tint view is $254$, but not $-2$. In signed int, $-2$ is coded as $0xFFFFFFFE$. 
So if we need to transfer $0xFE$ value from variable of \Tchar type to \Tint, 
we need to identify its sign and extend it. That is what \MOVSX does.}

\IFRU{См.также об этом раздел}
{See also in section} ``\IT{\SignedNumbersSectionName}''~\ref{sec:signednumbers}.

\IFRU{Хотя, конкретно здесь, компилятору врядли была особая надобность хранить значение \Tchar в регистре \EDX 
а не его восьмибитной части, скажем, \DL. Но получилось как получилось: должно быть, 
register allocator\footnote{функция компилятора распределяющая локальные переменные по регистрам процессора}
компилятора сработал именно так.}
{I'm not sure if compiler need to store \Tchar variable in \EDX, it could take 8-bit register part 
(let's say \DL). But probably, compiler's register allocator\footnote{compiler's function 
assigning local variables to CPU registers} works like that.}

\IFRU{Позже выполняется \TT{TEST EDX, EDX}. 
Об инструкции \TEST читайте в разделе о битовых полях~\ref{sec:bitfields}.
Но конкретно здесь, эта инструкция просто проверяет состояние регистра \EDX на $0$.}
{Then we see \TT{TEST EDX, EDX}. 
About \TEST instruction, read more in section about bit fields~\ref{sec:bitfields}.
But here, this instruction just checking \EDX value, if it is equals to $0$.}

\IFRU{Попробуем}{Let's try} GCC 4.4.1:

\lstinputlisting{10_strlen/10_3_gcc.asm}

\IFRU{Результат очень похож на MSVC, вот только здесь используется \MOVZX а не \MOVSX. 
\MOVZX означает \IT{MOV with Zero-Extent}. Эта инструкция перекладывает какое-либо значение 
в регистр и остальное добивает нулями. 
Фактически, преимущество этой инструкции только в том, что она позволяет 
заменить две инструкции сразу: \TT{xor eax, eax / mov al, [...]}.}
{The result almost the same as MSVC did, but here we see \MOVZX isntead of \MOVSX. 
\MOVZX mean \IT{MOV with Zero-Extent}. 
This instruction place 8-bit or 16-bit value into 32-bit register and set the rest bits to zero. 
In fact, this instruction is handy only because it is able to replace two instructions at once: 
\TT{xor eax, eax / mov al, [...]}.}

\IFRU{С другой стороны, нам очевидно, что здесь можно было бы написать вот так: 
\TT{mov al, byte ptr [eax] / test al, al} ~--- это тоже самое, хотя старшие биты \EAX будут ``замусорены''. 
Но, будем считать, что это погрешность компилятора ~--- он не смог сделать код более экономным или более понятным. 
Строго говоря, компилятор вообще не нацелен на то чтобы генерировать понятный (для человека) код.}
{On other hand, it is obvious to us that compiler could produce that code: 
\TT{mov al, byte ptr [eax] / test al, al} ~--- it is almost the same, however, 
highest \EAX register bits will contain random noise. 
But let's think it is compiler's drawback ~--- it can't produce more understandable code. 
Strictly speaking, compiler is not obliged to emit understandable (to humans) code at all.}

\IFRU{Следующая новая инструкция для нас ~--- \SETNZ. В данном случае, если в \AL был не ноль, 
то \TT{test al, al} выставит флаг \ZF в 0, а \SETNZ, если \TT{ZF==0} 
(\IT{NZ} значит \IT{not zero}) выставит единицу в \AL. 
Смысл этой процедуры в том, что, если говорить человеческим языком, 
\IT{если AL не ноль, то выполнить переход на} \TT{loc\_80483F0}.
Компилятор выдал немного избыточный код, но не будем забывать что оптимизация выключена.}
{Next new instruction for us is \SETNZ. Here, if \AL contain not zero, \TT{test al, al} 
will set zero to \ZF flag, but \SETNZ, if \TT{ZF==0} (\IT{NZ} mean \IT{not zero}) will set 1 to \AL. 
Speaking in natural language, \IT{if AL is not zero, let's jump to loc\_80483F0}. 
Compiler emitted slightly redundant code, but let's not forget that optimization is turned off.}

\IFRU{Теперь скомпилируем все то же самое в MSVC 2010, но с включенной оптимизацией (\Ox)}
{Now let's compile all this in MSVC 2010, with optimization turned on (\Ox)}:

\lstinputlisting{\IFRU{10_strlen/10_2_ru.asm}{10_strlen/10_2_en.asm}}

\IFRU{Здесь все попроще стало. Но следует отметить, что компилятор обычно может так хорошо использовать регистры 
только на не очень больших функциях с не очень большим количеством локальных переменных.}
{Now it's all simpler. But it is needless to say that compiler could use registers such efficiently 
only in small functions with small number of local variables.}

\INC/\DEC ~--- \IFRU{это инструкции инкремента-декремента, попросту говоря: 
увеличить на единицу или уменьшить.}
{are increment/decrement instruction, in other words: add 1 to variable or subtract.}

\IFRU{Попробуем GCC 4.4.1 с влюченной оптимизацией (ключ \Othree:}
{Let's check GCC 4.4.1 with optimization turned on (\Othree key):}

\lstinputlisting{10_strlen/10_3_gcc_O3.asm}

\IFRU{Здесь GCC не очень отстает от MSVC за исключением наличия \MOVZX.} 
{Here GCC is almost the same as MSVC, except of \MOVZX presence.}

\IFRU
{Впрочем, только кроме того что почему-то используется \MOVZX, который явно можно заменить на}
{However, \MOVZX could be replaced here to} \TT{mov dl, byte ptr [eax]}.

\IFRU{Но, возможно, компилятору GCC просто проще помнить что у него под переменную типа \Tchar отведен целый 
32-битный регистр и быть уверенным в том что старшие биты регистра не будут замусорены.}
{Probably, it is simpler for GCC compiler's code generator to \IT{remember} that whole register 
is allocated for \Tchar variable and it can be sure that highest bits will not contain noise 
at any point.}

\label{strlen_NOT_ADD}
\IFRU{Далее мы видим новую для нас инструкцию \NOT. Эта инструкция инвертирует все биты в операнде. 
Можно сказать что здесь это синонимично инструкции \TT{XOR ECX, 0ffffffffh}. 
\NOT и следующая за ней инструкция \ADD вычисляют разницу указателей и отнимают от результата единицу. 
Только происходит это слегка по-другому. Сначала \ECX, где хранится указатель на \IT{str}, 
инвертируется и от него отнимается единица.}
{After, we also see new instruction \NOT. This instruction inverts all bits in operand. 
It can be said, it is synonym to \TT{XOR ECX, 0ffffffffh} instruction. 
\NOT and following \ADD calculating pointer difference and subtracting 1. 
At the beginning \ECX, where pointer to str is stored, inverted and 1 is subtracted from it.}

\IFRU{См. также раздел:}{See also:} ``\SignedNumbersSectionName''~\ref{sec:signednumbers}.

\IFRU{Иными словами, в конце функции, после цикла, происходит примерно следующее:} 
{In other words, at the end of function, just after loop body, these operations are executed:}

\begin{lstlisting}
ecx=str;
eax=eos;
ecx=(-ecx)-1; 
eax=eax+ecx
return eax
\end{lstlisting}

\dots \IFRU{что эквивалентно}{and this is equivalent to}:

\begin{lstlisting}
ecx=str;
eax=eos;
eax=eax-ecx;
eax=eax-1;
return eax
\end{lstlisting}

\IFRU
{Но почему GCC решил что так будет лучше? Снова не берусь сказать. Но я не сомневаюсь, 
что эти оба варианта работают примерно равноценно в плане эффективности и скорости.}
{Why GCC decided it would be better? I cannot be sure. 
But I'm assure that both variants are equivalent in efficiency sense.}
