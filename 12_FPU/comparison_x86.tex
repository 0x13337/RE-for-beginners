\subsubsection{x86}

\IFRU{Несмотря на кажущуюся простоту этой функции, понять как она работает будет чуть сложнее.}
{Despite simplicity of that function, it will be harder to understand how it works.}

\IFRU{Вот что выдал MSVC 2010}{MSVC 2010 generated}:

\lstinputlisting[caption=MSVC 2010]{\IFRU{12_FPU/12_4_ru.asm}{12_FPU/12_4_en.asm}}

\index{x86!\Instructions!FLD}
\IFRU{Итак, \FLD загружает \TT{\_b} в регистр \STZERO.}
{So, \FLD loading \TT{\_b} into \STZERO register.}

\newcommand{\Czero}{\TT{C0}\xspace}
\newcommand{\Ctwo}{\TT{C2}\xspace}
\newcommand{\Cthree}{\TT{C3}\xspace}
\newcommand{\CThreeBits}{\Cthree/\Ctwo/\Czero}

\index{x86!\Instructions!FCOMP}
\IFRU{\FCOMP сравнивает содержимое \STZERO с тем что лежит в \TT{\_a} и выставляет биты \CThreeBits в 
регистре статуса FPU. Это 16-битный регистр отражающий текущее состояние FPU.} 
{\FCOMP compares \STZERO register state with what is in \TT{\_a} value and set \CThreeBits bits in FPU 
status word register. This is 16-bit register reflecting current state of FPU.} 

\IFRU{Итак, биты \CThreeBits выставлены, но, к сожалению, у процессоров до Intel P6 
\footnote{Intel P6 это Pentium Pro, Pentium II, и далее} нет инструкций условного перехода,
проверяющих эти биты. 
Возможно, так сложилось исторически (вспомните о том что FPU когда-то был вообще отдельным чипом). 
А у Intel P6 появились инструкции \FCOMI/\FCOMIP/\FUCOMI/\FUCOMIP ~--- делающие тоже самое, 
только напрямую модифицирующие флаги \ZF/\PF/\CF.}
{For now \CThreeBits bits are set, but unfortunately, CPU before Intel P6
\footnote{Intel P6 is Pentium Pro, Pentium II, etc} hasn't any conditional 
jumps instructions which are checking these bits. 
Probably, it is a matter of history (remember: FPU was separate chip in past). 
Modern CPU starting at Intel P6 has \FCOMI/\FCOMIP/\FUCOMI/\FUCOMIP instructions ~--- 
which does that same, but modifies CPU flags \ZF/\PF/\CF.}

\IFRU{После этого, инструкция \FCOMP выдергивает одно значение из стека. 
Это отличает её от \FCOM, которая просто сравнивает значения, оставляя стек в таком же состоянии.}
{After bits are set, the \FCOMP instruction popping one variable from stack. 
This is what distinguish it from \FCOM, which is just comparing values, leaving the stack at the same state.}

\index{x86!\Instructions!FNSTSW}
\IFRU{\FNSTSW копирует содержимое регистра статуса в \AX. Биты \CThreeBits занимают позиции, 
соответственно, 14, 10, 8, в этих позициях они и остаются в регистре \AX, 
и все они расположены в старшей части регистра ~--- \AH.}
{\FNSTSW copies FPU status word register to \AX. Bits \CThreeBits are placed at positions 14/10/8, 
they will be at the same positions in \AX registers and all them are placed in high part of \AX ~--- \AH.}

\begin{itemize}
\item
\IFRU{Если b>a в нашем случае, то биты \CThreeBits должны быть выставлены так:}
{If b>a in our example, then \CThreeBits bits will be set as following:} 0, 0, 0.
\item
\IFRU{Если a>b, то биты будут выставлены:}{If a>b, then bits will be set:} 0, 0, 1.
\item
\IFRU{Если a=b, то биты будут выставлены так:}{If a=b, then bits will be set:} 1, 0, 0.
\end{itemize}
% TODO: table here?

\IFRU{После исполнения \TT{test ah, 5}, бит \Cthree и \TT{C1} сбросится в ноль, 
на позициях 0 и 2 (внутри регистра \AH) 
останутся соответственно \Czero и \Ctwo.}
{After \TT{test ah, 5} execution, bits \Cthree and \TT{C1} will be set to 0, 
but at positions 0 and 2 (in \AH registers) 
\Czero and \Ctwo bits will be leaved.}

\index{x86!\Registers!\IFRU{Флаг четности}{Parity flag}}
\IFRU{Теперь немного о \IT{parity flag}\footnote{флаг четности}. Еще один замечательный рудимент}
{Now let's talk about parity flag. Another notable epoch rudiment}:

\begin{framed}
\begin{quotation}
One common reason to test the parity flag actually has nothing to do with parity. The FPU has four condition flags 
(C0 to C3), but they can not be tested directly, and must instead be first copied to the flags register. 
When this happens, C0 is placed in the carry flag, C2 in the parity flag and C3 in the zero flag. 
The C2 flag is set when e.g. incomparable floating point values (NaN or unsupported format) are compared 
with the FUCOM instructions.\footnote{\href{http://en.wikipedia.org/wiki/Parity_flag}{wikipedia: parity flag}}
\end{quotation}
\end{framed}

\IFRU{Этот флаг выставляется в 1 если количество единиц в последнем результате ~--- четно. 
И в ноль если ~--- нечетно.}
{This flag is to be set to 1 if ones number is even. And to zero if odd.}

\index{x86!\Instructions!JP}
\IFRU{Таким образом, что мы имеем, флаг \PF будет выставлен в единицу, если \Czero и \Ctwo 
оба нули или оба единицы. 
И тогда сработает последующий \JP (\IT{jump if PF==1}). 
Если мы вернемся чуть назад и посмотрим значения \CThreeBits 
для разных вариантов, то увидим, что условный переход \JP сработает в двух случаях: если b>a или если a==b 
(ведь бит \Cthree уже \IT{вылетел} после исполнения \TT{test ah, 5}).}
{Thus, \PF flag will be set to 1 if both \Czero and \Ctwo are set to zero or both are ones. 
And then following \JP (\IT{jump if PF==1}) will be triggered. 
If we recall values of \CThreeBits for different cases, we will see that conditional jump 
\JP will be triggered in two cases: if b>a or a==b 
(\Cthree bit is already not considering here, because it was cleared while execution of 
\TT{test ah, 5} instruction).}

\IFRU{Дальше все просто. Если условный переход сработал, то \FLD загрузит значение \TT{\_b} в \STZERO, 
а если не сработал, то загрузится \TT{\_a} и произойдет выход из функции.}
{It is all simple thereafter. If conditional jump was triggered, \FLD will load \TT{\_b} value 
to \STZERO, and if it's not triggered, \TT{\_a} will be loaded.}

\IFRU{Но это еще не все!}{But it is not over yet!}

\subsubsection{\IFRU{А теперь скомпилируем все это в MSVC 2010 с опцией \Ox}
{Now let's compile it with MSVC 2010 with optimization option \Ox}}

\lstinputlisting[caption=\Optimizing MSVC 2010]{\IFRU{12_FPU/12_5_ru.asm}{12_FPU/12_5_en.asm}}

\index{x86!\Instructions!FCOM}
\IFRU{\FCOM отличается от \FCOMP тем что просто сравнивает значения и оставляет стек в том же состоянии. 
В отличие от предыдущего примера, операнды здесь в другом порядке. 
Поэтому и результат сравнения в \CThreeBits будет другим чем раньше:}
{\FCOM is distinguished from \FCOMP is that sense that it just comparing values and leave FPU stack 
in the same state. 
Unlike previous example, operands here in reversed order. 
And that is why result of comparision in \CThreeBits will be different:}

\begin{itemize}
\item
\IFRU{Если a>b в нашем случае, то биты \CThreeBits должны быть выставлены так:}
{If a>b in our example, then \CThreeBits bits will be set as:} 0, 0, 0.
\item
\IFRU{Если b>a, то биты будут выставлены:}{If b>a, then bits will be set as:} 0, 0, 1.
\item
\IFRU{Если a=b, то биты будут выставлены так:}{If a=b, then bits will be set as:} 1, 0, 0.
\end{itemize}
% TODO: table?

\IFRU{Инструкция \TT{test ah, 65} как бы оставляет только два бита ~--- \Cthree и \Czero. 
Они оба будут нулями, если a>b: в таком случае переход \JNE не сработает. 
Далее имеется инструкция \TT{FSTP ST(1)} ~--- эта инструкция копирует 
значение \STZERO в указанный операнд и выдергивает одно значение из стека. В данном случае, 
она копирует \STZERO 
(где сейчас лежит \TT{\_a}) в \STONE. 
После этого на вершине стека два раза лежат \TT{\_a}. Затем одно значение выдергивается. 
После этого в \STZERO остается \TT{\_a} и функция завершается.}
{It can be said, \TT{test ah, 65} instruction just leave two bits ~--- \Cthree and \Czero. 
Both will be zeroes if a>b: in that case \JNE jump will not be triggered. 
Then \TT{FSTP ST(1)} is following ~--- this instruction copies \STZERO value into operand and 
popping one value from FPU stack.
In other words, that instruction copies \STZERO (where \TT{\_a} value now) into \STONE. 
After that, two values of {\_a} are at the top of stack now. 
After that, one value is popping.
After that, \STZERO will contain {\_a} and function is finished.}

\IFRU{Условный переход \JNE сработает в двух других случаях: если b>a или a==b. 
\STZERO скопируется в \STZERO, что как бы холостая операция, 
затем одно значение из стека вылетит и на вершине стека останется то что 
до этого лежало в \STONE (то есть, \TT{\_b}). И функция завершится. 
Эта инструкция используется здесь видимо потому что в FPU нет инструкции которая просто выдергивает 
значение из стека и больше ничего.}
{Conditional jump \JNE is triggered in two cases: of b>a or a==b. 
\STZERO into \STZERO will be copied, it is just like idle (\NOP) operation, then one value 
is popping from stack and top of stack (\STZERO) will contain what was in \STONE before 
(that is {\_b}). Then function finishes. 
That instruction used here probably because FPU has no instruction to pop value from stack and 
not to store it anywhere.}

\IFRU{Но и это еще не все.}{Well, but it is still not over.}

\subsubsection{GCC 4.4.1}

\lstinputlisting[caption=GCC 4.4.1]{\IFRU{12_FPU/12_6_ru.asm}{12_FPU/12_6_en.asm}}

\index{x86!\Instructions!FUCOMPP}
\IFRU{\FUCOMPP ~--- это почти то же что и \FCOM, только выкидывает из стека оба значения после сравнения, 
а также несколько иначе реагирует на ``не-числа''.}
{\FUCOMPP ~--- is almost like \FCOM, but popping both values from stack and handling 
``not-a-numbers'' differently.}

\index{\IFRU{Не-числа}{Non-a-numbers} (NaNs)}
\IFRU{Немного о \IT{не-числах}}{More about \IT{not-a-numbers}}:

\newcommand{\NANFN}{\IFRU{\footnote{\url{http://ru.wikipedia.org/wiki/NaN}}}
{\footnote{\url{http://en.wikipedia.org/wiki/NaN}}}}

\IFRU
{FPU умеет работать со специальными переменными, которые числами не являются и называются ``не числа'' или 
NaN\NANFN{}. 
Это бесконечность, результат деления на ноль, и так далее. Нечисла бывают ``тихие'' и ``сигнализирующие''. 
С первыми можно продолжать работать и далее, а вот если вы попытаетесь совершить какую-то операцию 
с сигнализирующим нечислом, то сработает исключение.}
{FPU is able to work with special values which are \IT{not-a-numbers} or 
NaNs\NANFN{}. 
These are infinity, result of dividing by zero, etc. 
Not-a-numbers can be ``quiet'' and ``signalling''. It is possible to continue to work with ``quiet'' NaNs, 
but if one try to do some operation with ``signalling'' NaNs ~--- an exception will be raised.}

\index{x86!\Instructions!FCOM}
\index{x86!\Instructions!FUCOM}
\IFRU{Так вот, \FCOM вызовет исключение если любой из операндов ~--- какое-либо нечисло.
\FUCOM же вызовет исключение только если один из операндов именно ``сигнализирующее нечисло''.}
{\FCOM will raise exception if any operand ~--- NaN. 
\FUCOM will raise exception only if any operand ~--- signalling NaN (SNaN).}

\index{x86!\Instructions!SAHF}
\IFRU{Далее мы видим \SAHF ~--- это довольно редкая инструкция в коде не использущим FPU. 
8 бит из \AH перекладываются в младшие 8 бит регистра статуса процессора в таком порядке: 
\TT{SF:ZF:-:AF:-:PF:-:CF <- AH}.}
{The following instruction is \SAHF ~--- this is rare instruction in the code which is not use FPU. 
8 bits from AH is movinto into lower 8 bits of CPU flags in the following order: 
\TT{SF:ZF:-:AF:-:PF:-:CF <- AH}.}

\index{x86!\Instructions!FNSTSW}
\IFRU{Вспомним, что \FNSTSW перегружает интересующие нас биты \CThreeBits в \AH, 
и соответственно они будут в позициях 6, 2, 0 в регистре \AH.}
{Let's remember that \FNSTSW is moving interesting for us bits \CThreeBits into \AH 
and they will be in positions 6, 2, 0 in \AH register.}

\IFRU{Иными словами, пара инструкций \TT{fnstsw  ax / sahf} перекладывает биты \CThreeBits в флаги \ZF, \PF, \CF.}
{In other words, \TT{fnstsw  ax / sahf} instruction pair is moving \CThreeBits into \ZF, \PF, \CF CPU flags.}

\IFRU{Теперь снова вспомним, какие значения бит \CThreeBits будут при каких результатах сравнения:}
{Now let's also recall, what values of \CThreeBits bits will be set:}

\begin{itemize}
\item
\IFRU{Если a больше b в нашем случае, то биты \CThreeBits должны быть выставлены так:}
{If a is greater than b in our example, then \CThreeBits bits will be set as:} 0, 0, 0.
\item
\IFRU{Если a меньше b, то биты будут выставлены:}{if a is less than b, then bits will be set as:} 0, 0, 1.
\item
\IFRU{Если a=b, то биты будут выставлены так:}{If a=b, then bits will be set:} 1, 0, 0.
\end{itemize}
% TODO: table?

\IFRU{Иными словами, после инструкций \FUCOMPP/\FNSTSW/\SAHF, мы получим такое состояние флагов:}
{In other words, after \FUCOMPP/\FNSTSW/\SAHF instructions, we will have these CPU flags states:}

\begin{itemize}
\item
\IFRU{Если a>b в нашем случае, то флаги будут выставлены так:}
{If a>b, CPU flags will be set as:} \TT{ZF=0, PF=0, CF=0}.
\item
\IFRU{Если a<b, то флаги будут выставлены:}{If a<b, then CPU flags will be set as:} \TT{ZF=0, PF=0, CF=1}.
\item
\IFRU{Если a=b, то флаги будут выставлены так:}{If a=b, then CPU flags will be set as:} \TT{ZF=1, PF=0, CF=0}.
\end{itemize}
% TODO: table?

\index{x86!\Instructions!SETNBE}
\index{x86!\Instructions!SETcc}
\index{x86!\Instructions!JNBE}
\IFRU{Инструкция \SETNBE выставит в \AL единицу или ноль, в зависимости от флагов и условий. 
Это почти аналог \JNBE, за тем лишь исключением, что \SETcc
\footnote{\IT{cc} это \IT{condition code}}
выставляет 1 или 0 в \AL, а \Jcc делает переход или нет. 
\SETNBE запишет 1 если только \TT{CF=0} и \TT{ZF=0}. Если это не так, то запишет 0 в \AL.}
{How \SETNBE instruction will store 1 or 0 to AL: it is depends of CPU flags. 
It is almost \JNBE instruction counterpart, with exception that \SETcc 
\footnote{\IT{cc} is \IT{condition code}} is storing 1 or 0 to \AL, but \Jcc do actual jump or not. 
\SETNBE store 1 only if \TT{CF=0} and \TT{ZF=0}. If it is not true, zero will be stored into \AL.}

\IFRU{\CF будет 0 и \ZF будет 0 одновременно только в одном случае: если a>b.}
{Both \CF is 0 and \ZF is 0 simultaneously only in one case: if a>b.}

\IFRU{Тогда в \AL будет записана единица, последующий условный переход \JZ взят не будет, 
и функция вернет \TT{\_a}. 
В остальных случаях, функция вернет \TT{\_b}.}
{Then one will be stored to \AL and the following \JZ will not be triggered and function will 
return {\_a}. On all other cases, {\_b} will be returned.}

\IFRU{Но и это еще не конец.}{But it is still not over.}

\subsubsection{GCC 4.4.1 \IFRU{с оптимизацией \Othree}{with \Othree optimization turned on}}

\lstinputlisting[caption=\Optimizing GCC 4.4.1]{\IFRU{12_FPU/12_7_ru.asm}{12_FPU/12_7_en.asm}}

\IFRU{Почти все что здесь есть уже описано мною, кроме одного: использование \JA после \SAHF. 
Действительно, инструкции условных переходов ``больше'', ``меньше'', ``равно'' для сравнения беззнаковых чисел 
(\JA, \JAE, \JB, \JBE, \JE/\JZ, \JNA, \JNAE, \JNB, \JNBE, \JNE/\JNZ) проверяют только флаги \CF и \ZF. 
И биты \CThreeBits после сравнения перекладываются в эти флаги аккурат так, 
чтобы перечисленные инструкции переходов могли работать. \JA сработает если \CF и \ZF обнулены.}
{It is almost the same except one: \JA usage instead of \SAHF. 
Actually, conditional jump instructions checking ``larger'', ``lesser'' or ``equal'' for unsigned number comparison 
(\JA, \JAE, \JB, \JBE, \JE/\JZ, \JNA, \JNAE, \JNB, \JNBE, \JNE/\JNZ) are checking only \CF and \ZF flags. 
And \CThreeBits bits after comparison are moving into these flags exactly in the same fashion 
so conditional jumps will work here. \JA will work if both \CF are \ZF zero.}

\IFRU{Таким образом, перечисленные инструкции условного перехода можно использовать после инструкций \FNSTSW/\SAHF.}
{Thereby, conditional jumps instructions listed here can be used after \FNSTSW/\SAHF instructions pair.}

\IFRU{Вполне возможно что биты статуса FPU \CThreeBits преднамерено были размещены таким образом, 
чтобы переноситься на базовые флаги процессора без перестановок.}
{It seems, FPU \CThreeBits status bits was placed there intentionally so to map them to base CPU flags 
without additional permutations.}
