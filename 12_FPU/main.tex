\section{\IFRU{Работа с FPU}{Work with FPU}}
\label{sec:FPU}

\newcommand{\FNURLSTACK}{\footnote{\url{http://en.wikipedia.org/wiki/Stack_machine}}}
\newcommand{\FNURLFORTH}{\footnote{\url{http://en.wikipedia.org/wiki/Forth_(programming_language)}}}
\newcommand{\FNURLIEEE}{\footnote{\url{http://en.wikipedia.org/wiki/IEEE_754-2008}}}
\newcommand{\FNURLSP}{\footnote{\url{http://en.wikipedia.org/wiki/Single-precision_floating-point_format}}}
\newcommand{\FNURLDP}{\footnote{\url{http://en.wikipedia.org/wiki/Double-precision_floating-point_format}}}
\newcommand{\FNURLEP}{\footnote{\url{http://en.wikipedia.org/wiki/Extended_precision}}}

FPU (\IT{Floating-point unit}) ~--- \IFRU{блок в процессоре работающий с числами с плавающей запятой.}
{is a device within main CPU specially designed to work with floating point numbers.}

\IFRU{Раньше он назывался сопроцессором. Он немного похож на программируемый калькулятор и 
стоит немного в стороне от основного процессора.}
{It was called coprocessor in past. It looks like programmable calculator in some way and 
stay aside of main processor.}

\IFRU{Перед изучением FPU полезно ознакомиться с тем как работают стековые машины\FNURLSTACK, 
или ознакомиться с основами языка Forth\FNURLFORTH.}
{It is worth to study stack machines\FNURLSTACK{} before FPU studying, or learn Forth language basics\FNURLFORTH.}

\index{x86!80486}
\IFRU{Интересен факт, что в свое время (до 80486) сопроцессор был отдельным чипом на материнской плате, 
и вследствии его высокой цены, он стоял не всегда. Его можно было докупить отдельно и поставить.}
{It is interesting to know that in past (before 80486 CPU) coprocessor was a separate chip 
and it was not always settled on motherboard. It was possible to buy it separately and install.}

\IFRU{Начиная с процессора 80486, FPU уже всегда входит в его состав.}
{Starting at 80486 CPU, FPU is always present in it.}

\index{IEEE 754}
\IFRU{FPU имеет стек из восьми 80-битных регистров, каждый может содержать число в формате IEEE 754\FNURLIEEE.}
{FPU has a stack capable to hold 8 80-bit registers, each register can hold a number 
in IEEE 754\FNURLIEEE format.}

\index{float}
\index{double}
\IFRU{В \CCpp имеются два типа для работы с числами с плавающей запятой, 
это \Tfloat (\IT{число одинарной точности}\FNURLSP, 32 бита)
\footnote{Формат представления float-чисел затрагивается в разделе 
\IT{\WorkingWithFloatAsWithStructSubSubSectionName}~\ref{sec:floatasstruct}.}
и \Tdouble (\IT{число двойной точности}\FNURLDP, 64 бита).}
{\CCpp language offer at least two floating number types, \Tfloat (\IT{single-precision}\FNURLSP, 32 bits)
\footnote{single precision float numbers format is also addressed in 
the \IT{\WorkingWithFloatAsWithStructSubSubSectionName}~\ref{sec:floatasstruct} section}
and \Tdouble (\IT{double-precision}\FNURLDP, 64 bits).}

\index{long double}
\IFRU{GCC также поддерживает тип \IT{long double} (\IT{extended precision}\FNURLEP, 80 бит), но MSVC ~--- нет.}
{GCC also supports \IT{long double} type (\IT{extended precision}\FNURLEP, 80 bit) but MSVC is not.}

\IFRU{Не смотря на то что \Tfloat занимает столько же места сколько \Tint на 32-битной архитектуре, 
представление чисел, разумеется, совершенно другое.}
{\Tfloat type requires the same number of bits as \Tint type in 32-bit environment, 
but number representation is completely different.}

\IFRU{Число с плавающей точкой состоит из знака, мантиссы\footnote{\IT{significand} или \IT{fraction} 
в анлоязчной литературе} и экспоненты.}
{Number consisting of sign, significand (also called \IT{fraction}) and exponent.}

\IFRU{Функция, имеющая \Tfloat или \Tdouble среди аргументов, получает эти значения через стек. 
Если функция возвращает \Tfloat или \Tdouble, она оставляет значение в регистре \STZERO ~--- то есть, 
на вершине FPU-стека.}
{Function having \Tfloat or \Tdouble among argument list is getting that value via stack. 
If function returns \Tfloat or \Tdouble value, it leaves that value in the \STZERO
register ~--- at top of FPU stack.}

\subsection{\IFRU{Простой пример}{Simple example}}

\IFRU{Рассмотрим простой пример}{Let's consider simple example}:

\lstinputlisting{12_FPU/simple.c}

\subsubsection{x86}

\IFRU{Компилируем в}{Compile it in} MSVC 2010:

\lstinputlisting[caption=MSVC 2010]{\IFRU{12_FPU/12_1_ru.asm}{12_FPU/12_1_en.asm}}

\IFRU{\FLD берет 8 байт из стека и загружает из в регистр \STZERO, автоматически конвертируя во внутренний 
80-битный формат (\IT{extended precision}).}
{\FLD takes 8 bytes from stack and load the number into \STZERO register, automatically converting 
it into internal 80-bit format \IT{extended precision}).}

\index{x86!\Instructions!FDIV}
\IFRU{\FDIV делит содержимое регистра \STZERO на число лежащее по адресу \TT{\_\_real@40091eb851eb851f} ~--- 
там закодировано значение 3.14. Синтаксис ассемблера не поддерживает подобные числа, 
так что то что мы там видим, это шестандцатиричное представление числа \IT{3.14} в формате IEEE 754.}
{\FDIV divide value in register \STZERO by number storing at address \TT{\_\_real@40091eb851eb851f} ~--- 
3.14 value is coded there. Assembler syntax missing floating point numbers, so, 
what we see here is hexadecimal representation of \IT{3.14} number in 64-bit IEEE 754 encoded.}

\IFRU{После выполнения \FDIV, в \STZERO остается частное\FNQUOTIENT.}
{After \FDIV execution, \STZERO will hold quotient\FNQUOTIENT.}

\index{x86!\Instructions!FDIVP}
\IFRU{Кстати, есть еще инструкция \FDIVP, которая делит \STONE на \STZERO, 
выталкивает эти числа из стека и заталкивает результат. 
Если вы знаете язык Forth\FNURLFORTH, то это как раз оно и есть ~--- стековая машина\FNURLSTACK.}
{By the way, there are also \FDIVP instruction, which divide \STONE by \STZERO, 
popping both these values from stack and then pushing result. 
If you know Forth language\FNURLFORTH, you will quickly understand that this is stack machine\FNURLSTACK.}

\IFRU{Следующая \FLD заталкивает в стек значение \IT{b}.}
{Next \FLD instruction pushing \IT{b} value into stack.}

\IFRU{После этого, в \STONE перемещается результат деления, а в \STZERO теперь будет \IT{b}.}
{After that, quotient is placed to \STONE, and \STZERO will hold \IT{b} value.}

\index{x86!\Instructions!FMUL}
\IFRU{Следующий \FMUL умножает \IT{b} из \STZERO на значение \TT{\_\_real@4010666666666666} ~--- 
там лежит число 4.1, и оставляет результат в \STZERO.}
{Next \FMUL instruction do multiplication: \IT{b} from \STZERO register by value at 
\TT{\_\_real@4010666666666666} (4.1 number is there) and leaves result in \STZERO.}

\index{x86!\Instructions!FADDP}
\IFRU{Самая последняя инструкция \FADDP складывает два значения из вершины стека, 
в \STONE и затем выталкивает значение лежащее в \STZERO, 
таким образом результат сложения остается на вершине стека в \STZERO.}
{Very last \FADDP instruction adds two values at top of stack, placing result at \STONE 
register and then popping value at \STONE, hereby leaving result at top of stack in \STZERO.}

\IFRU{Функция должна вернуть результат в \STZERO, так что больше ничего здесь не производится, 
кроме эпилога функции.}
{The function must return result in \STZERO register, so, after \FADDP there are no any 
other code except of function epilogue.}

\IFRU{GCC 4.4.1 (с опцией \Othree) генерирует похожий код, хотя и с некоторой разницей:}
{GCC 4.4.1 (with \Othree option) emitting the same code, however, slightly different:}

\lstinputlisting[caption=\Optimizing GCC 4.4.1]{\IFRU{12_FPU/12_2_ru.asm}{12_FPU/12_2_en.asm}}

\IFRU{Разница в том, что в стек сначала заталкивается 3.14 (в \STZERO), а затем значение 
из \TT{arg\_0} делится на то что лежит в регистре \STZERO.}
{The difference is that, first of all, 3.14 is pushing to stack (into \STZERO), and then value 
in \TT{arg\_0} is dividing by what is in \STZERO register.}

\index{x86!\Instructions!FDIVR}
\IFRU{\FDIVR означает \IT{Reverse Divide} ~--- делить поменяв делитель и делимое местами. 
Точно такой же инструкции для умножения нет, потому что она была бы бессмыслена (ведь умножение ~--- 
операция коммутативная), так что остается только \FMUL без соответсвующей ей \TT{-R} инструкции.}
{\FDIVR meaning \IT{Reverse Divide} ~--- to divide with divisor and dividend swapped with each other. 
There are no such instruction for multiplication, because multiplication is 
commutative operation, so we have just \FMUL without its \TT{-R} counterpart.}

\index{x86!\Instructions!FADDP}
\IFRU{\FADDP не только складывает два значения, но также и выталкивает из стека одно значение. 
После этого, в \STZERO остается только результат сложения.}
{\FADDP adding two values but also popping one value from stack. 
After that operation, \STZERO will contain sum.}

\IFRU{Этот фрагмент кода получен при помощи \IDA, которая регистр \STZERO называет для краткости просто \TT{ST}.}
{This fragment of disassembled code was produced using \IDA which named \STZERO register as \TT{ST} for short.}



\subsubsection{ARM: \OptimizingXcode + \ARMMode}

\IFRU{Пока в ARM не было стандартного набора инструкций для работы с плавающей точкой}
{Until ARM has floating standardized point support}, \IFRU{разные производители процессоров
могли добавлять свои расширения для работы с ними}{several processor manufacturers may add their own 
instructions extensions}.
\IFRU{Позже, был принят стандарт}{Then, } VFP (\IT{Vector Floating Point})\IFRU{}{ was standardized}.

\IFRU{Важное отличие от x86 в том, что там вы работаете с FPU-стеком, а здесь стека нет, 
здесь вы работаете просто с регистрами.}
{One important difference from x86, there you working with FPU-stack, but here, in ARM, there
are no any stack, you work just with registers.}

\lstinputlisting{12_FPU/simple_Xcode_ARM_O3.asm}

\index{ARM!D-\IFRU{регистры}{registers}}
\index{ARM!S-\IFRU{регистры}{registers}}
\IFRU{Итак, здесь мы видим использование новых регистров, с префиксом D.}
{So, we see here new registers used, with D prefix.}
\IFRU{Это 64-битные регистры, их 32, и их можно
использовать и для чисел с плавающей точкой двойной точности (double) и для 
SIMD (в ARM это называется NEON).}
{These are 64-bit registers, there are 32 of them, and these can be used both for floating-point numbers 
(double), but also for SIMD (it's called NEON here in ARM).}

\IFRU{Имеются также 32 32-битных S-регистра, они применяются для работы с числами 
с плавающей точкой одинарной точности (float).}
{There are also 32 32-bit S-registers, they are intended to be used for single precision 
floating pointer numbers (float).}

\IFRU{Запомнить легко: D-регистры предназначены для чисел double-точности, 
а S-регистры ~--- для чисел single-точности.}
{It's easy to remember: D-registers are intended for double precision numbers, while S-registers ~---
for single precision numbers.}

\IFRU{Обе константы}{Both} ($3.14$ \IFRU{и}{and} $4.1$) \IFRU{хранятся в памяти в формате IEEE 754.}
{constants are stored in memory in IEEE 754 form.}

\index{ARM!\Instructions!VLDR}
\index{ARM!\Instructions!VMOV}
\IFRU{Инструкции }{}\TT{VLDR} \IFRU{и}{and} \TT{VMOV}
\IFRU{, как можно догадаться, это аналоги обычных \TT{LDR} и \MOV, но они работают с D-регистрами.}
{instructions, as it can be easily deduced, are analogous to \TT{LDR} and \MOV, 
but they works with D-registers.}
\IFRU{Важно отметить, что эти инструкции, как и D-регистры, предназначены не только для работы 
с числами с плавающей точкой, но пригодны также и для работы с SIMD (NEON), и позже это также будет видно.}
{It should be noted that these instructions, just like D-registers, are intended not only for
floating point numbers, but can be also used for SIMD (NEON) operations and this will also be revealed soon.}

\IFRU{Аргументы передаются в функцию обычным путем, через R-регистры, однако, 
каждое число имеющее двойную точность
занимает 64 бита, так что для передачи каждого нужны два R-регистра.}
{Arguments are passed to function by usual way, via R-registers, however,
each number having double precision has size 64-bits, so, for passing each, two R-registers are needed.}

\TT{``VMOV D17, R0, R1''} \IFRU{в самом начале составляет два 32-битных значения из \Rzero и \Rone 
в одно 64-битное и сохраняет в}
{at the very beginning, composing two 32-bit values from \Rzero and \Rone into one 64-bit value
and saves it to} \TT{D17}.

\TT{``VMOV R0, R1, D16''} \IFRU{в конце это обратная процедура}{is inverse operation}, 
\IFRU{то что было в}{what was in} \TT{D16} 
\IFRU{остается в двух регистрах}{leaving in two} \Rzero \IFRU{и}{and} \Rone\IFRU{}{ registers},
\IFRU{потому что}{because}, \IFRU{число с двойной точностью}{double-precision number}, 
\IFRU{занимающее 64 бита}{needing 64 bits for storage}, \IFRU{возвращается в паре регистров \Rzero и \Rone}
{returning in \Rzero and \Rone registers pair}.

\index{ARM!\Instructions!VDIV}
\index{ARM!\Instructions!VMUL}
\index{ARM!\Instructions!VADD}
\TT{VDIV}, \TT{VMUL} \IFRU{и}{and} \TT{VADD}, \IFRU{это, собственно, инструкции для работы с числами 
с плавающей точкой, вычисляющие, соответственно, частное\FNQUOTIENT, произведение\FNPRODUCT и сумму\FNSUM.}
{are instruction for floating point numbers processing, computing, quotient\FNQUOTIENT, 
product\FNPRODUCT and sum\FNSUM, respectively.}

\IFRU{Код для thumb-2 такой же.}{The code for thumb-2 is same.}

\subsubsection{ARM: \OptimizingKeil + \ThumbMode}

\lstinputlisting{12_FPU/simple_Keil_O3_thumb.asm}

\IFRU{Keil компилировал для процессора, в котором может и не быть поддержки FPU или NEON.}
{Keil generates for processors not supporting FPU or NEON.}
\IFRU{Так что числа с двойной точностью передаются в парах обычных R-регистров}
{So, double-precision floating numbers are passed via usual R-registers},
\IFRU{а вместо FPU-инструкций вызываются сервисные библиотечные функции}
{and instead of FPU-instructions, service library functions are called, like}
\TT{\_\_aeabi\_dmul}, \TT{\_\_aeabi\_ddiv}, \TT{\_\_aeabi\_dadd}, 
\IFRU{эмулирующие умножение, деление и сложение чисел с плавающей точкой.}
{which emulating multiplication, division and addition floating-point numbers.}
\IFRU{Конечно, это медленнее чем FPU-сопроцессор, но лучше чем ничего.}
{Of course, that's slower than FPU-coprocessor, but it's better than nothing.}

\IFRU{Кстати, похожие библиотеки для эмуляции сопроцессорных инструкций были очень распространены в x86, 
когда сопроцессор был редким и дорогим, и стоял далеко не на всех компьютерах.}
{By the way, similar FPU-emulating libraries were very popular in x86 world when coprocessors were rare
and expensive, and were installed only on expensive computers.}

\index{ARM!soft float}
\index{ARM!armel}
\index{ARM!armhf}
\index{ARM!hard float}
\IFRU{Эмуляция FPU-сопроцессора в ARM называется \IT{soft float} или \IT{armel}, 
а использование FPU-инструкций сопроцессора ~--- \IT{hard float} или \IT{armhf}.}
{FPU-coprocessor emulating calling \IT{soft float} or \IT{armel} in ARM world, 
while coprocessor's FPU-instructions usage calling \IT{hard float} or \IT{armhf}.}

\index{Raspberry Pi}
\IFRU{Ядро Linux, например, для Raspberry Pi может поставляться в двух вариантах.}
{For example, Linux kernel for Raspberry Pi is compiled in two variants.}
\IFRU{В случае \IT{soft float}, аргументы будут передаваться через R-регистры, 
а в случае \IT{hard float}, через D-регистры.}
{In \IT{soft float} case, arguments will be passed via R-registers, and in \IT{hard float} case ~--- via
D-registers.}

\IFRU{И это то, что помешает использовать, например, armhf-библиотеки
из armel-кода или наоборот, поэтому, весь код в дистрибутиве Linux должен быть скомпилирован
в соответствии с выбранным соглашением о вызовах.}
{And that's what don't let you use, for example, armhf-libraries from armel-code or vice versa, so that's
why all code in Linux distribution should be compiled according to chosen calling convention.}



\subsection{\IFRU{Передача чисел с плавающей запятой в аргументах}{Passing floating point number via arguments}}

\lstinputlisting{12_FPU/pow.c}

\subsubsection{x86}

\IFRU{Посмотрим что у нас вышло}{Let's see what we got in} (MSVC 2010):

\lstinputlisting[caption=MSVC 2010]{\IFRU{12_FPU/12_3_ru.asm}{12_FPU/12_3_en.asm}}

\index{x86!\Instructions!FLD}
\index{x86!\Instructions!FSTP}
\IFRU{\FLD и \FSTP перемешают переменные из/в сегмента данных в FPU-стек. 
\TT{pow()}\footnote{стандартная функция Си, возводящая число в степень} достает оба значения из FPU-стека и 
возвращает результат в \STZERO. 
\printf берет 8 байт из стека и трактует их как переменную типа \Tdouble.}
{\FLD and \FSTP are moving variables from/to data segment to FPU stack. 
\TT{pow()}\footnote{standard C function, raises a number to the given power}
taking both values from FPU-stack and 
returns result in the \STZERO register.
\printf takes 8 bytes from local stack and interpret them as \Tdouble type variable.}


\subsubsection{ARM + \NonOptimizingXcode + \ThumbTwoMode}

\lstinputlisting{12_FPU/passing_floats_Xcode_thumb_O0.asm}

\IFRU{Как я уже писал, 64-битные числа с плавающей точкой передаются в парах R-регистров.}
{As I wrote before, 64-bit floating pointer numbers passing in R-registers pairs.}
\IFRU{Этот код слегка избыточен (наверное потому что не включена оптимизация), ведь, можно было бы 
загружать значения напрямую в R-регистры минуя загрузку в D-регистры.}
{This is code is redundant for a little (certainly because optimization is turned off), because,
it's actually possible to load values into R-registers straightforwardly without touching D-registers.}

\IFRU{Итак, видно что функция}{So, as we see,} \TT{\_pow} \IFRU{получает первый аргумент в}
{function receiving first argument in} \Rzero \IFRU{и}{and} \Rone, \IFRU{а второй в}{and the second one in} 
\Rtwo \IFRU{и}{and} \Rthree. 
\IFRU{Функция оставляет результат в}{Function leaves result in} \Rzero \IFRU{и}{and} \Rone.
\IFRU{Результат работы}{Result of} \TT{\_pow} \IFRU{перекладывается в}{is moved into} \TT{D16}, 
\IFRU{затем в пару}{then in} \Rone \IFRU{и}{and} \Rtwo\IFRU{}{ pair}, \IFRU{откуда}{from where} 
\printf \IFRU{будет читать это число}{will take this number}.

\subsubsection{ARM + \NonOptimizingKeil + \ARMMode}

\lstinputlisting{12_FPU/passing_floats_Keil_ARM_O0.asm}

\IFRU{Здесь не используются D-регистры, используются только пары R-регистров.}
{D-registers are not used here, only R-register pairs are used.}



\subsection{\IFRU{Пример с сравнением}{Comparison example}}

\IFRU{Попробуем теперь вот это:}{Let's try this:}

\lstinputlisting{12_FPU/d_max.c}

\subsubsection{x86}

\IFRU{Несмотря на кажущуюся простоту этой функции, понять как она работает будет чуть сложнее.}
{Despite simplicity of the function, it will be harder to understand how it works.}

\IFRU{Вот что выдал MSVC 2010}{MSVC 2010 generated}:

\lstinputlisting[caption=MSVC 2010]{\IFRU{12_FPU/12_4_ru.asm}{12_FPU/12_4_en.asm}}

\index{x86!\Instructions!FLD}
\IFRU{Итак, \FLD загружает \TT{\_b} в регистр \STZERO.}
{So, \FLD loading \TT{\_b} into the \STZERO register.}

\label{Czero_etc}
\newcommand{\Czero}{\TT{C0}\xspace}
\newcommand{\Ctwo}{\TT{C2}\xspace}
\newcommand{\Cthree}{\TT{C3}\xspace}
\newcommand{\CThreeBits}{\Cthree/\Ctwo/\Czero}

\index{x86!\Instructions!FCOMP}
\IFRU{\FCOMP сравнивает содержимое \STZERO с тем что лежит в \TT{\_a} и выставляет биты \CThreeBits в 
регистре статуса FPU. Это 16-битный регистр отражающий текущее состояние FPU.} 
{\FCOMP compares the value in the \STZERO register with what is in \TT{\_a} value 
and set \CThreeBits bits in FPU 
status word register. 
This is 16-bit register reflecting current state of FPU.} 

\IFRU{Итак, биты \CThreeBits выставлены, но, к сожалению, у процессоров до Intel P6 
\footnote{Intel P6 это Pentium Pro, Pentium II, и далее} нет инструкций условного перехода,
проверяющих эти биты. 
Возможно, так сложилось исторически (вспомните о том что FPU когда-то был вообще отдельным чипом). 
А у Intel P6 появились инструкции \FCOMI/\FCOMIP/\FUCOMI/\FUCOMIP ~--- делающие тоже самое, 
только напрямую модифицирующие флаги \ZF/\PF/\CF.}
{For now \CThreeBits bits are set, but unfortunately, CPU before Intel P6
\footnote{Intel P6 is Pentium Pro, Pentium II, etc} has not any conditional 
jumps instructions which are checking these bits. 
Probably, it is a matter of history (remember: FPU was separate chip in past). 
Modern CPU starting at Intel P6 has \FCOMI/\FCOMIP/\FUCOMI/\FUCOMIP 
instructions~---which does the same, but modifies CPU flags \ZF/\PF/\CF.}

\IFRU{После этого, инструкция \FCOMP выдергивает одно значение из стека. 
Это отличает её от \FCOM, которая просто сравнивает значения, оставляя стек в таком же состоянии.}
{After bits are set, the \FCOMP instruction popping one variable from stack. 
This is what distinguish it from \FCOM, which is just comparing values, leaving the stack at the same state.}

\index{x86!\Instructions!FNSTSW}
\IFRU{\FNSTSW копирует содержимое регистра статуса в \AX. Биты \CThreeBits занимают позиции, 
соответственно, 14, 10, 8, в этих позициях они и остаются в регистре \AX, 
и все они расположены в старшей части регистра ~--- \AH.}
{\FNSTSW copies FPU status word register to the \AX. Bits \CThreeBits are placed at positions 14/10/8, 
they will be at the same positions in the \AX register and all they are placed in high part of the \AX{}~---\AH{}.}

\begin{itemize}
\item
\IFRU{Если b>a в нашем случае, то биты \CThreeBits должны быть выставлены так:}
{If b>a in our example, then \CThreeBits bits will be set as following:} 0, 0, 0.
\item
\IFRU{Если a>b, то биты будут выставлены:}{If a>b, then bits will be set:} 0, 0, 1.
\item
\IFRU{Если a=b, то биты будут выставлены так:}{If a=b, then bits will be set:} 1, 0, 0.
\end{itemize}
% TODO: table here?

\IFRU{После исполнения \TT{test ah, 5}, бит \Cthree и \TT{C1} сбросится в ноль, 
на позициях 0 и 2 (внутри регистра \AH) 
останутся соответственно \Czero и \Ctwo.}
{After \TT{test ah, 5} execution, bits \Cthree and \TT{C1} will be set to 0, 
but at positions 0 and 2 (in the \AH registers) 
\Czero and \Ctwo bits will be leaved.}

\label{parity_flag}
\index{x86!\Registers!\IFRU{Флаг четности}{Parity flag}}
\IFRU{Теперь немного о \IT{parity flag}\footnote{флаг четности}. Еще один замечательный рудимент}
{Now let's talk about parity flag. Another notable epoch rudiment}:

\begin{framed}
\begin{quotation}
One common reason to test the parity flag actually has nothing to do with parity. The FPU has four condition flags 
(C0 to C3), but they can not be tested directly, and must instead be first copied to the flags register. 
When this happens, C0 is placed in the carry flag, C2 in the parity flag and C3 in the zero flag. 
The C2 flag is set when e.g. incomparable floating point values (NaN or unsupported format) are compared 
with the FUCOM instructions.\footnote{\href{http://en.wikipedia.org/wiki/Parity_flag}{wikipedia: parity flag}}
\end{quotation}
\end{framed}

\IFRU{Этот флаг выставляется в $1$ если количество единиц в последнем результате ~--- четно. 
И в $0$ если ~--- нечетно.}
{This flag is to be set to $1$ if ones number is even. And to $0$ if odd.}

\index{x86!\Instructions!JP}
\IFRU{Таким образом, что мы имеем, флаг \PF будет выставлен в $1$, если \Czero и \Ctwo 
оба $1$ или оба $0$. 
И тогда сработает последующий \JP (\IT{jump if PF==1}). 
Если мы вернемся чуть назад и посмотрим значения \CThreeBits 
для разных вариантов, то увидим, что условный переход \JP сработает в двух случаях: если b>a или если a==b 
(ведь бит \Cthree уже \IT{вылетел} после исполнения \TT{test ah, 5}).}
{Thus, \PF flag will be set to 1 if both \Czero and \Ctwo are set to $0$ or both are $1$.
And then following \JP (\IT{jump if PF==1}) will be triggered. 
If we recall values of the \CThreeBits for various cases,
we will see the conditional jump 
\JP will be triggered in two cases: if b>a or a==b 
(\Cthree bit is already not considering here since it was cleared while execution of 
the \TT{test ah, 5} instruction).}

\IFRU{Дальше все просто. Если условный переход сработал, то \FLD загрузит значение \TT{\_b} в \STZERO, 
а если не сработал, то загрузится \TT{\_a} и произойдет выход из функции.}
{It is all simple thereafter. If conditional jump was triggered, \FLD will load the \TT{\_b} value 
to the \STZERO register, and if it is not triggered, the value of the \TT{\_a} variable will be loaded.}

\IFRU{Но это еще не все!}{But it is not over yet!}

\subsubsection{\IFRU{А теперь скомпилируем все это в MSVC 2010 с опцией \Ox}
{Now let's compile it with MSVC 2010 with optimization option \Ox}}

\lstinputlisting[caption=\Optimizing MSVC 2010]{\IFRU{12_FPU/12_5_ru.asm}{12_FPU/12_5_en.asm}}

\index{x86!\Instructions!FCOM}
\IFRU{\FCOM отличается от \FCOMP тем что просто сравнивает значения и оставляет стек в том же состоянии. 
В отличие от предыдущего примера, операнды здесь в другом порядке. 
Поэтому и результат сравнения в \CThreeBits будет другим чем раньше:}
{\FCOM is distinguished from \FCOMP in that sense that it just comparing values and leaves FPU stack 
in the same state. 
Unlike previous example, operands here in reversed order. 
And that is why result of comparison in the \CThreeBits will be different:}

\begin{itemize}
\item
\IFRU{Если a>b в нашем случае, то биты \CThreeBits должны быть выставлены так:}
{If a>b in our example, then \CThreeBits bits will be set as:} 0, 0, 0.
\item
\IFRU{Если b>a, то биты будут выставлены:}{If b>a, then bits will be set as:} 0, 0, 1.
\item
\IFRU{Если a=b, то биты будут выставлены так:}{If a=b, then bits will be set as:} 1, 0, 0.
\end{itemize}
% TODO: table?

\IFRU{Инструкция \TT{test ah, 65} как бы оставляет только два бита ~--- \Cthree и \Czero. 
Они оба будут нулями, если a>b: в таком случае переход \JNE не сработает. 
Далее имеется инструкция \TT{FSTP ST(1)} ~--- эта инструкция копирует 
значение \STZERO в указанный операнд и выдергивает одно значение из стека. В данном случае, 
она копирует \STZERO 
(где сейчас лежит \TT{\_a}) в \STONE. 
После этого на вершине стека два раза лежат \TT{\_a}. Затем одно значение выдергивается. 
После этого в \STZERO остается \TT{\_a} и функция завершается.}
{It can be said, \TT{test ah, 65} instruction just leaves two bits~---\Cthree and \Czero. 
Both will be zeroes if \TT{a>b}: in that case \JNE jump will not be triggered. 
Then \TT{FSTP ST(1)} is following~---this instruction copies value in the \STZERO into operand and 
popping one value from FPU stack.
In other words, the instruction copies \STZERO (where \TT{\_a} value is now) into the \STONE.
After that, two values of the {\_a} are at the top of stack now. 
After that, one value is popping.
After that, \STZERO will contain {\_a} and function is finished.}

\IFRU{Условный переход \JNE сработает в двух других случаях: если b>a или a==b. 
\STZERO скопируется в \STZERO, что как бы холостая операция, 
затем одно значение из стека вылетит и на вершине стека останется то что 
до этого лежало в \STONE (то есть, \TT{\_b}). И функция завершится. 
Эта инструкция используется здесь видимо потому что в FPU нет инструкции которая просто выдергивает 
значение из стека и больше ничего.}
{Conditional jump \JNE is triggered in two cases: of b>a or a==b. 
\STZERO into \STZERO will be copied, it is just like idle (\ac{NOP}) operation, then one value 
is popping from stack and top of stack (\STZERO) will contain what was in the \STONE before 
(that is {\_b}). Then function finishes. 
The instruction used here probably since \ac{FPU} has no instruction to pop value from stack and 
not to store it anywhere.}

\IFRU{Но и это еще не все.}{Well, but it is still not over.}

\subsubsection{GCC 4.4.1}

\lstinputlisting[caption=GCC 4.4.1]{\IFRU{12_FPU/12_6_ru.asm}{12_FPU/12_6_en.asm}}

\index{x86!\Instructions!FUCOMPP}
\IFRU{\FUCOMPP ~--- это почти то же что и \FCOM, только выкидывает из стека оба значения после сравнения, 
а также несколько иначе реагирует на ``не-числа''.}
{\FUCOMPP{}~---is almost like \FCOM, but popping both values from stack and handling 
``not-a-numbers'' differently.}

\index{\IFRU{Не-числа}{Non-a-numbers} (NaNs)}
\IFRU{Немного о \IT{не-числах}}{More about \IT{not-a-numbers}}:

\newcommand{\NANFN}{\IFRU{\footnote{\url{http://ru.wikipedia.org/wiki/NaN}}}
{\footnote{\url{http://en.wikipedia.org/wiki/NaN}}}}

\IFRU
{FPU умеет работать со специальными переменными, которые числами не являются и называются ``не числа'' или 
\gls{NaN}\NANFN{}. 
Это бесконечность, результат деления на ноль, и так далее. Нечисла бывают ``тихие'' и ``сигнализирующие''. 
С первыми можно продолжать работать и далее, а вот если вы попытаетесь совершить какую-то операцию 
с сигнализирующим нечислом, то сработает исключение.}
{FPU is able to deal with a special values which are \IT{not-a-numbers} or 
\gls{NaN}s\NANFN{}. 
These are infinity, result of dividing by $0$, etc. 
Not-a-numbers can be ``quiet'' and ``signalling''. It is possible to continue to work with ``quiet'' NaNs, 
but if one try to do any operation with ``signalling'' NaNs~---an exception will be raised.}

\index{x86!\Instructions!FCOM}
\index{x86!\Instructions!FUCOM}
\IFRU{Так вот, \FCOM вызовет исключение если любой из операндов ~--- какое-либо нечисло.
\FUCOM же вызовет исключение только если один из операндов именно ``сигнализирующее нечисло''.}
{\FCOM will raise exception if any operand~---\gls{NaN}. 
\FUCOM will raise exception only if any operand~---signalling \gls{NaN} (SNaN).}

\index{x86!\Instructions!SAHF}
\label{SAHF}
\IFRU{Далее мы видим \SAHF ~--- это довольно редкая инструкция в коде не использущим FPU. 
8 бит из \AH перекладываются в младшие 8 бит регистра статуса процессора в таком порядке: 
\TT{SF:ZF:-:AF:-:PF:-:CF <- AH}.}
{The following instruction is \SAHF~---this is rare instruction in the code which is not use FPU. 
8 bits from AH is movinto into lower 8 bits of CPU flags in the following order: 
\TT{SF:ZF:-:AF:-:PF:-:CF <- AH}.}

\index{x86!\Instructions!FNSTSW}
\IFRU{Вспомним, что \FNSTSW перегружает интересующие нас биты \CThreeBits в \AH, 
и соответственно они будут в позициях 6, 2, 0 в регистре \AH.}
{Let's remember the \FNSTSW is moving interesting for us bits \CThreeBits into the \AH 
and they will be in positions 6, 2, 0 in the \AH register.}

\IFRU{Иными словами, пара инструкций \TT{fnstsw  ax / sahf} перекладывает биты \CThreeBits в флаги \ZF, \PF, \CF.}
{In other words, \TT{fnstsw  ax / sahf} instruction pair is moving \CThreeBits into \ZF, \PF, \CF CPU flags.}

\IFRU{Теперь снова вспомним, какие значения бит \CThreeBits будут при каких результатах сравнения:}
{Now let's also recall, what values of the \CThreeBits bits will be set:}

\begin{itemize}
\item
\IFRU{Если a больше b в нашем случае, то биты \CThreeBits должны быть выставлены так:}
{If a is greater than b in our example, then \CThreeBits bits will be set as:} 0, 0, 0.
\item
\IFRU{Если a меньше b, то биты будут выставлены:}{if a is less than b, then bits will be set as:} 0, 0, 1.
\item
\IFRU{Если a=b, то биты будут выставлены так:}{If a=b, then bits will be set:} 1, 0, 0.
\end{itemize}
% TODO: table?

\IFRU{Иными словами, после инструкций \FUCOMPP/\FNSTSW/\SAHF, мы получим такое состояние флагов:}
{In other words, after \FUCOMPP/\FNSTSW/\SAHF instructions, we will have these CPU flags states:}

\begin{itemize}
\item
\IFRU{Если a>b в нашем случае, то флаги будут выставлены так:}
{If a>b, CPU flags will be set as:} \TT{ZF=0, PF=0, CF=0}.
\item
\IFRU{Если a<b, то флаги будут выставлены:}{If a<b, then CPU flags will be set as:} \TT{ZF=0, PF=0, CF=1}.
\item
\IFRU{Если a=b, то флаги будут выставлены так:}{If a=b, then CPU flags will be set as:} \TT{ZF=1, PF=0, CF=0}.
\end{itemize}
% TODO: table?

\index{x86!\Instructions!SETNBE}
\index{x86!\Instructions!SETcc}
\index{x86!\Instructions!JNBE}
\IFRU{Инструкция \SETNBE выставит в \AL единицу или ноль, в зависимости от флагов и условий. 
Это почти аналог \JNBE, за тем лишь исключением, что \SETcc
\footnote{\IT{cc} это \IT{condition code}}
выставляет 1 или 0 в \AL, а \Jcc делает переход или нет. 
\SETNBE запишет 1 если только \TT{CF=0} и \TT{ZF=0}. Если это не так, то запишет $0$ в \AL.}
{How \SETNBE instruction will store 1 or 0 to AL: it is depends of CPU flags. 
It is almost \JNBE instruction counterpart, with the exception the \SETcc 
\footnote{\IT{cc} is \IT{condition code}} is storing 1 or 0 to the \AL, but \Jcc do actual jump or not. 
\SETNBE store 1 only if \TT{CF=0} and \TT{ZF=0}. If it is not true, $0$ will be stored into \AL.}

\IFRU{\CF будет 0 и \ZF будет 0 одновременно только в одном случае: если a>b.}
{Both \CF is 0 and \ZF is 0 simultaneously only in one case: if a>b.}

\IFRU{Тогда в \AL будет записана единица, последующий условный переход \JZ взят не будет, 
и функция вернет \TT{\_a}. 
В остальных случаях, функция вернет \TT{\_b}.}
{Then one will be stored to the \AL and the following \JZ will not be triggered and function will 
return {\_a}. In all other cases, {\_b} will be returned.}

\IFRU{Но и это еще не конец.}{But it is still not over.}

\subsubsection{GCC 4.4.1 \IFRU{с оптимизацией \Othree}{with \Othree optimization turned on}}

\lstinputlisting[caption=\Optimizing GCC 4.4.1]{\IFRU{12_FPU/12_7_ru.asm}{12_FPU/12_7_en.asm}}

\IFRU{Почти все что здесь есть уже описано мною, кроме одного: использование \JA после \SAHF. 
Действительно, инструкции условных переходов ``больше'', ``меньше'', ``равно'' для сравнения беззнаковых чисел 
(\JA, \JAE, \JB, \JBE, \JE/\JZ, \JNA, \JNAE, \JNB, \JNBE, \JNE/\JNZ) проверяют только флаги \CF и \ZF. 
И биты \CThreeBits после сравнения перекладываются в эти флаги аккурат так, 
чтобы перечисленные инструкции переходов могли работать. \JA сработает если \CF и \ZF обнулены.}
{It is almost the same except one: \JA usage instead of \SAHF. 
Actually, conditional jump instructions checking ``larger'', ``lesser'' or ``equal'' for unsigned number comparison 
(\JA, \JAE, \JB, \JBE, \JE/\JZ, \JNA, \JNAE, \JNB, \JNBE, \JNE/\JNZ) are checking only \CF and \ZF flags. 
And \CThreeBits bits after comparison are moving into these flags exactly in the same fashion 
so conditional jumps will work here. \JA will work if both \CF are \ZF zero.}

\IFRU{Таким образом, перечисленные инструкции условного перехода можно использовать после инструкций \FNSTSW/\SAHF.}
{Thereby, conditional jumps instructions listed here can be used after \FNSTSW/\SAHF instructions pair.}

\IFRU{Вполне возможно что биты статуса FPU \CThreeBits преднамерено были размещены таким образом, 
чтобы переноситься на базовые флаги процессора без перестановок.}
{It seems, FPU \CThreeBits status bits was placed there intentionally so to map them to base CPU flags 
without additional permutations.}


\subsubsection{ARM + \OptimizingXcode + \ARMMode}

\begin{lstlisting}
VMOV            D16, R2, R3 ; b
VMOV            D17, R0, R1 ; a
VCMPE.F64       D17, D16
VMRS            APSR_nzcv, FPSCR
VMOVGT.F64      D16, D17 ; copy b to D16
VMOV            R0, R1, D16
BX              LR
\end{lstlisting}

Очень простой случай. Входные величины кладутся в \TT{D17} и \TT{D16} и сравниваются инструкцией 
\TT{VCMPE}.
Как и в процессорах x86, сопроцессор в ARM имеет свой собственный регистр статуса и флагов (\TT{FPSCR}),
потому что есть необходимость хранить специфичные для его работы флаги.

И так же как и в x86, в ARM нет инструкций условного перехода, 
проверяющих биты в регистре статуса сопроцессора, так что имеется инструкция \TT{VMRS}, копирующая 4 бита
(N, Z, C, V) из статуса сопроцессора в биты \IT{общего} статуса (\TT{APSR}).

\TT{VMOVGT} это аналог \TT{MOVGT}, инструкция, сработающая если при сравнении один операнд 
был больше чем второй
(\IT{Greater Than}). Если она сработает, в \TT{D16} запишется значение $b$, лежащее в тот момент 
в \TT{D17}.
А если не сработает, то в \TT{D16} останется лежать значение $a$.

Предпоследняя инструкция подготовит то что было в \TT{D16} для возврата через пару регистров \Rzero 
и \Rone.

\subsubsection{ARM + \OptimizingXcode + \ThumbMode}

\begin{lstlisting}
VMOV            D16, R2, R3 ; b
VMOV            D17, R0, R1 ; a
VCMPE.F64       D17, D16
VMRS            APSR_nzcv, FPSCR
IT GT 
VMOVGT.F64      D16, D17
VMOV            R0, R1, D16
BX              LR
\end{lstlisting}

Почти то же самое что и в предыдущем примере, за парой отличий. Дело в том, многие инструкции в режиме ARM
можно дополнять условием, которое если справедливо, то инструкция выполнится.

Но в режиме thumb такого нет. В 16-битных инструкций просто нет места для лишних 4 битов, при помощи
которых можно было бы закодировать условие выполнения.

Поэтому в thumb-2 добавили возможность дополнять thumb-инструкции условиями.

Здесь, в листинге сгенерированном при помощи \IDA, мы видим инструкцию \TT{VMOVGT}, 
такую же как и в предыдущем
примере. Но в реальности, там закодирована обычная инструкция \TT{VMOV}, просто \IDA добавила 
суффикс \TT{-GT} к ней,
потому что перед этой инструкцией стоит \TT{``IT GT''}.

Инструкция \TT{IT} определяет так называемый \IT{if-then block}. После этой инструкции, можно указывать 
до четырех инструкций, к которым будет добавлен суффикс условия. В нашем примере, \TT{``IT GT''} означает,
что следующая за ней инструкция будет исполнена, если условие \IT{GT} (\IT{Greater Than}) справедливо.

Теперь более сложный пример, кстати, из Angry Birds (для iOS):

\begin{lstlisting}
ITE NE
VMOVNE          R2, R3, D16
VMOVEQ          R2, R3, D17
\end{lstlisting}

\TT{ITE} означает \IT{if-then-else} и кодирует суффиксы для двух следующих за ней инструкций. Первая из них
исполнится, если сработает условие закодированное в \TT{ITE} (\IT{NE, not equal}), а вторая ~--- 
если это условие не сработает. 
Обратное условие от \TT{NE} это \TT{EQ} (\IT{equal}).

Еще чуть сложнее, и снова этот фрагмент из Angry Birds:

\begin{lstlisting}
ITTTT EQ
MOVEQ           R0, R4
ADDEQ           SP, SP, #0x20
POPEQ.W         {R8,R10}
POPEQ           {R4-R7,PC}
\end{lstlisting}

4 символа ``T'' в инструкции означают что 4 следующие инструкции будут исполнены если условие соблюдается. 
Поэтому
\IDA добавила ко всем четырем инструкциям суффикс \TT{-EQ}. 

А если бы здесь было, например, \TT{ITEEE EQ} (\IT{if-then-else-else-else}), 
тогда суффиксы для следующих четырех инструкций были бы расставлены так:

\begin{lstlisting}
-EQ
-NE
-NE
-NE
\end{lstlisting}

Еще фрагмент из Angry Birds:

\begin{lstlisting}
CMP.W           R0, #0xFFFFFFFF
ITTE LE
SUBLE.W         R10, R0, #1
NEGLE           R0, R0
MOVGT           R10, R0
\end{lstlisting}

\TT{ITTE} (\IT{if-then-then-else}) означает что первая и вторая инструкции исполнятся, если 
условие \TT{LE} (\IT{Less or Equal}) справедливо,
а третья - если справедливо обратное условие (\TT{GT} ~--- \IT{Greater Than}).

Компиляторы способны генерировать далеко не все варианты, например, в вышеупомянутой игре Angry Birds
(версия \IT{classic} для iOS) попадаются только такие инструкции: \TT{IT}, \TT{ITE}, \TT{ITT}, \TT{ITTE}, 
\TT{ITTT}, \TT{ITTTT}.
Как я это узнал? В \IDA можно сгенерировать листинг, так я и сделал, только в опциях я установил так 
чтобы показывались 4 байта для каждого опкода. Затем, зная что старшая часть 16-битного опкода \TT{IT} 
это 0xBF, я сделал при помощи grep это:

\begin{lstlisting}
cat AngryBirdsClassic.lst | grep " BF" | grep "IT" > results.lst
\end{lstlisting}

Тем не менее, если писать на ассемблере для режима thumb-2 вручную, и дополнять инструкции суффиксами
условия, то ассемблер автоматически будет добавлять инструкцию \TT{IT} с соответствующими флагами, там
где надо.

\subsubsection{ARM + \NonOptimizingXcode + \ARMMode}

\begin{lstlisting}
b               = -0x20
a               = -0x18
val_to_return   = -0x10
saved_R7        = -4

                STR             R7, [SP,#saved_R7]!
                MOV             R7, SP
                SUB             SP, SP, #0x1C
                BIC             SP, SP, #7
                VMOV            D16, R2, R3
                VMOV            D17, R0, R1
                VSTR            D17, [SP,#0x20+a]
                VSTR            D16, [SP,#0x20+b]
                VLDR            D16, [SP,#0x20+a]
                VLDR            D17, [SP,#0x20+b]
                VCMPE.F64       D16, D17
                VMRS            APSR_nzcv, FPSCR
                BLE             loc_2E08
                VLDR            D16, [SP,#0x20+a]
                VSTR            D16, [SP,#0x20+val_to_return]
                B               loc_2E10

loc_2E08
                VLDR            D16, [SP,#0x20+b]
                VSTR            D16, [SP,#0x20+val_to_return]

loc_2E10
                VLDR            D16, [SP,#0x20+val_to_return]
                VMOV            R0, R1, D16
                MOV             SP, R7
                LDR             R7, [SP+0x20+b],#4
                BX              LR
\end{lstlisting}

Почти то же самое что мы уже видели, но много избыточного кода из-за хранения $a$ и $b$, а также 
выходного значения, в локальном стеке.

\subsubsection{ARM + \OptimizingKeil + \ThumbMode}

\begin{lstlisting}
                PUSH    {R3-R7,LR}
                MOVS    R4, R2
                MOVS    R5, R3
                MOVS    R6, R0
                MOVS    R7, R1
                BL      __aeabi_cdrcmple
                BCS     loc_1C0
                MOVS    R0, R6
                MOVS    R1, R7
                POP     {R3-R7,PC}

loc_1C0
                MOVS    R0, R4
                MOVS    R1, R5
                POP     {R3-R7,PC}
\end{lstlisting}

Keil не генерирует специальную инструкцию для сравнения чисел с плавающей запятой, потому что не 
расчитывает на то что она будет поддерживаться, а простым сравнением побитово, кстати, здесь не обойтись,
ведь равные числа с плавающей точкой могут быть по-разному закодированы в формате IEEE 754. 
Для сравнения вызывается библиотечная функция \TT{\_\_aeabi\_cdrcmple}. Обратите внимание, результат
сравнения эта функция оставляет в флагах, чтобы следующая за вызовом инструкция \TT{BCS} (\IT{Carry set - Greater than or equal}) могла сработать. Или не сработать.

 


