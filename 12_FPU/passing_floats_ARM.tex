\subsubsection{ARM + \NonOptimizingXcode + \ThumbTwoMode}

\lstinputlisting{12_FPU/passing_floats_Xcode_thumb_O0.asm}

\IFRU{Как я уже писал, 64-битные числа с плавающей точкой передаются в парах R-регистров.}
{As I wrote before, 64-bit floating pointer numbers passing in R-registers pairs.}
\IFRU{Этот код слегка избыточен (наверное потому что не включена оптимизация), ведь, можно было бы 
загружать значения напрямую в R-регистры минуя загрузку в D-регистры.}
{This is code is redundant for a little (certainly because optimization is turned off), because,
it is actually possible to load values into R-registers straightforwardly without touching D-registers.}

\IFRU{Итак, видно что функция}{So, as we see,} \TT{\_pow} \IFRU{получает первый аргумент в}
{function receiving first argument in} \Rzero \AndENRU \Rone, \IFRU{а второй в}{and the second one in} 
\Rtwo \AndENRU \Rthree. 
\IFRU{Функция оставляет результат в}{Function leaves result in} \Rzero \AndENRU \Rone.
\IFRU{Результат работы}{Result of} \TT{\_pow} \IFRU{перекладывается в}{is moved into} \TT{D16}, 
\IFRU{затем в пару}{then in} \Rone \AndENRU \Rtwo\IFRU{}{ pair}, \IFRU{откуда}{from where} 
\printf \IFRU{будет читать это число}{will take this number}.

\subsubsection{ARM + \NonOptimizingKeil + \ARMMode}

\lstinputlisting{12_FPU/passing_floats_Keil_ARM_O0.asm}

\IFRU{Здесь не используются D-регистры, используются только пары R-регистров.}
{D-registers are not used here, only R-register pairs are used.}

