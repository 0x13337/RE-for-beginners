\subsubsection{ARM + \NonOptimizingXcode + \ThumbMode}

\lstinputlisting{12_FPU/passing_floats_Xcode_thumb_O0.asm}

Как я уже писал, 64-битные числа с плавающей точкой передаются в парах R-регистров.
Здесь код слегка избыточен (наверное потому что не включена оптимизация), ведь, можно было бы 
загружать значения напрямую в R-регистры минуя загрузку в D-регистры.

Итак, видно что функция \TT{\_pow} получает первый аргумент в \TT{R0} и \TT{R1}, а второй в \TT{R2} и \TT{R3}. 
Оставляет результат в \TT{R0} и \TT{R1}.
Результат работы \TT{\_pow} перекладывается в \TT{D16}, затем в пару \TT{R1} и \TT{R2}, 
откуда \printf будет читать это число.

\subsubsection{ARM + \NonOptimizingKeil + \ARMMode}

\lstinputlisting{12_FPU/passing_floats_Keil_ARM_O0.asm}

Здесь не используются D-регистры, используются только пары R-регистров.

