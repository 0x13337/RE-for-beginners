\subsubsection{ARM: \OptimizingXcode + \ARMMode}

\IFRU{Пока в ARM не было стандартного набора инструкций для работы с плавающей точкой}
{Until ARM has floating standardized point support}, \IFRU{разные производители процессоров
могли добавлять свои расширения для работы с ними}{several processor manufacturers may add their own 
instructions extensions}.
\IFRU{Позже, был принят стандарт}{Then, } VFP (\IT{Vector Floating Point})\IFRU{}{ was standardized}.

\IFRU{Важное отличие от x86 в том, что там вы работаете с FPU-стеком, а здесь стека нет, 
здесь вы работаете просто с регистрами.}
{One important difference from x86, there you working with FPU-stack, but here, in ARM, there
are no any stack, you work just with registers.}

\lstinputlisting{12_FPU/simple_Xcode_ARM_O3.asm}

\index{ARM!D-\IFRU{регистры}{registers}}
\index{ARM!S-\IFRU{регистры}{registers}}
\IFRU{Итак, здесь мы видим использование новых регистров, с префиксом D.}
{So, we see here new registers used, with D prefix.}
\IFRU{Это 64-битные регистры, их 32, и их можно
использовать и для чисел с плавающей точкой двойной точности (double) и для 
SIMD (в ARM это называется NEON).}
{These are 64-bit registers, there are 32 of them, and these can be used both for floating-point numbers 
(double) but also for SIMD (it is called NEON here in ARM).}

\IFRU{Имеются также 32 32-битных S-регистра, они применяются для работы с числами 
с плавающей точкой одинарной точности (float).}
{There are also 32 32-bit S-registers, they are intended to be used for single precision 
floating pointer numbers (float).}

\IFRU{Запомнить легко: D-регистры предназначены для чисел double-точности, 
а S-регистры ~--- для чисел single-точности.}
{It is easy to remember: D-registers are intended for double precision numbers, while S-registers ~---
for single precision numbers.}

\IFRU{Обе константы}{Both} ($3.14$ \AndENRU $4.1$) \IFRU{хранятся в памяти в формате IEEE 754.}
{constants are stored in memory in IEEE 754 form.}

\index{ARM!\Instructions!VLDR}
\index{ARM!\Instructions!VMOV}
\IFRU{Инструкции }{}\TT{VLDR} \AndENRU \TT{VMOV}
\IFRU{, как можно догадаться, это аналоги обычных \TT{LDR} и \MOV, но они работают с D-регистрами.}
{instructions, as it can be easily deduced, are analogous to the \TT{LDR} and \MOV instructions,
but they works with D-registers.}
\IFRU{Важно отметить, что эти инструкции, как и D-регистры, предназначены не только для работы 
с числами с плавающей точкой, но пригодны также и для работы с SIMD (NEON), и позже это также будет видно.}
{It should be noted that these instructions, just like D-registers, are intended not only for
floating point numbers, but can be also used for SIMD (NEON) operations and this will also be revealed soon.}

\IFRU{Аргументы передаются в функцию обычным путем, через R-регистры, однако, 
каждое число имеющее двойную точность
занимает 64 бита, так что для передачи каждого нужны два R-регистра.}
{Arguments are passed to function by usual way, via R-registers, however,
each number having double precision has size 64-bits, so, for passing each, two R-registers are needed.}

\TT{``VMOV D17, R0, R1''} \IFRU{в самом начале составляет два 32-битных значения из \Rzero и \Rone 
в одно 64-битное и сохраняет в}
{at the very beginning, composing two 32-bit values from \Rzero and \Rone into one 64-bit value
and saves it to} \TT{D17}.

\TT{``VMOV R0, R1, D16''} \IFRU{в конце это обратная процедура}{is inverse operation}, 
\IFRU{то что было в}{what was in} \TT{D16} 
\IFRU{остается в двух регистрах}{leaving in two} \Rzero \AndENRU \Rone\IFRU{}{ registers},
\IFRU{потому что,}{since} \IFRU{число с двойной точностью}{double-precision number}, 
\IFRU{занимающее 64 бита}{needing 64 bits for storage}, \IFRU{возвращается в паре регистров \Rzero и \Rone}
{is returning in the \Rzero and \Rone registers pair}.

\index{ARM!\Instructions!VDIV}
\index{ARM!\Instructions!VMUL}
\index{ARM!\Instructions!VADD}
\TT{VDIV}, \TT{VMUL} \AndENRU \TT{VADD}, \IFRU{это, собственно, инструкции для работы с числами 
с плавающей точкой, вычисляющие, соответственно, частное\FNQUOTIENT, произведение\FNPRODUCT и сумму\FNSUM.}
{are instruction for floating point numbers processing, computing, quotient\FNQUOTIENT, 
product\FNPRODUCT and sum\FNSUM, respectively.}

\IFRU{Код для thumb-2 такой же.}{The code for thumb-2 is same.}

\subsubsection{ARM: \OptimizingKeil + \ThumbMode}

\lstinputlisting{12_FPU/simple_Keil_O3_thumb.asm}

\IFRU{Keil компилировал для процессора, в котором может и не быть поддержки FPU или NEON.}
{Keil generates for processors not supporting FPU or NEON.}
\IFRU{Так что числа с двойной точностью передаются в парах обычных R-регистров}
{So, double-precision floating numbers are passed via usual R-registers},
\IFRU{а вместо FPU-инструкций вызываются сервисные библиотечные функции}
{and instead of FPU-instructions, service library functions are called (like}
\TT{\_\_aeabi\_dmul}, \TT{\_\_aeabi\_ddiv}, \TT{\_\_aeabi\_dadd}
\IFRU{, эмулирующие умножение, деление и сложение чисел с плавающей точкой.}
{) which emulates multiplication, division and addition floating-point numbers.}
\IFRU{Конечно, это медленнее чем FPU-сопроцессор, но лучше чем ничего.}
{Of course, that's slower than FPU-coprocessor, but it is better than nothing.}

\IFRU{Кстати, похожие библиотеки для эмуляции сопроцессорных инструкций были очень распространены в x86, 
когда сопроцессор был редким и дорогим, и стоял далеко не на всех компьютерах.}
{By the way, similar FPU-emulating libraries were very popular in x86 world when coprocessors were rare
and expensive, and were installed only on expensive computers.}

\index{ARM!soft float}
\index{ARM!armel}
\index{ARM!armhf}
\index{ARM!hard float}
\IFRU{Эмуляция FPU-сопроцессора в ARM называется \IT{soft float} или \IT{armel}, 
а использование FPU-инструкций сопроцессора ~--- \IT{hard float} или \IT{armhf}.}
{FPU-coprocessor emulating called \IT{soft float} or \IT{armel} in ARM world, 
while using coprocessor's FPU-instructions called \IT{hard float} or \IT{armhf}.}

\index{Raspberry Pi}
\IFRU{Ядро Linux, например, для Raspberry Pi может поставляться в двух вариантах.}
{For example, Linux kernel for Raspberry Pi is compiled in two variants.}
\IFRU{В случае \IT{soft float}, аргументы будут передаваться через R-регистры, 
а в случае \IT{hard float}, через D-регистры.}
{In \IT{soft float} case, arguments will be passed via R-registers, and in \IT{hard float} case ~--- via
D-registers.}

\IFRU{И это то, что помешает использовать, например, armhf-библиотеки
из armel-кода или наоборот, поэтому, весь код в дистрибутиве Linux должен быть скомпилирован
в соответствии с выбранным соглашением о вызовах.}
{And that's what don't let you use, for example, armhf-libraries from armel-code or vice versa, so that's
why all code in Linux distribution should be compiled according to chosen calling convention.}

