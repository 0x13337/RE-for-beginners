\subsubsection{x86}

\IFRU{Компилируем в}{Compile it in} MSVC 2010:

\lstinputlisting{\IFRU{12_FPU/12_1_ru.asm}{12_FPU/12_1_en.asm}}

\IFRU{\FLD берет 8 байт из стека и загружает из в регистр \STZERO, автоматически конвертируя во внутренний 
80-битный формат (\IT{extended precision}).}
{\FLD takes 8 bytes from stack and load the number into \STZERO register, automatically converting 
it into internal 80-bit format \IT{extended precision}).}

\IFRU{\FDIV делит содержимое регистра \STZERO на число лежащее по адресу \TT{\_\_real@40091eb851eb851f} ~--- 
там закодировано значение 3.14. Синтаксис ассемблера не поддерживает подобные числа, 
так что то что мы там видим, это шестандцатиричное представление числа \IT{3.14} в формате IEEE 754.}
{\FDIV divide value in register \STZERO by number storing at address \TT{\_\_real@40091eb851eb851f} ~--- 
3.14 value is coded there. Assembler syntax missing floating point numbers, so, 
what we see here is hexadecimal representation of \IT{3.14} number in 64-bit IEEE 754 encoded.}

\IFRU{После выполнения \FDIV, в \STZERO остается частное\footnote{результат деления}.}
{After \FDIV execution, \STZERO will hold quotient\footnote{division result}.}

\IFRU{Кстати, есть еще инструкция \FDIVP, которая делит \STONE на \STZERO, 
выталкивает эти числа из стека и заталкивает результат. 
Если вы знаете язык Forth\FNURLFORTH, то это как раз оно и есть ~--- стековая машина\FNURLSTACK.}
{By the way, there are also \FDIVP instruction, which divide \STONE by \STZERO, 
popping both these values from stack and then pushing result. 
If you know Forth language\FNURLFORTH, you will quickly understand that this is stack machine\FNURLSTACK.}

\IFRU{Следующая \FLD заталкивает в стек значение \IT{b}.}
{Next \FLD instruction pushing \IT{b} value into stack.}

\IFRU{После этого, в \STONE перемещается результат деления, а в \STZERO теперь будет \IT{b}.}
{After that, quotient is placed to \STONE, and \STZERO will hold \IT{b} value.}

\IFRU{Следующий \FMUL умножает \IT{b} из \STZERO на значение \TT{\_\_real@4010666666666666} ~--- 
там лежит число 4.1, и оставляет результат в \STZERO.}
{Next \FMUL instruction do multiplication: \IT{b} from \STZERO register by value at 
\TT{\_\_real@4010666666666666} (4.1 number is there) and leaves result in \STZERO.}

\IFRU{Самая последняя инструкция \FADDP складывает два значения из вершины стека, 
в \STONE и затем выталкивает значение лежащее в \STZERO, 
таким образом результат сложения остается на вершине стека в \STZERO.}
{Very last \FADDP instruction adds two values at top of stack, placing result at \STONE 
register and then popping value at \STONE, hereby leaving result at top of stack in \STZERO.}

\IFRU{Функция должна вернуть результат в \STZERO, так что больше ничего здесь не производится, 
кроме эпилога функции.}
{The function must return result in \STZERO register, so, after \FADDP there are no any 
other code except of function epilogue.}

\IFRU{GCC 4.4.1 (с опцией \Othree) генерирует похожий код, хотя и с некоторой разницей:}
{GCC 4.4.1 (with \Othree option) emitting the same code, however, slightly different:}

\lstinputlisting{\IFRU{12_FPU/12_2_ru.asm}{12_FPU/12_2_en.asm}}

\IFRU{Разница в том, что в стек сначала заталкивается 3.14 (в \STZERO), а затем значение 
из \TT{arg\_0} делится на то что лежит в регистре \STZERO.}
{The difference is that, first of all, 3.14 is pushing to stack (into \STZERO), and then value 
in \TT{arg\_0} is dividing by what is in \STZERO register.}

\IFRU{\FDIVR означает \IT{Reverse Divide} ~--- делить поменяв делитель и делимое местами. 
Точно такой же инструкции для умножения нет, потому что она была бы бессмыслена (ведь умножение ~--- 
операция коммутативная), так что остается только \FMUL без соответсвующей ей \TT{-R} инструкции.}
{\FDIVR meaning \IT{Reverse Divide} ~--- to divide with divisor and dividend swapped with each other. 
There are no such instruction for multiplication, because multiplication is 
commutative operation, so we have just \FMUL without its \TT{-R} counterpart.}

\IFRU{\FADDP не только складывает два значения, но также и выталкивает из стека одно значение. 
После этого, в \STZERO остается только результат сложения.}
{\FADDP adding two values but also popping one value from stack. 
After that operation, \STZERO will contain sum.}

\IFRU{Этот фрагмент кода получен при помощи \IDA, которая регистр \STZERO называет для краткости просто \TT{ST}.}
{This fragment of disassembled code was produced using \IDA which named \STZERO register as \TT{ST} for short.}
