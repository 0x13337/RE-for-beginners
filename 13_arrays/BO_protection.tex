\subsection{\IFRU{Защита от переполнения буфера}{Buffer overflow protection methods}}

\newcommand{\URLWPB}{\href{http://en.wikipedia.org/wiki/Buffer_overflow_protection}
{Wikipedia: \IFRU{описания защит, которые компилятор может вставлять в код}
{compiler-side buffer overflow protection methods}}}

\IFRU{В наше время пытаются бороться с этой напастью, не взирая на халатность программистов на \CCpp. 
В MSVC есть опции вроде\footnote{\URLWPB}:}
{There are several methods to protect against it, regardless of \CCpp programmers' negligence.
MSVC has options like\footnote{\URLWPB}:}

\begin{verbatim}
 /RTCs Stack Frame runtime checking
 /GZ Enable stack checks (/RTCs)
\end{verbatim}

\IFRU{Один из методов, это в прологе функции вставлять в область локальных переменных 
некоторое случайное значение 
и в эпилоге функции, перед выходом, это число проверять. 
И если проверка не прошла, то не выполнять инструкцию \RET а остановиться (или зависнуть). 
Процесс зависнет, но это лучше чем удаленная атака на ваш хост.}
{One of the methods is to write random value among local variables to stack at function prologue 
and to check it in function epilogue before function exiting.
And if value is not the same, do not execute last instruction \RET, but halt (or hang).
Process will hang, but that's much better then remote attack to your host.}
    
\newcommand{\CANARYURL}
{
    \IFRU
    {
        \href{http://miningwiki.ru/wiki/\%D0\%9A\%D0\%B0\%D0\%BD\%D0\%B0\%D1\%80\%D0\%B5\%D0\%B9\%D0\%BA\%D0\%B0_\%D0\%B2_\%D1\%88\%D0\%B0\%D1\%85\%D1\%82\%D0\%B5}{Шахтерская энциклопедия: Канарейка в шахте}
    }
    {
        \href{http://en.wikipedia.org/wiki/Domestic\_Canary\#Miner.27s\_canary}{Wikipedia: Miner's canary}
    }
}

Это случайное значение иногда называют ``канарейкой''\footnote{``canary'' в англоязычной литературе}, 
по аналогии с шахтной канарейкой\footnote{\CANARYURL},
их использовали шахтеры в свое время, чтобы определять, есть ли в шахте опасный газ.
Канарейки очень к нему чувствительны и либо проявляли сильное беспокойство, либо гибли от газа.

Если скомпилировать наш простейший пример работы с массивом~\ref{arrays_simple} в MSVC с опцией RTC1 или RTCs, 
в конце функции будет вызов 
функции \TT{@\_RTC\_CheckStackVars@8}, проверяющей корректность ``канарейки''.

\subsubsection{\OptimizingXcode + \ThumbTwoMode}

Возвращаясь к нашему простому примеру~\ref{arrays_simple}, 
можно посмотреть как LLVM добавит проверку ``канарейки'':

\lstinputlisting{13_arrays/simple_Xcode_thumb_O3_en.asm}

Во-первых, как видно, LLVM "развернул" цикл и все значения записываются в массив по одному, уже вычисленные, 
потому что LLVM посчитал что так будет быстрее. Кстати, инструкции ARM позволяют сделать это еще быстрее и это
может быть вашим домашним заданием.

В конце функции мы видим сравнение ``канареек'' ~--- той что лежит в локальном стеке и корректной, на которую
ссылается регистр \TT{R8}. Если они равны, срабатывает блок из четырех инструкций \TT{``ITTTT EQ''}, это запись
$0$ в \Rzero, эпилог функции и выход из нее. Если блок ``канарейки'' не равны, блок не срабатывает и происходит
переход на функцию \TT{\_\_\_stack\_chk\_fail}, которая, вероятно, остановит работу программы.
% TODO illustrate this!

