\subsection{\IFRU{Защита от переполнения буфера}{Buffer overflow protection methods}}

\newcommand{\URLWPB}{\href{http://en.wikipedia.org/wiki/Buffer_overflow_protection}
{Wikipedia: \IFRU{описания защит, которые компилятор может вставлять в код}
{compiler-side buffer overflow protection methods}}}

\IFRU{В наше время пытаются бороться с этой напастью, не взирая на халатность программистов на \CCpp. 
В MSVC есть опции вроде\footnote{\URLWPB}:}
{There are several methods to protect against it, regardless of \CCpp programmers' negligence.
MSVC has options like\footnote{\URLWPB}:}

\begin{lstlisting}
 /RTCs Stack Frame runtime checking
 /GZ Enable stack checks (/RTCs)
\end{lstlisting}

\IFRU{Один из методов, это в прологе функции вставлять в область локальных переменных 
некоторое случайное значение 
и в эпилоге функции, перед выходом, это число проверять. 
И если проверка не прошла, то не выполнять инструкцию \RET а остановиться (или зависнуть). 
Процесс зависнет, но это лучше чем удаленная атака на ваш хост.}
{One of the methods is to write random value among local variables to stack at function prologue 
and to check it in function epilogue before function exiting.
And if value is not the same, do not execute last instruction \RET, but halt (or hang).
Process will hang, but that's much better then remote attack to your host.}
    
\newcommand{\CANARYURL}
{
    \IFRU
    {
        \href{http://miningwiki.ru/wiki/\%D0\%9A\%D0\%B0\%D0\%BD\%D0\%B0\%D1\%80\%D0\%B5\%D0\%B9\%D0\%BA\%D0\%B0_\%D0\%B2_\%D1\%88\%D0\%B0\%D1\%85\%D1\%82\%D0\%B5}{Шахтерская энциклопедия: Канарейка в шахте}
    }
    {
        \href{http://en.wikipedia.org/wiki/Domestic\_Canary\#Miner.27s\_canary}{Wikipedia: Miner's canary}
    }
}

\IFRU{Это случайное значение иногда называют ``канарейкой''\footnote{``canary'' в англоязычной литературе}, 
по аналогии с шахтной канарейкой\footnote{\CANARYURL},
их использовали шахтеры в свое время, чтобы определять, есть ли в шахте опасный газ.
}
{This random value is called ``canary'' sometimes, it's related to miner's canary\footnote{\CANARYURL},
they were used by miners in these days, in order to detect poisonous gases quickly.}
\IFRU{Канарейки очень к нему чувствительны и либо проявляли сильное беспокойство, либо гибли от газа.}
{Canaries are very sensetive to mine gases, they become very agitated in case of danger, or even dead.}

\IFRU{Если скомпилировать наш простейший пример работы с массивом}
{If to compile our very simple array example}~\ref{arrays_simple} \IFRU{в}{in} MSVC 
\IFRU{с опцией RTC1 или RTCs}{with RTC1 and RTCs option}, \IFRU{в конце функции будет вызов 
функции}{you will see call to} \TT{@\_RTC\_CheckStackVars@8}\IFRU{, проверяющей корректность ``канарейки''.}
{ function at the function end, checking ``canary'' correctness.}

\subsubsection{\OptimizingXcode + \ThumbTwoMode}

\IFRU{Возвращаясь к нашему простому примеру}{Let's back to our simple array example}~\ref{arrays_simple}, 
\IFRU{можно посмотреть как LLVM добавит проверку ``канарейки''}
{again, now we can see how LLVM will check ``canary'' correctness}:

\lstinputlisting{13_arrays/simple_Xcode_thumb_O3_en.asm}

\IFRU{Во-первых, как видно, LLVM ``развернул'' цикл и все значения записываются в массив по одному, 
уже вычисленные, 
потому что LLVM посчитал что так будет быстрее.}
{First of all, as we see, LLVM made loop ``unrolled'' and all values are written into array one-by-one,
already calculated, because LLVM conclued it will be faster.}
\IFRU{Кстати, инструкции режима ARM позволяют сделать это еще быстрее и это может быть вашим 
домашним заданием.}{By the way, ARM mode instructions may help to do this even faster, 
and finding this way could be your homework.}

\IFRU{В конце функции мы видим сравнение ``канареек'' ~--- той что лежит в локальном стеке и корректной, 
на которую ссылается регистр \TT{R8}.}
{At the function end wee see ``canaries'' comparison ~--- that laying in local stack and correct one,
to which \TT{R8} register pointing.}
\IFRU{Если они равны, срабатывает блок из четырех инструкций при помощи \TT{``ITTTT EQ''}, это запись
$0$ в \Rzero, эпилог функции и выход из нее.}
{If they are equal to each other, 4-instruction block is triggered by \TT{``ITTTT EQ''}, it is
writing $0$ into \Rzero, function epilogue and exit.}
\IFRU{Если ``канарейки'' не равны, блок не срабатывает и происходит
переход на функцию}{If ``canaries'' are not equial, block will not be triggered,
and jump to} \TT{\_\_\_stack\_chk\_fail}\IFRU{, которая, вероятно, остановит работу программы.}
{ function will be occured, which, as I suppose, will halt execution.}
% TODO illustrate this!

