\subsection{\IFRU{Многомерные массивы}{Multidimensional arrays}}

\IFRU{Внутри, многомерный массив выглядит так же как и линейный.}
{Internally, multidimensional array is essentially the same thing as linear array.}

\IFRU{Ведь память компьютера линейная, это одномерный массив.
Но для удобства, этот одномерный массив легко представить как многомерный.}
{Since computer memory in linear, it's one-dimensional array.
But this one-dimensional array can be easily represented as multidimensional for convenience.}

\IFRU{К примеру, элементы массива $a[3][4]$ будут так расположены в одномерном массиве из 12-и ячеек:}
{For example, that's how $a[3][4]$ array elements will be placed in one-dimensional array of 12 cells:}

\begin{center}
\begin{tabular}{ | l | l | l | l | }
\hline                        
0 & 1 & 2 & 3 \\
\hline  
4 & 5 & 6 & 7 \\
\hline  
8 & 9 & 10 & 11 \\
\hline  
\end{tabular}
\end{center}

\index{row-major order}
\IFRU{То есть, чтобы адресовать нужный элемент, в начале умножаем первый индекс на 4 (ширину матрицы), 
затем прибавляем второй индекс.}{So, in order to address elements we need, first multiply first index by
4 (matrix width) and then add second index.}
\IFRU{Это называется}{That's called} \IT{row-major order}, 
\IFRU{и такой способ представления массивов и матриц используется по крайней мере в}
{and this method of arrays and matrices representation is used in at least in} \CCpp, Python. 
\IFRU{Термин }\IT{row-major order} \IFRU{означает по-русски
примерно следующее: ``в начале записываем элементы первой строки, затем второй \dots и элементы последней 
строки в самом конце''.}
{term in plain English language means: ``first, write elements of first row, then second row \dots 
and finally elements of last row''.}

\index{column-major order}
\index{FORTRAN}
\IFRU{Другой способ представления называется}{Another method of representation called} 
\IT{column-major order} (\IFRU{индексы массива используются в обратном порядке}
{array indices used in reverse order}) \IFRU{и это используется по крайней мере в}{and it's used at least in} 
FORTRAN, MATLAB, R. 
\IFRU{Термин }\IT{column-major order} \IFRU{означает по-русски
следующее: ``в начале записываем элементы первого столбца, затем второго \dots и элементы последнего столбца
в самом конце''.}
{term in plain English language means: ``first, write elements of first column, then second columnt \dots
and finally elements of last column''.}

\IFRU{То же самое и для многомерных массивов.}{Same thing about multidimensional arrays.}

\IFRU{Попробуем}{Let's see}:

\lstinputlisting[caption=\IFRU{простой пример}{simple example}]{13_arrays/multi.c}

\subsubsection{x86}

\IFRU{В итоге}{We got} (MSVC 2010):

\lstinputlisting[caption=MSVC 2010]{13_arrays/multi_msvc.asm}

\IFRU{В принципе, ничего удивительного. В \TT{insert()} для вычисления адреса нужного элемента массива, 
три входных аргумента перемножаются по формуле $address=600 \cdot 4 \cdot x + 30 \cdot 4 \cdot y + 4z$, 
чтобы представить массив трехмерным.
Не забывайте также что тип \Tint 32-битный (4 байта), поэтому все коэффициенты нужно умножить на 4.}
{Nothing special. For index calculation, three input arguments are multiplying 
by formula $address=600 \cdot 4 \cdot x + 30 \cdot 4 \cdot y + 4z$ to represent array as multidimensional.
Do not forget that \Tint type is 32-bit (4 bytes), so all coefficients should be multiplied by 4.}

\lstinputlisting[caption=GCC 4.4.1]{13_arrays/multi_gcc.asm}

\IFRU{Компилятор GCC решил всё сделать немного иначе}{GCC compiler does it differently}.
\IFRU{Для вычисления одной из операций ($30y$), GCC создал код, где нет самой операции умножения.}
{For one of operations calculating ($30y$), GCC produced a code without multiplication instruction.}
\IFRU{Происходит это так}{This is how it done}: 
$(y+y) \ll 4 - (y+y) = (2y) \ll 4 - 2y = 2 \cdot 16 \cdot y - 2y = 32y - 2y = 30y$. 
\IFRU{Таким образом, для вычисления $30y$ используется только операция сложения, 
операция битового сдвига и операция вычитания.}{Thus, for $30y$ calculation, only one addition operation
used, one bitwise shift operation and one subtraction operation.}
\IFRU{Это работает быстрее}{That works faster}.

\subsubsection{ARM + \NonOptimizingXcode + \ThumbMode}

\lstinputlisting[caption=\NonOptimizingXcode + \ThumbMode]{13_arrays/multi_Xcode_thumb_O0_en.asm}

\NonOptimizing LLVM \IFRU{сохраняет все переменные в локальном стеке, хотя это и избыточно.}
{saves all variables in local stack, however, it's redundant.}
\IFRU{Адрес элемента массива вычисляется по уже рассмотренной формуле.}
{Address of array element is calculated by formula we already figured out.}

\subsubsection{ARM + \OptimizingXcode + \ThumbMode}

\lstinputlisting[caption=\OptimizingXcode + \ThumbMode]{13_arrays/multi_Xcode_thumb_O3_en.asm}

\IFRU{Тут используются уже описанные трюки для замены умножения на операции сдвига, сложения и вычитания.}
{Here is used tricks for replacing multiplication by shift, addition and subtraction we already considered.}

\index{ARM!\Instructions!RSB}
\index{ARM!\Instructions!SUB}
\IFRU{Также мы видим новую для себя инструкцию}{Here we also see new instruction for us:} 
\TT{RSB} (\IT{Reverse Subtract}).
\IFRU{Она работает так же как и \SUB, только меняет операнды местами.}
{It works just as \SUB, but swapping operands with each other.}
\IFRU{Зачем?}{Why?}
\index{ARM!Optional operators!LSL}
\SUB, \TT{RSB}, \IFRU{это те инструкции, ко второму операнду которых можно применить коэффициент сдвига, 
как мы видим и здесь}
{are those instructions, to the second operand of which shift coefficient may be applied}: (\TT{LSL\#4}). 
\IFRU{Но этот коэффициент можно применить только ко второму операнду.}
{But this coefficient may be applied only to second operand.}
\IFRU{Для коммутативных операций, таких как сложение или умножение, 
там операнды можно менять местами и это не влияет на результат.}
{That's fine for commutative operations like addition or multiplication, operands may be swapped there
without result affecting.}
\IFRU{Но вычитание ~--- операция некоммутативная, так что, для этих случаев существует инструкция \TT{RSB}.}
{But subtraction is non-commutative operation, so, for these cases, \TT{RSB} exist.}

\index{ARM!\Instructions!LDR.W}
\IFRU{Инструкция }\TT{``LDR.W R9, [R9]''} \IFRU{работает как}{works like} \LEA~\ref{sec:LEA} 
\IFRU{в x86, и здесь она ничего не делает, она избыточна.}{in x86, but here it does nothing, it's redundant.}
\IFRU{Вероятно, компилятор несоптимизировал её.}{Apparently, compiler not optimized it.}

