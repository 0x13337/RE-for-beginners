\subsubsection{x86}

\IFRU{Например в Win32 API:}{Win32 API example:}

\begin{lstlisting}
	HANDLE fh;

	fh=CreateFile ("file", GENERIC_WRITE | GENERIC_READ, FILE_SHARE_READ, NULL, OPEN_ALWAYS, FILE_ATTRIBUTE_NORMAL, NULL);
\end{lstlisting}

\IFRU{Получаем}{We got} (MSVC 2010):

\begin{lstlisting}[caption=MSVC 2010]
	push	0
	push	128					; 00000080H
	push	4
	push	0
	push	1
	push	-1073741824				; c0000000H
	push	OFFSET $SG78813
	call	DWORD PTR __imp__CreateFileA@28
	mov	DWORD PTR _fh$[ebp], eax
\end{lstlisting}

\IFRU{Заглянем в файл}{Let's take a look into} WinNT.h:

\begin{lstlisting}[caption=WinNT.h]
#define GENERIC_READ                     (0x80000000L)
#define GENERIC_WRITE                    (0x40000000L)
#define GENERIC_EXECUTE                  (0x20000000L)
#define GENERIC_ALL                      (0x10000000L)
\end{lstlisting}

\IFRU{Все ясно}{Everything is clear}, 
\TT{GENERIC\_READ | GENERIC\_WRITE = 0x80000000 | 0x40000000 = 0xC0000000}, 
\IFRU{и это значение используется как второй аргумент для}
{and that's value is used as second argument for} \TT{CreateFile()}\footnote{\href{http://msdn.microsoft.com/en-us/library/aa363858(VS.85).aspx}{MSDN: CreateFile function}} function.

\IFRU{Как \TT{CreateFile()} будет проверять флаги?}{How \TT{CreateFile()} will check flags?}

\IFRU{Заглянем в KERNEL32.DLL от Windows XP SP3 x86 и найдем в функции \TT{CreateFileW()} в том числе и 
такой фрагмент кода:}
{Let's take a look into KERNEL32.DLL in Windows XP SP3 x86 and we'll find
this fragment of code in the function \TT{CreateFileW}:}

\begin{lstlisting}[caption=KERNEL32.DLL (Windows XP SP3 x86)]
.text:7C83D429                 test    byte ptr [ebp+dwDesiredAccess+3], 40h
.text:7C83D42D                 mov     [ebp+var_8], 1
.text:7C83D434                 jz      short loc_7C83D417
.text:7C83D436                 jmp     loc_7C810817
\end{lstlisting}

\IFRU{Здесь мы видим инструкцию \TEST, впрочем, она берет не весь второй аргумент функции, 
но только его самый старший байт (\TT{ebp+dwDesiredAccess+3}) и проверяет его на флаг 0x40 
(имеется ввиду флаг \TT{GENERIC\_WRITE}).}
{Here we see \TEST instruction, it takes, however, not the whole second argument,
but only most significant byte (\TT{ebp+dwDesiredAccess+3}) and checks it for 0x40 flag
(meaning \TT{GENERIC\_WRITE} flag here)}

\IFRU{\TEST это то же что и \AND, только без сохранения результата 
(вспомните что \CMP это то же что и \SUB, только без сохранения результатов}
{\TEST is merely the same instruction as \AND, but without result saving 
(recall the fact \CMP instruction is merely the same as \SUB, but without result saving}~\ref{CMPandSUB}).

\IFRU{Логика данного фрагмента кода примерно такая:}{This fragment of code logic is as follows:}

\begin{lstlisting}
if ((dwDesiredAccess&0x40000000) == 0) goto loc_7C83D417
\end{lstlisting}

\IFRU{Если после операции \AND останется этот бит, то флаг \ZF не будет поднят и условный переход 
\JZ не сработает. 
Переход возможен только если в переменной \TT{dwDesiredAccess} отсутствует бит $0x40000000$ ~--- 
тогда результат \AND будет $0$, флаг \ZF будет поднят и переход сработает.}
{If \AND instruction leaving this bit, \ZF flag will be cleared and \JZ conditional jump will not 
be triggered.
Conditional jump will be triggered only if $0x40000000$ bit is absent in \TT{dwDesiredAccess} variable ~--- 
then \AND result will be $0$, \ZF flag will be set and conditional jump is to be triggered.}

\IFRU{Попробуем GCC 4.4.1 и Linux:}{Let's try GCC 4.4.1 and Linux:}

\begin{lstlisting}
#include <stdio.h>
#include <fcntl.h>

void main()
{
	int handle;

	handle=open ("file", O_RDWR | O_CREAT);
};
\end{lstlisting}

\IFRU{Получим}{We got}:

\lstinputlisting{14_bitfields/check.asm}[caption=GCC 4.4.1]

\IFRU{Заглянем в реализацию функции \TT{open()} в библиотеке \TT{libc.so.6}, но обнаружим что там 
только вызов сисколла:}
{Let's take a look into \TT{open()} function in \TT{libc.so.6} library, but there is only syscall calling:}

\begin{lstlisting}[caption=open() (libc.so.6)]
.text:000BE69B                 mov     edx, [esp+4+mode] ; mode
.text:000BE69F                 mov     ecx, [esp+4+flags] ; flags
.text:000BE6A3                 mov     ebx, [esp+4+filename] ; filename
.text:000BE6A7                 mov     eax, 5
.text:000BE6AC                 int     80h             ; LINUX - sys_open
\end{lstlisting}

\IFRU{Значит, битовые поля флагов \TT{open()} вероятно проверяются где-то в ядре Linux.}
{So, \TT{open()} bit fields apparently checked somewhere in Linux kernel.}

\IFRU{Разумеется, и стандартные библиотеки Linux и ядро Linux можно получить в виде исходников, 
но нам интересно попробовать разобраться без них.}
{Of course, it is easily to download both Glibc and Linux kernel source code, 
but we are interesting to understand the matter without it.}

\IFRU{Итак, при вызове сисколла \TT{sys\_open}, управление в конечном итоге передается в \TT{do\_sys\_open} в ядре Linux 2.6. 
Оттуда ~--- в \TT{do\_filp\_open()} (эта функция находится в исходниках ядра в файле \TT{fs/namei.c}).}
{So, as of Linux 2.6, when \TT{sys\_open} syscall is called, control eventually passed into \TT{do\_sys\_open} kernel function.
From there ~--- to \TT{do\_filp\_open()} function (this function located in kernel source tree in the file \TT{fs/namei.c}).}

\newcommand{\URLREGPARM}{\url{http://ohse.de/uwe/articles/gcc-attributes.html\#func-regparm}}

\IFRU{Важное отступление. Помимо передачи параметров функции через стек, существует также возможность передавать 
некоторые из них через регистры. Это называется в том числе fastcall~\ref{fastcall}. 
Это работает немного быстрее, так как процессору не нужно обращаться к стеку лежащему в памяти для чтения 
аргументов. 
В GCC есть опция \IT{regparm}\footnote{\URLREGPARM}, 
и с её помощью можно задать, сколько аргументов можно передать через регистры.}
{Important note. Aside from usual passing arguments via stack, there are also method to pass some of them
via registers. This is also called fastcall~\ref{fastcall}.
This works faster, because CPU not needed to access a stack in memory to read argument values.
GCC has option \IT{regparm}\footnote{\URLREGPARM},
and it's possible to set a number of arguments which might be passed via registers.}

\newcommand{\URLKERNELNEWB}{\url{http://kernelnewbies.org/Linux_2_6_20\#head-042c62f290834eb1fe0a1942bbf5bb9a4accbc8f}}
\newcommand{\CALLINGHFILE}{arch\textbackslash{}x86\textbackslash{}include\textbackslash{}asm\textbackslash{}calling.h}

\IFRU{Ядро Linux 2.6 собирается с опцией \TT{-mregparm=3}~\footnote{\URLKERNELNEWB}
\footnote{См. также файл \TT{\CALLINGHFILE} в исходниках ядра}.}
{Linux 2.6 kernel compiled with \TT{-mregparm=3} option~\footnote{\URLKERNELNEWB}
\footnote{See also \TT{\CALLINGHFILE} file in kernel tree}.}

\IFRU{И для нас это означает, что первые три аргумента функции будут передаваться через регистры \EAX, 
\EDX и \ECX, 
а остальные через стек. Разумеется, если аргументов у функции меньше трех, то будет задействована 
только часть регистров.}
{What it means to us, the first 3 arguments will be passed via \EAX, \EDX and \ECX registers, 
the other ones via stack. Of course, if arguments number is less than 3, only part of registers 
will be used.}

\IFRU{Итак, качаем ядро 2.6.31, собираем его в Ubuntu: \TT{make vmlinux}, открываем в \IDA, 
находим функцию \TT{do\_filp\_open()}. В начале мы увидим подобное (комментарии мои):}
{So, let's download Linux Kernel 2.6.31, compile it in Ubuntu: \TT{make vmlinux}, open it in \IDA, 
find the \TT{do\_filp\_open()} function. At the beginning, we will see (comments are mine):}

\lstinputlisting{\IFRU{14_bitfields/check2_ru.asm}{14_bitfields/check2_en.asm}}[caption=do\_filp\_open() (linux kernel 2.6.31)]

\IFRU{GCC сохраняет значения первых трех аргументов в локальном стеке. Иначе, если эти три регистра 
не трогать вообще, то функции компилятора, распределяющей переменные по регистрам (так называемый 
\IT{register allocator}), 
будет очень тесно.}
{GCC saves first 3 arguments values in local stack. 
Otherwise, if compiler would not touch these registers, 
it would be too tight environment for compiler's register allocator}.

\IFRU{Далее находим примерно такой фрагмент кода}{Let's find this fragment of code}:

\lstinputlisting{14_bitfields/check3.asm}[caption=do\_filp\_open() (linux kernel 2.6.31)]

\IFRU{$0x40$ ~--- это то чему равен макрос \TT{O\_CREAT}. 
\TT{open\_flag} проверяется на наличие бита $0x40$ и если бит равен $1$, то выполняется следующие 
за \JNZ инструкции.}
{$0x40$ ~--- is what \TT{O\_CREAT} macro equals to.
\TT{open\_flag} checked for $0x40$ bit presence, and if this bit is $1$, 
next \JNZ instruction is triggered.}

