\subsubsection{ARM + \OptimizingKeil + \ARMMode}

\begin{lstlisting}[caption=\OptimizingKeil + \ARMMode]
02 0C C0 E3          BIC     R0, R0, #0x200
01 09 80 E3          ORR     R0, R0, #0x4000
1E FF 2F E1          BX      LR
\end{lstlisting}

\TT{BIC} это ``логическое и'', аналог \AND в x86. \TT{ORR} это ``логическое или'', аналог \OR в x86.

Пока всё понятно.

\subsubsection{ARM + \OptimizingKeil + \ThumbMode}

\begin{lstlisting}[caption=\OptimizingKeil + \ThumbMode]
01 21 89 03          MOVS    R1, 0x4000
08 43                ORRS    R0, R1
49 11                ASRS    R1, R1, #5   ; generate 0x200 and place to R1
88 43                BICS    R0, R1
70 47                BX      LR
\end{lstlisting}

Вероятно, Keil решил что код в режиме thumb, получающий $0x200$ из $0x4000$, будет компактнее нежели код, 
записывающий $0x200$ в какой-нибудь регистр. 

Поэтому, при помощи инструкции \TT{ASRS} (арифметический сдвиг вправо),
это значение вычисляется как $0x4000 \gg 5$.

\subsubsection{ARM + \OptimizingXcode + \ARMMode}
\label{anomaly:LLVM}
\index{\CompilerAnomaly}

\begin{lstlisting}[caption=\OptimizingXcode + \ARMMode]
42 0C C0 E3          BIC             R0, R0, #0x4200
01 09 80 E3          ORR             R0, R0, #0x4000
1E FF 2F E1          BX              LR
\end{lstlisting}

Код, который был сгенерирован LLVM, на самом деле выглядел бы так:

\begin{lstlisting}
    REMOVE_BIT (rt, 0x4200);
    SET_BIT (rt, 0x4000);
\end{lstlisting}

И он делает то же самое что нам нужно. Но почему $0x4200$? 
\IFRU{Возможно, это артефакт оптимизатора LLVM.}
{Perhaps, that's LLVM optimizer's artifact.}
Можно сказать, ошибка оптимизатора компилятора, но создаваемый код все же работает верно.

Об аномалиях компиляторов, подробнее читайте здесь~\ref{anomaly:Intel}.

Для режима Thumb, \OptimizingXcode генерирует точно такой же код.

