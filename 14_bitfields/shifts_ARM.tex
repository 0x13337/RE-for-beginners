\subsubsection{ARM + \OptimizingXcode + \ARMMode}

\begin{lstlisting}[caption=\OptimizingXcode + \ARMMode]
                MOV             R1, R0
                MOV             R0, #0
                MOV             R2, #1
                MOV             R3, R0
loc_2E54
                TST             R1, R2,LSL R3 ; set flags according to R1 & (R2<<R3)
                ADD             R3, R3, #1    ; R3++
                ADDNE           R0, R0, #1    ; if ZF flag is cleared by TST, R0++
                CMP             R3, #32
                BNE             loc_2E54
                BX              LR
\end{lstlisting}

\TT{TST} \IFRU{это то же что и}{is the same things as} \TEST \IFRU{в}{in} x86.

\IFRU{Как я уже указывал}{As I mentioned before}~\ref{shifts_in_ARM_mode}, 
\IFRU{в режиме ARM нет отдельной инструкции для сдвигов.}
{there are no separate shifting instructions in ARM mode.}
\IFRU{Однако, модификаторами}{However, there are modifiers} 
LSL (\IT{Logical Shift Left}), 
LSR (\IT{Logical Shift Right}), 
ASR (\IT{Arithmetic Shift Right}), 
ROR (\IT{Rotate Right}) \IFRU{и}{and} 
RRX (\IT{Rotate Right with Extend}) \IFRU{можно дополнять некоторые инструкции, такие как}
{, which may be added to such instructions as} \MOV, \TT{TST},
\CMP, \ADD, \SUB, \TT{RSB}\footnote{\DataProcessingFunctionsFootNote}.

\IFRU{Эти модифицкаторы указывают, как сдвигать второй операнд, и на сколько.}
{These modificators are defines, how to shift second operand and by how many bits.}

\IFRU{Таким образом, инструкция }{Thus} \TT{``TST R1, R2,LSL R3''} 
\IFRU{здесь работает как}{instruction works here as} $R1 \land (R2 \ll R3)$.

\subsubsection{ARM + \OptimizingXcode + \ThumbTwoMode}

\IFRU{Почти такое же}{Almost the same}, 
\IFRU{только здесь применяется пара инструкций}{but here are two} 
\TT{LSL.W}/\TT{TST} 
\IFRU{вместо одной}{instructions are used instead of single} 
\TT{TST},
\IFRU{ведь в режиме thumb нельзя добавлять указывать модификатор}{becase, in thumb mode, it's not
possible to define} \TT{LSL} \IFRU{прямо в}{modifier right in} \TT{TST}.

\begin{lstlisting}
                MOV             R1, R0
                MOVS            R0, #0
                MOV.W           R9, #1
                MOVS            R3, #0
loc_2F7A
                LSL.W           R2, R9, R3
                TST             R2, R1
                ADD.W           R3, R3, #1
                IT NE
                ADDNE           R0, #1
                CMP             R3, #32
                BNE             loc_2F7A
                BX              LR
\end{lstlisting}

