\subsection{\StructurePackingSectionName}

\IFRU{Достаточно немаловажный момент, это упаковка полей в структурах\footnote{См.также: \URLWPDA}.}
{One important thing is fields packing in structures\footnote{See also: \URLWPDA}.}

\IFRU{Возьмем простой пример:}{Let's take a simple example:}

\lstinputlisting{15_structs/packing.c}

\IFRU{Как видно, мы имеем два поля \Tchar (занимающий один байт) и еще два ~--- \Tint (по 4 байта).}
{As we see, we have two \Tchar fields (each is exactly one byte) and two more ~--- \Tint (each - 4 bytes).}

\subsubsection{x86}

\IFRU{Компилируется это все в:}{That's all compiling into:}

\lstinputlisting{15_structs/packing.asm}

\IFRU{Мы видим здесь что адрес каждого поля в структуре выравнивается по 4-байтной границе. 
Так что каждый \Tchar здесь занимает те же 4 байта что и \Tint. Зачем? 
Затем что процессору удобнее обращаться по таким адресам и кешировать данные из памяти.}
{As we can see, each field's address is aligned on a 4-bytes border.
That's why each \Tchar occupies 4 bytes here (like \Tint). Why?
Thus it is easier for CPU to access memory at aligned addresses and to cache data from it.}

\IFRU{Но это не экономично по размеру данных.}{However, it is not very economical in size sense.}

\IFRU{Попробуем скомпилировать тот же исходник с опцией}{Let's try to compile it with option} (\TT{/Zp1}) 
(\IT{/Zp[n] pack structs on n-byte boundary}).

\lstinputlisting[caption=MSVC /Zp1]{15_structs/packing_msvc_Zp1.asm}

\IFRU{Теперь структура занимает 10 байт и все \Tchar занимают по байту. Что это дает? 
Экономию места. Недостаток ~--- процессор будет обращаться к этим полям не так эффективно 
по скорости как мог бы.}
{Now the structure takes only 10 bytes and each \Tchar value takes 1 byte. What it give to us?
Size economy. And as drawback ~--- CPU will access these fields without maximal performance it can.}

\IFRU{Как нетрудно догадаться, если структура используется много в каких исходниках и объектных файлах, 
все они должны быть откомпилированы с одним и тем же соглашением об упаковке структур.}
{As it can be easily guessed, if the structure is used in many source and object files,
all these should be compiled with the same convention about structures packing.}

\newcommand{\FNURLMSDNZP}{\footnote{\href{http://msdn.microsoft.com/en-us/library/ms253935.aspx}
{MSDN: Working with Packing Structures}}}
\newcommand{\FNURLGCCPC}{\footnote{\href{http://gcc.gnu.org/onlinedocs/gcc/Structure_002dPacking-Pragmas.html}
{Structure-Packing Pragmas}}}

\IFRU{Помимо ключа MSVC \TT{/Zp}, указывающего, по какой границе упаковывать поля структур, есть также 
опция компилятора \TT{\#pragma pack}, её можно указывать прямо в исходнике. 
Это справедливо и для MSVC\FNURLMSDNZP и GCC\FNURLGCCPC{}.}
{Aside from MSVC \TT{/Zp} option which set how to align each structure field, here is also
\TT{\#pragma pack} compiler option, it can be defined right in source code.
It is available in both MSVC\FNURLMSDNZP and GCC\FNURLGCCPC{}.}

\IFRU{Давайте теперь вернемся к \TT{SYSTEMTIME}, которая состоит из 16-битных полей. 
Откуда наш компилятор знает что их надо паковать по однобайтной границе?}
{Let's back to the \TT{SYSTEMTIME} structure consisting in 16-bit fields.
How our compiler know to pack them on 1-byte alignment method?}

\IFRU{В файле \TT{WinNT.h} попадается такое:}{\TT{WinNT.h} file has this:}

\begin{lstlisting}[caption=WinNT.h]
#include "pshpack1.h"
\end{lstlisting}

\IFRU{И такое:}{And this:}

\begin{lstlisting}[caption=WinNT.h]
#include "pshpack4.h"                   // 4 byte packing is the default
\end{lstlisting}

\IFRU{Сам файл PshPack1.h выглядит так:}{The file PshPack1.h looks like:}

\begin{lstlisting}[caption=PshPack1.h]
#if ! (defined(lint) || defined(RC_INVOKED))
#if ( _MSC_VER >= 800 && !defined(_M_I86)) || defined(_PUSHPOP_SUPPORTED)
#pragma warning(disable:4103)
#if !(defined( MIDL_PASS )) || defined( __midl )
#pragma pack(push,1)
#else
#pragma pack(1)
#endif
#else
#pragma pack(1)
#endif
#endif /* ! (defined(lint) || defined(RC_INVOKED)) */
\end{lstlisting}

\IFRU{Собственно, так и задается компилятору, как паковать объявленные после \TT{\#pragma pack} структуры.}
{That's how compiler will pack structures defined after \TT{\#pragma pack}.}

\subsubsection{ARM + \OptimizingKeil + \ThumbMode}

\lstinputlisting[caption=\OptimizingKeil + \ThumbMode]{15_structs/packing_Keil_thumb.asm}

\IFRU{Как мы помним, здесь передается не указатель на структуру, а сама структура, а так как в ARM первые 4 аргумента
функции передаются через регистры, то поля структуры передаются через}
{As we may recall, here a structure passed instead of pointer to structure,
and since first 4 function arguments in ARM are passed via registers,
so then structure fields are passed via} \TT{R0-R3}.

\index{ARM!\Instructions!LDRB}
\index{x86!\Instructions!MOVSX}
\IFRU{Инструкция }\TT{LDRB} \IFRU{загружает один байт из памяти и расширяет до 32-бит учитывая знак.}
{loads one byte from memory and extending it to 32-bit, taking into account its sign.}
\IFRU{Это похоже на инструкцию}{This is similar to} \MOVSX \IFRU{в}{instruction in} x86.
\IFRU{Она здесь применяется дла загрузки полей}{Here it is used for loading fields} $a$ \AndENRU $c$ 
\IFRU{из структуры}{from structure}.

\index{Function epilogue}
\IFRU{Еще что бросается в глаза, так это то что вместо эпилога функции, переход на эпилог другой функции!}
{One more thing we spot easily, instead of function epilogue, here is jump to another function's epilogue!}
\IFRU{Действительно, то была совсем другая, не относящаяся к этой, функция, однако, она имела точно такой же
эпилог}{Indeed, that was another function, not related in any way to our function, however, it has exactly
the same epilogue} 
(\IFRU{видимо, тоже хранила в стеке 5 локальных переменных}{probably because, it hold 5 local variables too} 
($5*4=0x14$)).
\IFRU{К тому же, она находится рядом (обратите внимание на адреса).}
{Also it is located nearly (take a look on addresses).}
\IFRU{Действительно, нет никакой разницы, какой эпилог исполнять, если он работает так же как нам нужно.}
{Indeed, there is no difference, which epilogue to execute,
if it works just as we need.}
\IFRU{Keil решил сделать так, вероятно, из-за экономии.}{Apparently, Keil decides to do this by a reason of economy.}
\IFRU{Эпилог занимает 4 байта, а переход ~--- только 2.}{Epilogue takes 4 bytes while jump ~--- only 2.}

\subsubsection{ARM + \OptimizingXcode + \ThumbTwoMode}

\lstinputlisting[caption=\OptimizingXcode + \ThumbTwoMode]{15_structs/packing_Xcode_thumb.asm}

\index{ARM!\Instructions!SXTB}
\index{x86!\Instructions!MOVSX}
\TT{SXTB} (\IT{Signed Extend Byte}) \IFRU{это так же аналог}{is analogous to} \MOVSX \InENRU 
x86\IFRU{, только работает не с памятью, а с регистром.}{ as well, but works not with memory, but with register.}
\IFRU{Всё остальное ~--- так же.}{All the rest ~--- just the same.}

