\subsection{\IFRU{Пример SYSTEMTIME}{SYSTEMTIME example}}

\newcommand{\FNSYSTEMTIME}{\footnote{\href{http://msdn.microsoft.com/en-us/library/ms724950(VS.85).aspx}{MSDN: SYSTEMTIME structure}}}

\IFRU{Возьмем, к примеру, структуру SYSTEMTIME\FNSYSTEMTIME{} из win32 описывающую время.}
{Let's take SYSTEMTIME\FNSYSTEMTIME{} win32 structure describing time.}

\IFRU{Она объявлена так:}{That's how it's defined:}

\begin{lstlisting}[caption=WinBase.h]
typedef struct _SYSTEMTIME {
  WORD wYear;
  WORD wMonth;
  WORD wDayOfWeek;
  WORD wDay;
  WORD wHour;
  WORD wMinute;
  WORD wSecond;
  WORD wMilliseconds;
} SYSTEMTIME, *PSYSTEMTIME;
\end{lstlisting}

\IFRU{Пишем на Си функцию для получения текущего системного времени:}
{Let's write a C function to get current time:}

\lstinputlisting{15_structs/systemtime.c}

\IFRU{Что в итоге}{We got} (MSVC 2010):

\lstinputlisting[caption=MSVC 2010]{15_structs/systemtime.asm}

\IFRU{Под структуру в стеке выделено 16 байт ~--- именно столько будет \TT{sizeof(WORD)*8}
(в структуре 8 переменных с типом WORD).}
{16 bytes are allocated for this structure in local stack ~--- that's exactly \TT{sizeof(WORD)*8}
(there are 8 WORD variables in the structure).}

\newcommand{\FNMSDNGST}{\footnote{\href{http://msdn.microsoft.com/en-us/library/ms724390(VS.85).aspx}{MSDN: GetSystemTime function}}}

\IFRU{Обратите внимание на тот факт что структура начинается с поля \TT{wYear}. 
Можно сказать что в качестве аргумента для \TT{GetSystemTime()}\FNMSDNGST передается указатель на структуру 
SYSTEMTIME, но можно также сказать, что передается указатель на поле \TT{wYear}, 
что одно и тоже! 
\TT{GetSystemTime()} пишет текущий год в тот WORD на который указывает переданный указатель, 
затем сдвигается на 2 байта вправо, пишет текущий месяц, итд, итд.}
{Pay attention to the fact the structure beginning with \TT{wYear} field.
It can be said, an pointer to SYSTEMTIME structure is passed to \TT{GetSystemTime()}\FNSYSTEMTIME,
but it's also can be said, pointer to \TT{wYear} field is passed, and that's the same!
\TT{GetSystemTime()} writting current year to the WORD pointer pointing to, then shifting 2 bytes
ahead, then writting current month, etc, etc.}

Тот факт что поля структуры это просто переменные расположенные рядом, я могу проиллюстрировать следующим образом.
Глядя на описание структуры SYSTEMTIME, мы можем переписать наш простой пример так:

\lstinputlisting{15_structs/systemtime2.c}

Компилятор немного поворчит:

\begin{lstlisting}
systemtime2.c(7) : warning C4133: 'function' : incompatible types - from 'WORD [8]' to 'LPSYSTEMTIME'
\end{lstlisting}

Тем не менее, выдаст такой код:

\lstinputlisting[caption=MSVC 2010]{15_structs/systemtime2.asm}

И это работает так же!

Любопытно что результат на ассемблере неотличим от предыдущего. Таким образом, глядя на этот код, 
никогда нельзя сказать с уверенностью, была ли там объявлена структура, либо просто набор переменных.

Тем не менее, никто в здравом уме делать так не будет. 
Потому что это неудобно. К тому же, иногда, поля в структуре могут меняться, переставляться местами, итд.


