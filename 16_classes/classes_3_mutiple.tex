\subsection{\IFRU{Множественное наследование в С++}{Multiple inheritance in C++}}

\IFRU{Множественное наследование это создание класса наследующего поля и методы от двух или более классов.}
{Multiple inheritance is a class creation which inherits fields and methods from two or more classes.}

\IFRU{Снова напишем простой пример:}{Let's write simple example again:}

\lstinputlisting{16_classes/classes3_multiple.cpp}

\IFRU{Снова скомпилируем в MSVC 2008 с опциями \Ox и \Obzero и посмотрим код методов \TT{box::dump()}, 
\TT{solid\_object::dump()}, \TT{solid\_box::dump()}:}
{Let's compile it in MSVC 2008 with \Ox and \Obzero options and let's see \TT{box::dump()}, 
\TT{solid\_object::dump()} and \TT{solid\_box::dump()} methods code:}

\lstinputlisting[caption=\Optimizing MSVC 2008 /Ob0]{16_classes/classes3_1.asm}

\lstinputlisting[caption=\Optimizing MSVC 2008 /Ob0]{16_classes/classes3_2.asm}

\lstinputlisting[caption=\Optimizing MSVC 2008 /Ob0]{16_classes/classes3_3.asm}

\IFRU{Выходит, имеем такую разметку в памяти для всех трех классов:}
{So, the memory layout for all three classes is:}

\IFRU{класс \IT{box}}{\IT{box} class}:

\begin{center}
\begin{tabular}{ | l | l | }
\hline
  \tableheader{} \\
  +0x0 & width \\
  +0x4 & height \\
  +0x8 & depth \\
\hline
\end{tabular}
\end{center}

\IFRU{класс \IT{solid\_object}}{\IT{solid\_object} class}:

\begin{center}
\begin{tabular}{ | l | l | }
\hline
  \tableheader{} \\
  +0x0 & density \\
\hline
\end{tabular}
\end{center}

\IFRU{Можно сказать, что разметка класса \IT{solid\_box} будет \IT{объедененной}:}
{It can be said, \IT{solid\_box} class memory layout will be \IT{united}:}

\IFRU{класс \IT{solid\_box}}{\IT{solid\_box} class}:

\begin{center}
\begin{tabular}{ | l | l | }
\hline
  \tableheader{} \\
  +0x0 & width \\
  +0x4 & height \\
  +0x8 & depth \\
  +0xC & density \\
\hline
\end{tabular}
\end{center}

\IFRU{Код методов \TT{box::get\_volume()} и \TT{solid\_object::get\_density()} тривиален:}
{The code of the \TT{box::get\_volume()} and \TT{solid\_object::get\_density()} methods is trivial:}

\lstinputlisting[caption=\Optimizing MSVC 2008 /Ob0]{16_classes/classes3_4.asm}

\lstinputlisting[caption=\Optimizing MSVC 2008 /Ob0]{16_classes/classes3_5.asm}

\IFRU{А вот код метода \TT{solid\_box::get\_weight()} куда интереснее:}
{But the code of the \TT{solid\_box::get\_weight()} method is much more interesting:}

\lstinputlisting[caption=\Optimizing MSVC 2008 /Ob0]{16_classes/classes3_6.asm}

\IFRU{\TT{get\_weight()} просто вызывает два метода, но для \TT{get\_volume()} он передает просто указатель на \TT{this}, а для \TT{get\_density()}, он передает указатель на \TT{this} сдвинутый на \IT{12} байт (либо \IT{0xC} байт), а там, 
в разметке класса \TT{solid\_box}, как раз начинаются поля класса \TT{solid\_object}.}
{\TT{get\_weight()} just calling two methods, but for \TT{get\_volume()} it just passing pointer to \TT{this},
and for \TT{get\_density()} it passing pointer to \TT{this} shifted by \IT{12} (or \IT{0xC}) bytes, and there,
in the \TT{solid\_box}
class memory layout, fields of the \TT{solid\_object} class are beginning.}

\IFRU{Так, метод \TT{solid\_object::get\_density()} будет полагать что работает с обычным
классом \TT{solid\_object}, а метод \TT{box::get\_volume()} будет работать только со своими тремя полями, полагая, 
что работает с обычным экземпляром класса \TT{box}.}
{Thus, \TT{solid\_object::get\_density()} method will believe it is working with usual \TT{solid\_object} class,
and \TT{box::get\_volume()} method will work with its three fields, believing this is usual object of the \TT{box} class.}

\IFRU{Таким образом, можно сказать, что экземпляр класса-наследника нескольких классов представляет в памяти просто 
\IT{объедененный} класс, содержащий все унаследованные поля. А каждый унаследованный метод вызывается с передачей
ему указателя на соответствующую часть структуры.}
{Thus, we can say, an object of some class, inheriting from several classes, representing in memory \IT{united} class, containing all inherited fields. And each inherited method called with a pointer to 
corresponding structure's part passed.}

