\subsection{\IFRU{Пример генератора случайных чисел}{Pseudo-random number generator example}}

\IFRU{Если нам нужны случайные значения с плавающей запятой в интервале от 0 до 1, самое простое это взять
генератор ПСЧ вроде Mersenne twister выдающий случайные 32-битные числа в виде DWORD, преобразовать
это число в \Tfloat и затем разделить на \TT{RAND\_MAX} (\IT{0xffffffff} в данном случае) ~--- 
полученное число будет в интервале от 0 до 1.}
{If we need float random numbers from 0 to 1, the most simplest thing is to use random numbers generator like
Mersenne twister produces random 32-bit values in DWORD form, transform this value to \Tfloat and then
dividing it by \TT{RAND\_MAX} (\IT{0xffffffff} in our case) ~--- value we got will be in 0..1 interval.}

\IFRU{Но как известно, операция деления это медленная операция. 
Сможем ли мы избежать её, как в случае с делением через умножение?}
{But as we know, division operation is slow.
Will it be possible to get rid of it, as in case of division by multiplication?}
~\ref{sec:divisionbynine}

\index{IEEE 754}
\IFRU{Вспомним состав числа с плавающей запятой: это бит знака, биты мантиссы и биты экпоненты. 
Для получения случайного числа, нам нужно просто заполнить случайными битами все биты мантиссы!}
{Let's recall what float number consisted of: sign bit, significand bits and exponent bits.
We need just to store random bits to all significand bits for getting random float number!}

\IFRU{Экспонента не может быть нулевой (иначе число будет денормализованным), 
так что в эти биты мы запишем \IT{01111111} ~--- 
это будет означать что экспонента равна единице. Далее заполняем мантиссу случайными битами, 
знак оставляем в виде 0 (что значит наше число положительное), и вуаля. 
Генерируемые числа будут в интервале от 1 до 2, так что нам еще нужно будет отнять единицу.}
{Exponent cannot be zero (number will be denormalized in this case), so we will store \IT{01111111} 
to exponent ~--- this means exponent will be 1. Then fill significand with random bits, set sign bit to
0 (which means positive number) and voilà.
Generated numbers will be in 1 to 2 interval, so we also must subtract 1 from it.}

\newcommand{\URLXOR}{\url{http://xor0110.wordpress.com/2010/09/24/how-to-generate-floating-point-random-numbers-efficiently}}

\IFRU{В моем примере\footnote{идея взята здесь: \URLXOR} 
применяется очень простой линейный конгруэнтный генератор случайных чисел, выдающий 32-битные числа.
Генератор инициализируется текущим временем в стиле UNIX.}
{Very simple linear congruential random numbers generator is used in my 
example\footnote{idea was taken from: \URLXOR}, produces 32-bit numbers. 
The PRNG initializing by current time in UNIX-style.}

\IFRU{Далее, тип \Tfloat представляется в виде \IT{union} ~--- это конструкция \CCpp позволяющая 
интерпретировать часть памти по-разному. В нашем случае, мы можем создать переменную типа \TT{union} 
и затем обращаться к ней как к \Tfloat или как к \IT{uint32\_t}. Можно сказать что это хак, причем грязный.}
{Then, \Tfloat type represented as \IT{union} ~--- it is the \CCpp construction enabling us
to interpret piece of memory as differently typed.
In our case, we are able to create a variable
of \TT{union} type and then access to it as it is \Tfloat or as it is \IT{uint32\_t}. 
It can be said, it is just a hack. A dirty one.}

\lstinputlisting{17_unions/FPU_PRNG.cpp}

MSVC 2010 (\Ox): 

\lstinputlisting{\IFRU{17_unions/FPU_PRNG_msvc_2010_Ox_ru.asm}{17_unions/FPU_PRNG_msvc_2010_Ox_en.asm}}

\IFRU{А результат GCC будет почти таким же.}{GCC produces very similar code.}

