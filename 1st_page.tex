\vspace*{\fill}

\ifdefined\ENGLISH

\huge
	Please take short survey
\normalsize

\bigskip
\bigskip
\bigskip

\dots here: \url{https://beginners.re/survey.html}.
This can be very helpful to author!

\fi % ENGLISH

\ifdefined\RUSSIAN

\huge
	Пожалуйста, заполните короткую анкету
\normalsize

\bigskip
\bigskip
\bigskip

\dots здесь: \url{https://beginners.re/survey.html}.
Это может очень помочь автору!

\fi % RUSSIAN

\ifdefined\ENGLISH

\bigskip
\bigskip
\bigskip

\huge
	\EN{My services}
\normalsize

\bigskip
\bigskip
\bigskip


The book you currently see is \href{http://beginners.re/}{free} and is \href{https://github.com/dennis714/RE-for-beginners/}{available in open source form}.
But sometimes I need to do something for money, so sorry in advance for placing my advertisement right here.

\iffalse
\Large Need documentation? \normalsize

I could try to write a documentation/reference/manual for some API, language, framework, etc.

Sometimes I'm good at finding concise and clear example for each API/language feature.
This book is an example of it.
I can try to do this in long and steady fashion.

On the other side, my English is far from fluent.
And I may need a long time for diving deep into product unknown to me.

But I'll glad to rework existing documentation project.

Example of reference I admire is Wolfram Mathematica one: \url{http://reference.wolfram.com/language/}.
\fi

\Large Reverse engineering \normalsize

I can't accept full-time job offers, I mostly work remotely on small tasks, like these:

\large Decrypting a database, managing unknown type of files \normalsize

Due to NDA agreement, I can't reveal many details about the last case, but the case in \myref{encrypted_DB1} section
is heavily based on a real case.

\large Rewriting some kind of old EXE or DLL file back to C/C++ \normalsize

\large Dongles \normalsize

Occasionally I do \href{https://en.wikipedia.org/wiki/Software_protection_dongle}{software copy-protection dongle} replacements or dongle emulators. In general, it is somewhat unlawful to break software protection, so I can do this only if these conditions are met:

\begin{itemize}
\item software company who developed the software product does not exist anymore to my best knowledge;
\item the software product is older than 10 years;
\item you have a dongle to read information from it. In other words, I can only help to those who still uses some very old software, completely satisfied with it, but afraid of dongle electrical breakage and there are no company who can still sell the dongle replacement. 
\end{itemize}

These includes ancient MS-DOS and UNIX software. Software for exotic computer architectures (like MIPS, DEC Alpha, PowerPC) accepted as well.

Examples of my work you may find here:

\begin{itemize}
\item My book devoted to reverse engineering has a part about copy-protection dongles: \ref{dongles}.
\item \href{http://yurichev.com/writings/z3_rockey.pdf}{Finding unknown algorithm using only input/output pairs and Z3 SMT solver article}
\item \href{http://yurichev.com/blog/56/}{About MicroPhar (93c46-based dongle) emulation in DosBox}.
\item \href{http://conus.info/dongle/src/microph.asm}{Source code of DOS MicroPhar emulator using EMM386 I/O interception API}
\end{itemize}

\large Contact me \normalsize

E-Mail: \GTT{\EMAIL}.

\large Still want to hire reverse engineer/security researcher on full-time basis? \normalsize

You may try \href{https://www.reddit.com/r/ReverseEngineering/comments/49cza0/rreverseengineerings_2015_triannual_hiring_thread/}{Reddit RE hiring thread}.
There is also Russian-speaking forum with a \href{https://forum.reverse4you.org/forumdisplay.php?f=252}{section devoted to RE jobs}.

\fi % ENGLISH

\vspace*{\fill}
\vfill
