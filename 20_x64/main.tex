\section{\IFRU{64 бита}{64 bits}}

\subsection{x86-64}
\index{x86-64}

\IFRU{Это расширение x86-архитуктуры до 64 бит.}{It's a 64-bit extension to x86-architecture.}

\IFRU{С точки зрения начинающего reverse engineer-а, наиболее важные отличия от 32-битного x86 это:}
{From the reverse engineer's perspective, most important differences are:}

\index{\CLanguageElements!\Pointers}
\begin{itemize}

\item
\IFRU{Почти все регистры (кроме FPU и SIMD) расширены до 64-бит и получили префикс r-. 
И еще 8 регистров добавлено. 
В итоге имеются эти регистры общего пользования:}
{Almost all registers (except FPU and SIMD) are extended to 64 bits and got r- prefix.
8 additional registers added.
Now general purpose registers are:} \TT{rax}, \TT{rbx}, \TT{rcx}, \TT{rdx}, 
\TT{rbp}, \TT{rsp}, \TT{rsi}, \TT{rdi}, \TT{r8}, \TT{r9}, \TT{r10}, 
\TT{r11}, \TT{r12}, \TT{r13}, \TT{r14}, \TT{r15}. 

\IFRU{К ним также можно обращаться так же как и прежде. Например, для доступа к младшим 32 битам \TT{RAX} 
можно использовать \EAX.}
{It's still possible to access to \IT{older} register parts as usual. 
For example, it's possible to access lower 32-bit part of the \TT{RAX} register using \EAX.}

\IFRU{У новых регистров \TT{r8-r15} также имеются их \IT{младшие части}: \TT{r8d-r15d} 
(младшие 32-битные части), 
\TT{r8w-r15w} (младшие 16-битные части), \TT{r8b-r15b} (младшие 8-битные части).}
{New \TT{r8-r15} registers also has its \IT{lower parts}: \TT{r8d-r15d} (lower 32-bit parts),
\TT{r8w-r15w} (lower 16-bit parts), \TT{r8b-r15b} (lower 8-bit parts).}

\IFRU{Удвоено количество SIMD-регистров: с 8 до 16:}
{SIMD-registers number are doubled: from 8 to 16:} \TT{XMM0-XMM15}.

\item
\IFRU{В win64 передача всех параметров немного иная, это немного похоже на fastcall~\ref{fastcall}. 
Первые 4 аргумента записываются в регистры \TT{RCX}, \TT{RDX}, \TT{R8}, \TT{R9}, а остальные ~--- в стек. 
Вызывающая функция также должна подготовить место из 32 байт чтобы вызываемая функция могла сохранить 
там первые 4 аргумента и использовать эти регистры по своему усмотрению. 
Короткие функции могут использовать аргументы прямо из регистров, но б\'{о}льшие функции могут сохранять 
их значения на будущее.}
{In Win64, function calling convention is slightly different, somewhat resembling fastcall~\ref{fastcall}.
First 4 arguments stored in the \TT{RCX}, \TT{RDX}, \TT{R8}, \TT{R9} registers, others ~--- in the stack.
Caller function should also allocate 32 bytes so the callee may save there 4 first arguments and use these 
registers for own needs.
Short functions may use arguments just from registers, but larger may save their values on the stack.}

\IFRU{См.также в соответствующем разделе о способах передачи аргументов через стек}
{See also section about calling conventions}~\ref{sec:callingconventions}.

\item
\IFRU{Сишный \Tint остается 32-битным для совместимости.}{C \Tint type is still 32-bit for compatibility.}

\item
\IFRU{Все указатели теперь 64-битные.}{All pointers are 64-bit now.}

\end{itemize}

\index{Register allocation}
\IFRU{Из-за того что регистров общего пользования теперь вдвое больше, у компиляторов теперь больше 
свободного места для маневра называемого \IT{register allocation}\footnote{распределение переменных по регистрам}. 
Для нас это означает, что в итоговом коде будет меньше локальных переменных.}
{Since now registers number are doubled, compilers has more space now for maneuvering calling 
\IT{register allocation}\footnote{assigning variables to registers}.
What it meanings for us, emitted code will contain less local variables.}

\index{DES}
\IFRU{Для примера, функция вычисляющая первый S-бокс алгоритма шифрования DES, 
она обрабатывает сразу 32/64/128/256 значений, в зависимости от типа \TT{DES\_type} (uint32, uint64, SSE2 или AVX), 
методом bitslice DES (больше об этом методе читайте здесь~\ref{bitslicedes}):}
{For example, function calculating first S-box of DES encryption algorithm, it processing
32/64/128/256 values at once (depending on \TT{DES\_type} type (uint32, uint64, SSE2 or AVX)) 
using bitslice DES method
(read more about this method here ~\ref{bitslicedes}):}

\lstinputlisting{20_x64/19_1.c}

\IFRU{Здесь много локальных переменных. Конечно, далеко не все они будут в локальном стеке. 
Компилируем обычным MSVC 2008 с опцией \Ox:}
{There is a lot of local variables. Of course, not them all will be in local stack.
Let's compile it with MSVC 2008 with \Ox option:}

\lstinputlisting[caption=\Optimizing MSVC 2008]{20_x64/19_2_msvc_Ox.asm}

\IFRU{5 переменных компилятору пришлось разместить в локальном стеке.}
{5 variables was allocated in local stack by compiler.}

\IFRU{Теперь попробуем то же самое только в 64-битной версии MSVC 2008:}
{Now let's try the same thing in 64-bit version of MSVC 2008:}

\lstinputlisting[caption=\Optimizing MSVC 2008]{20_x64/19_3_msvc_x64.asm}

\IFRU{Компилятор ничего не выделил в локальном стеке, а \TT{x36} это синоним для \TT{a5}.}
{Nothing allocated in local stack by compiler, \TT{x36} is synonym for \TT{a5}.}

\IFRU{Кстати, видно что функция сохраняет регистры \TT{RCX}, \TT{RDX} в отведенных для 
этого вызываемой функцией местах, 
а \TT{R8} и \TT{R9} не сохраняет, а начинает использовать их сразу.}
{By the way, we can see here, the function saved \TT{RCX} and \TT{RDX} registers in allocated by caller space,
but \TT{R8} and \TT{R9} are not saved but used from the beginning.}

\IFRU{Кстати, существуют процессоры с еще большим количеством регистров общего использования, например, 
Itanium ~--- 128 регистров.}
{By the way, there are CPUs with much more general purpose registers, Itanium, for example ~---
128 registers.}

\subsection{ARM}

\IFRU{64-битные инструкции в ARM появились в}{In ARM, 64-bit instructions are appeared in} ARMv8.

