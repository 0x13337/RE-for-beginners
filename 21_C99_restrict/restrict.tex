\section{C99 restrict}
\index{\CLanguageElements!C99!restrict}
\index{\CLanguageElements!restrict}
\index{FORTRAN}

\IFRU{А вот причина из-за которой программы на FORTRAN, в некоторых случаях, работают быстрее чем на Си.}
{Here is a reason why FORTRAN programs, in some cases, works faster than \CCpp ones.}

\begin{lstlisting}
void f1 (int* x, int* y, int* sum, int* product, int* sum_product, int* update_me, size_t s)
{
	for (int i=0; i<s; i++)
	{
		sum[i]=x[i]+y[i];
		product[i]=x[i]*y[i];
		update_me[i]=i*123; // some dummy value
		sum_product[i]=sum[i]+product[i];	
	};
};
\end{lstlisting}

\IFRU{Это очень простой пример, в котором есть одна особенность}{That's very simple example with one specific
thing in it}: 
\IFRU{указатель на массив}{pointer to} \TT{update\_me} \IFRU{может быть указателем на массив}{array could be
a pointer to}
\TT{sum}\IFRU{}{ array}, \TT{product}\IFRU{}{ array}, \IFRU{или даже}{or even} \TT{sum\_product}\IFRU{}{ array} 
~--- \IFRU{ведь нет ничего криминального в том чтобы аргументам функции быть такими, верно?}
{since there are no crime in it, right?}

\IFRU{Компилятор знает об этом, поэтому генерирует код, где в теле цикла будет 4 основных стадии:}
{Compiler is fully aware about it, so it generates a code with four stages in loop body:}
\begin{itemize}
\item \IFRU{вычислить следующий}{calculate next} \TT{sum[i]}
\item \IFRU{вычислить следующий}{calculate next} \TT{product[i]}
\item \IFRU{вычислить следующий}{calculate next} \TT{update\_me[i]}
\item \IFRU{вычислить следующий}{calculate next} \TT{sum\_product[i]} ~--- 
\IFRU{на этой стадии придется снова загружать из памяти подсчитанные}
{on this stage, we need to load from memory already calculated} \TT{sum[i]} \And \TT{product[i]}
\end{itemize}

\IFRU{Возможно ли соптимизировать последнюю стадию?}{Is it possible to optimize the last stage?}
\IFRU{Ведь подсчитанные}{Since already calculated} \TT{sum[i]} \And \TT{product[i]} 
\IFRU{не обязательно снова загружать из памяти, ведь мы их только что подсчитали.}
{are not necessary to load from memory again, becuse we already calculated them.}
\IFRU{Можно, но компилятор не уверен, что на третьей стадии ничего не затерлось!}
{Yes, but compiler isn't sure that nothing was overwritten on 3rd stage!}
\IFRU{Это называется}{This is called}
``pointer aliasing'', \IFRU{ситуация, когда компилятор не может быть уверен что память на которую указывает 
какой-то указатель, не изменилась.}
{a situation, when compiler can't be sure that a memory to which some pointer is pointing, wasn't changed.}

\IT{restrict} \IFRU{в стандарте Си C99}{in C99 standard}\cite[6.7.3.1]{C99TC3} 
\IFRU{это обещание, даваемое компилятору программистом, что аргументы функции отмеченные этим ключевым словом,
всегда будут указывать на разные места в памяти и пересекаться не будут.}
{is a promise, given by programmer to compiler that function arguments marked by this keyword will always
be pointing to different memory locations and never be crossed.}

\IFRU{Если быть более точным, и описывать это формально, \IT{restrict} показывает, что только данный указатель будет
использоваться для доступа к этому объекту, с которым мы работаем через этот указатель, больше никакой указатель для
этого использоваться не будет.}
{If to be more precise and describe this formally, \IT{restrict} shows that only this pointer is to be used
to access an object, with which we are working via this pointer, and no other pointer will be used for it.}
\IFRU{Можно даже сказать, что к всякому объекту, доступ будет осуществляться только через
один единственный указатель, если он отмечен как}
{It can be even said that some object will be accessed
only via one single pointer, if it's marked as} \IT{restrict}.

\IFRU{Добавим это ключевое слово к каждому аргументу-указателю}{Let's add this keyword to each argument-pointer}:

\begin{lstlisting}
void f2 (int* restrict x, int* restrict y, int* restrict sum, int* restrict product, int* restrict sum_product, 
	int* restrict update_me, size_t s)
{
	for (int i=0; i<s; i++)
	{
		sum[i]=x[i]+y[i];
		product[i]=x[i]*y[i];
		update_me[i]=i*123; // some dummy value
		sum_product[i]=sum[i]+product[i];	
	};
};
\end{lstlisting}

\IFRU{Посмотрим результаты}{Let's see results}:

\lstinputlisting[caption=GCC x64: f1()]{21_C99_restrict/f1.asm}

\lstinputlisting[caption=GCC x64: f2()]{21_C99_restrict/f2.asm}

\IFRU{Разница между скомпилированной функцией \TT{f1()} и \TT{f2()} такая}
{The difference between compiled \TT{f1()} and \TT{f2()} function is as follows}:
\In \TT{f1()}, \TT{sum[i]} \And \TT{product[i]} \IFRU{загружаются снова посреди тела цикла}
{are reloaded in the middle of loop},
\IFRU{а в}{and in} \TT{f2()} \IFRU{этого нет, используются уже подсчитанные значения}{there are no such thing,
already calculated values are used}, 
\IFRU{ведь мы ``пообещали'' компилятору}{since we ``promised'' to compiler}, 
\IFRU{что никто и ничто не изменит значения в}{that no one and nothing will change values in} \TT{sum[i]} 
\And \TT{product[i]} \IFRU{во время исполнения тела цикла}{while execution of loop body}, 
\IFRU{поэтому он ``уверен'', что значения из памяти можно не загружать снова}
{so it is ``sure'' that value from memory may not be again}.
\IFRU{Очевидно, второй вариант будет работать быстрее.}{Obviously, second example will work faster.}

\IFRU{Но что будет если указатели в аргументах функций все же будут пересекаться?}
{But what if pointers in function arguments will be crossed somehow?}
\IFRU{Это останется на совести программиста, а результаты вычислений будут неверными.}
{This will be on programmer's conscience, but results will be incorrect.}

\IFRU{Вернемся к}{Let's back to} FORTRAN. 
\IFRU{Компиляторы с этого ЯП, по умолчанию, все указатели считают таковыми}
{Compilers from this programming language treats all pointers as such}, 
\IFRU{поэтому, когда в Си не было возможности указать}{so when there are was not possible to set} \IT{restrict}, 
FORTRAN \IFRU{в этих случаях мог генерировать более быстрый код}{in these cases may generate faster code}.

\IFRU{Насколько это практично}{How practical is it}? 
\IFRU{Там где функция работает с несколькими большими блоками в памяти.}
{In the cases when function works with several big blocks in memory.}
\IFRU{Такого очень много в линейной алгебре, например.}
{There are a lot of such in linear algebra, for example.}
\IFRU{Очень много линейной алгебры используется на суперкомпьютерах/HPC,
возможно, поэтому, традиционно, там часто используется FORTRAN, до сих пор}
{A lot of linear algebra used on supercomputers/HPC, probably, that's why, traditionally, FORTRAN is still
used there}\cite{Loh:2010:IHP:1810226.1820518}.

\IFRU{Ну а когда итераций цикла не очень много, конечно, тогда прирост скорости не будет ощутимым.}
{But when there are not a big number of iterations, certainly, speed boost will not be significant.}

