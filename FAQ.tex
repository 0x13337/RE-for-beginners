\subsection*{mini-\RU{ЧаВО}\EN{FAQ}}

\newcommand{\HACKINGMdURL}{https://github.com/dennis714/RE-for-beginners/blob/master/HACKING.md}
\newcommand{\FNURLREDDIT}{\footnote{\href{http://go.yurichev.com/17027}{reddit}}}

Q: \EN{Why one should learn assembly language these days?}\RU{Зачем в наше время нужно изучать язык ассемблера?}\\
A: \EN{Unless you are an \ac{OS} developer, you probably don't need to code in assembly ---  
modern compilers are much better at performing optimizations than humans}
\RU{Если вы не разработчик \ac{OS}, вам наверное не нужно писать на ассемблере:
современные компиляторы оптимизируют код намного лучше человека}
\footnote{\RU{Очень хороший текст на эту тему}\EN{A very good text about this topic}: \cite{AgnerFog}}.
\EN{Also, modern \ac{CPU}s are very complex devices and assembly knowledge doesn't really help one to understand their internals.}
\RU{Современные \ac{CPU} это также крайне сложные устройства, и знание ассемблера вряд ли
поможет узнать его внутренности.}
\EN{That being said, there are at least two areas where a good understanding of assembly can be helpful: 
First and foremost, security/malware research. It is also a good way to gain a better understanding of your compiled code whilst debugging.}
\RU{Но все-таки остается по крайней мере две области, где знание ассемблера может хорошо
помочь:
1) исследование malware (\IT{зловредов}) в целях security research; 2) лучшее понимание
вашего скомпилированного кода в процессе отладки.}
\EN{This book is therefore intended for those who want to understand assembly language rather 
than to code in it, which is why there are many examples of compiler output contained within.}
\RU{Таким образом, эта книга предназначена для тех, кто хочет скорее понимать ассемблер,
нежели писать на нем, и вот почему здесь масса примеров связанных с результатами
работы компиляторов.}\\
\\
Q: \RU{Я кликнул на ссылку внутри PDF-документа, как теперь вернуться назад?}\EN{I clicked on a hyperlink inside of a PDF-document, how do I get back?}\\
A: (On Adobe Acrobat Reader) Alt + LeftArrow\\
\\
\ifx\LITE\undefined
Q: \RU{Ваша книга слишком большая! Нет ли чего покороче?}\EN{Your book is huge! Is there anything shorter?}\\
A: \RU{Имеется сокращенная lite-версия}\EN{There is shortened lite version found here}: \url{http://beginners.re/\#lite}.\\
\\
\fi
Q: \RU{Я не могу понять, стоит ли мне заниматься reverse engineering-ом}\EN{I'm not sure, it I should try to learn reverse engineering or not}.\\
A: \RU{Полагаю, среднее время для освоения сокращенной LITE-версии --- 1-2 месяца.}
\EN{I would say, the average time to become familiar with the contents of the shortened LITE-version is 1-2 month(s).}
\\
Q: \RU{Могу ли я распечатать эту книгу? Использовать её для обучения?}\EN{May I print this book? Use it for teaching?}\\
A: \RU{Конечно, поэтому книга и лицензирована под лицензией Creative Commons.}\EN{Of course! That's why book is licensed under Creative Commons license.}
\EN{One might also want to build one's own version of book --- read \href{\HACKINGMdURL}{here} to find out more.}
\RU{Кто-то может захотеть скомпилировать свою собственную версию книги, читайте \href{\HACKINGMdURL}{здесь} об этом.}\\
\\
Q: \RU{Я хочу перевести вашу книгу на другой язык}\EN{I want to translate your book to some other language}.\\
A: \RU{Прочитайте}\EN{Read} \href{https://github.com/dennis714/RE-for-beginners/blob/master/Translation.md}{\RU{мою заметку к переводчикам}\EN{my note to translators}}.\\
\\
Q: \RU{Как можно найти работу reverse engineer-а}\EN{How does one get a job in reverse engineering}? \\
A: \RU{На reddit, посвященному RE\FNURLREDDIT, время от времени бывают hiring thread}
\EN{There are hiring threads that appear from time to time on reddit devoted to RE\FNURLREDDIT}
(\href{http://go.yurichev.com/17333}{2013 Q3}, 
\href{http://go.yurichev.com/17334}{2014}).
\RU{Посмотрите там}\EN{Try looking there}.
\EN{A somewhat related hiring thread can be found in the ``netsec'' subreddit}\RU{В смежном субреддите ``netsec'' имеется похожий тред}: 
\href{http://go.yurichev.com/17335}{2014 Q2}.\\
\\
\RU{Q: Куда пойти учиться в Украине?\\
A: \href{http://go.yurichev.com/17336}{НТУУ <<КПИ>>: ``Аналіз програмного коду та бінарних вразливостей''};
\href{http://go.yurichev.com/17337}{факультативы}.\\
\\}
Q: \EN{I have a question}\RU{У меня есть вопрос}...\\
A: \EN{Send it to me by email}\RU{Напишите мне его емейлом} (\EMAIL) \EN{or ask your question on my forum}\RU{или задайте вопрос на моем форуме}: \href{http://forum.yurichev.com/}{forum.yurichev.com}.
