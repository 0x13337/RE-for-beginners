\subsection*{mini-%
	\RU{ЧаВО}%
	\EN{FAQ}%
	\ES{FAQ}%
	\PTBRph{}%
	\PLph{}%
	\ITAph{}%
	\DEph{}%
}

\newcommand{\HACKINGMdURL}{https://github.com/dennis714/RE-for-beginners/blob/master/HACKING.md}
\newcommand{\FNURLREDDIT}{\footnote{\href{http://go.yurichev.com/17027}{reddit.com/r/ReverseEngineering/}}}

Q:
\RU{Зачем в наше время нужно изучать язык ассемблера?}%
\EN{Why should one learn assembly language these days?}%
\ES{?`Por qu\'e deber\'ia aprender lenguaje ensamblador hoy en d\'ia?}%
\PTBRph{}%
\PLph{}%
\ITAph{}%
\DEph{}%
\\
A:
\RU{Если вы не разработчик \ac{OS}, вам наверное не нужно писать на ассемблере:
современные компиляторы оптимизируют код намного лучше человека}%
\EN{Unless you are an \ac{OS} developer, you probably don't need to code in assembly\textemdash{}modern compilers 
are much better at performing optimizations than humans}%
\ES{A menos que seas un desarrollador de \ac{OS}, probablemente no necesitas programar en ensamblador\textemdash{}los compiladores modernos
son mucho mejores generando optimizaciones que los humanos}%
\PTBRph{}%
\PLph{}%
\ITAph{}
\DEph{}%
\footnote{%
	\RU{Очень хороший текст на эту тему}%
	\EN{A very good text about this topic}%
	\ES{Un buen texto acerca de este tema}%
	\PTBRph{}%
	\PLph{}%
	\DEph{}%
	\ITAph{}: \cite{AgnerFog}}.
\RU{К тому же, современные \ac{CPU} это крайне сложные устройства и знание ассемблера вряд ли
поможет узнать их внутренности.}%
\EN{Also, modern \ac{CPU}s are very complex devices and assembly knowledge doesn't really help one to understand their internals.}%
\ES{Adem\'as, los \ac{CPU}s modernos son dispositivos muy complejos y el conocimiento de ensamblador en realidad no ayuda a comprender su funcionamiento interno.}%
\PTBRph{}%
\PLph{}%
\DEph{}%
\ITAph{}
\RU{Но все-таки остается по крайней мере две области, где знание ассемблера может хорошо
помочь:
1) исследование malware (\IT{зловредов}) с целью анализа; 2) лучшее понимание
вашего скомпилированного кода в процессе отладки.}%
\EN{That being said, there are at least two areas where a good understanding of assembly can be helpful: 
First and foremost, security/malware research. It is also a good way to gain a better understanding of your compiled code whilst debugging.}%
\ES{Una vez dicho eso, hay al menos dos \'areas donde un buen entendimiento de ensamblador puede ser \'util:
Antes que nada, la investigaci\'on de seguridad/malware. Tambi\'en es una buena manera de obtener un mejor entendimiento de tu c\'odigo compilado mientras es depurado.}%
\PTBRph{}%
\PLph{}%
\DEph{}%
\ITAph{}

\RU{Таким образом, эта книга предназначена для тех, кто хочет скорее понимать ассемблер,
нежели писать на нем, и вот почему здесь масса примеров, связанных с результатами
работы компиляторов.}%
\EN{This book is therefore intended for those who want to understand assembly language rather 
than to code in it, which is why there are many examples of compiler output contained within.}%
\ES{Por lo tanto, este libro est\'a dirigido a aquellos que desean comprender el lenguaje ensamblador en vez de codificar en \'el,
raz\'on por la cual contiene tantos ejemplos de c\'odigo generado por compilador.}%
\PTBRph{}\DEph{}\PLph{}\ITAph{} \\
\\
Q:
\RU{Я кликнул на ссылку внутри PDF-документа, как теперь вернуться назад?}%
\EN{I clicked on a hyperlink inside a PDF-document, how do I go back?}%
\ES{Di click en un link dentro del documento PDF, ?`c\'omo regreso?}%
\PTBRph{}%
\DEph{}%
\PLph{}%
\ITAph{}
\\
A:
\RU{В Adobe Acrobat Reader нажмите сочетание Alt+LeftArrow.}%
\EN{In Adobe Acrobat Reader click Alt+LeftArrow.}%
\ES{En Acrobat Reader, presiona Alt+FlechaIzquierda.}%
\PTBRph{}%
\DEph{}%
\PLph{}%
\ITAph{}
\\
\\
\ifx\LITE\undefined
Q:
\RU{Ваша книга слишком большая! Нет ли чего покороче?}%
\EN{Your book is huge! Is there anything shorter?}%
\ES{!`Tu libro es enorme! ?`Hay algo m\'as corto?}%
\PTBRph{}%
\DEph{}%
\PLph{}%
\ITAph{}
\\
A:
\RU{Есть сокращенная lite-версия:}%
\EN{There is a shortened, lite version found here:}%
\ES{Puedes encontrar una versi\'on reducida (LITE), aqu\'i:}%
\PTBRph{}%
\DEph{}%
\PLph{}%
\ITAph{}
\url{http://beginners.re/\#lite}.
\\
\\
\fi
Q:
\RU{Я не могу понять, стоит ли мне заниматься reverse engineering-ом.}%
\EN{I'm not sure if I should try to learn reverse engineering or not.}%
\ES{No estoy seguro de si deber\'ia tratar de aprender ingenier\'ia inversa o no.}%
\PTBRph{}%
\DEph{}%
\PLph{}%
\ITAph{}
\\
A:
\RU{Наверное, среднее время для освоения сокращенной LITE-версии\EMDASH{}1-2 месяца.
Вы можете попытаться также решать \href{http://challenges.re/}{задачи}).}%
\EN{Perhaps, the average time to become familiar with the contents of the shortened LITE-version is 1-2 month(s).
You may also try \href{http://challenges.re/}{reverse engineering challenges}).}%
\ES{Quiz\'a, el tiempo promedio para familiarizarse con los contenidos de la versi\'on LITE es de 1-2 meses.}% % ES-RESYNC
\PTBRph{}%
\DEph{}%
\PLph{}%
\ITAph{}
\\
\\
Q:
\RU{Могу ли я распечатать эту книгу? Использовать её для обучения?}%
\EN{May I print this book / use it for teaching?}%
\ES{?`Puedo imprimir este libro / usarlo para ense\~nanza?}%
\PTBRph{}%
\DEph{}%
\PLph{}%
\ITAph{}
\\
A:
\RU{Конечно, поэтому книга и лицензирована под лицензией Creative Commons.}%
\EN{Of course! That's why the book is licensed under the Creative Commons license.}%
\ES{!`Por supuesto! Por eso es que el libro est\'a registrado bajo Creative Commons.}%
\PTBRph{}%
\DEph{}%
\PLph{}%
\ITAph{}

\RU{Кто-то может захотеть скомпилировать свою собственную версию книги, читайте \href{\HACKINGMdURL}{здесь} об этом.}%
\EN{Someone might also want to build one's own version of book\textemdash{}read \href{\HACKINGMdURL}{here} to find out more.}%
\ES{Puede que alguien quiera generar su propia versi\'on del libro\textemdash{}lee \href{\HACKINGMdURL}{here} para m\'as informaci\'on al respecto.}%
\PTBRph{}%
\DEph{}%
\PLph{}%
\ITAph{}
\\
\\
\EN{Q: Why this book is free? You've done great job. This is suspicious, as many other free things.}%
\RU{Q: Почему эта книга бесплатная? Вы проделали большую работу. Это подозрительно, как и многие другие бесплатные вещи.}
\\
\EN{A: To my own experience, authors of technical literature do this mostly for self-advertisement purposes.
It's not possible to gain any decent money from such work.}%
\RU{A: По моему опыту, авторы технической литературы делают это, в основном ради само-рекламы.
Такой работой заработать приличные деньги невозможно.}
\\
\\
Q:
\RU{Как можно найти работу reverse engineer-а?}%
\EN{How does one get a job in reverse engineering?}%
\ES{?`C\'omo se consigue un trabajo en ingenier\'ia inversa?}%
\PTBRph{}%
\DEph{}%
\PLph{}%
\ITAph{}
\\
A:
\RU{На reddit, посвященному RE\FNURLREDDIT, время от времени бывают hiring thread}%
\EN{There are hiring threads that appear from time to time on reddit, devoted to RE\FNURLREDDIT}%
\ES{Existen threads de contrataci\'on que aparecen de vez en cuando en reddit, dedicados a reversing\FNURLREDDIT}%
\PTBRph{}%
\DEph{}%
\PLph{}%
\ITAph{}
(\href{http://go.yurichev.com/17333}{2013 Q3}, 
\href{http://go.yurichev.com/17334}{2014}).
\RU{Посмотрите там.}%
\EN{Try looking there.}%
\ES{Intenta buscando ah\'i.}%
\PTBRph{}%
\DEph{}%
\PLph{}%
\ITAph{}

\RU{В смежном субреддите \q{netsec} имеется похожий тред:}
\EN{A somewhat related hiring thread can be found in the \q{netsec} subreddit:}%
\ES{Un thread en ocasiones relacionado con contrataciones puede ser encontrado en el subreddit \q{netsec}:}%
\PTBRph{}%
\DEph{}%
\PLph{}%
\ITAph{}
\href{http://go.yurichev.com/17335}{2014 Q2}.
\\
\\
\RU{Q: Куда пойти учиться в Украине?\\
A: \href{http://go.yurichev.com/17336}{НТУУ \q{КПИ}: \q{Аналіз програмного коду та бінарних вразливостей}};
\href{http://go.yurichev.com/17337}{факультативы}.\\
\\}
Q:
\RU{У меня есть вопрос...}%
\EN{I have a question...}%
\ES{Tengo una pregunta...}%
\PTBRph{}%
\DEph{}%
\PLph{}%
\ITAph{}
\\
A:
\RU{Напишите мне его емейлом}%
\EN{Send it to me by email}%
\ES{Env\'iamela por email}%
\PTBRph{}%
\DEph{}%
\PLph{}%
\ITAph{}
(\EMAIL).
