\subsection*{mini-FAQ}

\par Q: What are prerequisites for reading this book?
\par A: Basic understanding of C/C++ is desirable.

\par Q: Can I buy Russian/English hardcopy/paper book?
\par A: Unfortunately no, no publisher got interested in publishing Russian or English version so far.
Meanwhile, you can ask your favorite copy shop to print/bind it.

\par Q: Is there epub/mobi version?
\par A: No, but there is PDF in A5 format for ebook readers.
The book is highly dependent on TeX/LaTeX-specific hacks, so converting to HTML (epub/mobi is a set of HTMLs)
will not be easy.

\par Q: Why should one learn assembly language these days?
\par A: Unless you are an \ac{OS} developer, you probably don't need to code in assembly\textemdash{}modern compilers 
are much better at performing optimizations than humans \footnote{A very good text about this topic: [\AgnerFog]}.

Also, modern \ac{CPU}s are very complex devices and assembly knowledge doesn't really help one to understand their internals.

That being said, there are at least two areas where a good understanding of assembly can be helpful: 
First and foremost, security/malware research. It is also a good way to gain a better understanding of your compiled code whilst debugging.
This book is therefore intended for those who want to understand assembly language rather 
than to code in it, which is why there are many examples of compiler output contained within.

\par Q: I clicked on a hyperlink inside a PDF-document, how do I go back?
\par A: In Adobe Acrobat Reader click Alt+LeftArrow. In Evince click ``<'' button.

\par Q: May I print this book / use it for teaching?
\par A: Of course! That's why the book is licensed under the Creative Commons license (CC BY-SA 4.0).

\par Q: Why is this book free? You've done great job. This is suspicious, as many other free things.
\par A: In my own experience, authors of technical literature do this mostly for self-advertisement purposes. It's not possible to get any decent money from such work.

\par Q: How does one get a job in reverse engineering?
\par A: There are hiring threads that appear from time to time on reddit, devoted to RE\FNURLREDDIT{}
(\href{http://go.yurichev.com/17333}{2013 Q3}, 
\href{http://go.yurichev.com/17334}{2014}).
Try looking there.

A somewhat related hiring thread can be found in the \q{netsec} subreddit:
\href{http://go.yurichev.com/17335}{2014 Q2}.

\par Q: I have a question...
\par A: Send it to me by email (\EMAIL).

