\subsection*{mini-FAQ}

\par P: Por que alguém deveria aprender assembly nos dias de hoje?
\par R: A não ser que você seja um desenvolvedor de sistemas operacionais, você provavelmente não precisará escrever códigos em assembly – visto que os compiladores modernos são muito melhores em executar operações 
do que nós humanos
\footnote{(\ac{TBT}: A very good text about this topic: \InSqBrackets{\AgnerFog}}.
Também, \ac{CPU}s modernas são dispositivos muito complexos e apenas o conhecimento de assembly não ajudará tanto ao seu entendimento interno. Sabendo disso, há ao menos duas áreas onde um bom entendimento de assembly pode ser proveitoso. Primeiro e mais importante, pesquisar sobre segurança e malwares. É também uma ótima maneira de acumular conhecimento sobre o seu código enquanto ele estiver “debugando”. Esse livro é mais voltado para os que querem entender a linguagem assembly do que codificar com ela, por isso há bastante exemplos de saídas de compiladores nele.

\par P: Eu cliquei em um link dentro de um documento PDF, como eu volto para ele?
\par R: No Adobe Acrobat Reader, pressione Alt+SetaEsquerda

\par P: Eu posso imprimir esse livro / usá-lo para ensinar?
\par R: Lógico! É por isso que o livro é registrado sob a “Creative Commons license” (CC BY-SA 4.0).

\par P: Por que esse livro é de graça? Você fez um ótimo trabalho. Isso é suspeito, assim como outras coisas grátis.
\par R: Pela minha própria experiência, autores de literaturas técnicas fazem isso mais para se divulgares. Não é possível ganhar uma quantia considerável de dinheiro com esse tipo de publicação.

\par P: Como alguém consegue um trabalho em engenharia reversa?
\par R: Há vários tópicos que aparecem de tempos em tempos no Reddit\FNURLREDDIT{}, em (\href{http://go.yurichev.com/17333}{2013 Q3}, \href{http://go.yurichev.com/17334}{2014}) 
e também tópicos relacionados a contratações podem ser encontrados no subreddit ``netsec'' (\href{http://go.yurichev.com/17335}{2014 Q2}).

\par P: Eu tenho uma pergunta...
\par R: Me envie por e-mail (\EMAIL).

