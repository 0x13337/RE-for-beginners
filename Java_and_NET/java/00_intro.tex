% TODO proof-reading
\subsection{\RU{Введение}\EN{Introduction}}

\newcommand{\JADURL}{\url{http://varaneckas.com/jad/}}

\RU{Есть немало известных декомпиляторов для Java (или для \ac{JVM}-байткода вообще)
\footnote{Например, JAD: \JADURL}.}
\EN{There are some well-known decompilers for Java (or \ac{JVM} bytecode in general)
\footnote{For example, JAD: \JADURL}.}

\RU{Причина в том что декомпиляция \ac{JVM}-байткода проще чем низкоуровневого x86-кода:}
\EN{The reason is the decompilation of \ac{JVM}-bytecode is somewhat easier 
than for lower level x86 code:}

\begin{itemize}
\item \RU{Здесь намного больше информации о типах.}
      \EN{There is much more information about the data types.}
\item \EN{The \ac{JVM} memory model is much more rigorous and outlined.}
      \RU{Модель памяти в \ac{JVM} более строгая и очерченная.}
\item \EN{The Java compiler don't do any optimizations (the \ac{JVM} \ac{JIT} does them at runtime),
      so the bytecode in the class files is usually pretty readable.}
      \RU{Java-компилятор не делает никаких оптимизаций (это делает \ac{JVM} \ac{JIT} во время 
       исполнения), так что байткод в class-файлах легко читаем.}
\end{itemize}

\EN{When can the knowledge of \ac{JVM} be useful?}
\RU{Когда знания \ac{JVM}-байткода могут быть полезны?}

\newcommand{\URLListOfJVMLangs}{\url{http://en.wikipedia.org/wiki/List_of_JVM_languages}}

\begin{itemize}
\item \EN{Quick-and-dirty patching tasks of class files without the need to recompile the decompiler's results.}
      \RU{Мелкая/несложная работа по патчингу class-файлов без необходимости снова компилировать результаты 
      декомпилятора.}
\item \EN{Analysing obfuscated code}\RU{Анализ обфусцированного кода}.
\item \EN{Building your own obfuscator}\RU{Создание вашего собственного обфускатора}.
\item \EN{Building a compiler codegenerator (back-end) targeting \ac{JVM} (like Scala, Clojure, etc
      \footnote{Full list: \URLListOfJVMLangs}).}
      \RU{Создание кодегенератора компилятора (back-end), создающего код для \ac{JVM} (как Scala, Clojure, итд
      \footnote{Полный список: \URLListOfJVMLangs}).}
\end{itemize}

\EN{Let's start with some simple pieces of code}\RU{Начнем с простых фрагментов кода}.
\EN{JDK 1.7 is used everywhere, unless mentioned otherwise.}
\RU{Если не указано иное, везде используется JDK 1.7.}

\EN{This is the command used to decompile class files everywhere}
\RU{Эта команда использовалась везде для декомпиляции class-файлов}: \GTT{javap -c -verbose}

\EN{This is the book I used while preparing all examples}
\RU{Эта книга использовалась мною для подготовки всех примеров}: \JavaBook.

