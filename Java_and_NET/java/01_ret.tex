% TODO proof-reading
\section{\RU{Возврат значения}\EN{Returning a value}}

\EN{Probably the simplest Java function is one which returns some value.}
\RU{Наверное, самая простая из всех возможных функций на Java это та, что возвращает 
некоторое значение.}
\EN{Oh, and we must keep in mind that there are no \q{free} functions in Java in common sense,
they are \q{methods}. }
\RU{О, и мы не должны забывать что в Java нет \q{свободных} функций в общем смысле,
это \q{методы}.}
\EN{Each method is related to some class, so it's not possible to define
method outside of a class.}
\RU{Каждый метод принадлежит какому-то классу, так что невозможно объявить метод 
вне какого-либо класса.}
\EN{But I'll also call them \q{functions} anyway, because I used to.}
\RU{Но я все равно буду называть их \q{функциями}, потому что я так привык.}

\begin{lstlisting}
public class ret
{
	public static int main(String[] args) 
	{
		return 0;
	}
}
\end{lstlisting}

\RU{Я компилирую это}\EN{I'll compile it}:

\begin{lstlisting}
javac ret.java
\end{lstlisting}

\dots \EN{and decompile it using standard Java utility}\RU{и декомпилирую используя стандартную утилиту в Java}:

\begin{lstlisting}
javap -c -verbose ret.class
\end{lstlisting}

\EN{And what I've got}\RU{И получаю}:

\begin{lstlisting}[caption=JDK 1.7 (excerpt)]
  public static int main(java.lang.String[]);
    flags: ACC_PUBLIC, ACC_STATIC
    Code:
      stack=1, locals=1, args_size=1
         0: iconst_0      
         1: ireturn       
\end{lstlisting}

\EN{Java developers decide that 0 is one of the busiest constants in programming, 
so there are separate short one-byte \TT{iconst\_0} instruction which pushes 0}
\RU{Разработчики Java решили что 0 это самая используемая константа в программировании,
так что здесь есть отдельная однобайтная инструкция \TT{iconst\_0}, заталкивающая 0 в стек}
\footnote{\EN{Just like in MIPS, where a separate register for zero constant exists}
\RU{Так же как и в MIPS, где для нулевой константы имеется отдельный регистр}: 
\myref{MIPS_zero_register}.}.
\EN{There are also \TT{iconst\_1} (which pushes 1), \TT{iconst\_2}, etc, 
up to \TT{iconst\_5}.}
\RU{Здесь есть также \TT{iconst\_1} (заталкивающая 1), \TT{iconst\_2}, итд, 
вплоть до \TT{iconst\_5}.}
\EN{There are also}\RU{Есть также} \TT{iconst\_m1} \RU{заталкивающая}\EN{which pushes} -1.

Stack is used in JVM for data passing into functions to be called and also returning values.
So \TT{iconst\_0} pushed 0 into stack.
\TT{ireturn} returns integer value (\IT{i} in name mean \IT{integer}) from the \ac{TOS}.

\RU{Немного перепишем наш пример, теперь возвращаем}
\EN{Let's rewrite our example slightly, now we return} 1234:

\begin{lstlisting}
public class ret
{
	public static int main(String[] args)
	{
		return 1234;
	}
}
\end{lstlisting}

\dots \RU{получаем}\EN{we got}:

\begin{lstlisting}[caption=JDK 1.7 (excerpt)]
  public static int main(java.lang.String[]);
    flags: ACC_PUBLIC, ACC_STATIC
    Code:
      stack=1, locals=1, args_size=1
         0: sipush        1234
         3: ireturn       
\end{lstlisting}

\TT{sipush} (\IT{short integer}) \RU{заталкивает значение}\EN{pushes} 1234 \RU{в стек}\EN{value into stack}.
\IT{short} \EN{in name implies a 16-bit value is to be pushed}\RU{в имени означает, что 16-битное значение будет заталкиваться в стек}. 
\EN{1234 number is indeed well fit in 16-bit value.}
\RU{Число 1234 действительно помещается в 16-битное значение.}

\EN{What about larger values?}
\RU{Как насчет б\'{о}льших значений?}

\begin{lstlisting}
public class ret
{
	public static int main(String[] args) 
	{
		return 12345678;
	}
}
\end{lstlisting}

\begin{lstlisting}[caption=Constant pool]
...
   #2 = Integer            12345678
...
\end{lstlisting}

\begin{lstlisting}
  public static int main(java.lang.String[]);
    flags: ACC_PUBLIC, ACC_STATIC
    Code:
      stack=1, locals=1, args_size=1
         0: ldc           #2                  // int 12345678
         2: ireturn       
\end{lstlisting}

\EN{It's not possible to encode a 32-bit number in a JVM instruction opcode, 
developers didn't left such possibility.}
\RU{Невозможно закодировать 32-битное число в опкоде какой-либо JVM-инструкции, 
разработчики не оставили такой возможности.}
\EN{So 32-bit number 12345678 is stored in so called \q{constant pool} which is, let's say,
library of most used constants (including strings, objects, etc).}
\RU{Так что 32-битное число 12345678 сохранено в так называемом \q{constant pool} (пул констант),
который, так скажем, является библиотекой наиболее используемых констант (включая строки, объекты,
итд).}

\EN{This way of passing constants is not unique to JVM.}
\RU{Этот способ передачи констант не уникален для JVM.}
\EN{MIPS, ARM and other RISC CPUs also can't encode 32-bit numbers
in 32-bit opcode, so RISC CPU code (including MIPS and ARM) has to construct values 
in several steps, or to keep them in data segment:}
\RU{MIPS, ARM и прочие RISC-процессоры не могут кодировать 32-битные числа в 32-битных опкодах,
так что код для RISC-процессоров (включая MIPS и ARM) должен конструировать значения 
в несколько шагов, или держать их в сегменте данных:}
\myref{ARM_big_constants}, \myref{MIPS_big_constants}.

\EN{MIPS code is also traditionally has constant pool, named \q{literal pool}, these are segments
called \q{.lit4} (for 32-bit single precision float point number constants storage) and \q{.lit8}
(for 64-bit double precision float point number constants storage).}
\RU{Код для MIPS также традиционно имеет пул констант, называемый \q{literal pool}, это сегменты
с названиями \q{.lit4} (для хранения 32-битных чисел с плавающей точкой одинарной точности) и
\q{.lit8}(для хранения 64-битных чисел с плавающей точкой двойной точности).}

\RU{Попробуем другие типы данных}\EN{Let's also try other data types}!

Boolean:

\begin{lstlisting}
public class ret
{
	public static boolean main(String[] args) 
	{
		return true;
	}
}
\end{lstlisting}

\begin{lstlisting}
  public static boolean main(java.lang.String[]);
    flags: ACC_PUBLIC, ACC_STATIC
    Code:
      stack=1, locals=1, args_size=1
         0: iconst_1      
         1: ireturn       
\end{lstlisting}

\EN{This JVM bytecode is no different from one returning integer 1.}
\RU{Этот JVM-байткод не отличается от того, что возвращает целочисленную 1.}
\EN{32-bit data slots in stack are also used here for boolean values, like in \CCpp.}
\RU{32-битные слоты данных в стеке также используются для булевых значений, как в \CCpp.}
\EN{But one could not use returned boolean value as integer or vice versa\EMDASH{}type information is stored in a class file and checked at runtime.}
\RU{Но нельзя использовать возвращаемое значение булевого типа как целочисленное и наоборот\EMDASH{}информация 
о типах сохраняется в class-файлах и проверяется при запуске.}

\EN{The same story about 16-bit}\RU{Та же история с 16-битным} \IT{short}:

\begin{lstlisting}
public class ret
{
	public static short main(String[] args) 
	{
		return 1234;
	}
}
\end{lstlisting}

\begin{lstlisting}
  public static short main(java.lang.String[]);
    flags: ACC_PUBLIC, ACC_STATIC
    Code:
      stack=1, locals=1, args_size=1
         0: sipush        1234
         3: ireturn       
\end{lstlisting}

\dots \AndENRU \IT{char}!

\begin{lstlisting}
public class ret
{
	public static char main(String[] args) 
	{
		return 'A';
	}
}
\end{lstlisting}

\begin{lstlisting}
  public static char main(java.lang.String[]);
    flags: ACC_PUBLIC, ACC_STATIC
    Code:
      stack=1, locals=1, args_size=1
         0: bipush        65
         2: ireturn       
\end{lstlisting}

\TT{bipush} \RU{означает}\EN{mean} \q{push byte}.
\EN{Needless to say that \IT{char} in Java is 16-bit UTF-16 character, 
and it's equivalent to \IT{short}, but ASCII code of \q{A} character is 65, and it's possible
to use instruction for passing byte into stack.}
\RU{Нужно сказать что \IT{char} в Java, это 16-битный символ в кодировке UTF-16,
и он эквивалентен \IT{short}, но ASCII-код символа \q{A} это 65, и можно воспользоваться
инструкцией для передачи байта в стек.}

\RU{Попробуем также}\EN{Let's also try} \IT{byte}:

\begin{lstlisting}
public class retc
{
	public static byte main(String[] args) 
	{
		return 123;
	}
}
\end{lstlisting}

\begin{lstlisting}
  public static byte main(java.lang.String[]);
    flags: ACC_PUBLIC, ACC_STATIC
    Code:
      stack=1, locals=1, args_size=1
         0: bipush        123
         2: ireturn       
\end{lstlisting}

\EN{One may ask, why to bother with 16-bit \IT{short} date type which is internally works
as 32-bit integer?}
\RU{Кто-то может спросить, зачем заморачиваться использованием 16-битного типа \IT{short}, который
внутри все равно 32-битный integer?}
\EN{Why use \IT{char} data type if it is the same as \IT{short} data type?}
\RU{Зачем использовать тип данных \IT{char}, если это то же самое что и тип \IT{short}?}

\EN{The answer is simple: for data type control and source code readability.}
\RU{Ответ прост: для контроля типов данных и читабельности исходников.}
\EN{\IT{char} may essentially be the same as \IT{short} but we quickly grasp that it's placeholder for
UTF-16 character, and not for some other integer value.}
\RU{\IT{char} может быть эквивалентом \IT{short}, но мы быстро понимаем что это ячейка
для символа в кодировке UTF-16, а не для какого-то другого целочисленного значения.}
\EN{When using \IT{short} we may show to everyone that a variable range is limited by 16 bits.}
\RU{Когда используем \IT{short}, мы можем показать всем, что диапазон этой переменной 
ограничен 16-ю битами.}
\EN{It's a very good idea to use \IT{boolean} type where it needs to, 
instead of C-style \IT{int} for the same purpose.}
\RU{Очень хорошая идея использовать тип \IT{boolean} где нужно, 
вместо \IT{int} для тех же целей, как это было в Си.}

\EN{There are also 64-bit integer data type in Java:}
\RU{В Java есть также 64-битный целочисленный тип:}

\begin{lstlisting}
public class ret3
{
	public static long main(String[] args)
	{
		return 1234567890123456789L;
	}
}
\end{lstlisting}

\begin{lstlisting}[caption=Constant pool]
...
   #2 = Long               1234567890123456789l
...
\end{lstlisting}

\begin{lstlisting}
  public static long main(java.lang.String[]);
    flags: ACC_PUBLIC, ACC_STATIC
    Code:
      stack=2, locals=1, args_size=1
         0: ldc2_w        #2                  // long 1234567890123456789l
         3: lreturn       
\end{lstlisting}

\EN{The 64-bit number is also stored in constant pool, \TT{ldc2\_w} loads it and \TT{lreturn} 
(\IT{long return}) returns it.}
\RU{64-битное число также хранится в пуле констант, \TT{ldc2\_w} загружает его и \TT{lreturn} 
(\IT{long return}) возвращает его.}

\EN{\TT{ldc2\_w} instruction is also used to load double precision floating point numbers 
(which also occupies 64 bits) from constant pool:}
\RU{Инструкция \TT{ldc2\_w} также используется для загрузки чисел с плавающей точкой двойной 
точности (которые также занимают 64 бита) из пула констант:}

\begin{lstlisting}
public class ret
{
	public static double main(String[] args)
	{
		return 123.456d;
	}
}
\end{lstlisting}

\begin{lstlisting}[caption=Constant pool]
...
   #2 = Double             123.456d
...
\end{lstlisting}

\begin{lstlisting}
  public static double main(java.lang.String[]);
    flags: ACC_PUBLIC, ACC_STATIC
    Code:
      stack=2, locals=1, args_size=1
         0: ldc2_w        #2                  // double 123.456d
         3: dreturn       
\end{lstlisting}

\EN{\TT{dreturn} stands for \q{return double}.}
\RU{\TT{dreturn} означает \q{return double}.}

\EN{And finally, single precision floating point number:}
\RU{И наконец, числа с плавающей точкой одинарной точности:}

\begin{lstlisting}
public class ret
{
	public static float main(String[] args)
	{
		return 123.456f;
	}
}
\end{lstlisting}

\begin{lstlisting}[caption=Constant pool]
...
   #2 = Float              123.456f
...
\end{lstlisting}

\begin{lstlisting}
  public static float main(java.lang.String[]);
    flags: ACC_PUBLIC, ACC_STATIC
    Code:
      stack=1, locals=1, args_size=1
         0: ldc           #2                  // float 123.456f
         2: freturn       
\end{lstlisting}

\EN{\TT{ldc} instruction used here is the same as used for 32-bit integer numbers loading
from constant pool.}
\RU{Используемая здесь инструкция \TT{ldc} та же, что и для загрузки 32-битных целочисленных чисел
из пула констант.}
\EN{\TT{freturn} stands for \q{return float}.}
\RU{\TT{freturn} означает \q{return float}.}

\EN{Now what about function returning nothing?}
\RU{А что насчет тех случаев, когда функция ничего не возвращает?}

\begin{lstlisting}
public class ret
{
	public static void main(String[] args) 
	{
		return;
	}
}
\end{lstlisting}

\begin{lstlisting}
  public static void main(java.lang.String[]);
    flags: ACC_PUBLIC, ACC_STATIC
    Code:
      stack=0, locals=1, args_size=1
         0: return        
\end{lstlisting}

\EN{This implies, \TT{return} instruction is used to return control without returning 
actual value.}
\RU{Это означает, что инструкция \TT{return} используется для возврата управления 
без возврата какого-либо значения.}
\EN{Knowing all this, it's very easy to deduct function (or method) returning type 
from the last instruction.}
\RU{Зная все это, по последней инструкции очень легко определить тип возвращаемого 
значения функции (или метода).}
