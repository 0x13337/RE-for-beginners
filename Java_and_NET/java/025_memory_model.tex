% TODO proof-reading
\section{\EN{\ac{JVM} memory model}\RU{Модель памяти в \ac{JVM}}}

\EN{x86 and other low-level environments uses stack for arguments passing and 
as local variables storage.}
\RU{x86 и другие низкоуровневые среды используют стек для передачи аргументов и как
хранилище локальных переменных.}
\ac{JVM} \EN{is slightly different}\RU{устроена немного иначе}.

\EN{It has}\RU{Тут есть}:

\begin{itemize}
\item \EN{Local variable array}\RU{Массив локальных переменных} (\ac{LVA}).
\EN{It is used as storage for incoming function arguments and local variables.}
\RU{Используется как хранилище для аргументов функций и локальных переменных.}
\EN{Instructions like \TT{iload\_0} loads values from it.}
\RU{Инструкции вроде \TT{iload\_0} загружают значения оттуда.}
\TT{istore} \EN{stores values to it}\RU{записывает значения туда}.
\EN{First, function arguments are came: starting at 0 or at 1 
(if zeroth argument is occupied by \IT{this} pointer).}
\RU{В начале идут аргументы функции: начиная с 0, или с 1 
(если нулевой аргумент занят указателем \IT{this}.}
\EN{Then local variables are allocated.}
\RU{Затем располагаются локальные переменные.}

\EN{Each slot has size of 32-bit.}
\RU{Каждый слот имеет размер 32 бита.}
\EN{Hence, values of \IT{long} and \IT{double} data types occupy two slots.}
\RU{Следовательно, значения типов \IT{long} и \IT{double} занимают два слота.}

\item \EN{Operand stack (or just \q{stack})}\RU{Стек операндов (или просто \q{стек})}.
\EN{It's used for computations and passing arguments while calling other functions.}
\RU{Используется для вычислений и для передачи аргументов во время вызова других функций.}
\EN{Unlike low-level environments like x86, it's not possible to access the stack without using
instructions which explicitely pushes or pops values to/from it.}
\RU{В отличие от низкоуровневых сред вроде x86, здесь невозможно работать со стеком
без использования инструкций, которые явно заталкивают или выталкивают значения туда/оттуда.}

\item \EN{Heap. It is used as storage for objects and arrays.}
\RU{Куча (heap). Используется как хранилище для объектов и массивов.}
\end{itemize}

\EN{These 3 areas are isolated from each other.}
\RU{Эти 3 области изолированы друг от друга.}
