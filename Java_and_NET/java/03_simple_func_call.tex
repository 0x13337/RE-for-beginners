% TODO translation
% TODO proof-reading
\section{Simple function calling}

\TT{Math.random()} returns a pseudorandom number in range of [0.0 \dots 1.0), but let's say,
by some reason, we need to devise a function returning number in range of [0.0 \dots 0.5):

\begin{lstlisting}
public class HalfRandom
{ 
	public static double f()
	{
		return Math.random()/2;
	}
}
\end{lstlisting}

\begin{lstlisting}[caption=Constant pool]
...
   #2 = Methodref          #18.#19        //  java/lang/Math.random:()D
   #3 = Double             2.0d
...
  #12 = Utf8               ()D
...
  #18 = Class              #22            //  java/lang/Math
  #19 = NameAndType        #23:#12        //  random:()D
  #22 = Utf8               java/lang/Math
  #23 = Utf8               random
\end{lstlisting}

\begin{lstlisting}
  public static double f();
    flags: ACC_PUBLIC, ACC_STATIC
    Code:
      stack=4, locals=0, args_size=0
         0: invokestatic  #2                  // Method java/lang/Math.random:()D
         3: ldc2_w        #3                  // double 2.0d
         6: ddiv          
         7: dreturn       
\end{lstlisting}

\TT{invokestatic} calls for the \TT{Math.random()} function and left result at the \ac{TOS}.
Then the result is divided by 0.5 and returned.
But how function name is encoded?
It's encoded in the constant pool using \TT{Methodref} clause. 
It defines class and method names.
First field of \TT{Methodref} is points to the \TT{Class} clause which is, in turn, points to
the usual text string (``java/lang/Math'').
Second \TT{Methodref} clause points to \TT{NameAndType} clause which also has two links 
to the strings.
First string is ``random'' which is the name of method.
The second string is ``()D'' which encodes function type. It means it returns double value
(hence \IT{D} in the string).
This is the way 1) JVM can check data type correctness; 2) Java decompilers can restore data types
from a compiled class file.

Now let's finally try ``Hello, world!'' example:

\begin{lstlisting}
public class HelloWorld
{
	public static void main(String[] args)
	{
		System.out.println("Hello, World");
	}
}
\end{lstlisting}

\begin{lstlisting}[caption=Constant pool]
...
   #2 = Fieldref           #16.#17        //  java/lang/System.out:Ljava/io/PrintStream;
   #3 = String             #18            //  Hello, World
   #4 = Methodref          #19.#20        //  java/io/PrintStream.println:(Ljava/lang/String;)V
...
  #16 = Class              #23            //  java/lang/System
  #17 = NameAndType        #24:#25        //  out:Ljava/io/PrintStream;
  #18 = Utf8               Hello, World
  #19 = Class              #26            //  java/io/PrintStream
  #20 = NameAndType        #27:#28        //  println:(Ljava/lang/String;)V
...
  #23 = Utf8               java/lang/System
  #24 = Utf8               out
  #25 = Utf8               Ljava/io/PrintStream;
  #26 = Utf8               java/io/PrintStream
  #27 = Utf8               println
  #28 = Utf8               (Ljava/lang/String;)V
...
\end{lstlisting}

\begin{lstlisting}
  public static void main(java.lang.String[]);
    flags: ACC_PUBLIC, ACC_STATIC
    Code:
      stack=2, locals=1, args_size=1
         0: getstatic     #2                  // Field java/lang/System.out:Ljava/io/PrintStream;
         3: ldc           #3                  // String Hello, World
         5: invokevirtual #4                  // Method java/io/PrintStream.println:(Ljava/lang/String;)V
         8: return        
\end{lstlisting}

\TT{ldc} at offset 3 is taking pointer to the ``Hello, World'' string in constant pool 
and pushes into stack.
It's called \IT{reference} in the Java world, but it's rather pointer, or address.

The familiar \TT{invokevirtual} instruction takes information about \TT{println} function (or method) 
from the constant pool and call it.
As we may know, there are several \TT{println} methods, dedicated for each data type.
Our case is \TT{println} for \IT{String} data type.

But what about the first \TT{getstatic} instruction?
This instruction takes a reference (or address of) field of object (\TT{System.out}) 
and pushes it into the stack.
This value is acting like \IT{this} pointer for \TT{println} method.
Thus, \TT{println} method takes two arguments internally:
1) \IT{this}, i.e., pointer to object; 
2) address of ``Hello, World'' string.
Indeed, \TT{println()} is called as a method within initialized \TT{System.out} object.

For convenience, \TT{javap} utility writes all this information in comments.
