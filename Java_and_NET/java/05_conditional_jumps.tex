% TODO proof-reading
\section{\EN{Conditional jumps}\RU{Условные переходы}}

\EN{Now let's proceed to conditional jumps}\RU{Перейдем к условным переходам}.

\begin{lstlisting}
public class abs
{
	public static int abs(int a)
	{
		if (a<0)
			return -a;
		return a;
	}
}
\end{lstlisting}

\begin{lstlisting}
  public static int abs(int);
    flags: ACC_PUBLIC, ACC_STATIC
    Code:
      stack=1, locals=1, args_size=1
         0: iload_0       
         1: ifge          7
         4: iload_0       
         5: ineg          
         6: ireturn       
         7: iload_0       
         8: ireturn       
\end{lstlisting}

\EN{\TT{ifge} jumps to offset 7 if value at \ac{TOS} is greater or equal to 0.}
\RU{\TT{ifge} переходит на смещение 7 если значение на \ac{TOS} больше или равно 0.}
\EN{Don't forget, any \TT{ifXX} instruction pops the value (to be compared) from stack.}
\RU{Не забывайте, любая инструкция \TT{ifXX} выталкивает значение (с которым будет производиться
сравнение) из стека.}

\EN{\TT{ineg} just negates value at \ac{TOS}.}
\RU{\TT{ineg} просто меняет знак значения на \ac{TOS}.}

\EN{Another example}\RU{Еще пример}:

\begin{lstlisting}
	public static int min (int a, int b)
	{
		if (a>b)
			return b;
		return a;
	}
\end{lstlisting}

\EN{What we've got}\RU{Получаем}:

\begin{lstlisting}
  public static int min(int, int);
    flags: ACC_PUBLIC, ACC_STATIC
    Code:
      stack=2, locals=2, args_size=2
         0: iload_0       
         1: iload_1       
         2: if_icmple     7
         5: iload_1       
         6: ireturn       
         7: iload_0       
         8: ireturn       
\end{lstlisting}

\TT{if\_icmple} \EN{pops two values and compares them}\RU{выталкивает два значения и сравнивает их}.
\EN{If the second one is lesser (or equal) than first, jump to offset 7 is occurred.}
\RU{Если второе меньше первого (или равно), происходит переход на смещение 7.}

\EN{When we define \TT{max()} function \dots}
\RU{Когда мы определяем функцию \TT{max()} \dots}

\begin{lstlisting}
	public static int max (int a, int b)
	{
		if (a>b)
			return a;
		return b;
	}
\end{lstlisting}

\EN{\dots resulting code is just the same, but two last \TT{iload} instructions 
(at offsets 5 and 7) are swapped:}
\RU{\dots итоговый код точно такой же, только последние инструкции \TT{iload} 
(на смещениях 5 и 7) поменены местами:}

\begin{lstlisting}
  public static int max(int, int);
    flags: ACC_PUBLIC, ACC_STATIC
    Code:
      stack=2, locals=2, args_size=2
         0: iload_0       
         1: iload_1       
         2: if_icmple     7
         5: iload_0       
         6: ireturn       
         7: iload_1       
         8: ireturn       
\end{lstlisting}

\EN{More complex example}\RU{Более сложный пример}:

\begin{lstlisting}
public class cond
{
	public static void f(int i)
	{
		if (i<100)
			System.out.print("<100");
		if (i==100)
			System.out.print("==100");
		if (i>100)
			System.out.print(">100");
		if (i==0)
			System.out.print("==0");
	}
}
\end{lstlisting}

\begin{lstlisting}
  public static void f(int);
    flags: ACC_PUBLIC, ACC_STATIC
    Code:
      stack=2, locals=1, args_size=1
         0: iload_0       
         1: bipush        100
         3: if_icmpge     14
         6: getstatic     #2                  // Field java/lang/System.out:Ljava/io/PrintStream;
         9: ldc           #3                  // String <100
        11: invokevirtual #4                  // Method java/io/PrintStream.print:(Ljava/lang/String;)V
        14: iload_0       
        15: bipush        100
        17: if_icmpne     28
        20: getstatic     #2                  // Field java/lang/System.out:Ljava/io/PrintStream;
        23: ldc           #5                  // String ==100
        25: invokevirtual #4                  // Method java/io/PrintStream.print:(Ljava/lang/String;)V
        28: iload_0       
        29: bipush        100
        31: if_icmple     42
        34: getstatic     #2                  // Field java/lang/System.out:Ljava/io/PrintStream;
        37: ldc           #6                  // String >100
        39: invokevirtual #4                  // Method java/io/PrintStream.print:(Ljava/lang/String;)V
        42: iload_0       
        43: ifne          54
        46: getstatic     #2                  // Field java/lang/System.out:Ljava/io/PrintStream;
        49: ldc           #7                  // String ==0
        51: invokevirtual #4                  // Method java/io/PrintStream.print:(Ljava/lang/String;)V
        54: return        
\end{lstlisting}

\TT{if\_icmpge} \EN{pops two values and compares them}\RU{Выталкивает два значения и сравнивает их}.
\EN{If the second one is larger than first, jump to offset 14 is occurred.}
\RU{Если второй больше первого, происходит переход на смещение 14.}
\EN{\TT{if\_icmpne} and \TT{if\_icmple} works just as the same, but different conditions are used.}
\RU{\TT{if\_icmpne} и \TT{if\_icmple} работают одинаково, но используются разные условия.}

\EN{There are also \TT{ifne} instruction at offset 43.}
\RU{По смещению 43 есть также инструкция \TT{ifne}.}
\EN{Its name is misnomer, I would name it rather \TT{ifnz} 
(jump if value at \ac{TOS} is not zero).}
\RU{Название неудачное, я бы скорее назвал её \TT{ifnz} 
(переход если переменная на \ac{TOS} не равна нулю).}
\EN{And that is what it does: it jumps to offset 54 if input value is not zero.}
\RU{И вот что она делает: производит переход на смещение 54, если входное значение не ноль.}
\EN{If zero, execution flow is proceeded to offset 46, where the ``==0'' string is printed.}
\RU{Если ноль, управление передается на смещение 46, где выводится строка ``==0''.}

\EN{N.B.: \ac{JVM} has no unsigned data types, so comparison instructions operates 
only on signed integer values.}
\RU{N.B.: В \ac{JVM} нет беззнаковых типов данных, так что инструкции сравнения работают
только с знаковыми челочисленными значениями.}
