% TODO proof-reading
\subsection{\EN{Loops}\RU{Циклы}}

\begin{lstlisting}
public class Loop
{
	public static void main(String[] args)
	{ 
		for (int i = 1; i <= 10; i++)
		{
			System.out.println(i); 
		}               
	}
}
\end{lstlisting}

\begin{lstlisting}
  public static void main(java.lang.String[]);
    flags: ACC_PUBLIC, ACC_STATIC
    Code:
      stack=2, locals=2, args_size=1
         0: iconst_1      
         1: istore_1      
         2: iload_1       
         3: bipush        10
         5: if_icmpgt     21
         8: getstatic     #2         // Field java/lang/System.out:Ljava/io/PrintStream;
        11: iload_1       
        12: invokevirtual #3         // Method java/io/PrintStream.println:(I)V
        15: iinc          1, 1
        18: goto          2
        21: return        
\end{lstlisting}

\EN{\TT{iconst\_1} loads 1 into \ac{TOS}, \TT{istore\_1} stores it in the \ac{LVA} at slot 1.}
\RU{\TT{iconst\_1} загружает 1 в \ac{TOS}, \TT{istore\_1} сохраняет её в первом слоте \ac{LVA}.}
\EN{Why not the zeroth slot}\RU{Почему не нулевой слот}?
\EN{Because the \main function has one argument (array of \TT{String}) 
and a pointer to it (or \IT{reference}) is now in the zeroth slot.}
\RU{Потому что функция \main имеет один аргумент (массив \TT{String}),
и указатель на него (или \IT{reference}) сейчас в нулевом слоте.}

\EN{So, the \IT{i} local variable will always be in 1st slot.}
\RU{Так что локальная переменная \IT{i} всегда будет в первом слоте.}

\EN{Instructions at offsets 3 and 5 compare \IT{i} with 10.}
\RU{Инструкции по смещениями 3 и 5 сравнивают \IT{i} с 10.}
\EN{If \IT{i} is larger, execution flow passes to offset 21, where the function ends.}
\RU{Если \IT{i} больше, управление передается на смещение 21, где функция заканчивает работу.}
\EN{If it's not, \TT{println} is called.}
\RU{Если нет, вызывается \TT{println}.}
\EN{\IT{i} is then reloaded at offset 11, for \TT{println}.}
\RU{\IT{i} перезагружается по смещению 11, для \TT{println}.}
\EN{By the way, we call the \TT{println} method for an \IT{integer}, 
and we see this in the comments: \q{(I)V}
(\IT{I} mean \IT{integer} and \IT{V} mean the return type is \IT{void}).}
\RU{Кстати, мы вызываем метод \TT{println} для типа данных \IT{integer}, 
и мы видим это в комментариях: \q{(I)V}
(\IT{I} означает \IT{integer} и \IT{V} означает, что возвращаемое значение имеет тип \IT{void}).}

\EN{When \TT{println} finishes, \IT{i} is incremented at offset 15.}
\RU{Когда \TT{println} заканчивается, \IT{i} увеличивается на 1 по смещению 15.}
\EN{The first operand of the instruction is the  number of a slot (1), 
the second is the number (1) to add to the variable.}
\RU{Первый операнд инструкции это номер слота (1), 
второй это число (1) для прибавления.}

\EN{\TT{goto} is just GOTO, it jumps to the beginning of the loop's body offset 2.}
\RU{\TT{goto} это просто GOTO, она переходит на начало цикла по смещению 2.}

\EN{Let's proceed with a more complex example:}
\RU{Перейдем к более сложному примеру:}

\begin{lstlisting}
public class Fibonacci
{
	public static void main(String[] args)
	{ 
		int limit = 20, f = 0, g = 1;

		for (int i = 1; i <= limit; i++)
		{
			f = f + g;
			g = f - g;
			System.out.println(f); 
		}
	}
}
\end{lstlisting}

\begin{lstlisting}
  public static void main(java.lang.String[]);
    flags: ACC_PUBLIC, ACC_STATIC
    Code:
      stack=2, locals=5, args_size=1
         0: bipush        20
         2: istore_1      
         3: iconst_0      
         4: istore_2      
         5: iconst_1      
         6: istore_3      
         7: iconst_1      
         8: istore        4
        10: iload         4
        12: iload_1       
        13: if_icmpgt     37
        16: iload_2       
        17: iload_3       
        18: iadd          
        19: istore_2      
        20: iload_2       
        21: iload_3       
        22: isub          
        23: istore_3      
        24: getstatic     #2         // Field java/lang/System.out:Ljava/io/PrintStream;
        27: iload_2       
        28: invokevirtual #3         // Method java/io/PrintStream.println:(I)V
        31: iinc          4, 1
        34: goto          10
        37: return        
\end{lstlisting}
        
\EN{Here is a map of the \ac{LVA} slots:}
\RU{Вот карта слотов в \ac{LVA}:}

\begin{itemize}
\item 0 --- \EN{the sole argument of \main}\RU{единственный аргумент функции \main}
\item 1 --- \IT{limit}, \EN{always contains}\RU{всегда содержит} 20
\item 2 --- \IT{f}
\item 3 --- \IT{g}
\item 4 --- \IT{i}
\end{itemize}

\EN{We can see that the Java compiler allocates variables in \ac{LVA} slots in the same order 
they were declared in the source code.}
\RU{Мы видим, что компилятор Java расположил переменные в слотах \ac{LVA} в точно таком же порядке,
в котором переменные были определены в исходном коде.}

\EN{There are separate \TT{istore} instructions for accessing slots 0, 1, 2 and 3, 
but not for 4 and larger, so there is \TT{istore} with an additional operand at offset 8 
which takes the slot number as an operand.}
\RU{Существуют отдельные инструкции \TT{istore} для слотов 0, 1, 2, 3, но не 4 и более, 
так что здесь есть \TT{istore} с дополнительным операндом по смещению 8, 
которая имеет номер слота в операнде.}
\EN{It's the same with \TT{iload} at offset 10.}
\RU{Та же история с \TT{iload} по смещению 10.}

\EN{But isn't it dubious to allocate another slot for the \IT{limit} variable, which 
always contains 20 (so it's a constant in essence), and reload its value so often?}
\RU{Но не слишком ли это сомнительно, выделить целый слот для переменной \IT{limit},
которая всегда содержит 20 (так что это по сути константа), и перезагружать её так часто?}
\EN{\ac{JVM} \ac{JIT} compiler is usually good enough to optimize such things.}
\RU{\ac{JIT}-компилятор в \ac{JVM} обычно достаточно хорош, чтобы всё это оптимизировать.}
\EN{Manual intervention in the code is probably not worth it.}
\RU{Самостоятельное вмешательство в код, наверное, того не стоит.}

