% TODO proof-reading
\subsectionold{\EN{Simple example}\RU{Простой пример}}

\EN{Let's first create an array of 10 integers and fill it:}
\RU{Создадим массив из 10-и чисел и заполним его:}

\begin{lstlisting}
	public static void main(String[] args) 
	{
		int a[]=new int[10];
		for (int i=0; i<10; i++)
			a[i]=i;
		dump (a);
	}
\end{lstlisting}

\begin{lstlisting}
  public static void main(java.lang.String[]);
    flags: ACC_PUBLIC, ACC_STATIC
    Code:
      stack=3, locals=3, args_size=1
         0: bipush        10
         2: newarray       int
         4: astore_1      
         5: iconst_0      
         6: istore_2      
         7: iload_2       
         8: bipush        10
        10: if_icmpge     23
        13: aload_1       
        14: iload_2       
        15: iload_2       
        16: iastore       
        17: iinc          2, 1
        20: goto          7
        23: aload_1       
        24: invokestatic  #4                  // Method dump:([I)V
        27: return        
\end{lstlisting}

\EN{The \TT{newarray} instruction creates an array object of 10 \IT{int} elements.}
\RU{Инструкция \TT{newarray} создает объект массива из 10 элементов типа \IT{int}.}
\EN{The array's size is set with \TT{bipush} and left at \ac{TOS}.}
\RU{Размер массива выставляется инструкцией \TT{bipush} и остается на \ac{TOS}.}
\EN{The array's type is set in \TT{newarray} instruction's operand.}
\RU{Тип массива выставляется в операнде инструкции \TT{newarray}.}
\EN{After \TT{newarray}'s execution, a \IT{reference} (or pointer) to the newly created array in the heap 
is left at the \ac{TOS}.}
\RU{После исполнения \TT{newarray}, \IT{reference} (или указатель) только что созданного 
в куче (heap) массива остается на \ac{TOS}.}
\EN{\TT{astore\_1} stores the \IT{reference} to the 1st slot in \ac{LVA}.}
\RU{\TT{astore\_1} сохраняет \IT{reference} на него в первом слоте \ac{LVA}.}
\EN{The second part of the \main function is the loop which stores \IT{i} into the corresponding
array element.}
\RU{Вторая часть функции \main это цикл, сохраняющий значение \IT{i} в соответствующий
элемент массива.}
\EN{\TT{aload\_1} gets a \IT{reference} of the array and places it in the stack.}
\RU{\TT{aload\_1} берет \IT{reference} массива и сохраняет его в стеке.}
\EN{\TT{iastore} then stores the integer value from the stack in the array, 
\IT{reference} of which is currently in \ac{TOS}.}
\RU{\TT{iastore} затем сохраняет значение из стека в массив, 
\IT{reference} на который в это время находится на \ac{TOS}.}
\EN{The third part of the \main function calls the \TT{dump()} function.}
\RU{Третья часть функции \main вызывает функцию \TT{dump()}.}
\EN{An argument for it is prepared by \TT{aload\_1} (offset 23).}
\RU{Аргумент для нее готовится инструкцией \TT{aload\_1} (смещение 23).}

\EN{Now let's proceed to the \TT{dump()} function:}
\RU{Перейдем к функции \TT{dump()}:}

\begin{lstlisting}
	public static void dump(int a[])
	{
		for (int i=0; i<a.length; i++)
			System.out.println(a[i]);
	}
\end{lstlisting}

\begin{lstlisting}
  public static void dump(int[]);
    flags: ACC_PUBLIC, ACC_STATIC
    Code:
      stack=3, locals=2, args_size=1
         0: iconst_0      
         1: istore_1      
         2: iload_1       
         3: aload_0       
         4: arraylength   
         5: if_icmpge     23
         8: getstatic     #2                  // Field java/lang/System.out:Ljava/io/PrintStream;
        11: aload_0       
        12: iload_1       
        13: iaload        
        14: invokevirtual #3                  // Method java/io/PrintStream.println:(I)V
        17: iinc          1, 1
        20: goto          2
        23: return        
\end{lstlisting}

\EN{The incoming \TT{reference} to the array is in the zeroth slot.}
\RU{Входящий \TT{reference} на массив в нулевом слоте.}
\EN{The \TT{a.length} expression in the source code is converted to an \TT{arraylength} instruction: 
it takes a \IT{reference} to the array and leaves the array size at \ac{TOS}.}
\RU{Выражение \TT{a.length} в исходном коде конвертируется в инструкцию \TT{arraylength},
она берет \IT{reference} на массив и оставляет размер массива на \ac{TOS}.}
\EN{\TT{iaload} at offset 13 is used to load array elements, 
it requires to array \IT{reference} be present
in the stack (prepared by \TT{aload\_0} at 11), 
and also an index (prepared by \TT{iload\_1} at offset 12).}
\RU{Инструкция \TT{iaload} по смещеню 13 используется для загрузки элементов массива, 
она требует чтобы в стеке присутствовал \IT{reference} на массив
(подготовленный \TT{aload\_0} на 11), 
а также индекс (подготовленный \TT{iload\_1} по смещеню 12).}

\EN{Needless to say, instructions prefixed with \IT{a} may be mistakenly comprehended 
as \IT{array} instructions.}
\RU{Нужно сказать что инструкции с префиксом \IT{a} могут быть неверно поняты, 
как инструкции работающие с массивами (\IT{array}).}
\EN{It's not correct}\RU{Это неверно}.
\EN{These instructions works with \IT{references} to objects.}
\RU{Эти инструкции работают с \IT{reference}-ами на объекты.}
\EN{And arrays and strings are objects too.}
\RU{А массивы и строки это тоже объекты.}
