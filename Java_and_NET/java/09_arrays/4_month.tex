% TODO proof-reading
\subsectionold{\EN{Pre-initialized array of strings}\RU{Заранее инициализированный массив строк}}
\label{Java_2D_array_month}

\begin{lstlisting}
class Month
{
	public static String[] months = 
	{
		"January", 
		"February", 
		"March", 
		"April",
		"May",
		"June",
		"July",
		"August",
		"September",
		"October",
		"November",
		"December"
	};

	public String get_month (int i)
	{
		return months[i];
	};
} 
\end{lstlisting}

\EN{The \TT{get\_month()} function is simple:}
\EN{Функция \TT{get\_month()} проста:}

\begin{lstlisting}
  public java.lang.String get_month(int);
    flags: ACC_PUBLIC
    Code:
      stack=2, locals=2, args_size=2
         0: getstatic     #2                  // Field months:[Ljava/lang/String;
         3: iload_1       
         4: aaload        
         5: areturn       
\end{lstlisting}

\EN{\TT{aaload} operates on an array of \IT{references}.}
\RU{\TT{aaload} работает с массивом \IT{reference}-ов.}
\EN{Java String are objects, so the \IT{a}-instructions are used to operate on them.}
\RU{Строка в Java это объект, так что используются \IT{a}-инструкции для работы с ними.}
\EN{\TT{areturn} returns a \IT{reference} to a \TT{String} object.}
\RU{\TT{areturn} возвращает \IT{reference} на объект \TT{String}.}

\EN{How is the \TT{months[]} array initialized?}
\RU{Как инициализируется массив \TT{months[]}?}

\begin{lstlisting}
  static {};
    flags: ACC_STATIC
    Code:
      stack=4, locals=0, args_size=0
         0: bipush        12
         2: anewarray     #3                  // class java/lang/String
         5: dup           
         6: iconst_0      
         7: ldc           #4                  // String January
         9: aastore       
        10: dup           
        11: iconst_1      
        12: ldc           #5                  // String February
        14: aastore       
        15: dup           
        16: iconst_2      
        17: ldc           #6                  // String March
        19: aastore       
        20: dup           
        21: iconst_3      
        22: ldc           #7                  // String April
        24: aastore       
        25: dup           
        26: iconst_4      
        27: ldc           #8                  // String May
        29: aastore       
        30: dup           
        31: iconst_5      
        32: ldc           #9                  // String June
        34: aastore       
        35: dup           
        36: bipush        6
        38: ldc           #10                 // String July
        40: aastore       
        41: dup           
        42: bipush        7
        44: ldc           #11                 // String August
        46: aastore       
        47: dup           
        48: bipush        8
        50: ldc           #12                 // String September
        52: aastore       
        53: dup           
        54: bipush        9
        56: ldc           #13                 // String October
        58: aastore       
        59: dup           
        60: bipush        10
        62: ldc           #14                 // String November
        64: aastore       
        65: dup           
        66: bipush        11
        68: ldc           #15                 // String December
        70: aastore       
        71: putstatic     #2                  // Field months:[Ljava/lang/String;
        74: return        
\end{lstlisting}

\EN{\TT{anewarray} creates a new array of \IT{references} (hence \IT{a} prefix).}
\RU{\TT{anewarray} создает новый массив \IT{reference}-ов (отсюда префикс \IT{a}).}
\EN{The object's type is defined in the \TT{anewarray}'s operand, it is the \q{java/lang/String} string.}
\RU{Тип объекта определяется в операнде \TT{anewarray}, там текстовая строка \q{java/lang/String}.}
\EN{The \TT{bipush 12} before \TT{anewarray} sets the array's size.}
\RU{\TT{bipush 12} перед \TT{anewarray} устанавливает размер массива.}
\EN{We see here a new instruction for us: \TT{dup}.}
\RU{Новая для нас здесь инструкция: \TT{dup}.}

\myindex{Forth}
\EN{It's a standard instruction in stack computers (including the Forth programming language) 
which just duplicates the value at \ac{TOS}.}
\RU{Это стандартная инструкция в стековых компьютерах (включая ЯП Forth),
которая делает дубликат значения на \ac{TOS}.}
\myindex{x86!\Instructions!FDUP}
\EN{By the way, FPU 80x87 is also a stack computer and it has similar instruction -- \INS{FDUP}.}
\RU{Кстати, FPU 80x87 это тоже стековый компьютер, и в нем есть аналогичная инструкция -- \INS{FDUP}.}

\EN{It is used here to duplicate a \IT{reference} to an array, because the \TT{aastore} instruction pops
the \IT{reference} to array from the stack, but subsequent \TT{aastore} will need it again.}
\RU{Она используется здесь для дублирования \IT{reference}-а на массив, 
потому что инструкция \TT{aastore} выталкивает из стека \IT{reference} на массив, 
но последующая инструкция \TT{aastore} снова нуждается в нем.}
\EN{The Java compiler concluded that it's better to generate a \TT{dup} instead of generating 
a \TT{getstatic} instruction before each array store operation (i.e., 11 times).}
\RU{Компилятор Java решил что лучше генерировать \TT{dup} вместо генерации инструкции
\TT{getstatic} перед каждой операцией записи в массив (т.е. 11 раз).}

\EN{\TT{aastore} puts a \IT{reference} (to string) into the array at an index which is 
taken from \ac{TOS}.}
\RU{\TT{aastore} кладет \IT{reference} (на строку) в массив по индексу взятому из \ac{TOS}.}

\EN{Finally, \TT{putstatic} puts \IT{reference} to the newly created array into the second field 
of our object, i.e., \IT{months} field.}
\RU{И наконец, \TT{putstatic} кладет \IT{reference} на только что созданный массив во второе поле
нашего объекта, т.е. в поле \IT{months}.}
