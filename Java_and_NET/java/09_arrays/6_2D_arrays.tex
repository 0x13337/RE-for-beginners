% TODO proof-reading
\subsection{\EN{Two-dimensional arrays}\RU{Двухмерные массивы}}

\EN{Two-dimensional arrays in Java are just one-dimensional arrays of \IT{references} to another 
one-dimensional arrays.}
\RU{Двухмерные массивы в Java это просто одномерные массивы \IT{reference}-в на другие одномерные 
массивы.}

\EN{Let's create two-dimensional array}\RU{Создадим двухмерный массив}:

\begin{lstlisting}
	public static void main(String[] args)
	{
		int[][] a = new int[5][10];
		a[1][2]=3;
	}
\end{lstlisting}

\begin{lstlisting}
  public static void main(java.lang.String[]);
    flags: ACC_PUBLIC, ACC_STATIC
    Code:
      stack=3, locals=2, args_size=1
         0: iconst_5      
         1: bipush        10
         3: multianewarray #2,  2             // class "[[I"
         7: astore_1      
         8: aload_1       
         9: iconst_1      
        10: aaload        
        11: iconst_2      
        12: iconst_3      
        13: iastore       
        14: return        
\end{lstlisting}

\EN{It's created using \TT{multianewarray} instruction: object type and dimensionality are passed
as operands.}
\RU{Он создается при помощи инструкции \TT{multianewarray}: тип объекта и размерность передаются
в операндах.}
\EN{Array size (10*5) is left in stack (using instructions \TT{iconst\_5} and \TT{bipush}).}
\RU{Размер массива (10*5) остается в стеке (используя инструкции \TT{iconst\_5} и \TT{bipush}).}

\EN{\IT{Reference} to row \#1 is loaded at offset 10 (\TT{iconst\_1} and \TT{aaload}).}
\RU{\IT{Reference} на строку \#1 загружается по смещению 10 (\TT{iconst\_1} и \TT{aaload}).}
\EN{Column is choosen using \TT{iconst\_2} instruction at offset 11.}
\RU{Выборка столбца происходит используя инструкцию \TT{iconst\_2} по смещению 11.}
\EN{Value to be written is set at offset 12.}
\RU{Значение для записи устанавливается по смещению 12.}
\EN{\TT{iastore} at 13 writes array element there.}
\RU{\TT{iastore} на 13 записывает элемент массива.}

\EN{How it is accessed}\RU{Как его прочитать}?

\begin{lstlisting}
	public static int get12 (int[][] in)
	{
		return in[1][2];
	}
\end{lstlisting}

\begin{lstlisting}
  public static int get12(int[][]);
    flags: ACC_PUBLIC, ACC_STATIC
    Code:
      stack=2, locals=1, args_size=1
         0: aload_0       
         1: iconst_1      
         2: aaload        
         3: iconst_2      
         4: iaload        
         5: ireturn       
\end{lstlisting}

\EN{\IT{Reference} to array row is loaded at offset 2, column is set at offset 3, 
\TT{iaload} loads array element.}
\RU{\IT{Reference} на строку массива загружается по смещению 2, 
столбец устанавливается по смещению 3, \TT{iaload} загружает элемент массива.}
