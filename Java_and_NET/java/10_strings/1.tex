% TODO proof-reading
\subsectionold{\EN{First example}\RU{Первый пример}}

\EN{Strings are objects and are constructed in the same way as other objects (and arrays).}
\RU{Строки это объекты, и конструируются так же как и другие объекты (и массивы).}

\begin{lstlisting}
	public static void main(String[] args)
	{
		System.out.println("What is your name?");
		String input = System.console().readLine();
		System.out.println("Hello, "+input);
	}
\end{lstlisting}

\begin{lstlisting}
  public static void main(java.lang.String[]);
    flags: ACC_PUBLIC, ACC_STATIC
    Code:
      stack=3, locals=2, args_size=1
         0: getstatic     #2                  // Field java/lang/System.out:Ljava/io/PrintStream;
         3: ldc           #3                  // String What is your name?
         5: invokevirtual #4                  // Method java/io/PrintStream.println:(Ljava/lang/String;)V
         8: invokestatic  #5                  // Method java/lang/System.console:()Ljava/io/Console;
        11: invokevirtual #6                  // Method java/io/Console.readLine:()Ljava/lang/String;
        14: astore_1      
        15: getstatic     #2                  // Field java/lang/System.out:Ljava/io/PrintStream;
        18: new           #7                  // class java/lang/StringBuilder
        21: dup           
        22: invokespecial #8                  // Method java/lang/StringBuilder."<init>":()V
        25: ldc           #9                  // String Hello, 
        27: invokevirtual #10                 // Method java/lang/StringBuilder.append:(Ljava/lang/String;)Ljava/lang/StringBuilder;
        30: aload_1       
        31: invokevirtual #10                 // Method java/lang/StringBuilder.append:(Ljava/lang/String;)Ljava/lang/StringBuilder;
        34: invokevirtual #11                 // Method java/lang/StringBuilder.toString:()Ljava/lang/String;
        37: invokevirtual #4                  // Method java/io/PrintStream.println:(Ljava/lang/String;)V
        40: return        
\end{lstlisting}

\EN{The \TT{readLine()} method is called at offset 11, a \IT{reference} to string (which is supplied by the user) 
is then stored at \ac{TOS}.}
\RU{Метод \TT{readLine()} вызывается по смещению 11, \IT{reference} на строку (введенную пользователем) 
остается на \ac{TOS}.}
\EN{At offset 14 the \IT{reference} to string is stored in slot 1 of \ac{LVA}.}
\RU{По смещению 14, \IT{reference} на строку сохраняется в первом слоте \ac{LVA}.}
\EN{The string the user entered is reloaded at offset 30 and concatenated with the \q{Hello, } string
using the \TT{StringBuilder} class.}
\RU{Строка введенная пользователем перезагружается по смещению 30 и складывается со строкой \q{Hello, }
используя класс \TT{StringBuilder}.}
\EN{The constructed string is then printed using \TT{println} at offset 37.}
\RU{Сконструированная строка затем выводится используя метод \TT{println} по смещению 37.}
