% TODO proof-reading
\section{\EN{Exceptions}\RU{Исключения}}

\EN{Let's rework our \IT{Month} example (\myref{Java_2D_array_month}) a bit:}
\RU{Немного переделаем пример \IT{Month} (\myref{Java_2D_array_month}):}

\begin{lstlisting}[caption=IncorrectMonthException.java]
public class IncorrectMonthException extends Exception
{
	private int index;

	public IncorrectMonthException(int index)
	{
		this.index = index;
	} 
	public int getIndex()
	{
		return index;
	}
}
\end{lstlisting}

\begin{lstlisting}[caption=Month2.java]
class Month2
{
	public static String[] months = 
	{
		"January", 
		"February", 
		"March", 
		"April",
		"May",
		"June",
		"July",
		"August",
		"September",
		"October",
		"November",
		"December"
	};

	public static String get_month (int i) throws IncorrectMonthException
	{
		if (i<0 || i>11)
			throw new IncorrectMonthException(i);
		return months[i];
	};

	public static void main (String[] args)
	{
		try
		{
			System.out.println(get_month(100));
		}
		catch(IncorrectMonthException e)
		{
			System.out.println("incorrect month index: "+ e.getIndex());
			e.printStackTrace();
		}
	};
}
\end{lstlisting}

\EN{Essentially, \TT{IncorrectMonthException.class} has just an object constructor 
and one getter method.}
\RU{Которко говоря, \TT{IncorrectMonthException.class} имеет только конструктор объетка и один
метод-акцессор.}

\EN{The \TT{IncorrectMonthException} class is derived from \TT{Exception}, 
so the \TT{IncorrectMonthException} constructor
first calls the constructor of the \TT{Exception} class, 
then it puts incoming integer value into the sole \TT{IncorrectMonthException} class field:}
\RU{Класс \TT{IncorrectMonthException} наследуется от \TT{Exception}, 
так что конструктор \TT{IncorrectMonthException}
в начале вызывает конструктор класса \TT{Exception}, 
затем он перекладывает входящее значение в единственное поле класса \TT{IncorrectMonthException}:}

\begin{lstlisting}
  public IncorrectMonthException(int);
    flags: ACC_PUBLIC
    Code:
      stack=2, locals=2, args_size=2
         0: aload_0       
         1: invokespecial #1                  // Method java/lang/Exception."<init>":()V
         4: aload_0       
         5: iload_1       
         6: putfield      #2                  // Field index:I
         9: return        
\end{lstlisting}

\TT{getIndex()} \EN{is just a getter}\RU{это просто акцессор}.
\EN{A \IT{reference} to \TT{IncorrectMonthException} is passed in the zeroth \ac{LVA} slot
(\IT{this}), \TT{aload\_0} takes it, \TT{getfield} loads an integer value from the object, 
\TT{ireturn} returns it.}
\RU{\IT{Reference} (указатель) на \TT{IncorrectMonthException} передается в нулевом слоте \ac{LVA}
(\IT{this}), \TT{aload\_0} берет его, \TT{getfield} загружает значение из объекта, 
\TT{ireturn} возвращает его.}

\begin{lstlisting}
  public int getIndex();
    flags: ACC_PUBLIC
    Code:
      stack=1, locals=1, args_size=1
         0: aload_0       
         1: getfield      #2                  // Field index:I
         4: ireturn       
\end{lstlisting}

\EN{Now let's take a look at \TT{get\_month()} in \TT{Month2.class}:}
\RU{Посмотрим на \TT{get\_month()} в \TT{Month2.class}:}

\begin{lstlisting}[caption=Month2.class]
  public static java.lang.String get_month(int) throws IncorrectMonthException;
    flags: ACC_PUBLIC, ACC_STATIC
    Code:
      stack=3, locals=1, args_size=1
         0: iload_0       
         1: iflt          10
         4: iload_0       
         5: bipush        11
         7: if_icmple     19
        10: new           #2                  // class IncorrectMonthException
        13: dup           
        14: iload_0       
        15: invokespecial #3                  // Method IncorrectMonthException."<init>":(I)V
        18: athrow        
        19: getstatic     #4                  // Field months:[Ljava/lang/String;
        22: iload_0       
        23: aaload        
        24: areturn       
\end{lstlisting}

\EN{\TT{iflt} at offset 1 is \IT{if less than}.}
\RU{\TT{iflt} по смещению 1 это \IT{if less than} (если меньше, чем).}

\EN{In case of invalid index, a new object is created using the \TT{new} instruction at offset 10.}
\RU{В случае неправильного индекса, создается новый объект при помощи инструкции \TT{new} 
по смещению 10.}
\EN{The object's type is passed as an operand to the instruction (which is \TT{IncorrectMonthException}).}
\RU{Тип объекта передается как операнд инструкции (и это \TT{IncorrectMonthException}).}
\EN{Then its constructor is called, and index is passed via \ac{TOS} (offset 15).}
\RU{Затем вызывается его конструктор, в который передается индекс (через \ac{TOS}) (по смещению 15).}
\EN{When the control flow is offset 18, the object is already constructed, 
so now the \TT{athrow} instruction takes a \IT{reference} 
to the newly constructed object and signals to \ac{JVM} to find the appropriate exception handler.}
\RU{В то время как управление находится на смещении 18, объект уже создан,
теперь инструкция \TT{athrow} берет указатель (\IT{reference})
на только что созданный объект и сигнализирует в \ac{JVM}, чтобы тот нашел подходящий обработчик
исключения.}

\EN{The \TT{athrow} instruction doesn't return the control flow here, 
so at offset 19 there is another \gls{basic block},
not related to exceptions business, where we can get from offset 7.}
\RU{Инструкция \TT{athrow} не возвращает управление сюда,
так что по смещению 19 здесь совсем другой \gls{basic block},
не имеющий отношения к исключениям, сюда можно попасть со смещения 7.}

\EN{How do handlers work}\RU{Как работает обработчик}?
\RU{Take a look at}\RU{Посмотрим на} \main \InENRU \TT{Month2.class}:

\begin{lstlisting}[caption=Month2.class]
  public static void main(java.lang.String[]);
    flags: ACC_PUBLIC, ACC_STATIC
    Code:
      stack=3, locals=2, args_size=1
         0: getstatic     #5                  // Field java/lang/System.out:Ljava/io/PrintStream;
         3: bipush        100
         5: invokestatic  #6                  // Method get_month:(I)Ljava/lang/String;
         8: invokevirtual #7                  // Method java/io/PrintStream.println:(Ljava/lang/String;)V
        11: goto          47
        14: astore_1      
        15: getstatic     #5                  // Field java/lang/System.out:Ljava/io/PrintStream;
        18: new           #8                  // class java/lang/StringBuilder
        21: dup           
        22: invokespecial #9                  // Method java/lang/StringBuilder."<init>":()V
        25: ldc           #10                 // String incorrect month index: 
        27: invokevirtual #11                 // Method java/lang/StringBuilder.append:(Ljava/lang/String;)Ljava/lang/StringBuilder;
        30: aload_1       
        31: invokevirtual #12                 // Method IncorrectMonthException.getIndex:()I
        34: invokevirtual #13                 // Method java/lang/StringBuilder.append:(I)Ljava/lang/StringBuilder;
        37: invokevirtual #14                 // Method java/lang/StringBuilder.toString:()Ljava/lang/String;
        40: invokevirtual #7                  // Method java/io/PrintStream.println:(Ljava/lang/String;)V
        43: aload_1       
        44: invokevirtual #15                 // Method IncorrectMonthException.printStackTrace:()V
        47: return        
      Exception table:
         from    to  target type
             0    11    14   Class IncorrectMonthException
\end{lstlisting}

\EN{Here is the \TT{Exception table}, which defines that from offsets 0 to 11 (inclusive) an exception 
\TT{IncorrectMonthException} may happen, and if it does, the control flow is to be passed to offset 14.}
\RU{Тут есть \TT{Exception table}, которая определяет, что между смещениями 0 и 11 (включительно)
может случиться исключение \TT{IncorrectMonthException}, и если это произойдет, то нужно передать
управление на смещение 14.}
\EN{Indeed, the main program ends at offset 11.}
\RU{Действительно, основная программа заканчивается на смещении 11.}
\EN{At offset 14 the handler starts. It's not possible to get here, 
there are no conditional/unconditional jumps to this area.}
\RU{По смещению 14 начинается обработчик, и сюда невозможно попасть, 
здесь нет никаких условных/безусловных переходов в эту область.}
\EN{But \ac{JVM} will transfer the execution flow here in case of an exception.}
\RU{Но \ac{JVM} передаст сюда управление в случае исключения.}
\EN{The very first \TT{astore\_1} (at 14) takes the incoming \IT{reference} to the exception object 
and stores it in \ac{LVA} slot 1.}
\RU{Самая первая \TT{astore\_1} (на 14) берет входящий указатель (\IT{reference}) на объект 
исключения и сохраняет его в слоте 1 \ac{LVA}.}
\EN{Later, the \TT{getIndex()} method (of this exception object) will be called at offset 31.}
\RU{Позже, по смещению 31 будет вызван метод этого объекта (\TT{getIndex()}).}
\EN{The \IT{reference} to the current exception object is passed right before that (offset 30).}
\RU{Указатель \IT{reference} на текующий объект исключения передался немного раньше (смещение 30).}
\EN{The rest of the code is does just string manipulation: 
first the integer value returned by \TT{getIndex()}
is converted to string by the \TT{toString()} method, then it's concatenated with 
the \q{incorrect month index: } text string (like we saw before),
then \TT{println()} and \TT{printStackTrace()} are called.}
\RU{Остальной код это просто код для манипуляции со строками: 
в начале значение возвращенное методом \TT{getIndex()}
конвертируется в строку используя метод \TT{toString()}, 
затем эта строка прибавляется к текстовой строке
\q{incorrect month index: } (как мы уже рассматривали ранее),
затем вызываются \TT{println()} и \TT{printStackTrace()}.}
\EN{After \TT{printStackTrace()} finishes, the exception is handled and we can continue with the normal execution.}
\RU{После того как \TT{printStackTrace()} заканчивается, исключение уже обработано, мы можем
возвращаться к нормальной работе.}
\EN{At offset 47 there is a \TT{return} which finishes the \main function, 
but there could be any other code which would execute as if no exceptions were raised.}
\RU{По смещению 47 есть \TT{return}, который заканчивает работу функции \main, 
но там может быть любой другой код, который исполнится, если исключения не произошло.}

\EN{Here is an example on how IDA shows exception ranges:}
\RU{Вот пример, как IDA показывает интервалы исключений:}

\begin{lstlisting}[caption=\EN{from some random .class file found on the author's computer}\RU{из какого-то случайного найденного на компьютере автора .class-файла}]
    .catch java/io/FileNotFoundException from met001_335 to met001_360\
 using met001_360
    .catch java/io/FileNotFoundException from met001_185 to met001_214\
 using met001_214
    .catch java/io/FileNotFoundException from met001_181 to met001_192\
 using met001_195
    .catch java/io/FileNotFoundException from met001_155 to met001_176\
 using met001_176
    .catch java/io/FileNotFoundException from met001_83 to met001_129 using \
met001_129
    .catch java/io/FileNotFoundException from met001_42 to met001_66 using \
met001_69
    .catch java/io/FileNotFoundException from met001_begin to met001_37\
 using met001_37
\end{lstlisting}
