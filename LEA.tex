\section{\IFRU{Инструкция LEA}{LEA instruction}}
\label{sec:LEA}
\index{x86!\Instructions!LEA}

\newcommand{\URLAM}{\url{http://en.wikipedia.org/wiki/Addressing_mode}}

\IFRU
{\LEA (\IT{Load Effective Address}) это инструкция которая задумывалась вовсе не для складывания чисел, 
а для формирования адреса например из указателя на массив и прибавления индекса к нему\footnote{См. также: \URLAM}.}
{\LEA (\IT{Load Effective Address}) is instruction intended not for values summing but for address forming, 
e.g., for forming address of array element by adding array address, element index, with 
multiplication of element size\footnote{See also: \URLAM}.}

\IFRU{Важная особенность \LEA в том что производимые ею вычисления не модифицируют флаги.}
{Important property of a \LEA instruction is that it does not alter processor flags.}

% TODO: дописать про x*n+m

\begin{lstlisting}
int f(int a, int b)
{
	return a*8+b;
};
\end{lstlisting}

\IFRU{Компилируем в MSVC 2010 с \Ox:}{MSVC 2010 with \Ox option:}

\begin{lstlisting}
_a$ = 8							; size = 4
_b$ = 12						; size = 4
_f	PROC
	mov	eax, DWORD PTR _b$[esp-4]
	mov	ecx, DWORD PTR _a$[esp-4]
	lea	eax, DWORD PTR [eax+ecx*8]
	ret	0
_f	ENDP
\end{lstlisting}
