\section{Windows NT: \IFRU{Критические секции}{Critical section}}

\label{critical_sections}

\IFRU{Критические секции в любой \ac{OS} очень важны в мультитредовой среде, используются в основном
для обеспечения гарантии что только один тред будет иметь доступ к данным,
блокируя остальные треды и прерывания}
{Critical sections in any \ac{OS} are very important in multithreaded environment,
mostly used for issuing a guarantee
that only one thread will access some data, while blocking other threads and interrupts}. \\
\\
\IFRU{Вот как объявлена структура}{That is how} \TT{CRITICAL\_SECTION} 
\IFRU{объявлена в линейке OS}{structure is declared in} \gls{Windows NT}\IFRU{}{ line OS}:

\begin{lstlisting}[caption=(Windows Research Kernel v1.2) public/sdk/inc/nturtl.h]
typedef struct _RTL_CRITICAL_SECTION {
    PRTL_CRITICAL_SECTION_DEBUG DebugInfo;

    //
    //  The following three fields control entering and exiting the critical
    //  section for the resource
    //

    LONG LockCount;
    LONG RecursionCount;
    HANDLE OwningThread;        // from the thread's ClientId->UniqueThread
    HANDLE LockSemaphore;
    ULONG_PTR SpinCount;        // force size on 64-bit systems when packed
} RTL_CRITICAL_SECTION, *PRTL_CRITICAL_SECTION;
\end{lstlisting}

\IFRU{Вот как работает ф-ция}{That's is how} EnterCriticalSection()\IFRU{}{ function works}:

\index{x86!\Instructions!LOCK}
\begin{lstlisting}[caption=Windows 2008/ntdll.dll/x86 (begin)]
_RtlEnterCriticalSection@4

var_C           = dword ptr -0Ch
var_8           = dword ptr -8
var_4           = dword ptr -4
arg_0           = dword ptr  8

                mov     edi, edi
                push    ebp
                mov     ebp, esp
                sub     esp, 0Ch
                push    esi
                push    edi
                mov     edi, [ebp+arg_0]
                lea     esi, [edi+4] ; LockCount
                mov     eax, esi
                lock btr dword ptr [eax], 0
                jnb     wait ; jump if CF=0

loc_7DE922DD:
                mov     eax, large fs:18h
                mov     ecx, [eax+24h]
                mov     [edi+0Ch], ecx
                mov     dword ptr [edi+8], 1
                pop     edi
                xor     eax, eax
                pop     esi
                mov     esp, ebp
                pop     ebp
                retn    4

... skipped
\end{lstlisting}

\index{x86!\Instructions!BTR}
\IFRU{Самая важная инструкция в этом фрагменте кода это}
{The most important instruction in this code fragment is} \TT{BTR} 
(\IFRU{с префиксом}{prefixed with} \TT{LOCK}): 
\IFRU{нулевой бит сохраняется в флаге CF и очищается в памяти}
{the zeroth bit is stored in CF flag and cleared in memory}.
\IFRU{Это \glslink{atomic operation}{атомарная операция}}{This is \gls{atomic operation}}, 
\IFRU{блокирующая доступ всех остальных процессоров
к этому значению в памяти (обратите внимание на префикс \TT{LOCK} перед инструкцией \TT{BTR}.}
{blocking all other CPUs to access this piece of memory 
(take a notice of \TT{LOCK} prefix before \TT{BTR} instruction).}
\IFRU{Если бит в}{If the bit at} \TT{LockCount} \IFRU{был}{was} 1, 
\IFRU{хорошо, сбросить его и вернуться из ф-ции: мы в критической секции}
{fine, reset it and return from the function: we are in critical section}.
\IFRU{Если нет ~--- критическая секция уже занята другим тредом, тогда ждем}
{If not~---critical section is already occupied by other thread, then wait}. \\
\IFRU{Ожидание там сделано через вызов}{Wait is done there using} WaitForSingleObject(). \\
\\
\IFRU{А вот как работает ф-ция}{And here is how} LeaveCriticalSection()\IFRU{}{ function works}:

\begin{lstlisting}[caption=Windows 2008/ntdll.dll/x86 (begin)]
_RtlLeaveCriticalSection@4 proc near

arg_0           = dword ptr  8

                mov     edi, edi
                push    ebp
                mov     ebp, esp
                push    esi
                mov     esi, [ebp+arg_0]
                add     dword ptr [esi+8], 0FFFFFFFFh ; RecursionCount
                jnz     short loc_7DE922B2
                push    ebx
                push    edi
                lea     edi, [esi+4]    ; LockCount
                mov     dword ptr [esi+0Ch], 0
                mov     ebx, 1
                mov     eax, edi
                lock xadd [eax], ebx
                inc     ebx
                cmp     ebx, 0FFFFFFFFh
                jnz     loc_7DEA8EB7

loc_7DE922B0:
                pop     edi
                pop     ebx

loc_7DE922B2:
                xor     eax, eax
                pop     esi
                pop     ebp
                retn    4

... skipped
\end{lstlisting}

\index{x86!\Instructions!XADD}
\TT{XADD} \IFRU{это ``обменять и прибавить''}{is ``exchange and add''}. 
\IFRU{В данном случае, это значит прибавить 1 к значению в \TT{LockCount}, сохранить результат
в регистре \TT{EBX}, и в то же время 1 записывается в \TT{LockCount}}
{In this case, it summing \TT{LockCount} value and 1 and stores result in \TT{EBX} register, 
and at the same time 1 goes to \TT{LockCount}}.
\IFRU{Эта операция также атомарная, потому что также имеет префикс \TT{LOCK}, что означает что другие CPU
или ядра CPU в системе не будут иметь доступа к этой ячейке памяти}
{This operation is atomic since it is prefixed by \TT{LOCK} as well,
meaning that all other CPUs or CPU cores in system are blocked from accessing this point of memory}.

\IFRU{Префикс }{}\TT{LOCK} \IFRU{очень важен}{prefix is very important}: 
\IFRU{два треда, каждый из которых работает на разных CPU или ядрах CPU, могут попытаться одновременно
войти в критическую секцию, одновременно модифицируя значение в памяти, и это может привести к
непредсказуемым результатам}
{two threads, each of which working on separate CPUs or CPU cores may try to
enter critical section and to modify the value in memory simultaneously,
this will result in unpredictable behaviour}.

% TODO linux

