\section{Windows NT: \RU{Критические секции}\EN{Critical section}}
\index{Windows}

\label{critical_sections}

\RU{Критические секции в любой \ac{OS} очень важны в мультитредовой среде, используются в основном
для обеспечения гарантии что только один тред будет иметь доступ к данным,
блокируя остальные треды и прерывания}
\EN{Critical sections in any \ac{OS} are very important in multithreaded environment,
mostly for giving a guarantee
that only one thread will access some data, while blocking other threads and interrupts}. \\
\\
\RU{Вот как объявлена структура}\EN{That is how a} \TT{CRITICAL\_SECTION} 
\RU{объявлена в линейке OS}\EN{structure is declared in} \gls{Windows NT}\EN{ line OS}:

\begin{lstlisting}[caption=(Windows Research Kernel v1.2) public/sdk/inc/nturtl.h]
typedef struct _RTL_CRITICAL_SECTION {
    PRTL_CRITICAL_SECTION_DEBUG DebugInfo;

    //
    //  The following three fields control entering and exiting the critical
    //  section for the resource
    //

    LONG LockCount;
    LONG RecursionCount;
    HANDLE OwningThread;        // from the thread's ClientId->UniqueThread
    HANDLE LockSemaphore;
    ULONG_PTR SpinCount;        // force size on 64-bit systems when packed
} RTL_CRITICAL_SECTION, *PRTL_CRITICAL_SECTION;
\end{lstlisting}

\RU{Вот как работает ф-ция}\EN{That's is how} EnterCriticalSection()\EN{ function works}:

\index{x86!\Instructions!LOCK}
\begin{lstlisting}[caption=Windows 2008/ntdll.dll/x86 (begin)]
_RtlEnterCriticalSection@4

var_C           = dword ptr -0Ch
var_8           = dword ptr -8
var_4           = dword ptr -4
arg_0           = dword ptr  8

                mov     edi, edi
                push    ebp
                mov     ebp, esp
                sub     esp, 0Ch
                push    esi
                push    edi
                mov     edi, [ebp+arg_0]
                lea     esi, [edi+4] ; LockCount
                mov     eax, esi
                lock btr dword ptr [eax], 0
                jnb     wait ; jump if CF=0

loc_7DE922DD:
                mov     eax, large fs:18h
                mov     ecx, [eax+24h]
                mov     [edi+0Ch], ecx
                mov     dword ptr [edi+8], 1
                pop     edi
                xor     eax, eax
                pop     esi
                mov     esp, ebp
                pop     ebp
                retn    4

... skipped
\end{lstlisting}

\index{x86!\Instructions!BTR}
\index{x86!\Prefixes!LOCK}
\RU{Самая важная инструкция в этом фрагменте кода --- это}
\EN{The most important instruction in this code fragment is} \TT{BTR} 
(\RU{с префиксом}\EN{prefixed with} \TT{LOCK}): 
\RU{нулевой бит сохраняется в флаге CF и очищается в памяти}
\EN{the zeroth bit is stored in the CF flag and cleared in memory}.
\RU{Это \glslink{atomic operation}{атомарная операция}}\EN{This is an \gls{atomic operation}}, 
\RU{блокирующая доступ всех остальных процессоров
к этому значению в памяти (обратите внимание на префикс \TT{LOCK} перед инструкцией \TT{BTR}.}
\EN{blocking all other CPUs' access to this piece of memory 
(see the \TT{LOCK} prefix before the \TT{BTR} instruction).}
\RU{Если бит в}\EN{If the bit at} \TT{LockCount} \RU{является}\EN{is} 1, 
\RU{хорошо, сбросить его и вернуться из ф-ции: мы в критической секции}
\EN{fine, reset it and return from the function: we are in a critical section}.
\RU{Если нет ~--- критическая секция уже занята другим тредом, тогда ждем}
\EN{If not~---the critical section is already occupied by other thread, so wait}. \\
\RU{Ожидание там сделано через вызов}\EN{The wait is done there using} WaitForSingleObject(). \\
\\
\RU{А вот как работает ф-ция}\EN{And here is how the} LeaveCriticalSection()\EN{ function works}:

\begin{lstlisting}[caption=Windows 2008/ntdll.dll/x86 (begin)]
_RtlLeaveCriticalSection@4 proc near

arg_0           = dword ptr  8

                mov     edi, edi
                push    ebp
                mov     ebp, esp
                push    esi
                mov     esi, [ebp+arg_0]
                add     dword ptr [esi+8], 0FFFFFFFFh ; RecursionCount
                jnz     short loc_7DE922B2
                push    ebx
                push    edi
                lea     edi, [esi+4]    ; LockCount
                mov     dword ptr [esi+0Ch], 0
                mov     ebx, 1
                mov     eax, edi
                lock xadd [eax], ebx
                inc     ebx
                cmp     ebx, 0FFFFFFFFh
                jnz     loc_7DEA8EB7

loc_7DE922B0:
                pop     edi
                pop     ebx

loc_7DE922B2:
                xor     eax, eax
                pop     esi
                pop     ebp
                retn    4

... skipped
\end{lstlisting}

\index{x86!\Instructions!XADD}
\TT{XADD} \RU{это ``обменять и прибавить''}\EN{is ``exchange and add''}. 
\RU{В данном случае, это значит прибавить 1 к значению в \TT{LockCount}, сохранить результат
в регистре \TT{EBX}, и в то же время 1 записывается в \TT{LockCount}}
\EN{In this case, it adds $1$ to \TT{LockCount} and stores the result in the \TT{EBX} register, 
and at the same time 1 goes to \TT{LockCount}}.
\RU{Эта операция также атомарная, потому что также имеет префикс \TT{LOCK}, что означает, что другие CPU
или ядра CPU в системе не будут иметь доступа к этой ячейке памяти}
\EN{This operation is atomic since it is prefixed by \TT{LOCK} as well,
meaning that all other CPUs or CPU cores in system are blocked from accessing this point in memory}.

\EN{The}\RU{Префикс} \TT{LOCK} \RU{очень важен}\EN{prefix is very important}: 
\RU{два треда, каждый из которых работает на разных CPU или ядрах CPU, могут попытаться одновременно
войти в критическую секцию, одновременно модифицируя значение в памяти, и это может привести к
непредсказуемым результатам}
\EN{without it two threads, each of which works on separate CPU or CPU core can try to
enter a critical section and to modify the value in memory,
which will result in non-deterministic behaviour}.

% TODO linux

