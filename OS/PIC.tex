\section{\CapitalPICcode}
\index{\PICcode}
\label{sec:PIC}

\RU{Во время анализа динамических библиотек (.so) в Linux, часто можно заметить такой шаблонный код}\EN{While analyzing Linux shared (.so) libraries, one may frequently spot such code pattern}:

\begin{lstlisting}[caption=libc-2.17.so x86]
.text:0012D5E3 __x86_get_pc_thunk_bx proc near         ; CODE XREF: sub_17350+3
.text:0012D5E3                                         ; sub_173CC+4 ...
.text:0012D5E3                 mov     ebx, [esp+0]
.text:0012D5E6                 retn
.text:0012D5E6 __x86_get_pc_thunk_bx endp

...

.text:000576C0 sub_576C0       proc near               ; CODE XREF: tmpfile+73

...

.text:000576C0                 push    ebp
.text:000576C1                 mov     ecx, large gs:0
.text:000576C8                 push    edi
.text:000576C9                 push    esi
.text:000576CA                 push    ebx
.text:000576CB                 call    __x86_get_pc_thunk_bx
.text:000576D0                 add     ebx, 157930h
.text:000576D6                 sub     esp, 9Ch

...

.text:000579F0                 lea     eax, (a__gen_tempname - 1AF000h)[ebx] ; "__gen_tempname"
.text:000579F6                 mov     [esp+0ACh+var_A0], eax
.text:000579FA                 lea     eax, (a__SysdepsPosix - 1AF000h)[ebx] ; "../sysdeps/posix/tempname.c"
.text:00057A00                 mov     [esp+0ACh+var_A8], eax
.text:00057A04                 lea     eax, (aInvalidKindIn_ - 1AF000h)[ebx] ; "! \"invalid KIND in __gen_tempname\""
.text:00057A0A                 mov     [esp+0ACh+var_A4], 14Ah
.text:00057A12                 mov     [esp+0ACh+var_AC], eax
.text:00057A15                 call    __assert_fail
\end{lstlisting}

\RU{Все указатели на строки корректируются при помощи некоторой константы из регистра \EBX, которая вычисляется в начале каждой функции.}
\EN{All pointers to strings are corrected by a constant and by value in the \EBX,
which calculated at the beginning of each function.}
\RU{Это так называемый адресно-независимый код (\ac{PIC}), он предназначен для исполнения будучи расположенным по любому адресу в памяти, вот почему он не содержит никаких абсолютных адресов в памяти}
\EN{This is so-called \ac{PIC}, it is intended to execute placed at any random point of memory, that is why it cannot contain any absolute memory addresses}.

\RU{\ac{PIC} был очень важен в ранних компьютерных системах и важен сейчас во встраиваемых\footnote{embedded}, не имеющих поддержки виртуальной памяти (все процессы расположены в одном непрерывном блоке памяти)}
\EN{\ac{PIC} was 
crucial in early computer systems and crucial now in embedded systems without 
virtual memory support (where processes are all placed in single continous memory block)}.
\RU{Он до сих пор используется в *NIX системах для динамических библиотек, потому что динамическая библиотека может использоваться одновременно в нескольких процессах, будучи загружена в память только один раз}
\EN{It is also still used in *NIX systems for shared libraries since shared libraries 
are shared across many processes while loaded in memory only once}.
\RU{Но все эти процессы могут загрузить одну и ту же динамическую библиотеку по разным адресам, вот почему динамическая библиотека должна работать корректно, не привязываясь к абсолютным адресам}\EN{But all these processes may 
map the same shared library on different addresses, so that is why
shared library should be working correctly without fixing on any absolute address}.

\RU{Простой эксперимент}\EN{Let's do a simple experiment}:

\begin{lstlisting}
#include <stdio.h>

int global_variable=123;

int f1(int var)
{
    int rt=global_variable+var;
    printf ("returning %d\n", rt);
    return rt;
};
\end{lstlisting}

\RU{Скомпилируем в GCC 4.7.3 и посмотрим итоговый файл .so в}\EN{Let's compile it in GCC 4.7.3 and see resulting .so file in} \IDA:

\begin{lstlisting}
gcc -fPIC -shared -O3 -o 1.so 1.c
\end{lstlisting}

\begin{lstlisting}[caption=GCC 4.7.3]
.text:00000440                 public __x86_get_pc_thunk_bx
.text:00000440 __x86_get_pc_thunk_bx proc near         ; CODE XREF: _init_proc+4
.text:00000440                                         ; deregister_tm_clones+4 ...
.text:00000440                 mov     ebx, [esp+0]
.text:00000443                 retn
.text:00000443 __x86_get_pc_thunk_bx endp

.text:00000570                 public f1
.text:00000570 f1              proc near
.text:00000570
.text:00000570 var_1C          = dword ptr -1Ch
.text:00000570 var_18          = dword ptr -18h
.text:00000570 var_14          = dword ptr -14h
.text:00000570 var_8           = dword ptr -8
.text:00000570 var_4           = dword ptr -4
.text:00000570 arg_0           = dword ptr  4
.text:00000570
.text:00000570                 sub     esp, 1Ch
.text:00000573                 mov     [esp+1Ch+var_8], ebx
.text:00000577                 call    __x86_get_pc_thunk_bx
.text:0000057C                 add     ebx, 1A84h
.text:00000582                 mov     [esp+1Ch+var_4], esi
.text:00000586                 mov     eax, ds:(global_variable_ptr - 2000h)[ebx]
.text:0000058C                 mov     esi, [eax]
.text:0000058E                 lea     eax, (aReturningD - 2000h)[ebx] ; "returning %d\n"
.text:00000594                 add     esi, [esp+1Ch+arg_0]
.text:00000598                 mov     [esp+1Ch+var_18], eax
.text:0000059C                 mov     [esp+1Ch+var_1C], 1
.text:000005A3                 mov     [esp+1Ch+var_14], esi
.text:000005A7                 call    ___printf_chk
.text:000005AC                 mov     eax, esi
.text:000005AE                 mov     ebx, [esp+1Ch+var_8]
.text:000005B2                 mov     esi, [esp+1Ch+var_4]
.text:000005B6                 add     esp, 1Ch
.text:000005B9                 retn
.text:000005B9 f1              endp
\end{lstlisting}

\newcommand{\retstring}{\IT{<<returning \%d\textbackslash{}n>>}}
\newcommand{\globvar}{\IT{global\_variable}}

\RU{Так и есть: указатели на строку \retstring{} и переменную \globvar{} корректируются при каждом исполнении функции}
\EN{That's it: pointers to \retstring{} string and \globvar{} are to be corrected at each function execution.}
\RU{Функция}\EN{The} \TT{\_\_x86\_get\_pc\_thunk\_bx()} \RU{возвращает адрес точки после вызова самой себя (здесь: \TT{0x57C}) в}\EN{function return address of the point after call to itself (\TT{0x57C} here) in the} \EBX.
\RU{Это очень простой способ получить значение указателя на текущую инструкцию (\EIP) в произвольном месте}
\EN{That's the simple way to get value of program counter (\EIP) at some point}.
\RU{Константа}\EN{The} \TT{0x1A84} \RU{связана с разницей между началом этой функции и так называемой}\EN{constant is related to the difference between this function begin and so-called}
\IT{Global Offset Table Procedure Linkage Table} (GOT PLT), \RU{секцией, сразу же за}\EN{the section right after} \IT{Global Offset Table} (GOT), \RU{где находится указатель на \globvar{}}\EN{where pointer to \globvar{} is}.
\IDA \RU{показывает смещения уже обработанными, чтобы их было проще понимать, но на самом деле код такой}\EN{shows these offset processed, so to understand them easily, but in fact the code is}:

\begin{lstlisting}
.text:00000577                 call    __x86_get_pc_thunk_bx
.text:0000057C                 add     ebx, 1A84h
.text:00000582                 mov     [esp+1Ch+var_4], esi
.text:00000586                 mov     eax, [ebx-0Ch]
.text:0000058C                 mov     esi, [eax]
.text:0000058E                 lea     eax, [ebx-1A30h]
\end{lstlisting}

\RU{Так что, \EBX указывает на секцию \TT{GOT PLT} и для вычисления указателя на \globvar{}, которая хранится в \TT{GOT}, нужно вычесть 0xC}\EN{So, \EBX pointing to the \TT{GOT PLT} section and to calculate pointer to \globvar{} which stored in 
the \TT{GOT}, \TT{0xC} must be subtracted}.
\RU{А чтобы вычислить указатель на \retstring{}, нужно вычесть \TT{0x1A30}}
\EN{To calculate pointer to the \retstring{} string, \TT{0x1A30} must be subtracted}.

\index{x86-64}
\index{x86!\Registers!RIP}
\RU{Кстати, вот зачем в AMD64 появилась поддержка адресации относительно RIP\footnote{указатель инструкций в AMD64}, просто для упрощения PIC-кода}
\EN{By the way, that is the reason why AMD64 instruction set supports RIP\footnote{program counter in AMD64}-relative addressing, just to simplify PIC-code}.

\RU{Скомпилируем тот же код на Си при помощи той же версии GCC, но для x64}\EN{Let's compile the same C code in the same GCC version, but for x64}.

\index{objdump}
\IDA \RU{упростит код на выходе убирая упоминания RIP, так что я буду использовать \IT{objdump} вместо}
\EN{would simplify output code but suppressing RIP-relative addressing details, so I will run \IT{objdump} instead to see the details}:

\begin{lstlisting}
0000000000000720 <f1>:
 720:	48 8b 05 b9 08 20 00 	mov    rax,QWORD PTR [rip+0x2008b9]        # 200fe0 <_DYNAMIC+0x1d0>
 727:	53                   	push   rbx
 728:	89 fb                	mov    ebx,edi
 72a:	48 8d 35 20 00 00 00 	lea    rsi,[rip+0x20]        # 751 <_fini+0x9>
 731:	bf 01 00 00 00       	mov    edi,0x1
 736:	03 18                	add    ebx,DWORD PTR [rax]
 738:	31 c0                	xor    eax,eax
 73a:	89 da                	mov    edx,ebx
 73c:	e8 df fe ff ff       	call   620 <__printf_chk@plt>
 741:	89 d8                	mov    eax,ebx
 743:	5b                   	pop    rbx
 744:	c3                   	ret    
\end{lstlisting}

\TT{0x2008b9} \RU{это разница между адресом инструкции по \TT{0x720} и \globvar{}, 
а \TT{0x20} это разница между инструкцией по \TT{0x72A} и строкой \retstring{}}
\EN{is the difference between address of instruction at \TT{0x720} and \globvar{} and 
\TT{0x20} is the difference between the address of the instruction at 
\TT{0x72A} and the \retstring{} string}.

\RU{Как видно, необходимость очень часто пересчитывать адреса делает исполнение немного медленнее 
(хотя это и стало лучше в x64)}
\EN{As you might see, the need to recalculate addresses frequently makes execution slower 
(it is better in x64, though)}.
\RU{Так что если вы заботитесь о скорости исполнения, то, наверное, нужно задуматься о статической
компоновке (static linking)}
\EN{So it is probably better to link statically if you aware of performance} (\cite{AgnerFogCPP}).

\subsection{Windows}

\RU{Такой механизм не используется в Windows DLL. Если загрузчику в Windows приходится загружать DLL 
в другое место, он ``патчит'' DLL прямо в памяти (на местах \IT{FIXUP}-ов) чтобы скорректировать 
все адреса.}\EN{The PIC mechanism is not used in Windows DLLs. If Windows loader needs to load DLL 
on another base address, it ``patches'' DLL in memory (at the \IT{FIXUP} places) in order to correct 
all addresses.}
\RU{Это приводит к тому что загруженную один раз DLL нельзя использовать одновременно в разных 
процессах, желающих расположить её по разным адресам ~--- потому что каждый загруженный в память 
экземпляр DLL \IT{доводится} до того чтобы работать только по этим адресам.}\EN{This means, several 
Windows processes cannot share once loaded DLL on different addresses in different process' memory 
blocks~---since each loaded into memory DLL instance \IT{fixed} to be work only at these addresses..}
