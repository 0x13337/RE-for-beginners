\subsection{\RU{Теперь вспомним MSVC}\EN{Now let's get back to MSVC}}

\index{\Cpp!\RU{исключения}\EN{exceptions}}
\RU{Должно быть, программистам Microsoft были нужны исключения в Си, но не в \Cpp, так что они добавили
нестандартное расширение Си в MSVC}
\EN{Supposedly, Microsoft programmers needed exceptions in C, but not in \Cpp, so they added a non-standard C extension
to MSVC}\footnote{\href{http://go.yurichev.com/17057}{MSDN}}.
\RU{Оно не связано с исключениями в Си++}\EN{It is not related to C++ \ac{PL} exceptions}.

\begin{lstlisting}
__try
{
    ...
}
__except(filter code)
{
    handler code
}
\end{lstlisting}

\RU{Блок \q{finally} может присутствовать вместо код обработчика}\EN{\q{Finally} block may be instead of handler code}:

\begin{lstlisting}
__try
{
    ...
}
__finally
{
    ...
}
\end{lstlisting}

\RU{Код-фильтр --- это выражение, отвечающее на вопрос, соответствует ли код этого обработчика к поднятому исключению}
\EN{The filter code is an expression, telling whether this handler code corresponds to the exception raised}.
\RU{Если ваш код слишком большой и не помещается в одно выражение, отдельная функция-фильтр может быть определена}
\EN{If your code is too big and cannot fit into one expression, a separate filter function can be defined}.\\
\\
\RU{Таких конструкций много в ядре Windows}\EN{There are a lot of such constructs in the Windows kernel}.
\RU{Вот несколько примеров оттуда}\EN{Here are a couple of examples from there} (\ac{WRK}):

\lstinputlisting[caption=WRK-v1.2/base/ntos/ob/obwait.c]{OS/SEH/2/wrk_ex1.c}

\lstinputlisting[caption=WRK-v1.2/base/ntos/cache/cachesub.c]{OS/SEH/2/wrk_ex2.c}

\RU{Вот пример кода-фильтра}\EN{Here is also a filter code example}:

\lstinputlisting[caption=WRK-v1.2/base/ntos/cache/copysup.c]{OS/SEH/2/wrk_ex3.c}

\RU{Внутри, SEH это расширение исключений поддерживаемых OS}\EN{Internally, SEH is an extension of the OS-supported exceptions}.
\RU{Но функция обработчик теперь или}\EN{But the handler function is} \TT{\_except\_handler3} (\ForENRU SEH3) 
\OrENRU \TT{\_except\_handler4} (\ForENRU SEH4).
\RU{Код обработчика от MSVC, расположен в его библиотеках, или же в}
\EN{The code of this handler is MSVC-related, it is located in its libraries, or in} msvcr*.dll.
\RU{Очень важно понимать, что SEH это специфичное для MSVC}\EN{It is very important to know that SEH is a MSVC thing}.
\RU{Другие win32-компиляторы могут предлагать что-то совершенно другое.}
\EN{Other win32-compilers may offer something completely different.}

\subsubsection{SEH3}

SEH3 \RU{имеет}\EN{has} \TT{\_except\_handler3} \RU{как функцию-обработчик, и расширяет структуру}
\EN{as a handler function, and extends the} \TT{\_EXCEPTION\_REGISTRATION} \RU{добавляя указатель на \IT{scope table}
и переменную \IT{previous try level}}\EN{table, adding
a pointer to the \IT{scope table} and \IT{previous try level} variable}.
SEH4 \RU{расширяет}\EN{extends the} \IT{scope table} \RU{добавляя еще 4 значения связанных с защитой от переполнения буфера}
\EN{by 4 values for buffer overflow protection}.\\
\\
\EN{The }\IT{scope table} \RU{это таблица, состоящая из указателей на код фильтра и обработчика, для каждого уровня вложенности
\IT{try/except}}\EN{is a table that consists of pointers to the filter and handler code blocks, for each nested level of \IT{try/except}}.

\begin{center}
\ifdefined\ebook
\begin{tikzpicture}[thick,scale=0.50, every node/.style={scale=0.50}]
\else
\begin{tikzpicture}[thick,scale=0.85, every node/.style={scale=0.85}]
\fi
	\tikzstyle{every path}=[thick]
	\tikzstyle{undefined}=[draw,rectangle,minimum height=1cm, minimum width=3.5cm, text width=3.5cm]
	\tikzstyle{undefinedw}=[draw,rectangle,minimum height=1cm, minimum width=4.5cm, text width=4.5cm]
	\tikzstyle{node}=[draw,rectangle,minimum height=1cm, minimum width=3.5cm, text width=3.5cm, fill=gray!20]
	\tikzstyle{nodew}=[draw,rectangle,minimum height=1cm, minimum width=4.5cm, text width=4.5cm, fill=gray!20]
	\tikzstyle{text_as_rect}=[rectangle,minimum height=1cm, minimum width=3.5cm, text width=3.5cm]
	\tikzstyle{point}=[inner sep=0pt]

	\node[node] (fs) [minimum width=1.5cm, text width=1.5cm] {FS:0};

\begin{scope}[start chain=1 going below, node distance=0cm]
	\node[on chain=1,node] (tib1) [right=1.5cm of fs] {+0: \_\_except\_list};
	\node[on chain=1,undefined] {+4: \dots};
	\node[on chain=1,undefined] {+8: \dots};
\end{scope}
	\node (tib_text) [above of=tib1] {TIB};
	
	\draw [->] (fs.east) -- (tib1.west);

\begin{scope}[start chain=3 going below, node distance=0cm]
	\node[on chain=3,undefined] (u1) [text centered, right=2.5cm of tib1] {\dots};
	%\node[on chain=3,node] (n1prev) {Prev=0xFFFFFFFF (1)};
	\node[on chain=3,node] (n1prev) {Prev=0xFFFFFFFF};
	\node[on chain=3,node] (n1handler) {Handle};
	\node[on chain=3,undefined] [text centered] {\dots};
	\node[on chain=3,node] (n2prev) {Prev};
	\node[on chain=3,node] (n2handler) {Handle};
	\node[on chain=3,undefined] [text centered] {\dots};
	\node[on chain=3,node] (n3prev) {Prev};
	\node[on chain=3,node] (n3handler) {Handle};
	\ifx\SEHfour\undefined	
	\node[on chain=3,node] (n3scopetable) {\ScopeTable};
	\else
	\node[on chain=3,node] (n3scopetable) {\ScopeTable$\oplus$security\_cookie};
	\fi
	\node[on chain=3,node] {\RU{предыдущий уровень try}\EN{previous try level}};
	\node[on chain=3,node] (n3ebp) {EBP};
	\ifx\SEHfour\undefined
	\else
	\node[on chain=3,undefined] [text centered] {\dots};
	\node[on chain=3,node] (AnotherCookie) {EBP$\oplus$security\_cookie};
	\node[on chain=3,undefined] [text centered] {\dots};
	\fi
\end{scope}
	
	\node [node] (n1handler_text) [right=1cm of n1handler] {\HandlerFunction};
	\draw [->] (n1handler.east) -- (n1handler_text.west);
	\node [node] (n2handler_text) [right=1cm of n2handler] {\HandlerFunction};
	\draw [->] (n2handler.east) -- (n2handler_text.west);

	\ifx\SEHfour\undefined	
	\node [node] (n3handler_text) [right=1cm of n3handler] {\_except\_handler3};
	\else
	\node [node] (n3handler_text) [right=1cm of n3handler] {\_except\_handler4};
	\fi

	\draw [->] (n3handler.east) -- (n3handler_text.west);
	\node[undefined] (u4) [text centered, below of=n3ebp] {\dots};
	\node (stack_text) [above of=u1] {Stack};
	
	\node[point] (n2block_pt2) [above left=0cm and 0.8cm of n2prev] {};
	\draw [->] (n3prev.west) .. controls +(left:0.8cm) and (n2block_pt2) .. (n2prev.north west);

	\node[point] (n1block_pt2) [above left=0cm and 0.8cm of n1prev] {};
	\draw [->] (n2prev.west) .. controls +(left:0.8cm) and (n1block_pt2) .. (n1prev.north west);
	
	\node[point] (n3block_pt2) [above left=0cm and 1.25cm of n3prev] {};
	\draw [->] (tib1.east) .. controls +(right:1.25cm) and (n3block_pt2) .. (n3prev.north west);

\begin{scope}[start chain=2 going below, node distance=0cm]
	% SEH3
	\ifx\SEHfour\undefined	
	\node[on chain=2,nodew] (scope_tbl) [left=3.5cm of n2handler] {0xFFFFFFFF (-1)};
	\else
	% SEH4 header
	\node[on chain=2,nodew] (scope_tbl) [left=3.5 cm of n2handler] {\RU{смещение GS Cookie}\EN{GS Cookie Offset}};
	\node[on chain=2,nodew] {\RU{смещение GS Cookie XOR}\EN{GS Cookie XOR Offset}};
	\node[on chain=2,nodew] (EHCookieOffset) {\RU{смещение EH Cookie}\EN{EH Cookie Offset}};
	\node[on chain=2,nodew] {\RU{смещение EH Cookie XOR}\EN{EH Cookie XOR Offset}};
	\node[on chain=2,minimum height=0.3cm] {};
	\node[on chain=2,nodew] {0xFFFFFFFF (-1)};
	\fi

	\node[on chain=2,nodew] (scope_tbl_filter1) {\FilterFunction};
	\node[on chain=2,nodew] {\HandlerFinallyFunction};
	\node[on chain=2,minimum height=0.3cm] {};
	\node[on chain=2,nodew] {0};
	\node[on chain=2,nodew] (scope_tbl_filter2) {\FilterFunction};
	\node[on chain=2,nodew] {\HandlerFinallyFunction};
	\node[on chain=2,minimum height=0.3cm] {};
	\node[on chain=2,nodew] {1};
	\node[on chain=2,nodew] (scope_tbl_filter3) {\FilterFunction};
	\node[on chain=2,nodew] {\HandlerFinallyFunction};
	\node[on chain=2,minimum height=0.3cm] {};
	\node[on chain=2,undefinedw] [text centered] {\dots \MoreEntries \dots};

\end{scope}
	
	\node[text_as_rect,left=0.5cm of scope_tbl_filter1] {\RU{информация для первого блока try/except}\EN{information about first try/except block}};
	\node[text_as_rect,left=0.5cm of scope_tbl_filter2] {\RU{информация для второго блока try/except}\EN{information about second try/except block}};
	\node[text_as_rect,left=0.5cm of scope_tbl_filter3] {\RU{информация для третьего блока try/except}\EN{information about third try/except block}};

	\node (scope_text) [above of=scope_tbl] {\ScopeTable};
	\node[point] (scope_table_pt2) [right=1.75cm of scope_tbl] {};
	\draw [->] (n3scopetable.west) .. controls +(left:1.75cm) and (scope_table_pt2) .. (scope_tbl.north east);

	\ifx\SEHfour\undefined
	\else
	\node[point] (AnotherCookie_pt2) [left=1.75cm of AnotherCookie] {};
	\draw [->] (EHCookieOffset.east) .. controls +(right:1.75cm) and (AnotherCookie_pt2) .. (AnotherCookie.west);
	\fi

\end{tikzpicture}
\end{center}


\RU{И снова, очень важно понимать, что OS заботится только о полях}
\EN{Again, it is very important to understand that the \ac{OS} takes care only of the} \IT{prev/handle}
\RU{, и больше ничего}\EN{ fields, and nothing more}.
\RU{Это работа функции}\EN{It is the job of the} \TT{\_except\_handler3} \RU{читать другие поля, читать \IT{scope table}
и решать, какой обработчик исполнять и когда}\EN{function to read the other fields and \IT{scope table}, and decide
which handler to execute and when}.\\
\\
\index{Wine}
\index{ReactOS}
\RU{Исходный код функции}\EN{The source code of the} \TT{\_except\_handler3} \RU{закрыт}\EN{function is closed}.
\RU{Хотя, Sanos OS, имеющая слой совместимости с win32, имеет некоторые функции написанные заново, которые
в каком-то смысле эквивалентны тем что в Windows}
\EN{However, Sanos OS, which has a win32 compatibility layer, has the same
functions reimplemented, which are somewhat equivalent to those in Windows}
\footnote{\url{http://go.yurichev.com/17058}}.
\RU{Другие попытки реализации имеются в}\EN{Another reimplementation is present in} 
Wine\footnote{\href{http://go.yurichev.com/17059}{GitHub}}
\AndENRU 
ReactOS\footnote{\url{http://go.yurichev.com/17060}}.\\
\\
\RU{Если указатель}\EN{If the} \IT{filter} \RU{ноль,}\EN{pointer is NULL, the} \IT{handler} \RU{указывает на код \IT{finally}}
\EN{pointer is the pointer to the \IT{finally} code block}.\\
\\
\RU{Во время исполнения, значение \IT{previous try level} в стеке меняется, чтобы функция
\TT{\_except\_handler3} знала о текущем уровне вложенности, чтобы знать, какой элемент таблицы
\IT{scope table} использовать.}
\EN{During execution, the \IT{previous try level} value in the stack changes, so
\TT{\_except\_handler3} can get information about the current level of nestedness, 
in order to know which \IT{scope table} entry to use.}

\subsubsection{SEH3: \RU{пример с одним блоком try/except}\EN{one try/except block example}}

\lstinputlisting{OS/SEH/2/2.c}

\lstinputlisting[caption=MSVC 2003]{OS/SEH/2/2_SEH3.asm}

\RU{Здесь мы видим, как структура SEH конструируется в стеке}\EN{Here we see how the SEH frame is constructed in the stack}.
\RU{\IT{Scope table} расположена в сегменте}\EN{The \IT{scope table} is located in the} \TT{CONST}\EN{ segment}\EMDASH{}
\RU{действительно, эти поля не будут меняться}\EN{indeed, these fields are not to be changed}.
\RU{Интересно, как меняется переменная}\EN{An interesting thing is how the} \IT{previous try level}\EN{ variable has changed}.
\RU{Исходное значение}\EN{The initial value is} \TT{0xFFFFFFFF} ($-1$).
\RU{Момент, когда тело}\EN{The moment when the body of the} \TT{try} \RU{открывается, обозначен инструкцией, записывающей}
\EN{statement is opened is marked with an instruction that writes} 0 \RU{в эту переменную}\EN{to the variable}.
\RU{В момент, когда тело}\EN{The moment when the body of the} \TT{try} \RU{закрывается}\EN{statement is closed}, $-1$ 
\RU{возвращается в нее назад}\EN{is written back to it}.
\RU{Мы также видим адреса кода фильтра и обработчика}\EN{We also see the addresses of filter and handler code}.
\RU{Так мы можем легко увидеть структуру конструкций \IT{try/except} в функции}
\EN{Thus we can easily see the structure of the \IT{try/except} constructs in the function}.\\
\\
\RU{Так как код инициализации SEH-структур в прологе функций может быть общим для нескольких функций, иногда компилятор
вставляет в прологе вызов функции}\EN{Since the SEH setup code in the function prologue may be shared between many functions,
sometimes the compiler inserts a call to the} \TT{SEH\_prolog()}\RU{, которая всё это делает}
\EN{ function in the prologue, which does just that}.
\RU{А код для деинициализации SEH в функции \TT{SEH\_epilog()}}
\EN{The SEH cleanup code is in the \TT{SEH\_epilog()} function}.\\
\\
\RU{Запустим этот пример в}\EN{Let's try to run this example in} \tracer{}:
\index{tracer}

\begin{lstlisting}
tracer.exe -l:2.exe --dump-seh
\end{lstlisting}

\begin{lstlisting}[caption=tracer.exe output]
EXCEPTION_ACCESS_VIOLATION at 2.exe!main+0x44 (0x401054) ExceptionInformation[0]=1
EAX=0x00000000 EBX=0x7efde000 ECX=0x0040cbc8 EDX=0x0008e3c8
ESI=0x00001db1 EDI=0x00000000 EBP=0x0018feac ESP=0x0018fe80
EIP=0x00401054
FLAGS=AF IF RF
* SEH frame at 0x18fe9c prev=0x18ff78 handler=0x401204 (2.exe!_except_handler3)
SEH3 frame. previous trylevel=0
scopetable entry[0]. previous try level=-1, filter=0x401070 (2.exe!main+0x60) handler=0x401088 (2.exe!main+0x78)
* SEH frame at 0x18ff78 prev=0x18ffc4 handler=0x401204 (2.exe!_except_handler3)
SEH3 frame. previous trylevel=0
scopetable entry[0]. previous try level=-1, filter=0x401531 (2.exe!mainCRTStartup+0x18d) handler=0x401545 (2.exe!mainCRTStartup+0x1a1)
* SEH frame at 0x18ffc4 prev=0x18ffe4 handler=0x771f71f5 (ntdll.dll!__except_handler4)
SEH4 frame. previous trylevel=0
SEH4 header:	GSCookieOffset=0xfffffffe GSCookieXOROffset=0x0
		EHCookieOffset=0xffffffcc EHCookieXOROffset=0x0
scopetable entry[0]. previous try level=-2, filter=0x771f74d0 (ntdll.dll!___safe_se_handler_table+0x20) handler=0x771f90eb (ntdll.dll!_TppTerminateProcess@4+0x43)
* SEH frame at 0x18ffe4 prev=0xffffffff handler=0x77247428 (ntdll.dll!_FinalExceptionHandler@16)
\end{lstlisting}

\RU{Мы видим, что цепочка SEH состоит из 4-х обработчиков}\EN{We see that the SEH chain consists of 4 handlers}.\\
\\
\index{CRT}
\RU{Первые два расположены в нашем примере}\EN{The first two are located in our example}. \RU{Два}\EN{Two}?
\RU{Но ведь мы же сделали только один}\EN{But we made only one}?
\RU{Да, второй был установлен в \ac{CRT}-функции}\EN{Yes, another one was set up in the \ac{CRT} function} 
\TT{\_mainCRTStartup()}, \RU{и судя по всему, он обрабатывает как минимум исключения связанные с \ac{FPU}}
\EN{and as it seems that it handles at least \ac{FPU} exceptions}.
\RU{Его код можно посмотреть в инсталляции MSVC}\EN{Its source code can found in the MSVC installation}: \TT{crt/src/winxfltr.c}.\\
\\
\RU{Третий это SEH4 в}\EN{The third is the SEH4 one in} ntdll.dll, \RU{и четвертый это обработчик, не имеющий отношения к MSVC,
расположенный в}
\EN{and the fourth handler is not MSVC-related and is located in} ntdll.dll,
\RU{имеющий \q{говорящее} название функции}\EN{and has a self-describing function name}.\\
\\
\RU{Как видно, в цепочке присутствуют обработчики трех типов}\EN{As you can see, there are 3 types of handlers in one chain}:
\RU{один не связан с MSVC вообще (последний) и два связанных с MSVC}
\EN{one is not related to MSVC at all (the last one) and two MSVC-related}: SEH3 \AndENRU SEH4.

\subsubsection{SEH3: \RU{пример с двумя блоками try/except}\EN{two try/except blocks example}}

\lstinputlisting{OS/SEH/2/3.c}

\RU{Теперь здесь два блока \TT{try}}\EN{Now there are two \TT{try} blocks}.
\RU{Так что}\EN{So the} \IT{scope table} \RU{теперь содержит два элемента, один элемент на каждый блок}
\EN{now has two entries, one for each block}.
\IT{Previous try level} \RU{меняется вместе с тем, как исполнение доходит до очередного \TT{try}-блока, либо
выходит из него}\EN{changes as execution flow enters or exits the \TT{try} block}.

\lstinputlisting[caption=MSVC 2003]{OS/SEH/2/3_SEH3.asm}

\RU{Если установить брякпойнт на функцию}\EN{If we set a breakpoint on the} \printf{} \RU{вызываемую из обработчика,
мы можем увидеть, что добавился еще один SEH-обработчик}\EN{function, which is called from the handler, 
we can also see how yet another SEH handler is added}.
\RU{Наверное, это еще какая-то дополнительная механика, скрытая внутри процесса обработки исключений}
\EN{Perhaps it's another machinery inside the SEH handling process}.
\RU{Тут мы также видим}\EN{Here we also see our} \IT{scope table} \RU{состоящую из двух элементов}
\EN{consisting of 2 entries}.

\begin{lstlisting}
tracer.exe -l:3.exe bpx=3.exe!printf --dump-seh
\end{lstlisting}

\begin{lstlisting}[caption=tracer.exe output]
(0) 3.exe!printf
EAX=0x0000001b EBX=0x00000000 ECX=0x0040cc58 EDX=0x0008e3c8
ESI=0x00000000 EDI=0x00000000 EBP=0x0018f840 ESP=0x0018f838
EIP=0x004011b6
FLAGS=PF ZF IF
* SEH frame at 0x18f88c prev=0x18fe9c handler=0x771db4ad (ntdll.dll!ExecuteHandler2@20+0x3a)
* SEH frame at 0x18fe9c prev=0x18ff78 handler=0x4012e0 (3.exe!_except_handler3)
SEH3 frame. previous trylevel=1
scopetable entry[0]. previous try level=-1, filter=0x401120 (3.exe!main+0xb0) handler=0x40113b (3.exe!main+0xcb)
scopetable entry[1]. previous try level=0, filter=0x4010e8 (3.exe!main+0x78) handler=0x401100 (3.exe!main+0x90)
* SEH frame at 0x18ff78 prev=0x18ffc4 handler=0x4012e0 (3.exe!_except_handler3)
SEH3 frame. previous trylevel=0
scopetable entry[0]. previous try level=-1, filter=0x40160d (3.exe!mainCRTStartup+0x18d) handler=0x401621 (3.exe!mainCRTStartup+0x1a1)
* SEH frame at 0x18ffc4 prev=0x18ffe4 handler=0x771f71f5 (ntdll.dll!__except_handler4)
SEH4 frame. previous trylevel=0
SEH4 header:	GSCookieOffset=0xfffffffe GSCookieXOROffset=0x0
		EHCookieOffset=0xffffffcc EHCookieXOROffset=0x0
scopetable entry[0]. previous try level=-2, filter=0x771f74d0 (ntdll.dll!___safe_se_handler_table+0x20) handler=0x771f90eb (ntdll.dll!_TppTerminateProcess@4+0x43)
* SEH frame at 0x18ffe4 prev=0xffffffff handler=0x77247428 (ntdll.dll!_FinalExceptionHandler@16)
\end{lstlisting}

\subsubsection{SEH4}

\index{\BufferOverflow}
\index{Security cookie}
\RU{Во время атаки переполнения буфера}\EN{During a buffer overflow} (\myref{subsec:bufferoverflow})
\RU{адрес}\EN{attack, the address of the} \IT{scope table} \RU{может быть перезаписан, так что начиная с}
\EN{can be rewritten, so starting from} MSVC 2005, SEH3 \RU{был дополнен защитой от переполнения буфера, до SEH4}
\EN{was upgraded to SEH4 in order to have buffer overflow protection}.
\RU{Указатель на}\EN{The pointer to the} \IT{scope table} \RU{теперь}\EN{is now} \glslink{xoring}{\RU{про-XOR-ен}\EN{xored}} 
\RU{с}\EN{with a} \gls{security cookie}.
\EN{The \IT{scope table} was extended to have a header consisting of two pointers to}
\RU{\IT{Scope table} расширена, теперь имеет заголовок, содержащий 2 указателя на} \IT{security cookies}.
\RU{Каждый элемент имеет смешение внутри стека на другое значение: это адрес \glslink{stack frame}{фрейма} (\EBP) также}
\EN{Each element has an offset inside the stack of another value: 
the address of the \gls{stack frame} (\EBP)} \glslink{xoring}{\RU{про-XOR-еный}\EN{xored}} \RU{с}\EN{with the} 
\TT{security\_cookie} \RU{расположенный в стеке}\EN{, placed in the stack}.
\RU{Это значение будет прочитано во время обработки исключения и проверено на правильность.}
\EN{This value will be read during exception handling and checked for correctness.}
\EN{The \IT{security cookie} in the stack is random each time, so hopefully a remote attacker can't predict it.}
\RU{\IT{Security cookie} в стеке случайное каждый раз, так что атакующий, как мы надеемся, не может предсказать его.}\\
\\
\RU{Изначальное значение}\EN{The initial} \IT{previous try level} \RU{это}\EN{is} $-2$ \InENRU SEH4 \RU{вместо}\EN{instead of} $-1$.

\def\SEHfour{1}
\begin{center}
\ifdefined\ebook
\begin{tikzpicture}[thick,scale=0.50, every node/.style={scale=0.50}]
\else
\begin{tikzpicture}[thick,scale=0.85, every node/.style={scale=0.85}]
\fi
	\tikzstyle{every path}=[thick]
	\tikzstyle{undefined}=[draw,rectangle,minimum height=1cm, minimum width=3.5cm, text width=3.5cm]
	\tikzstyle{undefinedw}=[draw,rectangle,minimum height=1cm, minimum width=4.5cm, text width=4.5cm]
	\tikzstyle{node}=[draw,rectangle,minimum height=1cm, minimum width=3.5cm, text width=3.5cm, fill=gray!20]
	\tikzstyle{nodew}=[draw,rectangle,minimum height=1cm, minimum width=4.5cm, text width=4.5cm, fill=gray!20]
	\tikzstyle{text_as_rect}=[rectangle,minimum height=1cm, minimum width=3.5cm, text width=3.5cm]
	\tikzstyle{point}=[inner sep=0pt]

	\node[node] (fs) [minimum width=1.5cm, text width=1.5cm] {FS:0};

\begin{scope}[start chain=1 going below, node distance=0cm]
	\node[on chain=1,node] (tib1) [right=1.5cm of fs] {+0: \_\_except\_list};
	\node[on chain=1,undefined] {+4: \dots};
	\node[on chain=1,undefined] {+8: \dots};
\end{scope}
	\node (tib_text) [above of=tib1] {TIB};
	
	\draw [->] (fs.east) -- (tib1.west);

\begin{scope}[start chain=3 going below, node distance=0cm]
	\node[on chain=3,undefined] (u1) [text centered, right=2.5cm of tib1] {\dots};
	%\node[on chain=3,node] (n1prev) {Prev=0xFFFFFFFF (1)};
	\node[on chain=3,node] (n1prev) {Prev=0xFFFFFFFF};
	\node[on chain=3,node] (n1handler) {Handle};
	\node[on chain=3,undefined] [text centered] {\dots};
	\node[on chain=3,node] (n2prev) {Prev};
	\node[on chain=3,node] (n2handler) {Handle};
	\node[on chain=3,undefined] [text centered] {\dots};
	\node[on chain=3,node] (n3prev) {Prev};
	\node[on chain=3,node] (n3handler) {Handle};
	\ifx\SEHfour\undefined	
	\node[on chain=3,node] (n3scopetable) {\ScopeTable};
	\else
	\node[on chain=3,node] (n3scopetable) {\ScopeTable$\oplus$security\_cookie};
	\fi
	\node[on chain=3,node] {\RU{предыдущий уровень try}\EN{previous try level}};
	\node[on chain=3,node] (n3ebp) {EBP};
	\ifx\SEHfour\undefined
	\else
	\node[on chain=3,undefined] [text centered] {\dots};
	\node[on chain=3,node] (AnotherCookie) {EBP$\oplus$security\_cookie};
	\node[on chain=3,undefined] [text centered] {\dots};
	\fi
\end{scope}
	
	\node [node] (n1handler_text) [right=1cm of n1handler] {\HandlerFunction};
	\draw [->] (n1handler.east) -- (n1handler_text.west);
	\node [node] (n2handler_text) [right=1cm of n2handler] {\HandlerFunction};
	\draw [->] (n2handler.east) -- (n2handler_text.west);

	\ifx\SEHfour\undefined	
	\node [node] (n3handler_text) [right=1cm of n3handler] {\_except\_handler3};
	\else
	\node [node] (n3handler_text) [right=1cm of n3handler] {\_except\_handler4};
	\fi

	\draw [->] (n3handler.east) -- (n3handler_text.west);
	\node[undefined] (u4) [text centered, below of=n3ebp] {\dots};
	\node (stack_text) [above of=u1] {Stack};
	
	\node[point] (n2block_pt2) [above left=0cm and 0.8cm of n2prev] {};
	\draw [->] (n3prev.west) .. controls +(left:0.8cm) and (n2block_pt2) .. (n2prev.north west);

	\node[point] (n1block_pt2) [above left=0cm and 0.8cm of n1prev] {};
	\draw [->] (n2prev.west) .. controls +(left:0.8cm) and (n1block_pt2) .. (n1prev.north west);
	
	\node[point] (n3block_pt2) [above left=0cm and 1.25cm of n3prev] {};
	\draw [->] (tib1.east) .. controls +(right:1.25cm) and (n3block_pt2) .. (n3prev.north west);

\begin{scope}[start chain=2 going below, node distance=0cm]
	% SEH3
	\ifx\SEHfour\undefined	
	\node[on chain=2,nodew] (scope_tbl) [left=3.5cm of n2handler] {0xFFFFFFFF (-1)};
	\else
	% SEH4 header
	\node[on chain=2,nodew] (scope_tbl) [left=3.5 cm of n2handler] {\RU{смещение GS Cookie}\EN{GS Cookie Offset}};
	\node[on chain=2,nodew] {\RU{смещение GS Cookie XOR}\EN{GS Cookie XOR Offset}};
	\node[on chain=2,nodew] (EHCookieOffset) {\RU{смещение EH Cookie}\EN{EH Cookie Offset}};
	\node[on chain=2,nodew] {\RU{смещение EH Cookie XOR}\EN{EH Cookie XOR Offset}};
	\node[on chain=2,minimum height=0.3cm] {};
	\node[on chain=2,nodew] {0xFFFFFFFF (-1)};
	\fi

	\node[on chain=2,nodew] (scope_tbl_filter1) {\FilterFunction};
	\node[on chain=2,nodew] {\HandlerFinallyFunction};
	\node[on chain=2,minimum height=0.3cm] {};
	\node[on chain=2,nodew] {0};
	\node[on chain=2,nodew] (scope_tbl_filter2) {\FilterFunction};
	\node[on chain=2,nodew] {\HandlerFinallyFunction};
	\node[on chain=2,minimum height=0.3cm] {};
	\node[on chain=2,nodew] {1};
	\node[on chain=2,nodew] (scope_tbl_filter3) {\FilterFunction};
	\node[on chain=2,nodew] {\HandlerFinallyFunction};
	\node[on chain=2,minimum height=0.3cm] {};
	\node[on chain=2,undefinedw] [text centered] {\dots \MoreEntries \dots};

\end{scope}
	
	\node[text_as_rect,left=0.5cm of scope_tbl_filter1] {\RU{информация для первого блока try/except}\EN{information about first try/except block}};
	\node[text_as_rect,left=0.5cm of scope_tbl_filter2] {\RU{информация для второго блока try/except}\EN{information about second try/except block}};
	\node[text_as_rect,left=0.5cm of scope_tbl_filter3] {\RU{информация для третьего блока try/except}\EN{information about third try/except block}};

	\node (scope_text) [above of=scope_tbl] {\ScopeTable};
	\node[point] (scope_table_pt2) [right=1.75cm of scope_tbl] {};
	\draw [->] (n3scopetable.west) .. controls +(left:1.75cm) and (scope_table_pt2) .. (scope_tbl.north east);

	\ifx\SEHfour\undefined
	\else
	\node[point] (AnotherCookie_pt2) [left=1.75cm of AnotherCookie] {};
	\draw [->] (EHCookieOffset.east) .. controls +(right:1.75cm) and (AnotherCookie_pt2) .. (AnotherCookie.west);
	\fi

\end{tikzpicture}
\end{center}


\RU{Оба примера скомпилированные в}\EN{Here are both examples compiled in} MSVC 2012 \RU{с}\EN{with} SEH4:

\lstinputlisting[caption=MSVC 2012: one try block example]{OS/SEH/2/2_SEH4.asm}

\lstinputlisting[caption=MSVC 2012: two try blocks example]{OS/SEH/2/3_SEH4.asm}

\RU{Вот значение}\EN{Here is the meaning of the} \IT{cookies}: \TT{Cookie Offset} 
\RU{это разница между адресом записанного в стеке значения EBP и значения}
\EN{is the difference between the address of the saved EBP value in the stack
and the} $EBP \oplus security\_cookie$ \RU{в стеке}\EN{value in the stack}.
\TT{Cookie XOR Offset} \RU{это дополнительная разница между значением}\EN{is an additional difference between the} 
$EBP \oplus security\_cookie$ \RU{и тем что записано в стеке}\EN{value and what is
stored in the stack}.
\RU{Если это уравнение не верно, то процесс остановится из-за разрушения стека}
\EN{If this equation is not true, the process is to halt due to stack corruption}:

\begin{center}
$security\_cookie \oplus (CookieXOROffset + address\_of\_saved\_EBP) == stack[address\_of\_saved\_EBP + CookieOffset]$
\end{center}

\RU{Если}\EN{If} \TT{Cookie Offset} \RU{равно}\EN{is} $-2$, \RU{это значит, что оно не присутствует}\EN{this implies that it is not present}.

\index{tracer}
\RU{Проверка \IT{cookies} также реализована в моем}\EN{\IT{Cookies} checking is also implemented in my} \tracer{},
\RU{смотрите}\EN{see} \href{http://go.yurichev.com/17061}{GitHub} \RU{для деталей}\EN{for details}.\\
\\
\RU{Возможность переключиться назад на SEH3 все еще присутствует в компиляторах после (и включая) MSVC 2005, нужно включить
опцию \TT{/GS-}, впрочем, \ac{CRT}-код будет продолжать использовать SEH4.}
\EN{It is still possible to fall back to SEH3 in the compilers after 
(and including) MSVC 2005 by setting the \TT{/GS-} option,
however, the \ac{CRT} code use SEH4 anyway.}

