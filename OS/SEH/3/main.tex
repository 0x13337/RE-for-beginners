\subsection{Windows x64}

\label{SEH_win64}
\RU{Как видно, это не самая быстрая штука, устанавливать SEH-структуры в каждом прологе функции}
\EN{As you might think, it is not very fast thing to set up SEH frame at each function prologue}.
\RU{Еще одна проблема производительности --- это менять переменную}\EN{Another performance problem is to change} 
\IT{previous try level} \RU{много раз в течении исполнении ф-ции}\EN{value many times while function execution}.
\RU{Так что в x64 всё сильно изменилось, теперь все указатели на \TT{try}-блоки, ф-ции фильтров и обработчиков,
теперь записаны в другом PE-сегменте}
\EN{So things are changed completely in x64: now all pointers to \TT{try} blocks, filter and handler functions are stored
in another PE-segment} \TT{.pdata}, \RU{откуда обработчик исключений OS берет всю информацию}
\EN{that is where OS exception handler takes all the information}.

\RU{Вот два примера из предыдущей секции, скомпилированных для}
\EN{These are two examples from the previous section compiled for} x64:

\lstinputlisting[caption=MSVC 2012]{OS/SEH/3/2_x64.asm}

\lstinputlisting[caption=MSVC 2012]{OS/SEH/3/3_x64.asm}

\RU{Смотрите}\EN{Read} \cite{IgorSkochinsky} \RU{для более детального описания}\EN{for more detailed information about this}.

\RU{Помимо информации об исключениях, секция}\EN{Aside from exception information,} \TT{.pdata} 
\RU{также содержит начала и концы почти всех ф-ций, так что эту информацию можно использовать в каких-либо
утилитах, предназначенных для автоматизации анализа}\EN{is a section containing addresses of almost all function starts and ends,
hence it may be useful for a tools targetting automated analysis}.


