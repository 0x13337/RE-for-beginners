\subsubsection{Windows x64}

\label{SEH_win64}

As you might think, it is not very fast to set up the SEH frame at each function prologue.
Another performance problem is changing the 
\IT{previous try level} value many times during the function's execution.

So things are changed completely in x64: now all pointers to \TT{try} blocks, filter and handler functions are stored
in another PE segment \TT{.pdata}, 
and from there the \ac{OS}'s exception handler takes all the information.

Here are the two examples from the previous section compiled for x64:

\lstinputlisting[caption=MSVC 2012]{OS/SEH/3/2_x64.asm}

\lstinputlisting[caption=MSVC 2012]{OS/SEH/3/3_x64.asm}

Read \IgorSkochinsky for more detailed information about this.

Aside from exception information, \TT{.pdata}
is a section that contains the addresses of almost all function starts and ends,
hence it may be useful for a tools targeted at automated analysis.

