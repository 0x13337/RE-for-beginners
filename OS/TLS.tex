\chapter{Thread Local Storage}
\label{TLS}
\index{TLS}

\RU{Это область данных, отдельная для каждого треда. Каждый тред может хранить там то, что ему нужно}
\EN{It is a data area, specific to each thread. Every thread can store there what it needs}.
\RU{Один из известных примеров, это стандартная глобальная переменная в Си}\EN{One famous example
is C standard global variable} \IT{errno}. 
\RU{Несколько тредов одновременно могут вызывать функции
возвращающие код ошибки в \IT{errno}, поэтому глобальная переменная здесь не будет работать корректно, 
для мультитредовых программ \IT{errno} нужно хранить в в \ac{TLS}.}
\EN{Multiple threads may simultaneously call a functions
which returns error code in the \IT{errno}, so global variable will not work correctly here, for multi-thread programs,
\IT{errno} must be stored in the \ac{TLS}.} \\
\\
\index{\Cpp!C++11}
\RU{В}\EN{In the} C++11 \RU{ввели модификатор}\EN{standard, a new} \IT{thread\_local} 
\RU{, показывающий что каждый тред будет иметь свою версию этой переменной}
\EN{modifier was added, showing that each thread will have its own version of the variable},
\RU{и её можно инициализировать, и она расположена в}\EN{it can be initialized, and it is located in the} \ac{TLS}
\footnote{
\index{C11}
\RU{В C11 также есть поддержка тредов, хотя и опциональная}
\EN{C11 also has thread support, optional though}}:

\begin{lstlisting}[caption=C++11]
#include <iostream>
#include <thread>

thread_local int tmp=3;

int main()
{
	std::cout << tmp << std::endl;
};
\end{lstlisting}
\footnote{\RU{Компилируется в}\EN{Compiled in} MinGW GCC 4.8.1, \RU{но не в}\EN{but not in} MSVC 2012}

\RU{Если говорить о PE-файлах, то в исполняемом файле значение}
\EN{If to say about PE-files, in the resulting executable file, the} \IT{tmp} 
\RU{будет именно в секции отведенной}\EN{variable will be stored in the section devoted to}
\ac{TLS}.
