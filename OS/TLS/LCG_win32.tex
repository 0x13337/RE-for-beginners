\subsection{Win32}

\subsubsection{\RU{Неинициализированные данные в \ac{TLS}}\EN{Uninitialized \ac{TLS} data}}

\RU{Одно из решений\EMDASH{}это добавить модификатор \TT{\_\_declspec( thread )} к глобальной переменной, 
и теперь она будет выделена в \ac{TLS} (строка 9):}
\EN{One solution is to add \TT{\_\_declspec( thread )} modifier to the global variable, 
then it will be allocated in the \ac{TLS} (line 9):}

\lstinputlisting[numbers=left]{OS/TLS/win32/rand_uninit.c}

\RU{Hiew показывает что в исполняемом файле теперь есть новая PE-секция:}
\EN{Hiew shows us that there is a new PE section in the executable file:} \TT{.tls}.
% TODO1 hiew screenshot?

\lstinputlisting[caption=\Optimizing MSVC 2013 x86]{OS/TLS/win32/rand_x86_uninit.asm}

\RU{\TT{rand\_state} теперь в \ac{TLS}-сегменте и у каждого потока есть своя версия этой переменной.}
\EN{\TT{rand\_state} is now in the \ac{TLS} segment, and each thread has its own version of this variable.}
\RU{Вот как к ней обращаться: загрузить адрес \ac{TIB} из FS:2Ch, затем прибавить дополнительный индекс 
(если нужно), затем вычислить адрес \ac{TLS}-сегмента.}
\EN{Here is how it's accessed: load the address of the \ac{TIB} from FS:2Ch, then add an additional index (if needed),
then calculate the address of the \ac{TLS} segment.}

\RU{Затем можно обращаться к переменной \TT{rand\_state} через регистр ECX, который указывает на свою
область в каждом потоке.}
\EN{Then it's possible to access the \TT{rand\_state} variable through the ECX register, which points to an unique area
in each thread.}

\index{x86!\Registers!FS}
\RU{Селектор \TT{FS:} знаком любому reverse engineer-у, он всегда указывает на \ac{TIB}, чтобы всегда можно было
загружать данные специфичные для текущего потока.}
\EN{The \TT{FS:} selector is familiar to every reverse engineer, it is specially used to always point to \ac{TIB},
so it would be fast to load the thread-specific data.}

\index{x86!\Registers!GS}
\RU{В Win64 используется селектор \TT{GS:} и адрес \ac{TLS} теперь 0x58:}
\EN{The \TT{GS:} selector is used in Win64 and the address of the \ac{TLS} is 0x58:}

\lstinputlisting[caption=\Optimizing MSVC 2013 x64]{OS/TLS/win32/rand_x64_uninit.asm}

\subsubsection{\RU{Инициализированные данные в \ac{TLS}}\EN{Initialized \ac{TLS} data}}

\RU{Скажем, мы хотим, чтобы в переменной \TT{rand\_state} в самом начале было какое-то значение, 
и если программист забудет инициализировать генератор, то \TT{rand\_state} все же будет инициализирована какой-то
константой (строка 9):}
\EN{Let's say, we want to set some fixed value to \TT{rand\_state}, so in case the programmer forgets to,
the \TT{rand\_state} variable would be initialized to some constant anyway (line 9):}

\lstinputlisting[numbers=left]{OS/TLS/win32/rand_init.c}

\RU{Код ничем не отличается от того, что мы уже видели, но вот что мы видим в IDA:}
\EN{The code is no differ from what we already saw, but in IDA we see:}

\lstinputlisting{OS/TLS/win32/rand_init_IDA.lst}

\RU{Там 1234 и теперь, во время запуска каждого нового потока, новый \ac{TLS} будет выделен для нового потока,
и все эти данные, включая 1234, будут туда скопированы.}
\EN{1234 is there and every time a new thread starts, a new \ac{TLS} is allocated for it, 
and all this data, including 1234, will be copied there.}

\RU{Вот типичный сценарий}\EN{This is a typical scenario}:

\begin{itemize}
\item \RU{Запустился поток А. \ac{TLS} создался для него, 1234 скопировалось в \TT{rand\_state}.}
\EN{Thread A is started. A \ac{TLS} is created for it, 1234 is copied to \TT{rand\_state}.}

\item \RU{Ф-ция \TT{my\_rand()} была вызвана несколько раз в потоке А. 
\TT{rand\_state} теперь содержит что-то неравное 1234.}
\EN{The \TT{my\_rand()} function is called several times in thread A. \TT{rand\_state} is different from 1234.}

\item 
\RU{Запустился поток Б. \ac{TLS} создался для него, 1234 скопировалось в \TT{rand\_state}, 
а в это же время, поток А имеет какое-то другое значение в этой переменной.}
\EN{Thread B is started. A \ac{TLS} is created for it, 1234 is copied to \TT{rand\_state}, 
while thread A has a different value in the same variable.}
\end{itemize}

\subsubsection{\RU{\ac{TLS}-коллбэки}\EN{\ac{TLS} callbacks}}
\index{TLS!\RU{Коллбэки}\EN{Callbacks}}

\RU{Но что если переменные в \ac{TLS} должны быть установлены в значения, которые должны быть подготовлены
каким-то необычным образом?}
\EN{But what if the variables in the \ac{TLS} have to be filled with some data that must be prepared in some unusual way?}
\RU{Скажем, у нас есть следующая задача:
программист может забыть вызвать функцию \TT{my\_srand()} для инициализации \ac{PRNG}, но генератор должен быть
инициализирован на старте чем-то по-настоящему случайным а не 1234.}
\EN{Let's say, we've got the following task:
the programmer can forget to call the \TT{my\_srand()} function to initialize the \ac{PRNG}, but the generator has to be 
initialized at start with something truly random, instead of 1234.}
\RU{Вот случай где можно применить \ac{TLS}-коллбэки}\EN{This is a case in which \ac{TLS} callbacks can be used}.

\RU{Нижеследующий код не очень портабельный из-за хака, но тем не менее, вы поймете идею.}
\EN{The following code is not very portable due to the hack, but nevertheless, you get the idea.}
\RU{Мы здесь добавляем функцию (\TT{tls\_callback()}), которая вызывается \IT{перед} стартом процесса и/или потока.}
\EN{What we do here is define a function (\TT{tls\_callback()}) which is to be called \IT{before} 
the process and/or thread start.}
\RU{Ф-ция будет инициализировать \ac{PRNG} значением возвращенным функцией \TT{GetTickCount()}.}
\EN{The function initializes the \ac{PRNG} with the value returned by \TT{GetTickCount()} function.}

\lstinputlisting{OS/TLS/win32/rand_TLS_callback.c}

\RU{Посмотрим в}\EN{Let's see it in} IDA:

\lstinputlisting[caption=\Optimizing MSVC 2013]{OS/TLS/win32/rand_TLS_callback.lst}

\RU{TLS-коллбэки иногда используются в процедурах распаковки для запутывания их работы.}
\EN{TLS callback functions are sometimes used in unpacking routines to obscure their processing.}
\RU{Некоторые люди могут быть в неведении что какой-то код уже был исполнен прямо перед \ac{OEP}.}
\EN{Some people may be confused and be in the dark that some code executed right before the \ac{OEP}.}
