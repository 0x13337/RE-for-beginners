\chapter{Thread Local Storage}
\label{TLS}
\index{TLS}

\RU{Это область данных, отдельная для каждого треда. Каждый тред может хранить там то, что ему нужно}
\EN{It is a data area, specific to each thread. Every thread can store there what it needs}.
\RU{Один из известных примеров, это стандартная глобальная переменная в Си}\EN{One famous example
is C standard global variable} \IT{errno}. 
\RU{Несколько тредов одновременно могут вызывать функции
возвращающие код ошибки в \IT{errno}, поэтому глобальная переменная здесь не будет работать корректно, 
для мультитредовых программ \IT{errno} нужно хранить в в \ac{TLS}.}
\EN{Multiple threads may simultaneously call a functions
which returns error code in the \IT{errno}, so global variable will not work correctly here, for multi-thread programs,
\IT{errno} must be stored in the \ac{TLS}.} \\
\\
\index{\Cpp!C++11}
\RU{В}\EN{In the} C++11 \RU{ввели модификатор}\EN{standard, a new} \IT{thread\_local} 
\RU{, показывающий что каждый тред будет иметь свою версию этой переменной}
\EN{modifier was added, showing that each thread will have its own version of the variable},
\RU{и её можно инициализировать, и она расположена в}\EN{it can be initialized, and it is located in the} \ac{TLS}
\footnote{
\index{C11}
\RU{В C11 также есть поддержка тредов, хотя и опциональная}
\EN{C11 also has thread support, optional though}}:

\begin{lstlisting}[caption=C++11]
#include <iostream>
#include <thread>

thread_local int tmp=3;

int main()
{
	std::cout << tmp << std::endl;
};
\end{lstlisting}

\RU{Компилируется в}\EN{Compiled in} MinGW GCC 4.8.1, \RU{но не в}\EN{but not in} MSVC 2012.

\RU{Если говорить о PE-файлах, то в исполняемом файле значение}
\EN{If to say about PE-files, in the resulting executable file, the} \IT{tmp} 
\RU{будет именно в секции отведенной}\EN{variable will be stored in the section devoted to}
\ac{TLS}.

\section{\RU{Вернемся к линейному конгруэнтному генератору}\EN{Linear congruential generator revisited}}
\label{LCG_TLS}

\RU{Рассмотренный раннее \myref{LCG_simple} генератор псевдослучайных чисел имеет недостаток:}
\EN{The pseudorandom number generator we considered earlier \myref{LCG_simple} has a flaw:}
\RU{он не пригоден для многопоточной среды, потому что переменная его внутреннего состояния может быть
прочитана и/или модифицирована в разных потоках одновременно.}
\EN{it's not thread-safe, because it has internal state variable which can be read and/or 
modified in different threads simultaneously.}

% subsections
\subsection{Win32}

\subsubsection{\RU{Неинициализированные данные в \ac{TLS}}\EN{Uninitialized \ac{TLS} data}}

\RU{Одно из решений --- это добавить модификатор \TT{\_\_declspec( thread )} к глобальной переменной, 
и теперь она будет выделена в \ac{TLS} (строка 9):}
\EN{One solution is to add \TT{\_\_declspec( thread )} modifier to the global variable, 
then it will be allocated in the \ac{TLS} (line 9):}

\lstinputlisting[numbers=left]{OS/TLS/win32/rand_uninit.c}

\RU{Hiew показывает что в исполняемом файле теперь есть новая PE-секция:}
\EN{Hiew shows us that there is a new PE section in the executable file:} \TT{.tls}.
% TODO hiew screenshot?

\lstinputlisting[caption=\Optimizing MSVC 2013 x86]{OS/TLS/win32/rand_x86_uninit.asm}

\RU{\TT{rand\_state} теперь в \ac{TLS}-сегменте и у каждого потока есть своя версия этой переменной.}
\EN{\TT{rand\_state} is now in the \ac{TLS} segment, and each thread has its own version of this variable.}
\RU{Вот как к ней обращаться: загрузить адрес \ac{TIB} из FS:2Ch, затем прибавить дополнительный индекс 
(если нужно), затем вычислить адрес \ac{TLS}-сегмента.}
\EN{Here is how it's accessed: load the address of the \ac{TIB} from FS:2Ch, then add an additional index (if needed),
then calculate the address of the \ac{TLS} segment.}

\RU{Затем можно обращаться к переменной \TT{rand\_state} через регистр ECX, который указывает на свою
область в каждом потоке.}
\EN{Then it's possible to access the \TT{rand\_state} variable through the ECX register, which points to an unique area
in each thread.}

\index{x86!\Registers!FS}
\RU{Селектор \TT{FS:} знаком любому reverse engineer-у, он всегда указывает на \ac{TIB}, чтобы всегда можно было
загружать данные специфичные для текущего потока.}
\EN{The \TT{FS:} selector is familiar to every reverse engineer, it is specially used to always point to \ac{TIB},
so it would be fast to load the thread-specific data.}

\index{x86!\Registers!GS}
\RU{В Win64 используется селектор \TT{GS:} и адрес \ac{TLS} теперь 0x58:}
\EN{The \TT{GS:} selector is used in Win64 and the address of the \ac{TLS} is 0x58:}

\lstinputlisting[caption=\Optimizing MSVC 2013 x64]{OS/TLS/win32/rand_x64_uninit.asm}

\subsubsection{\RU{Инициализированные данные в \ac{TLS}}\EN{Initialized \ac{TLS} data}}

\RU{Скажем, мы хотим, чтобы в переменной \TT{rand\_state} в самом начале было какое-то значение, 
и если программист забудет инициализировать генератор, то \TT{rand\_state} все же будет инициализирована какой-то
константой (строка 9):}
\EN{Let's say, we want to set some fixed value to \TT{rand\_state}, so in case the programmer forgets to,
the \TT{rand\_state} variable would be initialized to some constant anyway (line 9):}

\lstinputlisting[numbers=left]{OS/TLS/win32/rand_init.c}

\RU{Код ничем не отличается от того, что мы уже видели, но вот что мы видим в IDA:}
\EN{The code is no differ from what we already saw, but in IDA we see:}

\lstinputlisting{OS/TLS/win32/rand_init_IDA.lst}

\RU{Там 1234 и теперь, во время запуска каждого нового потока, новый \ac{TLS} будет выделен для нового потока,
и все эти данные, включая 1234, будут туда скопированы.}
\EN{1234 is there and every time a new thread starts, a new \ac{TLS} is allocated for it, 
and all this data, including 1234, will be copied there.}

\RU{Вот типичный сценарий}\EN{This is a typical scenario}:

\begin{itemize}
\item \RU{Запустился поток А. \ac{TLS} создался для него, 1234 скопировалось в \TT{rand\_state}.}
\EN{Thread A is started. A \ac{TLS} is created for it, 1234 is copied to \TT{rand\_state}.}

\item \RU{Ф-ция \TT{my\_rand()} была вызвана несколько раз в потоке А. 
\TT{rand\_state} теперь содержит что-то неравное 1234.}
\EN{The \TT{my\_rand()} function is called several times in thread A. \TT{rand\_state} is different from 1234.}

\item 
\RU{Запустился поток Б. \ac{TLS} создался для него, 1234 скопировалось в \TT{rand\_state}, 
а в это же время, поток А имеет какое-то другое значение в этой переменной.}
\EN{Thread B is started. A \ac{TLS} is created for it, 1234 is copied to \TT{rand\_state}, 
while thread A has a different value in the same variable.}
\end{itemize}

\subsubsection{\RU{\ac{TLS}-коллбэки}\EN{\ac{TLS} callbacks}}
\index{TLS!\RU{Коллбэки}\EN{Callbacks}}

\RU{Но что если переменные в \ac{TLS} должны быть установлены в значения, которые должны быть подготовлены
каким-то необычным образом?}
\EN{But what if the variables in the \ac{TLS} have to be filled with some data that must be prepared in some unusual way?}
\RU{Скажем, у нас есть следующая задача:
программист может забыть вызвать ф-цию \TT{my\_srand()} для инициализации \ac{PRNG}, но генератор должен быть
инициализирован на старте чем-то по-настоящему случайным а не 1234.}
\EN{Let's say, we've got the following task:
the programmer can forget to call the \TT{my\_srand()} function to initialize the \ac{PRNG}, but the generator has to be 
initialized at start with something truly random, instead of 1234.}
\RU{Вот случай где можно применить \ac{TLS}-коллбэки}\EN{This is a case in which \ac{TLS} callbacks can be used}.

\RU{Нижеследующий код не очень портабельный из-за хака, но тем не менее, вы поймете идею.}
\EN{The following code is not very portable due to the hack, but nevertheless, you get the idea.}
\RU{Мы здесь добавляем ф-цию (\TT{tls\_callback()}), которая вызывается \IT{перед} стартом процесса и/или потока.}
\EN{What we do here is define a function (\TT{tls\_callback()}) which is to be called \IT{before} 
the process and/or thread start.}
\RU{Ф-ция будет инициализировать \ac{PRNG} значением возвращенным ф-цией \TT{GetTickCount()}.}
\EN{The function initializes the \ac{PRNG} with the value returned by \TT{GetTickCount()} function.}

\lstinputlisting{OS/TLS/win32/rand_TLS_callback.c}

\RU{Посмотрим в}\EN{Let's see it in} IDA:

\lstinputlisting[caption=\Optimizing MSVC 2013]{OS/TLS/win32/rand_TLS_callback.lst}

\RU{TLS-коллбэки иногда используются в процедурах распаковки для запутывания их работы.}
\EN{TLS callback functions are sometimes used in unpacking routines to obscure their processing.}
\RU{Некоторые люди могут быть в неведении что какой-то код уже был исполнен прямо перед \ac{OEP}.}
\EN{Some people may be confused and be in the dark that some code executed right before the \ac{OEP}.}

\subsection{Linux}

\RU{Вот как глобальная переменная локальная для потока определяется в GCC:}
\EN{Here is how thread-local global variable declared in GCC:}

\begin{lstlisting}
__thread uint32_t rand_state=1234;
\end{lstlisting}

\RU{Этот модификатор не стандартный для \CCpp, он присутствует только в GCC}
\EN{This is not standard \CCpp modifier, but rather GCC-specific}
\footnote{\url{http://go.yurichev.com/17062}}.

\index{x86!\Registers!GS}
\RU{Селектор \TT{GS:} также используется для доступа к \ac{TLS}, но немного иначе:}
\EN{\TT{GS:} selector is also used to \ac{TLS} access, but in some different way:}

\lstinputlisting[caption=\Optimizing GCC 4.8.1 x86]{OS/TLS/linux/rand.lst}

% ??? Uninitialized data is allocated in \TT{.tbss} section, initialized --- in \TT{.tdata} section.

\RU{Еще об этом}\EN{More about it}: \cite{DrepperTLS}.

