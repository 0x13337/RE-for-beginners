\chapter{\RU{Способы передачи аргументов при вызове функций}\EN{Arguments passing methods (calling conventions)}}
\label{sec:callingconventions}

\section{cdecl}
\index{cdecl}
\label{cdecl}

\RU{Этот способ передачи аргументов через стек чаще всего используется в языках \CCpp.}
\EN{This is the most popular method for arguments passing to functions in \CCpp languages.}

\RU{Вызывающая функция заталкивает в стек аргументы в обратном порядке: сначала последний аргумент в стек, 
затем предпоследний, и в самом конце ~--- первый аргумент. 
Вызывающая функция должна также затем вернуть \glslink{stack pointer}{указатель стека} в нормальное состояние, 
после возврата вызываемой функции.}
\EN{\Gls{caller} pushing arguments to stack in reverse order: last argument, then penultimate element 
and finally~---first argument.
\Gls{caller} also must return back value of the \gls{stack pointer} (\ESP) to its initial state after \gls{callee} function exit.}

\begin{lstlisting}[caption=cdecl]
push arg3
push arg2
push arg1
call function
add esp, 12 ; returns ESP
\end{lstlisting}

\section{stdcall}
\label{sec:stdcall}
\index{stdcall}

\newcommand{\SIZEOFINT}{\RU{Размер переменной типа \Tint ~--- 4 в x86-системах и 8 в x64-системах}
\EN{Size of \Tint type variable is 4 in x86 systems and 8 in x64 systems}}

\RU{Это почти то же что и \IT{cdecl}, за исключением того, что вызываемая функция сама возвращает \ESP 
в нормальное состояние, выполнив инструкцию \TT{RET x} вместо \RET, где 
\TT{x = количество\_аргументов * sizeof(int)\footnote{\SIZEOFINT}}.
Вызывающая функция не будет корректировать \glslink{stack pointer}{указатель стека} при помощи инструкции \TT{add esp, x}.}
\EN{Almost the same thing as \IT{cdecl}, with the exception the \gls{callee} set \ESP to initial state executing \TT{RET x} instruction instead of \RET, where
\TT{x = arguments number * sizeof(int)\footnote{\SIZEOFINT}}.
\Gls{caller} will not adjust \gls{stack pointer} by \TT{add esp, x} instruction.}

\begin{lstlisting}[caption=stdcall]
push arg3
push arg2
push arg1
call function

function:
... do something ...
ret 12
\end{lstlisting}

\RU{Этот способ используется почти везде в системных библиотеках win32, 
но не в win64 (о win64 смотрите ниже).}
\EN{The method is ubiquitous in win32 standard libraries, but not in win64 (see below about win64).} \\
\\
\RU{Например, мы можем взять ф-цию из}\EN{For example, we may take the function from} 
\ref{src:passing_arguments_ex} \RU{и изменить её немного добавив модификатор}\EN{and change it
slightly by adding} \TT{\_\_stdcall}\EN{ modifier}:

\begin{lstlisting}
int __stdcall f2 (int a, int b, int c)
{
	return a*b+c;
};
\end{lstlisting}

\RU{Он будет скомпилирован почти так же как и}\EN{It will be compiled in almost the same way as} 
\ref{src:passing_arguments_ex_MSVC_cdecl},
\RU{но вы увидите}\EN{but you will see} \TT{RET 12} \RU{вместо}\EN{instead of} \TT{RET}. 
\ac{SP} \RU{не будет корректироваться в \glslink{caller}{вызывающей ф-ции}}
\EN{is not aligned in \gls{caller}}.

\RU{Как следствие, количество аргументов ф-ции легко узнать из инструкции}\EN{As a consequence, 
number of function arguments can be easily deduced from} \TT{RETN n} \RU{просто разделите
$n$ на 4}\EN{instruction: just divide $n$ by 4}.

\lstinputlisting[caption=MSVC 2010]{OS/calling_conventions/stdcall_ex.asm}

\subsection{\RU{Функции с переменным количеством аргументов}\EN{Variable arguments number functions}}

\RU{Функции вроде \printf, должно быть, единственный случай функций в \CCpp с переменным количеством аргументов,
но с их помощью можно легко проследить очень важную разницу между \IT{cdecl} и \IT{stdcall}.
Начнем с того, что компилятор знает сколько аргументов было у \printf.}
\EN{\printf-like functions are, probably, the only case of variable arguments functions in \CCpp,
but it is easy to illustrate an important difference between \IT{cdecl} and \IT{stdcall} with the help of it.
Let's start with the idea the compiler knows argument count of each \printf function calling.}
\RU{Однако, вызываемая функция \printf, которая уже давно скомпилирована 
и находится в системной библиотеке MSVCRT.DLL (если говорить о Windows), 
не знает сколько аргументов ей передали, хотя может установить их количество по строке формата.}
\EN{However, called \printf, which is already compiled and located in MSVCRT.DLL (if to talk about Windows),
has not any information about how much arguments were passed, however it can determine it from format string.}
\RU{Таким образом, если бы \printf была \IT{stdcall}-функцией и возвращала
\glslink{stack pointer}{указатель стека}
в первоначальное состояние 
подсчитав количество аргументов в строке формата, это была бы потенциально опасная ситуация, 
когда одна опечатка программиста могла бы вызывать неожиданные падения программы. 
Таким образом, для таких функций \IT{stdcall} явно не подходит, а подходит \IT{cdecl}.}
\EN{Thus, if \printf would be \IT{stdcall}-function and restored \gls{stack pointer} to its initial state by counting
number of arguments in format string, this could be dangerous situation, when one programmer's typo may
provoke sudden program crash.
Thus it is not suitable for such functions to use \IT{stdcall}, \IT{cdecl} is better.}

\section{fastcall}
\label{fastcall}
\index{fastcall}

\RU{Это общее название для передачи некоторых аргументов через регистры, а всех остальных ~--- через стек.На более старых процессорах, это работало потенциально быстрее чем \IT{cdecl}/\IT{stdcall} (ведь стек
в памяти использовался меньше).
Впрочем, на современных, намного более сложных CPU, выигрыша может не быть.}
\EN{That's general naming for a method of passing some of arguments via registers and all 
others~---via stack. It worked faster than \IT{cdecl}/\IT{stdcall} on older CPUs 
(because of smaller stack pressure).
It will not help to gain performance on modern much more complex CPUs, however.}

\RU{Это не стандартизированный способ, поэтому разные компиляторы делают это по-своему. 
Разумеется, если у вас есть, скажем, две DLL, одна использует другую, и обе они собраны с \IT{fastcall}
но разными компиляторами, очень вероятно, будут проблемы.}
\EN{it is not a standardized way, so, various compilers may do it differently.
Well known caveat: if you have two DLLs, one uses another, and they are built by different compilers with 
different \IT{fastcall} calling conventions.}

\RU{MSVC и GCC передает первый и второй аргумент через \ECX и \EDX а остальные аргументы через стек.}
\EN{Both MSVC and GCC passing first and second argument via \ECX and \EDX and other arguments via stack.}

\RU{\glslink{stack pointer}{Указатель стека} должен быть возвращен в первоначальное состояние вызываемой функцией, 
как в случае \IT{stdcall}.}
\EN{\Gls{stack pointer} must be restored to initial state by \gls{callee} (like in \IT{stdcall}).}

\begin{lstlisting}[caption=fastcall]
push arg3
mov edx, arg2
mov ecx, arg1
call function

function:
.. do something ..
ret 4
\end{lstlisting}

\RU{Например, мы можем взять ф-цию из}\EN{For example, we may take the function from} 
\ref{src:passing_arguments_ex} \RU{и изменить её немного добавив модификатор}\EN{and change it
slightly by adding} \TT{\_\_fastcall}\EN{ modifier}:

\begin{lstlisting}
int __fastcall f3 (int a, int b, int c)
{
	return a*b+c;
};
\end{lstlisting}

\RU{Вот как он будет скомпилирован}\EN{Here is how it will be compiled}:

\lstinputlisting[caption=\Optimizing MSVC 2010 /Ob0]{OS/calling_conventions/fastcall_ex.asm}

\RU{Видно, что \glslink{callee}{вызываемая ф-ция} сама возвращает}
\EN{We see that a \gls{callee} returns} \ac{SP} 
\RU{при помощи инструкции}\EN{by} \TT{RETN} \RU{с операндом}\EN{instruction with operand}.
\RU{Так что и здесь можно легко вычислять количество аргументов}\EN{Which means, number of arguments
can be deduced easily here as well}.

\subsection{GCC regparm}

\newcommand{\URLREGPARMM}{\url{http://go.yurichev.com/17040}}

\RU{Это в некотором роде, развитие \IT{fastcall}\footnote{\URLREGPARMM}. 
Опцией \TT{-mregparm=x} можно указывать, 
сколько аргументов компилятор будет передавать через регистры. Максимально 3. 
В этом случае будут задействованы регистры \EAX, \EDX и \ECX.}
\EN{It is \IT{fastcall} evolution\footnote{\URLREGPARMM} is some sense.
With the \TT{-mregparm} option it is possible to set, how many arguments will be passed via registers. 
3 at maximum.
Thus, \EAX, \EDX and \ECX registers are to be used.}

\RU{Разумеется, если аргументов у функции меньше трех, то будет задействована только часть регистров.}
\EN{Of course, if number of arguments is less then 3, not all 3 registers are to be used.}

\RU{Вызывающая функция возвращает \glslink{stack pointer}{указатель стека} в первоначальное состояние.}
\EN{\Gls{caller} restores \gls{stack pointer} to its initial state.}

\RU{Для примера, см.}\EN{For the example, see} (\ref{regparm}).

\subsection{Watcom/OpenWatcom}
\index{OpenWatcom}

\RU{Здесь это называется}\EN{It is called} ``register calling convention''\EN{ here}.
\RU{Первые 4 аргумента передаются через регистры}\EN{First 4 arguments are passed via}
\EAX, \EDX, \EBX \AndENRU \ECX\EN{ registers}.
\RU{Все остальные}\EN{All the rest}\EMDASH{}\RU{через стек}\EN{via stack}.
\RU{Функции имеют символ подчеркивания добавленный к концу имени ф-ции, для отличия их от тех,
которые имеют другой способ передачи аргументов}
\EN{Functions have underscore added to the function name in order to distinguish them from 
those having other calling convention}.

\section{thiscall}
\index{thiscall}

\RU{В \Cpp, это передача в функцию-метод указателя \ITthis на объект.}
\EN{In \Cpp, it is a \ITthis pointer to object passing into function-method.}

\RU{В MSVC указатель \ITthis обычно передается в регистре \ECX.}
\EN{In MSVC, \ITthis is usually passed in the \ECX register.}

\RU{В GCC указатель \ITthis обычно передается как самый первый аргумент. 
Таким образом, внутри будет видно: у всех функций-методов на один аргумент больше.}
\EN{In GCC, \ITthis pointer is passed as a first function-method argument.
Thus it will be seen: internally, all function-methods has extra argument.}

\RU{Для примера, см.}\EN{For the example, see} (\ref{thiscall}).

\section{x86-64}
\index{x86-64}

\subsection{Windows x64}
\label{sec:callingconventions_win64}

\RU{В win64 метод передачи всех параметров немного похож на \TT{fastcall}. 
Первые 4 аргумента записываются в регистры \RCX, \RDX, \Reg{8}, \Reg{9}, а остальные ~--- в стек. 
Вызывающая функция также должна подготовить место из 32 байт или для четырех 64-битных значений, 
чтобы вызываемая функция могла сохранить там первые 4 аргумента. 
Короткие функции могут использовать переменные прямо из регистров, 
но б\'{о}льшие могут сохранять их значения на будущее.}
\EN{The method of arguments passing in Win64 is somewhat resembling to \TT{fastcall}.
First 4 arguments are passed via \RCX, \RDX, \Reg{8}, \Reg{9}, other~---via stack.
\Gls{caller} also must prepare a space for 32 bytes or 4 64-bit values,
so then \gls{callee} can save there first 4 arguments.
Short functions may use argument values just from registers,
but larger may save its values for further use.}

\RU{Вызывающая функция должна вернуть \glslink{stack pointer}{указатель стека} 
в первоначальное состояние}
\EN{\Gls{caller} also must return \gls{stack pointer} into initial state}.

\RU{Это же соглашение используется и в системных библиотеках Windows x86-64 
(вместо \IT{stdcall} в win32).}
\EN{This calling convention is also used in Windows x86-64 system DLLs 
(instead of \IT{stdcall} in win32).}

\RU{Пример}\EN{Example}:

\lstinputlisting{OS/calling_conventions/x64.c}

\lstinputlisting[caption=MSVC 2012 /0b]{OS/calling_conventions/x64_MSVC_Ob.asm}

\index{Scratch space}
\RU{Здесь мы легко видим, как 7 аргументов передаются: 4 через регистры и остальные 3 через стек}
\EN{Here we clearly see how 7 arguments are passed: 4 via registers and the rest 3 via stack}.
\RU{Код пролога ф-ции f1() сохраняет аргументы в ``scratch space'' --- место в стеке предназначенное
именно для этого}
\EN{The code of f1() function's prologue saves the arguments in ``scratch space''---a space in the stack
intended exactly for the purpose}.
\RU{Это делается потому что компилятор может быть не уверен, достаточно ли ему будет остальных регистров
для работы исключая эти 4, которые иначе будут заняты аргументами до конца исполнения ф-ции}
\EN{It is done because compiler may not be sure if it will be enough to use other registers without these 4,
which will otherwise be occupied by arguments until function execution end}.
\RU{Выделение ``scratch space'' в стеке лежит на ответственности вызывающей ф-ции.}
\EN{The ``scratch space'' allocation in the stack is the caller's duty.}

\lstinputlisting[caption=\Optimizing MSVC 2012 /0b]{OS/calling_conventions/x64_MSVC_Ox_Ob.asm}

\RU{Если компилировать этот пример с оптимизацией, то выйдет почти то же самое, только ``scratch space''
не используется, потому что незачем}
\EN{If to compile the example with optimization switch, it is almost the same, but ``scratch space''
is not used, because no need to}.

\index{x86!\Instructions!LEA}
\label{using_MOV_and_pack_of_LEA_to_load_values}
\RU{Обратите также внимание на то как MSVC 2012 оптимизирует примитивную загрузку значений в регистры
используя \LEA (\ref{sec:LEA})}
\EN{Also take a look on how MSVC 2012 optimizes primitive value loads into registers by using 
\LEA (\ref{sec:LEA})}.
\RU{Я не уверен, что это того стоит, но может быть}\EN{I'm not sure if it worth so, but maybe}.

\RU{Еще один пример подобного}\EN{Another example of such thing is}: \ref{TaskMgr_LEA}.

\subsubsection{\RU{Передача \ITthis}\EN{\ITthis passing} (\CCpp)}

\RU{Указатель }\ITthis \RU{передается через}\EN{pointer is passed in} \RCX, 
\RU{первый аргумент метода через}\EN{first method argument in} \RDX, \RU{и т.д.}\EN{etc.}
\RU{Для примера, см. также}\EN{See also for an example}: \ref{simple_CPP_MSVC_x64}.
 
\subsection{Linux x64}

\RU{Метод передачи аргументов в Linux для x86-64 почти такой же, как и в Windows, но 6 регистров
используется вместо 4 (\RDI, \RSI, \RDX, \RCX, \Reg{8}, \Reg{9}), и здесь нет ``scratch space'', 
но \gls{callee} может сохранять значения регистров в стеке, если ему это нужно.}
\EN{The way arguments passed in Linux for x86-64 is almost the same as in Windows, but 6 registers are
used instead of 4 (\RDI, \RSI, \RDX, \RCX, \Reg{8}, \Reg{9}) and there are no ``scratch space'', 
although \gls{callee} may save register values in the stack, if it needs/wants to.}

\lstinputlisting[caption=\Optimizing GCC 4.7.3]{OS/calling_conventions/x64_linux_O3.s}

\RU{N.B.: здесь значения записываются в 32-битные части регистров (например EAX) а не в весь 64-битный
регистр (RAX).
Это связано с тем что в x86-64,
запись в младшую 32-битную часть 64-битного регистра автоматически обнуляет старшие 32 бита.
Вероятно, это сделано для упрощения портирования кода под x86-64.}
\EN{N.B.: here values are written into 32-bit parts of registers (e.g., EAX) but not to the whole 64-bit 
register (RAX).
This is because each write to low 32-bit part of register automatically clears high 32 bits.
Supposedly, it was done for x86-64 code porting simplification.}

\section{\RU{Возвращение переменных типа \Tfloat, \Tdouble}\EN{Returning values of \Tfloat and \Tdouble type}}
\index{float}
\index{double}

\RU{Во всех соглашениях кроме Win64, переменная типа \Tfloat или \Tdouble возвращается через регистр FPU \ST{0}.}
\EN{In all conventions except of Win64, values of type \Tfloat or \Tdouble are returning via the FPU register \ST{0}.}

\RU{В Win64 переменные типа \Tfloat и \Tdouble возвращаются в младших 16-и или 32-х битах 
регистра \XMM{0}.}
\EN{In Win64, values of \Tfloat and \Tdouble types are returned 
in the low 32 or 64 bits of \XMM{0} register.}

\section{\RU{Модификация аргументов}\EN{Modifying arguments}}

\RU{Иногда программисты на \CCpp{} (и не только этих \ac{PL}) задаются вопросом,
что будет если модифицировать аргументы?}
\EN{Sometimes, \CCpp{} programmers (not limited to these \ac{PL}, though),
may ask, what will happen if to modify arguments?}
\RU{Ответ прост: аргументы хранятся в стеке, именно там и будет происходит модификация.}
\EN{The answer is simple: arguments are stored in the stack, that is where modification will occurr.}
\RU{А вызывающие ф-ции не использует их после вызова ф-ции (обратного случая в своей практике я не видел
ни разу).}
\EN{Calling functions are not use them after \gls{callee} exit (I have not seen any opposite case in my practice).}

\lstinputlisting{OS/calling_conventions/change_arguments.c}

\lstinputlisting[caption=MSVC 2012]{OS/calling_conventions/change_arguments.asm}

% TODO (OllyDbg) пример как в стеке меняется $a$

\RU{Следовательно, модифицировать аргументы ф-ции можно запросто.}\EN{So yes, one may modify arguments easily.}
\RU{Разумеется, если это не}\EN{Of course, if it is not} \IT{references} \InENRU \Cpp{} (\ref{cpp_references}),
\RU{и если вы не модифицируете данные по указателю}\EN{and if you not modify data a pointer pointing to} 
(\RU{тогда эффект распространится не только на текущую ф-цию}
\EN{then the effect will be propagated outside of current function}).

\RU{Теоритически, после возврата из \gls{callee},
функция-\gls{caller} могла бы получить модифицированный аргумент и использовать его как-то.
Может быть, если бы она была написана на языке ассемблера.
Но языки \CCpp не предлагают никакого способа доступиться к ним.}
\EN{Theoretically, after \gls{callee} return, 
\gls{caller}-function could get modified argument and use it somehow.
Maybe if it will be written in assembly language.
But \CCpp languages doesn't offer any way to access them.}

% sections
\section{\RU{Указатель на аргумент функции}\EN{Taking a pointer to function argument}}
\label{pointer_to_argument}

\dots \RU{и даже более того, можно взять указатель на аргумент функции и передать его в другую функцию:}
\EN{even more than that, it's possible to take a pointer to the function's argument and pass
it to another function:}

\lstinputlisting{OS/calling_conventions/ptr_to_argument/10.c}

\RU{Трудно понять, как это работает, пока мы не посмотрим на код:}
\EN{It's hard to understand how it works until we can see the code:}

\lstinputlisting[caption=\Optimizing MSVC 2010]{OS/calling_conventions/ptr_to_argument/MSVC_2010_O3.asm}

\RU{Адрес места в стеке где была передана $a$ просто передается в другую функцию.
Она модифицирует переменную по этому адресу, и затем \printf выведет модифицированное значение.}
\EN{The address of the place in the stack where $a$ was passed is just passed to another function.
It modifies the value addressed by the pointer and then \printf prints the modified value.}

\par \RU{Наблюдательный читатель может спросить, а что насчет тех соглашений о вызовах, где аргументы функции
передаются в регистрах?}
\EN{The observant reader might ask, what about calling conventions where the function's arguments are
passed in registers?}

\RU{Это та ситуация где используется \IT{Shadow Space}.}
\EN{That's a situation where the \IT{Shadow Space} is used.}
\RU{Так что входящее значение копируется из регистра в \IT{Shadow Space} в локальном стеке и затем это адрес
передается в другую функцию:}
\EN{The input value is copied from the register
to the \IT{Shadow Space} in the local stack, and then this address is passed to the other function:}

\lstinputlisting[caption=\Optimizing MSVC 2012 x64]{OS/calling_conventions/ptr_to_argument/MSVC_2012_x64_O3.asm}

\RU{GCC также записывает входное значение в локальный стек:}
\EN{GCC also stores the input value in the local stack:}

\lstinputlisting[caption=\Optimizing GCC 4.9.1 x64]{OS/calling_conventions/ptr_to_argument/GCC491_x64_O3.s}

\RU{GCC для ARM64 делает то же самое, но это пространство здесь называется \IT{Register Save Area}:}
\EN{GCC for ARM64 does the same, but this space is called \IT{Register Save Area} here:}

\lstinputlisting[caption=\Optimizing GCC 4.9.1 ARM64]{OS/calling_conventions/ptr_to_argument/GCC49_ARM64_O3.s}

\EN{By the way, a similar usage of the \IT{Shadow Space} is also considered here}
\RU{Кстати, похожее использование \IT{Shadow Space} разбирается здесь}: \myref{variadic_arith_registers}.

