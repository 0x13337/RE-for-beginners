\chapter{\RU{Способы передачи аргументов при вызове функций}\EN{Arguments passing methods (calling conventions)}}
\label{sec:callingconventions}

\section{cdecl}
\index{cdecl}
\label{cdecl}

\RU{Этот способ передачи аргументов через стек чаще всего используется в языках \CCpp.}
\EN{This is the most popular method for passing arguments to functions in the \CCpp languages.}

\RU{Вызывающая функция заталкивает в стек аргументы в обратном порядке: сначала последний аргумент в стек, 
затем предпоследний, и в самом конце\EMDASH{}первый аргумент. 
Вызывающая функция должна также затем вернуть \glslink{stack pointer}{указатель стека} в нормальное состояние, 
после возврата вызываемой функции.}
\EN{The gls{caller} also must return the value of the \gls{stack pointer} (\ESP) to its initial state after the \gls{callee} function exits.}

\begin{lstlisting}[caption=cdecl]
push arg3
push arg2
push arg1
call function
add esp, 12 ; returns ESP
\end{lstlisting}

\section{stdcall}
\label{sec:stdcall}
\index{stdcall}

\newcommand{\SIZEOFINT}{\RU{Размер переменной типа \Tint\EMDASH{}4 в x86-системах и 8 в x64-системах}
\EN{The size of an \Tint type variable is 4 in x86 systems and 8 in x64 systems}}

\RU{Это почти то же что и \IT{cdecl}, за исключением того, что вызываемая функция сама возвращает \ESP 
в нормальное состояние, выполнив инструкцию \TT{RET x} вместо \RET, где 
\TT{x = количество\_аргументов * sizeof(int)\footnote{\SIZEOFINT}}.
Вызывающая функция не будет корректировать \glslink{stack pointer}{указатель стека},
там нет инструкции \TT{add esp, x}.}
\EN{It's almost the same as \IT{cdecl}, with the exception that the \gls{callee} must set \ESP to the initial state by executing the \TT{RET x} instruction instead of \RET, where
\TT{x = arguments number * sizeof(int)\footnote{\SIZEOFINT}}.
The \gls{caller} is not adjusting the \gls{stack pointer}, 
there are no \TT{add esp, x} instruction.}

\begin{lstlisting}[caption=stdcall]
push arg3
push arg2
push arg1
call function

function:
... do something ...
ret 12
\end{lstlisting}

\RU{Этот способ используется почти везде в системных библиотеках win32, 
но не в win64 (о win64 смотрите ниже).}
\EN{The method is ubiquitous in win32 standard libraries, but not in win64 (see below about win64).}

\par \RU{Например, мы можем взять функцию из}\EN{For example, we can take the function from} 
\myref{src:passing_arguments_ex} \RU{и изменить её немного добавив модификатор}\EN{and change it
slightly by adding the} \TT{\_\_stdcall}\EN{ modifier}:

\begin{lstlisting}
int __stdcall f2 (int a, int b, int c)
{
	return a*b+c;
};
\end{lstlisting}

\RU{Он будет скомпилирован почти так же как и}
\EN{It is to be compiled in almost the same way as} \myref{src:passing_arguments_ex_MSVC_cdecl},
\RU{но вы увидите}\EN{but you will see} \TT{RET 12} \RU{вместо}\EN{instead of} \TT{RET}. 
\ac{SP} \RU{не будет корректироваться в \glslink{caller}{вызывающей функции}}
\EN{is not update in the \gls{caller}}.

\RU{Как следствие, количество аргументов функции легко узнать из инструкции}\EN{As a consequence, 
the number of function arguments can be easily deduced from the} \TT{RETN n} \RU{просто разделите
$n$ на 4}\EN{instruction: just divide $n$ by 4}.

\lstinputlisting[caption=MSVC 2010]{OS/calling_conventions/stdcall_ex.asm}

\subsection{\RU{Функции с переменным количеством аргументов}\EN{Functions with variable number of arguments}}

\RU{Функции вроде \printf, должно быть, единственный случай функций в \CCpp с переменным количеством аргументов,
но с их помощью можно легко проследить очень важную разницу между \IT{cdecl} и \IT{stdcall}.
Начнем с того, что компилятор знает сколько аргументов было у \printf.}
\EN{\printf-like functions are, probably, the only case of functions with a variable number of arguments in \CCpp,
but it is easy to illustrate an important difference between \IT{cdecl} and \IT{stdcall} with their help.
Let's start with the idea that the compiler knows the argument count of each \printf function call.}
\RU{Однако, вызываемая функция \printf, которая уже давно скомпилирована 
и находится в системной библиотеке MSVCRT.DLL (если говорить о Windows), 
не знает сколько аргументов ей передали, хотя может установить их количество по строке формата.}
\EN{However, the called \printf, which is already compiled and located in MSVCRT.DLL (if we talk about Windows),
does not have any information about how much arguments were passed, however it can determine it from the format string.}
\RU{Таким образом, если бы \printf была \IT{stdcall}-функцией и возвращала
\glslink{stack pointer}{указатель стека}
в первоначальное состояние 
подсчитав количество аргументов в строке формата, это была бы потенциально опасная ситуация, 
когда одна опечатка программиста могла бы вызывать неожиданные падения программы. 
Таким образом, для таких функций \IT{stdcall} явно не подходит, а подходит \IT{cdecl}.}
\EN{Thus, if \printf would be a \IT{stdcall} function and restored \gls{stack pointer} to its initial state by counting
the number of arguments in the format string, this could be a dangerous situation, when one programmer's typo can
provoke a sudden program crash.
Thus it is not suitable for such functions to use \IT{stdcall}, \IT{cdecl} is better.}

\section{fastcall}
\label{fastcall}
\index{fastcall}

\RU{Это общее название для передачи некоторых аргументов через регистры, а всех остальных\EMDASH{}через стек.
На более старых процессорах, это работало потенциально быстрее чем \IT{cdecl}/\IT{stdcall} (ведь стек в памяти использовался меньше).
Впрочем, на современных (намного более сложных) CPU, существенного выигрыша может и не быть.}
\EN{That's the general naming for the method of passing some arguments via registers and the 
rest via the stack. It worked faster than \IT{cdecl}/\IT{stdcall} on older CPUs 
(because of smaller stack pressure).
It may not help to gain any significant performance on modern (much more complex) CPUs, however.}

\RU{Это не стандартизированный способ, поэтому разные компиляторы делают это по-своему. 
Разумеется, если у вас есть, скажем, две DLL, одна использует другую, и обе они собраны с \IT{fastcall}
но разными компиляторами, очень вероятно, будут проблемы.}
\EN{It is not standardized, so the various compilers can do it differently.
It's a well known caveat: if you have two DLLs and the one uses another one, and they are built by different compilers with 
different \IT{fastcall} calling conventions, you can expect problems.}

\RU{MSVC и GCC передает первый и второй аргумент через \ECX и \EDX а остальные аргументы через стек.}
\EN{Both MSVC and GCC pass the first and second arguments via \ECX and \EDX and the rest of the arguments via the stack.}

\RU{\glslink{stack pointer}{Указатель стека} должен быть возвращен в первоначальное состояние вызываемой функцией, 
как в случае \IT{stdcall}.}
\EN{The \gls{stack pointer} must be restored to its initial state by the \gls{callee} (like in \IT{stdcall}).}

\begin{lstlisting}[caption=fastcall]
push arg3
mov edx, arg2
mov ecx, arg1
call function

function:
.. do something ..
ret 4
\end{lstlisting}

\RU{Например, мы можем взять функцию из}\EN{For example, we may take the function from} 
\myref{src:passing_arguments_ex} \RU{и изменить её немного добавив модификатор}\EN{and change it
slightly by adding a} \TT{\_\_fastcall}\EN{ modifier}:

\begin{lstlisting}
int __fastcall f3 (int a, int b, int c)
{
	return a*b+c;
};
\end{lstlisting}

\RU{Вот как он будет скомпилирован}\EN{Here is how it is to be compiled}:

\lstinputlisting[caption=\Optimizing MSVC 2010 /Ob0]{OS/calling_conventions/fastcall_ex.asm}

\RU{Видно, что \glslink{callee}{вызываемая функция} сама возвращает}
\EN{We see that the \gls{callee} returns} \ac{SP} 
\RU{при помощи инструкции}\EN{by using the} \TT{RETN} \RU{с операндом}\EN{instruction with an operand}.
\RU{Так что и здесь можно легко вычислять количество аргументов.}
\EN{Which implies that the number of arguments can be deduced easily here as well.}

\subsection{GCC regparm}

\newcommand{\URLREGPARMM}{\url{http://go.yurichev.com/17040}}

\RU{Это в некотором роде, развитие \IT{fastcall}\footnote{\URLREGPARMM}. 
Опцией \TT{-mregparm=x} можно указывать, 
сколько аргументов компилятор будет передавать через регистры. Максимально 3. 
В этом случае будут задействованы регистры \EAX, \EDX и \ECX.}
\EN{It is the evolution of \IT{fastcall}\footnote{\URLREGPARMM} in some sense.
With the \TT{-mregparm} option it is possible to set how many arguments are to be passed via registers 
(3 is the maximum).
Thus, the \EAX, \EDX and \ECX registers are to be used.}

\RU{Разумеется, если аргументов у функции меньше трех, то будет задействована только часть регистров.}
\EN{Of course, if the number the of arguments is less than 3, not all 3 registers are to be used.}

\RU{Вызывающая функция возвращает \glslink{stack pointer}{указатель стека} в первоначальное состояние.}
\EN{The \gls{caller} restores the \gls{stack pointer} to its initial state.}

\RU{Для примера, см.}\EN{For example, see} (\myref{regparm}).

\subsection{Watcom/OpenWatcom}
\index{OpenWatcom}

\RU{Здесь это называется}\EN{Here it is called} \q{register calling convention}.
\RU{Первые 4 аргумента передаются через регистры}\EN{The first 4 arguments are passed via the}
\EAX, \EDX, \EBX \AndENRU \ECX\EN{ registers}.
\RU{Все остальные}\EN{All the rest}\EMDASH{}\RU{через стек}\EN{via the stack}.
\RU{Эти функции имеют символ подчеркивания, добавленный к концу имени функции, для отличия их от тех,
которые имеют другой способ передачи аргументов}
\EN{These functions has an underscore appended to the function name in order to distinguish them from 
those having a different calling convention}.

\section{thiscall}
\index{thiscall}

\RU{В \Cpp, это передача в функцию-метод указателя \ITthis на объект.}
\EN{This is passing the object's \ITthis pointer to the function-method, in \Cpp.}

\RU{В MSVC указатель \ITthis обычно передается в регистре \ECX.}
\EN{In MSVC, \ITthis is usually passed in the \ECX register.}

\RU{В GCC указатель \ITthis обычно передается как самый первый аргумент. 
Таким образом, внутри будет видно: у всех функций-методов на один аргумент больше.}
\EN{In GCC, the \ITthis pointer is passed as the first function-method argument.
Thus it will be very visible that internally: all function-methods have an extra argument.}

\RU{Для примера, см.}\EN{For an example, see} (\myref{thiscall}).

\section{x86-64}
\index{x86-64}

\subsection{Windows x64}
\label{sec:callingconventions_win64}

\RU{В win64 метод передачи всех параметров немного похож на \TT{fastcall}. 
Первые 4 аргумента записываются в регистры \RCX, \RDX, \Reg{8}, \Reg{9}, а остальные\EMDASH{}в стек. 
Вызывающая функция также должна подготовить место из 32 байт или для четырех 64-битных значений, 
чтобы вызываемая функция могла сохранить там первые 4 аргумента. 
Короткие функции могут использовать переменные прямо из регистров, 
но б\'{о}льшие могут сохранять их значения на будущее.}
\EN{The method of for passing arguments in Win64 somewhat resembles \TT{fastcall}.
The first 4 arguments are passed via \RCX, \RDX, \Reg{8} and \Reg{9}, the rest\EMDASH{}via the stack.
The \gls{caller} also must prepare space for 32 bytes or 4 64-bit values,
so then the \gls{callee} can save there the first 4 arguments.
Short functions may use the arguments' values just from the registers,
but larger ones may save their values for further use.}

\RU{Вызывающая функция должна вернуть \glslink{stack pointer}{указатель стека} 
в первоначальное состояние}
\EN{The \gls{caller} also must return the \gls{stack pointer} into its initial state}.

\RU{Это же соглашение используется и в системных библиотеках Windows x86-64 
(вместо \IT{stdcall} в win32).}
\EN{This calling convention is also used in Windows x86-64 system DLLs 
(instead of \IT{stdcall} in win32).}

\RU{Пример}\EN{Example}:

\lstinputlisting{OS/calling_conventions/x64.c}

\lstinputlisting[caption=MSVC 2012 /0b]{OS/calling_conventions/x64_MSVC_Ob.asm}

\index{Scratch space}
\RU{Здесь мы легко видим, как 7 аргументов передаются: 4 через регистры и остальные 3 через стек}
\EN{Here we clearly see how 7 arguments are passed: 4 via registers and the remaining 3 via the stack}.
\RU{Код пролога функции f1() сохраняет аргументы в \q{scratch space}\EMDASH{}место в стеке предназначенное
именно для этого}
\EN{The code of the f1() function's prologue saves the arguments in the \q{scratch space}\EMDASH{}a space in the stack
intended exactly for this purpose}.
\RU{Это делается потому что компилятор может быть не уверен, достаточно ли ему будет остальных регистров
для работы исключая эти 4, которые иначе будут заняты аргументами до конца исполнения функции}
\EN{This is done because the compiler can not be sure that there will be enough registers to use without these 4,
which will otherwise be occupied by the arguments until the function's execution end}.
\RU{Выделение \q{scratch space} в стеке лежит на ответственности вызывающей функции.}
\EN{The \q{scratch space} allocation in the stack is the caller's duty.}

\lstinputlisting[caption=\Optimizing MSVC 2012 /0b]{OS/calling_conventions/x64_MSVC_Ox_Ob.asm}

\RU{Если компилировать этот пример с оптимизацией, то выйдет почти то же самое, 
только \q{scratch space} не используется, потому что незачем.}
\EN{If we compile the example with optimizations, it is to be almost the same, 
but the \q{scratch space} will not be used, because it won't be needed.}

\index{x86!\Instructions!LEA}
\label{using_MOV_and_pack_of_LEA_to_load_values}
\RU{Обратите также внимание на то как MSVC 2012 оптимизирует примитивную загрузку значений в регистры
используя \LEA (\myref{sec:LEA})}
\EN{Also take a look on how MSVC 2012 optimizes the loading of primitive values into registers by using 
\LEA (\myref{sec:LEA})}.
\RU{Трудно сказать, стоит ли оно того, но может быть}\EN{It's hard to say if it worth doing so, but maybe}.

\RU{Еще один пример подобного}\EN{Another example of such thing is}: \myref{TaskMgr_LEA}.

\subsubsection{Windows x64: \RU{Передача \ITthis}\EN{Passing \ITthis} (\CCpp)}

\EN{The}\RU{Указатель} \ITthis \RU{передается через}\EN{pointer is passed in} \RCX, 
\RU{первый аргумент метода через}\EN{the first argument of the method is in} \RDX, \etc{}.
\RU{Для примера, см. также}\EN{For an example see}: \myref{simple_CPP_MSVC_x64}.
 
\subsection{Linux x64}

\RU{Метод передачи аргументов в Linux для x86-64 почти такой же, как и в Windows, но 6 регистров
используется вместо 4 (\RDI, \RSI, \RDX, \RCX, \Reg{8}, \Reg{9}), и здесь нет \q{scratch space}, 
но \gls{callee} может сохранять значения регистров в стеке, если ему это нужно.}
\EN{The way arguments are passed in Linux for x86-64 is almost the same as in Windows, but 6 registers are
used instead of 4 (\RDI, \RSI, \RDX, \RCX, \Reg{8}, \Reg{9}) and there is no \q{scratch space}, 
although the \gls{callee} may save the register values in the stack, if it needs/wants to.}

\lstinputlisting[caption=\Optimizing GCC 4.7.3]{OS/calling_conventions/x64_linux_O3.s}

\index{AMD}
\RU{N.B.: здесь значения записываются в 32-битные части регистров (например EAX) а не в весь 64-битный
регистр (RAX).
Это связано с тем что в x86-64,
запись в младшую 32-битную часть 64-битного регистра автоматически обнуляет старшие 32 бита.
Вероятно, это так решили в AMD для упрощения портирования кода под x86-64.}
\EN{N.B.: here the values are written into the 32-bit parts of the registers (e.g., EAX) but not in the whole 64-bit 
register (RAX).
This is because each write to the low 32-bit part of a register automatically clears the high 32 bits.
Supposedly, it was decided in AMD to do so to simplify porting code to x86-64.}

\section{\RU{Возвращение переменных типа \Tfloat, \Tdouble}\EN{Return values of \Tfloat and \Tdouble type}}
\index{float}
\index{double}

\RU{Во всех соглашениях кроме Win64, переменная типа \Tfloat или \Tdouble возвращается через регистр FPU \ST{0}.}
\EN{In all conventions except in Win64, the values of type \Tfloat or \Tdouble are returned via the FPU register \ST{0}.}

\RU{В Win64 переменные типа \Tfloat и \Tdouble возвращаются в младших 16-и или 32-х битах 
регистра \XMM{0}.}
\EN{In Win64, the values of \Tfloat and \Tdouble types are returned 
in the low 32 or 64 bits of the \XMM{0} register.}

\section{\RU{Модификация аргументов}\EN{Modifying arguments}}

\RU{Иногда программисты на \CCpp{} (и не только этих \ac{PL}) задаются вопросом,
что может случиться, если модифицировать аргументы?}
\EN{Sometimes, \CCpp{} programmers (not limited to these \ac{PL}s, though),
may ask, what can happen if they modify the arguments?}
\RU{Ответ прост: аргументы хранятся в стеке, именно там и будет происходит модификация.}
\EN{The answer is simple: the arguments are stored in the stack, 
that is where the modification takes place.}
\RU{А вызывающие функции не использует их после вызова функции (автор этих строк никогда не видел в своей практике обратного случая).}
\EN{The calling functions is not using them after the \gls{callee}'s exit (author of these lines have never seen any such case in his practice).}

\lstinputlisting{OS/calling_conventions/change_arguments.c}

\lstinputlisting[caption=MSVC 2012]{OS/calling_conventions/change_arguments.asm}

% TODO (OllyDbg) пример как в стеке меняется $a$

\RU{Следовательно, модифицировать аргументы функции можно запросто.}\EN{So yes, one can modify the arguments easily.}
\RU{Разумеется, если это не}\EN{Of course, if it is not} \IT{references} \InENRU \Cpp{} (\myref{cpp_references}),
\RU{и если вы не модифицируете данные по указателю}\EN{and if you not modify data to which a pointer points to}, 
\RU{то эффект не будет распространяться за пределами текущей функции}
\EN{then the effect will not propagate outside the current function}.

\RU{Теоретически, после возврата из \gls{callee},
функция-\gls{caller} могла бы получить модифицированный аргумент и использовать его как-то.
Может быть, если бы она была написана на языке ассемблера.
Но стандарты языков \CCpp не предлагают никакого способа доступиться к ним.}
\EN{Theoretically, after the \gls{callee}'s return, 
the \gls{caller} could get the modified argument and use it somehow.
Maybe if it is written directly in assembly language.
But the \CCpp languages standards don't offer any way to access them.}

% sections
\section{\RU{Указатель на аргумент функции}\EN{Taking a pointer to function argument}}
\label{pointer_to_argument}

\dots \RU{и даже более того, можно взять указатель на аргумент функции и передать его в другую функцию:}
\EN{even more than that, it's possible to take a pointer to the function's argument and pass
it to another function:}

\lstinputlisting{OS/calling_conventions/ptr_to_argument/10.c}

\RU{Трудно понять, как это работает, пока мы не посмотрим на код:}
\EN{It's hard to understand how it works until we can see the code:}

\lstinputlisting[caption=\Optimizing MSVC 2010]{OS/calling_conventions/ptr_to_argument/MSVC_2010_O3.asm}

\RU{Адрес места в стеке где была передана $a$ просто передается в другую функцию.
Она модифицирует переменную по этому адресу, и затем \printf выведет модифицированное значение.}
\EN{The address of the place in the stack where $a$ was passed is just passed to another function.
It modifies the value addressed by the pointer and then \printf prints the modified value.}

\par \RU{Наблюдательный читатель может спросить, а что насчет тех соглашений о вызовах, где аргументы функции
передаются в регистрах?}
\EN{The observant reader might ask, what about calling conventions where the function's arguments are
passed in registers?}

\RU{Это та ситуация где используется \IT{Shadow Space}.}
\EN{That's a situation where the \IT{Shadow Space} is used.}
\RU{Так что входящее значение копируется из регистра в \IT{Shadow Space} в локальном стеке и затем это адрес
передается в другую функцию:}
\EN{The input value is copied from the register
to the \IT{Shadow Space} in the local stack, and then this address is passed to the other function:}

\lstinputlisting[caption=\Optimizing MSVC 2012 x64]{OS/calling_conventions/ptr_to_argument/MSVC_2012_x64_O3.asm}

\RU{GCC также записывает входное значение в локальный стек:}
\EN{GCC also stores the input value in the local stack:}

\lstinputlisting[caption=\Optimizing GCC 4.9.1 x64]{OS/calling_conventions/ptr_to_argument/GCC491_x64_O3.s}

\RU{GCC для ARM64 делает то же самое, но это пространство здесь называется \IT{Register Save Area}:}
\EN{GCC for ARM64 does the same, but this space is called \IT{Register Save Area} here:}

\lstinputlisting[caption=\Optimizing GCC 4.9.1 ARM64]{OS/calling_conventions/ptr_to_argument/GCC49_ARM64_O3.s}

\EN{By the way, a similar usage of the \IT{Shadow Space} is also considered here}
\RU{Кстати, похожее использование \IT{Shadow Space} разбирается здесь}: \myref{variadic_arith_registers}.

