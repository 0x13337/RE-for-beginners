\section{\RU{Указатель на аргумент функции}\EN{Taking a pointer to function argument}}
\label{pointer_to_argument}

\dots \RU{и даже более того, можно взять указатель на аргумент функции и передать его в другую функцию:}
\EN{even more than that, it's possible to take a pointer to the function's argument and pass
it to another function:}

\lstinputlisting{OS/calling_conventions/ptr_to_argument/10.c}

\RU{Трудно понять, как это работает, пока мы не посмотрим на код:}
\EN{It's hard to understand how it works until we can see the code:}

\lstinputlisting[caption=\Optimizing MSVC 2010]{OS/calling_conventions/ptr_to_argument/MSVC_2010_O3.asm}

\RU{Адрес места в стеке где была передана $a$ просто передается в другую функцию.
Она модифицирует переменную по этому адресу, и затем \printf выведет модифицированное значение.}
\EN{The address of the place in the stack where $a$ was passed is just passed to another function.
It modifies the value addressed by the pointer and then \printf prints the modified value.}

\par \RU{Наблюдательный читатель может спросить, а что насчет тех соглашений о вызовах, где аргументы функции
передаются в регистрах?}
\EN{The observant reader might ask, what about calling conventions where the function's arguments are
passed in registers?}

\RU{Это та ситуация где используется \IT{Shadow Space}.}
\EN{That's a situation where the \IT{Shadow Space} is used.}
\RU{Так что входящее значение копируется из регистра в \IT{Shadow Space} в локальном стеке и затем это адрес
передается в другую функцию:}
\EN{The input value is copied from the register
to the \IT{Shadow Space} in the local stack, and then this address is passed to the other function:}

\lstinputlisting[caption=\Optimizing MSVC 2012 x64]{OS/calling_conventions/ptr_to_argument/MSVC_2012_x64_O3.asm}

\RU{GCC также записывает входное значение в локальный стек:}
\EN{GCC also stores the input value in the local stack:}

\lstinputlisting[caption=\Optimizing GCC 4.9.1 x64]{OS/calling_conventions/ptr_to_argument/GCC491_x64_O3.s}

\RU{GCC для ARM64 делает то же самое, но это пространство здесь называется \IT{Register Save Area}:}
\EN{GCC for ARM64 does the same, but this space is called \IT{Register Save Area} here:}

\lstinputlisting[caption=\Optimizing GCC 4.9.1 ARM64]{OS/calling_conventions/ptr_to_argument/GCC49_ARM64_O3.s}

\EN{By the way, a similar usage of the \IT{Shadow Space} is also considered here}
\RU{Кстати, похожее использование \IT{Shadow Space} разбирается здесь}: \myref{variadic_arith_registers}.
