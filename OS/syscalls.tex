\chapter{\RU{Системные вызовы (syscall-ы)}\EN{System calls (syscall-s)}}

\label{syscalls}
\index{syscall}

\index{kernel space}
\index{user space}
\RU{Как известно, все работающие процессы в \ac{OS} делятся на две категории}\EN{As we know, all running processes
inside an \ac{OS} are divided into two categories}:
\RU{имеющие полный доступ ко всему \q{железу}}\EN{those having full access to the hardware} (\q{kernel space}) 
\RU{и не имеющие}\EN{and those that do not} (\q{user space}).

\RU{В первой категории ядро \ac{OS} и, обычно, драйвера}
\EN{The \ac{OS} kernel and usually the drivers are in the first category}.

\RU{Во второй категории всё прикладное ПО}\EN{All applications are usually in the second category}.

\index{Glibc}
\EN{For example, Linux kernel is in \IT{kernel space}, but Glibc in \IT{user space}.}
\RU{Например, ядро Linux в \IT{kernel space}, но Glibc в \IT{user space}.}

\RU{Это разделение очень важно для безопасности \ac{OS}:
очень важно чтобы никакой процесс не мог испортить что-то в других процессах
или даже в самом ядре \ac{OS}}
\EN{This separation is crucial for the safety of the \ac{OS}: it is very important not to give to any process the possibility to screw up
something in other processes or even in the \ac{OS} kernel}.
\index{kernel panic}
\index{BSoD}
\RU{С другой стороны, падающий драйвер или ошибка внутри ядра \ac{OS} обычно приводит к kernel panic или \ac{BSOD}.}%
\EN{On the other hand, a failing driver or error inside the \ac{OS}'s kernel usually leads to a kernel panic or \ac{BSOD}.}

\RU{Защита x86-процессора устроена так что возможно разделить всё на 4 слоя защиты (rings), но и в Linux,
и в Windows, используются только 2}
\EN{The protection in the x86 processors allows to separate everything into 4 levels of protection (rings), but both in Linux
and in Windows only two are used}: ring0 (\q{kernel space}) \AndENRU ring3 (\q{user space}).

\RU{Системные вызовы}\EN{System calls} (syscall-\RU{ы}\EN{s})
\RU{это точка где соединяются вместе оба эти пространства}\EN{are a point where these two areas are connected}.
\RU{Это, можно сказать, самое главное \ac{API} предоставляемое прикладному ПО.}
\EN{It can be said that this is the main \ac{API} provided to applications.}

\RU{В \gls{Windows NT} таблица сисколлов находится в \ac{SSDT}}
\EN{As in \gls{Windows NT}, the syscalls table resides in the \ac{SSDT}}.

\index{Shellcode}
\RU{Работа через syscall-ы популярна у авторов шеллкодов и вирусов,
потому что там обычно бывает трудно определить адреса нужных функций в системных библиотеках,
а syscall-ами проще пользоваться, хотя и придется писать больше
кода из-за более низкого уровня абстракции этого \ac{API}}
\EN{The usage of syscalls is very popular among shellcode and computer viruses authors, 
because it is hard to determine the addresses of
needed functions in the system libraries, but it is easier to use syscalls. However, much more code has to be
written due to the lower level of abstraction of the \ac{API}}.
\RU{Также нельзя еще забывать, что номера syscall-ов могут отличаться от версии к версии OS.}
\EN{It is also worth noting that the syscall numbers may be different in various OS versions.}

\section{Linux}

\index{x86!\Instructions!INT!INT 0x80}
\RU{В Linux вызов syscall-а обычно происходит через}\EN{In Linux, a syscall is usually called via} \TT{int 0x80}.
\RU{В регистре}\EN{The call's number is passed in the} \EAX \RU{передается номер вызова,
в остальных регистрах\EMDASH{}параметры}\EN{register, and any other parameters~---in the other registers}.

\lstinputlisting[caption=\RU{Простой пример использования пары syscall-ов}\EN{A simple example of the usage of two syscalls}]
{OS/linux_syscall.s}

\RU{Компиляция}\EN{Compilation}:

\begin{lstlisting}
nasm -f elf32 1.s
ld 1.o
\end{lstlisting}

\RU{Полный список syscall-ов в}\EN{The full list of syscalls in} Linux: \url{http://go.yurichev.com/17319}.

\RU{Для перехвата и трассировки системных вызовов в Linux, можно применять}
\EN{For system calls interception and tracing in Linux,} strace(\myref{strace})\EN{ can be used}.

\section{Windows}

\index{x86!\Instructions!INT!INT 0x2e}
\index{x86!\Instructions!SYSENTER}

\RU{Вызов происходит через}\EN{Here they are called via} \TT{int 0x2e} 
\RU{либо используя специальную x86-инструкцию}\EN{or using the special x86 instruction} \TT{SYSENTER}.

\RU{Полный список syscall-ов в}\EN{The full list of syscalls in} Windows: \url{http://go.yurichev.com/17320}.

\RU{Смотрите также}\EN{Further reading}:

\q{Windows Syscall Shellcode} by Piotr Bania:\\
\url{http://go.yurichev.com/17321}.

