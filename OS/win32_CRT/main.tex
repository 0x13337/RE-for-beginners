\section{CRT (win32)}
\label{sec:CRT}
\index{CRT}

\RU{Начинается ли исполнение программы прямо с ф-ции \main{}}
\EN{Does program execution starts right at the \main{} function}?
\RU{Нет, не начинается}\EN{No, it is not}.
\RU{Если открыть любой исполняемый файл в \IDA или Hiew, то \ac{OEP} указывает на совсем другой код}
\EN{If we would open any executable file in \IDA or HIEW, we will see \ac{OEP} pointing to another code}.
\RU{Это код, который делает некоторые приготовления перед тем как запустить ваш код}
\EN{This is a code doing some maintenance and preparations before passing control flow to our code}.
\RU{Он называется стартап-код или CRT-код (C RunTime)}\EN{It is called startup-code or CRT-code (C RunTime)}. \\
\\
\RU{Ф-ция \main{} принимает на вход массив из параметров, переданных в командной строке, а также
переменные окружения}\EN{\main{} function takes an array of arguments passed in the command line, and also
environment variables}.
\RU{Но в реальности в программу передается командная строка в виде простой строки, это именно
CRT-код находит там пробелы и разрезает строку на части}\EN{But in fact, a generic string is passed to the program,
CRT-code will find spaces in it and cut by parts}.
\RU{CRT-код также готовит массив переменных окружения \TT{envp}}\EN{CRT-code is also prepares environment
variables array \TT{envp}}.
\RU{В \ac{GUI}-приложениях win32, вместо \main{} имеется ф-ция \TT{WinMain} со своими аргументами}
\EN{As of \ac{GUI} win32 applications, \TT{WinMain} is used instead of \main{}, having their own arguments}:

\begin{lstlisting}
int CALLBACK WinMain(
  _In_  HINSTANCE hInstance,
  _In_  HINSTANCE hPrevInstance,
  _In_  LPSTR lpCmdLine,
  _In_  int nCmdShow
);
\end{lstlisting}

\RU{CRT-код готовит и их}\EN{CRT-code prepares them as well}.

\RU{А также, число, возвращаемое ф-цией \main{}, это код ошибки возвращаемый программой}
\EN{Also, the number returned by \main{} function is an exit code}.
\RU{В CRT это значение передается в \TT{ExitProcess()}, принимающей в качестве аргумента код ошибки}
\EN{It may be passed in CRT to the \TT{ExitProcess()} function, taking exit code as argument}. \\
\\
\RU{Как правило, каждый компилятор имеет свой CRT-код}\EN{Usually, each compiler has its own CRT-code}. \\
\\
\RU{Вот типичный для MSVC 2008 CRT-код}\EN{Here is a typical CRT-code for MSVC 2008}.

\lstinputlisting{OS/win32_CRT/crt_msvc_2008.asm}

\RU{Здесь можно увидеть по крайней мере вызов
ф-ции}\EN{Here we may see calls to} \TT{GetCommandLineA()}, \RU{затем}\EN{then to} \TT{setargv()} \AndENRU 
\TT{setenvp()},
\RU{которые, видимо, заполняют глобальные переменные-указатели}\EN{which are, apparently, fills global variables}
\TT{argc}, \TT{argv}, \TT{envp}.

\RU{В итоге, вызывается \main{} с этими аргументами}\EN{Finally, \main{} is called with these arguments}.

\RU{Также имеются вызовы ф-ций с говорящими именами вроде}\EN{There are also calls to the functions
having self-describing names like} \TT{heap\_init()}, \TT{ioinit()}.

\RU{\glslink{heap}{Куча} действительно инициализируется в \ac{CRT}}
\EN{\glslink{heap}{Heap} is indeed initialized in \ac{CRT}}: 
\RU{если вы попытаетесь использовать}\EN{if you will try to use} \TT{malloc()}, 
\RU{программа упадет с такой ошибкой}\EN{the program exiting abnormally with the error}:

\begin{lstlisting}
runtime error R6030
- CRT not initialized
\end{lstlisting}

\RU{Инициализация глобальных объектов в Си++ происходит до вызова \main{}, именно в \ac{CRT}}
\EN{Global objects initializations in C++ is also occurred in the \ac{CRT} before \main{}}: 
\ref{sec:std_string_as_global_variable}.

\RU{Значение, возвращаемое из}\EN{A variable} \main{} \RU{передается или в}\EN{returns is passed to} \TT{cexit()}, 
\RU{или же в}\EN{or to} \TT{\$LN32}, \RU{которая далее вызывает}\EN{which in turn calling} \TT{doexit()}.

\RU{Можно ли обойтись без \ac{CRT}? Можно, если вы знаете что делаете.}\EN{Is it possible to get rid of \ac{CRT}?
Yes, if you know what you do.}

\RU{В линкере от \ac{MSVC} точка входа задается опцией \TT{/ENTRY}}
\EN{\ac{MSVC} linker has \TT{/ENTRY} option for setting entry point}.

\begin{lstlisting}
#include <windows.h>

int main()
{
	MessageBox (NULL, "hello, world", "caption", MB_OK);
};
\end{lstlisting}

\RU{Компилируем в}\EN{Let's compile it in} MSVC 2008.

\begin{lstlisting}
cl no_crt.c user32.lib /link /entry:main
\end{lstlisting}

\RU{Получаем вполне работающий .exe размером 2560 байт, внутри которого есть только PE-заголовок, инструкции, 
вызывающие \TT{MessageBox},
две строки в сегменте данных, импортируемая из \TT{user32.dll} ф-ция \TT{MessageBox}, и более ничего}
\EN{We will get a runnable .exe with size 2560 bytes, there are PE-header inside, instructions calling
\TT{MessageBox}, two strings in the data segment,
\TT{MessageBox} function imported from \TT{user32.dll} and nothing else}.

\RU{Это работает, но вы уже не сможете вместо \main{} написать \TT{WinMain} с его четырьмя аргументами}
\EN{This works, but you will not be able to write \TT{WinMain} with its 4 arguments instead of \main{}}.
\RU{Вернее, написать-то сможете, но доступа к этим аргументам не будет, 
потому что они не будут подготовлены на момент исполнения}\EN{To be correct, you will be able to write so,
but arguments will not be prepared at the moment of execution}.

\RU{Кстати, можно еще короче сделать .exe если уменьшить 
выравнивание \ac{PE}-секций (которое, по умолчанию, 4096 байт)}\EN{By the way, it is possible to make .exe even
shorter by doing \ac{PE}-section aligning less than default 4096 bytes}.

\begin{lstlisting}
cl no_crt.c user32.lib /link /entry:main /align:16
\end{lstlisting}

\RU{Линкер скажет}\EN{Linker will say}:

\begin{lstlisting}
LINK : warning LNK4108: /ALIGN specified without /DRIVER; image may not run
\end{lstlisting}

\RU{Получим .exe размером 720 байт}\EN{We getting .exe of 720 bytes size}.
\RU{Он запускается в}\EN{It running in} Windows 7 x86, \RU{но не}\EN{but not in} x64 
(\RU{там выдает ошибку при загрузке}\EN{the error message will be showed when trying to execute}).
\RU{При желании, размер можно еще сильнее ужать, но, как видно, 
возникают проблемы с совместимостью с разными версиями Windows}\EN{By applying even more efforts, it is possible
to make executable even shorter, but as you can see, compatibility problems may arise quickly}.

