\section{CRT (win32)}
\label{sec:CRT}
\index{CRT}

\RU{Начинается ли исполнение программы прямо с функции \main{}}
\EN{Does the program execution start right at the \main{} function}?
\RU{Нет, не начинается}\EN{No, it does not}.
\RU{Если открыть любой исполняемый файл в \IDA или Hiew, 
то \ac{OEP} указывает на какой-то совсем другой код.}
\EN{If we would open any executable file in \IDA or HIEW, 
we can see \ac{OEP} pointing to some another code block.}
\RU{Это код, который делает некоторые приготовления перед тем как запустить ваш код}
\EN{This code is doing some maintenance and preparations before passing control flow to our code}.
\RU{Он называется стартап-код или CRT-код (C RunTime)}\EN{It is called startup-code or CRT code (C RunTime)}. \\
\\
\RU{Функция \main{} принимает на вход массив из параметров, переданных в командной строке, а также
переменные окружения}\EN{The \main{} function takes an array of the arguments passed on the command line, and also
one with environment variables}.
\RU{Но в реальности в программу передается командная строка в виде простой строки, это именно
CRT-код находит там пробелы и разрезает строку на части}\EN{But in fact a generic string is passed to the program,
the CRT code finds the spaces in it and cuts it in parts}.
\RU{CRT-код также готовит массив переменных окружения \TT{envp}}\EN{The CRT code also prepares the environment
variables array \TT{envp}}.
\RU{В \ac{GUI}-приложениях win32, вместо \main{} имеется функция \TT{WinMain} со своими аргументами}
\EN{As for \ac{GUI} win32 applications, \TT{WinMain} is used instead of \main{}, having its own arguments}:

\begin{lstlisting}
int CALLBACK WinMain(
  _In_  HINSTANCE hInstance,
  _In_  HINSTANCE hPrevInstance,
  _In_  LPSTR lpCmdLine,
  _In_  int nCmdShow
);
\end{lstlisting}

\RU{CRT-код готовит и их}\EN{The CRT code prepares them as well}.

\RU{А также, число, возвращаемое функцией \main{}, это код ошибки возвращаемый программой}
\EN{Also, the number returned by the \main{} function is the exit code}.
\RU{В CRT это значение передается в \TT{ExitProcess()}, принимающей в качестве аргумента код ошибки}
\EN{It may be passed in CRT to the \TT{ExitProcess()} function, which takes the exit code as an argument}. \\
\\
\RU{Как правило, каждый компилятор имеет свой CRT-код}\EN{Usually, each compiler has its own CRT code}. \\
\\
\RU{Вот типичный для MSVC 2008 CRT-код}\EN{Here is a typical CRT code for MSVC 2008}.

\lstinputlisting[numbers=left]{OS/win32_CRT/crt_msvc_2008.asm}

\RU{Здесь можно увидеть по крайней мере вызов
функции}\EN{Here we can see calls to} \TT{GetCommandLineA()} (\LineENRU 62), 
\RU{затем}\EN{then to} \TT{setargv()} (\LineENRU 66) \AndENRU \TT{setenvp()} (\LineENRU 74),
\RU{которые, видимо, заполняют глобальные переменные-указатели}\EN{which apparently fill the global variables}
\TT{argc}, \TT{argv}, \TT{envp}.

\RU{В итоге, вызывается \main{} с этими аргументами}\EN{Finally, \main{} is called with these arguments} 
(\LineENRU 97).

\RU{Также имеются вызовы функций с говорящими именами вроде}\EN{There are also calls to functions
with self-describing names like} \TT{heap\_init()} (\LineENRU 35), \TT{ioinit()} (\LineENRU 54).

\RU{\glslink{heap}{Куча} действительно инициализируется в \ac{CRT}}
\EN{The \glslink{heap}{heap} is indeed initialized in the \ac{CRT}}.
\RU{Если вы попытаетесь использовать}\EN{If you try to use} \TT{malloc()} 
\RU{в программе без}\EN{in a program without} CRT,
\RU{программа упадет с такой ошибкой}\EN{it will exit abnormally with the following error}:

\begin{lstlisting}
runtime error R6030
- CRT not initialized
\end{lstlisting}

\RU{Инициализация глобальных объектов в \Cpp происходит до вызова \main{}, именно в \ac{CRT}}
\EN{Global object initializations in \Cpp is also occur in the \ac{CRT} before the execution of \main{}}: 
\myref{sec:std_string_as_global_variable}.

\RU{Значение, возвращаемое из}\EN{The value that} \main{} \RU{передается или в}\EN{returns is passed to} \TT{cexit()}, 
\RU{или же в}\EN{or in} \TT{\$LN32}, \RU{которая далее вызывает}\EN{which in turn calls} \TT{doexit()}.

\RU{Можно ли обойтись без \ac{CRT}? Можно, если вы знаете что делаете.}\EN{Is it possible to get rid of the \ac{CRT}?
Yes, if you know what you are doing.}

\RU{В линкере от \ac{MSVC} точка входа задается опцией \TT{/ENTRY}}
\EN{The \ac{MSVC}'s linker has the \TT{/ENTRY} option for setting an entry point}.

\begin{lstlisting}
#include <windows.h>

int main()
{
	MessageBox (NULL, "hello, world", "caption", MB_OK);
};
\end{lstlisting}

\RU{Компилируем в}\EN{Let's compile it in} MSVC 2008.

\begin{lstlisting}
cl no_crt.c user32.lib /link /entry:main
\end{lstlisting}

\RU{Получаем вполне работающий .exe размером 2560 байт, внутри которого есть только PE-заголовок, инструкции, 
вызывающие \TT{MessageBox},
две строки в сегменте данных, импортируемая из \TT{user32.dll} функция \TT{MessageBox}, и более ничего.}
\EN{We are getting a runnable .exe with size 2560 bytes, that has a PE header in it, instructions calling
\TT{MessageBox}, two strings in the data segment,
the \TT{MessageBox} function imported from \TT{user32.dll} and nothing else.}

\RU{Это работает, но вы уже не сможете вместо \main{} написать \TT{WinMain} с его четырьмя аргументами}
\EN{This works, but you cannot write \TT{WinMain} with its 4 arguments instead of \main{}}.
\RU{Вернее, если быть точным, написать-то сможете, но доступа к этим аргументам не будет, 
потому что они не подготовлены на момент исполнения.}
\EN{To be precise, you can, but the arguments are not prepared at the moment of execution.}

\RU{Кстати, можно еще короче сделать .exe если уменьшить 
выравнивание \ac{PE}-секций (которое, по умолчанию, 4096 байт).}
\EN{By the way, it is possible to make the .exe even 
shorter by aligning the \ac{PE} sections at less than the default 4096 bytes.}

\begin{lstlisting}
cl no_crt.c user32.lib /link /entry:main /align:16
\end{lstlisting}

\RU{Линкер скажет}\EN{Linker says}:

\begin{lstlisting}
LINK : warning LNK4108: /ALIGN specified without /DRIVER; image may not run
\end{lstlisting}

\RU{Получим .exe размером 720 байт}\EN{We get an .exe that's 720 bytes}.
\RU{Он запускается в}\EN{It can be exectued in} Windows 7 x86, \RU{но не}\EN{but not in} x64 
(\RU{там выдает ошибку при загрузке}\EN{an error message will be shown when you try to execute it}).
\RU{При желании, размер можно еще сильнее ужать, но, как видно, 
возникают проблемы с совместимостью с разными версиями Windows.}
\EN{With even more efforts, it is possible
to make the executable even shorter, but as you can see, compatibility problems arise quickly.}
