\sectionold{CRT (win32)}
\label{sec:CRT}
\myindex{CRT}

Does the program execution start right at the \main{} function?
No, it does not.

If we would open any executable file in \IDA or HIEW, 
we can see \ac{OEP} pointing to some another code block.

This code is doing some maintenance and preparations before passing control flow to our code.
It is called startup-code or CRT code (C RunTime). \\
\\
The \main{} function takes an array of the arguments passed on the command line, and also
one with environment variables.
But in fact a generic string is passed to the program,
the CRT code finds the spaces in it and cuts it in parts.
The CRT code also prepares the environment
variables array \TT{envp}.

As for \ac{GUI} win32 applications, \TT{WinMain} is used instead of \main{}, having its own arguments:

\begin{lstlisting}
int CALLBACK WinMain(
  _In_  HINSTANCE hInstance,
  _In_  HINSTANCE hPrevInstance,
  _In_  LPSTR lpCmdLine,
  _In_  int nCmdShow
);
\end{lstlisting}

The CRT code prepares them as well.

Also, the number returned by the \main{} function is the exit code.

It may be passed in CRT to the \TT{ExitProcess()} function, which takes the exit code as an argument. \\
\\
Usually, each compiler has its own CRT code. \\
\\
Here is a typical CRT code for MSVC 2008.

\lstinputlisting[numbers=left]{OS/win32_CRT/crt_msvc_2008.asm}

Here we can see calls to \TT{GetCommandLineA()} (line 62),
then to \TT{setargv()} (line 66) and \TT{setenvp()} (line 74),
which apparently fill the global variables
\TT{argc}, \TT{argv}, \TT{envp}.

Finally, \main{} is called with these arguments (line 97).

There are also calls to functions
with self-describing names like \TT{heap\_init()} (line 35), \TT{ioinit()} (line 54).

The \glslink{heap}{heap} is indeed initialized in the \ac{CRT}.
If you try to use \TT{malloc()} in a program without CRT, it will exit abnormally with the following error:

\begin{lstlisting}
runtime error R6030
- CRT not initialized
\end{lstlisting}

Global object initializations in \Cpp is also occur in the \ac{CRT} before the execution of \main{}: 
\myref{sec:std_string_as_global_variable}.

The value that \main{} returns is passed to \TT{cexit()}, 
or in \TT{\$LN32}, which in turn calls \TT{doexit()}.

Is it possible to get rid of the \ac{CRT}?
Yes, if you know what you are doing.

The \ac{MSVC}'s linker has the \TT{/ENTRY} option for setting an entry point.

\begin{lstlisting}
#include <windows.h>

int main()
{
	MessageBox (NULL, "hello, world", "caption", MB_OK);
};
\end{lstlisting}

Let's compile it in MSVC 2008.

\begin{lstlisting}
cl no_crt.c user32.lib /link /entry:main
\end{lstlisting}

We are getting a runnable .exe with size 2560 bytes, that has a PE header in it, instructions calling
\TT{MessageBox}, two strings in the data segment,
the \TT{MessageBox} function imported from \TT{user32.dll} and nothing else.

This works, but you cannot write \TT{WinMain} with its 4 arguments instead of \main{}.

To be precise, you can, but the arguments are not prepared at the moment of execution.

By the way, it is possible to make the .exe even 
shorter by aligning the \ac{PE} sections at less than the default 4096 bytes.

\begin{lstlisting}
cl no_crt.c user32.lib /link /entry:main /align:16
\end{lstlisting}

Linker says:

\begin{lstlisting}
LINK : warning LNK4108: /ALIGN specified without /DRIVER; image may not run
\end{lstlisting}

We get an .exe that's 720 bytes.
It can be exectued in Windows 7 x86, but not in x64 
(an error message will be shown when you try to execute it).

With even more efforts, it is possible
to make the executable even shorter, but as you can see, compatibility problems arise quickly.

