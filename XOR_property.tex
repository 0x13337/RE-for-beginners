% included twice... a bit of redundancy, but it's OK

\RU{XOR (исключающее ИЛИ) часто используется для того чтобы поменять какой-то бит(ы) на противоположный.}
\EN{XOR is widely used when one need just to flip specific bit(s).}
\RU{Действительно, операция XOR с 1 на самом деле просто инвертирует бит:}
\EN{Indeed, the XOR operation applied with 1 is effectively inverting a bit:}

\begin{center}
\begin{tabular}{ | l | l | l | }
\hline
\cellcolor{blue!25} \RU{вход А}\EN{input A} & 
\cellcolor{blue!25} \RU{вход Б}\EN{input B} & 
\cellcolor{blue!25} \RU{выход}\EN{output} \\
\hline
0 & 0 & 0 \\
\hline
{\color{red} 0} & {\color{red} 1} & {\color{red} 1} \\
\hline
{\color{red} 1} & {\color{red} 0} & {\color{red} 1} \\
\hline
1 & 1 & 0 \\
\hline
\end{tabular}
\end{center}

\RU{И наоборот, операция XOR с 0 ничего не делает, т.е. это холостая операция.}
\EN{And on the contrary, the XOR operation applied with 0 does nothing, i.e., it's an idle operation.}
\RU{Это очень важное свойство операции XOR и очень важно помнить его.}
\EN{This is a very important property of the XOR operation and it's highly recommended to memorize it.}
