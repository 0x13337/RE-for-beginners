\part*z{\RU{Список принятых сокращений}\EN{Acronyms used}}
\addcontentsline{toc}{part}{\RU{Список принятых сокращений}\EN{Acronyms used}}
\begin{acronym}
\RU{
	\acro{OS}[ОС]{Операционная Система}
	\acro{FAQ}[ЧаВО]{Часто задаваемые вопросы}
	\acro{OOP}[ООП]{Объектно-Ориентированное Программирование}
	\acro{PL}[ЯП]{Язык Программирования}
	\acro{PRNG}[ГПСЧ]{Генератор псевдослучайных чисел}
	\acro{ROM}[ПЗУ]{Постоянное запоминающее устройство}
	\acro{ALU}[АЛУ]{Арифметико-логическое устройство}
}
\EN{
	\acro{OS}{Operating System}
	\acro{FAQ}{Frequently Asked Questions}
	\acro{OOP}{Object-Oriented Programming}
	\acro{PL}{Programming language}
	\acro{PRNG}{Pseudorandom number generator}
	\acro{ROM}{Read-only memory}
	\acro{ALU}{Arithmetic logic unit}
}
\acro{RA}{\RU{Адрес возврата}\EN{Return Address}}
\acro{PE}{Portable Executable: \ref{win32_pe}}
\acro{SP}{\gls{stack pointer}. SP/ESP/RSP \InENRU x86/x64. SP \InENRU ARM.}
\acro{DLL}{Dynamic-link library}
\acro{PC}{Program Counter. IP/EIP/RIP \InENRU x86/64. PC \InENRU ARM.}
\acro{LR}{Link Register}
\acro{IDA}{Interactive Disassembler}
\acro{IAT}{Import Address Table}
\acro{INT}{Import Name Table}
\acro{RVA}{Relative Virtual Address}
\acro{VA}{Virtual Address}
\acro{OEP}{Original Entry Point}
\acro{MSVC}{Microsoft Visual C++}
\acro{MSVS}{Microsoft Visual Studio}
\acro{ASLR}{Address Space Layout Randomization}
\acro{MFC}{Microsoft Foundation Classes}
\acro{TLS}{Thread Local Storage}
\acro{AKA}{Also Known As\RU{ (Также известный как)}}
\acro{CRT}{C runtime library: {sec:CRT}}
\acro{CPU}{Central processing unit}
\acro{FPU}{Floating-point unit}
\acro{CISC}{Complex instruction set computing}
\acro{RISC}{Reduced instruction set computing}
\acro{GUI}{Graphical user interface}
\acro{RTTI}{Run-time type information}
\acro{BSS}{Block Started by Symbol}
\acro{SIMD}{Single instruction, multiple data}
\acro{BSOD}{Black Screen of Death}
\acro{DBMS}{Database management systems}
\acro{ISA}{Instruction Set Architecture\RU{ (Архитектура набора команд)}}
\acro{CGI}{Common Gateway Interface}
\acro{HPC}{High-Performance Computing}
\acro{SOC}{System on Chip}
\acro{SEH}{Structured Exception Handling: \ref{sec:SEH}}
\acro{ELF}{\RU{Формат исполняемых файлов, использующийся в Linux и некоторых других *NIX}
	\EN{Executable file format widely used in *NIX system including Linux}}
\acro{TIB}{Thread Information Block}
\acro{TEA}{Tiny Encryption Algorithm}
\acro{PIC}{Position Independent Code: \ref{sec:PIC}}
\acro{NAN}{Not a Number}
\acro{NOP}{No OPeration}
\acro{BEQ}{(PowerPC, ARM) Branch if Equal}
\acro{BNE}{(PowerPC, ARM) Branch if Not Equal}
\acro{BLR}{(PowerPC) Branch to Link Register}
\acro{XOR}{eXclusive OR\RU{ (исключающее ``ИЛИ'')}}
\acro{MCU}{Microcontroller unit}
\acro{RAM}{Random-access memory}
\acro{EGA}{Enhanced Graphics Adapter}
\acro{VGA}{Video Graphics Array}
\acro{API}{Application programming interface}
\acro{ASCII}{American Standard Code for Information Interchange}
\acro{ASCIIZ}{ASCII Zero (\RU{ASCII-строка заканчивающаяся нулем}\EN{null-terminated ASCII string})}
\acro{IA64}{Intel Architecture 64 (Itanium): \ref{itanium}}
\acro{EPIC}{Explicitly parallel instruction computing}
\acro{OOE}{Out-of-order execution}
\acro{MSDN}{Microsoft Developer Network}
\acro{MSB}{Most significant bit/byte\RU{ (самый старший бит/байт)}}
\acro{LSB}{Least significant bit/byte\RU{ (самый младший бит/байт)}}
\acro{STL}{(\Cpp) Standard Template Library: \ref{sec:STL}}
\acro{PODT}{(\Cpp) Plain Old Data Type}
\acro{HDD}{Hard disk drive}
\acro{VM}{Virtual Memory\RU{ (виртуальная память)}}
\acro{WRK}{Windows Research Kernel}
\acro{GPR}{General Purpose Registers\RU{ (регистры общего пользования)}}
\acro{SSDT}{System Service Dispatch Table}
\acro{RE}{Reverse Engineering}
\acro{SSE}{Streaming SIMD Extensions}
\acro{BCD}{Binary-coded decimal}
\acro{BOM}{Byte order mark}
\acro{GDB}{GNU debugger}
\acro{FP}{Frame Pointer}
\acro{MBR}{Master Boot Record}
\acro{JPE}{Jump Parity Even (\RU{инструкция x86}\EN{x86 instruction})}
\acro{CIDR}{Classless Inter-Domain Routing}
\acro{STMFD}{Store Multiple Full Descending (\RU{инструкция ARM}\EN{ARM instruction})}
\acro{LDMFD}{Load Multiple Full Descending (\RU{инструкция ARM}\EN{ARM instruction})}
\acro{STMED}{Store Multiple Empty Descending (\RU{инструкция ARM}\EN{ARM instruction})}
\acro{LDMED}{Load Multiple Empty Descending (\RU{инструкция ARM}\EN{ARM instruction})}
\acro{STMFA}{Store Multiple Full Ascending (\RU{инструкция ARM}\EN{ARM instruction})}
\acro{LDMFA}{Load Multiple Full Ascending (\RU{инструкция ARM}\EN{ARM instruction})}
\acro{STMEA}{Store Multiple Empty Ascending (\RU{инструкция ARM}\EN{ARM instruction})}
\acro{LDMEA}{Load Multiple Empty Ascending (\RU{инструкция ARM}\EN{ARM instruction})}
\acro{APSR}{(ARM) Application Program Status Register}
\acro{FPSCR}{(ARM) Floating-Point Status and Control Register}
\acro{PID}{\RU{ID программы/процесса}\EN{Program/process ID}}
\acro{LF}{Line feed\RU{ (подача строки)} (10 \OrENRU '\textbackslash{}n' \InENRU \CCpp)}
\acro{CR}{Carriage return\RU{ (возврат каретки)} (13 \OrENRU '\textbackslash{}r' \InENRU \CCpp)}

\end{acronym}

