% TODO translate to Russian
\section{Double negation}

A popular way\footnote{This way is also controversial, because it leads to hard-to-read code} to convert non-zero value into 1 (or boolean \IT{true}) and zero value into 0 (or boolean \IT{false}) is \IT{!!statement}:

\begin{lstlisting}
int convert_to_bool(int a)
{
	return !!a;
};
\end{lstlisting}

Optimizing GCC 5.4 x86:

\begin{lstlisting}
convert_to_bool:
        mov     edx, DWORD PTR [esp+4]
        xor     eax, eax
        test    edx, edx
        setne   al
        ret
\end{lstlisting}

\INS{XOR} always clears return value in \EAX in case \INS{SETNE} will not trigger.
I.e., \INS{XOR} sets default return value to zero.

\myindex{x86!\Instructions!SET}
If the input value is not equal to zero (\TT{-NE} suffix in \INS{SET} instruction),
1 is set to \AL, otherwise \AL isn't touched.

Why \INS{SETNE} operates on low 8-bit part of \EAX register?
Because the matter is just in the last bit (0 or 1), while other bits are cleared by \INS{XOR}.

Therefore, that C/C++ code could be rewritten like this:

\begin{lstlisting}
int convert_to_bool(int a)
{
	if (a!=0)
		return 1;
	else
		return 0;
};
\end{lstlisting}

\dots or even:

\begin{lstlisting}
int convert_to_bool(int a)
{
	if (a)
		return 1;
	else
		return 0;
};
\end{lstlisting}

Compilers targeting \ac{CPU}s lacking instruction similar to \INS{SET}, in this case,
generates branching instructions, etc.

