\chapter{\RU{Конвертирование температуры}\EN{Temperature converting}}

\RU{Еще один крайне популярный пример из книг по программированию для начинающих, это простейшая программа
для конвертирования температуры по Фаренгейту в температуру по Цельсию}\EN{Another very popular example in 
programming books for beginners is a small program that converts Fahrenheit temperature to Celsius or back}.

\[
	C=\frac{5 \cdot (F-32)}{9}
\]

\RU{Мы также добавим простейшую обработку ошибок}\EN{We can also add simple error handling}:
1) \RU{мы должны проверять правильность ввода пользователем}\EN{we must check if the user has entered a correct number};
2) \RU{мы должны проверять результат, не ниже ли он $-273$ по Цельсию (что, как мы можем помнить из школьных
уроков физики, ниже абсолютного ноля)}\EN{we must check if the Celsius temperature is not below $-273$ 
(which is below absolute zero, as we may remember from school physics lessons)}.

\index{\CStandardLibrary!exit()}
\EN{The}\RU{Функция} \TT{exit()} \RU{заканчивает программу тут же, без возврата в вызывающую функцию}\EN{function terminates 
the program instantly, without returning to the \gls{caller} function}.

\section{\RU{Целочисленные значения}\EN{Integer values}}

\lstinputlisting{\CURPATH/i.c}

\subsection{\Optimizing MSVC 2012 x86}

\lstinputlisting[caption=\Optimizing MSVC 2012 x86]{\CURPATH/i_MSVC_2012_Ox_x86.asm.\LANG}

\RU{Что мы можем сказать об этом}\EN{What we can say about it}:

\begin{itemize}
\item \RU{Адрес функции}\EN{The address of} \printf \RU{в начале загружается в регистр}\EN{is first loaded in the} 
\ESI \RU{так что последующие вызовы}\EN{register, so the subsequent}
\printf \RU{происходят просто при помощи инструкции}\EN{calls are done just by the} \TT{CALL ESI}\EN{ instruction}.
\RU{Это очень популярная техника компиляторов, может присутствовать, если имеются несколько вызовов
одной и той же функции в одном месте, и/или имеется свободный регистр для этого}\EN{It's a very popular compiler 
technique, possible if several consequent calls to the same function are present
in the code, and/or if there is a free register which can be used for this}.

\item \RU{Мы видим инструкцию}\EN{We see the} \TT{ADD EAX, -32} \RU{в том месте где от значения должно отняться 32}
\EN{instruction at the place where 32 has to be subtracted from the value}.
$EAX=EAX+(-32)$ \RU{эквивалентно}\EN{is equivalent to} $EAX=EAX-32$ \RU{и как-то компилятор решил использовать}
\EN{and somehow, the compiler decided to use} \TT{ADD} \RU{вместо}\EN{instead of} \TT{SUB}.
\RU{Может быть оно того стоит, но я не уверен.}\EN{Maybe it's worth it, but I'm not sure.}

\index{x86!\Instructions!LEA}
\item \EN{The}\RU{Инструкция} \LEA \RU{используются там, где нужно умножить значение на}\EN{instruction is used when 
the value is to be multiplied by} 5: \TT{lea ecx, DWORD PTR [eax+eax*4]}.
\RU{Да}\EN{Yes}, $i+i*4$ \RU{эквивалентно}\EN{is equivalent to} $i*5$ \AndENRU \LEA \RU{работает быстрее чем}
\EN{works faster then} \TT{IMUL}.
\RU{Кстати, пара инструкций}\EN{By the way, the} \TT{SHL EAX, 2 / ADD EAX, EAX} \RU{может быть использована здесь
вместо \LEA\EMDASH{}некоторые компиляторы так и делают}\EN{instruction pair could be also used here instead---
some compilers do it like}.

\item \RU{Деление через умножение}\EN{The division by multiplication trick} (\myref{sec:divisionbynine}) 
\RU{также используется здесь}\EN{is also used here}.

\item \RU{Функция }\main \RU{возвращает}\EN{returns} 0 \RU{хотя}\EN{if we don't have} \TT{return 0} 
\RU{в конце функции отсутствует}\EN{at its end}.
\RU{В стандарте C99}\EN{The C99 standard tells us} \cite[5.1.2.2.3]{C99TC3} \RU{указано что}\EN{that} \main 
\RU{будет возвращать}\EN{will return} 0 \RU{в случае отсутствия выражения}\EN{in case the} 
\TT{return}\EN{ statement is missing}.
\RU{Это правило работает только для функции \main}\EN{This rule works only for the \main function}.
\RU{И хотя, MSVC официально не поддерживает C99, может быть частично и поддерживает?}
\EN{Though, MSVC doesn't officially support C99, but maybe it support it partially?}
\end{itemize}

\subsection{\Optimizing MSVC 2012 x64}

\RU{Код почти такой же, хотя я заметил инструкцию}\EN{The code is almost the same, but I've found} 
\TT{INT 3} \RU{после каждого вызова}\EN{instructions after each} \TT{exit()}\EN{ call}:

\begin{lstlisting}
	xor	ecx, ecx
	call	QWORD PTR __imp_exit
	int	3
\end{lstlisting}

\TT{INT 3} \RU{это точка останова для отладчика}\EN{is a debugger breakpoint}.

\RU{Известно что функция}\EN{It is known that} \TT{exit()} \RU{из тех, что никогда не возвращают
управление}\EN{is one of the functions which can never return}
\footnote{\RU{еще одна популярная из того же ряда это}\EN{another popular one is} \TT{longjmp()}},
\RU{так что если управление все же возвращается, значит происходит что-то крайне странное, и пришло
время запускать отладчик}\EN{so if it does, something really odd has happened and it's time to load the debugger}.

\section{\RU{Числа с плавающей запятой}\EN{Floating-point values}}

\lstinputlisting{\CURPATH/f.c}

MSVC 2010 x86 \RU{использует инструкции}\EN{uses} \ac{FPU}\EN{ instructions}\dots

\lstinputlisting[caption=\Optimizing MSVC 2010 x86]{\CURPATH/f_MSVC_2010_x86_Ox.asm.\LANG}

\dots \RU{но}\EN{but} MSVC \RU{от года} 2012 \RU{использует инструкции}\EN{uses} \ac{SIMD} 
\RU{вместо этого}\EN{instructions instead}:

\lstinputlisting[caption=\Optimizing MSVC 2010 x86]{\CURPATH/f_MSVC_2012_x86_Ox.asm.\LANG}

\RU{Конечно}\EN{Of course}, \ac{SIMD}\RU{-инструкции доступны и в x86-режиме, включая те что работают
с числами с плавающей запятой}\EN{ instructions are available in x86 mode, 
including those working with floating point numbers}.
\RU{Их использовать в каком-то смысле проще, так что новый компилятор от Microsoft теперь применяет их}
\EN{It's somewhat easier to use them for calculations, so the new Microsoft compiler uses them}.

\RU{Мы можем также заметить, что значение}\EN{We can also see that the} $-273$ \RU{загружается в регистр}\EN{value 
is loaded into} \XMM{0} \RU{слишком рано}\EN{register too early}.
\RU{И это нормально, потому что компилятор может генерировать инструкции далеко не в том порядке, в котором
они появляются в исходном коде}\EN{And that's OK, because the compiler may emit instructions not in 
the order they are in the source code}.
