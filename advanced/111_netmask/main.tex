\chapter{\RU{Пример вычисления адреса сети}\EN{Network address calculation example}}

\EN{As we know, a TCP/IP address (IPv4) consists of four numbers in the $0 \ldots 255$ range, i.e., four bytes.}
\RU{Как мы знаем, TCP/IP-адрес (IPv4) состоит из четырех чисел в пределах $0 \ldots 255$, т.е. 4 байта.}
\EN{Four bytes can be fit in a 32-bit variable easily, so an IPv4 host address, network mask or network address
can all be 32-bit integers.}
\RU{4 байта легко помещаются в 32-битную переменную, так что адрес хоста в IPv4, сетевая маска или адрес сети
могут быть 32-битными числами.}

\EN{From the user's point of view, the network mask is defined as four numbers and is formatted like 255.255.255.0 or so,
but network engineers (sysadmins) use a more compact notation (\ac{CIDR}), like /8, /16 or similar.}
\RU{С точки зрения пользователя, маска сети определяется четырьмя числами в формате вроде 255.255.255.0,
но сетевые инженеры (сисадмины) используют более компактную нотацию (\ac{CIDR}), вроде  /8, /16, \etc{}.}
\EN{This notation just defines the number of bits the mask has, starting at the \ac{MSB}.}
\RU{Эта нотация просто определяет количество бит в сетевой маске, начиная с \ac{MSB}.}

\begin{center}
\begin{tabular}{ | l | l | l | l | l | l | }
\hline
\cellcolor{blue!25} \EN{Mask}\RU{Маска} & 
\cellcolor{blue!25} \EN{Hosts}\RU{Хосты} & 
\cellcolor{blue!25} \EN{Usable}\RU{Свободно} &
\cellcolor{blue!25} \EN{Netmask}\RU{Сетевая маска} &
\cellcolor{blue!25} \EN{Hex mask}\RU{В шестнадцатеричном виде} &
\cellcolor{blue!25} \\
\hline
/30  & 4        & 2        & 255.255.255.252  & \TT{fffffffc}  & \\
\hline
/29  & 8        & 6        & 255.255.255.248  & \TT{fffffff8}  & \\
\hline
/28  & 16       & 14       & 255.255.255.240  & \TT{fffffff0}  & \\
\hline
/27  & 32       & 30       & 255.255.255.224  & \TT{ffffffe0}  & \\
\hline
/26  & 64       & 62       & 255.255.255.192  & \TT{ffffffc0}  & \\
\hline
/24  & 256      & 254      & 255.255.255.0    & \TT{ffffff00}  & \RU{сеть класса C}\EN{class C network} \\
\hline
/23  & 512      & 510      & 255.255.254.0    & \TT{fffffe00}  & \\
\hline
/22  & 1024     & 1022     & 255.255.252.0    & \TT{fffffc00}  & \\
\hline
/21  & 2048     & 2046     & 255.255.248.0    & \TT{fffff800}  & \\
\hline
/20  & 4096     & 4094     & 255.255.240.0    & \TT{fffff000}  & \\
\hline
/19  & 8192     & 8190     & 255.255.224.0    & \TT{ffffe000}  & \\
\hline
/18  & 16384    & 16382    & 255.255.192.0    & \TT{ffffc000}  & \\
\hline
/17  & 32768    & 32766    & 255.255.128.0    & \TT{ffff8000}  & \\
\hline
/16  & 65536    & 65534    & 255.255.0.0      & \TT{ffff0000}  & \RU{сеть класса B}\EN{class B network} \\
\hline
/8   & 16777216 & 16777214 & 255.0.0.0        & \TT{ff000000}  & \RU{сеть класса A}\EN{class A network} \\
\hline
\end{tabular}
\end{center}

\EN{Here is a small example, which calculates the network address by applying the network mask to the host address.}
\RU{Вот простой пример, вычисляющий адрес сети используя сетевую маску и адрес хоста.}

\lstinputlisting{\CURPATH/netmask.c}

\section{calc\_network\_address()}

\RU{Функция }\TT{calc\_network\_address()} \RU{самая простая}\EN{function is simplest one}: 
\EN{it just ANDs the host address with the network mask, resulting in the network address.}
\RU{она просто умножает (логически, используя \AND) адрес хоста на сетевую маску, в итоге давая адрес
сети.}

\lstinputlisting[caption=\Optimizing MSVC 2012 /Ob0,numbers=left]{\CURPATH/calc_network_address_MSVC_2012_Ox.asm}

\EN{At line 22 we see the most important \AND\EMDASH{}here the network address is calculated.}
\RU{На строке 22 мы видим самую важную инструкцию \AND\EMDASH{}так вычисляется адрес сети.}

\section{form\_IP()}

\EN{The}\RU{Функция} \TT{form\_IP()} \EN{function just puts all 4 bytes into a 32-bit value}\RU{просто собирает
все 4 байта в одно 32-битное значение}.

\RU{Вот как это обычно происходит}\EN{Here is how it is usually done}:

\begin{itemize}
\item \RU{Выделите переменную для возвращаемого значения. Обнулите её.}\EN{Allocate a variable for the return value. 
Set it to 0.}

\item \EN{Take the fourth (lowest) byte, apply OR operation to this byte and return the value.
The return value contain the 4th byte now.}
\RU{Возьмите четвертый (самый младший) байт, сложите его (логически, инструкцией \OR) с возвращаемым
значением. Оно содержит теперь 4-й байт.}

\item \EN{Take the third byte, shift it left by 8 bits}\RU{Возьмите третий байт, сдвиньте его на 8 бит влево}.
\EN{You'll get a value like \TT{0x0000bb00} where \TT{bb} is your third byte.}
\RU{Получится значение в виде \TT{0x0000bb00}, где \TT{bb} это третий байт.}
\EN{Apply the OR operation to the resulting value and it.}
\RU{Сложите итоговое значение (логически, инструкцией \OR) с возвращаемым значением.}
\EN{The return value has contained \TT{0x000000aa} so far, so ORing the values will produce a value 
like \TT{0x0000bbaa}.}
\RU{Возвращаемое значение пока что содержит \TT{0x000000aa}, так что логическое сложение
в итоге выдаст значение вида \TT{0x0000bbaa}.}

\item \EN{Take the second byte, shift it left by 16 bits.}
\RU{Возьмите второй байт, сдвиньте его на 16 бит влево.}
\EN{You'll get a value like \TT{0x00cc0000}, where \TT{cc} is your second byte.}
\RU{Вы получите значение вида \TT{0x00cc0000}, где \TT{cc} это второй байт.}
\EN{Apply the OR operation to the resulting value and return it.}
\RU{Сложите (логически) результат и возвращаемое значение.}
\EN{The return value has contained \TT{0x0000bbaa} so far, so ORing the values will produce
a value like \TT{0x00ccbbaa}.}
\RU{Выходное значение содержит пока что \TT{0x0000bbaa}, так что логическое сложение
в итоге выдаст значение вида \TT{0x00ccbbaa}.}

\item \EN{Take the first byte, shift it left by 24 bits.}
\RU{Возьмите первый байт, сдвиньте его на 24 бита влево.}
\EN{You'll get a value like \TT{0xdd000000}, where \TT{dd} is your first byte.}
\RU{Вы получите значение вида \TT{0xdd000000}, где \TT{dd} это первый байт.}
\EN{Apply the OR operation to the resulting value and return it.}
\RU{Сложите (логически) результат и выходное значение.}
\EN{The return value contain \TT{0x00ccbbaa} so far, so ORing the values will produce
a value like \TT{0xddccbbaa}.}
\RU{Выходное значение содержит пока что \TT{0x00ccbbaa}, так что сложение выдаст в итоге значение
вида \TT{0xddccbbaa}.}

\end{itemize}

\EN{And this is how it's done by non-optimizing MSVC 2012:}
\RU{И вот как работает неоптимизирующий MSVC 2012:}

\lstinputlisting[caption=\NonOptimizing MSVC 2012]{\CURPATH/form_IP_MSVC_2012.asm.\LANG}

\EN{Well, the order is different, but, of course, the order of the operations doesn't matters.}
\RU{Хотя, порядок операций другой, но, конечно, порядок роли не играет.}

\Optimizing MSVC 2012 \EN{does essentially the same, but in a different way}\RU{делает то же самое,
но немного иначе}:

\lstinputlisting[caption=\Optimizing MSVC 2012 /Ob0]{\CURPATH/form_IP_MSVC_2012_Ox.asm.\LANG}

\EN{We could say that each byte is written to the lowest 8 bits of the return value, 
and then the return value is shifted left by one byte at each step.}
\RU{Можно сказать, что каждый байт записывается в младшие 8 бит возвращаемого значения,
и затем возвращаемое значение сдвигается на один байт влево на каждом шаге.}
\EN{Repeat 4 times for each input byte.}
\RU{Повторять 4 раза, для каждого байта.}\ESph{}\PTBRph{}\\
\\
\RU{Вот и всё}\EN{That's it}! \EN{Unfortunately, there are probably no other ways to do it.}
\RU{К сожалению, наверное, нет способа делать это иначе.}
\EN{There are no popular \ac{CPU}s or \ac{ISA}s which has instruction for composing a value from
bits or bytes.}%
\RU{Не существует более-менее популярных \ac{CPU} или \ac{ISA}, где имеется инструкция для сборки значения из бит или байт.}
\EN{It's all usually done by bit shifting and ORing.}%
\RU{Обычно всё это делает сдвигами бит и логическим сложением (OR).}

\section{print\_as\_IP()}

\TT{print\_as\_IP()} \EN{does the inverse: splitting a 32-bit value into 4 bytes.}
\RU{делает наоборот: расщепляет 32-битное значение на 4 байта.}

\EN{Slicing works somewhat simpler: just shift input value by 24, 16, 8 or 0 bits, take the 
bits from zeroth to seventh (lowest byte), and that's it:}
\RU{Расщепление работает немного проще: просто сдвигайте входное значение на 24, 16, 8 или 0 бит,
берите биты с нулевого по седьмой (младший байт), вот и всё:}

\lstinputlisting[caption=\NonOptimizing MSVC 2012]{\CURPATH/print_as_IP_MSVC_2012.asm.\LANG}

\Optimizing MSVC 2012 \EN{does almost the same, but without unnecessary reloading of the input value:}
\RU{делает почти всё то же самое, только без ненужных перезагрузок входного значения:}

\lstinputlisting[caption=\Optimizing MSVC 2012 /Ob0]{\CURPATH/print_as_IP_MSVC_2012_Ox.asm}

\section{form\_netmask() \AndENRU set\_bit()}

\TT{form\_netmask()} \EN{makes a network mask value from \ac{CIDR} notation}\RU{делает сетевую маску из
\ac{CIDR}-нотации}.
\EN{Of course, it would be much effective to use here some kind of a precalculated table, but we consider it in this
way intentionally, to demonstrate bit shifts.}
\RU{Конечно, было бы куда эффективнее использовать здесь какую-то уже готовую таблицу, но мы рассматриваем
это именно так, сознательно, для демонстрации битовых сдвигов.}
\RU{Мы также сделаем отдельную функцию}\EN{We will also write a separate function} \TT{set\_bit()}. 
\EN{It's a not very good idea to create a function
for such primitive operation, but it would be easy to understand how it all works.}
\RU{Не очень хорошая идея выделять отдельную функцию для такой примитивной операции, но так будет проще понять,
как это всё работает.}

\lstinputlisting[caption=\Optimizing MSVC 2012 /Ob0]{\CURPATH/form_netmask_MSVC_2012_Ox.asm}

\TT{set\_bit()} \RU{примитивна: просто сдвигает единицу на нужное количество бит, затем складывает (логически) с
входным значением \q{input}}\EN{is primitive: it just shift left 1 to number of bits we need and then 
ORs it with the \q{input} value}.
\TT{form\_netmask()} \EN{has a loop: it will set as many bits (starting from the \ac{MSB}) as 
passed in the \TT{netmask\_bits} argument}\RU{имеет цикл: он выставит столько бит (начиная с \ac{MSB}), 
сколько передано в аргументе \TT{netmask\_bits}.}

\section{\RU{Итог}\EN{Summary}}

\RU{Вот и всё}\EN{That's it}!
\RU{Мы запускаем и видим}\EN{We run it and getting}:

\begin{lstlisting}
netmask=255.255.255.0
network address=10.1.2.0
netmask=255.0.0.0
network address=10.0.0.0
netmask=255.255.255.128
network address=10.1.2.0
netmask=255.255.255.192
network address=10.1.2.64
\end{lstlisting}
