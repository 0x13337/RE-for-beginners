\chapter{\RU{Циклы: несколько итераторов}\EN{Loops: several iterators}}
\label{loop_iterators}

\RU{Часто, у цикла только один итератор, но в итоговом коде их может быть несколько.}
\EN{In most cases loops have only one iterator, but there could be several in the resulting code.}

\RU{Вот очень простой пример}\EN{Here is a very simple example}:

\lstinputlisting{\CURPATH/iterators.c.\LANG}

\RU{Здесь два умножения на каждой итерации, а это дорогая операция.}
\EN{There are two multiplications at each iteration and they are costly operations.}
\RU{Сможем ли мы соптимизировать это как-то}\EN{Can we optimize it somehow}?
\RU{Да, если мы заметим, что индексы обоих массивов перескакивают на места, рассчитать которые мы
можем легко и без умножения.}
\EN{Yes, if we notice that both array indices are jumping on values that we can easily calculate without 
multiplication.}

\section{\RU{Три итератора}\EN{Three iterators}}

\lstinputlisting[caption=\Optimizing MSVC 2013 x64]{\CURPATH/MSVC_2013_x64_Ox.asm.\LANG}

\RU{Теперь здесь три итератора: переменная \IT{cnt} и два индекса, они увеличиваются на 12 и 28 на каждой
итерации, указывая на новые элементы массивов.}
\EN{Now there are 3 iterators: the \IT{cnt} variable and two indices, which are increased by 12 and 28 at 
each iteration.}
\RU{Мы можем переписать этот код на}\EN{We can rewrite this code in} \CCpp:

\lstinputlisting{\CURPATH/iterators3.c.\LANG}

\RU{Так что, ценой модификации трех итераторов на каждой итерации вместо одного, 
мы избавлены от двух операций умножения.}
\EN{So, at the cost of updating 3 iterators at each iteration instead of one, 
we can remove two multiplication operations.}

\section{\RU{Два итератора}\EN{Two iterators}}

\RU{GCC 4.9 сделал еще больше, оставив только 2 итератора:}
\EN{GCC 4.9 does even more, leaving only 2 iterators:}

\lstinputlisting[caption=\Optimizing GCC 4.9 x64]{\CURPATH/GCC491_x64_O3.asm.\LANG}

\RU{Здесь больше нет переменной-\IT{счетчика}: GCC рассудил, что она не нужна.}
\EN{There is no \IT{counter} variable any more: GCC concluded that it is not needed.}
\RU{Последний элемент массива \IT{a2} вычисляется перед началом цикла (а это просто: $cnt*7$),
и при помощи этого цикл останавливается: просто исполняйте его пока второй индекс не сравняется
с предвычисленным значением.}
\EN{The last element of the \IT{a2} array is calculated before the loop begins (which is easy: $cnt*7$) 
and that's how the loop is to be stopped: just iterate until the second index has not reached this precalculated value.}

\RU{Об умножении используя сдвиги/сложения/вычитания, читайте здесь:}
\EN{You can read more about multiplication using shifts/additions/subtractions here:} 
\myref{multiplication_using_shifts_adds_subs}.

\RU{Этот код можно переписать на \CCpp вот так:}
\EN{This code can be rewritten into \CCpp like that:}

\lstinputlisting{\CURPATH/iterators2.c.\LANG}

\ifdefined\IncludeARM
GCC (Linaro) 4.9 \ForENRU ARM64 \RU{делает тоже самое, только предвычисляет последний индекс массива \IT{a1} вместо
\IT{a2}, а это, конечно, имеет тот же эффект}\EN{does the same, but it precalculates the last index of \IT{a1} 
instead of \IT{a2}, which, of course has the same effect}:

\lstinputlisting[caption=\Optimizing GCC (Linaro) 4.9 ARM64,label=GCC_no_number_sign]{\CURPATH/ARM64_GCC_49_O3.s.\LANG}

% FIXME1 -> post-increment
\fi

\ifdefined\IncludeMIPS
GCC 4.4.5 \ForENRU MIPS \RU{делает то же самое}\EN{does the same}:

\lstinputlisting[caption=\Optimizing GCC 4.4.5 \ForENRU MIPS (IDA)]{\CURPATH/MIPS_O3_IDA.lst.\LANG}
\fi

\ifx\LITE\undefined
\section{\RU{Случай }Intel C++ 2011\EN{ case}}
\index{\CompilerAnomaly}
\label{loops_iterators_loop_anomaly}

\RU{Оптимизации компилятора могут быть очень странными, но, тем не менее, корректными.}
\EN{Compiler optimizations can also be weird, but nevertheless, still correct.}
\RU{Вот что делает Intel C++ 2011}\EN{Here is what the Intel C++ compiler 2011 does}:

\lstinputlisting[caption=\Optimizing Intel C++ 2011 x64]{\CURPATH/intel_2011_x64_Ox.asm}

\RU{В начале, принимаются какие-то решения, затем исполняется одна из процедур.}
\EN{First, there are some decisions taken, then one of the routines is executed.}
\RU{Вероятно, это проверка, не пересекаются ли массивы.}
\EN{Looks like it is a check if arrays intersect.}
\RU{Это хорошо известный способ оптимизации процедур копирования блоков в памяти.}
\EN{This is very well known way of optimizing memory block copy routines.}
\RU{Но копирующие процедуры одинаковые}\EN{But copy routines are the same}!
\RU{Видимо, это ошибка оптимизатора Intel C++, который, тем не менее, генерирует работоспособный код.}
\EN{This is probably an error of the Intel C++ optimizer, which still produces workable code, though.}

\RU{Мы намеренно изучаем примеры такого кода в этой книге чтобы читатель мог понимать, что результаты работы
компилятором иногда бывают крайне странными, но корректными, потому что когда компилятор тестировали, 
тесты прошли нормально.}%
\EN{We intentionally considering such example code in this book so the reader would understand that compiler output is weird at times,
but still correct, because when the compiler was tested, is passed the tests.}
\fi
