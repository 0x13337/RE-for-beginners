\chapter{Duff's device}
\index{Duff's device}

Duff's device 
\EN{\footnote{\href{http://go.yurichev.com/17137}{wikipedia}}}
\RU{\footnote{\href{http://go.yurichev.com/17138}{wikipedia}}}
\RU{это развернутый цикл с возможностью перехода внутри цикла.}
\EN{is an unrolled loop with the possibility to jump inside it.}
\RU{Развернутый цикл реализован используя fallthrough-выражение switch().}
\EN{The unrolled loop is implemented using a fallthrough switch() statement.}

\RU{Мы будем использовать здесь упрощенную версию кода Тома Даффа.}
\EN{We would use here a slightly simplified version of Tom Duff's original code.}

\RU{Скажем, нам нужно написать функцию, очищающую регион в памяти.}
\EN{Let's say, we need to write a function that clears a region in memory.}
\RU{Кто-то может подумать о простом цикле, очищающем байт за байтом.}
\EN{One can come with a simple loop, clearing byte by byte.}
\RU{Это, очевидно, медленно, так как все современные компьютеры имеют намного более широкую шину памяти.}
\EN{It's obviously slow, since all modern computers have much wider memory bus.}
\RU{Так что более правильный способ\EMDASH{}это очищать регион в памяти блоками по 4 или 8 байт.}
\EN{So the better way is to clear the memory region using 4 or 8 byte blocks.}
\RU{Так как мы будем работать с 64-битным примером, мы будем очищать память блоками по 8 байт.}
\EN{Since we are going to work with a 64-bit example here, we are going to clear the memory in 8 byte blocks.}
\RU{Пока всё хорошо}\EN{So far so good}.
\RU{Но что насчет хвоста}\EN{But what about the tail}? 
\RU{Функция очистки памяти будет также вызываться и для блоков с длиной не кратной 8.}
\EN{Memory clearing routine can also be called for regions of size that's not a multiple of 8.}

\RU{Вот алгоритм}\EN{So here is the algorithm}:

\begin{itemize}
\item \RU{вычислить количество 8-байтных блоков, очистить их используя 8-байтный (64-битный) доступ к памяти;}
\EN{calculate the number of 8-byte blocks, clear them using 8-byte (64-bit) memory accesses;}

\item \RU{вычислить размер хвоста, очистить его используя 1-байтный доступ к памяти.}
\EN{calculate the size of the tail, clear it using 1-byte memory accesses.}
\end{itemize}

\RU{Второй шаг можно реализовать, используя простой цикл}\EN{The second step can be implemented using a simple loop}.
\RU{Но давайте реализуем его используя развернутый цикл}\EN{But let's implement it as an unrolled loop}:

\lstinputlisting{\CURPATH/duff.c.\LANG}

\RU{В начале разберемся, как происходят вычисления.}\EN{Let's first understand how the calculation is done.}
\RU{Размер региона в памяти приходит в 64-битном значении.}\EN{The memory region size comes as a 64-bit value.}
\RU{И это значение можно разделить на две части:}\EN{And this value can be divided in two parts:}

% .... 7 6 5 4 3 2 1 0
%|....|B|B|B|B|B|S|S|S|

\begin{center}
\begin{bytefield}[endianness=big,bitwidth=0.03\linewidth]{11}
\bitheader{7,6,5,4,3,2,1,0} \\
\bitbox{3}{\dots} & 
\bitbox{1}{B} & 
\bitbox{1}{B} & 
\bitbox{1}{B} & 
\bitbox{1}{B} & 
\bitbox{1}{B} & 
\bitbox{1}{S} & 
\bitbox{1}{S} & 
\bitbox{1}{S}
\end{bytefield}
\end{center}

\RU{( \q{B} это количество 8-байтных блоков и \q{S} это длина хвоста в байтах ).}
\EN{( \q{B} is number of 8-byte blocks and \q{S} is length of the tail in bytes ).}

\RU{Если разделить размер входного блока в памяти на 8, то значение просто сдвигается на 3 бита вправо.}
\EN{When we divide the input memory region size by 8, the value is just shifted right by 3 bits.}
\RU{Но для вычисления остатка, нам нужно просто изолировать младшие 3 бита!}
\EN{But to calculate the remainder, we just need to isolate the lowest 3 bits!}
\RU{Так что количество 8-байтных блоков вычисляется как $count>>3$, а остаток как $count \& 7$.}
\EN{So the number of 8-byte blocks is calculated as $count>>3$ and remainder as $count \& 7$.}

\RU{В начале, нужно определить, будем ли мы вообще исполнять 8-байтную процедуру,
так что нам нужно узнать, не больше ли $count$ чем 7.}
\EN{We also need to find out if we are going to execute the 8-byte procedure at all, so we need
to check if the value of $count$ is greater than 7.}
\RU{Мы делаем это очищая младшие 3 бита и сравнивая результат с нулем, потому что,
всё что нам нужно узнать, это ответ на вопрос, содержит ли старшая часть значения $count$ ненулевые биты.}
\EN{We do this by clearing the 3 lowest bits and comparing the resulting number with zero, because 
all we need here is to answer the question, is the high part of $count$ non-zero.}

\RU{Конечно, это работает потому что 8 это $2^{3}$, так что деление на числа вида $2^n$ это легко.}
\EN{Of course, this works because 8 is $2^{3}$ and division by numbers that are $2^n$ is easy.}
\RU{Это невозможно с другими числами}\EN{It's not possible for other numbers}.

\RU{А на самом деле, трудно сказать, стоит ли пользоваться такими хакерскими трюками, потому что они
приводят к коду, который затем тяжело читать.}
\EN{It's actually hard to say if these hacks are worth using, because they lead
to hard-to-read code.}
\RU{С другой стороны, эти трюки очень популярны и практикующий программист, хотя может и не использовать
их, всё же должен их понимать.}
\EN{However, these tricks are very popular and a practicing programmer, 
even if he/she is not using them, nevertheless has to understand them.}

\RU{Так что первая часть простая: получить количество 8-байтных блоков и записать 64-битные нулевые значения
в память.}
\EN{So the first part is simple: get the number of 8-byte blocks and write 64-bit zero values to memory.}

\RU{Вторая часть\EMDASH{}это развернутый цикл реализованный как fallthrough-выражение switch().}
\EN{The second part is an unrolled loop implemented as fallthrough switch() statement.}
\RU{В начале, выразим на обычном русском языке, что мы хотим сделать.}
\EN{First, let's express in plain English what we have to do here.}
\RU{Мы должны \q{записать столько нулевых байт в память, сколько значение $count\&7$ нам говорит}.}
\EN{We have to \q{write as many zero bytes in memory, as $count\&7$ value tells us}.}
\RU{Если это 0, перейти на конец, больше ничего делать не нужно.}
\EN{If it's 0, jump to the end, there is no work to do.}
\RU{Если это 1, перейти на место внутри выражения switch(), где произойдет только одна операция записи.}
\EN{If it's 1, jump to the place inside switch() statement where only one storage operation
is to be executed.}
\RU{Если это 2, перейти на другое место, где две операции записи будут исполнены,}
\EN{If it's 2, jump to another place, where two storage operation are to be executed,}\etc{}.
\RU{7 во входном значении приведет к тому что исполнятся все 7 операций.}
\EN{7 as input value leads to the execution of all 7 operations.}
\RU{8 здесь нет, потому что регион памяти размером в 8 байт будет обработан первой частью нашей функции.}
\EN{There is no 8, because a memory region of 8 bytes is to be processed by the first part of our function.}

\RU{Так что мы сделали развернутый цикл}\EN{So we wrote an unrolled loop}.
\RU{Это однозначно работало быстрее обычных циклов на старых компьютерах
(и наоборот, на современных процессорах короткие циклы работают быстрее развернутых).}
\EN{It was definitely faster on older computers than normal loops (and conversely,
modern CPUs works better for short loops than for unrolled ones).}
\RU{Может быть, это всё еще может иметь смысл на современных маломощных дешевых \ac{MCU}.}
\EN{Maybe this is still meaningful on modern low-cost embedded \ac{MCU}s.}

\RU{Посмотрим, что сделает оптимизирующий MSVC 2012}\EN{Let's see what the optimizing MSVC 2012 does}:

\lstinputlisting{\CURPATH/duff_MSVC2012_x64_Ox.asm.\LANG}

\RU{Первая часть функции выглядит для нас предсказуемо.}
\EN{The first part of the function is predictable.}
\RU{Вторая часть\EMDASH{}это просто развернутый цикл и переход передает управление на нужную инструкцию
внутри него.}
\EN{The second part is just an unrolled loop and a jump passing control flow to the correct instruction
inside it.}
\RU{Между парами инструкций \TT{MOV}/\TT{INC} никакого другого кода нет, так что исполнение
продолжается до самого конца, исполняются столько пар, сколько нужно.}
\EN{There is no other code between the \TT{MOV}/\TT{INC} instruction pairs, 
so the execution is to fall until the very end, executing as many pairs as needed.}

\RU{Кстати, мы можем заметить, что пара \TT{MOV}/\TT{INC} занимает какое-то фиксированное количество
байт (3+3).}
\EN{By the way, we can observe that the \TT{MOV}/\TT{INC} pair consumes a fixed number of bytes (3+3).}
\RU{Так что пара занимает 6 байт}\EN{So the pair consumes 6 bytes}.
\RU{Зная это, мы можем избавиться от таблицы переходов в switch(), мы можем просто умножить входное значение
на 6 и перейти на \TT{текущий\_RIP + входное\_значение * 6}.}
\EN{Knowing that, we can get rid of the switch() jumptable, we can just multiple the input value by 6
and jump to $current\_RIP + input\_value * 6$.}
\RU{Это будет также быстрее, потому что не нужно будет загружать элемент из таблицы переходов (\IT{jumptable}).}
\EN{This can also be faster because we are not in need to fetch a value from the jumptable.}
\RU{Может быть, 6 не самая подходящая константа для быстрого умножения, и может быть оно того и не стоит,
но вы поняли идею\footnote{В качестве упражнения, вы можете попробовать переработать этот код и избавиться
от таблицы переходов.
\index{x86!\Instructions!STOSB}
Пару инструкций тоже можно переписать так что они будут занимать 4 байта или 8.
1 байт тоже возможен (используя инструкцию \TT{STOSB}).}.}
\EN{It's possible that 6 probably is not a very good constant for fast multiplication and maybe it's not worth it,
but you get the idea\footnote{As an exercise, you can try to rework the code to get rid of 
the jumptable. 
\index{x86!\Instructions!STOSB}
The instruction pair can be rewritten in a way that it will consume 4 bytes or maybe 8. 
1 byte is also possible (using \TT{STOSB} instruction).}.}
\RU{Так в прошлом делали с развернутыми циклами олд-скульные демомейкеры.}
\EN{That is what old-school demomakers did in the past with unrolled loops.}
