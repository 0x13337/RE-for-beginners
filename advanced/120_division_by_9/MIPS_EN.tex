\section{MIPS}

For some reason, optimizing GCC 4.4.5 generate just a division instruction:

\lstinputlisting[caption=\Optimizing GCC 4.4.5 (IDA)]{\CURPATH/MIPS_O3_IDA_EN.lst}

\myindex{MIPS!\Instructions!BREAK}
Here we see here a new instruction: BREAK. It just raises an exception.

In this case, an exception is raised if the divisor is zero (it's not possible to divide by zero 
in conventional math).

But GCC probably did not do very well the optimization job and did not see that \$V0 is never zero.

So the check is left here.
So if \$V0 is zero somehow, BREAK is to be executed, signaling to the \ac{OS} about the exception.

\myindex{MIPS!\Instructions!MFLO}
Otherwise, MFLO executes, which takes the result of the division from the LO register and copies it in \$V0.

\myindex{MIPS!\Registers!LO}
\myindex{MIPS!\Registers!HI}
By the way, as we may know, the MUL instruction leaves the high 32 bits of 
the result in register HI and the low 32 bits in register LO.

DIV leaves the result in the LO register, and remainder in the HI register.

\myindex{MIPS!\Instructions!MFHI}
If we alter the statement to \q{a \% 9}, 
the MFHI instruction is to be used here instead of MFLO.

