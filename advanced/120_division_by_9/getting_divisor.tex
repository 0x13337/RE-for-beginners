\section{\RU{Определение делителя}\EN{Getting the divisor}}

\subsection{\RU{Вариант}\EN{Variant} \#1}

\RU{Часто, код имеет вид}\EN{Often, the code looks like this}:

\lstinputlisting{\CURPATH/form.asm.\LANG}

\RU{Определим 32-битную \IT{magic}-коэффициент через}
\EN{Let's denote the 32-bit \IT{magic} coefficient as} $M$, 
\RU{коэффициент сдвига через}\EN{the shifting coefficient as} $C$ \RU{и делитель через}\EN{and the divisor as} $D$.

\RU{Делитель, который нам нужен это}\EN{The divisor we need to get is}:

\[
D=\frac{2^{32 + C}}{M}
\]

\RU{Например}\EN{For example}:

\lstinputlisting[caption=\Optimizing MSVC 2012]{\CURPATH/ex1.asm}

\RU{Это}\EN{This is}:

\[
D=\frac{2^{32 + 3}}{2021161081}
\]

\index{Wolfram Mathematica}
\RU{Числа больше чем 32-битные, так что мы можем использовать}
\EN{The numbers are larger than 32-bit, so we can use} Wolfram Mathematica \RU{для удобства}\EN{for convenience}:

\begin{lstlisting}[caption=Wolfram Mathematica]
In[1]:=N[2^(32+3)/2021161081]
Out[1]:=17.
\end{lstlisting}

\RU{Так что искомый делитель это}\EN{So the divisor from the code we used as example is} 17.

\RU{При делении в x64, всё то же самое, только нужно использовать}\EN{For x64 division, things 
are the same, but} $2^{64}$ \RU{вместо}\EN{has to be used instead of} $2^{32}$:

\begin{lstlisting}
uint64_t f1234(uint64_t a)
{
	return a/1234;
};
\end{lstlisting}

\begin{lstlisting}[caption=\Optimizing MSVC 2012 x64]
f1234	PROC
	mov	rax, 7653754429286296943		; 6a37991a23aead6fH
	mul	rcx
	shr	rdx, 9
	mov	rax, rdx
	ret	0
f1234	ENDP
\end{lstlisting}

\begin{lstlisting}[caption=Wolfram Mathematica]
In[1]:=N[2^(64+9)/16^^6a37991a23aead6f]
Out[1]:=1234.
\end{lstlisting}

\subsection{\RU{Вариант}\EN{Variant} \#2}

\RU{Бывает также вариант с пропущенным арифметическим сдвигом, например}\EN{A variant with 
omitted arithmetic shift also exist}:

\begin{lstlisting}
		mov     eax, 55555556h ; 1431655766
		imul    ecx
		mov     eax, edx
		shr     eax, 1Fh
\end{lstlisting}

\RU{Метод определения делителя упрощается}\EN{The method of getting divisor is simplified}:

\[
D=\frac{2^{32}}{M}
\]

\RU{Для моего примера, это}\EN{As of my example, this is}:

\[
D=\frac{2^{32}}{1431655766}
\]

\index{Wolfram Mathematica}
\RU{Снова используем}
\EN{And again we use} Wolfram Mathematica:

\begin{lstlisting}[caption=Wolfram Mathematica]
In[1]:=N[2^32/16^^55555556]
Out[1]:=3.
\end{lstlisting}

\RU{Искомый делитель это}\EN{The divisor is} 3.
