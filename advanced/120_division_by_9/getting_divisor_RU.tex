\section{Определение делителя}

\subsection{Вариант \#1}

Часто, код имеет вид:

\lstinputlisting{\CURPATH/form_RU.asm}

Определим 32-битную \IT{magic}-коэффициент через $M$, коэффициент сдвига через $C$ и делитель через $D$.

Делитель, который нам нужен это:

\[
D=\frac{2^{32 + C}}{M}
\]

Например:

\lstinputlisting[caption=\Optimizing MSVC 2012]{\CURPATH/ex1.asm}

Это:

\[
D=\frac{2^{32 + 3}}{2021161081}
\]

\myindex{Wolfram Mathematica}
Числа больше чем 32-битные, так что мы можем использовать Wolfram Mathematica для удобства:

\begin{lstlisting}[caption=Wolfram Mathematica]
In[1]:=N[2^(32+3)/2021161081]
Out[1]:=17.
\end{lstlisting}

Так что искомый делитель это 17.

При делении в x64, всё то же самое, только нужно использовать $2^{64}$ вместо $2^{32}$:

\begin{lstlisting}
uint64_t f1234(uint64_t a)
{
	return a/1234;
};
\end{lstlisting}

\begin{lstlisting}[caption=\Optimizing MSVC 2012 x64]
f1234	PROC
	mov	rax, 7653754429286296943		; 6a37991a23aead6fH
	mul	rcx
	shr	rdx, 9
	mov	rax, rdx
	ret	0
f1234	ENDP
\end{lstlisting}

\begin{lstlisting}[caption=Wolfram Mathematica]
In[1]:=N[2^(64+9)/16^^6a37991a23aead6f]
Out[1]:=1234.
\end{lstlisting}

\subsection{Вариант \#2}

Бывает также вариант с пропущенным арифметическим сдвигом, например:

\begin{lstlisting}
		mov     eax, 55555556h ; 1431655766
		imul    ecx
		mov     eax, edx
		shr     eax, 1Fh
\end{lstlisting}

Метод определения делителя упрощается:

\[
D=\frac{2^{32}}{M}
\]

Для моего примера, это:

\[
D=\frac{2^{32}}{1431655766}
\]

\myindex{Wolfram Mathematica}
Снова используем Wolfram Mathematica:

\begin{lstlisting}[caption=Wolfram Mathematica]
In[1]:=N[2^32/16^^55555556]
Out[1]:=3.
\end{lstlisting}

Искомый делитель это 3.
