\section{\RU{Как это работает}\EN{How it works}}

\RU{Вот как деление может быть заменено на умножение и деление на числа $2^{n}$}
\EN{That's how division can be replaced by multiplication and division with $2^{n}$ numbers}:

\[
	result = 
	\frac{input}{divisor} = 
	\frac{input \cdot \frac{2^{n}}{divisor}}{2^{n}} = 
	\frac{input \cdot M}{2^{n}}
\]

\RU{Где}\EN{Where} $M$ \RU{это}\EN{is a} \IT{magic}\RU{-коэффициент}\EN{ coefficient}.

\RU{Как вычислить $M$}\EN{This is how $M$ can be computed}:

\[
	M = \frac{2^{n}}{divisor}
\]

\RU{Так что эти фрагменты кода обычно имеют форму}
\EN{So these code snippets usually have this form}:

\[
	result = \frac{input \cdot M}{2^{n}}
\]

\RU{Деление на $2^{n}$ производится обычным битовым сдвигом вправо.}
\EN{Division by $2^{n}$ is usually done by simply shifting to the right.}
\RU{Если}\EN{If} $n<32$, 
\RU{то тогда сдвигается младшая часть \glslink{product}{произведения}}
\EN{then the low part of the \gls{product} is shifted} (\InENRU \EAX \OrENRU \RAX).
\RU{Если}\EN{If} $n\geq32$, 
\RU{то тогда сдвигается старшая часть \glslink{product}{произведения}}
\EN{then the high part of the \gls{product} is shifted} (\InENRU \EDX \OrENRU \RDX).

$n$ \RU{выбирается так, чтобы улучшить точность результата}\EN{is chosen in order to minimize
the error}.

\RU{Если делать знаковое деление, знак результата умножения также добавляется к результату}
\EN{When doing signed division, the sign of the multiplication result is also added to the output result}.

\RU{Посмотрите на разницу}\EN{Take a look at the difference}:

\begin{lstlisting}
int f3_32_signed(int a)
{
	return a/3;
};

unsigned int f3_32_unsigned(unsigned int a)
{
	return a/3;
};
\end{lstlisting}

\RU{В беззнаковой версии функции, }\EN{In the unsigned version of the function, the}
\IT{magic}\RU{-коэффициент это}\EN{ coefficient is} \TT{0xAAAAAAAB} \RU{и результат умножения делится на}
\EN{and the multiplication result is divided by} $2^{33}$.

\RU{В знаковой версии функции,}\EN{In the signed version of the function, the} \IT{magic}\RU{-коэффициент это}
\EN{ coefficient is} \TT{0x55555556} \RU{и результат умножения делится на}
\EN{and the multiplication result is divided by} $2^{32}$.
\RU{Впрочем здесь нет инструкции деления: результат просто берется из}\EN{There are no 
division instruction though: the result is just taken from} \EDX. 

\RU{Знак результата умножения также учитывается: старшие 32 бита результата сдвигаются на 31
(таким образом, оставляя знак в самом младшем бите \EAX).}
\EN{The sign is also taken from the multiplication result: the high 32 bits of the result are shifted by 31
(leaving the sign in the least significant bit of \EAX).}
1 \RU{прибавляется к конечному результату, если знак отрицательный, для коррекции результата}
\EN{is added to the final result if the sign is negative, for result correction}.

\lstinputlisting[caption=\Optimizing MSVC 2012]{\CURPATH/2.asm.\LANG}

\subsection{\EN{More theory}\RU{Больше теории}}

\EN{It works, because that is how it's possible to replace division by multiplication:}
\RU{Это работает, потому что можно заменить деление на умножение вот так:}

\[
	\frac{x}{c} = x\frac{1}{c}
\]

\EN{$\frac{1}{c}$ is called \IT{multiplicative inverse} and can be precomputed by compiler.}
\RU{$\frac{1}{c}$ называется \IT{обратное число} и может быть предвычислено компилятором.}

\EN{But this is for real numbers}\RU{Но это для вещественных чисел.}
\EN{What about integers?}\RU{Что насчет целых чисел?}
\EN{It's possible to find \IT{multiplicative inverse} for integer in the environment of modulo arithmetics}
\RU{Можно найти \IT{обратное число по модулю} для целого числа в среде модульной арифметики}
\footnote{\href{http://go.yurichev.com/17359}{Wikipedia}}.
\EN{\ac{CPU} registers fits nicely: each is limited by 32 or 64 bits, so almost any arithmetic operation on registers are in fact
opeartions on modulo $2^{32}$ or $2^{64}$.}
\RU{Регистры \ac{CPU} подходят идеально: каждый ограничен 32 или 64-ю битами, так что практически любая арифметическая операция над регистрами
это на самом деле операции по модулю $2^{32}$ или $2^{64}$.}

\RU{Читайте больше об этом в}\EN{Read more about it in} \cite[10-3]{Warren:2002:HD:515297}.
