\section{\RU{Как это работает}\EN{How it works}}

\RU{Вот как деление может быть заменено на умножение и деление на числа $2^{n}$}
\EN{That's how division can be replaced by multiplication and division by $2^{n}$ numbers}:

\[
	result = 
	\frac{input}{divisor} = 
	\frac{input \cdot \frac{2^{n}}{divisor}}{2^{n}} = 
	\frac{input \cdot M}{2^{n}}
\]

\RU{Где}\EN{Where} $M$ \RU{это}\EN{is} \IT{magic}-\RU{коэффициент}\EN{coefficient}.

\RU{Как вычислить $M$}\EN{That's how $M$ can be computed}:

\[
	M = \frac{2^{n}}{divisor}
\]

\RU{Так что эти фрагменты кода обычно имеют форму}
\EN{So these code snippets are usually have this form}:

\[
	result = \frac{input \cdot M}{2^{n}}
\]

\RU{Деление на $2^{n}$ производится обычным битовым сдвигом вправо.}
\EN{Division by $2^{n}$ is usually done by simple right bit shift.}
\RU{Если}\EN{If} $n<32$, 
\RU{то тогда сдвигается младшая часть \glslink{product}{произведения}}
\EN{then low part of \gls{product} is shifted} (\InENRU \EAX \OrENRU \RAX).
\RU{Если}\EN{If} $n\geq32$, 
\RU{то тогда сдвигается старшая часть \glslink{product}{произведения}}
\EN{then the high part of \gls{product} is shifted} (\InENRU \EDX \OrENRU \RDX).

$n$ \RU{выбирается так, чтобы улучшить точность результата}\EN{is chosen in order to minimize
error}.

\RU{Если делать знаковое деление, знак результата умножения также добавляется к результату}
\EN{When doing signed division, sign of multiplication result also added to the output result}.

\RU{Посмотрите на разницу}\EN{Take a look at the difference}:

\begin{lstlisting}
int f3_32_signed(int a)
{
	return a/3;
};

unsigned int f3_32_unsigned(unsigned int a)
{
	return a/3;
};
\end{lstlisting}

\RU{В беззнаковой версии функции}\EN{In the unsigned version of function}, 
\IT{magic}-\RU{коэффициент это}\EN{coefficient is} \TT{0xAAAAAAAB} \RU{и результат умножения делится на}
\EN{and multiplication result is divided by} $2^{33}$.

\RU{В знаковой версии функции}\EN{In the signed version of function}, \IT{magic}-\RU{коэффициент это}
\EN{coefficient is} \TT{0x55555556} \RU{и результат умножения делится на}
\EN{and multiplication result is divided by} $2^{32}$.
\RU{Впрочем здесь нет инструкции деления: результат просто берется из}\EN{There are no 
division instruction though: result is just taken from} \EDX. 

\RU{Знак результата умножения также учитывается: старшие 32 бита результата сдвигаются на 31
(таким образом, оставляя знак в самом младшем бите \EAX).}
\EN{Sign is also taken from multiplication result: high 32 bits of result is shifted by 31
(leaving sign in least significant bit of \EAX).}
$1$ \RU{прибавляется к конечному результату, если знак отрицательный, для коррекции результата}
\EN{is added to the final result if sign is negative, for result correction}.

\lstinputlisting[caption=\Optimizing MSVC 2012]{\CURPATH/2.asm.\LANG}

\RU{Читайте больше об этом в}\EN{Read more about it in} \cite[10-3]{Warren:2002:HD:515297}.
