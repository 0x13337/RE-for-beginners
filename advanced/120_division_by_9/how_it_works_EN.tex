\subsection{How it works}

That's how division can be replaced by multiplication and division with $2^{n}$ numbers:

\[
	result = 
	\frac{input}{divisor} = 
	\frac{input \cdot \frac{2^{n}}{divisor}}{2^{n}} = 
	\frac{input \cdot M}{2^{n}}
\]

Where $M$ is a \IT{magic} coefficient.

This is how $M$ can be computed:

\[
	M = \frac{2^{n}}{divisor}
\]

So these code snippets usually have this form:

\[
	result = \frac{input \cdot M}{2^{n}}
\]

%
Division by $2^{n}$ is usually done by simply shifting to the right.

If $n<32$, then the low part of the \gls{product} is shifted (in \EAX or \RAX).

$n$ is chosen in order to minimize the error.

When doing signed division, the sign of the multiplication result is also added to the output result.

Take a look at the difference:

\begin{lstlisting}
int f3_32_signed(int a)
{
	return a/3;
};

unsigned int f3_32_unsigned(unsigned int a)
{
	return a/3;
};
\end{lstlisting}

In the unsigned version of the function, the \IT{magic} coefficient is \TT{0xAAAAAAAB} 
and the multiplication result is divided by $2^{33}$.

In the signed version of the function, the \IT{magic} coefficient is \TT{0x55555556} 
and the multiplication result is divided by $2^{32}$.
There are no division instruction though: the result is just taken from \EDX. 

The sign is also taken from the multiplication result: the high 32 bits of the result are shifted by 31
(leaving the sign in the least significant bit of \EAX).
1 is added to the final result if the sign is negative, for result correction.

\lstinputlisting[caption=\Optimizing MSVC 2012]{\CURPATH/2_EN.asm}

\subsubsection{More theory}

It works, because that is how it's possible to replace division by multiplication:

\[
	\frac{x}{c} = x\frac{1}{c}
\]

$\frac{1}{c}$ is called \IT{multiplicative inverse} and can be precomputed by compiler during compilation
time.

But this is for real numbers
What about integers?
It's possible to find \IT{multiplicative inverse} for integer in the environment of modular arithmetic
\footnote{\href{http://go.yurichev.com/17359}{Wikipedia}}.
\ac{CPU} registers fits nicely: each is limited by 32 or 64 bits, so almost any arithmetic operation on registers are in fact
operations on modulo $2^{32}$ or $2^{64}$.

Read more about it in \InSqBrackets{\HenryWarren 10-3}.

