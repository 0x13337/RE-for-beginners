\chapter{\RU{Конверсия строки в число}\EN{String to number conversion} (atoi())}

\index{\CStandardLibrary!atoi()}
\RU{Попробуем реализовать стандарту ф-цию Си atoi().}
\EN{Let's try to reimplement the standard atoi() C function.}

\section{\RU{Простой пример}\EN{Simple example}}

\RU{Это самый простой способ прочитать число, представленное в кодировке \ac{ASCII}.}
\EN{Here is the simplest possible way to read a number represented in \ac{ASCII} encoding.}
\RU{Он не защищен от ошибок: символ отличный от цифры приведет к неверному результату.}
\EN{It's not error-prone: a character other than a digit will lead to incorrect result.}

\lstinputlisting{\CURPATH/atoi.c}

\RU{То, что делает алгоритм это просто считывает цифры слева направо.}
\EN{So what the algorithm does is just reading digits from left to right.}
\RU{Символ нуля в \ac{ASCII} вычитается из каждой цифры.}
\EN{The zero \ac{ASCII} character is subtracted from each digit. }
\RU{Цифры от ``0'' до ``9'' расположены по порядку в таблице \ac{ASCII}, так что мы даже можем
и не знать точного значения символа ``0''.}
\EN{The digits from ``0'' to ``9'' are consecutive in the \ac{ASCII} table, so 
we do not even need to know the exact value of the ``0'' character.}
\RU{Всё что нам нужно знать это то что ``0'' минус ``0'' --- это 0, а ``9'' минус ``0'' это 9, и т.д.}
\EN{All we need to know is that ``0'' minus ``0'' is 0, ``9'' minus ``0'''is 9 and so on.}
\RU{Вычитание ``0'' от каждого символа в итоге дает число от 0 до 9 включительно.}
\EN{Subtracting ``0'' from each character results in a number from 0 to 9 inclusive.}
\RU{Любой другой символ, конечно, приведет к неверному результату!}
\EN{Any other character will lead to an incorrect result, of course!}
\RU{Каждая цифра добавляется к итоговому результату (в переменной ``rt''), но итоговый результат
также умножается на 10 на каждой цифре.}
\EN{Each digit has to be added to the final result (in variable ``rt''), but the final result
is also multiplied by 10 at each digit.}
\RU{Другими словами, на каждой итерации, результат сдвигается влево на одну позицию в десятичном виде.}
\EN{In other words, the result is shifted left by one position in decimal form on each iteration.}
\RU{Самая последняя цифра прибавляется, но не сдвигается.}
\EN{The last digit is added, but there is no no shift.}

\subsection{\Optimizing MSVC 2013 x64}

\lstinputlisting[caption=\Optimizing MSVC 2013 x64]{\CURPATH/atoi.asm.MSVC2013.x64.Ox.\LANG}

\RU{Символы загружаются в двух местах: первый символ и все последующие символы.}
\EN{A character can be loaded in two places: the first character and all subsequent characters.}
\RU{Это сделано для перегруппировки цикла.}\EN{This is done for loop regrouping.}
\index{x86!\Instructions!LEA}
\RU{Здесь нет инструкции для умножения на 10, вместо этого две LEA делают это же.}
\EN{There is no instruction for multiplication by 10, two LEA instruction do this instead.}
\index{x86!\Instructions!ADD}
\index{x86!\Instructions!SUB}
\RU{MSVC иногда использует инструкцию ADD с отрицательной константой вместо SUB.}
\EN{MSVC sometimes uses the ADD instruction with a negative constant instead of SUB.}
\RU{Это тот случай}\EN{This is the case}.
\RU{Честно говоря, я не знаю, чем это лучше, чем SUB.}
\EN{I truly don't know why this is better then SUB.}
\RU{Но MSVC делает так часто}\EN{But MSVC does this often}.

\subsection{\Optimizing GCC 4.9.1 x64}

\Optimizing GCC 4.9.1 \RU{более краток, но здесь есть одна лишняя инструкция RET в конце.}
\EN{is more concise, but there is one redundant RET instruction at the end.}
\RU{Одной было бы достаточно}\EN{One would be enough}.

\lstinputlisting[caption=\Optimizing GCC 4.9.1 x64]{\CURPATH/atoi.s.GCC491.O3.x64.\LANG}

\ifdefined\IncludeARM
\subsection{\OptimizingKeilVI (\ARMMode)}

\lstinputlisting[caption=\OptimizingKeilVI (\ARMMode)]{\CURPATH/atoi.s.ARM.O3.\LANG}

\subsection{\OptimizingKeilVI (\ThumbMode)}

\lstinputlisting[caption=\OptimizingKeilVI (\ThumbMode)]{\CURPATH/atoi.s.thumb.O3.\LANG}

\RU{Интересно, из школьного курса математики мы можем помнить что порядок операций сложения и вычитания
не играет роли.}
\EN{Interestingly, from school mathematics we may remember that the order of addition and 
subtraction operations doesn't matter.}
\RU{Это наш случай: в начале вычисляется выражение $rt*10 - '0'$, затем к нему прибавляется 
значение входного символа.}
\EN{That's our case: first, the $rt*10 - '0'$ expression is computed, then the input character value 
is added to it.}
\RU{Действительно, результат тот же, но компилятор немного всё перегруппировал.}
\EN{Indeed, the result is the same, but the compiler did some regrouping.}

\subsection{\Optimizing GCC 4.9.1 ARM64}

\RU{Компилятор для ARM64 может использовать суффикс инструкции, задающий пре-инкремент:}
\EN{The ARM64 compiler can use the pre-increment instruction suffix:}

\lstinputlisting[caption=\Optimizing GCC 4.9.1 ARM64]{\CURPATH/atoi.s.GCC49.ARM64.O3.\LANG}
\fi

\section{\RU{Немного расширенный пример}\EN{A slightly advanced example}}

\RU{Новый пример более расширенный, теперь здесь есть проверка знака ``минус'' в самом начале,
и еще он может сообщать об ошибке если не-цифра была найдена во входной строке:}
\EN{My new code snippet is more advanced, now it checks for the ``minus'' sign at the first character
and reports an error if a non-digit was found in the input string:}

\lstinputlisting{\CURPATH/atoi2.c}

\subsection{\Optimizing GCC 4.9.1 x64}

\lstinputlisting[caption=\Optimizing GCC 4.9.1 x64]{\CURPATH/atoi2.s.GCC491.O3.x64.\LANG}

\index{x86!\Instructions!NEG}
\RU{Если знак ``минус'' был найден в начале строки, инструкция NEG будет исполнена в конце.}
\EN{If the ``minus'' sign was encountered at the string start, the NEG instruction will be executed at the end.}
\RU{Она просто меняет знак числа}\EN{It just negates the number}.

\RU{Еще кое-что надо отметить}\EN{There is one more thing that needs mentioning}.
\RU{Как среднестатистический программист будет проверять, является ли символ цифрой?}
\EN{How would a common programmer check if the character is not a digit?}
\RU{Так же, как я и написал в исходном коде}\EN{Just how I wrote it in the source code}:

\begin{lstlisting}
if (*s<'0' || *s>'9')
    ...
\end{lstlisting}

\RU{Здесь две операции сравнения}\EN{There are two comparison operations}.
\RU{Но что интересно, так это то что мы можем заменить обе операции на одну:}
\EN{What is interesting is that we can replace both operations by single one:}
\RU{просто вычитайте ``0'' из значения символа}\EN{just subtract ``0'' from character value},
\RU{считается результат за беззнаковое значение (это важно) и проверьте, не больше ли он чем 9.}
\EN{treat result as unsigned value (this is important) and check if it's greater than 9.}

\RU{Например, скажем, строка на входе имеет символ точки (``.''), которая имеет код 46 в таблице \ac{ASCII}.}
\EN{For example, let's say that the user input contains the dot character (``.'') which has \ac{ASCII} code 46.}
$46-48=-2$ \RU{если считать результат за знаковое число}\EN{if we treat the result as a signed number}.
\RU{Действительно, символ точки расположен на два места раньше, чем символ ``0'' в таблице \ac{ASCII}.}
\EN{Indeed, the dot character is located two places earlier than the ``0'' character in the \ac{ASCII} table.}
\RU{Но это}\EN{But it is} \TT{0xFFFFFFFE} ($4294967294$) \RU{если считать результат за беззнаковое значение, 
и это точно больше чем 9!}
\EN{if we treat the result as an unsigned value, and that's definitely bigger than 9!}

\RU{Компиляторы часто так делают, важно распознавать эти трюки.}
\EN{The compilers do this often, so it's important to recognize these tricks.}

\Optimizing MSVC 2013 x64 \RU{применяет те же трюки}\EN{does the same tricks}.

\ifdefined\IncludeARM
\subsection{\OptimizingKeilVI (\ARMMode)}

\lstinputlisting[caption=\OptimizingKeilVI (\ARMMode),numbers=left]{\CURPATH/atoi2.s.ARM.O3.\LANG}

\RU{В 32-битном ARM нет инструкции NEG, так что вместо этого используется операция ``Reverse Subtraction''
(строка 31).}
\EN{There is no NEG instruction in 32-bit ARM, so the ``Reverse Subtraction'' operation (line 31) 
is used here.}
\RU{Она сработает если результат инструкции CMP (на строке 29) был ``Not Equal'' 
(не равно, отсюда суффикс -NE suffix).}
\EN{It is triggered if the result of the CMP instruction (at line 29) was ``Not Equal'' (hence -NE suffix).}
\index{ARM!\Instructions!RSB}
\RU{Что делает RSBNE это просто вычитает результирующее значение из нуля.}
\EN{So what RSBNE does is to subtract the resulting value from 0.}
\RU{Она работает, как и обычное вычитание, но меняет местами операнды.}
\EN{It works just like the regular subtraction operation, but swaps operands.}
\RU{Вычитание любого числа из нуля это смена знака}
\EN{Subtracting any number from 0 results in negation}: $0-x=-x$.

\RU{Код для режима Thumb почти такой же.}
\EN{Thumb mode code is mostly the same.}

\index{ARM!\Instructions!NEG}
GCC 4.9 \ForENRU ARM64 \RU{может использовать инструкцию NEG, доступную в}
\EN{can use the NEG instruction, which is available in} ARM64.
\fi
