\section{\RU{Функции работы со строками и памятью}\EN{Strings and memory functions}}

\RU{Другая очень частая оптимизация это вставка кода строковых функций таких как}
\EN{Another very common automatic optimization tactic is the inlining of string functions like}
\IT{strcpy()}, \IT{strcmp()}, \IT{strlen()}, \IT{memset()}, \IT{memcmp()}, \IT{memcpy()}, \etc{}.

\RU{Иногда это быстрее, чем вызывать отдельную функцию.}\EN{Sometimes it's faster than to call a separate function.}

\RU{Это очень часто встречающиеся шаблонные вставки, которые желательно распознавать 
reverse engineer-ам \q{на глаз}.}
\EN{These are very frequent patterns and it is highly advisable for reverse engineers 
to learn to detect automatically.}

% subsections
\subsection{strcmp()}
\index{\CStandardLibrary!strcmp()}

\lstinputlisting[caption=\RU{пример с strcmp()}\EN{strcmp() example}]{\CURPATH/str_mem/strcmp.c}

\lstinputlisting[caption=\Optimizing GCC 4.8.1]{\CURPATH/str_mem/strcmp_GCC_O3.s}

\lstinputlisting[caption=\Optimizing MSVC 2010]{\CURPATH/str_mem/strcmp_MSVC_2010_Ox.asm}

\subsection{strlen()}
\myindex{\CStandardLibrary!strlen()}

\lstinputlisting[caption=\RU{пример с strlen()}\EN{strlen() example}]{\CURPATH/str_mem/strlen.c}

\lstinputlisting[caption=\Optimizing MSVC 2010]{\CURPATH/str_mem/strlen_MSVC_2010_Ox.asm}

\subsection{strcpy()}
\index{\CStandardLibrary!strcpy()}

\lstinputlisting[caption=\RU{пример с strcpy()}\EN{strcpy() example}]{\CURPATH/str_mem/strcpy.c}

\lstinputlisting[caption=\Optimizing MSVC 2010]{\CURPATH/str_mem/strcpy_MSVC_2010_Ox.asm.\LANG}

\subsection{memset()}
\index{\CStandardLibrary!memset()}

\subsubsection{\Example \#1}

\lstinputlisting[caption=32 \RU{байта}\EN{bytes}]{\CURPATH/str_mem/memset_32.c}

\index{x86!\Instructions!MOV}
\EN{Many compilers don't generate a call to memset() for short blocks, but rather insert a \MOV:}
\RU{Многие компиляторы не генерируют вызов memset() для коротких блоков, а просто вставляют \MOV:}

\lstinputlisting[caption=\Optimizing GCC 4.9.1 x64]{\CURPATH/str_mem/memset_32_GCC491_x64_O3.s}

\index{Unrolled loop}
\RU{Кстати, это напоминает развернутые циклы}\EN{By the way, that remind us of unrolled loops}: 
\myref{ARM_unrolled_loops}.

\subsubsection{\Example \#2}

\lstinputlisting[caption=67 \RU{байт}\EN{bytes}]{\CURPATH/str_mem/memset_67.c}

\EN{When the block size is not a multiple of 4 or 8, the compilers can behave differently.}
\RU{Когда размер блока не кратен 4 или 8, разные компиляторы могут вести себя по-разному.}

\index{x86!\Instructions!MOV}
\EN{For instance, MSVC 2012 continues to insert \TT{MOV}s:}
\RU{Например, MSVC 2012 продолжает вставлять \MOV:}

\lstinputlisting[caption=\Optimizing MSVC 2012 x64]{\CURPATH/str_mem/memset_67_MSVC2012_x64_Ox.asm}

\index{x86!\Instructions!STOSQ}
\EN{\dots while GCC uses \TT{REP STOSQ}, concluding that this would be shorter than a pack of \TT{MOV}s:}
\RU{\dots а GCC использует \TT{REP STOSQ}, полагая, что так будет короче, чем пачка \TT{MOV}'s:}

\lstinputlisting[caption=\Optimizing GCC 4.9.1 x64]{\CURPATH/str_mem/memset_67_GCC491_x64_O3.s}

\subsectionold{memcpy()}
\myindex{\CStandardLibrary!memcpy()}

\subsubsectionold{\RU{Короткие блоки}\EN{Short blocks}}
\label{copying_short_blocks}

\myindex{x86!\Instructions!MOV}
\RU{Если нужно скопировать немного байт, то, нередко, 
\TT{memcpy()} заменяется на несколько инструкций \MOV.}
\EN{The routine to copy short blocks is often implemented as a sequence of \MOV instructions.}

\lstinputlisting[caption=\RU{пример с memcpy()}\EN{memcpy() example}]{\CURPATH/str_mem/memcpy_7.c}

\lstinputlisting[caption=\Optimizing MSVC 2010]{\CURPATH/str_mem/memcpy_7_MSVC_2010_Ox.asm}

\lstinputlisting[caption=\Optimizing GCC 4.8.1]{\CURPATH/str_mem/memcpy_7_GCC_O3.s}

\RU{Обынчо это происходит так: в начале копируются 4-байтные блоки, затем 16-битное слово (если нужно), 
затем последний байт (если нужно).}
\EN{That's usually done as follows: 4-byte blocks are copied first, then a 16-bit word (if needed), 
then the last byte (if needed).}

\RU{Точно так же при помощи \MOV копируются структуры}\EN{Structures are also copied using
\MOV}: \myref{short_struct_copying_using_MOV}.

\subsubsectionold{\RU{Длинные блоки}\EN{Long blocks}}

\RU{Здесь компиляторы ведут себя по-разному.}\EN{The compilers behave differently in this case.}

\lstinputlisting[caption=\RU{пример с memcpy()}\EN{memcpy() example}]{\CURPATH/str_mem/memcpy.c}

\myindex{x86!\Instructions!MOVSD}
\RU{При копировании 128 байт, MSVC может обойтись одной инструкцией \TT{MOVSD} (ведь 128 кратно 4):}
\EN{For copying 128 bytes, MSVC uses a single \TT{MOVSD} instruction (because 128 
divides evenly by 4):}

\lstinputlisting[caption=\Optimizing MSVC 2010]{\CURPATH/str_mem/memcpy_128_MSVC_2010_Ox.asm}

\RU{При копировании 123-х байт, в начале копируется 30 32-битных слов при помощи \TT{MOVSD} 
(это 120 байт), 
затем копируется 2 байта при помощи \TT{MOVSW}, 
затем еще один байт при помощи \TT{MOVSB}.}
\EN{When copying 123 bytes, 30 32-byte words are copied first using \TT{MOVSD}
(that's 120 bytes),
then 2 bytes are copied using \TT{MOVSW}, 
then one more byte using \TT{MOVSB}.}

\lstinputlisting[caption=\Optimizing MSVC 2010]{\CURPATH/str_mem/memcpy_123_MSVC_2010_Ox.asm}

\RU{GCC во всех случаях вставляет большую универсальную функцию, работающую для всех размеров блоков:}
\EN{GCC uses one big universal functions, that works for any block size:}

\lstinputlisting[caption=\Optimizing GCC 4.8.1]{\CURPATH/str_mem/memcpy_GCC.s}

\RU{Универсальные функции копирования блоков обычно работают по следующей схеме: 
вычислить, сколько 32-битных слов
можно скопировать, затем сделать это при помощи \TT{MOVSD}, затем скопировать остатки.}
\EN{Universal memory copy functions usually work as follows:
calculate how many 32-bit words can be copied, then copy them using \TT{MOVSD}, then copy
the remaining bytes.}

\myindex{SIMD}
\RU{Более сложные функции копирования используют \ac{SIMD} и учитывают выравнивание в памяти.}
\EN{More complex copy functions use \ac{SIMD} instructions and also take the memory alignment
in consideration.}
\RU{Как пример функции strlen() использующую SIMD}
\EN{As an example of SIMD strlen() function}: \myref{SIMD_strlen}.


\subsectionold{memcmp()}
\myindex{\CStandardLibrary!memcmp()}

\lstinputlisting[caption=\RU{пример с memcmp()}\EN{memcmp() example}]{\CURPATH/str_mem/memcmp.c}

\RU{Для блоков разной длины, MSVC 2010 вставляет одну и ту же универсальную функцию:}
\EN{For any block size, MSVC 2010 inserts the same universal function:}

\lstinputlisting[caption=\Optimizing MSVC 2010]{\CURPATH/str_mem/memcmp_MSVC_2010_Ox.asm}

\subsection{strcat()}
\index{\CStandardLibrary!strcat()}

\ifdefined\RUSSIAN
Это ф-ция strcat() в том виде, в котором её сгенерировала MSVC 6.0.
Здесь видны 3 части:
1) измерение длины исходной строки (первый \INS{scasb});
2) измерение длины целевой строки (второй \INS{scasb});
3) копирование исходной строки в конец целевой (пара \INS{movsd}/\INS{movsb}).
\fi % RUSSIAN

\ifdefined\ENGLISH
This is inlined strcat() as it has been generated by MSVC 6.0.
There are 3 parts visible:
1) getting source string length (first \INS{scasb});
2) getting destination string length (second \INS{scasb});
3) copying source string into the end of destination string (\INS{movsd}/\INS{movsb} pair).
\fi % ENGLISH

\lstinputlisting[caption=strcat()]{\CURPATH/str_mem/strcat.lst}



\subsection{\RU{Скрипт для IDA}\EN{IDA script}}

\index{IDA}
\RU{Есть также небольшой скрипт для \IDA для поиска и сворачивания таких очень часто попадающихся inline-функций:}
\EN{There is also a small \IDA script for searching and folding such very frequently seen pieces of inline code:}
\par \href{\YurichevIDAIDCScripts}{GitHub}.
