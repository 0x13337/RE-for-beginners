\subsectionold{memcpy()}
\myindex{\CStandardLibrary!memcpy()}

\subsubsectionold{\RU{Короткие блоки}\EN{Short blocks}}
\label{copying_short_blocks}

\myindex{x86!\Instructions!MOV}
\RU{Если нужно скопировать немного байт, то, нередко, 
\TT{memcpy()} заменяется на несколько инструкций \MOV.}
\EN{The routine to copy short blocks is often implemented as a sequence of \MOV instructions.}

\lstinputlisting[caption=\RU{пример с memcpy()}\EN{memcpy() example}]{\CURPATH/str_mem/memcpy_7.c}

\lstinputlisting[caption=\Optimizing MSVC 2010]{\CURPATH/str_mem/memcpy_7_MSVC_2010_Ox.asm}

\lstinputlisting[caption=\Optimizing GCC 4.8.1]{\CURPATH/str_mem/memcpy_7_GCC_O3.s}

\RU{Обынчо это происходит так: в начале копируются 4-байтные блоки, затем 16-битное слово (если нужно), 
затем последний байт (если нужно).}
\EN{That's usually done as follows: 4-byte blocks are copied first, then a 16-bit word (if needed), 
then the last byte (if needed).}

\RU{Точно так же при помощи \MOV копируются структуры}\EN{Structures are also copied using
\MOV}: \myref{short_struct_copying_using_MOV}.

\subsubsectionold{\RU{Длинные блоки}\EN{Long blocks}}

\RU{Здесь компиляторы ведут себя по-разному.}\EN{The compilers behave differently in this case.}

\lstinputlisting[caption=\RU{пример с memcpy()}\EN{memcpy() example}]{\CURPATH/str_mem/memcpy.c}

\myindex{x86!\Instructions!MOVSD}
\RU{При копировании 128 байт, MSVC может обойтись одной инструкцией \TT{MOVSD} (ведь 128 кратно 4):}
\EN{For copying 128 bytes, MSVC uses a single \TT{MOVSD} instruction (because 128 
divides evenly by 4):}

\lstinputlisting[caption=\Optimizing MSVC 2010]{\CURPATH/str_mem/memcpy_128_MSVC_2010_Ox.asm}

\RU{При копировании 123-х байт, в начале копируется 30 32-битных слов при помощи \TT{MOVSD} 
(это 120 байт), 
затем копируется 2 байта при помощи \TT{MOVSW}, 
затем еще один байт при помощи \TT{MOVSB}.}
\EN{When copying 123 bytes, 30 32-byte words are copied first using \TT{MOVSD}
(that's 120 bytes),
then 2 bytes are copied using \TT{MOVSW}, 
then one more byte using \TT{MOVSB}.}

\lstinputlisting[caption=\Optimizing MSVC 2010]{\CURPATH/str_mem/memcpy_123_MSVC_2010_Ox.asm}

\RU{GCC во всех случаях вставляет большую универсальную функцию, работающую для всех размеров блоков:}
\EN{GCC uses one big universal functions, that works for any block size:}

\lstinputlisting[caption=\Optimizing GCC 4.8.1]{\CURPATH/str_mem/memcpy_GCC.s}

\RU{Универсальные функции копирования блоков обычно работают по следующей схеме: 
вычислить, сколько 32-битных слов
можно скопировать, затем сделать это при помощи \TT{MOVSD}, затем скопировать остатки.}
\EN{Universal memory copy functions usually work as follows:
calculate how many 32-bit words can be copied, then copy them using \TT{MOVSD}, then copy
the remaining bytes.}

\myindex{SIMD}
\RU{Более сложные функции копирования используют \ac{SIMD} и учитывают выравнивание в памяти.}
\EN{More complex copy functions use \ac{SIMD} instructions and also take the memory alignment
in consideration.}
\RU{Как пример функции strlen() использующую SIMD}
\EN{As an example of SIMD strlen() function}: \myref{SIMD_strlen}.

