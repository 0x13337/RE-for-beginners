\subsection{memset()}
\index{\CStandardLibrary!memset()}

\subsubsection{\Example \#1}

\lstinputlisting[caption=32 \RU{байта}\EN{bytes}]{\CURPATH/str_mem/memset_32.c}

\index{x86!\Instructions!MOV}
\EN{Many compilers don't generate a call to memset() for short blocks, but rather insert a pack of \MOV{}s:}
\RU{Многие компиляторы не генерируют вызов memset() для коротких блоков, а просто вставляют набор \MOV{}-ов:}

\lstinputlisting[caption=\Optimizing GCC 4.9.1 x64]{\CURPATH/str_mem/memset_32_GCC491_x64_O3.s}

\index{Unrolled loop}
\RU{Кстати, это напоминает развернутые циклы}\EN{By the way, that remind us of unrolled loops}: 
\myref{ARM_unrolled_loops}.

\subsubsection{\Example \#2}

\lstinputlisting[caption=67 \RU{байт}\EN{bytes}]{\CURPATH/str_mem/memset_67.c}

\EN{When the block size is not a multiple of 4 or 8, the compilers can behave differently.}
\RU{Когда размер блока не кратен 4 или 8, разные компиляторы могут вести себя по-разному.}

\index{x86!\Instructions!MOV}
\EN{For instance, MSVC 2012 continues to insert \TT{MOV}s:}
\RU{Например, MSVC 2012 продолжает вставлять \MOV:}

\lstinputlisting[caption=\Optimizing MSVC 2012 x64]{\CURPATH/str_mem/memset_67_MSVC2012_x64_Ox.asm}

\index{x86!\Instructions!STOSQ}
\EN{\dots while GCC uses \TT{REP STOSQ}, concluding that this would be shorter than a pack of \TT{MOV}s:}
\RU{\dots а GCC использует \TT{REP STOSQ}, полагая, что так будет короче, чем пачка \TT{MOV}'s:}

\lstinputlisting[caption=\Optimizing GCC 4.9.1 x64]{\CURPATH/str_mem/memset_67_GCC491_x64_O3.s}
