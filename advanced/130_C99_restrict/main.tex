\chapter{C99 restrict}
\myindex{\CLanguageElements!C99!restrict}
\myindex{FORTRAN}

\RU{А вот причина, из-за которой программы на FORTRAN, в некоторых случаях, работают быстрее чем на Си.}
\EN{Here is a reason why FORTRAN programs, in some cases, work faster than \CCpp ones.}

\begin{lstlisting}
void f1 (int* x, int* y, int* sum, int* product, int* sum_product, int* update_me, size_t s)
{
	for (int i=0; i<s; i++)
	{
		sum[i]=x[i]+y[i];
		product[i]=x[i]*y[i];
		update_me[i]=i*123; // some dummy value
		sum_product[i]=sum[i]+product[i];	
	};
};
\end{lstlisting}

\RU{Это очень простой пример, в котором есть одна особенность}
\EN{That's very simple example with one specific
thing in it}: 
\RU{указатель на массив}\EN{the pointer to the} \TT{update\_me} \RU{может быть указателем на массив}\EN{array could be
a pointer to the}
\TT{sum}\EN{ array}, \TT{product}\EN{ array}, \RU{или даже}\EN{or even the} 
\TT{sum\_product}\EN{ array}\EMDASH\RU{ведь нет ничего криминального в том 
чтобы аргументам функции быть такими, верно?}\EN{nothing forbids that, right?}

\RU{Компилятор знает об этом, поэтому генерирует код, где в теле цикла будет 4 основных стадии:}
\EN{The compiler is fully aware of this, so it generates code with four stages in the loop body:}
\begin{itemize}
\item \RU{вычислить следующий}\EN{calculate next} \TT{sum[i]}
\item \RU{вычислить следующий}\EN{calculate next} \TT{product[i]}
\item \RU{вычислить следующий}\EN{calculate next} \TT{update\_me[i]}
\item \RU{вычислить следующий}\EN{calculate next} \TT{sum\_product[i]}\EMDASH\RU{на этой стадии придется снова загружать из памяти подсчитанные}
\EN{on this stage, we need to load from memory the already calculated} \TT{sum[i]} \AndENRU \TT{product[i]}
\end{itemize}

\RU{Возможно ли соптимизировать последнюю стадию?}\EN{Is it possible to optimize the last stage?}
\RU{Ведь подсчитанные}\EN{Since we have already calculated} \TT{sum[i]} \AndENRU \TT{product[i]} 
\RU{не обязательно снова загружать из памяти, ведь мы их только что подсчитали.}
\EN{it is not necessary to load them again from memory.}
\RU{Можно, но компилятор не уверен, что на третьей стадии ничего не затерлось!}
\EN{Yes, but compiler is not sure that nothing was overwritten in the 3rd stage!}
\RU{Это называется}\EN{This is called}
\q{pointer aliasing}, \RU{ситуация, когда компилятор не может быть уверен, что память на которую указывает 
какой-то указатель, не изменилась.}
\EN{a situation when the compiler cannot be sure that a memory to which a pointer is pointing was not changed.}

\IT{restrict} \RU{в стандарте Си C99}\EN{in the C99 standard}\cite[6.7.3/1]{C99TC3} 
\RU{это обещание, данное компилятору программистом, что аргументы функции, отмеченные этим ключевым словом,
всегда будут указывать на разные места в памяти и пересекаться не будут.}
\EN{is a promise, given by programmer to the compiler that the function arguments 
marked by this keyword always points to different memory locations and never intersects.}

\RU{Если быть более точным, и описывать это формально, \IT{restrict} показывает, что только данный указатель будет
использоваться для доступа к этому объекту, больше никакой указатель для
этого использоваться не будет.}
\EN{To be more precise and describe this formally, \IT{restrict} shows that only this pointer is to be used
to access an object, and no other pointer will be used for it.}
\RU{Можно даже сказать, что к всякому объекту, доступ будет осуществляться только через
один единственный указатель, если он отмечен как}
\EN{It can be even said the object will be accessed
only via one single pointer, if it is marked as} \IT{restrict}.

\RU{Добавим это ключевое слово к каждому аргументу-указателю}\EN{Let's add this keyword to each pointer argument}:

\begin{lstlisting}
void f2 (int* restrict x, int* restrict y, int* restrict sum, int* restrict product, int* restrict sum_product, 
	int* restrict update_me, size_t s)
{
	for (int i=0; i<s; i++)
	{
		sum[i]=x[i]+y[i];
		product[i]=x[i]*y[i];
		update_me[i]=i*123; // some dummy value
		sum_product[i]=sum[i]+product[i];	
	};
};
\end{lstlisting}

\RU{Посмотрим результаты}\EN{Let's see results}:

\lstinputlisting[caption=GCC x64: f1()]{\CURPATH/f1.asm.\LANG}

\lstinputlisting[caption=GCC x64: f2()]{\CURPATH/f2.asm.\LANG}

\RU{Разница между скомпилированной функцией \TT{f1()} и \TT{f2()} такая}
\EN{The difference between the compiled \TT{f1()} and \TT{f2()} functions is as follows}:
\InENRU \TT{f1()}, \TT{sum[i]} \AndENRU \TT{product[i]} \RU{загружаются снова посреди тела цикла}
\EN{are reloaded in the middle of the loop},
\RU{а в}\EN{and in} \TT{f2()} \RU{этого нет, используются уже подсчитанные значения}
\EN{there is no such thing,
the already calculated values are used}, 
\RU{ведь мы \q{пообещали} компилятору,}\EN{since we \q{promised} the compiler} 
\RU{что никто и ничто не изменит значения в}
\EN{that no one and nothing will change the values in} \TT{sum[i]} 
\AndENRU \TT{product[i]} \RU{во время исполнения тела цикла}\EN{during the execution of the loop's body}, 
\RU{поэтому он \q{уверен}, что значения из памяти можно не загружать снова}
\EN{so it is \q{sure} that there is no need to load the value from memory again}.
\RU{Очевидно, второй вариант работает быстрее.}
\EN{Obviously, the second example works faster.}

\RU{Но что будет если указатели в аргументах функций все же будут пересекаться?}
\EN{But what if the pointers in the function's arguments intersect somehow?}
\RU{Это на совести программиста, а результаты вычислений будут неверными.}
\EN{This is on the programmer's conscience, and the results will be incorrect.}

\RU{Вернемся к}\EN{Let's go back to} FORTRAN. 
\RU{Компиляторы с этого ЯП, по умолчанию, все указатели считают таковыми,}
\EN{Compilers of this programming language treats all pointers as such,} 
\RU{поэтому, когда не было возможности указать}
\EN{so when it was not possible to set} \IT{restrict} \InENRU \RU{Си}\EN{C}, 
\RU{то} FORTRAN \RU{в этих случаях мог генерировать более быстрый код}\EN{can generate faster code in these cases}.

\RU{Насколько это практично}\EN{How practical is it}? 
\RU{Там, где функция работает с несколькими большими блоками в памяти.}
\EN{In the cases when the function works with several big blocks in memory.}
\RU{Такого очень много в линейной алгебре, например.}
\EN{There are a lot of such in linear algebra, for instance.}
\RU{Очень много линейной алгебры используется на суперкомпьютерах/\ac{HPC},
возможно, поэтому, традиционно, там часто используется FORTRAN, до сих пор}
\EN{A lot of linear algebra is done on supercomputers/\ac{HPC}, so probably that is why, traditionally, FORTRAN is still
used there} \cite{Loh:2010:IHP:1810226.1820518}.

\RU{Ну а когда итераций цикла не очень много, конечно, 
тогда прирост скорости может и не быть ощутимым.}
\EN{But when the number of iterations is not very big,
certainly, the speed boost may not to be significant.}
