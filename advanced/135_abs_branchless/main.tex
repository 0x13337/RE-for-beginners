\chapter{\RU{Ф-ция \IT{abs()} без переходов}\EN{Branchless \IT{abs()} function}}
\label{chap:branchless_abs}

\RU{Снова вернемся к уже рассмотренному раннее примеру \ref{sec:abs} и спросим себя, возможно ли
сделать версию этого кода под x86 без переходов?}
\EN{Let's revisit an example I gave earlier \ref{sec:abs} and ask ourselves, is it possible
to make a branchless version of the function in x86 code?}

\lstinputlisting{abs.c}

\RU{И ответ положительный}\EN{And the answer is yes}.

\section{\Optimizing GCC 4.9.1 x64}

\RU{Мы можем это увидеть если скомпилируем оптимизирующим GCC 4.9:}
\EN{We could see it if we compile it using optimizing GCC 4.9:}

\lstinputlisting[caption=\Optimizing GCC 4.9 x64]{\CURPATH/abs_GCC491_x64_O3.asm.\LANG}

\RU{И вот как он работает}\EN{This is how it works}:

\RU{Арифметически сдвигаем входное значение влево на 31.}
\EN{Arithmetically shift the input value left by 31.}
\RU{Арифметический сдвиг означает знаковое расширение, так что если \ac{MSB} это 1, то все 32 бита будут
заполнены единицами, либо нулями в противном случае.}
\EN{Arithmetical shift means sign extension, so if the \ac{MSB} is 1, all 32 bits will be filled with 1, or with 0
if otherwise.}
\index{x86!\Instructions!SAR}
\RU{Другими словами, инструкция \TT{SAR REG, 31} делает \TT{0xFFFFFFFF} если знак был отрицательным либо 0 если
положительным.}
\EN{In other words, the \TT{SAR REG, 31} instruction makes \TT{0xFFFFFFFF} if the sign was negative or 0 if positive.}
\RU{После исполнения \TT{SAR}, это значение у нас в \EDX.}
\EN{After the execution of \TT{SAR}, we have this value in \EDX.}
\index{x86!\Instructions!XOR}
\RU{Затем, если значение \TT{0xFFFFFFFF} (т.е., знак отрицательный) входное значение инвертируется
(потому что \TT{XOR REG, 0xFFFFFFFF} работает как операция инвертирования всех бит).}
\EN{Then, if the value is \TT{0xFFFFFFFF} (i.e., the sign is negative), the input value is inverted 
(because \TT{XOR REG, 0xFFFFFFFF} is effectively an inverse all bits operation).}
\RU{Затем, снова, если значение \TT{0xFFFFFFFF} (т.е., знак отрицательный), 1 прибавляется к итоговому результату
(потому что вычитание $-1$ из значения это то же что и инкремент).}
\EN{Then, again, if the value is \TT{0xFFFFFFFF} (i.e., the sign is negative), 1 is added to the final result (because
subtracting $-1$ from some value resulting in incrementing it).}
\RU{Инвертирование всех бит и инкремент, это то, как меняется знак у значения в формате two's complement: 
\ref{sec:signednumbers}.}
\EN{Inversion of all bits and incrementing is exactly how two's complement value is negated: 
\ref{sec:signednumbers}.}

\RU{Мы можем заметить, что последние две инструкции делают что-то если знак входного значения отрицательный.}
\EN{We may observe that the last two instruction do something if the sign of the input value is negative.}
\RU{В противном случае (если знак положительный) они не делают ничего, оставляя входное значение нетронутым.}
\EN{Otherwise (if the sign is positive) they do nothing at all, leaving the input value untouched.}

\RU{Алгоритм разъяснен в}\EN{The algorithm is explained in} \cite[2-4]{Warren:2002:HD:515297}.
\RU{Не уверен, как именно GCC сгенерировал его, соптимизировал сам или просто нашел подходящий шаблон среди
известных?}
\EN{I'm not sure, how GCC did it, deduced it by itself or found a suitable pattern among known ones?}

\section{\Optimizing GCC 4.9 ARM64}

\RU{GCC 4.9 для ARM64 делает почти то же, только использует полные 64-битные регистры.}
\EN{GCC 4.9 for ARM64 generates mostly the same, just decides to use the full 64-bit registers.}
\RU{Здесь меньше инструкций, потому что входное значение может быть сдвинуто используя суффикс инструкции (``asr'')
вместо отдельной инструкции.}
\EN{There are less instructions, because the input value can be shifted using a suffixed instruction (``asr'')
instead of using a separate instruction.}

\lstinputlisting[caption=\Optimizing GCC 4.9 ARM64]{\CURPATH/abs_GCC49_ARM64_O3.s.\LANG}
