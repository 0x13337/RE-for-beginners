\section{Branchless \IT{abs()} function}
\label{chap:branchless_abs}

Let's revisit an example we considered earlier \myref{sec:abs} and ask ourselves, is it possible
to make a branchless version of the function in x86 code?

\lstinputlisting{abs.c}

And the answer is yes.

\subsection{\Optimizing GCC 4.9.1 x64}

We could see it if we compile it using optimizing GCC 4.9:

\lstinputlisting[caption=\Optimizing GCC 4.9 x64]{\CURPATH/abs_GCC491_x64_O3_EN.asm}

This is how it works:

Arithmetically shift the input value right by 31.

Arithmetical shift implies sign extension, so if the \ac{MSB} is 1, 
all 32 bits are to be filled with 1, or with 0 if otherwise.
\myindex{x86!\Instructions!SAR}

In other words, the \TT{SAR REG, 31} instruction makes \TT{0xFFFFFFFF} if the sign was negative or 0 if positive.

After the execution of \TT{SAR}, we have this value in \EDX.
\myindex{x86!\Instructions!XOR}

Then, if the value is \TT{0xFFFFFFFF} (i.e., the sign is negative), the input value is inverted \\
(because \TT{XOR REG, 0xFFFFFFFF} is effectively an inverse all bits operation).

Then, again, if the value is \TT{0xFFFFFFFF} (i.e., the sign is negative), 1 is added to the final result (because
subtracting $-1$ from some value resulting in incrementing it).

Inversion of all bits and incrementing is exactly how two's complement value is negated: 
\myref{sec:signednumbers}.

We may observe that the last two instruction do something if the sign of the input value is negative.

Otherwise (if the sign is positive) they do nothing at all, leaving the input value untouched.

The algorithm is explained in \InSqBrackets{\HenryWarren 2-4}.

It's hard to say, how GCC did it, deduced it by itself or found a suitable pattern among known ones?

\subsection{\Optimizing GCC 4.9 ARM64}

GCC 4.9 for ARM64 generates mostly the same, just decides to use the full 64-bit registers.

There are less instructions, because the input value can be shifted using a suffixed instruction (\q{asr})
instead of using a separate instruction.

\lstinputlisting[caption=\Optimizing GCC 4.9 ARM64]{\CURPATH/abs_GCC49_ARM64_O3_EN.s}
