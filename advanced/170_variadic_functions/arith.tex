\section{\RU{Вычисление среднего арифметического}\EN{Computing arithmetic mean}}

\RU{Представим, что нам нужно вычислить \glslink{arithmetic mean}{среднее арифметическое}, 
и по какой-то странной причине, 
нам нужно задать все числа в аргументах функции.}
\EN{Let's imagine that we need to calculate \gls{arithmetic mean}, and for some weird reason 
we need to specify all the values as function arguments.}

\RU{Но в \CCpp функции с переменным кол-вом аргументов невозможно определить кол-во аргументов,
так что обозначим значение $-1$ как конец списка.}%
\EN{But it's impossible to get the number of arguments in a variadic function in \CCpp, so let's denote 
the value of $-1$ as a terminator.}

\RU{Имеется стандартный заголовочный файл stdarg.h, который определяет макросы для работы с такими аргументами.}
\EN{There is the standard stdarg.h header file which define macros for dealing with such arguments.}
\RU{Их так же используют функции \printf и \scanf.}
\EN{The \printf and \scanf functions use them as well.}

\lstinputlisting{\CURPATH/arith.c.\LANG}

\RU{Самый первый аргумент должен трактоваться как обычный аргумент.}
\EN{The first argument has to be treated just like a normal argument.}
\index{\CStandardLibrary!va\_arg}
\RU{Остальные аргументы загружаются используя макрос \TT{va\_arg}, и затем суммируются.}
\EN{All other arguments are loaded using the \TT{va\_arg} macro and then summed.}

\RU{Так что внутри}\EN{So what is inside}?

\subsection{\RU{Соглашение о вызовах \IT{cdecl}}\EN{\IT{cdecl} calling conventions}}

\lstinputlisting[caption=\Optimizing MSVC 6.0]{\CURPATH/arith_MSVC60_Ox.asm.\LANG}

\RU{Аргументы, как мы видим, передаются в \main один за одним.}
\EN{The arguments, as we may see, are passed to \main one-by-one.}
\RU{Первый аргумент заталкивается в локальный стек первым.}
\EN{The first argument is pushed into the local stack as first.}
\RU{Терминатор (оконечивающее значение $-1$) заталкивается последним.}
\EN{The terminating value ($-1$) is pushed last.}

\RU{Функция \TT{arith\_mean()} берет первый аргумент и сохраняет его значение в переменной $sum$.}
\EN{The \TT{arith\_mean()} function takes the value of the first argument and stores it in the $sum$ variable.}
\RU{Затем, она записывает адрес второго аргумента в регистр \EDX, берет значение оттуда, прибавляет к $sum$,
и делает это в бесконечном цикле, до тех пор, пока не встретится $-1$.}
\EN{Then, it sets the \EDX register to the address of the second argument, takes the value from it, adds it to $sum$,
and does this in an infinite loop, until $-1$ is found.}

\RU{Когда встретится, сумма делится на число всех значений (исключая $-1$) и \glslink{quotient}{частное} 
возвращается.}
\EN{When it's found, the sum is divided by the number of all values (excluding $-1$) and the \gls{quotient} is returned.}

\RU{Так что, другими словами, я бы сказал, функция обходится с фрагментом стека как с массивом целочисленных
значений, бесконечной длины.}
\EN{So, in other words, the function treats the stack fragment as an array of integer values of infinite
length.}
\RU{Теперь нам легче понять почему в соглашениях о вызовах \IT{cdecl} первый аргумент 
заталкивается в стек последним.}
\EN{Now we can understand why the \IT{cdecl} calling convention forces us to push the first argument 
into the stack as last.}
\RU{Потому что иначе будет невозможно найти первый аргумент, или, для функции вроде \printf, невозможно
будет найти строку формата.}
\EN{Because otherwise, it would not be possible to find the first argument, 
or, for printf-like functions, it would not be possible to find the address of the format-string.}

\subsection{\RU{Соглашения о вызовах на основе регистров}\EN{Register-based calling conventions}}
\label{variadic_arith_registers}

\RU{Наблюдательный читатель может спросить, что насчет тех соглашений о вызовах, где первые аргументы передаются
в регистрах?}
\EN{The observant reader may ask, what about calling conventions where the first few arguments are passed in registers?}
\RU{Посмотрим}\EN{Let's see}:

\lstinputlisting[caption=\Optimizing MSVC 2012 x64]{\CURPATH/arith_MSVC_2012_Ox_x64.asm.\LANG}

\RU{Мы видим, что первые 4 аргумента передаются в регистрах и еще два\EMDASH{}в стеке.}
\EN{We see that the first 4 arguments are passed in the registers and two more\EMDASH{}in the stack.}
\RU{Функция \TT{arith\_mean()} в начале сохраняет эти 4 аргумента в \IT{Shadow Space} и затем обходится
с \IT{Shadow Space} и стеком за ним как с единым непрерывным массивом!}
\EN{The \TT{arith\_mean()} function first places these 4 arguments into the \IT{Shadow Space} and then treats
the \IT{Shadow Space} and stack behind it as a single continuous array!}

\RU{Что насчет GCC? Тут немного неуклюже всё, потому что функция делится на две части: первая часть
сохраняет регистры в \q{red zone}, обрабатывает это пространство, а вторая часть функции обрабатывает стек:}
\EN{What about GCC? Things are slightly clumsier here, because now the function is divided in two parts:
the first part saves the registers into the \q{red zone}, processes that space, and the second part of the function processes 
the stack:}

\lstinputlisting[caption=\Optimizing GCC 4.9.1 x64]{\CURPATH/arith_GCC491_x64_O3.s.\LANG}

\EN{By the way, a similar usage of the \IT{Shadow Space} is also considered here}
\RU{Кстати, похожее использование \IT{Shadow Space} разбирается здесь}: \myref{pointer_to_argument}.
