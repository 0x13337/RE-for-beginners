\section{ARM: \OptimizingKeilVI (\ARMMode)}

\RU{И снова, компилятор пользуется условными инструкциями в режиме ARM, поэтому код более 
компактный.}
\EN{And again, the compiler took advantage of ARM mode's conditional instructions, 
so the code is much more compact.}

\lstinputlisting[caption=\OptimizingKeilVI (\ARMMode)]{\CURPATH/Keil_ARM_O3.s.\LANG}

\section{ARM: \OptimizingKeilVI (\ThumbMode)}
\index{\CompilerAnomaly}
\label{Keil_anomaly}

\RU{В режиме Thumb куда меньше условных инструкций, так что код более простой.}
\EN{There are less conditional instructions in Thumb mode, so the code is simpler.}
\RU{Но здесь есть одна странность со сдвигами на 0x20 и 0x19.}
\EN{But there are is really weird thing with the 0x20 and 0x19 offsets.}
\RU{Почему компилятор Keil сделал так}\EN{Why did the Keil compiler do so}?
\RU{Честно говоря, трудно сказать}\EN{Honestly, it's hard to say}.
\RU{Возможно, это выверт процесса оптимизации компилятора.}
\EN{Probably, this is a quirk of Keil's optimization process.}
\RU{Тем не менее, код будет работать корректно.}
\EN{Nevertheless, the code works correctly.}

\lstinputlisting[caption=\OptimizingKeilVI (\ThumbMode)]{\CURPATH/Keil_thumb_O3.s.\LANG}
