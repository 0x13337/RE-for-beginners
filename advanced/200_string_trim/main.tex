\chapter{\RU{Обрезка строк}\EN{Strings trimming}}
\newcommand{\CRLF}{\ac{CR}/\ac{LF}}

\RU{Весьма востребованная операция со строками\EMDASH{}это удаление некоторых символов в начале и/или конце
строки.}
\EN{A very common string processing task is to remove some characters at the start and/or at the end.}

\RU{В этом примере, мы будем работать с функцией, удаляющей все символы перевода строки 
(\CRLF{}) в конце входной строки:}
\EN{In this example, we are going to work with a function which removes all newline characters 
(\CRLF{}) from the end of the input string:}

\lstinputlisting{\CURPATH/strtrim.c.\LANG}

\RU{Входной аргумент всегда возвращается на выходе, это удобно, когда вам нужно объединять
функции обработки строк в цепочки, как это сделано здесь в функции \main.}
\EN{The input argument is always returned on exit, this is convenient when you need to chain 
string processing functions, like it was done here in the \main function.}

\RU{Вторая часть for() (\TT{str\_len>0 \&\& (c=s[str\_len-1])}) называется в \CCpp \q{short-circuit} 
(короткое замыкание) и это очень удобно: \cite[1.3.8]{CBook}.}
\EN{The second part of for() (\TT{str\_len>0 \&\& (c=s[str\_len-1])}) is the so called \q{short-circuit} 
in \CCpp and is very convenient \cite[1.3.8]{CBook}.}
\RU{Компиляторы \CCpp гарантируют последовательное вычисление слева направо.}
\EN{The \CCpp compilers guarantee an evaluation sequence from left to right.}
\RU{Так что если первое условие не истинно после вычисления, второе никогда не будет
вычисляться.}
\EN{So if the first clause is false after evaluation, the second one is never to be evaluated.}

% subsections
\subsection{x64: \RU{8 аргументов}\EN{8 arguments}}

\index{x86-64}
\label{example_printf8_x64}
\RU{Для того чтобы посмотреть, как остальные аргументы будут передаваться через стек, 
изменим пример ещё раз, 
увеличив количество передаваемых аргументов до 9 
(строка формата \printf и 8 переменных типа \Tint)}%
\EN{To see how other arguments are passed via the stack, let's change our example again 
by increasing the number of arguments to 9 (\printf format string + 8 \Tint variables)}:

\lstinputlisting{patterns/03_printf/2.c}

\subsubsection{MSVC}

\RU{Как уже было сказано раннее, первые 4 аргумента в Win64 передаются в регистрах}
\EN{As it was mentioned earlier, the first 4 arguments has to be passed through the} \RCX, \RDX, \Reg{8}, \Reg{9}
\RU{, а остальные~--- через стек}\EN{ registers in Win64, while all the rest---via the stack}.
\RU{Здесь мы это и видим}\EN{That is exactly what we see here}.
\RU{Впрочем, инструкция \PUSH не используется, вместо неё при помощи \MOV значения сразу записываются в стек}%
\EN{However, the \MOV instruction, instead of \PUSH, is used for preparing the stack, so the values are stored
to the stack in a straightforward manner}.

\lstinputlisting[caption=MSVC 2012 x64]{patterns/03_printf/x86/2_MSVC_x64.asm.\LANG}

\RU{Наблюдательный читатель может спросить, почему для значений типа \Tint отводится 8 байт,
ведь нужно только 4?}
\EN{The observant reader may ask why are 8 bytes allocated for \Tint values, when 4 is enough?}
\RU{Да, это нужно запомнить: для значений всех типов более коротких чем 64-бита, отводится 8 байт.}
\EN{Yes, one has to remember: 8 bytes are allocated for any data type shorter than 64 bits.}
\RU{Это сделано для удобства: так всегда легко рассчитать адрес того или иного аргумента.}
\EN{This is established for the convenience's sake: it makes it easy to calculate the address of arbitrary argument.}
\RU{К тому же, все они расположены по выровненным адресам в памяти.}
\EN{Besides, they are all located at aligned memory addresses.}
% also for local variables?
\RU{В 32-битных средах точно также: для всех типов резервируется 4 байта в стеке.}
\EN{It is the same in the 32-bit environments: 4 bytes are reserved for all data types.}

\ifdefined\IncludeGCC
\subsubsection{GCC}

\RU{В *NIX-системах для x86-64 ситуация похожая, вот только первые 6 аргументов передаются через}
\EN{The picture is similar for x86-64 *NIX OS-es, except that the first 6 arguments are passed through the} \RDI, \RSI,
\RDX, \RCX, \Reg{8}, \Reg{9}\EN{ registers}.
\RU{Остальные~--- через стек}\EN{All the rest---via the stack}.
\RU{GCC генерирует код, записывающий указатель на строку в \EDI вместо \RDI~--- 
это мы уже рассмотрели чуть раньше}\EN{GCC generates the code storing the string pointer into \EDI instead of \RDI{}---we noted that previously}: \myref{hw_EDI_instead_of_RDI}.

\RU{Почему перед вызовом \printf очищается регистр \EAX мы уже рассмотрели ранее}%
\EN{We also noted earlier that the \EAX register has been cleared before a \printf call}: \myref{SysVABI_input_EAX}.

\lstinputlisting[caption=\Optimizing GCC 4.4.6 x64]{patterns/03_printf/x86/2_GCC_x64.s.\LANG}

\ifdefined\IncludeGDB
\subsubsection{GCC + GDB}
\index{GDB}

\RU{Попробуем этот пример в}\EN{Let's try this example in} \ac{GDB}.

\begin{lstlisting}
$ gcc -g 2.c -o 2
\end{lstlisting}

\begin{lstlisting}
$ gdb 2
GNU gdb (GDB) 7.6.1-ubuntu
Copyright (C) 2013 Free Software Foundation, Inc.
License GPLv3+: GNU GPL version 3 or later <http://gnu.org/licenses/gpl.html>
This is free software: you are free to change and redistribute it.
There is NO WARRANTY, to the extent permitted by law.  Type "show copying"
and "show warranty" for details.
This GDB was configured as "x86_64-linux-gnu".
For bug reporting instructions, please see:
<http://www.gnu.org/software/gdb/bugs/>...
Reading symbols from /home/dennis/polygon/2...done.
\end{lstlisting}

\begin{lstlisting}[caption=\RU{ставим точку останова на \printf{,} запускаем}\EN{let's set the breakpoint to \printf{,} and run}]
(gdb) b printf
Breakpoint 1 at 0x400410
(gdb) run
Starting program: /home/dennis/polygon/2 

Breakpoint 1, __printf (format=0x400628 "a=%d; b=%d; c=%d; d=%d; e=%d; f=%d; g=%d; h=%d\n") at printf.c:29
29	printf.c: No such file or directory.
\end{lstlisting}

\RU{В регистрах}\EN{Registers} \RSI/\RDX/\RCX/\Reg{8}/\Reg{9} 
\RU{всё предсказуемо}\EN{have the expected values}.
\RU{А }\RIP \RU{содержит адрес самой первой инструкции функции}\EN{has the address of the very first instruction
of the} \printf\EN{ function}.

\begin{lstlisting}
(gdb) info registers
rax            0x0	0
rbx            0x0	0
rcx            0x3	3
rdx            0x2	2
rsi            0x1	1
rdi            0x400628	4195880
rbp            0x7fffffffdf60	0x7fffffffdf60
rsp            0x7fffffffdf38	0x7fffffffdf38
r8             0x4	4
r9             0x5	5
r10            0x7fffffffdce0	140737488346336
r11            0x7ffff7a65f60	140737348263776
r12            0x400440	4195392
r13            0x7fffffffe040	140737488347200
r14            0x0	0
r15            0x0	0
rip            0x7ffff7a65f60	0x7ffff7a65f60 <__printf>
...
\end{lstlisting}

\begin{lstlisting}[caption=\RU{смотрим на строку формата}\EN{let's inspect the format string}]
(gdb) x/s $rdi
0x400628:	"a=%d; b=%d; c=%d; d=%d; e=%d; f=%d; g=%d; h=%d\n"
\end{lstlisting}

\RU{Дампим стек на этот раз с командой x/g}\EN{Let's dump the stack with the x/g command this time}\EMDASH{}g 
\RU{означает}\EN{stands for} \IT{giant words}, \RU{т.е., 64-битные слова}\EN{i.e., 64-bit words}.

\begin{lstlisting}
(gdb) x/10g $rsp
0x7fffffffdf38:	0x0000000000400576	0x0000000000000006
0x7fffffffdf48:	0x0000000000000007	0x00007fff00000008
0x7fffffffdf58:	0x0000000000000000	0x0000000000000000
0x7fffffffdf68:	0x00007ffff7a33de5	0x0000000000000000
0x7fffffffdf78:	0x00007fffffffe048	0x0000000100000000
\end{lstlisting}

\RU{Самый первый элемент стека, как и в прошлый раз, это}\EN{The very first stack element, 
just like in the previous case, is the} \ac{RA}.
\RU{Через стек также передаются 3 значения}\EN{3 values are also passed through the stack}: 6, 7, 8.
\RU{Видно, что 8 передается с неочищенной старшей 32-битной частью}\EN{We also see that 8 is passed
with the high 32-bits not cleared}: \TT{0x00007fff00000008}.
\RU{Это нормально, ведь передаются числа типа \Tint, а они 32-битные}\EN{That's OK, because the values have
\Tint type, which is 32-bit}.
\RU{Так что в старшей части регистра или памяти стека остался ``случайный мусор''}\EN{So, the high register
or stack element part may contain ``random garbage''}.

\RU{\ac{GDB} показывает всю функцию \main, если попытаться посмотреть, куда вернется управление после исполнения \printf}%
\EN{If you take a look at where the control will return after the \printf execution,
\ac{GDB} will show the entire \main function}:

\begin{lstlisting}
(gdb) set disassembly-flavor intel
(gdb) disas 0x0000000000400576
Dump of assembler code for function main:
   0x000000000040052d <+0>:	push   rbp
   0x000000000040052e <+1>:	mov    rbp,rsp
   0x0000000000400531 <+4>:	sub    rsp,0x20
   0x0000000000400535 <+8>:	mov    DWORD PTR [rsp+0x10],0x8
   0x000000000040053d <+16>:	mov    DWORD PTR [rsp+0x8],0x7
   0x0000000000400545 <+24>:	mov    DWORD PTR [rsp],0x6
   0x000000000040054c <+31>:	mov    r9d,0x5
   0x0000000000400552 <+37>:	mov    r8d,0x4
   0x0000000000400558 <+43>:	mov    ecx,0x3
   0x000000000040055d <+48>:	mov    edx,0x2
   0x0000000000400562 <+53>:	mov    esi,0x1
   0x0000000000400567 <+58>:	mov    edi,0x400628
   0x000000000040056c <+63>:	mov    eax,0x0
   0x0000000000400571 <+68>:	call   0x400410 <printf@plt>
   0x0000000000400576 <+73>:	mov    eax,0x0
   0x000000000040057b <+78>:	leave  
   0x000000000040057c <+79>:	ret    
End of assembler dump.
\end{lstlisting}

\RU{Заканчиваем исполнение \printf, исполняем инструкцию обнуляющую \EAX, 
удостоверяемся что в регистре \EAX именно ноль}\EN{Let's finish executing \printf, execute the instruction
zeroing \EAX, and note that the \EAX register has a value of exactly zero}.
\RIP \RU{указывает сейчас на инструкцию}\EN{now points to the} \TT{LEAVE}\RU{, т.е., предпоследнюю в функции \main}
\EN{ instruction, i.e., the penultimate one in the \main function}.

\begin{lstlisting}
(gdb) finish
Run till exit from #0  __printf (format=0x400628 "a=%d; b=%d; c=%d; d=%d; e=%d; f=%d; g=%d; h=%d\n") at printf.c:29
a=1; b=2; c=3; d=4; e=5; f=6; g=7; h=8
main () at 2.c:6
6		return 0;
Value returned is $1 = 39
(gdb) next
7	};
(gdb) info registers
rax            0x0	0
rbx            0x0	0
rcx            0x26	38
rdx            0x7ffff7dd59f0	140737351866864
rsi            0x7fffffd9	2147483609
rdi            0x0	0
rbp            0x7fffffffdf60	0x7fffffffdf60
rsp            0x7fffffffdf40	0x7fffffffdf40
r8             0x7ffff7dd26a0	140737351853728
r9             0x7ffff7a60134	140737348239668
r10            0x7fffffffd5b0	140737488344496
r11            0x7ffff7a95900	140737348458752
r12            0x400440	4195392
r13            0x7fffffffe040	140737488347200
r14            0x0	0
r15            0x0	0
rip            0x40057b	0x40057b <main+78>
...
\end{lstlisting}
\fi
\fi

\subsection{ARM64}

\subsubsection{\Optimizing GCC (Linaro) 4.9}

\index{Fused multiply–add}
\index{ARM!\Instructions!MADD}
\RU{Тут всё просто}\EN{Everything here is simple}.
\EN{\TT{MADD} is just an instruction doing fused multiply/add (similar to the \TT{MLA} we already saw).}
\RU{\TT{MADD} это просто инструкция, производящая умножение и сложение одновременно (как \TT{MLA}, 
которую мы уже видели).}
\EN{All 3 arguments are passed in the 32-bit parts of X-registers.}
\RU{Все 3 аргумента передаются в 32-битных частях X-регистров.}
\EN{Indeed, the argument types are 32-bit \IT{int}'s.}
\RU{Действительно, типы аргументов это 32-битные \IT{int}'ы.}
\EN{The result is returned in \TT{W0}.}
\RU{Результат возвращается в \TT{W0}.}

\lstinputlisting[caption=\Optimizing GCC (Linaro) 4.9]{patterns/05_passing_arguments/ARM/ARM64_O3.s.\LANG}

\EN{Let's also extend all data types to 64-bit \TT{uint64\_t} and test:}%
\RU{Также расширим все типы данных до 64-битных \TT{uint64\_t} и попробуем:}

\lstinputlisting{patterns/05_passing_arguments/ex64.c}

\begin{lstlisting}
f:
	madd	x0, x0, x1, x2
	ret
main:
	mov	x1, 13396
	adrp	x0, .LC8
	stp	x29, x30, [sp, -16]!
	movk	x1, 0x27d0, lsl 16
	add	x0, x0, :lo12:.LC8
	movk	x1, 0x122, lsl 32
	add	x29, sp, 0
	movk	x1, 0x58be, lsl 48
	bl	printf
	mov	w0, 0
	ldp	x29, x30, [sp], 16
	ret

.LC8:
	.string	"%lld\n"
\end{lstlisting}

\EN{The \ttf{} function is the same, only the whole 64-bit X-registers are now used.}%
\RU{Функция \ttf{} точно такая же, только теперь используются полные части 64-битных X-регистров.}
\RU{Длинные 64-битные значения загружаются в регистры по частям, это описано здесь}%
\EN{Long 64-bit values are loaded into the registers by parts, this is also described here}: \myref{ARM_big_constants_loading}.

\subsubsection{\NonOptimizing GCC (Linaro) 4.9}

\EN{The non-optimizing compiler is more redundant:}
\RU{Неоптимизирующий компилятор выдает немного лишнего кода:}

\begin{lstlisting}
f:
	sub	sp, sp, #16
	str	w0, [sp,12]
	str	w1, [sp,8]
	str	w2, [sp,4]
	ldr	w1, [sp,12]
	ldr	w0, [sp,8]
	mul	w1, w1, w0
	ldr	w0, [sp,4]
	add	w0, w1, w0
	add	sp, sp, 16
	ret
\end{lstlisting}

\EN{The code saves its input arguments in the local stack, 
in case someone (or something) in this function needs using the \TT{W0...W2} 
registers. This prevents overwriting the original
function arguments, which may be needed again in the future.}
\RU{Код сохраняет входные аргументы в локальном стеке на случай если кому-то (или чему-то) в этой функции
понадобится использовать регистры \TT{W0...W2}, перезаписывая оригинальные аргументы функции, которые
могут понадобится в будущем.}
\RU{Это называется}\EN{This is called} \IT{Register Save Area.} \cite{ARM64_PCS}
\RU{Вызываемая функция не обязана сохранять их.}\EN{ The callee, however, is not obliged to save them.}
\RU{Это то же что и}\EN{This is somewhat similar to} \q{Shadow Space}: \myref{shadow_space}.

\RU{Почему оптимизирующий GCC 4.9 убрал этот, сохраняющий аргументы, код?}
\EN{Why did the optimizing GCC 4.9 drop this argument saving code?}
\EN{Because it did some additional optimizing work and concluded
that the function arguments will not be needed in the future 
and also that the registers \TT{W0...W2} will not be used.}
\RU{Потому что он провел дополнительную работу по оптимизации и сделал вывод, 
что аргументы функции не понадобятся в будущем и регистры \TT{W0...W2} также не будут использоваться.}

\index{ARM!\Instructions!MUL}
\index{ARM!\Instructions!ADD}
\RU{Также мы видим пару инструкций \TT{MUL}/\TT{ADD} вместо одной \TT{MADD}.}
\EN{We also see a \TT{MUL}/\TT{ADD} instruction pair instead of single a \TT{MADD}.}

\subsection{ARM}

\subsubsection{\OptimizingKeilVI (\ThumbMode)}

\begin{lstlisting}
.text:00000042             scanf_main
.text:00000042
.text:00000042             var_8           = -8
.text:00000042
.text:00000042 08 B5                       PUSH    {R3,LR}
.text:00000044 A9 A0                       ADR     R0, aEnterX     ; "Enter X:\n"
.text:00000046 06 F0 D3 F8                 BL      __2printf
.text:0000004A 69 46                       MOV     R1, SP
.text:0000004C AA A0                       ADR     R0, aD          ; "%d"
.text:0000004E 06 F0 CD F8                 BL      __0scanf
.text:00000052 00 99                       LDR     R1, [SP,#8+var_8]
.text:00000054 A9 A0                       ADR     R0, aYouEnteredD___ ; "You entered %d...\n"
.text:00000056 06 F0 CB F8                 BL      __2printf
.text:0000005A 00 20                       MOVS    R0, #0
.text:0000005C 08 BD                       POP     {R3,PC}
\end{lstlisting}

\index{\CLanguageElements!\Pointers}
\RU{Чтобы \scanf мог вернуть значение, ему нужно передать указатель на переменную типа \Tint.}
\EN{In order for \scanf to be able to read item it needs a parameter---pointer to an \Tint.}
\Tint\RU{~--- 32-битное значение, для его хранения нужно только 4 байта, и оно помещается в 
32-битный регистр.}
\EN{is 32-bit, so we need 4 bytes to store it somewhere in memory, and it fits exactly 
in a 32-bit register.}
\index{IDA!var\_?}
\RU{Место для локальной переменной \TT{x} выделяется в стеке, \IDA наименовала её \IT{var\_8}. 
Впрочем, место для неё выделять не обязательно, т.к. \glslink{stack pointer}{указатель стека} \ac{SP} уже указывает на место, 
свободное для использования.}\EN{A place for the local variable \TT{x} is allocated in the stack and \IDA
has named it \IT{var\_8}. It is not necessary, however, to allocate a such since \ac{SP} (\gls{stack pointer}) is already pointing to that space and it can be used directly.}
\RU{Так что значение указателя \ac{SP} копируется в регистр \Reg{1}, и вместе с format-строкой, 
передается в \scanf.}
\EN{So, \ac{SP}'s value is copied to the \Reg{1} register and, together with the format-string, passed
to \scanf.}
\index{ARM!\Instructions!LDR}
\RU{Позже, при помощи инструкции \TT{LDR}, это значение перемещается из стека в регистр \Reg{1}, 
чтобы быть переданным в \printf.}\EN{Later, with the help of the \TT{LDR} instruction, this value is moved
from the stack to the \Reg{1} register in order to be passed to \printf.}

\subsubsection{ARM64}

\lstinputlisting[caption=\NonOptimizing GCC 4.9.1 ARM64,numbers=left]{patterns/04_scanf/1_simple/ARM64_GCC491_O0.s.\LANG}

\RU{Под стековый фрейм выделяется 32 байта, что больше чем нужно. Вероятно, это связано с выравниваем по границе памяти?}%
\EN{There is 32 bytes are allocated for stack frame, which is bigger than it needed. Perhaps, some memory aligning issue?}
\RU{Самая интересная часть~--- это поиск места под переменную $x$ в стековом фрейме (строка 22).}
\EN{The most interesting part is finding space for the $x$ variable in the stack frame (line 22).}
\RU{Почему 28? Почему-то, компилятор решил расположить эту переменную в конце стекового фрейма, а не в начале.}%
\EN{Why 28? Somehow, compiler decided to place this variable at the end of stack frame instead of beginning.}
\RU{Адрес потом передается в \scanf, которая просто сохраняет значение, введенное пользователем, в памяти
по этому адресу.}
\EN{The address is passed to \scanf, which just stores the user input value in the memory at that address.}
\RU{Это 32-битное значение типа \Tint}\EN{This is 32-bit value of type \Tint}.
\RU{Значение загружается в строке 27 и затем передается в \printf.}
\EN{The value is fetched at line 27 and then passed to \printf.}


\section{MIPS}

\lstinputlisting[caption=\Optimizing GCC 4.4.5]{patterns/05_passing_arguments/MIPS_O3_IDA.lst}

\RU{Первые 4 аргумента ф-ции передаются в четырех регистрах с префиксами A-.}
\EN{First four function arguments are passed in four registers prefixed by A-.}

\index{MIPS!\Instructions!MULT}
\RU{В MIPS есть два специальных регистра: HI и LO, которые выставляются в 64-битный результат умножения
во время исполнения инструцкии \TT{MULT}.}
\EN{There are two special registers in MIPS: HI and LO which are filled by 64-bit result of multiplication while
execution of \TT{MULT} instruction.}
\index{MIPS!\Instructions!MFLO}
\index{MIPS!\Instructions!MFHI}
\RU{К регистрам можно обращаться только используя инструкции \TT{MFLO} и \TT{MFHI}.}
\EN{Registers are accessible only using \TT{MFLO} and \TT{MFHI} instructions.}
\RU{Здесь, \TT{MFLO} берет младшую часть результата умножения и записывает в \$V0.}
\EN{\TT{MFLO} here is taking low-part of result of multiplication and putting it into \$V0.}

\RU{Так что старшая 32-битная часть результата игнорируется (содержимое регистра HI не используется).}
\EN{So high 32-bit part of multiplication result is dropped (contents of HI register is not used).}
\RU{Действительно: мы ведь работаем с 32-битным типом \Tint.}
\EN{Indeed: we work with 32-bit \Tint data type here.}

\index{MIPS!\Instructions!ADDU}
\RU{И наконец, \TT{ADDU} (``Add Unsigned'') прибавляет значение третьего аргумента к результату.}
\EN{Finally, \TT{ADDU} (``Add Unsigned'') adds value of the third argument to the result.}

\index{MIPS!\Instructions!ADD}
\index{MIPS!\Instructions!ADDU}
\index{Ada}
\index{Integer overflow}
\RU{В MIPS есть две разных инструкции сложения:}
\EN{There are two different addition instructions in MIPS:} \TT{ADD} \AndENRU \TT{ADDU}.
\RU{На самом деле, дело не в знаковых числах, но в исключениях: \TT{ADD} может вызвать исключение
во время переполнения, а это иногда полезно\footnote{\url{http://blog.regehr.org/archives/1154}} и поддерживается,
например, в \ac{PL} Ada.}
\EN{In fact, it's not about signedness, but about exceptions: \TT{ADD} can raise exception on overflow,
which is sometimes useful\footnote{\url{http://blog.regehr.org/archives/1154}} and supported in Ada \ac{PL}, 
for instance.}
\TT{ADDU} \RU{не вызывает исключения во время переполнения}\EN{do not raise exceptions on overflow}.
\RU{А так как \CCpp не поддерживает всё это, мы видим здесь \TT{ADDU} вместо \TT{ADD}.}
\EN{Since, \CCpp doesn't support this, here we see \TT{ADDU} instead of \TT{ADD}.}

\RU{32-битный результат оставляется в}\EN{32-bit result is leaved in} \$V0.

\index{MIPS!\Instructions!JAL}
\index{MIPS!\Instructions!JALR}
\RU{В \main есть новая для нас инструкция:}
\EN{There are new instruction for us in \main:} \TT{JAL} (``Jump and Link''). 
\RU{Разница между JAL и JALR в том что относительное смещение кодируется в первой инструкции,
а JALR переходит по абсолютному адресу записанному в регистр (``Jump and Link Register'').}
\EN{Difference between JAL and JALR is that relative offset is encoded in first instruction, 
while JALR jumping to the absolute address stored in register (``Jump and Link Register'').}
\RU{Обе ф-ции \ttf и \main расположены в одном объектном файле, так что относительный адрес
\ttf известен и фиксирован.}
\EN{Both \ttf and \main functions are located in the same object file, so relative address of \ttf 
is known and fixed.}


