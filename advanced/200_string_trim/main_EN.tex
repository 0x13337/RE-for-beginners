\section{Strings trimming}
\newcommand{\CRLF}{\ac{CR}/\ac{LF}}

A very common string processing task is to remove some characters at the start and/or at the end.

In this example, we are going to work with a function which removes all newline characters 
(\CRLF{}) from the end of the input string:

\lstinputlisting{\CURPATH/strtrim_EN.c}

The input argument is always returned on exit, this is convenient when you need to chain 
string processing functions, like it was done here in the \main function.

The second part of for() (\TT{str\_len>0 \&\& (c=s[str\_len-1])}) is the so called \q{short-circuit} 
in \CCpp and is very convenient [\CNotes 1.3.8].

The \CCpp compilers guarantee an evaluation sequence from left to right.

So if the first clause is false after evaluation, the second one is never to be evaluated.

% subsections
\subsection{x64}
\label{subsec:popcnt}

Let's modify the example slightly to extend it to 64-bit:

\lstinputlisting[label=popcnt_x64_example]{patterns/14_bitfields/4_popcnt/shifts64.c}

\subsubsection{\NonOptimizing GCC 4.8.2}

So far so easy.

\lstinputlisting[caption=\NonOptimizing GCC 4.8.2]{patterns/14_bitfields/4_popcnt/shifts64_GCC_O0_EN.s}

\subsubsection{\Optimizing GCC 4.8.2}

\lstinputlisting[caption=\Optimizing GCC 4.8.2,numbers=left,label=shifts64_GCC_O3]{patterns/14_bitfields/4_popcnt/shifts64_GCC_O3_EN.s}

This code is terser, but has a quirk.

In all examples that we see so far, we were incrementing the \q{rt} value after comparing a specific bit,
but the code here increments \q{rt} before (line 6), writing the new value into register \EDX .
Thus, if the last bit is 1, the \CMOVNE\footnote{Conditional MOVe if Not Equal} instruction
(which is a synonym for \CMOVNZ\footnote{Conditional MOVe if Not Zero}) \IT{commits} 
the new value of \q{rt}
by moving \EDX (\q{proposed rt value}) into \EAX (\q{current rt} to be returned at the end).

Hence, the incrementing is done at each step of loop, i.e., 64 times, without any relation to the input value.

The advantage of this code is that it contain only one conditional jump (at the end of the loop) instead of 
two jumps (skipping the \q{rt} value increment and at the end of loop).
And that might work faster on the modern CPUs with branch predictors: \myref{branch_predictors}.

\label{FATRET}
\myindex{x86!\Instructions!FATRET}
The last instruction is \INS{REP RET} (opcode \TT{F3 C3}) 
which is also called \INS{FATRET} by MSVC.
This is somewhat optimized version of \RET, 
which is recommended by AMD to be placed at the end of function, if \RET goes right after conditional jump: 
[\AMDOptimization p.15]
\footnote{More information on it: \url{http://go.yurichev.com/17328}}.

\subsubsection{\Optimizing MSVC 2010}

\lstinputlisting[caption=MSVC 2010]{patterns/14_bitfields/4_popcnt/MSVC_2010_x64_Ox_EN.asm}

\myindex{x86!\Instructions!ROL}
Here the \ROL instruction is used instead of 
\SHL, which is in fact \q{rotate left} 
instead of \q{shift left},
but in this example it works just as \TT{SHL}.

You can read more about the rotate instruction here: \myref{ROL_ROR}.

\Reg{8} here is counting from 64 to 0.
It's just like an inverted $i$.

Here is a table of some registers during the execution:

\begin{center}
\begin{tabular}{ | l | l | }
\hline
\HeaderColor RDX & \HeaderColor R8 \\
\hline
0x0000000000000001 & 64 \\
\hline
0x0000000000000002 & 63 \\
\hline
0x0000000000000004 & 62 \\
\hline
0x0000000000000008 & 61 \\
\hline
... & ... \\
\hline
0x4000000000000000 & 2 \\
\hline
0x8000000000000000 & 1 \\
\hline
\end{tabular}
\end{center}

\myindex{x86!\Instructions!FATRET}
At the end we see the \INS{FATRET} instruction, which was explained here: \myref{FATRET}.

\subsubsection{\Optimizing MSVC 2012}

\lstinputlisting[caption=MSVC 2012]{patterns/14_bitfields/4_popcnt/MSVC_2012_x64_Ox_EN.asm}

\myindex{\CompilerAnomaly}
\label{MSVC2012_anomaly}
\Optimizing MSVC 2012 does almost the same job as 
optimizing MSVC 2010, but somehow, it generates two identical loop bodies and the loop count is now 32 instead of 64.

To be honest, it's not possible to say why. Some optimization trick? Maybe it's better for the loop body to be slightly 
longer?

Anyway, such code is relevant here to show that sometimes the compiler output may be really weird and 
illogical, but perfectly working.


\subsubsectionold{ARM64}

\myparagraphold{\Optimizing GCC (Linaro) 4.9}

\lstinputlisting{patterns/10_strings/1_strlen/ARM/ARM64_GCC_O3_EN.lst}

The algorithm is the same as in \myref{strlen_MSVC_Ox}: 
find a zero 
byte, calculate the difference between the pointers and decrement the result by 1.
Some comments were added by the author of this book.

The only thing worth noting is that our example is somewhat wrong: \TT{my\_strlen()}
returns 32-bit \Tint, while it has to return \TT{size\_t} or another 64-bit type.

The reason is that, theoretically, \TT{strlen()} can be called for a huge blocks in memory that exceeds
4GB, so it must able to return a 64-bit value on 64-bit platforms.

Because of my mistake, the last \SUB instruction operates on a 32-bit part of register, while the penultimate
\SUB instruction works on full the 64-bit register (it calculates the difference between the pointers).

It's my mistake, it is better to leave it as is, as an example of how the code could look like in such case.

\myparagraphold{\NonOptimizing GCC (Linaro) 4.9}

\lstinputlisting{patterns/10_strings/1_strlen/ARM/ARM64_GCC_O0_EN.lst}

It's more verbose.
The variables are often tossed here to and from memory (local stack).
The same mistake here: the decrement operation happens on a 32-bit register part.


\subsection{ARM}

\subsubsection{\OptimizingKeilVI (\ThumbMode)}

\lstinputlisting[caption=\OptimizingKeilVI (\ThumbMode)]{patterns/15_structs/4_packing/packing_Keil_thumb.asm}

As we may recall, here a structure is passed instead of pointer to one,
and since the first 4 function arguments in ARM are passed via registers,
the structure's fields are passed via \TT{R0-R3}.

\myindex{ARM!\Instructions!LDRB}
\myindex{x86!\Instructions!MOVSX}
\TT{LDRB} loads one byte from memory and extends it to 32-bit, taking its sign into account.
This is similar to \MOVSX in x86.
Here it is used to load fields $a$ and $c$ from the structure.

\myindex{Function epilogue}

One more thing we spot easily is that instead of function epilogue, there is jump to another function's epilogue!
Indeed, that was quite different function, not related in any way to ours, however, it has exactly
the same epilogue 
(probably because, it hold 5 local variables too 
($5*4=0x14$)).

Also it is located nearby (take a look at the addresses).

Indeed, it doesn't matter which epilogue gets executed,
if it works just as we need.

Apparently, Keil decides to reuse a part of another function to economize.

The epilogue takes 4 bytes while jump~---only 2.

\subsubsection{ARM + \OptimizingXcodeIV (\ThumbTwoMode)}

\lstinputlisting[caption=\OptimizingXcodeIV (\ThumbTwoMode)]{patterns/15_structs/4_packing/packing_Xcode_thumb.asm}

\myindex{ARM!\Instructions!SXTB}
\myindex{x86!\Instructions!MOVSX}
\TT{SXTB} (\IT{Signed Extend Byte}) is analogous to \MOVSX in x86.
All the rest~---just the same.


\subsectionold{MIPS}

\lstinputlisting[caption=\Optimizing GCC 4.4.5 (IDA)]{patterns/12_FPU/2_passing_floats/MIPS_O3_IDA_EN.lst}

And again, we see here \INS{LUI} loading a 32-bit part of a \Tdouble number into \$V0.
And again, it's hard to comprehend why.

\myindex{MIPS!\Instructions!MFC1}

The new instruction for us here is \INS{MFC1} (\q{Move From Coprocessor 1}).
The FPU is coprocessor number 1, hence \q{1} in the instruction name.
This instruction transfers values from the coprocessor's registers to the registers of the CPU (\ac{GPR}).
So in the end the result from \TT{pow()} is moved to registers \$A3 and \$A2, 
and \printf takes a 64-bit double value from this register pair.



