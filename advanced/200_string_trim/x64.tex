\section{x64: \Optimizing MSVC 2013}

\lstinputlisting[caption=\Optimizing MSVC 2013 x64]{\CURPATH/MSVC2013_x64_Ox.asm.\LANG}

\RU{В начале, MSVC вставил тело функции \strlen{} прямо в код, потому что решил, что так будет
быстрее чем обычная работа \strlen{} + время на вызов её и возврат из нее.}
\EN{First, MSVC inlined the \strlen{} function code, because it concluded this 
is to be faster than the usual \strlen{} work + the cost of calling it and returning from it.}
\RU{Это также называется \IT{inlining}}\EN{This is called inlining}: \myref{inline_code}.

\index{x86!\Instructions!OR}
\index{\CStandardLibrary!strlen()}
\RU{Первая инструкция функции \strlen{} вставленная здесь, это}\EN{The first instruction of 
the inlined \strlen{} is} \TT{OR RAX, 0xFFFFFFFFFFFFFFFF}. 
\RU{Трудно сказать, почему MSVC использовала \TT{OR} вместо \TT{MOV RAX, 0xFFFFFFFFFFFFFFFF}, но он делает это часто.}
\EN{It's hard to say why MSVC uses \TT{OR} instead of \TT{MOV RAX, 0xFFFFFFFFFFFFFFFF}, but it does this often.}
\RU{И конечно, это эквивалентно друг другу: все биты просто выставляются, а все выставленные
биты это -1 в дополнительном коде (two's complement): \myref{sec:signednumbers}.}
\EN{And of course, it is equivalent: all bits are set, and a number with all bits set is $-1$ 
in two's complement arithmetic: \myref{sec:signednumbers}.}

\RU{Кто-то мог бы спросить, зачем вообще нужно использовать число -1 в функции \strlen{}?}
\EN{Why would the $-1$ number be used in \strlen{}, one might ask.}
\RU{Вследствие оптимизации, конечно}\EN{Due to optimizations, of course}.
\RU{Вот что сделал MSVC}\EN{Here is the code that MSVC generated}:

\lstinputlisting[caption=\RU{Вставленная \strlen{} сгенерированная}\EN{Inlined \strlen{} by} MSVC 2013 x64]{\CURPATH/strlen_MSVC.asm.\LANG}

\RU{Попробуйте написать короче, если хотите инициализировать счетчик нулем!}
\EN{Try to write shorter if you want to initialize the counter at 0!}
\RU{Ну, например}\EN{OK, let' try}:

\lstinputlisting[caption=\RU{Наша версия}\EN{Our version of} \strlen{}]{\CURPATH/my_strlen.asm.\LANG}

\RU{Не получилось. Нам придется вводить дополнительную инструкцию JMP!}
\EN{We failed. We need to use additional JMP instruction!}

\RU{Что сделал MSVC 2013, так это передвинул инструкцию \TT{INC} в место перед загрузкой символа.}
\EN{So what the MSVC 2013 compiler did is to move the \TT{INC} instruction to the place before 
the actual character loading.}
\RU{Если самый первый символ\EMDASH{}нулевой, всё нормально, \RAX содержит 0 в этот момент, так что
итоговая длина строки будет 0.}
\EN{If the first character is 0, that's OK, \RAX is 0 at this moment, 
so the resulting string length is 0.}

\RU{Остальную часть функции проще понять.}
\EN{The rest in this function seems easy to understand.}
\RU{Еще один трюк в конце}\EN{There is another trick at the end}.
\RU{Если не учитывать вставленный код \strlen{}, здесь только 3 условных перехода в функции.}
\EN{If we do not count \strlen{}'s inlined code, there are only 3 conditional jumps in the function.}
\RU{Должно было быть 4: четвертый должен был быть в конце функции, проверяющий, является
ли символ нулевым.}
\EN{There should be 4: the 4th has to be located at the function end, to check 
if the character is zero.}
\RU{Но там безусловный переход на метку}\EN{But there is an unconditional jump to the} 
\q{\$LN18@str\_trim}\RU{, где мы видим \TT{JE}, которая в начале служила
для проверки пустоты входящей строки, прямо после завершения \strlen{}.}\EN{ label, where 
we see \TT{JE}, which was first used to check if the input string is empty, right after \strlen{} 
finishes.}
\RU{Так что код использует инструкцию}\EN{So the code uses the} \TT{JE} \RU{в этом месте
для двух целей}\EN{instruction at this place for two purposes}!
\RU{Это может быть излишним, но так или иначе, MSVC так сделал.}
\EN{This may be overkill, but nevertheless, MSVC did it.}

\RU{О том, почему так важно обходиться без условных переходов, если это возможно,
читайте здесь:}
\EN{You can read more on why it's important to do the job without conditional jumps, 
if possible:} \myref{branch_predictors}.

\section{x64: \NonOptimizing GCC 4.9.1}

\lstinputlisting{\CURPATH/GCC491_x64_O0.asm.\LANG}

\RU{Комментарии\EMDASH{}автора книги}\EN{Comments are added by the author of the book}.
\RU{После исполнения \strlen{}, управление передается на метку L2,
и там проверяются два выражения, одно после другого.}
\EN{After the execution of \strlen{}, the control is passed to the L2 label, 
and there two clauses are checked, one after another.}
\RU{Второе никогда не будет проверяться, если первое выражение не истинно (\IT{str\_len==0})
(это \q{short-circuit}).}
\EN{The second will never be checked, if the first one (\IT{str\_len==0}) is false 
(this is \q{short-circuit}).}

\RU{Теперь посмотрим на эту функцию в коротком виде:}
\EN{Now let's see this function in short form:}

\begin{itemize}
\item \RU{Первая часть for() (вызов \strlen{})}\EN{First for() part (call to \strlen{})}
\item goto L2
\item L5: \RU{Тело for(). переход на выход, если нужно}\EN{for() body. goto exit, if needed}
\item \RU{Третья часть for() (декремент str\_len)}\EN{for() third part (decrement of str\_len)}
\item L2: \RU{Вторая часть for(): проверить первое выражение, затем второе. 
переход на начало тела цикла, или выход.}
\EN{for() second part: check first clause, then second. goto loop body begin or exit.}
\item L4: // \RU{выход}\EN{exit}
\item return s
\end{itemize}

\section{x64: \Optimizing GCC 4.9.1}
\label{string_trim_GCC_x64_O3}

\lstinputlisting{\CURPATH/GCC491_x64_O3.asm.\LANG}

\RU{Тут более сложный результат}\EN{Now this is more complex}.
\RU{Код перед циклом исполняется только один раз, но также содержит проверку символов
\CRLF{}!}
\EN{The code before the loop's body start is executed only once, but it has the \CRLF{} 
characters check too!}
\RU{Зачем нужна это дублирование кода}\EN{What is this code duplication for}?

\RU{Обычная реализация главного цикла это\EMDASH{}наверное, такая:}
\EN{The common way to implement the main loop is probably this:}

\begin{itemize}
\item (\RU{начало цикла}\EN{loop start}) \RU{проверить символы}\EN{check for} 
\CRLF{}\RU{, принять решения}\EN{ characters, make decisions}
\item \RU{записать нулевой символ}\EN{store zero character}
\end{itemize}

\RU{Но GCC решил поменять местами эти два шага}\EN{But GCC has decided to reverse these two steps}. 
\RU{Конечно, шаг \IT{записать нулевой символ} не может быть первым, так что нужна еще одна
проверка:}
\EN{Of course, \IT{store zero character} cannot be first step, so another check is needed:}

\begin{itemize}
\item 
\RU{обработать первый символ. сравнить его с \CRLF{}, выйти если символ не равен \CRLF{}}
\EN{workout first character. match it to \CRLF{}, exit if character is not \CRLF{}}
\item (\RU{начало цикла}\EN{loop begin}) \RU{записать нулевой символ}\EN{store zero character}
\item 
\RU{проверить символы \CRLF{}, принять решения}
\EN{check for \CRLF{} characters, make decisions}
\end{itemize}

\RU{Теперь основной цикл очень короткий, а это очень хорошо для современных процессоров.}
\EN{Now the main loop is very short, which is good for modern \ac{CPU}s.}

\RU{Код не использует переменную str\_len, но str\_len-1.}
\EN{The code doesn't use the str\_len variable, but str\_len-1.}
\RU{Так что это больше похоже на индекс в буфере}\EN{So this is more like an index in a buffer}.
\RU{Должно быть, GCC заметил, что выражение str\_len-1 используется дважды.}
\EN{Apparently, GCC notices that the str\_len-1 statement is used twice.}
\RU{Так что будет лучше выделить переменную, которая всегда содержит значение равное 
текущей длине строки минус 1, и уменьшать его на 1 (это тот же эффект, что и уменьшать
переменную str\_len).}
\EN{So it's better to allocate a variable which always holds a value that's smaller than 
the current string length by one, 
and decrement it (this is the same effect as decrementing the str\_len variable).}
