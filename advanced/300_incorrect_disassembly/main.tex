\chapter{\RU{Неверно дизассемблированный код}\EN{Incorrectly disassembled code}}
\label{sec:incorrectly_disasmed_code}

\RU{Практикующие reverse engineer-ы часто сталкиваются с неверно дизассемблированным кодом}
\EN{Practicing reverse engineers often have to deal with incorrectly disassembled code}.

\section{\RU{Дизассемблирование началось в неверном месте}\EN{Disassembling from an incorrect start} (x86)}

\RU{В отличие от ARM и MIPS (где у каждой инструкции длина или 2 или 4 байта), x86-инструкции имеют переменную длину,
так что, любой дизассемблер, начиная работу с середины x86-инструкции, может выдать неверные результаты.}
\EN{Unlike ARM and MIPS (where any instruction has a length of 2 or 4 bytes), x86 instructions have variable size,
so any disassembler that starts in the middle of a x86 instruction may produce incorrect results.}

\RU{Как пример}\EN{As an example}:

\lstinputlisting{\CURPATH/x86_wrong_start.asm.\LANG}

\RU{В начале мы видим неверно дизассемблированные инструкции, но потом, так или иначе, дизассемблер находит верный след}
\EN{There are incorrectly disassembled instructions at the beginning, but eventually the disassembler gets on the right 
track}.

\section{\RU{Как выглядят случайные данные в дизассемблированном виде}\EN{How does random noise looks disassembled}?}

\RU{Общее, что можно сразу заметить, это}\EN{Common properties that can be spotted easily are}:

\begin{itemize}
\item \RU{Необычно большой разброс инструкций}\EN{Unusually big instruction dispersion}.
\index{x86!\Instructions!PUSH}
\index{x86!\Instructions!MOV}
\index{x86!\Instructions!CALL}
\index{x86!\Instructions!IN}
\index{x86!\Instructions!OUT}
\RU{Самые частые x86-инструкции это}\EN{The most frequent x86 instructions are} \PUSH{}, \MOV{}, \CALL{}, 
\RU{но здесь мы видим
инструкции из любых групп: \ac{FPU}-инструкции, инструкции \TT{IN}/\TT{OUT}, редкие и системные инструкции, всё друг с другом смешано 
в одном месте.}
\EN{but here we see
instructions from all instruction groups: \ac{FPU} instructions, \TT{IN}/\TT{OUT} instructions, rare and system instructions,
everything mixed up in one single place}.

\item \RU{Большие и случайные значения, смещения,}\EN{Big and random values, offsets and} immediates.

\item \RU{Переходы с неверными смещениями часто имеют адрес перехода в середину другой инструкции}
\EN{Jumps having incorrect offsets, often jumping in the middle of another instructions}.
\end{itemize}

\lstinputlisting[caption=\randomNoise{} (x86)]{\CURPATH/x86.asm}

\index{x86-64}
\lstinputlisting[caption=\randomNoise{} (x86-64)]{\CURPATH/x64.asm}

\index{ARM}
\lstinputlisting[caption=\randomNoise{} (ARM (\ARMMode))]{\CURPATH/ARM.asm}

\lstinputlisting[caption=\randomNoise{} (ARM (\ThumbMode))]{\CURPATH/ARM_thumb.asm}

\index{MIPS}
\lstinputlisting[caption=\randomNoise{} (MIPS little endian)]{\CURPATH/MIPS.asm}

\RU{Также важно помнить, что хитрым образом написанный код для распаковки и дешифровки (включая самомодифицирующийся),
также может выглядеть как случайный шум, тем не менее, он исполняется корректно.}
\EN{It is also important to keep in mind that 
cleverly constructed unpacking and decryption code 
(including self-modifying) may looks like noise as well, but still execute correctly.}
% TODO таких примеров тоже бы добавить
