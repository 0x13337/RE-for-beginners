\chapter{\RU{Обфускация}\EN{Obfuscation}}

\RU{Обфускация это попытка спрятать код (или его значение) от reverse engineer-а}
\EN{The obfuscation is an attempt to hide the code (or its meaning) from reverse engineers}.

\section{\RU{Текстовые строки}\EN{Text strings}}

\RU{Как я указывал в}\EN{As I explained in} (\myref{sec:digging_strings}) \RU{текстовые строки могут быть крайне
полезны}\EN{, text strings may be really helpful}.
\RU{Знающие об этом программисты могут попытаться их спрятать так, чтобы их не было видно в \IDA{} или любом
шестнадцатеричном редакторе}
\EN{Programmers who are aware of this try to hide them, making it impossible to find the string in \IDA{} or any hex editor}.

\RU{Вот простейший метод}\EN{Here is the simplest method}.

\RU{Вот как строка может быть сконструирована}\EN{This is how the string can be constructed}:

\begin{lstlisting}
mov     byte ptr [ebx], 'h'
mov     byte ptr [ebx+1], 'e'
mov     byte ptr [ebx+2], 'l'
mov     byte ptr [ebx+3], 'l'
mov     byte ptr [ebx+4], 'o'
mov     byte ptr [ebx+5], ' '
mov     byte ptr [ebx+6], 'w'
mov     byte ptr [ebx+7], 'o'
mov     byte ptr [ebx+8], 'r'
mov     byte ptr [ebx+9], 'l'
mov     byte ptr [ebx+10], 'd'
\end{lstlisting}

\RU{Строка также может сравниваться с другой}\EN{The string is also can be compared with another one like this}:

\begin{lstlisting}
mov	ebx, offset username
cmp	byte ptr [ebx], 'j'
jnz	fail
cmp	byte ptr [ebx+1], 'o'
jnz	fail
cmp	byte ptr [ebx+2], 'h'
jnz	fail
cmp	byte ptr [ebx+3], 'n'
jnz	fail
jz	it_is_john
\end{lstlisting}

\RU{В обоих случаях, эти строки нельзя так просто найти в шестнадцатеричном редакторе.}
\EN{In both cases, it is impossible to find these strings straightforwardly in a hex editor.}

\index{Shellcode}
\RU{Кстати, точно также со строками можно работать в тех случаях, 
когда строку нельзя разместить в сегменте данных, например, в \ac{PIC}, или в шелл-коде.}
\EN{By the way, this is a way to work with the strings when it is impossible
to allocate space for them in the data segment, for example in a \ac{PIC} or in shellcode.}

\RU{Еще один виденный мною метод с использованием функции}\EN{Another method I once saw is to use} \TT{sprintf()} 
\RU{для конструирования}\EN{for the construction}:

\begin{lstlisting}
sprintf(buf, "%s%c%s%c%s", "hel",'l',"o w",'o',"rld");
\end{lstlisting}

\RU{Код выглядит ужасно, но как простейшая мера для анти-реверсинга, это может помочь.}
\EN{The code looks weird, but as a simple anti-reversing measure, it may be helpful.}

\RU{Текстовые строки могут также присутствовать в зашифрованном виде, в таком случае,
их использование будет предварять вызов функции для дешифровки.}
\EN{Text strings may also be present in encrypted form, 
then every string usage is to be preceded by a string decrypting routine.}
\RU{Например}\EN{For example}: \myref{examples_SCO}.

\section{\RU{Исполняемый код}\EN{Executable code}}

\subsection{\RU{Вставка мусора}\EN{Inserting garbage}}

\RU{Обфускация исполняемого кода\EMDASH{}это вставка случайного мусора (между настоящим кодом), который исполняется, но не делает
ничего полезного}\EN{Executable code obfuscation implies inserting random garbage code between real one,
which executes but does nothing useful}.

\RU{Просто пример}\EN{A simple example}:

\begin{lstlisting}[caption=\RU{оригинальный код}\EN{original code}]
add	eax, ebx
mul	ecx
\end{lstlisting}

\lstinputlisting[caption=obfuscated code]{\CURPATH/1.asm.\LANG}

\RU{Здесь код-мусор использует регистры, которые не используются в настоящем коде}
\EN{Here the garbage code uses registers which are not used in the real code} (\TT{ESI} \AndENRU \TT{EDX}).
\RU{Впрочем, промежуточные результаты полученные при исполнении настоящего кода вполне могут использоваться
кодом-мусором для б\'{о}льшей путаницы}\EN{However, the intermediate results produced by the real code 
may be used by the garbage instructions for some extra mess}\EMDASH{}\RU{почему нет}\EN{why not}?

\subsection{\RU{Замена инструкций на раздутые эквиваленты}\EN{Replacing instructions with bloated equivalents}}

\begin{itemize}
\item \TT{MOV op1, op2} \RU{может быть заменена на пару}\EN{can be replaced by the} \TT{PUSH op2 / POP op1}\EN{ pair}.
\item \TT{JMP label} \RU{может быть заменена на пару}\EN{can be replaced by the} \TT{PUSH label / RET}\EN{ pair}. 
\IDA{} \RU{не покажет ссылок на эту метку}\EN{will not show the references to the label}.
\item \TT{CALL label} \RU{может быть заменена на следующую тройку инструкций}\EN{can be replaced by the following instructions triplet}:
\TT{PUSH label\_after\_CALL\_instruction / PUSH label / RET}.
\item \TT{PUSH op} \RU{также можно заменить на пару инструкций}\EN{can also be replaced with the following instructions pair}: 
\TT{SUB ESP, 4 (\OrENRU 8) / MOV [ESP], op}.
\end{itemize}

\subsection{\RU{Всегда исполняющийся/никогда не исполняющийся код}\EN{Always executed/never executed code}}

\RU{Если разработчик уверен, что в}\EN{If the developer is sure that} ESI \RU{всегда будет 0 в этом месте}
\EN{at always 0 at that point}:

\lstinputlisting{\CURPATH/2.asm.\LANG}

\EN{The r}\RU{R}everse engineer\RU{-у понадобится какое-то время чтобы с этим разобраться}\EN{ needs some time to get into it}.

\index{opaque predicate}
\RU{Это также называется}\EN{This is also called an} \IT{opaque predicate}.

\RU{Еще один пример}\EN{Another example} (\RU{и снова разработчик уверен, что}
\EN{and again, the developer is sure that} ESI\RU{\EMDASH{}всегда ноль}\EN{ is always zero}):

\lstinputlisting{\CURPATH/3.asm.\LANG}

\subsection{\RU{Сделать побольше путаницы}\EN{Making a lot of mess}}

\begin{lstlisting}
instruction 1
instruction 2
instruction 3
\end{lstlisting}

\RU{Можно заменить на}\EN{Can be replaced with}:

\begin{lstlisting}
begin:		jmp	ins1_label

ins2_label:	instruction 2
		jmp	ins3_label

ins3_label:	instruction 3
		jmp	exit:

ins1_label:	instruction 1
		jmp	ins2_label
exit:
\end{lstlisting}

\subsection{\RU{Использование косвенных указателей}\EN{Using indirect pointers}}

\begin{lstlisting}
dummy_data1	db	100h dup (0)
message1	db	'hello world',0

dummy_data2	db	200h dup (0)
message2	db	'another message',0

func		proc
		...
		mov	eax, offset dummy_data1 ; PE or ELF reloc here
		add	eax, 100h
		push	eax
		call	dump_string
		...
		mov	eax, offset dummy_data2 ; PE or ELF reloc here
		add	eax, 200h
		push	eax
		call	dump_string
		...
func		endp
\end{lstlisting}

\IDA{} \RU{покажет ссылки на}\EN{will show references only to} \TT{dummy\_data1} \AndENRU \TT{dummy\_data2}, 
\RU{но не на сами текстовые строки}\EN{but not to the text strings}.

\RU{К глобальным переменным и даже функциям можно обращаться так же}
\EN{Global variables and even functions may be accessed like that}.

\section{\RU{Виртуальная машина / псевдо-код}\EN{Virtual machine / pseudo-code}}

\RU{Программист может также создать свой собственный}
\EN{A programmer can construct his/her own} \ac{PL} \OrENRU \ac{ISA} \RU{и интерпретатор для него}\EN{and interpreter for it}.
(\RU{Как версии Visual Basic перед 5.0}\EN{Like the pre-5.0 Visual Basic}, .NET or Java machines).
\EN{The reverse engineer will have to spend some time to understand the meaning 
and details of all of the \ac{ISA}'s instructions.}
\RU{Reverse engineer-у придется потратить какое-то время для понимания деталей всех инструкций 
в \ac{ISA}.}
\RU{Ему также возможно придется писать что-то вроде дизассемблера/декомпилятора}
\EN{Probably, he/she will also have to write a disassembler/decompiler of some sort}.

\section{\RU{Еще кое-что}\EN{Other things to mention}}

\RU{Моя попытка (хотя и слабая) пропатчить компилятор Tiny C чтобы он выдавал обфусцированный код}
\EN{My own (yet weak) attempt to patch the Tiny C compiler to produce obfuscated code}: \url{http://go.yurichev.com/17220}.

\RU{Использование инструкции}\EN{Using the} \MOV \RU{для сложных вещей}
\EN{instruction for really complicated things}: \cite{MOV_is_TM}.

\section{\Exercises}

\subsection{\Exercise \#1}
\label{exercise_obfuscation_1}

\RU{Это очень короткая программа, скомпилированная моим пропатченным компилятором Tiny C}
\EN{This is a very short program, compiled using the patched Tiny C compiler}
\footnote{\href{http://go.yurichev.com/17220}{blog.yurichev.com}}.
\RU{Попробуйте разобраться, что она делает}\EN{Try to find out what it does}.

\href{http://go.yurichev.com/17219}{beginners.re}.

\RU{Ответ}\EN{Answer}: \myref{exercise_solutions_obfuscation_1}.
