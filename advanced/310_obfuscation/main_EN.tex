\chapter{Obfuscation}

The obfuscation is an attempt to hide the code (or its meaning) from reverse engineers.

\section{Text strings}

As we saw in (\myref{sec:digging_strings}), text strings may be really helpful.

Programmers who are aware of this try to hide them, making it impossible to find the string in \IDA{} or any hex editor.

Here is the simplest method.

This is how the string can be constructed:

\begin{lstlisting}
mov     byte ptr [ebx], 'h'
mov     byte ptr [ebx+1], 'e'
mov     byte ptr [ebx+2], 'l'
mov     byte ptr [ebx+3], 'l'
mov     byte ptr [ebx+4], 'o'
mov     byte ptr [ebx+5], ' '
mov     byte ptr [ebx+6], 'w'
mov     byte ptr [ebx+7], 'o'
mov     byte ptr [ebx+8], 'r'
mov     byte ptr [ebx+9], 'l'
mov     byte ptr [ebx+10], 'd'
\end{lstlisting}

The string can also be compared with another one like this:

\begin{lstlisting}
mov	ebx, offset username
cmp	byte ptr [ebx], 'j'
jnz	fail
cmp	byte ptr [ebx+1], 'o'
jnz	fail
cmp	byte ptr [ebx+2], 'h'
jnz	fail
cmp	byte ptr [ebx+3], 'n'
jnz	fail
jz	it_is_john
\end{lstlisting}

In both cases, it is impossible to find these strings straightforwardly in a hex editor.

\myindex{Shellcode}

By the way, this is a way to work with the strings when it is impossible
to allocate space for them in the data segment, for example in a \ac{PIC} or in shellcode.

Another method is to use \TT{sprintf()} 
for the construction:

\begin{lstlisting}
sprintf(buf, "%s%c%s%c%s", "hel",'l',"o w",'o',"rld");
\end{lstlisting}

The code looks weird, but as a simple anti-reversing measure, it may be helpful.

Text strings may also be present in encrypted form, 
then every string usage is to be preceded by a string decrypting routine.
For example: \myref{examples_SCO}.

\section{Executable code}

\subsection{Inserting garbage}

Executable code obfuscation implies inserting random garbage code between real one,
which executes but does nothing useful.

A simple example:

\begin{lstlisting}[caption=original code]
add	eax, ebx
mul	ecx
\end{lstlisting}

\lstinputlisting[caption=obfuscated code]{\CURPATH/1_EN.asm}

Here the garbage code uses registers which are not used in the real code (\TT{ESI} and \TT{EDX}).
However, the intermediate results produced by the real code 
may be used by the garbage instructions for some extra mess\EMDASH{}why not?

\subsection{Replacing instructions with bloated equivalents}

\begin{itemize}
\item \TT{MOV op1, op2} can be replaced by the \TT{PUSH op2 / POP op1} pair.
\item \TT{JMP label} can be replaced by the \TT{PUSH label / RET} pair. 
\IDA{} will not show the references to the label.
\item \TT{CALL label} can be replaced by the following instructions triplet:
\TT{PUSH label\_after\_CALL\_instruction / PUSH label / RET}.
\item \TT{PUSH op} can also be replaced with the following instructions pair: \TT{SUB ESP, 4 (or 8) / MOV [ESP], op}.%

\end{itemize}

\subsection{Always executed/never executed code}

If the developer is sure that ESI 
at always 0 at that point:

\lstinputlisting{\CURPATH/2_EN.asm}

The reverse engineer needs some time to get into it.

\myindex{opaque predicate}
This is also called an \IT{opaque predicate}.

Another example (and again, the developer is sure that ESI is always zero):

\lstinputlisting{\CURPATH/3_EN.asm}

\subsection{Making a lot of mess}

\begin{lstlisting}
instruction 1
instruction 2
instruction 3
\end{lstlisting}

Can be replaced with:

\begin{lstlisting}
begin:		jmp	ins1_label

ins2_label:	instruction 2
		jmp	ins3_label

ins3_label:	instruction 3
		jmp	exit:

ins1_label:	instruction 1
		jmp	ins2_label
exit:
\end{lstlisting}

\subsection{Using indirect pointers}

\begin{lstlisting}
dummy_data1	db	100h dup (0)
message1	db	'hello world',0

dummy_data2	db	200h dup (0)
message2	db	'another message',0

func		proc
		...
		mov	eax, offset dummy_data1 ; PE or ELF reloc here
		add	eax, 100h
		push	eax
		call	dump_string
		...
		mov	eax, offset dummy_data2 ; PE or ELF reloc here
		add	eax, 200h
		push	eax
		call	dump_string
		...
func		endp
\end{lstlisting}

\IDA{} will show references only to \TT{dummy\_data1} and \TT{dummy\_data2}, 
but not to the text strings.

Global variables and even functions may be accessed like that.

\section{Virtual machine / pseudo-code}

A programmer can construct his/her own \ac{PL} or \ac{ISA} and interpreter for it.

(Like the pre-5.0 Visual Basic, .NET or Java machines).
The reverse engineer will have to spend some time to understand the meaning 
and details of all of the \ac{ISA}'s instructions.

Probably, he/she will also have to write a disassembler/decompiler of some sort.
% TODO about obfuscation VMs

\section{Other things to mention}

My own (yet weak) attempt to patch the Tiny C compiler to produce obfuscated code: \url{http://go.yurichev.com/17220}.

Using the \MOV 
instruction for really complicated things: 
[Stephen Dolan, \IT{mov is Turing-complete}, (2013)]
\footnote{\AlsoAvailableAs \url{http://www.cl.cam.ac.uk/~sd601/papers/mov.pdf}}. 

\section{\Exercise}

\begin{itemize}
	\item \url{http://challenges.re/29}
\end{itemize}

