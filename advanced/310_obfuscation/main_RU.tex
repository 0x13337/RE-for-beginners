\section{Обфускация}

Обфускация это попытка спрятать код (или его значение) от reverse engineer-а.

\subsection{Текстовые строки}

Как мы увидели в (\myref{sec:digging_strings}) текстовые строки могут быть крайне
полезны.
Знающие об этом программисты могут попытаться их спрятать так, чтобы их не было видно в \IDA{} или любом
шестнадцатеричном редакторе.

Вот простейший метод.

Вот как строка может быть сконструирована:

\begin{lstlisting}
mov     byte ptr [ebx], 'h'
mov     byte ptr [ebx+1], 'e'
mov     byte ptr [ebx+2], 'l'
mov     byte ptr [ebx+3], 'l'
mov     byte ptr [ebx+4], 'o'
mov     byte ptr [ebx+5], ' '
mov     byte ptr [ebx+6], 'w'
mov     byte ptr [ebx+7], 'o'
mov     byte ptr [ebx+8], 'r'
mov     byte ptr [ebx+9], 'l'
mov     byte ptr [ebx+10], 'd'
\end{lstlisting}

Строка также может сравниваться с другой:

\begin{lstlisting}
mov	ebx, offset username
cmp	byte ptr [ebx], 'j'
jnz	fail
cmp	byte ptr [ebx+1], 'o'
jnz	fail
cmp	byte ptr [ebx+2], 'h'
jnz	fail
cmp	byte ptr [ebx+3], 'n'
jnz	fail
jz	it_is_john
\end{lstlisting}

В обоих случаях, эти строки нельзя так просто найти в шестнадцатеричном редакторе.

\myindex{Shellcode}
Кстати, точно также со строками можно работать в тех случаях, 
когда строку нельзя разместить в сегменте данных, например, в \ac{PIC}, или в шелл-коде.

Еще метод с использованием функции \TT{sprintf()} для конструирования:

\begin{lstlisting}
sprintf(buf, "%s%c%s%c%s", "hel",'l',"o w",'o',"rld");
\end{lstlisting}

Код выглядит ужасно, но как простейшая мера для анти-реверсинга, это может помочь.

Текстовые строки могут также присутствовать в зашифрованном виде, в таком случае,
их использование будет предварять вызов функции для дешифровки.

Например: \myref{examples_SCO}.

\subsection{Исполняемый код}

\subsubsection{Вставка мусора}

Обфускация исполняемого кода\EMDASH{}это вставка случайного мусора (между настоящим кодом), который исполняется, но не делает
ничего полезного.

Просто пример:

\begin{lstlisting}[caption=оригинальный код]
add	eax, ebx
mul	ecx
\end{lstlisting}

\lstinputlisting[caption=obfuscated code]{\CURPATH/1_RU.asm}

Здесь код-мусор использует регистры, которые не используются в настоящем коде (\TT{ESI} и \TT{EDX}).
Впрочем, промежуточные результаты полученные при исполнении настоящего кода вполне могут использоваться
кодом-мусором для б\'{о}льшей путаницы\EMDASH{}почему нет?

\subsubsection{Замена инструкций на раздутые эквиваленты}

\begin{itemize}
\item \TT{MOV op1, op2} может быть заменена на пару \TT{PUSH op2 / POP op1}.
\item \TT{JMP label} может быть заменена на пару \TT{PUSH label / RET}. 
\IDA{} не покажет ссылок на эту метку.
\item \TT{CALL label} может быть заменена на следующую тройку инструкций:\\
\TT{PUSH label\_after\_CALL\_instruction / PUSH label / RET}.
\item \TT{PUSH op} также можно заменить на пару инструкций: \TT{SUB ESP, 4 (или 8) / MOV [ESP], op}.
\end{itemize}

\subsubsection{Всегда исполняющийся/никогда не исполняющийся код}

Если разработчик уверен, что в ESI всегда будет 0 в этом месте:

\lstinputlisting{\CURPATH/2_RU.asm}

Reverse engineer-у понадобится какое-то время чтобы с этим разобраться.

\myindex{opaque predicate}
Это также называется \IT{opaque predicate}.

Еще один пример (и снова разработчик уверен, что ESI\EMDASH{}всегда ноль):

\lstinputlisting{\CURPATH/3_RU.asm}

\subsubsection{Сделать побольше путаницы}

\begin{lstlisting}
instruction 1
instruction 2
instruction 3
\end{lstlisting}

Можно заменить на:

\begin{lstlisting}
begin:		jmp	ins1_label

ins2_label:	instruction 2
		jmp	ins3_label

ins3_label:	instruction 3
		jmp	exit:

ins1_label:	instruction 1
		jmp	ins2_label
exit:
\end{lstlisting}

\subsubsection{Использование косвенных указателей}

\begin{lstlisting}
dummy_data1	db	100h dup (0)
message1	db	'hello world',0

dummy_data2	db	200h dup (0)
message2	db	'another message',0

func		proc
		...
		mov	eax, offset dummy_data1 ; PE or ELF reloc here
		add	eax, 100h
		push	eax
		call	dump_string
		...
		mov	eax, offset dummy_data2 ; PE or ELF reloc here
		add	eax, 200h
		push	eax
		call	dump_string
		...
func		endp
\end{lstlisting}

\IDA{} покажет ссылки на \TT{dummy\_data1} и \TT{dummy\_data2}, но не на сами текстовые строки.

К глобальным переменным и даже функциям можно обращаться так же.

\subsection{Виртуальная машина / псевдо-код}

Программист может также создать свой собственный \ac{PL} или \ac{ISA} и интерпретатор для него.

(Как версии Visual Basic перед 5.0, .NET or Java machines).

Reverse engineer-у придется потратить какое-то время для понимания деталей всех инструкций в \ac{ISA}.
Ему также возможно придется писать что-то вроде дизассемблера/декомпилятора.
% TODO about obfuscation VMs

\subsection{Еще кое-что}

Моя попытка (хотя и слабая) пропатчить компилятор Tiny C чтобы он выдавал обфусцированный код:
\url{http://go.yurichev.com/17220}.

Использование инструкции \MOV для действительно сложных вещей: 
[Stephen Dolan, \IT{mov is Turing-complete}, (2013)]
\footnote{\AlsoAvailableAs \url{http://www.cl.cam.ac.uk/~sd601/papers/mov.pdf}}. 

\subsection{\Exercise}

\begin{itemize}
	\item \url{http://challenges.re/29}
\end{itemize}

