\subsection{std::list}
\index{\Cpp!STL!std::list}
\label{std_list}

\RU{Хорошо известный всем двусвязный список: каждый элемент имеет два указателя, на следующий и на предыдущий
элементы}\EN{This is the well-known doubly-linked list: each element has two pointers, to the previous and next
elements}.

\RU{Это означает что расход памяти увеличивается на 2 \glslink{word}{слова} на каждый элемент (8 байт в 32-битной среде или
16 байт в 64-битной)}
\EN{This implies that the memory footprint is enlarged by 2 \glslink{word}{words} for each element (8 bytes in 32-bit environment or 16
bytes in 64-bit)}.

\RU{STL в \Cpp просто добавляет указатели ``next'' и ``previous'' к той вашей структуре, которую вы желаете
объединить в список}
\EN{\Cpp STL just adds the ``next'' and ``previous'' pointers to the existing structure of the type that you want to 
unite in a list}.

\RU{Попробуем разобраться с примером в котором простая структура из двух переменных, мы объединим её в список}
\EN{Let's work out an example with a simple 2-variable structure that we want to store in a list}.

\RU{Хотя и стандарт \Cpp \cite{CPP11} не указывает, как он должен быть реализован, реализации
MSVC и GCC простые и похожи друг на друга, так что этот исходный код для обоих:}
\EN{Although the \Cpp standard \cite{CPP11} does not say how to implement it, 
both MSVC's and GCC's implementations are straightforward and similar, so here is only one
source code for both:}

\lstinputlisting{\CURPATH/STL/list/2.cpp.\LANG}

\subsubsection{GCC}

\RU{Начнем с}\EN{Let's start with} GCC.

\RU{При запуске увидим длинный вывод, будем разбирать его по частям}
\EN{When we run the example, we'll see a long dump, let's work with it in pieces}.

\begin{lstlisting}
* empty list:
ptr=0x0028fe90 _Next=0x0028fe90 _Prev=0x0028fe90 x=3 y=0
\end{lstlisting}

\RU{Видим пустой список}\EN{Here we see an empty list}.
\RU{Не смотря на то что он пуст, имеется один элемент с мусором (\ac{AKA} узел-пустышка (\IT{dummy node})) 
в переменных $x$ и $y$.}
\EN{Despite the fact it is empty, it has one element with garbage (\ac{AKA} \IT{dummy node})
in $x$ and $y$.}
\RU{Оба указателя}\EN{Both the} ``next'' \AndENRU ``prev'' \RU{указывают на себя}
\EN{pointers are pointing to the self node}:

\begin{center}
\begin{tikzpicture}[thick,scale=0.85, every node/.style={scale=0.85}]
	\tikzstyle{every path}=[thick]
	\tikzstyle{node1}=[draw,rectangle,minimum height=1cm, minimum width=2.5cm, text width=2.5cm, fill=gray!20]

	\node [node1] (n1next) {Next};
	\node [node1] (n1prev) [below of=n1next] {Prev};
	\node [node1] (n1x) [below of=n1prev] {X=\garbage{}};
	\node [node1] (n1y) [below of=n1x] {Y=\garbage{}};

	\node [node1] (variable) [above=1.5cm of n1next] {\RU{Переменная}\EN{Variable} std::list};
	
	\node [node1] (itbegin) [above left=1.5cm and 0.2cm of n1next] {list.begin()};
	\node [node1] (itend) [above right=1.5cm and 0.2cm of n1next] {list.end()};
	
	\draw [->] (itbegin.south) -- (n1next.north);
	
	\draw [->] (variable.south) -- (n1next.north);
	
	\draw [->] (itend.south) -- (n1next.north);
	
	\node (ia1) [inner sep=0pt, above left=0.5cm and 0.5cm of n1next] {};
	\node (ia2) [inner sep=0pt, left=0.5cm of n1next] {};
	\node (ib1) [inner sep=0pt, above right=0.5cm and 0.5cm of n1next] {};
	\node (ib2) [inner sep=0pt, right=0.5cm of n1next] {};
	
	\node (oa1) [inner sep=0pt, above left=1cm and 1cm of n1next] {};
	\node (oa2) [inner sep=0pt, left=1cm of n1prev] {};
	\node (ob1) [inner sep=0pt, above right=1cm and 1cm of n1next] {};
	\node (ob2) [inner sep=0pt, right=1cm of n1prev] {};

	\draw [->] (n1next.east) -- (ib2) -- (ib1) -- (ia1) -- (ia2) -- (n1next.west);
	\draw [->] (n1prev.west) -- (oa2) -- (oa1) -- (ob1) -- (ob2) -- (n1prev.east);
\end{tikzpicture}
\end{center}


\RU{Это тот момент, когда итераторы .begin и .end равны друг другу}
\EN{At this moment, the .begin and .end iterators are equal to each other}.

\RU{Вставим 3 элемента и список в памяти будет представлен так:}
\EN{If we push 3 elements, the list internally will be:}

\begin{lstlisting}
* 3-elements list:
ptr=0x000349a0 _Next=0x00034988 _Prev=0x0028fe90 x=3 y=4
ptr=0x00034988 _Next=0x00034b40 _Prev=0x000349a0 x=1 y=2
ptr=0x00034b40 _Next=0x0028fe90 _Prev=0x00034988 x=5 y=6
ptr=0x0028fe90 _Next=0x000349a0 _Prev=0x00034b40 x=5 y=6
\end{lstlisting}

\RU{Последний элемент всё еще на 0x0028fe90, 
он не будет передвинут куда-либо до самого уничтожения списка.}
\EN{The last element is still at 0x0028fe90,
it not to be moved until the list's disposal.}
\RU{Он все еще содержит случайный мусор в полях $x$ и $y$ (5 и 6)}
\EN{It still contain random garbage in $x$ and $y$ (5 and 6)}. 
\RU{Случайно совпало так, что эти значения точно такие же, как и в последнем элементе, но это не значит,
что они имеют какое-то значение}\EN{By coincidence, these values are the same as in the last element, but it doesn't mean that they are meaningful}.

\RU{Вот как эти 3 элемента хранятся в памяти}\EN{Here is how these 3 elements are stored in memory}:

\begin{center}
\begin{tikzpicture}[thick,scale=0.85, every node/.style={scale=0.85}]
	\tikzstyle{every path}=[thick]
	\tikzstyle{node1}=[draw,rectangle,minimum height=1cm, minimum width=2.5cm, text width=2.5cm, fill=gray!20]

	\node [node1] (n1next) {Next};
	\node [node1] (n1prev) [below of=n1next] {Prev};
	\node [node1] (n1x) [below of=n1prev] {X=\RU{1-й элемент}\EN{1st element}};
	\node [node1] (n1y) [below of=n1x] {Y=\RU{1-й элемент}\EN{1st element}};

	\node [node1] (n2next) [right=0.5cm of n1next] {Next};
	\node [node1] (n2prev) [below of=n2next] {Prev};
	\node [node1] (n2x) [below of=n2prev] {X=\RU{2-й элемент}\EN{2nd element}};
	\node [node1] (n2y) [below of=n2x] {Y=\RU{2-й элемент}\EN{2nd element}};
	
	\node [node1] (n3next) [right=0.5cm of n2next] {Next};
	\node [node1] (n3prev) [below of=n3next] {Prev};
	\node [node1] (n3x) [below of=n3prev] {X=\RU{3-й элемент}\EN{3rd element}};
	\node [node1] (n3y) [below of=n3x] {Y=\RU{3-й элемент}\EN{3rd element}};
	
	\node [node1] (n4next) [right=0.5cm of n3next] {Next};
	\node [node1] (n4prev) [below of=n4next] {Prev};
	\node [node1] (n4x) [below of=n4prev] {X=\RU{мусор}\EN{garbage}};
	\node [node1] (n4y) [below of=n4x] {Y=\RU{мусор}\EN{garbage}};
	
	\node [node1] (variable) [above=3cm of n1next] {\RU{Переменная}\EN{Variable} std::list};
	
	\node [node1] (itbegin) [above=2cm of n1next] {list.begin()};
	\draw [->] (itbegin.south) -- (n1next.north);
	
	\draw [->] (variable.south) -- (itbegin.north);

	\node [node1] (itend) [above=2cm of n4next] {list.end()};
	\draw [->] (itend.south) -- (n4next.north);
	
	\node (ia1) [inner sep=0pt, above left=0.5cm and 0.5cm of n1next] {};
	\node (ia2) [inner sep=0pt, left=0.5cm of n1next] {};
	\node (ib1) [inner sep=0pt, above right=0.5cm and 0.5cm of n4next] {};
	\node (ib2) [inner sep=0pt, right=0.5cm of n4next] {};
	
	\node (oa1) [inner sep=0pt, above left=1cm and 1cm of n1next] {};
	\node (oa2) [inner sep=0pt, left=1cm of n1prev] {};
	\node (ob1) [inner sep=0pt, above right=1cm and 1cm of n4next] {};
	\node (ob2) [inner sep=0pt, right=1cm of n4prev] {};

	\draw [->] (n1next.east) -- (n2next.west);
	\draw [->] (n2next.east) -- (n3next.west);
	\draw [->] (n3next.east) -- (n4next.west);

	\draw [<-] (n1prev.east) -- (n2prev.west);
	\draw [<-] (n2prev.east) -- (n3prev.west);
	\draw [<-] (n3prev.east) -- (n4prev.west);
	
	\draw [->] (n4next.east) -- (ib2) -- (ib1) -- (ia1) -- (ia2) -- (n1next.west);
	\draw [->] (n1prev.west) -- (oa2) -- (oa1) -- (ob1) -- (ob2) -- (n4prev.east);
\end{tikzpicture}
\end{center}


\RU{Переменная}\EN{The} $l$ \RU{всегда указывает на первый элемент}\EN{variable always points to the first node}.

\RU{Итераторы .begin() и .end() это не переменные, а функции,
возвращающие указатели на соответствующие узлы.}
\EN{The .begin() and .end() iterators are not variables, but functions, 
which when called return pointers to the corresponding nodes.}

\RU{Иметь элемент-пустышку (\IT{dummy node} \OrENRU \IT{sentinel node}) 
это очень популярная практика в реализации двусвязных списков.}
\EN{Having a dummy element (\ac{AKA} \IT{sentinel node}) 
is a very popular practice in implementing doubly-linked lists.}
\RU{Без него, многие операции были бы сложнее, и, следовательно, медленнее}
\EN{Without it, a lot of operations may become slightly more complex and, hence, slower}.

\RU{Итератор на самом деле это просто указатель на элемент}\EN{The iterator is in fact just a pointer to a node}.
list.begin() \AndENRU list.end() \RU{просто возвращают указатели}\EN{just return pointers}.

\begin{lstlisting}
node at .begin:
ptr=0x000349a0 _Next=0x00034988 _Prev=0x0028fe90 x=3 y=4
node at .end:
ptr=0x0028fe90 _Next=0x000349a0 _Prev=0x00034b40 x=5 y=6
\end{lstlisting}

\RU{Тот факт, что что последний элемент имеет указатель на первый 
и первый имеет указатель на последний, напоминает нам циркулярный список.}
\EN{The fact that the last element has a pointer to the first and the first element has a pointer 
to the last one remind us circular list.}

\RU{Это очень помогает: если иметь указатель только на первый элемент, т.е.,
тот что в переменной $l$, очень легко получить указатель на последний элемент, без необходимости
обходить все элементы списка}
\EN{This is very helpful here: having a pointer to the first list element,
i.e., that is in the $l$ variable,
it is easy to get a pointer to the last one quickly, without the need to traverse the whole list}.
\RU{Вставка элемента в конец списка также быстра благодаря этой особенности}
\EN{Inserting an element at the list end is also quick, thanks to this feature}.

\TT{operator--} \AndENRU \TT{operator++} \RU{просто выставляют текущее значение итератора на}
\EN{just set the current iterator's value to the} \TT{current\_node->prev}
\OrENRU \\
\TT{current\_node->next}\EN{ values}.
\RU{Обратные итераторы}\EN{The reverse iterators} (.rbegin, .rend) \RU{работают точно так же,
только наоборот}
\EN{work just as the same, but in reverse}.

\TT{operator*} \RU{на итераторе просто возвращает указатель на место в структуре, где начинается пользовательская
структура, т.е., указатель на самый первый элемент структуры}\EN{just returns a pointer to the point 
in the node structure, where the user's structure starts, i.e., a pointer to the 
first element of the structure} ($x$).

\RU{Вставка в список и удаление очень просты: просто выделите новый элемент (или освободите) и исправьте
все указатели так, чтобы они были верны}
\EN{The list insertion and deletion are trivial: just allocate a new node (or deallocate) and update all pointers
to be valid}.

\RU{Вот почему итератор может стать недействительным после удаления элемента:
он может всё еще указывать на уже освобожденный элемент.}
\EN{That's why an iterator may become invalid after element deletion: 
it may still point to the node that was already deallocated.}
\RU{Это также называется}\EN{This is also called a} \IT{dangling pointer}.
\RU{И конечно же, информация из освобожденного элемента, на который указывает итератор, 
не может использоваться более.}
\EN{And of course, the information from the freed node (to which iterator still points) 
cannot be used anymore.}

\RU{В реализации GCC (по крайней мере 4.8.1) не сохраняется текущая длина списка: это выливается в медленный
метод .size(): он должен пройти по всему списку считая элементы, просто потому что нет
другого способа получить эту информацию}
\EN{The GCC implementation (as of 4.8.1) doesn't store the current size of the list: this implies a slow .size() method:
it has to traverse the whole list to count the elements, because it doesn't have any other way to get the information}.
\RU{Это означает что эта операция $O(n)$, т.е., она работает тем медленнее, чем больше элементов в списке}
\EN{This mean that this operation is $O(n)$, i.e., it gets slower steadily as the list grows}.

\lstinputlisting[caption=\Optimizing GCC 4.8.1 -fno-inline-small-functions]{\CURPATH/STL/list/GCC.asm.\LANG}

\lstinputlisting[caption=\RU{Весь вывод}\EN{The whole output}]{\CURPATH/STL/list/GCC.txt}

\subsubsection{MSVC}
\label{MSVC_std_list}

\RU{Реализация MSVC (2012) точно такая же, только еще и сохраняет текущий размер списка}
\EN{MSVC's implementation (2012) is just the same, but it also stores the current size of the list}.
\RU{Это означает что метод .size() очень быстр ($O(1)$): просто прочитать одно значение из памяти}
\EN{This implies that the .size() method is very fast ($O(1)$): it just reads one value from memory}.
\RU{С другой стороны, переменная хранящая размер должна корректироваться при каждой вставке/удалении}
\EN{On the other hand, the size variable must be updated at each insertion/deletion}.

\RU{Реализация MSVC также немного отлична в смысле расстановки элементов:}
\EN{MSVC's implementation is also slightly different in the way it arranges the nodes:}

\begin{center}
\begin{tikzpicture}[thick,scale=0.85, every node/.style={scale=0.85}]
	\tikzstyle{every path}=[thick]
	\tikzstyle{node1}=[draw,rectangle,minimum height=1cm, minimum width=2.5cm, text width=2.5cm, fill=gray!20]

	\node [node1] (n1next) {Next};
	\node [node1] (n1prev) [below of=n1next] {Prev};
	\node [node1] (n1x) [below of=n1prev] {X=\IFRU{мусор}{garbage}};
	\node [node1] (n1y) [below of=n1x] {Y=\IFRU{мусор}{garbage}};

	\node [node1] (n2next) [right=0.5cm of n1next] {Next};
	\node [node1] (n2prev) [below of=n2next] {Prev};
	\node [node1] (n2x) [below of=n2prev] {X=\IFRU{1-й элемент}{1st element}};
	\node [node1] (n2y) [below of=n2x] {Y=\IFRU{1-й элемент}{1st element}};
	
	\node [node1] (n3next) [right=0.5cm of n2next] {Next};
	\node [node1] (n3prev) [below of=n3next] {Prev};
	\node [node1] (n3x) [below of=n3prev] {X=\IFRU{2-й элемент}{2nd element}};
	\node [node1] (n3y) [below of=n3x] {Y=\IFRU{2-й элемент}{2nd element}};
	
	\node [node1] (n4next) [right=0.5cm of n3next] {Next};
	\node [node1] (n4prev) [below of=n4next] {Prev};
	\node [node1] (n4x) [below of=n4prev] {X=\IFRU{3-й элемент}{3rd element}};
	\node [node1] (n4y) [below of=n4x] {Y=\IFRU{3-й элемент}{3rd element}};
	
	\node [node1] (variable) [above=3cm of n1next] {\IFRU{Переменная}{Variable} std::list};

	\node [node1] (itend) [above=2cm of n1next] {list.end()};
	\draw [->] (itend.south) -- (n1next.north);
	
	\draw [->] (variable.south) -- (itend.north);
	
	\node [node1] (itbegin) [above=2cm of n2next] {list.begin()};
	\draw [->] (itbegin.south) -- (n2next.north);

	\node (ia1) [inner sep=0pt, above left=0.5cm and 0.5cm of n1next] {};
	\node (ia2) [inner sep=0pt, left=0.5cm of n1next] {};
	\node (ib1) [inner sep=0pt, above right=0.5cm and 0.5cm of n4next] {};
	\node (ib2) [inner sep=0pt, right=0.5cm of n4next] {};
	
	\node (oa1) [inner sep=0pt, above left=1cm and 1cm of n1next] {};
	\node (oa2) [inner sep=0pt, left=1cm of n1prev] {};
	\node (ob1) [inner sep=0pt, above right=1cm and 1cm of n4next] {};
	\node (ob2) [inner sep=0pt, right=1cm of n4prev] {};

	\draw [->] (n1next.east) -- (n2next.west);
	\draw [->] (n2next.east) -- (n3next.west);
	\draw [->] (n3next.east) -- (n4next.west);

	\draw [<-] (n1prev.east) -- (n2prev.west);
	\draw [<-] (n2prev.east) -- (n3prev.west);
	\draw [<-] (n3prev.east) -- (n4prev.west);
	
	\draw [->] (n4next.east) -- (ib2) -- (ib1) -- (ia1) -- (ia2) -- (n1next.west);
	\draw [->] (n1prev.west) -- (oa2) -- (oa1) -- (ob1) -- (ob2) -- (n4prev.east);
\end{tikzpicture}
\end{center}


\RU{У GCC его элемент-пустышка в самом конце списка, а у MSVC в самом начале.}
\EN{GCC has its dummy element at the end of the list, while MSVC's is at the beginning.}

\lstinputlisting[caption=\Optimizing MSVC 2012 /Fa2.asm /GS- /Ob1]{\CURPATH/STL/list/MSVC.asm.\LANG}

\RU{В отличие от GCC, код MSVC выделяет элемент-пустышку в самом начале функции при помощи
функции ``Buynode'', она также используется и во время выделения остальных элементов}
\EN{Unlike GCC, MSVC's code allocates the dummy element at the start of the function with the help of the ``Buynode'' function,
it is also used to allocate the rest of the nodes} (\RU{код GCC выделяет самый первый элемент в локальном стеке}
\EN{GCC's code allocates the first element in the local stack}).

\lstinputlisting[caption=\RU{Весь вывод}\EN{The whole output}]{\CURPATH/STL/list/MSVC.txt}

\subsubsection{C++11 std::forward\_list}
\index{\Cpp!STL!std::forward\_list}

\RU{Это то же самое что и std::list, но только односвязный список, т.е., имеющий только поле ``next'' в каждом
элементе}
\EN{The same thing as std::list, but singly-linked one, i.e., having only the ``next'' field at each node}.
\RU{Таким образом расход памяти меньше, но возможности идти по списку назад здесь нет}
\EN{It has a smaller memory footprint, but also don't offer the ability to traverse list backwards}.

