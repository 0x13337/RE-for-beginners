\subsection{std::map and std::set}
\myindex{\Cpp!STL!std::map}
\myindex{\Cpp!STL!std::set}
\myindex{Binary tree}

The binary tree is another fundamental data structure.

As its name states, this is a tree where each node has at most 2 links to other nodes.
Each node has key and/or value.

Binary trees are usually the structure used in the implementation of \q{dictionaries} of key-values (\ac{AKA} \q{associative arrays}).

There are at least three important properties that a binary trees has:
\begin{itemize}
\item All keys are always stored in sorted form.
\item Keys of any types can be stored easily.
Binary tree algorithms are unaware of the key's type, 
only a key comparison function is required.
\item 
Finding a specific key is relatively fast in comparison with lists and arrays.
\end{itemize}

Here is a very simple example: let's store these numbers in a binary tree:
0, 1, 2, 3, 5, 6, 9, 10, 11, 12, 20, 99, 100, 101, 107, 1001, 1010.

\tikzset
{
  treenode/.style = {align=center, inner sep=0pt, text centered},
  key/.style = {treenode, circle, white, draw=black, fill=black, minimum size=2.0em, text width=2.0em},
  empty/.style = {treenode, rectangle, draw=black, minimum width=0.5em, minimum height=0.5em}
}

\begin{center}
\begin{tikzpicture}[->,>=stealth',
		level 1/.style={sibling distance=6cm},
		level 2/.style={sibling distance=2.7cm},
		level 3/.style={sibling distance=1.7cm},
		level 4/.style={sibling distance=0.7cm}] 
	\node [key] {10}
	child { node [key] {1} 
		child{ node [key] {0} 
		}
		child{ node [key] {5}
			child{ node [key] {3}
				child{ node[key] {2}}
				child{ node[empty] {}}
			}
			child{ node [key] {6}
				child{ node[empty] {}}
				child{ node[key] {9}}
			}
		}                            
	}
	child { node [key] {100}
		child { node [key] {20}
			child{ node [key] {12}
				child{ node[key] {11}}
				child{ node[empty] {}}
			}
			child{ node [key] {99}
			}
		}
		child{ node [key] {107}
			child{ node [key] {101}}
			child{ node [key] {1001}
				child{ node[empty] {}}
				child{ node[key] {1010}}
			}
		}
	}
	; 
\end{tikzpicture}
\end{center}


All keys that are smaller than the node key's value are stored on the left side.

All keys that are bigger than the node key's value are stored on the right side.

Hence, the lookup algorithm is straightforward: if the value that you are looking for is smaller than the current node's key value:
move left, if it is bigger: move right, stop if the value required is equal to the node key's value.

That is why the searching algorithm may search for numbers, text strings, etc, as long as 
a key comparison function is provided.

All keys have unique values.

Having that, one needs $\approx \log_{2} n$ steps in order to find a key in a balanced binary tree with $n$ keys.
This implies that $\approx 10$ steps are needed $\approx 1000$ keys, or $\approx 13$ 
steps for $\approx 10000$ keys.

Not bad, but the tree has always to be balanced for this: i.e., the keys has to be distributed evenly on all levels.
The insertion and removal operations do some maintenance to keep the tree in a balanced state.

There are several popular balancing algorithms available, including the AVL tree and the red-black tree.

The latter extends each node with a \q{color} value to simplify the balancing process, hence, 
each node may be \q{red} or \q{black}.

Both GCC's and MSVC's \TT{std::map} and \TT{std::set} template implementations use red-black trees.

\TT{std::set} contains only keys.
\TT{std::map} is the \q{extended} version of std::set: it also has a value at each node.

\subsubsection{MSVC}

\lstinputlisting{\CURPATH/STL/map_set/MSVC_EN.cpp}

\lstinputlisting[caption=MSVC 2012]{\CURPATH/STL/map_set/MSVC.txt}

The structure is not packed, so both \Tchar values occupy 4 bytes each.

As for \TT{std::map}, \TT{first} and \TT{second} can be viewed as a single value of type \TT{std::pair}.
\TT{std::set} 
has only one value at this address in the structure instead.

The current size of the tree is always present, as in the case of the implementation of \TT{std::list} in MSVC (\myref{MSVC_std_list}).

As in the case of \TT{std::list}, 
the iterators are just pointers to nodes.
The \TT{.begin()} iterator points to the minimal key.

That pointer is not stored anywhere (as in lists), the minimal key of the tree is looked up every time.
\TT{operator--} and \TT{operator++} 
move the current node pointer to the predecessor or successor respectively, i.e., the nodes which have the previous or next key.

The algorithms for all these operations are explained in
[Cormen, Thomas H. and Leiserson, Charles E. and Rivest, Ronald L. and Stein, Clifford,
\IT{Introduction to Algorithms, Third Edition}, (2009)].

The \TT{.end()} 
iterator points to the dummy node, it has 1 in \TT{Isnil}, which implies that 
the node has no key and/or value.
It can be viewed as a \q{landing zone} in \ac{HDD}.

The \q{parent} field of the dummy node points to the root node, which serves
as a vertex of the tree and contains information.

\subsubsection{GCC}

\lstinputlisting{\CURPATH/STL/map_set/GCC.cpp}

\lstinputlisting[caption=GCC 4.8.1]{\CURPATH/STL/map_set/GCC.txt}

GCC's implementation is very similar
\footnote{\url{http://go.yurichev.com/17084}}.
The only difference is the absence of the \TT{Isnil} field,
so the structure occupies slightly less space in memory than its implementation in MSVC.

The dummy node is also used as a place to point the \TT{.end()} iterator also has no key and/or value.

\subsubsection{Rebalancing demo (GCC)}

Here is also a demo showing us how a tree is rebalanced after some insertions.

\lstinputlisting[caption=GCC]{\CURPATH/STL/map_set/GCC_rebalancing_demo.cpp}

\lstinputlisting[caption=GCC 4.8.1]{\CURPATH/STL/map_set/GCC_rebalancing_demo.txt}

