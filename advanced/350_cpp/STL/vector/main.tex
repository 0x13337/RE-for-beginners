\subsection{std::vector}
\index{\Cpp!STL!std::vector}

\RU{Я бы назвал}\EN{I would call} \TT{std::vector} \EN{a }``\RU{безопасной оболочкой (wrapper)}\EN{safe wrapper}'' \EN{of the}\ac{PODT} \RU{массива в Си}\EN{C array}.
\RU{Изнутри он очень похож на}\EN{Internally it is somewhat similar to} \TT{std::string} (\myref{std_string}):
\RU{он имеет указатель на выделенный буфер, указатель на конец массива и указатель на конец выделенного буфера.}
\EN{it has a pointer to the allocated buffer, a pointer to the end of the array, and a pointer to the end of the allocated buffer.}

\RU{Элементы массива просто лежат в памяти впритык друг к другу, так же, как и в обычном массиве}
\EN{The array's elements  lie in memory adjacently to each other, just like in a normal array} (\myref{arrays}).
\index{\Cpp!C++11}
\RU{В}\EN{In} C++11 \RU{появился метод}\EN{there is a new method called} \TT{.data()} \RU{возвращающий указатель на этот буфер, это похоже на}\EN{, that returns a pointer to the buffer, like} \TT{.c\_str()} \InENRU \TT{std::string}.

\RU{Выделенный буфер в \glslink{heap}{куче} может быть больше чем сам массив.}
\EN{The buffer allocated in the \gls{heap} can be larger than the array itself.}

\RU{Реализации MSVC и GCC почти одинаковые, отличаются только имена полей в структуре}
\EN{Both MSVC's and GCC's implementations are similar, just the names of the structure's fields are slightly different}\footnote
{\RU{внутренности GCC}\EN{GCC internals}: \url{http://go.yurichev.com/17086}}, \RU{так что здесь один исходник работающий для обоих компиляторов}\EN{so here is one source
code that works for both compilers}.
\RU{И снова здесь Си-подобный код для вывода структуры}\EN{Here is again the 
C-like code for dumping the structure of} \TT{std::vector}:

\lstinputlisting{\CURPATH/STL/vector/2.cpp.\LANG}

\RU{Примерный вывод программы скомпилированной в}\EN{Here is the output of this program when compiled in} MSVC:

\lstinputlisting{\CURPATH/STL/vector/MSVC.txt}

\RU{Как можно заметить, выделенного буфера в самом начале функции \main пока нет}
\EN{As it can be seen, there is no allocated buffer when \main starts}.
\RU{После первого вызова}\EN{After the first} \TT{push\_back()} \RU{буфер выделяется}\EN{call, a buffer is allocated}.
\RU{И далее, после каждого вызова}\EN{And then, after each} \TT{push\_back()} 
\RU{и длина массива и вместимость буфера (\IT{capacity}) увеличиваются}
\EN{call, both array size and buffer size (\IT{capacity}) are increased}.
\RU{Но адрес буфера также меняется, потому что вызов функции}\EN{But the buffer address changes as well, because} \TT{push\_back()} \RU{перевыделяет буфер в \glslink{heap}{куче} каждый раз}
\EN{reallocates the buffer in the \gls{heap} each time}.
\RU{Это дорогая операция, вот почему очень важно предсказать размер будущего массива и зарезервировать место для него
при помощи метода}\EN{It is costly operation, that's why it is very important to predict the size of the array in the future and reserve 
enough space for it with the} \TT{.reserve()}\EN{ method}.
\RU{Самое последнее число\EMDASH{}это мусор: там нет элементов массива в этом месте, вот откуда это случайное число}
\EN{The last number is garbage: there are no array elements at this point, so a random number is printed}.
\RU{Это иллюстрация того факта что метод}\EN{This illustrates the fact that} \TT{operator[]} \RU{в}\EN{of} 
\TT{std::vector} \RU{не проверяет индекс на правильность}\EN{does not check of the index is in the array's bounds}.
\RU{Более медленный метод}\EN{The slower} \TT{.at()} \RU{с другой стороны, проверяет, и подкидывает исключение}
\EN{method, however, does this checking and throws an} \TT{std::out\_of\_range} 
\RU{в случае ошибки}\EN{exception in case of error}.

\RU{Давайте посмотрим код}\EN{Let's see the code}:

\lstinputlisting[caption=MSVC 2012 /GS- /Ob1]{\CURPATH/STL/vector/MSVC.asm}

\RU{Мы видим, как метод}\EN{We see how the} \TT{.at()} \RU{проверяет границы и подкидывает исключение в случае ошибки}
\EN{method checks the bounds and throws an exception in case of error}.
\RU{Число, которое выводит последний вызов}\EN{The number that the last} \printf \RU{берется из памяти, без всяких
проверок}\EN{call prints is just taken from the memory, without any checks}.

\RU{Читатель может спросить, почему бы не использовать переменные}
\EN{One may ask, why not use the variables like} ``size'' \AndENRU ``capacity'', 
\RU{как это сделано в}\EN{like it was done in} \TT{std::string}.
\RU{Я подозреваю что это для более быстрой проверки границ}\EN{I suppose that this was done for faster bounds checking}.
\RU{Но я не уверен}\EN{But I'm not sure}.

\index{Inline code}
\RU{Код генерируемый GCC почти такой же, в целом, но метод}
\EN{The code GCC generates is in general almost the same, but the} \TT{.at()} \RU{вставлен прямо в код}\EN{method is inlined}:

\lstinputlisting[caption=GCC 4.8.1 -fno-inline-small-functions -O1]{\CURPATH/STL/vector/GCC.asm}

\RU{Метод }\TT{.reserve()} \RU{точно так же вставлен прямо в код \main}\EN{is inlined as well}.
\RU{Он вызывает}\EN{It calls} \TT{new()} \RU{если буфер слишком мал для нового массива}
\EN{if the buffer is too small for the new size}, \RU{вызывает}\EN{calls} \TT{memmove()} 
\RU{для копирования содержимого буфера}\EN{to copy the contents of the buffer},
\RU{и вызывает}\EN{and calls} \TT{delete()} \RU{для освобождения старого буфера}\EN{to free the old buffer}.

\RU{Посмотрим, что выводит программа будучи скомпилированная GCC}
\EN{Let's also see what the compiled program outputs if compiled with GCC}:

\lstinputlisting{\CURPATH/STL/vector/GCC.txt}

\RU{Мы можем заметить, что буфер растет иначе чем в}\EN{We can spot that the buffer size grows in a different way that in} MSVC.

\RU{При помощи простых экспериментов становится ясно, что в реализации MSVC буфер увеличивается
на}\EN{Simple experimentation shows that in MSVC's implementation the buffer grows by} \textasciitilde{}50\% \RU{каждый раз,
когда он должен был увеличен}\EN{each time it needs to be enlarged},
\RU{а у GCC он увеличивается на}\EN{while GCC's code enlarges it by} 100\% \RU{каждый раз, т.е., удваивается}
\EN{each time, i.e., doubles it}.

