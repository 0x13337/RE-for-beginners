\subsectionold{A simple example}

Internally, the representation of \Cpp classes is almost the same as the structures.

Let's try an example with two variables, two constructors and one method:

\lstinputlisting{\CURPATH/classes/simple_EN.cpp}

\subsubsectionold{MSVC\EMDASH{}x86}

Here is how the \main function looks like, translated 
into assembly language:

\lstinputlisting[caption=MSVC]{\CURPATH/classes/simple_1_msvc.asm}

Here's what's going on.
For each object (instance of class $c$) 8 bytes are allocated,
exactly the size needed to store the 2 variables.

For \IT{c1} a default argumentless constructor \TT{??0c@@QAE@XZ} is called.
For \IT{c2} another constructor \TT{??0c@@QAE@HH@Z} is called and two numbers are passed as arguments.

\myindex{thiscall}
\label{thiscall}
\myindex{x86!\Registers!ECX}

A pointer to the object (\ITthis in \Cpp terminology) is passed in the \ECX register.
This is called thiscall~(\myref{thiscall})\EMDASH{}the method for passing a pointer to the object.

MSVC does it using the \ECX register. Needless to say, it is not a standardized method, other compilers can do it
differently, e.g., via the first function argument (like GCC).

\label{namemangling}
\myindex{Name mangling}
\myindex{OOP!Polymorphism}

Why do these functions have such odd names?
That's \gls{name mangling}.

A \Cpp class may contain several methods sharing the same name but having different 
arguments\EMDASH{}that is polymorphism.
And of course, different classes may have their own methods with the same name.

\myindex{Linker}

\IT{Name mangling} enable us to encode the class name + method name + all method argument types
in one ASCII string, which is then used as an internal function name.
That's all because neither the linker, nor the DLL \ac{OS} loader (mangled names may be among 
the DLL exports as well) knows anything about \Cpp or \ac{OOP}.

The \TT{dump()} function is called two times.

Now let's see the constructors' code:

\lstinputlisting[caption=MSVC]{\CURPATH/classes/simple_2_msvc.asm}

The constructors are just functions, they use a pointer to the structure in \ECX,
copying the pointer into their own local variable, however, it is not necessary.

From the \Cpp standard (C++11 12.1) we know that constructors are not required to return any values.

In fact, internally, the constructors return a pointer to the newly created object, i.e., \ITthis.

Now the \TT{dump()} method:

\lstinputlisting[caption=MSVC]{\CURPATH/classes/simple_3_msvc.asm}


Simple enough: \TT{dump()} takes a pointer to the structure that contains the two \Tint's from \ECX,
takes both values from it and passes them to \printf.

The code is much shorter if compiled with optimizations (\Ox):

\lstinputlisting[caption=MSVC]{\CURPATH/classes/simple_4_msvc_Ox.asm}

\myindex{x86!\Instructions!RET}

That's all. The other thing we need to note is that the \gls{stack pointer} was not corrected
with \TT{add esp, X} after the constructor was called. 
At the same time, the constructor has \TT{ret 8} instead of \RET at the end.

\myindex{thiscall}

This is all because the thiscall~(\myref{thiscall}) calling convention is used here,
which together with the stdcall~(\myref{sec:stdcall}) method offers the \gls{callee} to correct the stack
instead of the \gls{caller}.
The \TT{ret x} instruction adds \TT{X} to the value in \ESP, then passes the control to the \gls{caller} function.

See also the section about calling conventions~(\myref{sec:callingconventions}).

It also has to be noted that the compiler decides when to call the constructor and 
destructor\EMDASH{}but we already know that from the \Cpp language basics.

\subsubsectionold{MSVC\EMDASH{}x86-64}
\label{simple_CPP_MSVC_x64}

As we already know, the first 4 function arguments in x86-64 are passed in \RCX, \RDX, \Reg{8} and 
\Reg{9} registers, all the rest\EMDASH{}via the stack.

Nevertheless, the \ITthis pointer to the object is passed in \RCX, the first argument of the method in \RDX,\etc{}.
We can see this in the 
\TT{c(int a, int b)} method internals:

\lstinputlisting[caption=\Optimizing MSVC 2012 x64]{\CURPATH/classes/simple_MSVC_x64_EN.asm}

The \Tint data type is still 32-bit in x64
\footnote{Apparently, for easier porting of 32-bit \CCpp code to x64}, 
so that is why 32-bit register parts are used here.

We also see \TT{JMP printf} instead of \RET in the \TT{dump()} method, that \IT{hack} we already saw
earlier: \myref{JMP_instead_of_RET}.

\subsubsectionold{GCC\EMDASH{}x86}

It is almost the same story in GCC 4.4.1, with a few exceptions.

\lstinputlisting[caption=GCC 4.4.1]{\CURPATH/classes/simple_GCC.asm}

Here we see another \IT{name mangling} style, specific to GNU
\footnote{There is a good document about the various name mangling conventions in different compilers:
[\AgnerFogCC].}
It can also be noted that the pointer to the object is passed as the first function 
argument\EMDASH{}invisible to programmer, of course.

First constructor:

\begin{lstlisting}
                public _ZN1cC1Ev ; weak
_ZN1cC1Ev       proc near               ; CODE XREF: main+10

arg_0           = dword ptr  8

                push    ebp
                mov     ebp, esp
                mov     eax, [ebp+arg_0]
                mov     dword ptr [eax], 667
                mov     eax, [ebp+arg_0]
                mov     dword ptr [eax+4], 999
                pop     ebp
                retn
_ZN1cC1Ev       endp
\end{lstlisting}

It just writes two numbers using the pointer passed in the first (and only) argument.

Second constructor:

\begin{lstlisting}
                public _ZN1cC1Eii
_ZN1cC1Eii      proc near

arg_0           = dword ptr  8
arg_4           = dword ptr  0Ch
arg_8           = dword ptr  10h

                push    ebp
                mov     ebp, esp
                mov     eax, [ebp+arg_0]
                mov     edx, [ebp+arg_4]
                mov     [eax], edx
                mov     eax, [ebp+arg_0]
                mov     edx, [ebp+arg_8]
                mov     [eax+4], edx
                pop     ebp
                retn
_ZN1cC1Eii      endp
\end{lstlisting}

This is a function, the analog of which can look like this:

\begin{lstlisting}
void ZN1cC1Eii (int *obj, int a, int b)
{
  *obj=a;
  *(obj+1)=b;
};
\end{lstlisting}

\dots and that is completely predictable.

Now the \TT{dump()} function:

\begin{lstlisting}
                public _ZN1c4dumpEv
_ZN1c4dumpEv    proc near

var_18          = dword ptr -18h
var_14          = dword ptr -14h
var_10          = dword ptr -10h
arg_0           = dword ptr  8

                push    ebp
                mov     ebp, esp
                sub     esp, 18h
                mov     eax, [ebp+arg_0]
                mov     edx, [eax+4]
                mov     eax, [ebp+arg_0]
                mov     eax, [eax]
                mov     [esp+18h+var_10], edx
                mov     [esp+18h+var_14], eax
                mov     [esp+18h+var_18], offset aDD ; "%d; %d\n"
                call    _printf
                leave
                retn
_ZN1c4dumpEv    endp
\end{lstlisting}

This function in its \IT{internal representation} has only one argument, 
used as pointer to the object (\ITthis).

This function could be rewritten in C like this:

\begin{lstlisting}
void ZN1c4dumpEv (int *obj)
{
  printf ("%d; %d\n", *obj, *(obj+1));
};
\end{lstlisting}

Thus, if we base our judgment on these simple examples, the difference between MSVC and GCC
is the style of the encoding of function names (\IT{name mangling}) and the method for passing a pointer to the object
(via the \ECX register or via the first argument).

\subsubsectionold{GCC\EMDASH{}x86-64}

The first 6 arguments, as we already know, are passed in the \RDI, \RSI, \RDX, \RCX, \Reg{8} and 
\Reg{9} (\SysVABI) registers,
and the pointer to \ITthis via the first one (\RDI) and that is what we see here.
The \Tint data type is also 32-bit here.

The \JMP instead of \RET \IT{hack} is also used here.

\lstinputlisting[caption=GCC 4.4.6 x64]{\CURPATH/classes/simple_GCC_x64_EN.s}

