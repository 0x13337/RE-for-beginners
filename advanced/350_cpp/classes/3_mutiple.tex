\subsection{\RU{Множественное наследование}\EN{Multiple inheritance}}

\RU{Множественное наследование\EMDASH{}это создание класса наследующего поля и методы от двух или более классов.}
\EN{Multiple inheritance is creating a class which inherits fields and methods from two or more classes.}

\RU{Снова напишем простой пример:}\EN{Let's write a simple example again:}

\lstinputlisting{\CURPATH/classes/classes3_multiple.cpp}

\RU{Снова скомпилируем в MSVC 2008 с опциями \Ox и \Obzero и посмотрим код методов \TT{box::dump()}, 
\TT{solid\_object::dump()}, \TT{solid\_box::dump()}:}
\EN{Let's compile it in MSVC 2008 with the \Ox and \Obzero options and see the code of \TT{box::dump()}, 
\TT{solid\_object::dump()} and \TT{solid\_box::dump()}:}

\lstinputlisting[caption=\Optimizing MSVC 2008 /Ob0]{\CURPATH/classes/classes3_1.asm}

\lstinputlisting[caption=\Optimizing MSVC 2008 /Ob0]{\CURPATH/classes/classes3_2.asm}

\lstinputlisting[caption=\Optimizing MSVC 2008 /Ob0]{\CURPATH/classes/classes3_3.asm}

\RU{Выходит, имеем такую разметку в памяти для всех трех классов:}
\EN{So, the memory layout for all three classes is:}

\RU{класс \IT{box}}\EN{\IT{box} class}:

\begin{center}
\begin{tabular}{ | l | l | }
\hline
  \tableheader{} \\
\hline
  +0x0 & width \\
\hline
  +0x4 & height \\
\hline
  +0x8 & depth \\
\hline
\end{tabular}
\end{center}

\RU{класс \IT{solid\_object}}\EN{\IT{solid\_object} class}:

\begin{center}
\begin{tabular}{ | l | l | }
\hline
  \tableheader{} \\
\hline
  +0x0 & density \\
\hline
\end{tabular}
\end{center}

\RU{Можно сказать, что разметка класса \IT{solid\_box} \IT{объединённая}:}
\EN{It can be said that the \IT{solid\_box} class memory layout is \IT{united}:}

\RU{Класс \IT{solid\_box}}\EN{\IT{solid\_box} class}:

\begin{center}
\begin{tabular}{ | l | l | }
\hline
  \tableheader{} \\
\hline
  +0x0 & width \\
\hline
  +0x4 & height \\
\hline
  +0x8 & depth \\
\hline
  +0xC & density \\
\hline
\end{tabular}
\end{center}

\RU{Код методов \TT{box::get\_volume()} и \TT{solid\_object::get\_density()} тривиален:}
\EN{The code of the \TT{box::get\_volume()} and \TT{solid\_object::get\_density()} methods is trivial:}

\lstinputlisting[caption=\Optimizing MSVC 2008 /Ob0]{\CURPATH/classes/classes3_4.asm}

\lstinputlisting[caption=\Optimizing MSVC 2008 /Ob0]{\CURPATH/classes/classes3_5.asm}

\RU{А вот код метода \TT{solid\_box::get\_weight()} куда интереснее:}
\EN{But the code of the \TT{solid\_box::get\_weight()} method is much more interesting:}

\lstinputlisting[caption=\Optimizing MSVC 2008 /Ob0]{\CURPATH/classes/classes3_6.asm}

\RU{\TT{get\_weight()} просто вызывает два метода, но для \TT{get\_volume()} он передает просто указатель на \TT{this}, а для \TT{get\_density()}, он передает указатель на \TT{this} сдвинутый на 12 байт (либо \TT{0xC} байт), а там, 
в разметке класса \TT{solid\_box}, как раз начинаются поля класса \TT{solid\_object}.}
\EN{\TT{get\_weight()} just calls two methods, but for \TT{get\_volume()} it just passes pointer to \TT{this},
and for \TT{get\_density()} it passes a pointer to \TT{this} incremented by 12 (or \TT{0xC}) bytes, and there,
in the \TT{solid\_box}
class memory layout, the fields of the \TT{solid\_object} class start.}

\RU{Так, метод \TT{solid\_object::get\_density()} будет полагать что работает с обычным
классом \TT{solid\_object}, а метод \TT{box::get\_volume()} будет работать только со своими тремя полями, полагая, 
что работает с обычным экземпляром класса \TT{box}.}
\EN{Thus, the \TT{solid\_object::get\_density()} method will believe like it is dealing with the usual \TT{solid\_object} class,
and the \TT{box::get\_volume()} method will work with its three fields, believing this is just the usual object of class \TT{box}.}

\RU{Таким образом, можно сказать, что экземпляр класса-наследника нескольких классов представляет в памяти просто 
\IT{объединённый} класс, содержащий все унаследованные поля. А каждый унаследованный метод вызывается с передачей
ему указателя на соответствующую часть структуры.}
\EN{Thus, we can say, an object of a class,
that inherits from several other classes, is representing in memory as a \IT{united} class, that contains all inherited fields.
And each inherited method is called with a pointer to 
the corresponding structure's part.}

