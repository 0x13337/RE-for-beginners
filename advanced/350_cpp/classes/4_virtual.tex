\subsection{\RU{Виртуальные методы}\EN{Virtual methods}}

\RU{И снова простой пример:}\EN{Yet another simple example:}

\lstinputlisting{\CURPATH/classes/classes4_virtual.cpp}

\RU{У класса \IT{object} есть виртуальный метод \TT{dump()}, 
впоследствии заменяемый в классах-наследниках \IT{box} и \IT{sphere}.}
\EN{Class \IT{object} has a virtual method \TT{dump()} that is being replaced in the inheriting \IT{box} and \IT{sphere} classes.}

\RU{Если в какой-то среде, где неизвестно, какого типа является экземпляр класса, как в функции \main в примере, 
вызывается виртуальный метод \TT{dump()}, где-то должна сохраняться информация о том, какой же метод в итоге 
вызвать.}
\EN{If we are in an environment where it is not known the type of an object, as in the \main function in example,
where the virtual method \TT{dump()} is called, the information about its type must be stored somewhere, to
be able to call the relevant virtual method.}

\RU{Скомпилируем в MSVC 2008 с опциями \Ox и \Obzero и посмотрим код функции \main:}
\EN{Let's compile it in MSVC 2008 with the \Ox and \Obzero options and see the code of \main:}

\lstinputlisting{\CURPATH/classes/classes4_1.asm}

\RU{Указатель на функцию \TT{dump()} берется откуда-то из экземпляра класса (объекта). 
Где мог записаться туда адрес нового метода-функции?
Только в конструкторах, больше негде: ведь в функции \main ничего более не вызывается.}
\EN{A pointer to the \TT{dump()} function is taken somewhere from the object.
Where could we store the address of the new method?
Only somewhere in the constructors: there is no other place since nothing else is called in the \main function.}
\footnote{\RU{Об указателях на функции читайте больше в соответствующем разделе}\EN{You can read more about pointers to functions in the relevant section}:(\myref{sec:pointerstofunctions})}

\RU{Посмотрим код конструктора класса \IT{box}:}
\EN{Let's see the code of the constructor of the \IT{box} class:}

\lstinputlisting{\CURPATH/classes/classes4_2.asm}

\RU{Здесь мы видим, что разметка класса немного другая: в качестве первого поля имеется указатель 
на некую таблицу \TT{box::`vftable'} (название оставлено компилятором MSVC).}
\EN{Here we see a slightly different memory layout:
the first field is a pointer to some table
\TT{box::`vftable'} (the name was set by the MSVC compiler).}

\label{RTTI}
\index{\Cpp!RTTI}
\RU{В этой таблице есть ссылка на таблицу с названием \TT{box::`RTTI Complete Object Locator'} и еще ссылка на 
метод \TT{box::dump()}.}
\EN{In this table we see a link to a table named \TT{box::`RTTI Complete Object Locator'} and also a link
to the \TT{box::dump()} method.}
\RU{Итак, это называется таблица виртуальных методов и \ac{RTTI}.
Таблица виртуальных методов хранит в себе адреса методов, а \ac{RTTI} хранит информацию о типах вообще.}
\EN{These are called virtual methods table and \ac{RTTI}.
The table of virtual methods contains the addresses of methods and the \ac{RTTI} table contains information about types.}
\RU{Кстати, \ac{RTTI}-таблицы\EMDASH{}это именно те таблицы, информация из которых используются при вызове \IT{dynamic\_cast} и \IT{typeid} в \Cpp. 
Вы можете увидеть, что здесь хранится даже имя класса в виде обычной строки.}
\EN{By the way, the \ac{RTTI} tables are used while calling \IT{dynamic\_cast} and \IT{typeid} in \Cpp.
You can also see here the class name as a plain text string.}
\RU{Так, какой-нибудь метод базового класса \IT{object} может вызвать виртуальный метод \TT{object::dump()} что 
в итоге вызовет нужный метод унаследованного класса, потому что информация о нем присутствует прямо в этой 
структуре класса.}
\EN{Thus, a method of the base \IT{object} class may call the virtual method \IT{object::dump()}, which in turn will call
a method of an inherited class, since that information is present right in the object's structure.}

\RU{Работа с этими таблицами и поиск адреса нужного метода, занимает какое-то время процессора, возможно, 
поэтому считается что работа с виртуальными методами медленна.}
\EN{Some additional CPU time is needed for doing look-ups in these tables and finding the right virtual method address, 
thus virtual methods are widely considered as slightly slower than common methods.}

\RU{В сгенерированном коде от GCC \ac{RTTI}-таблицы устроены чуть-чуть иначе.}
\EN{In GCC-generated code the \ac{RTTI} tables are constructed slightly differently.}
% TODO: добавить...
