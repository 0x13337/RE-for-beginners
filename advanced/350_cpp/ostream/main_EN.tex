\sectionold{ostream}
\myindex{\Cpp!ostream}

Let's start again with a \q{hello world} example, but now we are going to use ostream:

\lstinputlisting{\CURPATH/ostream/1.cpp}

Almost any \Cpp textbook tells us that the \TT{<<} operation can be replaced (overloaded) for other types.
That is what is done in ostream.
We see that \TT{operator<<} is called for ostream:

\lstinputlisting[caption=MSVC 2012 (reduced listing)]{\CURPATH/ostream/1.asm}

Let's modify the example:

\lstinputlisting{\CURPATH/ostream/2.cpp}

And again, from many \Cpp textbooks we know that the result of each \TT{operator<<} in ostream is forwarded to the
next one.
Indeed:

\lstinputlisting[caption=MSVC 2012]{\CURPATH/ostream/2_EN.asm}

If we would rename \TT{operator<<} method name to \ttf{}, that code will looks like:

\begin{lstlisting}
f(f(std::cout, "Hello, "), "world!");
\end{lstlisting}

GCC generates almost the same code as MSVC.

