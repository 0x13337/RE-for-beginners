\section{References}
\index{\Cpp!References}
\label{cpp_references}

\RU{References в \Cpp это тоже указатели (\ref{label_pointers}), но их называют \IT{безопасными} (safe), потому что работая с ними, 
труднее сделать
ошибку}
\EN{In \Cpp, references are pointers (\ref{label_pointers}) as well, but they are called \IT{safe}, because it is harder to make a mistake while
dealing with them} \cite[8.3.2]{CPP11}.
\RU{Например, reference всегда должен указывать объект того же типа и не может быть NULL}
\EN{For example, reference must always be pointing to the object of corresponding type and cannot be NULL}
\cite[8.6]{ParashiftCPPFAQ}.
\RU{Более того, reference нельзя менять, нельзя его заставить указывать на другой объект (reseat)}
\EN{Even more than that, reference cannot be changed, it is impossible to point it to another object (reseat)}
\cite[8.5]{ParashiftCPPFAQ}.

\RU{Если мы попробуем изменить пример с указателями (\ref{label_pointers}) чтобы он использовал reference вместо указателей:}
\EN{If we will try to change the pointers example (\ref{label_pointers}) to use references instead of pointers:}

\begin{lstlisting}
void f2 (int x, int y, int & sum, int & product)
{
	sum=x+y;
	product=x*y;
};
\end{lstlisting}

\RU{То выяснится, что скомпилированный код абсолютно такой 
же как и в примере с указателями (\ref{label_pointers}):}
\EN{Then we'll figure out the compiled code is just the same 
as in pointers example (\ref{label_pointers}):}

\begin{lstlisting}[caption=\Optimizing MSVC 2010]
_x$ = 8							; size = 4
_y$ = 12						; size = 4
_sum$ = 16						; size = 4
_product$ = 20						; size = 4
?f2@@YAXHHAAH0@Z PROC					; f2
	mov	ecx, DWORD PTR _y$[esp-4]
	mov	eax, DWORD PTR _x$[esp-4]
	lea	edx, DWORD PTR [eax+ecx]
	imul eax, ecx
	mov ecx, DWORD PTR _product$[esp-4]
	push esi
	mov	esi, DWORD PTR _sum$[esp]
	mov	DWORD PTR [esi], edx
	mov	DWORD PTR [ecx], eax
	pop	esi
	ret	0
?f2@@YAXHHAAH0@Z ENDP					; f2
\end{lstlisting}

(\RU{Почему у функций в \Cpp такие странные имена, описано здесь}
\EN{A reason why \Cpp functions has such strange names, is explained here}: \ref{namemangling}.)
