\chapter{\RU{Отрицательные индексы массивов}\EN{Negative array indices}}
\label{negative_array_indices}

\EN{It's possible to address the space \IT{before} an array by supplying a negative index, e.g.}\RU{Возможно
адресовать место в памяти \IT{перед} массивом задавая отрицательный индекс, например}, $array[-1]$.

\index{FORTRAN}
\EN{It's very hard to say why one should use it, there is probably only one known practical application
of this technique}\RU{Трудно сказать, зачем это вообще может понадобиться, известно только одно практическое
применение этому}.
% TODO: разобраться:
% another use (which seems to be used more often) can be found in "Transaction processing" by Jim Gray, p. 755
% basically, it's used in structures that describe pages, and the entries are indexed from the beginning,
% and the directory entries - from the end with negative indices.
\EN{\CCpp array elements indices start at 0, but some \ac{PL}s have a first index at 1 
(at least FORTRAN).}
\RU{Элементы массивов в \CCpp индексируются, начиная с 0, но в некоторых \ac{PL}, первый элемент 
индексируется через 1 (как минимум FORTRAN).}
\EN{Programmers may still have this habit, so using this little trick, it's possible to address the first element
in \CCpp using index 1}\RU{Привычка у программистов может остаться, так что все еще возможно адресовать
первый элемент через 1 в \CCpp используя этот трюк}:

\lstinputlisting{\CURPATH/neg_array.c}

\lstinputlisting[caption=\NonOptimizing MSVC 2010,label=neg_array_c,numbers=left]{\CURPATH/neg_array.asm.\LANG}

\EN{So we have \TT{array[]} of ten elements, filled with $0 \ldots 9$ bytes.}
\RU{Так что у нас тут массив \TT{array[]} из десяти элементов, заполненный байтами $0 \ldots 9$.}
\EN{Then we have the \TT{fakearray[]} pointer, which points one byte before \TT{array[]}.}
\RU{Затем у нас указатель \TT{fakearray[]} указывающий на один байт перед \TT{array[]}.}
\TT{fakearray[1]} \RU{указывает точно на}\EN{points exactly to} \TT{array[0]}.
\EN{But we are still curious, what is there before}\RU{Но нам все еще любопытно, что же находится перед} \TT{array[]}?
\RU{Мы добавляем}\EN{We have added} \TT{random\_value} \RU{перед}\EN{before} \TT{array[]} \RU{и установим её в}\EN{and set 
it to} \TT{0x11223344}.
\EN{The non-optimizing compiler allocated the variables in the order they were declared, so yes, the 32-bit \TT{random\_value}
is right before the array}\RU{Неоптимизирующий компилятор выделяет переменные в том же порядке, в котором они 
объявлены, так что да, 32-битная \TT{random\_value} находится точно перед массивом}.

\RU{Запускаем, и}\EN{We ran it, and}:

\begin{lstlisting}
first element 0
second element 1
last element 9
array[-1]=11, array[-2]=22, array[-3]=33, array[-4]=44
\end{lstlisting}

\EN{Here is the stack fragment we will copypaste from \olly's stack window (with comments added by the author):}
\RU{Фрагмент стека, который мы скопипастим из окна стека в \olly (включая комментарии автора):}

\lstinputlisting[caption=\NonOptimizing MSVC 2010]{\CURPATH/stack.txt.\LANG}

\RU{Указатель на}\EN{The pointer to the} \TT{fakearray[]} (\TT{0x001DFBD3}) \RU{это действительно адрес}\EN{is indeed 
the address of} \TT{array[]} \RU{в стеке}\EN{in the stack} (\TT{0x001DFBD4}), 
\RU{но минус 1 байт}\EN{but minus 1 byte}.

\EN{It's still very hackish and dubious trick. Doubtfully anyone should use it in production code,
but as a demonstration, it fits perfectly here.}
\RU{Трюк этот все-таки слишком хакерский и сомнительный. Вряд ли кто-то будет его использовать в своем коде,
но для демонстрации, он здесь очень уместен.}

