\chapter{Negative array indices}
\label{negative_array_indices}

It's possible to address the space \IT{before} an array by supplying a negative index, e.g., $array[-1]$.

\section{Addressing string from the end}

\myindex{Python}
Python \ac{PL} allows to address arrays and strings from the end.
For example, \IT{string[-1]} returns the last character, \IT{string[-2]} returns penultimate, etc.
Hard to believe, but this is also possible in C/C++:

\lstinputlisting{\CURPATH/pythonesque_addressing.c}

It works, but \textit{s\_end} must always has an address of terminating zero byte at the end of \textit{s} string.
If \textit{s} string's size get changed, \textit{s\_end} must be updated.

The trick is dubious, but again, this is a demonstration of negative indices.

\section{Adressing some kind of block from the end}

Let's first recall why stack grows backwards (\myref{stack_grow_backwards}).
There is some kind of block in memory and you need to store both heap and stack there, and you are not sure,
how big they both can grow during runtime.

You can set a \IT{heap} pointer to the beginning of the block,
then you can set a \IT{stack} pointer to the end of the block (\IT{heap + size\_of\_block}),
and then you can address \IT{nth} element of stack like \IT{stack[-n]}.
For example, \IT{stack[-1]} for 1st element, \IT{stack[-2]} for 2nd, etc.

This will work in the same fashion, as our trick of addressing string from the end.

You can easily check if the structures has not begun to overlap each other:
just be sure that address of the last element in \IT{heap} is below the address of the last element of \IT{stack}.

Unfortunately, $-0$ as index will not work,
since two's complement way of representing negative numbers (\myref{sec:signednumbers}) don't allow negative zero,
so it cannot be distinguished from positive zero.

This method is also mentioned in ``Transaction processing'', Jim Gray, 1993,
``The Tuple-Oriented File System'' chapter, p. 755.

\section{Arrays started at 1}

\myindex{Fortran}
\myindex{Mathematica}
Fortran and Mathematica defined first element of array as 1th, probably because this is tradition in mathematics.
Other \ac{PL}s like \CCpp defined it as 0th.
Which is better?
\myindex{Edsger W. Dijkstra}
Edsger W. Dijkstra argued that latter is better
\footnote{See \url{https://www.cs.utexas.edu/users/EWD/transcriptions/EWD08xx/EWD831.html}}.

But programmers may still have a habit after Fortran, so using this little trick, it's possible to address the first element
in \CCpp using index 1:

\lstinputlisting{\CURPATH/neg_array.c}

\lstinputlisting[caption=\NonOptimizing MSVC 2010,label=neg_array_c,numbers=left]{\CURPATH/neg_array_EN.asm}

So we have \TT{array[]} of ten elements, filled with $0 \ldots 9$ bytes.

Then we have the \TT{fakearray[]} pointer, which points one byte before \TT{array[]}.

\TT{fakearray[1]} points exactly to \TT{array[0]}.
But we are still curious, what is there before \TT{array[]}?
We have added \TT{random\_value} before \TT{array[]} and set 
it to \TT{0x11223344}.
The non-optimizing compiler allocated the variables in the order they were declared, so yes, the 32-bit \TT{random\_value}
is right before the array.

We ran it, and:

\begin{lstlisting}
first element 0
second element 1
last element 9
array[-1]=11, array[-2]=22, array[-3]=33, array[-4]=44
\end{lstlisting}

Here is the stack fragment we will copypaste from \olly's stack window (with comments added by the author):

\lstinputlisting[caption=\NonOptimizing MSVC 2010]{\CURPATH/stack_EN.txt}

The pointer to the \TT{fakearray[]} (\TT{0x001DFBD3}) is indeed 
the address of \TT{array[]} in the stack (\TT{0x001DFBD4}), 
but minus 1 byte.

It's still very hackish and dubious trick. Doubtfully anyone should use it in production code,
but as a demonstration, it fits perfectly here.


