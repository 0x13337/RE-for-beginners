\section{\Example{} \#5}
\label{win16_near_far_pointers}

\lstinputlisting{\CURPATH/ex5.c}

\lstinputlisting{\CURPATH/ex5.lst}

\index{Intel!8086!\RU{Модель памяти}\EN{Memory model}}
\RU{Здесь мы можем увидеть разницу между указателями}
\EN{Here we see a difference between so-called} ``near'' \RU{и указателями}\EN{pointers and} ``far'' 
\RU{еще один ужасный артефакт сегментированной памяти 16-битного 8086}
\EN{pointers: another weird artefact of segmented memory in 16-bit 8086}.

\RU{Читайте больше об этом}\EN{Read more about it}: \ref{8086_memory_model}.

\RU{Указатели }``near'' \RU{(``близкие'') это те которые указывают в пределах текущего сегмента}
\EN{pointers are those which points within current data segment}.
\RU{Поэтому}\EN{Hence}, \RU{ф-ция }\TT{string\_compare()} \RU{берет на вход только 2 16-битных
значения и работает с данными расположенными в сегменте, на который указывает \TT{DS}}\EN{function takes only
two 16-bit pointers, and accesses data as it is located in the segment \TT{DS} pointing to} 
(\RU{инструкция }\TT{mov al, [bx]} \RU{на самом деле работает как}\EN{instruction actually works like} 
\TT{mov al, ds:[bx]}\EMDASH{}\TT{DS} \RU{используется здесь неявно}\EN{is implicitly used here}).

\RU{Указатели }``far'' \RU{(далекие) могут указывать на данные в другом сегменте памяти}
\EN{pointers are those which may point to data in another segment memory}.
\RU{Поэтому}\EN{Hence} \TT{string\_compare\_far()} \RU{берет на вход 16-битную пару как указатель, загружает старшую
часть в сегментный регистр \TT{ES} и обращается к данным через него}
\EN{takes 16-bit pair as a pointer, loads high part of it to \TT{ES} segment register and accessing
data through it} (\TT{mov al, es:[bx]}).
\RU{Указатели }``far'' \RU{также используются в моем win16-примере касательно}
\EN{pointers are also used in my} \TT{MessageBox()}\EN{ win16 example}: \ref{win16_messagebox}. 
\RU{Действительно, ядро Windows должно знать, из какого сегмента данных читать текстовые строки, так что ему нужна
полная информация}\EN{Indeed, Windows kernel is not aware which data segment to use when accessing text strings,
so it need more complete information}.

\RU{Причина этой разница в том, что компактная программа вполне может обойтись одним сегментом данных размером 64 килобайта,
так что старшую часть указателя передавать не нужна (ведь она одинаковая везде)}
\EN{The reason for this distinction is that compact program may use just one 64kb data segment, so it doesn't need
to pass high part of the address, which is always the same}.
\RU{Б\`{о}льшие программы могут использовать несколько сегментов данных размером 64 килобайта,
так что нужно указывать каждый раз, в каком сегменте расположены данные}
\EN{Bigger program may use several 64kb data segments, so it needs to specify each time, in which segment data is located}.

\RU{То же касается и сегментов кода}\EN{The same story for code segments}.
\RU{Компактная программа может расположиться в пределах одного 64kb-сегмента, тогда
ф-ции в ней будут вызываться инструкцией}\EN{Compact program may have all executable code within one 64kb-segment, 
then all functions will be called in it using} 
\TT{CALL NEAR}\RU{, а возвращаться управление используя}\EN{ instruction, and code flow will be returned using} \TT{RETN}.
\RU{Но если сегментов кода несколько, тогда и адрес вызываемой ф-ции будет задаваться парой, 
вызываться она будет используя}
\EN{But if there are several code segments, then the address of the function will be specified by pair,
it will be called using}
\TT{CALL FAR}\RU{, а возвращаться управление используя}\EN{ instruction, and the code flow will be returned using} \TT{RETF}.

\RU{Это то что задается в компиляторе указывая}\EN{This is what to be set in compiler by specifying} ``memory model''.

\RU{Компиляторы под MS-DOS и Win16 имели разные библиотеки под разные модели памяти: они отличались типами указателей для
кода и данных}\EN{Compilers targeting MS-DOS and Win16 has specific libraries for each memory model: they were differ
by pointer types for code and data}.

