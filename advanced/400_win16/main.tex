\chapter{Windows 16-bit}
\index{Windows!Windows 3.x}

\RU{16-битные программы под Windows в наше время редки, хотя я иногда вожусь с ними, в смысле ретрокомпьютинга,
либо защищенные донглами (\ref{dongles})}
\EN{16-bit Windows programs are rare nowadays, but in the cases of retrocomputing
or dongle hacking (\ref{dongles}), I sometimes dig into these}.

\RU{16-битные версии Windows были вплоть до}\EN{16-bit Windows versions were up to} 3.11.
96/98/ME \RU{также поддерживает 16-битный код, как и все 32-битные OS линейки}
\EN{also support 16-bit code, as well as the 32-bit versions of the} \gls{Windows NT}\EN{ line}.
\RU{64-битные версии}\EN{The 64-bit versions of} \gls{Windows NT} \RU{не поддерживают 16-битный код вообще}
\EN{line do not support 16-bit executable code at all}.

\RU{Код напоминает тот что под MS-DOS}\EN{The code resembles MS-DOS's one}.

\RU{Исполняемые файлы имеют NE-тип (так называемый ``new executable'').}
\EN{Executable files are of type NE-type (so-called ``new executable'').}

\RU{Все рассмотренные здесь примеры скомпилированы компилятором}
\EN{All examples considered here were compiled by the} OpenWatcom 1.9 \RU{используя эти опции}\EN{compiler, using these switches}:\\
\TT{wcl.exe -i=C:/WATCOM/h/win/ -s -os -bt=windows -bcl=windows example.c}

\section{\Example \#1}
\begin{lstlisting}
#include <windows.h>

int PASCAL WinMain( HINSTANCE hInstance,
                    HINSTANCE hPrevInstance,
                    LPSTR lpCmdLine,
                    int nCmdShow )
{
	MessageBeep(MB_ICONEXCLAMATION);
	return 0;
};
\end{lstlisting}

\begin{lstlisting}
WinMain         proc near
                push    bp
                mov     bp, sp
                mov     ax, 30h ; '0'   ; MB_ICONEXCLAMATION constant
                push    ax
                call    MESSAGEBEEP
                xor     ax, ax          ; return 0
                pop     bp
                retn    0Ah
WinMain         endp
\end{lstlisting}

\RU{Пока всё просто}\EN{Seems to be easy, so far}.

\sectionold{\Example{} \#2}
\label{win16_messagebox}

\begin{lstlisting}
#include <windows.h>

int PASCAL WinMain( HINSTANCE hInstance,
                    HINSTANCE hPrevInstance,
                    LPSTR lpCmdLine,
                    int nCmdShow )
{
	MessageBox (NULL, "hello, world", "caption", MB_YESNOCANCEL);
	return 0;
};
\end{lstlisting}

\begin{lstlisting}
WinMain         proc near
                push    bp
                mov     bp, sp
                xor     ax, ax          ; NULL
                push    ax
                push    ds
                mov     ax, offset aHelloWorld ; 0x18. "hello, world"
                push    ax
                push    ds
                mov     ax, offset aCaption ; 0x10. "caption"
                push    ax
                mov     ax, 3           ; MB_YESNOCANCEL
                push    ax
                call    MESSAGEBOX
                xor     ax, ax          ; return 0
                pop     bp
                retn    0Ah
WinMain         endp

dseg02:0010 aCaption        db 'caption',0
dseg02:0018 aHelloWorld     db 'hello, world',0
\end{lstlisting}

\myindex{x86!\Instructions!RET}
\RU{Пара важных моментов: соглашение о передаче аргументов здесь \TT{PASCAL}: оно указывает что самый
первый аргумент должен передаваться первым}
\EN{Couple important things here: the \TT{PASCAL} calling convention dictates passing the first argument first} 
(\TT{MB\_YESNOCANCEL}), \RU{а самый последний аргумент\EMDASH{}последним}\EN{and the last argument\EMDASH{}last} (NULL).
\RU{Это соглашение также указывает вызываемой функции восстановить}
\EN{This convention also tells the \gls{callee} to restore the} \gls{stack pointer}:
\RU{поэтому инструкция}\EN{hence the} \TT{RETN} \RU{имеет аргумент}\EN{instruction has} \TT{0Ah} 
\RU{означая что указатель нужно сдвинуть вперед на 10 байт во время возврата из функции}
\EN{as argument, which implies that the pointer has to be increased by 10 bytes when the function exits}.
\RU{Это как}\EN{It is like} stdcall (\myref{sec:stdcall}), \EN{but the arguments are passed in 
\q{natural} order}\RU{только аргументы передаются в \q{естественном} порядке}.

\RU{Указатели передаются парами: сначала сегмент данных, потом указатель внутри сегмента}
\EN{The pointers are passed in pairs: first the data segment is passed, then the pointer inside the segment}.
\RU{В этом примере только один сегмент, так что \TT{DS} всегда указывает на сегмент данных в исполняемом
файле}\EN{There is only one segment in this example, so \TT{DS} always points to the data segment of the executable}.


\sectionold{\Example{} \#3}

\lstinputlisting{\CURPATH/ex3.c}

\lstinputlisting{\CURPATH/ex3.lst}

\RU{Немного расширенная версия примера из предыдущей секции}
\EN{Somewhat extended example from the previous section}.

\section{\Example{} \#4}

\label{win16_32bit_values}

\lstinputlisting{\CURPATH/ex4.c}

\lstinputlisting{\CURPATH/ex4.lst}

\index{MS-DOS}
\RU{32-битные значения (тип данных \TT{long} означает 32-бита, а \Tint здесь 16-битный) 
в 16-битном коде (и в MS-DOS и в Win16) передаются парами)}
\EN{32-bit values (the \TT{long} data type implies 32 bits, while \Tint is 16-bit)
in 16-bit code (both MS-DOS and Win16) are passed in pairs}.
\RU{Это так же как и 64-битные значения передаются в 32-битной среде}
\EN{It is just like when 64-bit values are used in a 32-bit environment} (\myref{sec:64bit_in_32_env}).

\TT{sub\_B2 here} \RU{здесь это библиотечная ф-ция написанная разработчиками компилятора, делающая}
\EN{is a library function written by the compiler's developers that does} ``long multiplication'', \RU{т.е., перемножает
два 32-битных значения}\EN{i.e., multiplies two 32-bit values}.
\RU{Другие ф-ции компиляторов делающие то же самое перечислены здесь}
\EN{Other compiler functions that do the same are listed here}: \myref{sec:MSVC_library_func}, \myref{sec:GCC_library_func}.

\index{x86!\Instructions!ADD}
\index{x86!\Instructions!ADC}
\EN{The}\RU{Пара инструкций} \TT{ADD}/\TT{ADC} \RU{используется для сложения этих составных значений}
\EN{instruction pair is used for addition of compound values}: 
\RU{\TT{ADD} может установить или сбросить флаг \TT{CF}, а \TT{ADC} будет использовать его после.}
\EN{\TT{ADD} may set/clear the \TT{CF} flag, and \TT{ADC} uses it after.}

\EN{The}\RU{Пара инструкций} \TT{SUB}/\TT{SBB} \RU{используется для вычитания}\EN{instruction pair is used for subtraction}: 
\RU{\TT{SUB} может установить или сбросить флаг \TT{CF}, \TT{SBB} будет использовать его после.}
\EN{\TT{SUB} may set/clear the \TT{CF} flag, \TT{SBB} uses it after.}

\RU{32-битные значения возвращаются из ф-ций в паре регистров \TT{DX:AX}}
\EN{32-bit values are returned from functions in the \TT{DX:AX} register pair}.

\RU{Константы так же передаются как пары в}\EN{Constants are also passed in pairs in} \TT{WinMain()}\EN{ here}.

\index{x86!\Instructions!CWD}
\RU{Константа 123 типа \Tint в начале конвертируется (учитывая знак) в 32-битное значение 
используя инструкция \TT{CWD}}
\EN{The \Tint{}-typed 123 constant is first converted according to its sign into a 32-bit value using the \TT{CWD} instruction}.


\section{\Example{} \#5}
\label{win16_near_far_pointers}

\lstinputlisting{\CURPATH/ex5.c}

\lstinputlisting{\CURPATH/ex5.lst}

\index{Intel!8086!\RU{Модель памяти}\EN{Memory model}}
\RU{Здесь мы можем увидеть разницу между указателями}
\EN{Here we see a difference between so-called} ``near'' \RU{и указателями}\EN{pointers and} ``far'' 
\RU{еще один ужасный артефакт сегментированной памяти 16-битного 8086}
\EN{pointers: another weird artefact of segmented memory in 16-bit 8086}.

\RU{Читайте больше об этом}\EN{Read more about it}: \ref{8086_memory_model}.

\RU{Указатели }``near'' \RU{(``близкие'') это те которые указывают в пределах текущего сегмента}
\EN{pointers are those which points within current data segment}.
\RU{Поэтому}\EN{Hence}, \RU{ф-ция }\TT{string\_compare()} \RU{берет на вход только 2 16-битных
значения и работает с данными расположеными в сегменте, на который указывает \TT{DS}}\EN{function takes only
two 16-bit pointers, and accesses data as it is located in the segment \TT{DS} pointing to} 
(\RU{инструкция }\TT{mov al, [bx]} \RU{на самом деле работает как}\EN{instruction actually works like} 
\TT{mov al, ds:[bx]}\EMDASH{}\TT{DS} \RU{используется здесь неявно}\EN{is implicitly used here}).

\RU{Указатели }``far'' \RU{(далекие) могут указывать на данные в другом сегменте памяти}
\EN{pointers are those which may point to data in another segment memory}.
\RU{Поэтому}\EN{Hence} \TT{string\_compare\_far()} \RU{берет на вход 16-битную пару как указатель, загружает старшую
часть в сегментный регистр \TT{ES} и обращается к данным через него}
\EN{takes 16-bit pair as a pointer, loads high part of it to \TT{ES} segment register and accessing
data through it} (\TT{mov al, es:[bx]}).
\RU{Указатели }``far'' \RU{также используются в моем win16-примере касательно}
\EN{pointers are also used in my} \TT{MessageBox()}\EN{ win16 example}: \ref{win16_messagebox}. 
\RU{Действительно, ядро Windows должно знать, из какого сегмента данных читать текстовые строки, так что ему нужна
полная информация}\EN{Indeed, Windows kernel is not aware which data segment to use when accessing text strings,
so it need more complete information}.

\RU{Причина этой разница в том, что компактная программа вполне может обойтись одним сегментом данных размером 64 килобайта,
так что старшую часть указателя передавать не нужна (ведь она одинаковая везде)}
\EN{The reason for this distinction is that compact program may use just one 64kb data segment, so it doesn't need
to pass high part of the address, which is always the same}.
\RU{Б\`{о}льшие программы могут использовать несколько сегментов данных размером 64 килобайта,
так что нужно указывать каждый раз, в каком сегменте расположены данные}
\EN{Bigger program may use several 64kb data segments, so it needs to specify each time, in which segment data is located}.

\RU{То же касается и сегментов кода}\EN{The same story for code segments}.
\RU{Компактная программа может расположиться в пределах одного 64kb-сегмента, тогда
ф-ции в ней будут вызываться инструкцией}\EN{Compact program may have all executable code within one 64kb-segment, 
then all functions will be called in it using} 
\TT{CALL NEAR}\RU{, а возвращаться управление используя}\EN{ instruction, and code flow will be returned using} \TT{RETN}.
\RU{Но если сегментов кода несколько, тогда и адрес вызываемой ф-ции будет задаваться парой, 
вызываться она будет используя}
\EN{But if there are several code segments, then the address of the function will be specified by pair,
it will be called using}
\TT{CALL FAR}\RU{, а возвращаться управление используя}\EN{ instruction, and the code flow will be returned using} \TT{RETF}.

\RU{Это то что задается в компиляторе указывая}\EN{This is what to be set in compiler by specifying} ``memory model''.

\RU{Компиляторы под MS-DOS и Win16 имели разные библиотеки под разные модели памяти: они отличались типами указателей для
кода и данных}\EN{Compilers targeting MS-DOS and Win16 has specific libraries for each memory model: they were differ
by pointer types for code and data}.


\section{\Example{} \#6}

\lstinputlisting{\CURPATH/ex6.c}

\lstinputlisting{\CURPATH/ex6.lst}

\index{\CStandardLibrary!time()}
\index{\CStandardLibrary!localtime()}
\RU{Время в формате UNIX это 32-битное значение, так что оно возвращается в паре регистров \TT{DX:AX} и сохраняется
в двух локальны 16-битных переменных}
\EN{UNIX time is a 32-bit value, so it is returned in the \TT{DX:AX} register pair and stored in two local 16-bit variables}.
\RU{Потом указатель на эту пару передается в ф-цию}\EN{Then a pointer to the pair is passed to the}
\TT{localtime()}\EN{ function}.
\RU{Ф-ция}\EN{The} \TT{localtime()} \RU{имеет структуру}\EN{function has a} \TT{struct tm} \RU{расположенную у себя
где-то внутри, так что только указатель на нее возвращается}
\EN{allocated somewhere in the guts of the C library, so only a pointer to it is returned}. 
\RU{Кстати, это также означает что функцию нельзя вызывать еще раз, пока её результаты не были использованы}
\EN{By the way, this also implies that the function cannot be called again until its results are used}.

\RU{Для ф-ций}\EN{For the} \TT{time()} \AndENRU \TT{localtime()} \RU{используется
Watcom-соглашение о вызовах: первые четыре аргумента передаются через регистры}
\EN{functions, a Watcom calling convention is used here:
the first four arguments are passed in the} \TT{AX}, \TT{DX}, \TT{BX} \AndENRU \TT{CX}, \RU{а остальные аргументы через стек}
\EN{registers, and the rest arguments are via the stack}.
\RU{Ф-ции, использующие это соглашение, маркируется символом подчеркивания в конце имени}
\EN{The functions using this convention are also marked by underscore at the end of their name}.

\RU{Для вызова ф-ции }\TT{sprintf()} \RU{используется обычное соглашение \IT{cdecl} (\myref{cdecl}) вместо 
\TT{PASCAL} или Watcom, так что аргументы передаются привычным образом}
\EN{does not use the \TT{PASCAL} calling convention, nor the Watcom one,
so the arguments are passed in the normal \IT{cdecl} way (\myref{cdecl})}.

\subsection{\RU{Глобальные переменные}\EN{Global variables}}

\RU{Это тот же пример, только переменные теперь глобальные}
\EN{This is the same example, but now these variables are global}:

\lstinputlisting{\CURPATH/ex6_global.c}

\lstinputlisting{\CURPATH/ex6_global.lst}

\RU{\TT{t} не будет использоваться, но компилятор создал код, записывающий в эту переменную.}
\EN{\TT{t} is not to be used, but the compiler emitted the code which stores the value.}
\RU{Потому что он не уверен, может быть это значение будет прочитано где-то в другом модуле.}
\EN{Because it is not sure, maybe that value will eventually be used in some other module.}


