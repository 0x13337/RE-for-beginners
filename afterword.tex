\part*{\RU{Послесловие}\EN{Afterword}}
\addcontentsline{toc}{part}{\RU{Послесловие}\EN{Afterword}}

\chapter{\RU{Вопросы?}\EN{Questions?}}

\RU{Совершенно по любым вопросам вы можете не раздумывая писать автору}%
\EN{Do not hesitate to mail any questions to the author}: \TT{<\EMAIL>}

\EN{Any suggestions what also should be added to my book?}
\RU{Есть идеи о том, что ещё можно добавить в эту книгу?}

\EN{There is also a support forum, you can ask any questions there}%
\RU{Есть также форум поддержки, вы можете задавать там абсолютно любые вопросы}:\\
\begin{center}
\url{http://go.yurichev.com/17010}
\end{center}
 
\RU{Пожалуйста, присылайте мне информацию о замеченных ошибках (включая грамматические),}
\EN{Please, do not hesitate to send me any corrections (including grammar (you see how horrible my English is?)),}\etc.\\
\\
\RU{Я много работаю над книгой, поэтому номера страниц, листингов, \etc. очень часто меняются.}
\EN{I'm working on the book a lot, so the page and listing numbers, \etc. are changing very rapidly.}
\RU{Пожалуйста, в своих письмах мне не ссылайтесь на номера страниц и листингов.}
\EN{Please, do not refer to page and listing numbers in your emails to me.}
\RU{Есть метод проще: сделайте скриншот страницы, затем в графическом редакторе подчеркните место, где вы видите
ошибку, и отправьте мне. Так я исправлю её намного быстрее.}
\EN{There is a much simpler method: make a screenshot of the page, in a graphics editor underline the place where you see the error,
and send it to me. I'll fix it much faster.}
\RU{Ну а если вы знакомы с git и \LaTeX, вы можете исправить ошибку прямо в исходных текстах:}\EN{And if you familiar with git and \LaTeX\, you can fix the error right in the source code:}\\
\href{http://go.yurichev.com/17089}{GitHub}.