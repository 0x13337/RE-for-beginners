\part*{\RU{Послесловие}\EN{Afterword}}
\addcontentsline{toc}{part}{\RU{Послесловие}\EN{Afterword}}

\chapter{\RU{Вопросы?}\EN{Questions?}}

\RU{Совершенно по любым вопросам, вы можете не раздумывая писать автору}
\EN{Do not hesitate to mail any questions to the author}: \TT{<\EMAIL>}

\EN{There is also supporting forum, you may ask any questions there}
\RU{Есть также форум поддержки, вы можете задавать там абсолютно любые вопросы}:\\
\begin{center}
\url{http://forum.yurichev.com/}
\end{center}
 
\RU{Пожалуйста, присылайте мне информацию о замеченных ошибках 
(включая грамматические), и т.д.}
\EN{Please, also do not hesitate to send me any corrections 
(including grammar ones (you see how horrible my English is?)), etc.}\\
\\
\RU{Я много работаю над книгой, поэтому номера страниц, листингов, итд, очень часто меняются.}
\EN{I'm working on book a lot, so page, listings numbers, etc, are changing very often.}
\RU{Пожалуйста, в своих письмах мне не ссылайтесь на номера страниц и листингов.}
\EN{Please, do not refer to page/listing numbers in your emails to me.}
\RU{Есть метод проще: сделайте скриншот страницы, затем в графическом редакторе подчеркните место, где вы видите
ошибку, и отправьте мне. Так я исправлю её намного быстрее.}
\EN{There is much simpler method: just make page screenshot, then underline a place in a graphics editor,
where you see error and send  me it. I'll fix it much faster in this manner.}
\RU{Ну а если вы знакомы с git и \LaTeX\, вы можете исправить ошибку прямо в исходных текстах:}\EN{And if you familiar with git and \LaTeX\, you can fix error right in source code:}\\
\url{https://github.com/dennis714/RE-for-beginners}.