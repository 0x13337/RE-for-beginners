\chapterold{ARM}
\myindex{ARM}

\sectionold{\RU{Терминология}\EN{Terminology}}

ARM \RU{изначально разрабатывался как 32-битный}\EN{was initially developed as 32-bit} \ac{CPU}, 
\RU{поэтому \IT{слово} здесь, в отличие от x86, 32-битное}\EN{so that's why a \IT{word} here, unlike x86, is 32-bit}.

\begin{description}
	\item[byte] 8-\bitENRU.
		\RU{Для определения переменных и массива байт используется директива ассемблера DCB}
		\EN{The DB assembly directive is used for defining variables and arrays of bytes}.
	\item[halfword] 16-\bitENRU. \RU{\dittoclosing директива ассемблера DCW}
					\EN{DCW assembly directive \dittoclosing}.	
	\item[word] 32-\bitENRU. \RU{\dittoclosing директива ассемблера DCD}
					\EN{DCD assembly directive \dittoclosing}.
	\item[doubleword] 64-\bitENRU.
	\item[quadword] 128-\bitENRU.
\end{description}

\sectionold{\RU{Версии}\EN{Versions}}

\begin{itemize}
\item ARMv4: \RU{появился режим Thumb}\EN{Thumb mode introduced}.

\item ARMv6: \RU{использовался в}\EN{used in} iPhone 1st gen., iPhone 3G 
(Samsung 32-bit RISC ARM 1176JZ(F)-S \RU{поддерживающий}\EN{that supports} Thumb-2)

\item ARMv7: \RU{появился }Thumb-2\EN{ was added} (2003).
\RU{Использовался в}\EN{was used in} iPhone 3GS, iPhone 4, iPad 1st gen. (ARM Cortex-A8), iPad 2 (Cortex-A9),
iPad 3rd gen.

\item ARMv7s: \RU{Добавлены новые инструкции}\EN{New instructions added}.
\RU{Использовался в}\EN{Was used in} iPhone 5, iPhone 5c, iPad 4th gen. (Apple A6).

\item ARMv8: 64-\RU{битный процессор}\EN{bit CPU}, \ac{AKA} ARM64 \ac{AKA} AArch64.
\RU{Использовался в}\EN{Was used in} iPhone 5S, iPad Air (Apple A7).
\RU{В 64-битном режиме, режима Thumb больше нет, только режим ARM (4-байтные инструкции).}
\EN{There is no Thumb mode in 64-bit mode, only ARM (4-byte instructions).}
\end{itemize}

% sections
\section{32-\RU{битный}\EN{bit} ARM (AArch32)}

\subsection{\RU{Регистры общего пользования}\EN{General purpose registers}}

\begin{itemize}
\index{ARM!\Registers!R0}
	\item R0\EMDASH{}\RU{результат функции обычно возвращается через R0}
		\EN{function result is usually returned using R0}
	\item R1...R12\EMDASH{}\ac{GPR}s
	\item R13\EMDASH{}\ac{AKA} SP (\gls{stack pointer})
\index{ARM!\Registers!Link Register}
	\item R14\EMDASH{}\ac{AKA} LR (\gls{link register})
	\item R15\EMDASH{}\ac{AKA} PC (program counter)
\end{itemize}

\index{ARM!\Registers!scratch registers}
\Reg{0}-\Reg{3} \RU{называются также \q{scratch registers}: аргументы функции обычно передаются через них,
и эти значения не обязательно восстанавливать перед выходом из функции}
\EN{are also called \q{scratch registers}: the function's arguments are usually passed in them,
and the values in them are not required to be restored upon the function's exit}.

\subsection{Current Program Status Register (CPSR)}

\begin{center}
\begin{tabular}{ | l | l | }
\hline
\headercolor\ \RU{Бит}\EN{Bit} &
\headercolor\ \RU{Описание}\EN{Description} \\
\hline
0..4           & M\EMDASH{}processor mode \\
\hline
5              & T\EMDASH{}Thumb state \\
\hline
6              & F\EMDASH{}FIQ disable \\
\hline
7              & I\EMDASH{}IRQ disable \\
\hline
8              & A\EMDASH{}imprecise data abort disable \\
\hline
9              & E\EMDASH{}data endianness \\
\hline
10..15, 25, 26 & IT\EMDASH{}if-then state \\
\hline
16..19         & GE\EMDASH{}greater-than-or-equal-to \\
\hline
20..23         & DNM\EMDASH{}do not modify \\
\hline
24             & J\EMDASH{}Java state \\
\hline
27             & Q\EMDASH{}sticky overflow \\
\hline
28             & V\EMDASH{}overflow \\
\hline
29             & C\EMDASH{}carry/borrow/extend \\
\hline
\index{ARM!\Registers!Z}
30             & Z\EMDASH{}zero bit \\
\hline
31             & N\EMDASH{}negative/less than \\
\hline
\end{tabular}
\end{center}

% TODO
% \index{ARM!\Registers!APSR}
% \subsection{Application Program Status Register (APSR)}

% TODO
% \index{ARM!\Registers!FPSCR}
% \subsection{Floating-Point Status and Control Register (FPPSR)}
% http://infocenter.arm.com/help/index.jsp?topic=/com.arm.doc.ddi0344b/Chdfafia.html

\subsection{\RU{Регистры VPF (для чисел с плавающей точкой) и NEON}
\EN{VFP (floating point) and NEON registers}}
\label{ARM_VFP_registers}

% http://infocenter.arm.com/help/index.jsp?topic=/com.arm.doc.dht0002a/ch01s03s02.html

\index{ARM!D-\registers{}}
\index{ARM!S-\registers{}}
\begin{center}
\begin{tabular}{ | l | l | l | l | }
\hline
0..31\textsuperscript{bits} & 32..64 & 65..96 & 97..127 \\
\hline
\multicolumn{4}{ | c | }{Q0\textsuperscript{128 bits}} \\
\hline
\multicolumn{2}{ | c | }{D0\textsuperscript{64 bits}} & \multicolumn{2}{ c | }{D1} \\
\hline
S0\textsuperscript{32 bits} & S1 & S2 & S3 \\
\hline
\end{tabular}
\end{center}

\RU{S-регистры 32-битные, используются для хранения чисел с одинарной точностью}
\EN{S-registers are 32-bit, used for the storage of single precision numbers}.

\RU{D-регистры 64-битные, используются для хранения чисел с двойной точностью}
\EN{D-registers are 64-bit ones, used for the storage of double precision numbers}.

\RU{D- и S-регистры занимают одно и то же место в памяти CPU\EMDASH{}
можно обращаться к D-регистрам через S-регистры (хотя это и бессмысленно)}
\EN{D- and S-registers share the same physical space in the CPU\EMDASH{}it is possible to access 
a D-register via the S-registers (it is senseless though)}.

\RU{Точно также, \gls{NEON} Q-регистры имеют размер 128 бит и занимают то же физическое место 
в памяти CPU что и остальные регистры, предназначенные для чисел с плавающей точкой}
\EN{Likewise, the \gls{NEON} Q-registers are 128-bit ones and share the same physical space in the CPU 
with the other floating point registers}.

\RU{В VFP присутствует 32 S-регистров: S0..S31}
\EN{In VFP 32 S-registers are present: S0..S31}.

\RU{В VPFv2 были добавлены 16 D-регистров, которые занимают то же место что и S0..S31}
\EN{In VFPv2 there 16 D-registers are added, which in fact occupy the same space as S0..S31}.

\RU{В}\EN{In} VFPv3 (\gls{NEON} \OrENRU \q{Advanced SIMD}) 
\RU{добавили еще 16 D-регистров, в итоге это D0..D31, но регистры D16..D31 не делят место
с другими S-регистрами}
\EN{there are 16 more D-registers, D0..D31, but the D16..D31 registers are not 
sharing space with any other S-registers}.

\RU{В}\EN{In} \gls{NEON} \OrENRU \q{Advanced SIMD} \RU{были добавлены также 16 128-битных Q-регистров,
делящих место с регистрами D0..D31}
\EN{another 16 128-bit Q-registers were added, which share the same space as D0..D31}.

\section{64-\RU{битный}\EN{bit} ARM (AArch64)}

\subsection{\RU{Регистры общего пользования}\EN{General purpose registers}}
\label{ARM64_GPRs}

\RU{Количество регистров было удвоено со времен}\EN{The register count was doubled since} AArch32.

\begin{itemize}
\index{ARM!\Registers!X0}
	\item X0\EMDASH{}\RU{результат функции обычно возвращается через X0}
		\EN{function result is usually returned using X0}
        \item X0...X7\EMDASH{}\RU{Здесь передаются аргументы функции}\EN{Function arguments are passed here}.
	\item X8
	\item X9...X15\EMDASH{}\RU{временные регистры, вызываемая функция может их использовать и не восстанавливать 
их}\EN{are temporary registers, the callee function can use and not restore them}.
	\item X16
	\item X17
	\item X18
	\item X19...X29\EMDASH{}\RU{вызываемая функция может их использовать, но должна восстанавливать их по 
завершению}\EN{callee function can use them, but must restore them upon exit}.
	\item X29\EMDASH{}\EN{used as}\RU{используется как} \ac{FP} (\EN{at least}\RU{как минимум в} GCC)
	\item X30\EMDASH{}\q{Procedure Link Register} \ac{AKA} \ac{LR} (\gls{link register}).
	\item X31\EMDASH{}\EN{register always contains zero}\RU{регистр, всегда содержащий ноль}
\ac{AKA} XZR \OrENRU \q{Zero Register}. \RU{Его 32-битная часть называется}\EN{It's 32-bit part is called} WZR.
	\item \ac{SP}, \RU{больше не регистр общего пользования}\EN{not a general purpose register anymore}.
\end{itemize}

\RU{См.также}\EN{See also}: \cite{ARM64_PCS}.

\EN{The 32-bit part of each X-register is also accessible via W-registers (W0, W1, \etc{}).}
\RU{32-битная часть каждого X-регистра также доступна как W-регистр (W0, W1, \etc{}.).}

\begin{center}
\begin{tabular}{ | l | l | }
\hline
\RU{Старшие 32 бита}\EN{High 32-bit part}\ES{Parte alta de 32 bits}\PTBRph{}\PLph{}\ITAph{}\DEph{}\THAph{} & \RU{младшие 32 бита}\EN{low 32-bit part}\ES{parte baja de 32 bits}\PTBRph{}\PLph{}\ITAph{}\DEph{}\THAph{} \\
\hline
\multicolumn{2}{ | c | }{X0} \\
\hline
\multicolumn{1}{ | c | }{} & \multicolumn{1}{ c | }{W0} \\
\hline
\end{tabular}
\end{center}


\section{\RU{Инструкции}\EN{Instructions}}

\RU{В ARM имеется также для некоторых инструкций суффикс \IT{-S}, указывающий, 
что эта инструкция будет модифицировать флаги, а при отсутствии суффикса ~--- не будет.}
\EN{There is \IT{-S} suffix for some instructions in ARM,
indicating the instruction will set the flags according to the result, and without 
it~---the flags will not be touched.}
\index{ARM!\Instructions!ADD}
\index{ARM!\Instructions!ADDS}
\index{ARM!\Instructions!CMP}
\RU{Например, инструкция}\EN{For example} \TT{ADD} \RU{в отличие от}\EN{unlike} \TT{ADDS}
\RU{сложит два числа, но флаги не изменит}
\EN{will add two numbers, but flags will not be touched}.
\RU{Такие инструкции удобно использовать
между \CMP где выставляются флаги и, например, инструкциями перехода, где флаги используются.}
\EN{Such instructions are convenient to use between \CMP where flags are set and, 
e.g. conditional jumps, where flags are used.}

% ADD
% ADDAL
% ADDCC
% ADDS
% ADR
% ADREQ
% ADRGT
% ADRHI
% ADRNE
% ASRS
% B
% BCS
% BEQ
% BGE
% BIC
% BL
% BLE
% BLEQ
% BLGT
% BLHI
% BLS
% BLT
% BLX
% BNE
% BX
% CMP
% IDIV
% IT
% LDMCSFD
% LDMEA
% LDMED
% LDMFA
% LDMFD
% LDMGEFD
% LDR.W
% LDR
% LDRB.W
% LDRB
% LDRSB
% LSL.W
% LSL
% LSLS
% MLA
% MOV
% MOVT.W
% MOVT
% MOVW
% MULS
% MVNS
% ORR
% POP
% PUSH
% RSB
% SMMUL
% STMEA
% STMED
% STMFA
% STMFD
% STMIA
% STMIB
% STR
% SUB
% SUBEQ
% SXTB
% TEST
% TST
% VADD
% VDIV
% VLDR
% VMOV
% VMOVGT
% VMRS
% VMUL
%\index{ARM!Optional operators!ASR
%\index{ARM!Optional operators!LSL
%\index{ARM!Optional operators!LSR
%\index{ARM!Optional operators!ROR
%\index{ARM!Optional operators!RRX

% AArch64
% RET is BR X30 or BR LR but with additional hint to CPU

\ifx\RUSSIAN\undefined
\subsection{Conditional codes table}

\begin{center}
\begin{tabular}{ | l | l | l | }
\hline
\cellcolor{blue!25} Code & \cellcolor{blue!25} Description & \cellcolor{blue!25} Flags \\
\hline
EQ & Equal & Z == 1 \\
\hline
NE & Not equal & Z == 0 \\
\hline
CS \ac{AKA} HS (unsigned higher or same) & Carry set / Greater than, equal & C == 1 \\
\hline
CC \ac{AKA} LO (unsigned lower) & Carry clear / Less than & C == 0 \\
\hline
MI & Minus, negative / Less than & N == 1 \\
\hline
PL & Plus, positive or zero / Greater than, equal & N == 0 \\
\hline
VS & Overflow & V == 1 \\
\hline
VC & No overflow & V == 0 \\
\hline
HI & Unsigned higher / Greater than, or unordered & C == 1 and Z == 0 \\
\hline
LS & Unsigned lower or same / Less than or equal & C == 0 or Z == 1 \\
\hline
GE & Signed greater than or equal / Greater than or equal & N == V \\
\hline
LT & Signed less than / Less than & N != V \\
\hline
GT & Signed greater than / Greater than & Z == 0 and N == V \\
\hline
LE & Signed less than or equal / Less than, equal & Z == 1 or N != V \\
\hline
None / AL & Always & Any \\
\hline
\end{tabular}
\end{center}
\fi

