% to be proofreaded
\subsection{\IFRU{Регистры общего пользования}{General purpose registers}}

\begin{itemize}
\index{ARM!\Registers!R0}
	\item R0 --- \IFRU{результат ф-ции обычно возвращается через R0}
		{function result is usually returned using R0}
	\item R1
	\item R2
	\item R3
	\item R4
	\item R5
	\item R6
	\item R7
	\item R8
	\item R9
	\item R10
	\item R11
	\item R12
	\item R13 --- \ac{AKA} SP (\gls{stack pointer})
\index{ARM!\Registers!Link Register}
	\item R14 --- \ac{AKA} LR (\gls{link register})
	\item R15 --- \ac{AKA} PC (program counter)
\end{itemize}

\index{ARM!\Registers!scratch registers}
\IFRU{R0-R3 называются также ``scratch registers'': аргументы ф-ции обычно передаются через них,
и эти значения не обязательно восстанавливать перед выходом из ф-ции}
{R0-R3 are also called ``scratch registers'': function arguments are usually passed in them,
and values in them are not necessary to restore upon function exit}.

\subsection{Current Program Status Register (CPSR)}

\begin{center}
\begin{tabular}{ | l | l | }
\hline
\headercolor{} \IFRU{Бит}{Bit} &
\headercolor{} \IFRU{Описание}{Description} \\
\hline
0..4           & M --- processor mode \\
\hline
5              & T --- Thumb state \\
\hline
6              & F --- FIQ disable \\
\hline
7              & I --- IRQ disable \\
\hline
8              & A --- imprecise data abort disable \\
\hline
9              & E --- data endianness \\
\hline
10..15, 25, 26 & IT --- if-then state \\
\hline
16..19         & GE --- greater-than-or-equal-to \\
\hline
20..23         & DNM --- do not modify \\
\hline
24             & J --- Java state \\
\hline
27             & Q --- sticky overflow \\
\hline
28             & V --- overflow \\
\hline
29             & C --- carry/borrow/extend \\
\hline
\index{ARM!\Registers!Z}
30             & Z --- zero bit \\
\hline
31             & N --- negative/less than \\
\hline
\end{tabular}
\end{center}

% TODO
% \index{ARM!\Registers!APSR}
% \subsection{Application Program Status Register (APSR)}

% TODO
% \index{ARM!\Registers!FPSCR}
% \subsection{Floating-Point Status and Control Register (FPPSR)}
% http://infocenter.arm.com/help/index.jsp?topic=/com.arm.doc.ddi0344b/Chdfafia.html

\subsection{\IFRU{Регистры VPF (для чисел с плавающей точкой) и NEON}
{VFP (floating point) and NEON registers}}

% http://infocenter.arm.com/help/index.jsp?topic=/com.arm.doc.dht0002a/ch01s03s02.html

\index{ARM!D-\IFRU{регистры}{registers}}
\index{ARM!S-\IFRU{регистры}{registers}}
\begin{center}
\begin{tabular}{ | l | l | l | l | }
\hline
0..31\textsuperscript{bits} & 32..64 & 65..96 & 97..127 \\
\hline
\multicolumn{4}{ | c | }{Q0\textsuperscript{128 bits}} \\
\hline
\multicolumn{2}{ | c | }{D0\textsuperscript{64 bits}} & \multicolumn{2}{ c | }{D1} \\
\hline
S0\textsuperscript{32 bits} & S1 & S2 & S3 \\
\hline
\end{tabular}
\end{center}

\IFRU{S-регистры 32-битные, используются для хранения чисел с одинарной точностью}
{S-registers are 32-bit ones, used for single precision numbers storage}.

\IFRU{D-регистры 64-битные, используются для хранения чисел с двойной точностью}
{D-registers are 64-bit ones, used for double precision numbers storage}.

\IFRU{D- и S-регистры занимают одно и то же место в памяти CPU --- 
можно обращаться к D-регистрам через S-регистры (хотя это и бессмысленно)}
{D- and S-registers share the same physical space in CPU---it is possible to access 
D-register via S-registers (it is senseless though)}.

\IFRU{Точно также, \gls{NEON} Q-регистры имеют размер 128 бит и занимают то же физическое место 
в памяти CPU что и остальные регистры предназначенные для чисел с плавающей точкой}
{Likewise, \gls{NEON} Q-registers are 128-bit ones and share the same physical space in CPU 
with other floating point registers}.

\IFRU{В VFP присутствует 32 S-регистров: S0..S31}
{In VFP 32 S-registers are present: S0..S31}.

\IFRU{В VPFv2 были добавлены 16 D-регистров, которые занимают то же место что и S0..S31}
{In VFPv2 there are 16 D-registers added, which are, in fact, occupy the same space as S0..S31}.

\IFRU{В}{In} VFPv3 (\gls{NEON} \OrENRU ``Advanced SIMD'') 
\IFRU{добавили еще 16 D-регистров, в итоге это D0..D31, но регистры D16..D31 не делят место
с другими S-регистрами}
{there are 16 more D-registers added, resulting D0..D31, but D16..D31 registers are not 
sharing a space with other S-registers}.

\IFRU{В}{In} \gls{NEON} \OrENRU ``Advanced SIMD'' \IFRU{были добавлены также 16 128-битных Q-регистров,
делящих место с регистрами D0..D31}
{there are also 16 128-bit Q-registers added, which share the same space as D0..D31}.

