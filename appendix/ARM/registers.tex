\section{32-\RU{битный}\EN{bit} ARM (AArch32)}

\subsection{\RU{Регистры общего пользования}\EN{General purpose registers}}

\begin{itemize}
\index{ARM!\Registers!R0}
	\item R0 --- \RU{результат ф-ции обычно возвращается через R0}
		\EN{function result is usually returned using R0}
	\item R1...R12\EMDASH{}\ac{GPR}s
	\item R13 --- \ac{AKA} SP (\gls{stack pointer})
\index{ARM!\Registers!Link Register}
	\item R14 --- \ac{AKA} LR (\gls{link register})
	\item R15 --- \ac{AKA} PC (program counter)
\end{itemize}

\index{ARM!\Registers!scratch registers}
\Reg{0}-\Reg{3} \RU{называются также ``scratch registers'': аргументы ф-ции обычно передаются через них,
и эти значения не обязательно восстанавливать перед выходом из ф-ции}
\EN{are also called ``scratch registers'': function arguments are usually passed in them,
and values in them are not necessary to restore upon function exit}.

\subsection{Current Program Status Register (CPSR)}

\begin{center}
\begin{tabular}{ | l | l | }
\hline
\headercolor\ \RU{Бит}\EN{Bit} &
\headercolor\ \RU{Описание}\EN{Description} \\
\hline
0..4           & M --- processor mode \\
\hline
5              & T --- Thumb state \\
\hline
6              & F --- FIQ disable \\
\hline
7              & I --- IRQ disable \\
\hline
8              & A --- imprecise data abort disable \\
\hline
9              & E --- data endianness \\
\hline
10..15, 25, 26 & IT --- if-then state \\
\hline
16..19         & GE --- greater-than-or-equal-to \\
\hline
20..23         & DNM --- do not modify \\
\hline
24             & J --- Java state \\
\hline
27             & Q --- sticky overflow \\
\hline
28             & V --- overflow \\
\hline
29             & C --- carry/borrow/extend \\
\hline
\index{ARM!\Registers!Z}
30             & Z --- zero bit \\
\hline
31             & N --- negative/less than \\
\hline
\end{tabular}
\end{center}

% TODO
% \index{ARM!\Registers!APSR}
% \subsection{Application Program Status Register (APSR)}

% TODO
% \index{ARM!\Registers!FPSCR}
% \subsection{Floating-Point Status and Control Register (FPPSR)}
% http://infocenter.arm.com/help/index.jsp?topic=/com.arm.doc.ddi0344b/Chdfafia.html

\subsection{\RU{Регистры VPF (для чисел с плавающей точкой) и NEON}
\EN{VFP (floating point) and NEON registers}}

% http://infocenter.arm.com/help/index.jsp?topic=/com.arm.doc.dht0002a/ch01s03s02.html

\index{ARM!D-\registers{}}
\index{ARM!S-\registers{}}
\begin{center}
\begin{tabular}{ | l | l | l | l | }
\hline
0..31\textsuperscript{bits} & 32..64 & 65..96 & 97..127 \\
\hline
\multicolumn{4}{ | c | }{Q0\textsuperscript{128 bits}} \\
\hline
\multicolumn{2}{ | c | }{D0\textsuperscript{64 bits}} & \multicolumn{2}{ c | }{D1} \\
\hline
S0\textsuperscript{32 bits} & S1 & S2 & S3 \\
\hline
\end{tabular}
\end{center}

\RU{S-регистры 32-битные, используются для хранения чисел с одинарной точностью}
\EN{S-registers are 32-bit ones, used for single precision numbers storage}.

\RU{D-регистры 64-битные, используются для хранения чисел с двойной точностью}
\EN{D-registers are 64-bit ones, used for double precision numbers storage}.

\RU{D- и S-регистры занимают одно и то же место в памяти CPU --- 
можно обращаться к D-регистрам через S-регистры (хотя это и бессмысленно)}
\EN{D- and S-registers share the same physical space in CPU---it is possible to access 
D-register via S-registers (it is senseless though)}.

\RU{Точно также, \gls{NEON} Q-регистры имеют размер 128 бит и занимают то же физическое место 
в памяти CPU что и остальные регистры, предназначенные для чисел с плавающей точкой}
\EN{Likewise, \gls{NEON} Q-registers are 128-bit ones and share the same physical space in CPU 
with other floating point registers}.

\RU{В VFP присутствует 32 S-регистров: S0..S31}
\EN{In VFP 32 S-registers are present: S0..S31}.

\RU{В VPFv2 были добавлены 16 D-регистров, которые занимают то же место что и S0..S31}
\EN{In VFPv2 there are 16 D-registers added, which are, in fact, occupy the same space as S0..S31}.

\RU{В}\EN{In} VFPv3 (\gls{NEON} \OrENRU ``Advanced SIMD'') 
\RU{добавили еще 16 D-регистров, в итоге это D0..D31, но регистры D16..D31 не делят место
с другими S-регистрами}
\EN{there are 16 more D-registers added, resulting D0..D31, but D16..D31 registers are not 
sharing a space with other S-registers}.

\RU{В}\EN{In} \gls{NEON} \OrENRU ``Advanced SIMD'' \RU{были добавлены также 16 128-битных Q-регистров,
делящих место с регистрами D0..D31}
\EN{there are also 16 128-bit Q-registers added, which share the same space as D0..D31}.

\section{64-\RU{битный}\EN{bit} ARM (AArch64)}

\subsection{\RU{Регистры общего пользования}\EN{General purpose registers}}
\label{ARM64_GPRs}

\RU{Количество регистров было удвоено со времен}\EN{Register count was doubled since} AArch32.

\begin{itemize}
\index{ARM!\Registers!X0}
	\item X0\EMDASH{}\RU{результат ф-ции обычно возвращается через X0}
		\EN{function result is usually returned using X0}
        \item X0...X7\EMDASH{}\RU{Здесь передаются аргументы ф-ции}\EN{Function arguments are passed here}.
	\item X8
	\item X9...X15\EMDASH{}\RU{временные регистры, вызываемая ф-ция может их использовать и не восстанавливать 
их}\EN{are temporary registers, callee function may use it and not restore}.
	\item X16
	\item X17
	\item X18
	\item X19...X29\EMDASH{}\RU{вызываемая ф-ция может их использовать, но должна восстанавливать их по 
завершению}\EN{callee function may use, but should restore them upon exit}.
	\item X29\EMDASH{}\EN{used as}\RU{используется как} \ac{FP} (\EN{at least}\RU{как минимум в} GCC)
	\item X30\EMDASH{}``Procedure Link Register'' \ac{AKA} \ac{LR} (\gls{link register}).
	\item X31\EMDASH{}\EN{register always containing zero}\RU{регистр, всегда содержащий ноль}
\ac{AKA} XZR \OrENRU ``Zero Register''. \RU{Его 32-битная часть называется}\EN{It's 32-bit part called} WZR.
	\item \ac{SP}, \RU{больше не регистр общего пользования}\EN{not general register anymore}.
\end{itemize}

\RU{См.также}\EN{See also}: \cite{ARM64_PCS}.

\EN{32-bit part of each X-register is also accessible via W-registers}\RU{32-битная часть 
каждого X-регистра также доступна как W-регистр} (W0, W1, \RU{и т.д.}\EN{etc}).

\begin{center}
\begin{tabular}{ | l | l | }
\hline
\RU{Старшие 32 бита}\EN{High 32-bit part}\ES{Parte alta de 32 bits}\PTBRph{}\PLph{}\ITAph{}\DEph{}\THAph{} & \RU{младшие 32 бита}\EN{low 32-bit part}\ES{parte baja de 32 bits}\PTBRph{}\PLph{}\ITAph{}\DEph{}\THAph{} \\
\hline
\multicolumn{2}{ | c | }{X0} \\
\hline
\multicolumn{1}{ | c | }{} & \multicolumn{1}{ c | }{W0} \\
\hline
\end{tabular}
\end{center}

