\section{GCC}
\index{GCC}

\IFRU{Некоторые полезные опции, которые я использовал в книге}
{Some useful options I used through this book}.

\begin{center}
\begin{tabular}{ | l | l | }
\hline
\cellcolor{blue!25} \IFRU{опция}{option} & 
\cellcolor{blue!25} \IFRU{значение}{meaning} \\
\hline
-Os		& \IFRU{оптимизация по размеру кода}{code size optimization} \\
-O3		& \IFRU{максимальная оптимизация}{maximum optimization} \\
-regparm=	& \IFRU{как много аргументов будет передаваться через регистры}
			{how many arguments will be passed in registers} \\
-o file		& \IFRU{задать имя выходного файла}{set name of output file} \\
-g		& \IFRU{генерировать отладочную информацию в итоговом исполняемом файле}
			{produce debugging information in resulting executable} \\
-S		& \IFRU{генерировать листинг на ассемблере}
			{generate assembly listing file} \\
-masm=intel	& \IFRU{генерировать листинг в Intel-синтаксисе}{produce listing in Intel syntax} \\
\hline
\end{tabular}
\end{center}


