\section{GCC}
\index{GCC}

\RU{Некоторые полезные опции, которые я использовал в книге}
\EN{Some useful options I used through this book}.

\begin{center}
\begin{tabular}{ | l | l | }
\hline
\cellcolor{blue!25} \RU{опция}\EN{option} & 
\cellcolor{blue!25} \RU{значение}\EN{meaning} \\
\hline
-Os		& \RU{оптимизация по размеру кода}\EN{code size optimization} \\
-O3		& \RU{максимальная оптимизация}\EN{maximum optimization} \\
-regparm=	& \RU{как много аргументов будет передаваться через регистры}
			\EN{how many arguments are to be passed in registers} \\
-o file		& \RU{задать имя выходного файла}\EN{set name of output file} \\
-g		& \RU{генерировать отладочную информацию в итоговом исполняемом файле}
			\EN{produce debugging information in resulting executable} \\
-S		& \RU{генерировать листинг на ассемблере}
			\EN{generate assembly listing file} \\
-masm=intel	& \RU{генерировать листинг в синтаксисе Intel}\EN{produce listing in Intel syntax} \\
-fno-inline	& \RU{не вставлять тело функции там, где она вызывается}\EN{do not inline functions} \\
\hline
\end{tabular}
\end{center}


