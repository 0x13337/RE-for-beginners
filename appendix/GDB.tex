\section{GDB}
\index{GDB}
\label{sec:GDB_cheatsheet}

\IFRU{Некоторые команды, которые я использовал в книге}{Some of commands I used in this book}:

\begin{center}
\begin{tabular}{ | l | l | }
\hline
\cellcolor{blue!25} \IFRU{опция}{option} & 
\cellcolor{blue!25} \IFRU{значение}{meaning} \\
\hline
break filename.c:number		& \IFRU{установить брякпойнт на номере строки в исходном файле}
					{set a breakpoint on line number in source code} \\
break function			& \IFRU{установить брякпойнт на ф-ции}{set a breakpoint on function} \\
break *address			& \IFRU{установить брякпойнт на адресе}{set a breakpoint on address} \\
b				& \dittoclosing \\
p variable			& \IFRU{вывести значение переменной}{print value of variable} \\
run				& \IFRU{запустить}{run} \\
r				& \dittoclosing \\
cont				& \IFRU{продолжить исполнение}{continue execution} \\
c				& \dittoclosing \\
bt				& \IFRU{вывести стек}{print stack} \\
set disassembly-flavor intel	& \IFRU{установить Intel-синтаксис}{set Intel syntax} \\
disas				& disassemble current function \\
disas function			& \IFRU{дизассемблировать ф-цию}{disassemble function} \\
disas function,+50		& disassemble portion \\
disas \$eip,+0x10		& \dittoclosing \\
info registers			& \IFRU{вывести все регистры}{print all registers} \\
info locals			& \IFRU{вывести локальные переменные (если известны)}{dump local variables (if known)} \\
x/w ...				& \IFRU{вывести память как 32-битные слова}{dump memory as 32-bit word} \\
x/10w ...			& \IFRU{вывести 10 слов памяти}{dump 10 memory words} \\
x/s ...				& \IFRU{вывести строку из памяти}{dump memory as string} \\
x/i ...				& \IFRU{трактовать память как код}{dump memory as code} \\
x/10c ...			& \IFRU{вывести 10 символов}{dump 10 characters} \\
x/b ...				& \IFRU{вывести байты}{dump bytes} \\
x/h ...				& \IFRU{вывести 16-битные полуслова}{dump 16-bit halfwords} \\
x/g ...				& \IFRU{вывести 64-битные слова}{dump giant (64-bit) words} \\
finish				& \IFRU{исполнять до конца ф-ции}{execute till the end of function} \\
next				& \IFRU{следующая инструкция (не заходить в ф-ции)}
					{next instruction (don't dive into functions)} \\
step				& \IFRU{следующая инструкция (заодить в ф-ции)}
					{next instruction (dive into functions)} \\
frame n				& \IFRU{переключить фрейм стека}{switch stack frame} \\
info break			& \IFRU{список брякпойнтов}{list of breakpoints} \\
del n				& \IFRU{удалить брякпойнт}{delete breakpoint} \\
\hline
\end{tabular}
\end{center}


