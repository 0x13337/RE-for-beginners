\section{IDA}
\index{IDA}
\label{sec:IDA_cheatsheet}

\ShortHotKeyCheatsheet:

\begin{center}
\begin{tabular}{ | l | l | }
\hline
\cellcolor{blue!25} \IFRU{клавиша}{key} & \cellcolor{blue!25} \IFRU{значение}{meaning} \\
\hline
Space 	& \IFRU{переключать между листингом и просмотром кода в виде графа}{switch listing and graph view} \\
C 	& \IFRU{конвертировать в код}{convert to code} \\
D 	& \IFRU{конвертировать в данные}{convert to data} \\
A 	& \IFRU{конвертировать в строку}{convert to string} \\
* 	& \IFRU{конвертировать в массив}{convert to array} \\
U 	& \IFRU{сделать неопределенным}{undefine} \\
O 	& \IFRU{сделать смещение из операнда}{make offset of operand} \\
H 	& \IFRU{сделать десятичное число}{make decimal number} \\
R 	& \IFRU{сделать символ}{make char} \\
B 	& \IFRU{сделать двоичное число}{make binary number} \\
Q 	& \IFRU{сделать шестнадцатеричное число}{make hexadecimal number} \\
N 	& \IFRU{переменовать идентификатор}{rename identificator} \\
? 	& \IFRU{калькулятор}{calculator} \\
G 	& \IFRU{переход на адрес}{jump to address} \\
: 	& \IFRU{добавить комментарий}{add comment} \\
Ctrl-X 	& \IFRU{показать ссылки на текущую ф-цию, метку, переменную (в т.ч., в стеке)}
		{show refernces to the current function, label, variable (incl. in local stack)} \\
X 	& \IFRU{показать ссылки на ф-цию, метку, переменную, итд}
		{show references to the function, label, variable, etc} \\
Alt-I 	& \IFRU{искать константу}{search for constant} \\
Ctrl-I 	& \IFRU{искать следующее вхождение константы}{search for the next occurrence of constant} \\
Alt-B 	& \IFRU{искать последовательность байт}{search for byte sequence} \\
Ctrl-B 	& \IFRU{искать следующее вхождение последовательности байт}
		{search for the next occurrence of byte sequence} \\
Alt-T 	& \IFRU{искать текст (включая инструкции, итд)}{search for text (including instructions, etc)} \\
Ctrl-T 	& \IFRU{искать следующее вхождение текста}{search for the next occurrence of text} \\
Alt-P 	& \IFRU{редактировать текущую функцию}{edit current function} \\
Enter 	& \IFRU{перейти к ф-ции, переменной, итд}{jump to function, variable, etc} \\
Esc 	& \IFRU{вернуться назад}{get back} \\
Num -   & \IFRU{свернуть ф-цию или отмеченную область}{fold function or selected area} \\
Num + 	& \IFRU{снова показать ф-цию или область}{unhide function or area}\\
\hline
\end{tabular}
\end{center}

\IFRU{Сворачивание ф-ции или области может быть удобно чтобы прятать те части ф-ции,
чья функция вам стала уже ясна}
{Function/area folding may be useful for hiding function parts when you realize what they do}.
\IFRU{это используется в моем скрипте\footnote{\url{\YurichevIDAIDCScripts}}}{this is used in my}
\IFRU{для сворачивания некоторых очень часто используемых фрагментов inline-кода}
{script\footnote{\url{\YurichevIDAIDCScripts}} for hiding some often used patterns of inline code}.

