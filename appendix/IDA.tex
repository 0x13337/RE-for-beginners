\section{IDA}
\index{IDA}
\label{sec:IDA_cheatsheet}

\ShortHotKeyCheatsheet:

\begin{center}
\begin{tabular}{ | l | l | }
\hline
\cellcolor{blue!25} \RU{клавиша}\EN{key} & \cellcolor{blue!25} \RU{значение}\EN{meaning} \\
\hline
Space 	& \RU{переключать между листингом и просмотром кода в виде графа}\EN{switch listing and graph view} \\
C 	& \RU{конвертировать в код}\EN{convert to code} \\
D 	& \RU{конвертировать в данные}\EN{convert to data} \\
A 	& \RU{конвертировать в строку}\EN{convert to string} \\
* 	& \RU{конвертировать в массив}\EN{convert to array} \\
U 	& \RU{сделать неопределенным}\EN{undefine} \\
O 	& \RU{сделать смещение из операнда}\EN{make offset of operand} \\
H 	& \RU{сделать десятичное число}\EN{make decimal number} \\
R 	& \RU{сделать символ}\EN{make char} \\
B 	& \RU{сделать двоичное число}\EN{make binary number} \\
Q 	& \RU{сделать шестнадцатеричное число}\EN{make hexadecimal number} \\
N 	& \RU{переименовать идентификатор}\EN{rename identificator} \\
? 	& \RU{калькулятор}\EN{calculator} \\
G 	& \RU{переход на адрес}\EN{jump to address} \\
: 	& \RU{добавить комментарий}\EN{add comment} \\
Ctrl-X 	& \RU{показать ссылки на текущую функцию, метку, переменную (в т.ч., в стеке)}
		\EN{show references to the current function, label, variable (incl. in local stack)} \\
X 	& \RU{показать ссылки на функцию, метку, переменную, и т.д.}
		\EN{show references to the function, label, variable, etc} \\
Alt-I 	& \RU{искать константу}\EN{search for constant} \\
Ctrl-I 	& \RU{искать следующее вхождение константы}\EN{search for the next occurrence of constant} \\
Alt-B 	& \RU{искать последовательность байт}\EN{search for byte sequence} \\
Ctrl-B 	& \RU{искать следующее вхождение последовательности байт}
		\EN{search for the next occurrence of byte sequence} \\
Alt-T 	& \RU{искать текст (включая инструкции, и т.д.)}\EN{search for text (including instructions, etc)} \\
Ctrl-T 	& \RU{искать следующее вхождение текста}\EN{search for the next occurrence of text} \\
Alt-P 	& \RU{редактировать текущую функцию}\EN{edit current function} \\
Enter 	& \RU{перейти к функции, переменной, и т.д.}\EN{jump to function, variable, etc} \\
Esc 	& \RU{вернуться назад}\EN{get back} \\
Num -   & \RU{свернуть функцию или отмеченную область}\EN{fold function or selected area} \\
Num + 	& \RU{снова показать функцию или область}\EN{unhide function or area}\\
\hline
\end{tabular}
\end{center}

\RU{Сворачивание функции или области может быть удобно чтобы прятать те части функции,
чья функция вам стала уже ясна}
\EN{Function/area folding may be useful for hiding function parts when you realize what they do}.
\RU{это используется в моем скрипте\footnote{\href{\YurichevIDAIDCScripts}{GitHub}}}\EN{this is used in my}
\RU{для сворачивания некоторых очень часто используемых фрагментов inline-кода}
\EN{script\footnote{\href{\YurichevIDAIDCScripts}{GitHub}} for hiding some often used patterns of inline code}.

