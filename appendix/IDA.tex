\subsection{IDA}
\myindex{IDA}
\label{sec:IDA_cheatsheet}

\ShortHotKeyCheatsheet:

\begin{center}
\begin{tabular}{ | l | l | }
\hline
\HeaderColor \RU{клавиша}\EN{key}\DE{Taste} & \HeaderColor \RU{значение}\EN{meaning}\DE{Bedeutung} \\
\hline
Space 	& \RU{переключать между листингом и просмотром кода в виде графа}
            \EN{switch listing and graph view}
            \DE{Zwischen Quellcode und grafischer Ansicht wechseln}\\
C 	& \RU{конвертировать в код}\EN{convert to code}\DE{zu Code konvertieren} \\
D 	& \RU{конвертировать в данные}\EN{convert to data}\DE{zu Daten konvertieren} \\
A 	& \RU{конвертировать в строку}\EN{convert to string}\DE{zu Zeichenkette konvertieren} \\
* 	& \RU{конвертировать в массив}\EN{convert to array}\DE{zu Array konvertieren} \\
U 	& \RU{сделать неопределенным}\EN{undefine}\DE{undefinieren}%not sure \\
O 	& \RU{сделать смещение из операнда}\EN{make offset of operand}\DE{Offset von Operanden}%not sure \\
H 	& \RU{сделать десятичное число}\EN{make decimal number}\DE{Dezimalzahl erstellen} \\
R 	& \RU{сделать символ}\EN{make char}\DE{Zeichen erstellen} \\
B 	& \RU{сделать двоичное число}\EN{make binary number}\DE{Binärzahl erstellen} \\
Q 	& \RU{сделать шестнадцатеричное число}\EN{make hexadecimal number}\DE{Hexadezimalzahl erstellen} \\
N 	& \RU{переименовать идентификатор}\EN{rename identificator}\DE{Identifikator umbenennen} \\
? 	& \RU{калькулятор}\EN{calculator}\DE{Rechner} \\
G 	& \RU{переход на адрес}\EN{jump to address}\DE{zu Adresse springen} \\
: 	& \RU{добавить комментарий}\EN{add comment}\DE{Kommentar einfügen} \\
Ctrl-X 	& \RU{показать ссылки на текущую функцию, метку, переменную}
		\EN{show references to the current function, label, variable }
        \DE{Referenz zu aktueller Funktion, Variable, ... zeigen}\\
	& \RU{(в т.ч., в стеке)}\EN{(incl. in local stack)}\DE{(inkl. lokalem Stack)} \\
X 	& \RU{показать ссылки на функцию, метку, переменную, итд}\EN{show references to the function, label, variable, etc.}
        \DE{Referenz zu Funktion, Variable, ... zeigen}\\
Alt-I 	& \RU{искать константу}\EN{search for constant}\DE{Konstante suchen} \\
Ctrl-I 	& \RU{искать следующее вхождение константы}\EN{search for the next occurrence of constant}\DE{Nächstes Auftreten der Konstante suchen} \\
Alt-B 	& \RU{искать последовательность байт}\EN{search for byte sequence}\DE{Byte-Sequenz suchen} \\
Ctrl-B 	& \RU{искать следующее вхождение последовательности байт}
		\EN{search for the next occurrence of byte sequence}
        \DE{Nächstes Auftreten der Byte-Sequenz suchen} \\
Alt-T 	& \RU{искать текст (включая инструкции, итд.)}\EN{search for text (including instructions, etc.)}\EN{Text suchen (inkl. Anweisungen, usw.)} \\
Ctrl-T 	& \RU{искать следующее вхождение текста}\EN{search for the next occurrence of text}\DE{nächstes Aufreten des Textes suchen} \\
Alt-P 	& \RU{редактировать текущую функцию}\EN{edit current function}\DE{akutelle Funktion editieren} \\
		Enter 	& \RU{перейти к функции, переменной, итд.}\EN{jump to function, variable, etc.}\DE{zu Funktion, Variable, ... springen} \\
Esc 	& \RU{вернуться назад}\EN{get back}\DE{zurückgehen} \\
Num -   & \RU{свернуть функцию или отмеченную область}\EN{fold function or selected area}\DE{Funktion oder markierten Bereich einklappen} \\
Num + 	& \RU{снова показать функцию или область}\EN{unhide function or area}\DE{Funktion oder Bereich anzeigen}\\
\hline
\end{tabular}
\end{center}

\RU{Сворачивание функции или области может быть удобно чтобы прятать те части функции,
чья функция вам стала уже ясна}
\EN{Function/area folding may be useful for hiding function parts when you realize what they do}.
\DE{Das Einklappen ist nützlich um Teile von Funktionen zu verstecken, wenn bekannt ist was sie tun}.
\RU{это используется в моем скрипте\footnote{\href{\YurichevIDAIDCScripts}{GitHub}}}\EN{this is used in my}\DE{dies wird genutzt im}
\RU{для сворачивания некоторых очень часто используемых фрагментов inline-кода}
\EN{script\footnote{\href{\YurichevIDAIDCScripts}{GitHub}} for hiding some often used patterns of inline code}.
\DE{Script\footnote{\href{\YurichevIDAIDCScripts}{GitHub}} um häufig genutzte Inline-Code-Stellen zu verstecken}.

