\chapter{MIPS}

\section{\Registers}
\label{MIPS_registers_ref}

\index{MIPS!O32}
( \RU{Соглашение о вызовах O32}\EN{O32 calling convention} )

\subsection{\RU{Регистры общего пользования}\EN{General purpose registers} \ac{GPR}}

\begin{center}
\begin{tabular}{ | l | l | l | }
\hline
\cellcolor{blue!25} \RU{Номер}\EN{Number} & 
\cellcolor{blue!25} \RU{Псевдоимя}\EN{Pseudoname} & 
\cellcolor{blue!25} \RU{Описание}\EN{Description} \\
\hline
\$0             & \$ZERO          & \RU{Всегда ноль. Запись в этот регистр работает как холостая инструкция}\EN{Always zero. Writing to this register is effectively idle instruction} (\ac{NOP}). \\
\hline
\$1             & \$AT            & \RU{Используется как временный регистр для ассемблерных макросов и псевдоинструкций}\EN{Used as a temporary register for assembly macros and pseudoinstructions}. \\
\hline
\$2 \dots \$3   & \$V0 \dots \$V1 & \RU{Здесь возвращается результат ф-ции}\EN{Function result returned here}. \\
\hline
\$4 \dots \$7   & \$A0 \dots \$A3 & \RU{Аргументы ф-ции}\EN{Function arguments}. \\
\hline
\$8 \dots \$15  & \$T0 \dots \$T7 & \RU{Используется для временных данных}\EN{Used for temporary data}. \\
\hline
\$16 \dots \$23 & \$S0 \dots \$S7 & \RU{Используется для временных данных}\EN{Used for temporary data}\AsteriskOne{}. \\
\hline
\$24 \dots \$25 & \$T8 \dots \$T9 & \RU{Используется для временных данных}\EN{Used for temporary data}. \\
\hline
\$26 \dots \$27 & \$K0 \dots \$K1 & \RU{Зарезервировано для ядра \ac{OS}}\EN{Reserved for \ac{OS} kernel}. \\
\hline
\$28            & \$GP            & \RU{Глобальный указатель}\EN{Global Pointer}\AsteriskTwo{}. \\
\hline
\$29            & \$SP            & \ac{SP}\AsteriskOne{}. \\
\hline
\$30            & \$FP            & \ac{FP}\AsteriskOne{}. \\
\hline
\$31            & \$RA            & \ac{RA}. \\
\hline
n/a             & PC              & \ac{PC}. \\
\hline
n/a             & HI              & \RU{старшие 32 бита результата умножения или остаток от деления}\EN{high 32 bit of multiplication or division remainder}\AsteriskThree{}. \\
\hline
n/a             & LO              & \RU{младшие 32 бита результата умножения или результат деления}\EN{low 32 bit of multiplication and division remainder}\AsteriskThree{}. \\
\hline
\end{tabular}
\end{center}

\subsection{\RU{Регистры для работы с числами в формате плавающей точки}\EN{Floating-point registers}}
\label{MIPS_FPU_registers}

\begin{center}
\begin{tabular}{ | l | l | l | }
\hline
\cellcolor{blue!25} \RU{Название}\EN{Name} & \cellcolor{blue!25} \RU{Описание}\EN{Description} \\
\hline
\$F0..\$F1   & \RU{Здесь возвращается результат ф-ции}\EN{Function result returned here}. \\
\hline
\$F2..\$F3   & \RU{Не используется}\EN{Not used}. \\
\hline
\$F4..\$F11  & \RU{Используется для временных данных}\EN{Used for temporary data}. \\
\hline
\$F12..\$F15 & \RU{Первые два аргумента ф-ции}\EN{First two function arguments}. \\
\hline
\$F16..\$F19 & \RU{Используется для временных данных}\EN{Used for temporary data}. \\
\hline
\$F20..\$F31 & \RU{Используется для временных данных}\EN{Used for temporary data}\AsteriskOne{}. \\
\hline
%fcr31 & Control/status register. \\
%\hline
\end{tabular}
\end{center}

\AsteriskOne{} --- \Gls{callee} \RU{должен сохранять}\EN{must preserve}.\\
\AsteriskTwo{} --- \Gls{callee} \RU{должен сохранять}\EN{must preserve} (\RU{кроме \ac{PIC}-кода}
\EN{except \ac{PIC} code}).\\
\index{MIPS!\Instructions!MFLO}
\index{MIPS!\Instructions!MFHI}
\AsteriskThree{} --- \RU{доступны используя инструкции}\EN{accessible using} \TT{MFHI} \AndENRU \TT{MFLO}\EN{ instructions}.\\

\section{\Instructions}

\RU{Есть три типа инструкций}\EN{There are 3 kinds of instructions}:

\begin{itemize}

\item \RU{Тип R: имеющие 3 регистра. R-инструкции обычно имеют такой вид:}
\EN{R-type: those which has 3 registers. R-instruction are usually has the following form:}

\begin{lstlisting}
instruction destination, source1, source2
\end{lstlisting}

\RU{Важно помнить что если первый и второй регистр один и тот же, IDA может показать инструкцию в сокращенной
форме:}
\EN{One important thing to remember is that when first and second register is the same, 
IDA may show instruction in shorter form:}

\begin{lstlisting}
instruction destination/source1, source2
\end{lstlisting}

\RU{Это немного напоминает Интеловский синтаксис ассемблера x86.}
\EN{That somewhat reminds us Intel-syntax of x86 assembly language.}

\item \RU{Тип I: имеющие 2 регистра и 16-битное ``immediate''-значение.}
\EN{I-type: those which has 2 registers and 16-bit immediate value.}

\item \RU{Тип J: инструкции перехода, имеют 26 бит для кодирования смещения.}
\EN{J-type: jump/branch instructions, has 26 bits for offset encoding.}

\end{itemize}

\subsection{\RU{Инструкции перехода}\EN{Jump instructions}}

\RU{Какая разница между инструкциями начинающихся с B- (BEQ, B, итд) и с J- (JAL, JALR, итд)?}
\EN{What is the difference between B- instructions (BEQ, B, etc) and J- ones (JAL, JALR, etc)?}

\RU{B-инструкции имеют тип I, так что, смещение в этих инструкциях кодируется как 16-битное значение.}
\EN{B-instructions has I-type, hence, B-instructions offset is encoded as 16-bit immediate.}
\RU{Инструкции JR и JALR имеют тип R, и они делают переход по абсолютному адресу указанному в регистре.}
\EN{JR and JALR are R-type, which jumps to absolute address specified in register.}
\RU{J и JAL имеют тип J, так что смещение кодируется как 26-битное значение.}
\EN{J and JAL are J-type, hence, offset is encoded as 26-bit immediate.}

\RU{Коротко говоря, в B-инструкциях можно кодировать условие 
(B на самом деле это псевдоинструкция для \TT{BEQ \$ZERO, \$ZERO, LABEL}),
а в J-инструкциях нельзя.}
\EN{In short, B-instructions can encode condition 
(B is in fact pseudoinstruction for \TT{BEQ \$ZERO, \$ZERO, LABEL}), 
while J-instructions can't.}
