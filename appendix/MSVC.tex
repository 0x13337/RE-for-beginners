\section{MSVC}
\index{MSVC}
\label{sec:MSVC_options}

\RU{Некоторые полезные опции, которые были использованы в книге}
\EN{Some useful options which were used through this book}.

\begin{center}
\begin{tabular}{ | l | l | }
\hline
\cellcolor{blue!25} \RU{опция}\EN{option} & 
\cellcolor{blue!25} \RU{значение}\EN{meaning} \\
\hline
/O1		& \RU{оптимизация по размеру кода}\EN{minimize space}\\
/Ob0		& \RU{не заменять вызовы inline-функций их кодом}\EN{no inline expansion}\\
/Ox		& \RU{максимальная оптимизация}\EN{maximum optimizations}\\
/GS-		& \RU{отключить проверки переполнений буфера}
		\EN{disable security checks (buffer overflows)}\\
/Fa(file)	& \RU{генерировать листинг на ассемблере}\EN{generate assembly listing}\\
/Zi		& \RU{генерировать отладочную информацию}\EN{enable debugging information}\\
/Zp(n)		& \RU{паковать структуры по границе в $n$ байт}\EN{pack structs on $n$-byte boundary}\\
/MD		& \RU{выходной исполняемый файл будет использовать}
			\EN{produced executable will use} \TT{MSVCR*.DLL}\\
\hline
\end{tabular}
\end{center}

\RU{Кое-как информация о версиях MSVC}\EN{Some information about MSVC versions}:
\myref{MSVC_versions}.
