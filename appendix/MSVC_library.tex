\chapter{\RU{Некоторые библиотечные функции MSVC}\EN{Some MSVC library functions}}
\index{MSVC}
\label{sec:MSVC_library_func}

\TT{ll} \RU{в имени функции означает}\EN{in function name means} ``long long'', \RU{т.е., 64-битный тип данных}
\EN{e.g., a 64-bit data type}.

\begin{center}
\begin{tabular}{ | l | l | }
\hline
\cellcolor{blue!25} \RU{имя}\EN{name} & \cellcolor{blue!25} \RU{значение}\EN{meaning} \\
\hline \TT{\_\_alldiv} & \RU{знаковое деление}\EN{signed division} \\
\hline \TT{\_\_allmul} & \RU{умножение}\EN{multiplication} \\
\hline \TT{\_\_allrem} & \RU{остаток от знакового деления}\EN{remainder of signed division} \\
\hline \TT{\_\_allshl} & \RU{сдвиг влево}\EN{shift left} \\
\hline \TT{\_\_allshr} & \RU{знаковый сдвиг вправо}\EN{signed shift right} \\
\hline \TT{\_\_aulldiv} & \RU{беззнаковое деление}\EN{unsigned division} \\
\hline \TT{\_\_aullrem} & \RU{остаток от беззнакового деления}\EN{remainder of unsigned division} \\
\hline \TT{\_\_aullshr} & \RU{беззнаковый сдвиг вправо}\EN{unsigned shift right} \\
\hline
\end{tabular}
\end{center}

\RU{Процедуры умножения и сдвига влево, одни и те же и для знаковых чисел, и для беззнаковых,
поэтому здесь только одна ф-ция для каждой операции}
\EN{Multiplication and shift left procedures are the same for both signed and unsigned numbers, hence there is only one function 
for each operation here}. \\
\\
\RU{Исходные коды этих ф-ций можно найти в установленной \ac{MSVS}, в}\EN{The source code of these function
can be found in the installed \ac{MSVS}, in} \TT{VC/crt/src/intel/*.asm}.

