\section{\IFRU{Некоторые библиотечные функции MSVC}{Some MSVC library functions}}
\index{MSVC}
\label{sec:MSVC_library_func}

\TT{ll} \IFRU{в имени функции означает}{in function name mean} ``long long'', \IFRU{т.е., 64-битный тип данных}
{e.g., 64-bit data type}.

\begin{center}
\begin{tabular}{ | l | l | }
\hline
\cellcolor{blue!25} \IFRU{имя}{name} & \cellcolor{blue!25} \IFRU{значение}{meaning} \\
\hline \TT{\_\_alldiv} & \IFRU{знаковое деление}{signed division} \\
\hline \TT{\_\_allmul} & \IFRU{умножение}{multiplication} \\
\hline \TT{\_\_allrem} & \IFRU{остаток от знакового деления}{remainder of signed division} \\
\hline \TT{\_\_allshl} & \IFRU{сдвиг влево}{shift left} \\
\hline \TT{\_\_allshr} & \IFRU{знаковый сдвиг вправо}{signed shift right} \\
\hline \TT{\_\_aulldiv} & \IFRU{беззнаковое деление}{unsigned division} \\
\hline \TT{\_\_aullrem} & \IFRU{остаток от беззнакового деления}{remainder of unsigned division} \\
\hline \TT{\_\_aullshr} & \IFRU{беззнаковый сдвиг вправо}{unsigned shift right} \\
\hline
\end{tabular}
\end{center}

\IFRU{Процедуры умножения и сдвига влево, одни и те же и для знаковых чисел и для беззнаковых,
поэтому здесь только одна ф-ция для каждой операции}
{Multiplication and shift left procedures are the same for both signed and unsigned numbers, hence only one function 
for each operation here}. \\
\\
\IFRU{Исходные коды этих ф-ций можно найти в установленной \ac{MSVS}, в}{The source code of these function
can be founded in the installed \ac{MSVS}, in} \TT{VC/crt/src/intel/*.asm}.

