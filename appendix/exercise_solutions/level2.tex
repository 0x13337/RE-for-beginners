\section{\IFRU{Уровень}{Level} 2}

\subsection{\Exercise 2.1}

\index{\CStandardLibrary!toupper()}
\IFRU{Решение}{Solution}: \TT{toupper()}.

\IFRU{Исходник на Си}{C source code}:

\begin{lstlisting}
char toupper ( char c )
{
    if( c >= 'a' && c <= 'z' ) {
        c = c - 'a' + 'A';
    }
    return( c );
}
\end{lstlisting}

\subsection{\Exercise 2.2}

\index{\CStandardLibrary!atoi()}
\IFRU{Ответ}{Solution}: \TT{atoi()}

\IFRU{Исходник на Си}{C source code}:

\begin{lstlisting}
#include <stdio.h>
#include <string.h>
#include <ctype.h>

int atoi ( const *p )  /* convert ASCII string to integer */
{
    int i;
    char s;

    while( isspace ( *p ) )
        ++p;
    s = *p;
    if( s == '+' || s == '-' )
        ++p;
    i = 0;
    while( isdigit(*p) ) {
        i = i * 10 + *p - '0';
        ++p;
    }
    if( s == '-' )
        i = - i;
    return( i );
}
\end{lstlisting}

\subsection{\Exercise 2.3}

\index{\CStandardLibrary!rand()}
\index{\CStandardLibrary!srand()}
\IFRU{Ответ}{Solution}: \TT{srand()} / \TT{rand()}.

\IFRU{Исходник на Си}{C source code}:

\begin{lstlisting}
static unsigned int v;

void srand (unsigned int s)
{
        v = s;
}

int rand ()
{
        return( ((v = v * 214013L
            + 2531011L) >> 16) & 0x7fff );
}
\end{lstlisting}

\subsection{\Exercise 2.4}

\index{\CStandardLibrary!strstr()}
\IFRU{Ответ}{Solution}: \TT{strstr()}.

\IFRU{Исходник на Си}{C source code}:

\begin{lstlisting}
char * strstr (
        const char * str1,
        const char * str2
        )
{
        char *cp = (char *) str1;
        char *s1, *s2;

        if ( !*str2 )
            return((char *)str1);

        while (*cp)
        {
                s1 = cp;
                s2 = (char *) str2;

                while ( *s1 && *s2 && !(*s1-*s2) )
                        s1++, s2++;

                if (!*s2)
                        return(cp);

                cp++;
        }

        return(NULL);

}
\end{lstlisting}

\subsection{\Exercise 2.5}

\IFRU{Подсказка}{Hint} \#1: \IFRU{Не забывайте, что}{Keep in mind that} 
\TT{\_\_v}\EMDASH\IFRU{глобальная переменная}{global variable}.

\IFRU{Подсказка}{Hint} \#2: \IFRU{Эта функция вызывается startup-кодом перед вызовом \main}
{The function is called in \ac{CRT} startup code, before \main execution}.

\index{x86!\Instructions!FDIV}
\IFRU{Ответ: это проверка на наличие FDIV-ошибки в ранних процессорах Pentium}
{Solution: early Pentium CPU FDIV bug checking}\footnote{\url{http://en.wikipedia.org/wiki/Pentium_FDIV_bug}}.

\IFRU{Исходник на Си}{C source code}:

\begin{lstlisting}
unsigned _v; // _v

enum e {
    PROB_P5_DIV = 0x0001
};

void f( void ) // __verify_pentium_fdiv_bug
{
    /*
        Verify we have got the Pentium FDIV problem.
        The volatiles are to scare the optimizer away.
    */
    volatile double     v1     = 4195835;
    volatile double     v2   = 3145727;

    if( (v1 - (v1/v2)*v2) > 1.0e-8 ) {
        _v |= PROB_P5_DIV;
    }
}
\end{lstlisting}

\subsection{\Exercise 2.6}

\IFRU{Подсказка: если погуглить применяемую здесь константу, это может помочь.}
{Hint: it might be helpful to google a constant used here.}

\IFRU{Ответ: шифрование алгоритмом \ac{TEA}}{Solution: \ac{TEA} encryption algorithm}.

\IFRU{Исходник на Си}{C source code} (\IFRU{взято с}{taken from} \url{http://en.wikipedia.org/wiki/Tiny_Encryption_Algorithm}):

\begin{lstlisting}
void f (unsigned int* v, unsigned int* k) {
    unsigned int v0=v[0], v1=v[1], sum=0, i;           /* set up */
    unsigned int delta=0x9e3779b9;                     /* a key schedule constant */
    unsigned int k0=k[0], k1=k[1], k2=k[2], k3=k[3];   /* cache key */
    for (i=0; i < 32; i++) {                       /* basic cycle start */
        sum += delta;
        v0 += ((v1<<4) + k0) ^ (v1 + sum) ^ ((v1>>5) + k1);
        v1 += ((v0<<4) + k2) ^ (v0 + sum) ^ ((v0>>5) + k3);  
    }                                              /* end cycle */
    v[0]=v0; v[1]=v1;
}
\end{lstlisting}

\subsection{\Exercise 2.7}

\IFRU{Подсказка: таблица содержит заранее вычисленные значения. 
Можно было бы обойтись и без нее, но тогда функция работала бы чуть медленнее.}
{Hint: the table contain pre-calculated values.
It is possible to implement the function without it, but it will work slower, though.}

\IFRU{Ответ: эта функция переставляет все биты во входном 32-битном слове наоборот. 
Это \TT{lib/bitrev.c} из ядра Linux.}
{Solution: this function reverse all bits in input 32-bit integer. 
It is \TT{lib/bitrev.c} from Linux kernel.}

\IFRU{Исходник на Си}{C source code}:

\begin{lstlisting}
const unsigned char byte_rev_table[256] = {
	0x00, 0x80, 0x40, 0xc0, 0x20, 0xa0, 0x60, 0xe0,
	0x10, 0x90, 0x50, 0xd0, 0x30, 0xb0, 0x70, 0xf0,
	0x08, 0x88, 0x48, 0xc8, 0x28, 0xa8, 0x68, 0xe8,
	0x18, 0x98, 0x58, 0xd8, 0x38, 0xb8, 0x78, 0xf8,
	0x04, 0x84, 0x44, 0xc4, 0x24, 0xa4, 0x64, 0xe4,
	0x14, 0x94, 0x54, 0xd4, 0x34, 0xb4, 0x74, 0xf4,
	0x0c, 0x8c, 0x4c, 0xcc, 0x2c, 0xac, 0x6c, 0xec,
	0x1c, 0x9c, 0x5c, 0xdc, 0x3c, 0xbc, 0x7c, 0xfc,
	0x02, 0x82, 0x42, 0xc2, 0x22, 0xa2, 0x62, 0xe2,
	0x12, 0x92, 0x52, 0xd2, 0x32, 0xb2, 0x72, 0xf2,
	0x0a, 0x8a, 0x4a, 0xca, 0x2a, 0xaa, 0x6a, 0xea,
	0x1a, 0x9a, 0x5a, 0xda, 0x3a, 0xba, 0x7a, 0xfa,
	0x06, 0x86, 0x46, 0xc6, 0x26, 0xa6, 0x66, 0xe6,
	0x16, 0x96, 0x56, 0xd6, 0x36, 0xb6, 0x76, 0xf6,
	0x0e, 0x8e, 0x4e, 0xce, 0x2e, 0xae, 0x6e, 0xee,
	0x1e, 0x9e, 0x5e, 0xde, 0x3e, 0xbe, 0x7e, 0xfe,
	0x01, 0x81, 0x41, 0xc1, 0x21, 0xa1, 0x61, 0xe1,
	0x11, 0x91, 0x51, 0xd1, 0x31, 0xb1, 0x71, 0xf1,
	0x09, 0x89, 0x49, 0xc9, 0x29, 0xa9, 0x69, 0xe9,
	0x19, 0x99, 0x59, 0xd9, 0x39, 0xb9, 0x79, 0xf9,
	0x05, 0x85, 0x45, 0xc5, 0x25, 0xa5, 0x65, 0xe5,
	0x15, 0x95, 0x55, 0xd5, 0x35, 0xb5, 0x75, 0xf5,
	0x0d, 0x8d, 0x4d, 0xcd, 0x2d, 0xad, 0x6d, 0xed,
	0x1d, 0x9d, 0x5d, 0xdd, 0x3d, 0xbd, 0x7d, 0xfd,
	0x03, 0x83, 0x43, 0xc3, 0x23, 0xa3, 0x63, 0xe3,
	0x13, 0x93, 0x53, 0xd3, 0x33, 0xb3, 0x73, 0xf3,
	0x0b, 0x8b, 0x4b, 0xcb, 0x2b, 0xab, 0x6b, 0xeb,
	0x1b, 0x9b, 0x5b, 0xdb, 0x3b, 0xbb, 0x7b, 0xfb,
	0x07, 0x87, 0x47, 0xc7, 0x27, 0xa7, 0x67, 0xe7,
	0x17, 0x97, 0x57, 0xd7, 0x37, 0xb7, 0x77, 0xf7,
	0x0f, 0x8f, 0x4f, 0xcf, 0x2f, 0xaf, 0x6f, 0xef,
	0x1f, 0x9f, 0x5f, 0xdf, 0x3f, 0xbf, 0x7f, 0xff,
};

unsigned char bitrev8(unsigned char byte)
{
	return byte_rev_table[byte];
}

unsigned short bitrev16(unsigned short x)
{
	return (bitrev8(x & 0xff) << 8) | bitrev8(x >> 8);
}

/**
 * bitrev32 - reverse the order of bits in a unsigned int value
 * @x: value to be bit-reversed
 */

unsigned int bitrev32(unsigned int x)
{
	return (bitrev16(x & 0xffff) << 16) | bitrev16(x >> 16);
}
\end{lstlisting}

\subsection{\Exercise 2.8}

\IFRU{Ответ: сложение двух матриц размером 100 на 200 элементов типа \Tdouble.}
{Solution: two 100*200 matrices of \Tdouble type addition.}

\IFRU{Исходник на \CCpp}{\CCpp source code}:

\begin{lstlisting}
#define M    100
#define N    200

void s(double *a, double *b, double *c)
{
  for(int i=0;i<N;i++)
    for(int j=0;j<M;j++)
      *(c+i*M+j)=*(a+i*M+j) + *(b+i*M+j);
};
\end{lstlisting}

\subsection{\Exercise 2.9}

\IFRU{Ответ: умножение двух матриц размерами 100*200 и 100*300 элементов типа \Tdouble, результат: матрица 100*300.}
{Solution: two matrices (one is 100*200, second is 100*300) of \Tdouble type multiplication, result: 100*300
matrix.}

\IFRU{Исходник на \CCpp}{\CCpp source code}:

\begin{lstlisting}
#define M     100
#define N     200
#define P     300

void m(double *a, double *b, double *c)
{
  for(int i=0;i<M;i++)
    for(int j=0;j<P;j++)
    {
      *(c+i*M+j)=0;
      for (int k=0;k<N;k++) *(c+i*M+j)+=*(a+i*M+j) * *(b+i*M+j);
    }
};
\end{lstlisting}

\subsection{\Exercise 2.11}

\IFRU{Подсказка: Task Manager узнает количество процессоров/ядер используя вызов ф-ции \\
\TT{NtQuerySystemInformation(SystemBasicInformation, ..., ..., ...)}, этот вызов можно найти и подставлять
туда другое значение.}
{Hint: Task Manager get CPU/CPU cores count using function call \\
\TT{NtQuerySystemInformation(SystemBasicInformation, ..., ..., ...)}, it is possible to find that
call and to substitute resulting number.}

\IFRU{И разумеется, Task Manager будет показывать некорректные результаты в графиках 
истории загрузки процессоров/ядер}{And of course, 
the Task Manager will show incorrect results in CPU usage history}.

\subsection{\Exercise 2.12}

\IFRU{Это алгоритм примитивного шифрования ROT13, некогда популярный в UseNet и мейл-листах}
{This is a primitive cryptographic algorithm named ROT13, once popular in UseNet and mailing lists}
\footnote{\url{https://en.wikipedia.org/wiki/ROT13}}.

\Sourcecode: \url{http://yurichev.com/RE-exercise-solutions/2/12/ROT13.c}

\subsection{\Exercise 2.13}

\IFRU{Это криптоалгоритм}{The cryptoalgorithm is} 
linear feedback shift register\RU{ (регистр сдвига с линейной обратной связью)}
\footnote{\url{https://en.wikipedia.org/wiki/Linear_feedback_shift_register}}.

\Sourcecode: \url{http://yurichev.com/RE-exercise-solutions/2/13/LFSR.c}

\subsection{\Exercise 2.14}

\IFRU{Это алгоритм нахождения наименьшего общего делителя}{This is algorithm of finding greater
common divisor (GCD)}.

\Sourcecode: \url{http://yurichev.com/RE-exercise-solutions/2/14/GCD.c}

\subsection{\Exercise 2.15}

\IFRU{Это вычисление числа Пи методом Монте-Карло}{Pi value calculation using Monte-Carlo method}.

\Sourcecode: \url{http://yurichev.com/RE-exercise-solutions/2/15/monte.c}

\subsection{\Exercise 2.16}

\IFRU{Это функция Аккермана}{It is Ackermann function}
\footnote{\url{https://en.wikipedia.org/wiki/Ackermann_function}}.

\begin{lstlisting}
int ack (int m, int n)
{
	if (m==0)
		return n+1;
	if (n==0)
		return ack (m-1, 1);
	return ack(m-1, ack (m, n-1));
};
\end{lstlisting}

\subsection{\Exercise 2.17}

\IFRU{Это одномерный клеточный автомат по \IT{правилу 110}}{This is 1D cellular automation working
by \IT{Rule 110}}:\\
\url{https://en.wikipedia.org/wiki/Rule_110}.

\Sourcecode: \url{http://yurichev.com/RE-exercise-solutions/2/17/CA.c}

\subsection{\Exercise 2.18}

\Sourcecode: \url{http://yurichev.com/RE-exercise-solutions/2/18/}

\subsection{\Exercise 2.19}

\Sourcecode: \url{http://yurichev.com/RE-exercise-solutions/2/19/}

