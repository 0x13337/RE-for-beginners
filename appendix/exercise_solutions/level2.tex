\section{\RU{Уровень}\EN{Level} 2}

% 2.1
% 2.2
% 2.3

\subsection{\Exercise 2.4}

\index{\CStandardLibrary!strstr()}
\RU{Ответ}\EN{Solution}: \TT{strstr()}.

\RU{Исходник на Си}\EN{C source code}:

\begin{lstlisting}
char * strstr (
        const char * str1,
        const char * str2
        )
{
        char *cp = (char *) str1;
        char *s1, *s2;

        if ( !*str2 )
            return((char *)str1);

        while (*cp)
        {
                s1 = cp;
                s2 = (char *) str2;

                while ( *s1 && *s2 && !(*s1-*s2) )
                        s1++, s2++;

                if (!*s2)
                        return(cp);

                cp++;
        }

        return(NULL);

}
\end{lstlisting}

\subsection{\Exercise 2.5}

\RU{Подсказка}\EN{Hint} \#1: \RU{Не забывайте, что}\EN{Keep in mind that} 
\TT{\_\_v}\EMDASH\RU{глобальная переменная}\EN{global variable}.

\RU{Подсказка}\EN{Hint} \#2: \RU{Эта функция вызывается startup-кодом перед вызовом \main}
\EN{The function is called in \ac{CRT} startup code, before \main execution}.

\index{x86!\Instructions!FDIV}
\RU{Ответ: это проверка на наличие FDIV-ошибки в ранних процессорах Pentium}
\EN{Solution: early Pentium CPU FDIV bug checking}\footnote{\url{http://en.wikipedia.org/wiki/Pentium_FDIV_bug}}.

\RU{Исходник на Си}\EN{C source code}:

\begin{lstlisting}
unsigned _v; // _v

enum e {
    PROB_P5_DIV = 0x0001
};

void f( void ) // __verify_pentium_fdiv_bug
{
    /*
        Verify we have got the Pentium FDIV problem.
        The volatiles are to scare the optimizer away.
    */
    volatile double     v1     = 4195835;
    volatile double     v2   = 3145727;

    if( (v1 - (v1/v2)*v2) > 1.0e-8 ) {
        _v |= PROB_P5_DIV;
    }
}
\end{lstlisting}

\subsection{\Exercise 2.6}

\RU{Подсказка: если погуглить применяемую здесь константу, это может помочь.}
\EN{Hint: it might be helpful to google a constant used here.}

\RU{Ответ: шифрование алгоритмом \ac{TEA}}\EN{Solution: \ac{TEA} encryption algorithm}.

\RU{Исходник на Си}\EN{C source code} (\RU{взято с}\EN{taken from} \url{http://en.wikipedia.org/wiki/Tiny_Encryption_Algorithm}):

\begin{lstlisting}
void f (unsigned int* v, unsigned int* k) {
    unsigned int v0=v[0], v1=v[1], sum=0, i;           /* set up */
    unsigned int delta=0x9e3779b9;                     /* a key schedule constant */
    unsigned int k0=k[0], k1=k[1], k2=k[2], k3=k[3];   /* cache key */
    for (i=0; i < 32; i++) {                       /* basic cycle start */
        sum += delta;
        v0 += ((v1<<4) + k0) ^ (v1 + sum) ^ ((v1>>5) + k1);
        v1 += ((v0<<4) + k2) ^ (v0 + sum) ^ ((v0>>5) + k3);  
    }                                              /* end cycle */
    v[0]=v0; v[1]=v1;
}
\end{lstlisting}

% 2.7
% 2.8
% 2.9
% 2.10
% 2.11
% 2.12

\subsection{\Exercise 2.13}

\RU{Это криптоалгоритм}\EN{The cryptoalgorithm is} 
linear feedback shift register\RU{ (регистр сдвига с линейной обратной связью)}
\footnote{\url{https://en.wikipedia.org/wiki/Linear_feedback_shift_register}}.

\Sourcecode: \url{http://beginners.re/exercise-solutions/2/13/LFSR.c}

\subsection{\Exercise 2.14}

\RU{Это алгоритм нахождения наименьшего общего делителя}\EN{This is algorithm of finding greater
common divisor (GCD)}.

\Sourcecode: \url{http://beginners.re/exercise-solutions/2/14/GCD.c}

\subsection{\Exercise 2.15}

\RU{Это вычисление числа Пи методом Монте-Карло}\EN{Pi value calculation using Monte-Carlo method}.

\Sourcecode: \url{http://beginners.re/exercise-solutions/2/15/monte.c}

\subsection{\Exercise 2.16}

\RU{Это функция Аккермана}\EN{It is Ackermann function}
\footnote{\url{https://en.wikipedia.org/wiki/Ackermann_function}}.

\begin{lstlisting}
int ack (int m, int n)
{
	if (m==0)
		return n+1;
	if (n==0)
		return ack (m-1, 1);
	return ack(m-1, ack (m, n-1));
};
\end{lstlisting}

\subsection{\Exercise 2.17}

\RU{Это одномерный клеточный автомат по \IT{правилу 110}}\EN{This is 1D cellular automation working
by \IT{Rule 110}}:\\
\url{https://en.wikipedia.org/wiki/Rule_110}.

\Sourcecode: \url{http://beginners.re/exercise-solutions/2/17/CA.c}

\subsection{\Exercise 2.18}

\Sourcecode: \url{http://beginners.re/exercise-solutions/2/18/}

\subsection{\Exercise 2.19}

\Sourcecode: \url{http://beginners.re/exercise-solutions/2/19/}

