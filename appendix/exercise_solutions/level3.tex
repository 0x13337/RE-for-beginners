\section{\RU{Уровень}\EN{Level} 3}

% 3.1

\subsection{\Exercise 3.2}
% GOST

\RU{Подсказка: проще всего конечно же искать по значениями в таблицах}
\EN{Hint: easiest way is to find by values in the tables}.

\RU{Исходник на Си с комментариями}\EN{Commented C source code}:\\
\url{http://beginners.re/exercise-solutions/3/2/gost.c}

\subsection{\Exercise 3.3}
% entropy

\RU{Исходник на Си с комментариями}\EN{Commented C source code}:\\
\url{http://beginners.re/exercise-solutions/3/3/entropy.c}

\subsection{\Exercise 3.4}
\RU{Исходник на Си с комментариями, а также расшифрованный файл}
\EN{Commented C source code, and also decrypted file}:
\url{http://beginners.re/exercise-solutions/3/4/}

\subsection{\Exercise 3.5}
% CRC16

\RU{Подсказка: как видно, строка где указано имя пользователя занимает не весь ключевой файл}
\EN{Hint: as we can see, the string with user name occupies not the whole file}.

\RU{Байты за терминирующим нулем вплоть до смещения \TT{0x7F} игнорируются программой}
\EN{Bytes after terminated zero till offset \TT{0x7F} are ignored by program}.

\RU{Исходник на Си с комментариями}\EN{Commented C source code}:\\
\url{http://beginners.re/exercise-solutions/3/5/crc16_keyfile_check.c}

\subsection{\Exercise 3.6}
% webserv

{\RU{Исходник на Си с комментариями}\EN{Commented C source code}}:\\
\url{http://beginners.re/exercise-solutions/3/6/}

\RU{В качестве еще одного упражнения, теперь вы можете попробовать исправить уязвимости в этом веб-сервере.}
\EN{As another exercise, now you may try to fix all vulnerabilities you found in this web-server.}

\subsection{\Exercise 3.8}
% LZSS

{\RU{Исходник на Си с комментариями}\EN{Commented C source code}}:\\
\url{http://beginners.re/exercise-solutions/3/8/}

