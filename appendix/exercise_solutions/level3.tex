\section{\RU{Уровень}\EN{Level} 3}

% 3.1

\subsection{\Exercise 3.2}
% GOST

\RU{Подсказка: проще всего конечно же искать по значениями в таблицах}
\EN{Hint: the easiest way is to search by value in the tables}.

\RU{Исходник на Си с комментариями}\EN{Commented C source code}:\\
\href{http://go.yurichev.com/17156}{beginners.re}

\subsection{\Exercise 3.3}
% entropy

\RU{Исходник на Си с комментариями}\EN{Commented C source code}:\\
\href{http://go.yurichev.com/17157}{beginners.re}

\subsection{\Exercise 3.4}
\RU{Исходник на Си с комментариями, а также расшифрованный файл}
\EN{Commented C source code, and also the decrypted file}:
\href{http://go.yurichev.com/17158}{beginners.re}

\subsection{\Exercise 3.5}
% CRC16

\RU{Подсказка: как видно, строка где указано имя пользователя занимает не весь ключевой файл}
\EN{Hint: as we can see, the string with the user name does not occupy the whole file}.

\RU{Байты за терминирующим нулем вплоть до смещения \TT{0x7F} игнорируются программой}
\EN{Bytes after the terminating zero till offset \TT{0x7F} are ignored by the program}.

\RU{Исходник на Си с комментариями}\EN{Commented C source code}:\\
\href{http://go.yurichev.com/17159}{beginners.re}

\subsection{\Exercise 3.6}
% webserv

{\RU{Исходник на Си с комментариями}\EN{Commented C source code}}:\\
\href{http://go.yurichev.com/17160}{beginners.re}

\RU{В качестве еще одного упражнения, теперь вы можете попробовать исправить уязвимости в этом веб-сервере.}
\EN{As another exercise, now you may try to fix all the vulnerabilities you found in this web server.}

\subsection{\Exercise 3.8}
% LZSS

{\RU{Исходник на Си с комментариями}\EN{Commented C source code}}:\\
\href{http://go.yurichev.com/17161}{beginners.re}

