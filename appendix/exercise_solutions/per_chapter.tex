\section{\RU{По главам}\EN{Per chapter}}

\subsection{\RU{Глава ``\Stack''}\EN{``\Stack'' chapter}}
%02..

\subsubsection{\Exercise \#1}
\label{exercise_solutions_stack_1}

\Exercise: \ref{exercise_stack_1}.

\EN{\NonOptimizing MSVC, these numbers are: saved \EBP value}\RU{Если MSVC без оптимизации, 
то эти числа таковы: сохраненное значение \EBP}, \ac{RA} \AndENRU \TT{argc}.
\RU{В этом легко убедиться, если запускать пример с разным количеством аргументов в командной 
строке}\EN{It's easy to be assured in that by running the example with different number
of arguments in command-line}.

\RU{Если MSVC с оптимизацией, то числа таковы}\EN{\Optimizing MSVC, these numbers are}: 
\ac{RA}, \TT{argc} \RU{и указатель на массив}\EN{and a pointer to} \TT{argv[]}\EN{ array}.

GCC 4.8.x \RU{выделяет в прологе ф-ции \main 16-байтное пространство, поэтому числа на 
выходе совсем другие}\EN{allocates 16-byte space in \main function prologue, 
hence different output numbers}.

\subsubsection{\Exercise \#2}
\label{exercise_solutions_stack_2}

\Exercise: \ref{exercise_stack_2}.

\RU{Этот код выводит время в формате UNIX}\EN{This code prints UNIX time}.

\begin{lstlisting}
#include <stdio.h>
#include <time.h>

int main()
{
	printf ("%d\n", time(NULL));
};
\end{lstlisting}

\subsection{\RU{Глава ``\SwitchCaseDefaultSectionName''}\EN{``\SwitchCaseDefaultSectionName'' chapter}}
%08...

\subsection{\Exercise \#1}
\label{exercise_solutions_switch_1}

\Exercise: \ref{exercise_switch_1}.

\RU{Подсказка}\EN{Hint}: \printf \EN{may be called only from the one single place}\RU{вполне может 
вызываться только из одного места}.

\subsection{\RU{Глава ``\Loops''}\EN{``\Loops'' chapter}}
%09...

\subsection{\Exercise \#3}
\label{exercise_solutions_loops_3}

\Exercise: \ref{exercise_loops_3}.

\begin{lstlisting}
#include <stdio.h>

int main()
{
	int i;
	for (i=100; i>0; i--)
		printf ("%d\n", i);
};
\end{lstlisting}

\subsection{\Exercise \#4}
\label{exercise_solutions_loops_4}

\Exercise: \ref{exercise_loops_4}.

\begin{lstlisting}
#include <stdio.h>

int main()
{
	int i;
	for (i=1; i<100; i=i+3)
		printf ("%d\n", i);
};
\end{lstlisting}

\subsection{\RU{Глава ``\SimpleStringsProcessings''}\EN{``\SimpleStringsProcessings'' chapter}}
%10..

\subsubsection{\Exercise \#1}
\label{exercise_solutions_strlen_1}

\Exercise: \ref{exercise_strlen_1}.

\RU{Эта ф-ция подсчитывает пробелы во входящей Си-строке.}
\EN{This is a function counting spaces in the input C-string.}

\begin{lstlisting}
int f(char *s)
{
	int rt=0;
	for (;*s;s++)
	{
		if (*s==' ')
			rt++;
	};
	return rt;
};
\end{lstlisting}

\subsection{\RU{Глава ``\ArithOptimizations''}\EN{``\ArithOptimizations'' chapter}}
%11..

\subsubsection{\Exercise \#2}
\label{exercise_solutions_arith_optimizations_2}

\Exercise: \ref{exercise_arith_optimizations_2}.

\begin{lstlisting}
int f(int a)
{
	return a*7;
};
\end{lstlisting}

\subsection{\RU{Глава ``\FPUChapterName''}\EN{``\FPUChapterName'' chapter}}
%12..

\subsubsection{\Exercise \#1}
\label{exercise_solutions_FPU_2}

\Exercise: \ref{exercise_FPU_2}.

\EN{Calculating arithmetic mean for 5 \Tdouble values.}
\RU{Вычисление среднего арифметического для пяти значений типа \Tdouble.}

\begin{lstlisting}
double f(double a1, double a2, double a3, double a4, double a5)
{
	return (a1+a2+a3+a4+a5) / 5;
};
\end{lstlisting}

\subsection{\RU{Глава ``\Arrays''}\EN{``\Arrays'' chapter}}
%13..

\subsubsection{\Exercise \#1}
\label{exercise_solutions_arrays_1}

\Exercise: \ref{exercise_array_1}.

% FIXME: make example of it!
\RU{Ответ: сложение двух матриц размером 100 на 200 элементов типа \Tdouble.}
\EN{Solution: two 100*200 matrices of \Tdouble type addition.}

\RU{Исходник на \CCpp}\EN{\CCpp source code}:

\begin{lstlisting}
#define M    100
#define N    200

void s(double *a, double *b, double *c)
{
  for(int i=0;i<N;i++)
    for(int j=0;j<M;j++)
      *(c+i*M+j)=*(a+i*M+j) + *(b+i*M+j);
};
\end{lstlisting}

\subsubsection{\Exercise \#2}
\label{exercise_solutions_arrays_2}

\Exercise: \ref{exercise_array_2}.

% FIXME: make example of it?
\RU{Ответ: умножение двух матриц размерами 100*200 и 100*300 элементов типа \Tdouble, результат: матрица 100*300.}
\EN{Solution: two matrices (one is 100*200, second is 100*300) of \Tdouble type multiplication, result: 100*300
matrix.}

\RU{Исходник на \CCpp}\EN{\CCpp source code}:

\begin{lstlisting}
#define M     100
#define N     200
#define P     300

void m(double *a, double *b, double *c)
{
  for(int i=0;i<M;i++)
    for(int j=0;j<P;j++)
    {
      *(c+i*M+j)=0;
      for (int k=0;k<N;k++) *(c+i*M+j)+=*(a+i*M+j) * *(b+i*M+j);
    }
};
\end{lstlisting}

\subsubsection{\Exercise \#3}
\label{exercise_solutions_arrays_3}

\Exercise: \ref{exercise_array_3}.

\begin{lstlisting}
double f(double array[50][120], int x, int y)
{
	return array[x][y];
};
\end{lstlisting}

\subsubsection{\Exercise \#4}
\label{exercise_solutions_arrays_4}

\Exercise: \ref{exercise_array_4}.

\begin{lstlisting}
int f(int array[50][60][80], int x, int y, int z)
{
	return array[x][y][z];
};
\end{lstlisting}

\subsubsection{\Exercise \#5}
\label{exercise_solutions_arrays_5}

\Exercise: \ref{exercise_array_5}.

\EN{This code just calculates multiplication table.}
\RU{Этот код просто вычисляет таблицу умножения.}

\begin{lstlisting}
int tbl[10][10];

int main()
{
	int x, y;
	for (x=0; x<10; x++)
		for (y=0; y<10; y++)
			tbl[x][y]=x*y;
};
\end{lstlisting}

% TODO MSVC 2012 optimizations!

\subsection{\RU{Глава ``\BitfieldsChapter''}\EN{``\BitfieldsChapter'' chapter}}
%14..

\subsubsection{\Exercise \#1}
\label{exercise_solutions_bitfields_1}

\Exercise: \ref{exercise_bitfields_1}.
% FIXME: make example of it!
\RU{Эта ф-ция меняет}\EN{This is a function which changes} \gls{endianness} 
\RU{в 32-битном значении}\EN{in 32-bit value}.

\lstinputlisting{appendix/exercise_solutions/change_endiannes.c}

\RU{Дополнительный вопрос: в x86 есть инструкция делающая всё это. Какая?}\EN{Additional question: x86 
instruction can do this. Which one?}
% answer: BSWAP

\subsubsection{\Exercise \#2}
\label{exercise_solutions_bitfields_2}

\Exercise: \ref{exercise_bitfields_2}.

\RU{Эта ф-ция конвертирует значение запакованное в формате \ac{BCD} в обычное.}
\EN{This function converts \ac{BCD}-packed 32-bit value into usual one.}

\begin{lstlisting}
#include <stdio.h>

unsigned int f(unsigned int a)
{
	int i=0;
	int j=1;
	unsigned int rt=0;
	for (;i<=28; i+=4, j*=10)
		rt+=((a>>i)&0xF) * j;
	return rt;
};

int main()
{
	// test
	printf ("%d\n", f(0x12345678));
	printf ("%d\n", f(0x1234567));
	printf ("%d\n", f(0x123456));
	printf ("%d\n", f(0x12345));
	printf ("%d\n", f(0x1234));
	printf ("%d\n", f(0x123));
	printf ("%d\n", f(0x12));
	printf ("%d\n", f(0x1));
};
\end{lstlisting}

\subsubsection{\Exercise \#3}
\label{exercise_solutions_bitfields_3}

\Exercise: \ref{exercise_bitfields_3}.

\begin{lstlisting}
#include <windows.h>

int main()
{
	MessageBox(NULL, "hello, world!", "caption", 
		MB_TOPMOST | MB_ICONINFORMATION | MB_HELP | MB_YESNOCANCEL);
};
\end{lstlisting}

\subsubsection{\Exercise \#4}
\label{exercise_solutions_bitfields_4}

\Exercise: \ref{exercise_bitfields_4}.

\EN{This function just multiplies two 32-bit numbers, returning 64-bit \gls{product}.}
\RU{Эта ф-ция просто перемножает два 32-битных числа, возвращая 64-битное \glslink{product}{произведение}.}
\EN{Well, this is a case when simple observing input/outputs may solve problem faster.}
\RU{Да, это тот случай, когда простое наблюдение входных и выходных значений может решить проблему быстрее.}

\begin{lstlisting}
#include <stdio.h>
#include <stdint.h>

// source code taken from
// http://www4.wittenberg.edu/academics/mathcomp/shelburne/comp255/notes/binarymultiplication.pdf

uint64_t mult (uint32_t m, uint32_t n)
{
    uint64_t p = 0; // initialize product p to 0 
    while (n != 0) // while multiplier n is not 0 
    { 
        if (n & 1) // test LSB of multiplier 
            p = p + m; // if 1 then add multiplicand m 
        m = m << 1; // left shift multiplicand 
        n = n >> 1; // right shift multiplier 
    }
    return p;
}

int main()
{
    printf ("%d\n", mult (2, 7));
    printf ("%d\n", mult (3, 11));
    printf ("%d\n", mult (4, 111));
};
\end{lstlisting}

\subsection{\RU{Глава ``\StructuresChapterName''}\EN{``\StructuresChapterName'' chapter}}
%15..

\subsubsection{\Exercise \#1}
\label{exercise_solutions_struct_1}

\Exercise: \ref{exercise_struct_1}.

\RU{Эта программа показывает ID пользователя владеющего файлом}\EN{This program shows user ID of file owner}.

\begin{lstlisting}
#include <sys/types.h>
#include <sys/stat.h>
#include <time.h>
#include <stdio.h>
#include <stdlib.h>

int main(int argc, char *argv[])
{
    struct stat sb;

    if (argc != 2) 
    {
        fprintf(stderr, "Usage: %s <pathname>\n", argv[0]);
        return 0;
    }

    if (stat(argv[1], &sb) == -1) 
    {
    	// error
        return 0;
    }

    printf("%ld\n",(long) sb.st_uid);
}
\end{lstlisting}

\subsubsection{\Exercise \#1}
\label{exercise_solutions_struct_2}

\Exercise: \ref{exercise_struct_2}.

\EN{Hint (x86): you may get some information on how values are treated with \TT{Jcc}, \MOVSX and \MOVZX instructions.}
\RU{Подсказка (x86): вы можете получить какую-то информацию глядя на то как значения обрабатываются при помощи
инструкций \TT{Jcc}, \MOVSX и \MOVZX.}

\begin{lstlisting}
#include <stdio.h>

struct some_struct
{
	int a;
	unsigned int b;
	float f;
	double d;
	char c;
	unsigned char uc;
};

void f(struct some_struct *s)
{
	if (s->a > 1000)
	{
		if (s->b > 10)
		{
			printf ("%f\n", s->f * 444 + s->d * 123);
			printf ("%c, %d\n", s->c, s->uc);
		}
		else
		{
			printf ("error #2\n");
		};
	}
	else
	{
		printf ("error #1\n");
	};
};
\end{lstlisting}

\subsection{\RU{Глава ``Обфускация''}\EN{``Obfuscation'' chapter}}

\subsubsection{\Exercise \#1}
\label{exercise_solutions_obfuscation_1}

\Exercise: \ref{exercise_obfuscation_1}.

\RU{Исходный код}\EN{Source code}: \href{http://go.yurichev.com/17162}{beginners.re}.

\subsection{\RU{Глава ``\DivisionByNineSectionName''}\EN{``\DivisionByNineSectionName'' chapter}}

\subsubsection{\Exercise \#1}
\label{exercise_solutions_arith_optimizations_1}

\Exercise: \ref{exercise_arith_optimizations_1}.

\begin{lstlisting}
int f(int a)
{
	return a/661;
};
\end{lstlisting}
