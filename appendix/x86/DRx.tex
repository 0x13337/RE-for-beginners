% to be proofreaded (begin)
\subsection{\IFRU{Отладочные регистры}{Debugging registers}}

\IFRU{Применяются для работы с т.н.}{Used for} hardware breakpoints\IFRU{}{ control}.

\begin{itemize}
	\item DR0 --- \IFRU{адрес точки останова}{address of breakpoint} \#1
	\item DR1 --- \IFRU{адрес точки останова}{address of breakpoint} \#2
	\item DR2 --- \IFRU{адрес точки останова}{address of breakpoint} \#3
	\item DR3 --- \IFRU{адрес точки останова}{address of breakpoint} \#4
	\item DR6 --- \IFRU{здесь отображается причина останова}{a cause of break is reflected here}
	\item DR7 --- \IFRU{здесь можно задать типы точек останова}{breakpoint types are set here}
\end{itemize}

\subsubsection{DR6}

\begin{center}
\begin{tabular}{ | l | l | }
\hline
\headercolor{} \IFRU{Бит}{Bit} (\IFRU{маска}{mask}) &
\headercolor{} \IFRU{Описание}{Description} \\
\hline
0 (1)       &  B0 --- \IFRU{сработала точка останова \#1}{breakpoint \#1 was triggered} \\
\hline
1 (2)       &  B1 --- \IFRU{сработала точка останова \#2}{breakpoint \#2 was triggered} \\
\hline
2 (4)       &  B2 --- \IFRU{сработала точка останова \#3}{breakpoint \#3 was triggered} \\
\hline
3 (8)       &  B3 --- \IFRU{сработала точка останова \#4}{breakpoint \#4 was triggered} \\
\hline
13 (0x2000) &  BD --- \IFRU{была попытка модифицировать один из регистров DRx.}
               {modification attempt of one of DRx registers.} \\
            &  \IFRU{может быть выставлен если бит GD выставлен.}
	       {may be raised if GD is enabled} \\
\hline
14 (0x4000) &  BS --- \IFRU{точка останова типа single step (флаг TF был выставлен в EFLAGS)}
               {single step breakpoint (TF flag was set in EFLAGS)}. \\
	    &  \IFRU{Наивысший приоритет. Другие биты также могут быть выставлены}
	       {Highest priority. Other bits may also be set}. \\
\hline
% FIXME: describe BT
15 (0x8000) &  BT (task switch flag) \\
\hline
\end{tabular}
\end{center}

N.B. \IFRU{Точка останова single step это срабатывающая после каждой инструкции}
{Single step breakpoint is a breakpoint occuring after each instruction}.
\IFRU{Может быть включена выставлением флага TF в EFLAGS}
{It can be enabled by setting TF in EFLAGS} (\ref{EFLAGS}).

\subsubsection{DR7}

\IFRU{В этом регистре задаются типы точек останова}{Breakpoint types are set here}.

\begin{center}
\begin{tabular}{ | l | l | }
\hline
\headercolor{} \IFRU{Бит}{Bit} (\IFRU{маска}{mask}) &
\headercolor{} \IFRU{Описание}{Description} \\
\hline
0 (1)       &  L0 --- \IFRU{разрешить точку останова}{enable breakpoint} \#1 \IFRU{для текущей задачи}{for the current task} \\
\hline
1 (2)       &  G0 --- \IFRU{разрешить точку останова}{enable breakpoint} \#1 \IFRU{для всех задач}{for all tasks} \\
\hline
2 (4)       &  L1 --- \IFRU{разрешить точку останова}{enable breakpoint} \#2 \IFRU{для текущей задачи}{for the current task} \\
\hline
3 (8)       &  G1 --- \IFRU{разрешить точку останова}{enable breakpoint} \#2 \IFRU{для всех задач}{for all tasks} \\
\hline
4 (0x10)    &  L2 --- \IFRU{разрешить точку останова}{enable breakpoint} \#3 \IFRU{для текущей задачи}{for the current task} \\
\hline
5 (0x20)    &  G2 --- \IFRU{разрешить точку останова}{enable breakpoint} \#3 \IFRU{для всех задач}{for all tasks} \\
\hline
6 (0x40)    &  L3 --- \IFRU{разрешить точку останова}{enable breakpoint} \#4 \IFRU{для текущей задачи}{for the current task} \\
\hline
7 (0x80)    &  G3 --- \IFRU{разрешить точку останова}{enable breakpoint} \#4 \IFRU{для всех задач}{for all tasks} \\
\hline
8 (0x100)   &  LE --- \IFRU{не поддерживается начиная с P6}{not supported since P6} \\
\hline
9 (0x200)   &  GE --- \IFRU{не поддерживается начиная с P6}{not supported since P6} \\
\hline
13 (0x2000) &  GD --- \IFRU{исключение будет вызвано если какая-либо инструкция MOV}
			{exception will be raised if any MOV instruction} \\
            & \IFRU{попытается модифицировать один из DRx-регистров}
			{tries to modify one of DRx registers} \\
\hline
16,17 (0x30000)    &  \IFRU{точка останова}{breakpoint} \#1: R/W --- \IFRU{тип}{type} \\
\hline
18,19 (0xC0000)    &  \IFRU{точка останова}{breakpoint} \#1: LEN --- \IFRU{длина}{length} \\
\hline
20,21 (0x300000)   &  \IFRU{точка останова}{breakpoint} \#2: R/W --- \IFRU{тип}{type} \\
\hline
22,23 (0xC00000)   &  \IFRU{точка останова}{breakpoint} \#2: LEN --- \IFRU{длина}{length} \\
\hline
24,25 (0x3000000)  &  \IFRU{точка останова}{breakpoint} \#3: R/W --- \IFRU{тип}{type} \\
\hline
26,27 (0xC000000)  &  \IFRU{точка останова}{breakpoint} \#3: LEN --- \IFRU{длина}{length} \\
\hline
28,29 (0x30000000) &  \IFRU{точка останова}{breakpoint} \#4: R/W --- \IFRU{тип}{type} \\
\hline
30,31 (0xC0000000) &  \IFRU{точка останова}{breakpoint} \#4: LEN --- \IFRU{длина}{length} \\
\hline
\end{tabular}
\end{center}

\IFRU{Так задается тип точки останова}{Breakpoint type is to be set as follows} (R/W):

\begin{itemize}
\item 00 --- \IFRU{исполнение инструкции}{instruction execution}
\item 01 --- \IFRU{запись в память}{data writes}
\item 10 --- \IFRU{обращения к I/O-портам (недоступно из user-mode)}{I/O reads or writes (not available in user-mode)}
\item 11 --- \IFRU{обращение к памяти (чтение или запись)}{on data reads or writes}
\end{itemize}

N.B.: \IFRU{отдельного типа для чтения из памяти действительно нет}
{breakpoint type for data reads is absent, indeed}. \\
\\
\IFRU{Так задается длина точки останова}{Breakpoint length is to be set as follows} (LEN):

\begin{itemize}
\item 00 --- \IFRU{1 байт}{one-byte}
\item 01 --- \IFRU{2 байта}{two-byte}
\item 10 --- \IFRU{не определено для 32-битного режима, 8 байт для 64-битного}
		{undefined for 32-bit mode, eight-byte in 64-bit mode}
\item 11 --- \IFRU{4 байта}{four-byte}
\end{itemize}

% to be proofreaded (end)
