\index{\CStandardLibrary!memcmp()}
\index{x86!\Instructions!CMPSB}
\index{x86!\Instructions!CMPSW}
\index{x86!\Instructions!CMPSD}
\index{x86!\Instructions!CMPSQ}
\item[CMPSB/CMPSW/CMPSD/CMPSQ] (M) \RU{сравнить}\EN{compare} \RU{байт}\EN{byte}/
16-\RU{битное слово}\EN{bit word}/
32-\RU{битное слово}\EN{bit word}/
64-\RU{битное слово}\EN{bit word} \RU{из места, адрес которого находится в}\EN{from the place
address of which is in the} SI/ESI/RSI \RU{со значением, адрес которого находится
в}\EN{with a variable address of which is in the} DI/EDI/RDI.
\RU{Выставить флаги так же, как это делает \CMP}\EN{Set flags as \CMP does}.

\label{REPE_CMPSx}
\RU{Вместе с префиксом REPE, инструкция будет исполняться в цикле, счетчик будет
находится в регистре CX/ECX/RCX, процесс будет продолжаться пока флаг ZF=0 (т.е., до тех пор,
пока все сравниваемые значения равны, отсюда ``E'' в REPE)}
\EN{Together with REP prefix, it will repeated in loop, count is stored in the CX/ECX/RCX register,
the process will be running until ZF flag is zero (e.g., until compared values are equal to each
other, hence ``E'' in REPE)}.

\RU{Это работает как}\EN{It works like} memcmp() \RU{в Си}\EN{in C}.

\RU{Пример из ядра Windows NT}\EN{Example from Windows NT kernel} (\ac{WRK} v1.2):

\lstinputlisting[caption=base\textbackslash{}ntos\textbackslash{}rtl\textbackslash{}i386\textbackslash{}movemem.asm]{appendix/x86/instructions/RtlCompareMemory_WRK12.asm}

N.B.: \RU{эта ф-ция использует сравнение 32-битных слов (CMPSD) если длина блоков
кратна 4-м байтам, либо побайтовое сравнение (CMPSB) если не кратна}
\EN{this function uses 32-bit words comparison (CMPSD) if block size is multiple of 4,
or per-byte comparison (CMPSB) otherwise}.

