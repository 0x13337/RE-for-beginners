\index{x86!\Instructions!INT}

\index{MS-DOS}
\item[INT]: \TT{INT x} \IFRU{аналогична}{is analogous to} \TT{PUSHF; CALL dword ptr [x*4]} 
\IFRU{в 16-битной среде}{in 16-bit environment}.
  \IFRU{Она активно использовалась в MS-DOS, работая как сисколл. Аргументы записывались в регистры
  AX/BX/CX/DX/SI/DI и затем происходил переход на таблицу векторов прерываний (расположенную в самом
  начале адресного пространства)}
  {It was widely used in MS-DOS, functioning as syscalls. Registers AX/BX/CX/DX/SI/DI were filled
  by arguments and jump to the address in the Interrupt Vector Table 
  (located at the address space beginning)
  will be occured}.
  \IFRU{Она была очень популярна потому что имела короткий опкод (2 байта) и программе использующая
  сервисы MS-DOS не нужно было заморачиваться узнавая адреса всех ф-ций этих сервисов}
  {It was popular because INT has short opcode (2 bytes) and the program which needs
  some MS-DOS services is not bothering by determining service's entry point address}.
\index{x86!\Instructions!IRET}
  \IFRU{Обработчик прерываний возвращал управление назад при помощи инструкции IRET}
  {Interrupt handler return control flow to called using IRET instruction}.

  \IFRU{Самое используемое прерывание в MS-DOS было 0x21, там была основная часть его \ac{API}}
  {Most busy MS-DOS interrupt number was 0x21, serving a huge ammount of its \ac{API}}.
  \IFRU{См.также}{Refer to} \cite{RalfBrown} 
  \IFRU{самый крупный список всех известных прерываний и вообще там много информации о MS-DOS}
  {for the most comprehensive interrupt lists and other MS-DOS information}.

\index{x86!\Instructions!SYSENTER}
\index{x86!\Instructions!SYSCALL}
  \IFRU{Во времена после MS-DOS, эта инструкция все еще использовалась как сискол, и в Linux
  и в Windows (\ref{syscalls}), но позже была заменена инструкцией SYSENTER или SYSCALL}
  {In post-MS-DOS era, this instruction was still used as syscall both in Linux and 
  Windows (\ref{syscalls}), but later replaced by SYSENTER or SYSCALL instruction}.

