% to be proofreaded
\index{\CStandardLibrary!memcpy()}
\index{x86!\Instructions!MOVSB}
\index{x86!\Instructions!MOVSW}
\index{x86!\Instructions!MOVSD}
\index{x86!\Instructions!MOVSQ}
\item[MOVSB/MOVSW/MOVSD/MOVSQ] 
\RU{скопировать}\EN{copy} \RU{байт}\EN{byte}/
16-\RU{битное слово}\EN{bit word}/
32-\RU{битное слово}\EN{bit word}/
64-\RU{битное слово}\EN{bit word} \RU{на который указывает}\EN{address of which is in the} SI/ESI/RSI 
\RU{куда указывает}\EN{into the place address of which is in the} DI/EDI/RDI.

\label{REP_MOVSx}
\RU{Вместе с префиксом REP, инструкция будет исполняться в цикле, счетчик будет
находится в регистре CX/ECX/RCX}
\EN{Together with REP prefix, it will repeated in loop, count is stored in the CX/ECX/RCX register}:
\RU{это работает как}\EN{it works like} memcpy() \RU{в Си}\EN{in C}.
\RU{Если размер блока известен компилятору на стадии компиляции,
memcpy() часто компилируется в короткий фрагмент кода использующий
REP MOVSx, иногда даже несколько инструкций}
\EN{If block size is known to compiler on compile stage, 
memcpy() is often inlined into short code fragment
using REP MOVSx, sometimes even as several instructions}.

\RU{Эквивалент }memcpy(EDI, ESI, 15)\EN{ equivalent is}:

\lstinputlisting{appendix/x86/instructions/MOVSB_ex1_\LANG.asm}

(\RU{Вероятно, так быстрее чем копировать 15 байт используя просто одну}
\EN{Supposedly, it will work faster then copying 15 bytes using just one} REP MOVSB).

