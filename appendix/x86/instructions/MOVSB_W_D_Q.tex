\index{\CStandardLibrary!memcpy()}
\index{x86!\Instructions!MOVSB}
\index{x86!\Instructions!MOVSW}
\index{x86!\Instructions!MOVSD}
\index{x86!\Instructions!MOVSQ}
\item[MOVSB/MOVSW/MOVSD/MOVSQ] 
\RU{скопировать}\EN{copy} \RU{байт}\EN{byte}/
16-\RU{битное слово}\EN{bit word}/
32-\RU{битное слово}\EN{bit word}/
64-\RU{битное слово}\EN{bit word} \RU{на который указывает}\EN{from the address which is in} SI/ESI/RSI 
\RU{куда указывает}\EN{into the address which is in} DI/EDI/RDI.

\label{REP_MOVSx}
\index{x86!\Prefixes!REP}
\RU{Вместе с префиксом REP, инструкция исполняется в цикле, 
счетчик находится в регистре CX/ECX/RCX:}
\EN{Together with the REP prefix, it is to be repeated in a loop, 
the count is to be stored in the CX/ECX/RCX register:}
\RU{это работает как}\EN{it works like} memcpy() \RU{в Си}\EN{in C}.
\RU{Если размер блока известен компилятору на стадии компиляции,
memcpy() часто компилируется в короткий фрагмент кода использующий
REP MOVSx, иногда даже несколько инструкций}
\EN{If the block size is known to the compiler in the compile stage, 
memcpy() is often inlined into a short code fragment
using REP MOVSx, sometimes even as several instructions}.

\EN{The}\RU{Эквивалент} memcpy(EDI, ESI, 15)\EN{ equivalent is}:

\lstinputlisting{appendix/x86/instructions/MOVSB_ex1.asm.\LANG}

(\RU{Вероятно, так быстрее чем копировать 15 байт используя просто одну REP MOVSB}
\EN{Supposedly, it works faster than copying 15 bytes using just one REP MOVSB}).

