% to be proofreaded
\index{\CStandardLibrary!memcpy()}
\index{x86!\Instructions!MOVSB}
\index{x86!\Instructions!MOVSW}
\index{x86!\Instructions!MOVSD}
\index{x86!\Instructions!MOVSQ}
\item[MOVSB/MOVSW/MOVSD/MOVSQ] 
\IFRU{скопировать}{copy} \IFRU{байт}{byte}/
16-\IFRU{битное слово}{bit word}/
32-\IFRU{битное слово}{bit word}/
64-\IFRU{битное слово}{bit word} \IFRU{на который указывает}{address of which is in the} SI/ESI/RSI 
\IFRU{куда указывает}{into the place address of which is in the} DI/EDI/RDI.

\label{REP_MOVSx}
\IFRU{Вместе с префиксом REP, инструкция будет исполняться в цикле, счетчик будет
находится в регистре CX/ECX/RCX}
{Together with REP prefix, it will repeated in loop, count is stored in the CX/ECX/RCX register}:
\IFRU{это работает как}{it works like} memcpy() \IFRU{в Си}{in C}.
\IFRU{Если размер блока известен компилятору на стадии компиляции,
memcpy() часто компилируется в короткий фрагмент кода использующий
REP MOVSx, иногда даже несколько инструкций}
{If block size is known to compiler on compile stage, 
memcpy() is often inlined into short code fragment
using REP MOVSx, sometimes even as several instructions}.

\IFRU{Эквивалент }{}memcpy(EDI, ESI, 15)\IFRU{}{ equivalent is}:

\lstinputlisting{appendix/x86/instructions/MOVSB_ex1_\IFRU{ru}{en}.asm}

(\IFRU{Вероятно, так быстрее чем копировать 15 байт используя просто одну}
{Supposedly, it will work faster then copying 15 bytes using just one} REP MOVSB).

