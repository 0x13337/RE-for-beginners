\index{x86!\Instructions!POPCNT}
  \item[POPCNT] population count. \RU{Считает количество бит выставленных в 1 в значении}
  \EN{Counts the number of 1 bits in the value}.
  \ac{AKA} \q{hamming weight}.
  \ac{AKA} \q{NSA instruction} \RU{из-за слухов}\EN{due to some rumors}:

\begin{framed}
\begin{quotation}
  This branch of cryptography is fast-paced and very politically charged.
  Most designs are secret; a majority of military encryptions systems in use today are 
  based on LFSRs. 
  In fact, most Cray computers (Cray 1, Cray X-MP, Cray Y-MP) have a rather curious 
  instruction generally known as “population count.” It counts the 1 bits in a register 
  and can be used both to efficiently calculate the Hamming distance between two binary 
  words and to implement a vectorized version of a LFSR. I’ve heard this called the canonical 
  NSA instruction, demanded by almost all computer contracts.
\end{quotation}
\end{framed}
\cite{Schneier}

