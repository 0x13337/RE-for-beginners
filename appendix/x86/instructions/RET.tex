\index{x86!\Instructions!RET}
\index{MS-DOS}
\item[RET] \RU{возврат из процедуры}\EN{return from subroutine}: \TT{POP tmp; JMP tmp}.

\RU{В реальности}\EN{In fact}, RET 
\RU{это макрос ассемблера, в среде Windows и *NIX транслирующийся в}
\EN{is an assembly language macro, in Windows and *NIX environment it is translated into}
RETN (``return near'') 
\RU{ибо, во времена MS-DOS, где память адресовалась немного иначе}
\EN{or, in MS-DOS times, where the memory was addressed differently}
(\myref{8086_memory_model}), \RU{в}\EN{into} RETF (``return far'').

\TT{RET} \RU{может иметь операнд}\EN{can have an operand}.
\RU{Тогда его работа будет такой}\EN{Then it will work like this}: \TT{POP tmp; ADD ESP op1; JMP tmp}.
\TT{RET} \RU{с операндом обычно завершает ф-ции с соглашением о вызовах \IT{stdcall}, см. также}
\EN{with an operand usually ends functions in the \IT{stdcall} calling convention, see also}: \myref{sec:stdcall}.

