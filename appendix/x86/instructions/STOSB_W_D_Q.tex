% to be proofreaded
\index{\CStandardLibrary!memset()}
\index{x86!\Instructions!STOSB}
\index{x86!\Instructions!STOSW}
\index{x86!\Instructions!STOSD}
\index{x86!\Instructions!STOSQ}
\item[STOSB/STOSW/STOSD/STOSQ] \IFRU{записать}{store} \IFRU{байт}{byte}/
16-\IFRU{битное слово}{bit word}/
32-\IFRU{битное слово}{bit word}/
64-\IFRU{битное слово}{bit word} \IFRU{из}{from} AX/EAX/RAX \IFRU{в место, адрес которого находится
в}{into the place address of which is in the} DI/EDI/RDI.

\label{REP_STOSx}
\IFRU{Вместе с префиксом REP, инструкция будет исполняться в цикле, счетчик будет
находится в регистре CX/ECX/RCX}
{Together with REP prefix, it will repeated in loop, count is stored in the CX/ECX/RCX register}:
\IFRU{это работает как}{it works like} memset() \IFRU{в Си}{in C}.
\IFRU{Если размер блока известен компилятору на стадии компиляции,
memset() часто компилируется в короткий фрагмент кода использующий
REP STOSx, иногда даже несколько инструкций}
{If block size is known to compiler on compile stage, 
memset() is often inlined into short code fragment
using REP MOVSx, sometimes even as several instructions}.

\IFRU{Эквивалент }{}memset(EDI, 0xAA, 15)\IFRU{}{ equivalent is}:

\lstinputlisting{appendix/x86/instructions/STOSB_ex1_\IFRU{ru}{en}.asm}

(\IFRU{Вероятно, так быстрее чем заполнять 15 байт используя просто одну}
{Supposedly, it will work faster then storing 15 bytes using just one} REP STOSB).

