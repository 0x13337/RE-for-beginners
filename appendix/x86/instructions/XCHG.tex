\myindex{x86!\Instructions!XCHG}
  \item[XCHG] (M) \RU{обменять местами значения в операндах}\EN{exchange the values in the operands}

\RU{Это редкая инструкция: компиляторы её не генерируют, потому что начиная с Pentium, XCHG с операндом в памяти
исполняется так, как если имеет префикс LOCK ([\MAbrash глава 19]).
Вероятно, в Intel так сделали для совместимости с синхронизирующими примитивами.
Таким образом, XCHG начиная с Pentium может быть медленной.
С другой стороны, XCHG была очень популярна у программистов на ассемблере.
Так что, если вы видите XCHG в коде, это может быть знаком, что код написан вручную.}
% TODO translate

