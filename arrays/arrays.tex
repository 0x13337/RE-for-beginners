% done

\section{\IFRU{Массивы}{Arrays}}
\label{arrays}

\IFRU{Массив это просто набор переменных в памяти, обязательно лежащих рядом, и обязательно одного типа.}
{Array is just a set of variables in memory, always lying next to each other, always has same type.}

\begin{lstlisting}
#include <stdio.h>

int main() 
{
	int a[20];
	int i;

	for (i=0; i<20; i++)
		a[i]=i*2;

	for (i=0; i<20; i++)
		printf ("a[%d]=%d\n", i, a[i]);

	return 0;
};
\end{lstlisting}

\IFRU{Компилируем:}{Let's compile:}

\lstinputlisting{arrays/13_2_msvc.asm}

\IFRU{Однако, ничего особенного, просто два цикла, один заполняет цикл, второй печатает его содержимое. 
Команда \TT{shl ecx, 1} используется для умножения \ECX на 2, об этом ниже~\ref{SHR}.}
{Nothing very special, just two loops: first is filling loop and second is printing loop.
\TT{shl ecx, 1} instruction is used for \ECX value multiplication by 2, more about below~\ref{SHR}.}

\IFRU{Под массив выделено в стеке 80 байт, это 20 элементов по 4 байта.}
{80 bytes are allocated in stack for array, that's 20 elements of 4 bytes.}

\IFRU{То что делает GCC 4.4.1:}{Here is what GCC 4.4.1 does:}

\lstinputlisting{arrays/13_2_gcc.asm}

\subsection{\IFRU{Переполнение буфера}{Buffer overflow}}
\label{subsec:bufferoverflow}

\IFRU{Итак, индексация массива это просто \IT{массив\lbrack{}индекс\rbrack}.  % TODO как-то плохо отображаются []
Если вы присмотритесь к коду, в цикле печати значений массива через \printf вы 
не увидите проверок индекса, \IT{меньше ли он двадцати?} 
А что будет если он будет больше двадцати? 
Эта одна из особенностей \CCpp, за которую их, собственно, и ругают.}
{So, array indexing is just \IT{array\lbrack{}index\rbrack}.
If you study generated code closely, you'll probably note missing index bounds checking,
which could check index, \IT{if it is less than 20}.
What if index will be greater than 20?
That's the one \CCpp feature it's often blamed for.}

\IFRU{Вот код который и компилится и работает:}
{Here is a code successfully compiling and working:}

\begin{lstlisting}
#include <stdio.h>

int main() 
{
	int a[20];
	int i;

	for (i=0; i<20; i++)
		a[i]=i*2;

	printf ("a[100]=%d\n", a[100]);

	return 0;
};
\end{lstlisting}

\IFRU{Вот в это}{Compilation results} (MSVC 2010):

\lstinputlisting{arrays/13_3_msvc.asm}

\IFRU{У меня оно при запуске выдало вот это:}{I'm running it, and I got:}

\begin{lstlisting}
a[100]=760826203
\end{lstlisting}

\IFRU{Это просто \IT{что-то}, что волею случая лежало в стеке рядом с массивом, 
через 400 байт от его первого элемента.}
{It is just \IT{something}, lying in the stack near to array, 400 bytes from its first element.}

\IFRU{Действительно, а как могло бы быть иначе? Компилятор мог бы встроить какой-то код, 
каждый раз проверяющий индекс на соответствие пределам массива, как в языках программирования 
более высокого уровня\footnote{Java, Python, итд}, что делало бы запускаемый код медленнее.}
{Indeed, how it could be done differently? Compiler may incorporate some code, checking index value to be always
in array's bound, like in higher-level programming languages\footnote{Java, Python, etc}, 
but this makes running code slower.}

\IFRU{Итак, мы прочитали какое-то число из стека явно \IT{нелегально}, а что если мы запишем?}
{OK, we read some values in stack \IT{illegally}, but what if we could write something to it?}

\IFRU{Вот что мы пишем:}{Here is what we will write:}

\begin{lstlisting}
#include <stdio.h>

int main() 
{
	int a[20];
	int i;

	for (i=0; i<30; i++)
		a[i]=i;

	return 0;
};
\end{lstlisting}

\IFRU{И вот что имеем на ассемблере:}{And what we've got:}

\lstinputlisting{\IFRU{arrays/13_1_ru.asm}{arrays/13_1_en.asm}}

\IFRU{Запускаете скомпилированную программу, и она падает. Немудрено. Но давайте теперь узнаем, где именно.}
{Run compiled program and its crashing. No wonder. Let's see, where exactly it's crashing.}

\IFRU{Отладчик я уже давно не использую, так как надоело для всяких мелких задач вроде подсмотреть состояние 
регистров, запускать что-то, двигать мышью, итд. 
Поэтому я написал очень минималистическую утилиту для себя, \IT{tracer}~\ref{tracer}, коей обхожусь.}
{I'm not using debugger anymore, because I tired to run it each time, move mouse, etc, when I need just to
spot some register's state at specific point.
That's why I wrote very minimalistic tool for myself, \IT{tracer}~\ref{tracer}, which is enough for my tasks.}

\IFRU{Помимо всего прочего, я могу использовать мою утилиту просто чтобы посмотреть 
где и какое исключение произошло. 
Итак, пробую:}
{I can also use it just to see, where debuggee is crashed.
So let's see:}

\begin{lstlisting}
generic tracer 0.4 (WIN32), http://conus.info/gt

New process: C:\PRJ\...\1.exe, PID=7988
EXCEPTION_ACCESS_VIOLATION: 0x15 (<symbol (0x15) is in unknown module>), ExceptionInformation[0]=8
EAX=0x00000000 EBX=0x7EFDE000 ECX=0x0000001D EDX=0x0000001D
ESI=0x00000000 EDI=0x00000000 EBP=0x00000014 ESP=0x0018FF48
EIP=0x00000015
FLAGS=PF ZF IF RF
PID=7988|Process exit, return code -1073740791
\end{lstlisting}

\IFRU{Итак, следите внимательно за регистрами.}
{So, please keep an eye on registers.}

\IFRU{Исключение произошло по адресу 0x15. Это явно нелегальный адрес для кода ~--- по крайней мере, win32-кода! 
Мы там как-то очутились, причем, сами того не хотели. Интересен также тот факт что в \EBP хранится 0x14, 
а в \ECX и \EDX ~--- 0x1D.}
{Exception occured at address 0x15. It's not legal address for code ~--- at least for win32 code!
We trapped there somehow against our will. It's also interesting fact that \EBP register contain 0x14,
\ECX and \EDX ~--- 0x1D.}

\IFRU{И еще немного изучим разметку стека.}{Let's study stack layout more.}

\IFRU{После того как управление передалось в \TT{\_main}, в стек было сохранено значение \EBP. 
Затем, для массива + переменной \IT{i} было выделено 84 байта. Это \TT{(20+1)*sizeof(int)}. 
\ESP сейчас указывает на переменную \TT{\_i} в локальном стеке и при исполнении следующего \TT{PUSH что-либо}, 
\IT{что-либо} появится рядом с \TT{\_i}.}
{After control flow was passed into \TT{\main}, \EBP register value was saved into stack.
Then, 84 bytes was allocated for array and \IT{i} variable. That's \TT{(20+1)*sizeof(int)}.
\ESP pointing now to \TT{\_i} variable in local stack and after execution of next \TT{PUSH something},
\IT{something} will be appeared next to \TT{\_i}.}

\IFRU{Вот так выглядит разметка стека пока управление находится внутри}
{That's stack layout while control is inside} \TT{\_main}:

\begin{center}
\begin{tabular}{ | l | l | }
\hline
  \TT{ESP}    & \IFRU{4 байта для \IT{i}}{4 bytes for \IT{i}} \\
  \TT{ESP+4}  & \IFRU{80 байт для массива \TT{a[20]}}{80 bytes for \TT{a[20]} array} \\
  \TT{ESP+84} & \IFRU{сохраненное значение \EBP}{saved \EBP value} \\
  \TT{ESP+88} & \IFRU{адрес возврата}{returning address} \\
\hline
\end{tabular}
\end{center}

\IFRU{Команда \TT{a[19]=чего\_нибудь} записывает последний \Tint в пределах массива (пока что в пределах!)}
{Instruction \TT{a[19]=something} writes last \Tint in array bounds (in bounds yet!)}

\IFRU{Команда \TT{a[20]=чего\_нибудь} записывает \IT{чего\_нибудь} на место где сохранено значение \EBP.}
{Instruction \TT{a[20]=something} writes \IT{something} to the place where \EBP value is saved.}

\IFRU{Обратите внимание на состояние регистров на момент падения процесса. В нашем случае, 
в 20-й элемент записалось значение 20. 
И вот все дело в том, что заканчиваясь, эпилог функции восстанавливал значение \EBP. 
(20 в десятичной системе это как раз 0x14 в шестнадцетиричной). 
Далее выполнилась инструкция \RET, которая на самом деле эквивалентна \TT{POP EIP}.}
{Please take a look at registers state at the crash moment. In our case,
number 20 was written to 20th element. 
By the function ending, function epilogue restore \EBP value.
(20 in decimal system is 0x14 in hexadecimal).
Then, \RET instruction was executed, which is equivalent to \TT{POP EIP} instruction.}

\IFRU{Инструкция \RET вытащила из стека адрес возврата (это адрес в какой-то CRT\footnote{C Run-Time}-функции, 
которая вызвала \TT{\_main}), 
а там было записано 21 в десятичной системе, то есть 0x15 в шестнадцетиричной. 
И вот процессор оказался по адресу 0x15, но исполняемого кода там нет, так что случилось исключение.}
{\RET instruction taking returning adddress from stack (that's address in some CRT\footnote{C Run-Time}-function,
which was called \TT{\_main}),
and 21 was stored there (0x15 in hexadecimal).
The CPU trapped at the address 0x15, but there are no executable code, so exception was raised.}

\IFRU{Добро пожаловать! Это называется}
{Welcome! It's called} \IT{buffer overflow}\footnote{\url{http://en.wikipedia.org/wiki/Stack_buffer_overflow}}.

\newcommand{\URLPHRACK}{\href{http://www.phrack.com/issues.html?issue=49&id=14}
{Smashing The Stack For Fun And Profit}}

\IFRU{Замените массив \Tint на строку (массив \Tchar), нарочно создайте слишком длинную строку, 
просуньте её в ту программу, 
в ту функцию, которая не проверяя длину строки скопирует её в слишком короткий буфер, 
и вы сможете указать программе, по какому именно адресу перейти. 
Не все так просто в реальности, конечно, но началось все с этого\footnote{Классическая статья об этом: \URLPHRACK}.}
{Replace \Tint array by string (\Tchar array), create a long string deliberately,
pass it to the program, to the function which is not checking string length and copies it to short buffer,
and you'll able to point to a program an address to which it should jump.
Not that simple in reality, but that's how it was emerged\footnote{Classic article about it: \URLPHRACK}.}

\newcommand{\URLWPB}{\href{http://en.wikipedia.org/wiki/Buffer_overflow_protection}
{Wikipedia: \IFRU{описания защит, которые компилятор может вставлять в код}
{compiler-side buffer overflow protection methods}}}

\IFRU{В наше время пытаются бороться с этой напастью, не взирая на халатность программистов на \CCpp. 
В MSVC есть опции вроде\footnote{\URLWPB}:}
{There are several methods to protect against it, regardless of \CCpp programmers' negligence.
MSVC has options like\footnote{\URLWPB}:}

\begin{verbatim}
 /RTCs Stack Frame runtime checking
 /GZ Enable stack checks (/RTCs)
\end{verbatim}

\IFRU{Один из методов, это в прологе функции вставлять в область локальных переменных 
некоторое случайное значение 
и в эпилоге функции, перед выходом, это число проверять. 
И если проверка не прошла, то не выполнять инструкцию \RET а остановиться (или зависнуть). 
Процесс зависнет, но это лучше чем удаленная атака на ваш хост.}
{One of the methods is to write random value among local variables to stack at function prologue 
and to check it in function epilogue before function exiting.
And if value is not the same, do not execute last instruction \RET, but halt (or hang).
Process will hang, but that's much better then remote attack to your host.}

\IFRU{Попробуем то же самое в GCC 4.4.1. У нас выходит такое:}{Let's try the same code in GCC 4.4.1. We got:}

\lstinputlisting{arrays/13_4_gcc.asm}

\IFRU{Запуск этого в Linux выдаст:}{Running this in Linux will produce:} \TT{Segmentation fault}.

\IFRU{Если запустить полученное в отладчике GDB, получим:}
{If we run this in GDB debugger, we getting this:}

\begin{lstlisting}
(gdb) r
Starting program: /home/dennis/RE/1 

Program received signal SIGSEGV, Segmentation fault.
0x00000016 in ?? ()
(gdb) info registers
eax            0x0	0
ecx            0xd2f96388	-755407992
edx            0x1d	29
ebx            0x26eff4	2551796
esp            0xbffff4b0	0xbffff4b0
ebp            0x15	0x15
esi            0x0	0
edi            0x0	0
eip            0x16	0x16
eflags         0x10202	[ IF RF ]
cs             0x73	115
ss             0x7b	123
ds             0x7b	123
es             0x7b	123
fs             0x0	0
gs             0x33	51
(gdb) 
\end{lstlisting}

\IFRU{Значения регистров немного другие чем в примере win32, это потому что разметка стека чуть другая.}
{Register values are slightly different then in win32 example, 
that's because stack layout is slightly different too.}

\subsection{\IFRU{Еще немного о массивах}{One more word about arrays}}

\IFRU{Теперь понятно, почему нельзя написать в исходном коде на \CCpp что-то вроде:
\footnote{GCC способен это сделать выделяя место под массив динамически в стеке (как alloca()), 
но это расширение не является частью стандарта}}
{Now we understand, why it's not possible to write something like that in \CCpp code
\footnote{GCC can actually do this by allocating array dynammically in stack (like alloca()), 
but it's not standard langauge extension}:}

\begin{lstlisting}
void f(int size)
{
    int a[size];
...
};
\end{lstlisting}

\IFRU{Все просто потому, чтобы выделять место под массив в локальном стеке или же сегменте данных 
(если массив глобальный), компилятору нужно знать его размер, чего он, на стадии компиляции, 
разумеется знать не может.}
{That's just because compiler should know exact array size to allocate place for it in local stack layout or
in data segment (in case of global variable) on compiling stage.}

\IFRU{Если вам нужен массив произвольной длины, то выделите столько, сколько нужно, через \TT{malloc()}, 
затем обращайтесь к выделенному блоку байт как к массиву того типа, который вам нужен.}
{If you need array of arbitrary size, allocate it by \TT{malloc()}, then access allocated memory block
as array of variables of type you need.}

\subsection{\IFRU{Многомерные массивы}{Multidimensional arrays}}

\IFRU{Многомерный массив выглядит внутри так же как и линейный.}
{Internally, multidimensional array is essentially the same thing as linear array.}

\IFRU{Попробуем:}{Let's see:}

\begin{lstlisting}
#include <stdio.h>

int a[10][20][30];

void insert(int x, int y, int z, int value)
{
	a[x][y][z]=value;
};
\end{lstlisting}

\IFRU{В итоге}{We got} (MSVC 2010):

\lstinputlisting{arrays/13_5_msvc.asm}

\IFRU{В принципе, ничего удивительного. В \TT{insert()} для индексирования нужного элемента массива, 
три входных аргумента перемножаются так, чтобы представить массив трехмерным.}
{Nothing special. For index calculation, three input arguments are multiplying 
in such way to represent array as multidimensional.}

GCC 4.4.1:

\lstinputlisting{arrays/13_5_gcc.asm}

