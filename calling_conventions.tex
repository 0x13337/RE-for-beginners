% done

\section{\IFRU{Способы передачи аргументов при вызове функций}{Arguments passing methods (calling conventions)}}
\label{sec:callingconventions}

\subsection{cdecl}

\IFRU{Этот способ передачи аргументов через стек чаще всего используется в языках \CCpp.}
{This is the most popular method for arguments passing to functions in \CCpp languages.}

\IFRU{Вызывающая функция заталкивает в стек аргументы в обратном порядке: сначала последний аргумент в стек, 
затем предпоследний, и в самом конце ~--- первый аргумент. 
Вызывающая функция должна также затем вернуть указатель \ESP в нормальное состояние, 
после возврата вызываемой функции.}
{Caller pushing arguments to stack in reverse order: last argument, then penultimate element 
and finally ~--- first argument.
Caller also should return back \ESP to its initial state after callee function exit.}

\begin{lstlisting}
push arg3
push arg2
push arg3
call function
add esp, 12 ; return ESP
\end{lstlisting}

\subsection{stdcall}
\label{stdcall}

\newcommand{\SIZEOFINT}{\IFRU{Размер переменной типа \Tint ~--- 4 в x86-системах и 8 в x64-системах}
{Size of \Tint type variable is 4 in x86 systems and 8 in x64 systems}}

\IFRU{Это почти то же что и \IT{cdecl}, за исключением того что вызываемая функция сама возвращает \ESP 
в нормальное состояние, выполнив инструкцию \TT{RET x} вместо \RET, где 
\TT{x = количество\_аргументов * sizeof(int)\footnote{\SIZEOFINT}}.
Вызывающая функция не будет корректировать указатель стека при помощи инструкции \TT{add esp, x}.}
{Almost the same thing as \IT{cdecl}, with the exception that callee set \ESP to initial state executing \TT{RET x} instruction instead of \RET, where
\TT{x = arguments number * sizeof(int)\footnote{\SIZEOFINT}}.
Caller will not adjust stack pointer by \TT{add esp, x} instruction.}

\begin{lstlisting}
push arg3
push arg2
push arg1
call function

function:
... do something ...
ret 12
\end{lstlisting}

\IFRU{Этот способ используется почти везде в системных библиотеках win32, но не в win64 (о win64 смотрите ниже).}
{This method is ubiquitous in win32 standard libraries, but not in win64 (see below about win64).}

\subsubsection{\IFRU{Функции с переменным количеством аргументов}{Variable arguments number functions}}

\IFRU{Функции вроде \printf, должно быть, единственный случай функций в \CCpp с переменным количеством аргументов,
но с их помощью можно легко проследить очень важную разницу между \IT{cdecl} и \IT{stdcall}.
Начнем с того, что компилятор знает сколько аргументов было у \printf.}
{\printf-like functions are, probably, the only case of variable arguments functions in \CCpp,
but it's easy to illustrate an important difference between \IT{cdecl} and \IT{stdcall} with help of it.
Let's start with the idea that compiler knows argument count of each \printf function calling.}
\IFRU{Однако, вызываемая функция \printf, которая уже давно скомпилена 
и находится в системной библиотеке MSVCRT.DLL (если говорить о Windows), 
не знает сколько аргументов ей передали, хотя может установить их количество по строке формата.}
{However, called \printf, which is already compiled and located in MSVCRT.DLL (if to talk about Windows),
do not have information about how much arguments were passed, however it can determine it from format string.}
\IFRU{Таким образом, если бы \printf была \IT{stdcall}-функцией и возвращала указатель стека 
в первоначальное состояние 
подсчитав количество аргументов в строке формата, это была бы потенциально опасная ситуация, 
когда одна опечатка программиста могла бы вызывать неожиданные падения программы. 
Таким образом, для таких функций \IT{stdcall} явно не подходит, а подходит \IT{cdecl}.}
{Thus, if \printf would be \IT{stdcall}-function and restored stack pointer to its initial state by counting
number of arguments in format string, this could be dangerous situation, when one programmer's typo may
provoke sudden program crash.
Thus it's not suitable for such functions to use \IT{stdcall}, \IT{cdecl} is better.}

\subsection{fastcall}
\label{fastcall}

\IFRU{Это общее название для передачи некоторых аргументов через регистры а всех остальных ~--- через стек. 
На более старых процессорах, это работало потенциально быстрее чем \IT{cdecl}/\IT{stdcall}.}
{That's general naming for a method for passing some of arguments via registers and all others ~--- via stack.
It worked faster than \IT{cdecl}/\IT{stdcall} on older CPUs.}
\IFRU{Это не стандартизированый способ, поэтому разные компиляторы делают это по-своему. 
Разумеется, если у вас есть, скажем, две DLL, одна использует другую, и обе они собраны с \IT{fastcall}
но разными компиляторами, возможно будут проблемы.}
{It's not a standardized way, so, different compilers may do it differently.
Of course, if you have two DLLs, one use another, and they are built by different compilers with \IT{fastcall}
calling conventions, there will be a problems.}

\IFRU{MSVC и GCC передает первый и второй аргумент через \ECX и \EDX а остальные аргументы через стек. 
Вызываемая функция возвращает указатель стека в первоначальное состояние.}
{Both MSVC and GCC passing first and second argument via \ECX and \EDX and other arguments via stack.
Caller should restore stack pointer into initial state.}

\IFRU{Указатель стека должен быть возвращен в первоначальное состояние вызываемой функцией, 
как в случае \IT{stdcall}.}
{Stack pointer should be restored to initial state by callee, like in \IT{stdcall}.}

\begin{lstlisting}
push arg3
mov edx, arg2
mov ecx, arg1
call function

function:
.. do something ..
ret 4
\end{lstlisting}

\subsubsection{GCC regparm}

\newcommand{\URLREGPARMM}{\url{http://www.ohse.de/uwe/articles/gcc-attributes.html\#func-regparm}}

\IFRU{Это в некотором роде, развитие \IT{fastcall}\footnote{\URLREGPARMM}. 
Опцией \TT{-mregparm=x} можно указывать, 
сколько аргументов компилятор будет передавать через регистры. Максимально 3. 
В этом случае будут задействованы регистры \EAX, \EDX и \ECX.}
{It's \IT{fastcall} evolution\footnote{\URLREGPARMM} is some sense.
With the \TT{-mregparm} option it's possible to set, how many arguments will be passed via registers. 
3 at maximum.
Thus, \EAX, \EDX and \ECX registers will be used.}

\IFRU{Разумеется, если аргументов у функции меньше трех, то будет задействована только часть регистров.}
{Of course, if number of arguments is less then 3, not all registers 3 will be used.}

\IFRU{Вызывающая функция возвращает указатель стека в первоначальное состояние.}
{Caller restores stack pointer to its initial state.}

% TODO: example

\subsection{thiscall}
\label{thiscall}

\IFRU{В С++, это передача в функцию-метод указателя \IT{this} на объект.}
{In C++, it's a \IT{this} pointer to object passing into function-method.}

\IFRU{В MSVC указатель \IT{this} обычно передается в регистре \ECX.}
{In MSVC, \IT{this} is usually passed in \ECX register.}

\IFRU{В GCC указатель \IT{this} обычно передается как самый первый аргумент. 
Таким образом, внутри будет видно, что у всех функций-методов на один аргумент больше.}
{In GCC, \IT{this} pointer is passed as a first function-method argument.
Thus it will be seen that internally all function-methods has extra argument for it.}

% TODO: example

\subsection{x86-64}

\subsubsection{win64}

\IFRU{В win64 метод передачи всех параметров немного похож на \TT{fastcall}. 
Первые 4 аргумента записываются в регистры \TT{RCX}, \TT{RDX}, \TT{R8}, \TT{R9}, а остальные ~--- в стек. 
Вызывающая функция также должна подготовить место из 32 байт или для четырех 64-битных значений, 
чтобы вызываемая функция могла сохранить там первые 4 аргумента. 
Короткие функции могут использовать переменные прямо из регистров, 
но б\'{о}льшие могут сохранять их значения на будущее.}
{The method of arguments passing in Win64 is somewhat resembling to \TT{fastcall}.
First 4 arguments are passed via \TT{RCX}, \TT{RDX}, \TT{R8}, \TT{R9}, others ~--- via stack.
Caller also must prepare a place for 32 bytes or 4 64-bit values,
so then callee can save there first 4 arguments.
Short functions may use argument values just from registers,
but larger may save its values for further use.}

\IFRU{Вызывающая функция должна вернуть указатель стека в первоначальное состояние.}
{Caller also should return stack pointer into initial state.}

\IFRU{Это же соглашение используется и в системных библиотеках Windows x86-64 (вместо \IT{stdcall} в win32).}
{This calling convention is also used in Windows x86-64 system DLLs (instead if \IT{stdcall} in win32).}

% TODO: example

\subsection{\IFRU{Возвращение переменных типа \Tfloat, \Tdouble}{Returning values of \Tfloat and \Tdouble type}}

\IFRU{Во всех соглашениях кроме Win64, переменная типа \Tfloat или \Tdouble возвращается через регистр FPU \STZERO.}
{In all conventions except of Win64, values of type \Tfloat or \Tdouble returning via FPU register \STZERO.}

\IFRU{В Win64 переменные типа \Tfloat и \Tdouble возвращаются в регистре \XMMZERO вместо \STZERO.}
{In Win64, return values of \Tfloat and \Tdouble types are returned in \XMMZERO register instead of \STZERO.}

