% done

\section{\IFRU{Классы в Си++}{C++ classes}}

\IFRU{Я преднамеренно расположил описание классов здесь сразу за структурами, 
потому что внутреннее представление классов в Си++ почти такое же как и представление структур.}
{I placed a C++ classes description here intentionally after structures description,
because internally, C++ classes representation is almost the same as structures representation.}

\IFRU{Давайте попробуем простой пример с парой переменных, парой конструкторов и одним методом:}
{Let's try an example with two variables, couple of constructors and one method:}

\lstinputlisting{classes/16_1.cpp}

\IFRU{Вот как выглядит \main на ассемблере:}{Here is how \main function looks like translated into assembler:}

\lstinputlisting{classes/16_1_msvc.asm}

\IFRU{Вот что происходит. 
Под каждый экземпляр класса \IT{c} выделяется по 8 байт, столько же, сколько нужно 
для хранения двух переменных.}
{So what's going on.
For each object (instance of class \IT{c}) 8 bytes allocated, that's exactly size of 2 variables storage.}

\IFRU{Для \IT{c1} вызывается конструктор по умолчанию без аргументов \TT{??0c@@QAE@XZ}. 
Для \IT{c2} вызывается другой конструктор \TT{??0c@@QAE@HH@Z} и передаются два числа в качестве аргументов.}
{For \IT{c1} a default argumentless constructor \TT{??0c@@QAE@XZ} is called.
For \IT{c2} another constructor \TT{??0c@@QAE@HH@Z} is called and two numbers are passed as arguments.}

\IFRU{А указатель на объект (\IT{this} в терминологии Си++) передается в регистре \ECX. 
Это называется thiscall~\ref{thiscall} ~--- метод передачи указателя на объект.}
{A pointer to object (\IT{this} in C++ terminology) is passed in \ECX register.
This is called thiscall~\ref{thiscall} ~--- a pointer to object passing method.}

\IFRU{В данном случае, MSVC делает это через \ECX. Необходимо помнить, что это не стандартизированный метод, 
и другие компиляторы могут делать это иначе, например через первый аргумент функции (как GCC).}
{MSVC doing it using \ECX register. Needless to say, it's not a standardized method, other compilers could do it
differently, for example, via first function argument (like GCC).}

%\newcommand{\URLNM}{\href{http://en.wikipedia.org/wiki/Visual_C\%2B\%2B_name_mangling}{Wikipedia: Visual C++ name mangling}}
\newcommand{\URLNM}{\href{http://en.wikipedia.org/wiki/Name_mangling}{Wikipedia: Name mangling}}

\IFRU{Почему у имен функций такие странные имена? Это \IT{name mangling}\footnote{\URLNM}.}
{Why these functions has so odd names? That's \IT{name mangling}\footnote{\URLNM}.}

\IFRU{В Си++, у класса, может иметься несколько методов с одинаковыми именами 
но аргументами разных типов ~--- это полиморфизм. 
Ну и конечно, у разных классов могут быть методы с одинаковыми именами.}
{C++ class may contain several methods sharing the same name but having different arguments ~--- 
that's polymorphism.
And of course, different classes may own methods sharing the same name.}

\IFRU{\IT{Name mangling} позволяет закодировать имя класса + имя метода + типы всех аргументов метода 
в одной ASCII-строке, которая затем используется как внутреннее имя функции. 
Это все потому что ни линкер, ни загрузчик DLL операционной системы 
(мангленные имена могут быть среди экспортов/импортов в DLL), 
ничего не знают о Си++ или ООП.}
{\IT{Name mangling} allows to encode class name + method name + all method argument types 
in one ASCII-string, which will be used as internal function name.
That's all because neither linker, nor DLL operation system loader (mangled names may be among 
DLL exports as well) knows nothing about C++ or OOP.}

\IFRU{Далее вызывается два раза \TT{dump()}.}{\TT{dump()} function called two times after.}

\IFRU{Теперь смотрим на код в конструкторах:}{Now let's see constructors' code:}

\lstinputlisting{classes/16_2_msvc.asm}

\IFRU{Конструкторы это просто функции, они используют указатель на структуру в \ECX, 
перекладывают его себе в локальную переменную, хотя это и не обязательно.}
{Constructors are just functions, they use pointer to structure in \ECX,
moving the pointer into own local variable, however, it's not necessary.}

\IFRU{И еще метод \TT{dump()}:}{Now \TT{dump()} method:}

\lstinputlisting{classes/16_3_msvc.asm}

\IFRU{Все очень просто, \TT{dump()} берет указатель на структуру состоящую из двух \Tint через \ECX, 
выдергивает оттуда две переменные и передает их в \printf.}
{Simple enough: \TT{dump()} taking pointer to the structure containing two \Tint's in \ECX,
takes two values from it and passing it into \printf.}

\IFRU{А если скомпилировать с оптимизацией (\Ox), то будет намного меньше всего:}
{The code is much shorter if compiled with optimization (\Ox):}

\lstinputlisting{classes/16_4_msvc_Ox.asm}

\IFRU{Вот и все. Единственное о чем еще нужно сказать, это о том что в функции \main, 
когда вызывался второй конструктор с двумя аргументами, за ним не корректировался стек при помощи 
\TT{add esp, X}. В то же время, в конце у конструктора вместо \RET имеется \TT{RET 8}.}
{That's all. One more thing to say is that stack pointer after second constructor calling wasn't corrected
with \TT{add esp, X}. Please also note that, constructor has \TT{ret 8} instead of \RET at the end.}

\IFRU{Это потому что здесь используется thiscall~\ref{thiscall}, который, вместе с stdcall~\ref{stdcall} 
(все это ~--- методы передачи аргументов через стек), предлагает вызываемой функции корректировать стек. 
Инструкция \TT{ret X} сначала прибавляет \TT{X} к \ESP, затем передает управление вызывающей функции.}
{That's all because here used thiscall~\ref{thiscall} calling convention, the method of passing values through the
stack, which is, together with stdcall~\ref{stdcall} method, offers to correct stack to callee 
rather then to caller.
\TT{ret x} instruction adding \TT{X} to \ESP, then passes control to caller function.}

\IFRU{См.также в соответствующем разделе о способах передачи аргументов через стек}
{See also section about calling conventions}~\ref{sec:callingconventions}.

\IFRU{Еще, кстати, нужно отметить, что именно компилятор решает, когда вызывать конструктор и деструктор ~--- 
но это итак известно из основ языка Си++.}
{It's also should be noted that compiler deciding when to call constructor and destructor ~--- but that's 
we already know from C++ language basics.}

\subsection{GCC}

\IFRU{В GCC 4.4.1 все почти так же, за исключением некоторых различий.}
{It's almost the same situation in GCC 4.4.1, with few exceptions.}

\lstinputlisting{classes/16_5_gcc.asm}

\newcommand{\URLAGNER}{\url{http://www.agner.org/optimize/calling_conventions.pdf}}

\IFRU{Здесь мы видим что применяется иной \IT{name mangling} характерный для стандартов 
GNU\footnote{Еще о name mangling разных компиляторов: \URLAGNER}. Во-вторых, указатель на экземпляр передается как первый аргумент функции ~--- конечно же, скрыто от программиста.}
{Here we see another \IT{name mangling} style, specific to GNU\footnote{One more document about different compilers name mangling types: \URLAGNER} standards. It's also can be noted that pointer to object is passed as first function argument ~--- hiddenly from programmer, of course.}

\IFRU{Это первый конструктор:}{First constructor:}

\begin{lstlisting}
                public _ZN1cC1Ev ; weak
_ZN1cC1Ev       proc near               ; CODE XREF: main+10

arg_0           = dword ptr  8

                push    ebp
                mov     ebp, esp
                mov     eax, [ebp+arg_0]
                mov     dword ptr [eax], 667
                mov     eax, [ebp+arg_0]
                mov     dword ptr [eax+4], 999
                pop     ebp
                retn
_ZN1cC1Ev       endp
\end{lstlisting}

\IFRU{Он просто записывает два числа по указателю переданному в первом (и единственном) аргументе.}
{What it does is just writes two numbers using pointer passed in first (and sole) argument.}

\IFRU{Второй конструктор:}{Second constructor:}

\begin{lstlisting}
                public _ZN1cC1Eii
_ZN1cC1Eii      proc near

arg_0           = dword ptr  8
arg_4           = dword ptr  0Ch
arg_8           = dword ptr  10h

                push    ebp
                mov     ebp, esp
                mov     eax, [ebp+arg_0]
                mov     edx, [ebp+arg_4]
                mov     [eax], edx
                mov     eax, [ebp+arg_0]
                mov     edx, [ebp+arg_8]
                mov     [eax+4], edx
                pop     ebp
                retn
_ZN1cC1Eii      endp
\end{lstlisting}

\IFRU{Эта функция, аналог которой мог бы выглядеть так:}{This is a function, analog of which could be looks like:}

\begin{lstlisting}
void ZN1cC1Eii (int *obj, int a, int b)
{
  *obj=a;
  *(obj+1)=b;
};
\end{lstlisting}

\IFRU{... что, в общем, предсказуемо.}{... and that's completely predictable.}

\IFRU{И функция \TT{dump()}:}{Now \TT{dump()} function:}

\begin{lstlisting}
                public _ZN1c4dumpEv
_ZN1c4dumpEv    proc near

var_18          = dword ptr -18h
var_14          = dword ptr -14h
var_10          = dword ptr -10h
arg_0           = dword ptr  8

                push    ebp
                mov     ebp, esp
                sub     esp, 18h
                mov     eax, [ebp+arg_0]
                mov     edx, [eax+4]
                mov     eax, [ebp+arg_0]
                mov     eax, [eax]
                mov     [esp+18h+var_10], edx
                mov     [esp+18h+var_14], eax
                mov     [esp+18h+var_18], offset aDD ; "%d; %d\n"
                call    _printf
                leave
                retn
_ZN1c4dumpEv    endp
\end{lstlisting}

\IFRU{Эта функция \IT{во внутреннем представлении} имеет один аргумент, через который передается указатель на 
объект\footnote{экземпляр класса} (\IT{this}).}
{This function in its \IT{internal representation} has sole argument, 
used as pointer to the object (\IT{this}).}

\IFRU{Таким образом, если брать в учет только эти простые примеры, разница между MSVC и GCC 
в способе кодирования имен функций (\IT{name mangling}) и передаче указателя на экземпляр класса 
(через \ECX или через первый аргумент).}
{Thus, if to base our judgment on these simple examples, the difference between MSVC and GCC
is style of function names encoding (\IT{name mangling}) and passing pointer to object
(via \ECX register or via first argument).}
