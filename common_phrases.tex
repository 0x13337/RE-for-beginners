% for index
\newcommand{\GrepUsage}{\IFRU{Использование grep}{grep usage}}
\newcommand{\SyntacticSugar}{\IFRU{Синтаксический сахар}{Syntactic Sugar}}
\newcommand{\CompilerAnomaly}{\IFRU{Аномалии компиляторов}{Compiler's anomalies}}
\newcommand{\CLanguageElements}{\IFRU{Элементы языка Си}{C language elements}}
\newcommand{\CStandardLibrary}{\IFRU{Стандартная библиотека Си}{C standard library}}
\newcommand{\Flags}{\IFRU{Флаги}{Flags}}
\newcommand{\Registers}{\IFRU{Регистры}{Registers}}
\newcommand{\Stack}{\IFRU{Стек}{Stack}}
\newcommand{\Recursion}{\IFRU{Рекурсия}{Recursion}}
\newcommand{\RAM}{\IFRU{ОЗУ}{RAM}}
\newcommand{\ROM}{\IFRU{ПЗУ}{ROM}}
\newcommand{\Pointers}{\IFRU{Указатели}{Pointers}}

\newcommand{\Task}{\IFRU{Задача}{Task}\xspace}
\newcommand{\CCpp}{\IFRU{Си/Си++}{C/C++}\xspace}
\newcommand{\NonOptimizing}{\IFRU{Неоптимизирующий}{Non-optimizing}\xspace}
\newcommand{\Optimizing}{\IFRU{Оптимизирующий}{Optimizing}\xspace}
\newcommand{\NonOptimizingKeil}{\NonOptimizing Keil\xspace}
\newcommand{\OptimizingKeil}{\Optimizing Keil\xspace}
\newcommand{\NonOptimizingXcode}{\NonOptimizing Xcode (LLVM)\xspace}
\newcommand{\OptimizingXcode}{\Optimizing Xcode (LLVM)\xspace}
\newcommand{\ARMMode}{\IFRU{Режим ARM}{ARM mode}\xspace}
\newcommand{\ThumbMode}{\IFRU{Режим thumb}{thumb mode}\xspace}
\newcommand{\ThumbTwoMode}{\IFRU{Режим thumb-2}{thumb-2 mode}\xspace}

\newcommand{\FNQUOTIENT}{\footnote{\IFRU{результат деления}{result of division}}}
\newcommand{\FNPRODUCT}{\footnote{\IFRU{результат умножения}{result of multiplication}}}
\newcommand{\FNSUM}{\footnote{\IFRU{результат сложения}{result of addition}}}

\newcommand{\DataProcessingInstructionsFootNote}{\IFRU{Эти инструкции также называются}
{These instructions are also called} ``data processing instructions''}

\newcommand{\Instructions}{\IFRU{Инструкции}{Instructions}}

% section names
\newcommand{\ShiftsSectionName}{\IFRU{Сдвиги}{Shifts}}
\newcommand{\SignedNumbersSectionName}{\IFRU{Представление знака в числах}{Signed number representations}}
\newcommand{\HelloWorldSectionName}{Hello, world!}
\newcommand{\SwitchCaseDefaultSectionName}{switch()/case/default}
\newcommand{\PrintfSeveralArgumentsSectionName}{\printf \IFRU{с несколькими агрументами}{with several arguments}}
\newcommand{\DivisionByNineSectionName}{\IFRU{Деление на 9}{Division by 9}}
\newcommand{\WorkingWithFloatAsWithStructSubSubSectionName}{\IFRU
{Работа с типом float как со структурой}{Working with the float type as with a structure}}

\newcommand{\StructurePackingSectionName}{\IFRU{Упаковка полей в структуре}{Fields packing in structure}}

\newcommand{\PICcode}{\IFRU{адресно-независимый код}{position-independent code}}
\newcommand{\CapitalPICcode}{\IFRU{Адресно-независимый код}{Position-independent code}}

% C
\newcommand{\PostIncrement}{\IFRU{Пост-инкремент}{Post-increment}}
\newcommand{\PostDecrement}{\IFRU{Пост-декремент}{Post-decrement}}
\newcommand{\PreIncrement}{\IFRU{Пре-инкремент}{Pre-increment}}
\newcommand{\PreDecrement}{\IFRU{Пре-декремент}{Pre-decrement}}

% other
\newcommand{\IntelSyntax}{\IFRU{Синтаксис Intel}{Intel syntax}}
\newcommand{\ATTSyntax}{\IFRU{Синтаксис AT\&T}{AT\&T syntax}}

