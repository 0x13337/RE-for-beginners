% for index
\newcommand{\GrepUsage}{\RU{Использование grep}\EN{grep usage}}
\newcommand{\SyntacticSugar}{\RU{Синтаксический сахар}\EN{Syntactic Sugar}}
\newcommand{\CompilerAnomaly}{\RU{Аномалии компиляторов}\EN{Compiler's anomalies}}
\newcommand{\CLanguageElements}{\RU{Элементы языка Си}\EN{C language elements}}
\newcommand{\CStandardLibrary}{\RU{Стандартная библиотека Си}\EN{C standard library}}
\newcommand{\Flags}{\RU{Флаги}\EN{Flags}}
\newcommand{\Registers}{\RU{Регистры}\EN{Registers}}
\newcommand{\Stack}{\RU{Стек}\EN{Stack}}
\newcommand{\Recursion}{\RU{Рекурсия}\EN{Recursion}}
\newcommand{\RAM}{\RU{ОЗУ}\EN{RAM}}
\newcommand{\ROM}{\RU{ПЗУ}\EN{ROM}}
\newcommand{\Pointers}{\RU{Указатели}\EN{Pointers}}
\newcommand{\BufferOverflow}{\RU{Переполнение буфера}\EN{Buffer Overflow}}

\newcommand{\Exercise}{\RU{Упражнение}\EN{Exercise}\xspace}
\newcommand{\Exercises}{\RU{Упражнения}\EN{Exercises}\xspace}
\newcommand{\Arrays}{\RU{Массивы}\EN{Arrays}}
\newcommand{\Cpp}{\RU{Си++}\EN{C++}\xspace}
\newcommand{\CCpp}{\RU{Си/Си++}\EN{C/C++}\xspace}
\newcommand{\NonOptimizing}{\RU{Неоптимизирующий}\EN{Non-optimizing}\xspace}
\newcommand{\Optimizing}{\RU{Оптимизирующий}\EN{Optimizing}\xspace}
\newcommand{\NonOptimizingKeil}{\NonOptimizing Keil\xspace}
\newcommand{\OptimizingKeil}{\Optimizing Keil\xspace}
\newcommand{\NonOptimizingXcode}{\NonOptimizing Xcode (LLVM)\xspace}
\newcommand{\OptimizingXcode}{\Optimizing Xcode (LLVM)\xspace}
\newcommand{\ARMMode}{\RU{Режим ARM}\EN{ARM mode}\xspace}
\newcommand{\ThumbMode}{\RU{Режим thumb}\EN{thumb mode}\xspace}
\newcommand{\ThumbTwoMode}{\RU{Режим thumb-2}\EN{thumb-2 mode}\xspace}
\newcommand{\AndENRU}{\RU{и}\EN{and}\xspace}
\newcommand{\OrENRU}{\RU{или}\EN{or}\xspace}
\newcommand{\InENRU}{\RU{в}\EN{in}\xspace}
\newcommand{\ForENRU}{\RU{для}\EN{for}\xspace}
\newcommand{\LineENRU}{\RU{строка}\EN{line}\xspace}

\newcommand{\FNQUOTIENT}{\footnote{\RU{результат деления}\EN{result of division}}}
\newcommand{\FNPRODUCT}{\footnote{\RU{результат умножения}\EN{result of multiplication}}}
\newcommand{\FNSUM}{\footnote{\RU{результат сложения}\EN{result of addition}}}

\newcommand{\DataProcessingInstructionsFootNote}{\RU{Эти инструкции также называются}
\EN{These instructions are also called} ``data processing instructions''}

\newcommand{\Instructions}{\RU{Инструкции}\EN{Instructions}}

% for .bib files
\newcommand{\AlsoAvailableAs}{\RU{Также доступно здесь:}\EN{Also available as}\xspace}

% section names
\newcommand{\ShiftsSectionName}{\RU{Сдвиги}\EN{Shifts}}
\newcommand{\SignedNumbersSectionName}{\RU{Представление знака в числах}\EN{Signed number representations}}
\newcommand{\HelloWorldSectionName}{Hello, world!}
\newcommand{\SwitchCaseDefaultSectionName}{switch()/case/default}
\newcommand{\PrintfSeveralArgumentsSectionName}{\printf \RU{с несколькими аргументами}\EN{with several arguments}}
\newcommand{\DivisionByNineSectionName}{\RU{Деление на 9}\EN{Division by 9}}
\newcommand{\WorkingWithFloatAsWithStructSubSubSectionName}{
\RU{Работа с типом float как со структурой}\EN{Working with the float type as with a structure}}

\newcommand{\StructurePackingSectionName}{\RU{Упаковка полей в структуре}\EN{Fields packing in structure}}

\newcommand{\PICcode}{\RU{адресно-независимый код}\EN{position-independent code}}
\newcommand{\CapitalPICcode}{\RU{Адресно-независимый код}\EN{Position-independent code}}
\newcommand{\Loops}{\RU{Циклы}\EN{Loops}}

% C
\newcommand{\PostIncrement}{\RU{Пост-инкремент}\EN{Post-increment}}
\newcommand{\PostDecrement}{\RU{Пост-декремент}\EN{Post-decrement}}
\newcommand{\PreIncrement}{\RU{Пре-инкремент}\EN{Pre-increment}}
\newcommand{\PreDecrement}{\RU{Пре-декремент}\EN{Pre-decrement}}

% other
\newcommand{\IntelSyntax}{\RU{Синтаксис Intel}\EN{Intel syntax}}
\newcommand{\ATTSyntax}{\RU{Синтаксис AT\&T}\EN{AT\&T syntax}}
\newcommand{\randomNoise}{\RU{случайный шум}\EN{random noise}}
\newcommand{\Example}{\RU{Пример}\EN{Example}}
\newcommand{\argument}{\RU{аргумент}\EN{argument}}
\newcommand{\MarkedInIDAAs}{\RU{маркируется в \IDA как}\EN{marked in \IDA as}}
\newcommand{\HERMIT}{\RU{Андрей}\EN{Andrey} ``herm1t'' \RU{Баранович}\EN{Baranovich}}
\newcommand{\stepover}{\RU{сделать шаг не входя в ф-цию}\EN{step over}}
\newcommand{\ShortHotKeyCheatsheet}{\RU{Краткий справочник хот-кеев}\EN{Short hot-keys cheatsheet}}

