\subsection{std::list}
\index{\Cpp!STL!std::list}

\IFRU{Хорошо известный всем двусвязный список: каждый элемент имеет два указателя, на следующий и на предыдущий
элементы}{This is a well-known doubly-linked list: each element has two pointers, to the previous and the next
elements}.

\IFRU{Это означает что расход памяти увеличивается на 2 слова на каждый элемент (8 байт в 32-битной среде или
16 байт в 64-битной)}
{This mean that a memory footprint is enlarged by 2 words for each element (8 bytes in 32-bit environment or 16
bytes in 64-bit)}.

\IFRU{Это также циркулярный список, что означает что последний элемент имеет указатель на первый и наоборот}
{This is also a circular list, meaning that the last element has a pointer to the first and vice versa}.

\IFRU{C++ STL просто добавляет указатели ``next'' и ``previous'' к той вашей структуре, которую вы желаете
объеденить в список}
{C++ STL just append ``next'' and ``previous'' pointers to your existing structure you wish to 
unite into a list}.

\IFRU{Попробуем разобраться с примером в котором простая структура из двух переменных, мы объеденим её в список}
{Let's work out an example with a simple 2-variable structure we want to store in the list}.

\IFRU{Хотя и стандарт Си++ \cite{CPP11} не предлагает, как он должен быть реализован, реализации
MSVC и GCC прямолинейны и похожи друг на друга, так что этот исходный код для обоих}
{Although standard C++ standard \cite{CPP11} does not offer how to implement it, 
MSVC and GCC implementations are straightforward and similar to each other, so here is only one
source code for both}:

\lstinputlisting{cpp/STL/list/2.cpp}

\subsubsection{GCC}

\IFRU{Начнем с}{Let's start with} GCC.

\IFRU{При запуске увидим длинный вывод, будем разбирать его по частям}
{When we run the example, we'll see a long dump, let's work with it part by part}.

\begin{lstlisting}
* empty list:
ptr=0x0028fe90 _Next=0x0028fe90 _Prev=0x0028fe90 x=3 y=0
\end{lstlisting}

\IFRU{Видим пустой список}{Here we see an empty list}.
\IFRU{Не смотря на то что он пуст, имеется один элемент с мусором в переменных $x$ и $y$}
{Despite the fact it is empty, it has one element with garbage in $x$ and $y$ variables}.
\IFRU{Оба указателя}{Both} ``next'' \AndENRU ``prev'' \IFRU{указывают на себя}
{pointers are pointing to the self node}:

\begin{center}
\begin{tikzpicture}[thick,scale=0.85, every node/.style={scale=0.85}]
	\tikzstyle{every path}=[thick]
	\tikzstyle{node1}=[draw,rectangle,minimum height=1cm, minimum width=2.5cm, text width=2.5cm, fill=gray!20]

	\node [node1] (n1next) {Next};
	\node [node1] (n1prev) [below of=n1next] {Prev};
	\node [node1] (n1x) [below of=n1prev] {X=garbage};
	\node [node1] (n1y) [below of=n1x] {Y=garbage};

	\node [node1] (variable) [above=1.5cm of n1next] {\IFRU{Переменная}{Variable} std::list};
	
	\node [node1] (itbegin) [above left=1.5cm and 0.2cm of n1next] {list.begin()};
	\node [node1] (itend) [above right=1.5cm and 0.2cm of n1next] {list.end()};
	
	\draw [->] (itbegin.south) -- (n1next.north);
	
	\draw [->] (variable.south) -- (n1next.north);
	
	\draw [->] (itend.south) -- (n1next.north);
	
	\node (ia1) [inner sep=0pt, above left=0.5cm and 0.5cm of n1next] {};
	\node (ia2) [inner sep=0pt, left=0.5cm of n1next] {};
	\node (ib1) [inner sep=0pt, above right=0.5cm and 0.5cm of n1next] {};
	\node (ib2) [inner sep=0pt, right=0.5cm of n1next] {};
	
	\node (oa1) [inner sep=0pt, above left=1cm and 1cm of n1next] {};
	\node (oa2) [inner sep=0pt, left=1cm of n1prev] {};
	\node (ob1) [inner sep=0pt, above right=1cm and 1cm of n1next] {};
	\node (ob2) [inner sep=0pt, right=1cm of n1prev] {};

	\draw [->] (n1next.east) -- (ib2) -- (ib1) -- (ia1) -- (ia2) -- (n1next.west);
	\draw [->] (n1prev.west) -- (oa2) -- (oa1) -- (ob1) -- (ob2) -- (n1prev.east);
\end{tikzpicture}
\end{center}


\IFRU{Это тот момент, когда итераторы .begin и .end равны друг другу}
{That's is the moment when .begin and .end iterators are equal to each other}.

\IFRU{Вставим 3 элемента и список в памяти будет представлен так}
{Let's push 3 elements, and the list internally will be}:

\begin{lstlisting}
* 3-elements list:
ptr=0x000349a0 _Next=0x00034988 _Prev=0x0028fe90 x=3 y=4
ptr=0x00034988 _Next=0x00034b40 _Prev=0x000349a0 x=1 y=2
ptr=0x00034b40 _Next=0x0028fe90 _Prev=0x00034988 x=5 y=6
ptr=0x0028fe90 _Next=0x000349a0 _Prev=0x00034b40 x=5 y=6
\end{lstlisting}

\IFRU{Последний элемент всё еще на}{The last element is still at} 0x0028fe90, 
\IFRU{он не будет передвинут куда-либо до самого уничтожения списка}{it will not be moved until list disposal}.
\IFRU{Он все еще содержит случайный мусор в полях $x$ и $y$ (5 и 6)}
{It still contain random garbage in $x$ and $y$ fields (5 and 6)}. 
\IFRU{Случайно совпало так, что эти значения точно такие же как и в последнем элементе, но это не значит,
что они имеют какое-то значение}{By occasion, these values are the same as in the last element, but it doesn't mean they are meaningful}.

\IFRU{Вот как эти 3 элемента хранятся в памяти}{Here is how 3 elements will be stored in memory}:

\begin{center}
\begin{tikzpicture}[thick,scale=0.85, every node/.style={scale=0.85}]
	\tikzstyle{every path}=[thick]
	\tikzstyle{node1}=[draw,rectangle,minimum height=1cm, minimum width=2.5cm, text width=2.5cm, fill=gray!20]

	\node [node1] (n1next) {Next};
	\node [node1] (n1prev) [below of=n1next] {Prev};
	\node [node1] (n1x) [below of=n1prev] {X=\IFRU{1-й элемент}{1st element}};
	\node [node1] (n1y) [below of=n1x] {Y=\IFRU{1-й элемент}{1st element}};

	\node [node1] (n2next) [right=0.5cm of n1next] {Next};
	\node [node1] (n2prev) [below of=n2next] {Prev};
	\node [node1] (n2x) [below of=n2prev] {X=\IFRU{2-й элемент}{2nd element}};
	\node [node1] (n2y) [below of=n2x] {Y=\IFRU{2-й элемент}{2nd element}};
	
	\node [node1] (n3next) [right=0.5cm of n2next] {Next};
	\node [node1] (n3prev) [below of=n3next] {Prev};
	\node [node1] (n3x) [below of=n3prev] {X=\IFRU{3-й элемент}{3rd element}};
	\node [node1] (n3y) [below of=n3x] {Y=\IFRU{3-й элемент}{3rd element}};
	
	\node [node1] (n4next) [right=0.5cm of n3next] {Next};
	\node [node1] (n4prev) [below of=n4next] {Prev};
	\node [node1] (n4x) [below of=n4prev] {X=\IFRU{мусор}{garbage}};
	\node [node1] (n4y) [below of=n4x] {Y=\IFRU{мусор}{garbage}};
	
	\node [node1] (variable) [above=3cm of n1next] {\IFRU{Переменная}{Variable} std::list};
	
	\node [node1] (itbegin) [above=2cm of n1next] {list.begin()};
	\draw [->] (itbegin.south) -- (n1next.north);
	
	\draw [->] (variable.south) -- (itbegin.north);

	\node [node1] (itend) [above=2cm of n4next] {list.end()};
	\draw [->] (itend.south) -- (n4next.north);
	
	\node (ia1) [inner sep=0pt, above left=0.5cm and 0.5cm of n1next] {};
	\node (ia2) [inner sep=0pt, left=0.5cm of n1next] {};
	\node (ib1) [inner sep=0pt, above right=0.5cm and 0.5cm of n4next] {};
	\node (ib2) [inner sep=0pt, right=0.5cm of n4next] {};
	
	\node (oa1) [inner sep=0pt, above left=1cm and 1cm of n1next] {};
	\node (oa2) [inner sep=0pt, left=1cm of n1prev] {};
	\node (ob1) [inner sep=0pt, above right=1cm and 1cm of n4next] {};
	\node (ob2) [inner sep=0pt, right=1cm of n4prev] {};

	\draw [->] (n1next.east) -- (n2next.west);
	\draw [->] (n2next.east) -- (n3next.west);
	\draw [->] (n3next.east) -- (n4next.west);

	\draw [<-] (n1prev.east) -- (n2prev.west);
	\draw [<-] (n2prev.east) -- (n3prev.west);
	\draw [<-] (n3prev.east) -- (n4prev.west);
	
	\draw [->] (n4next.east) -- (ib2) -- (ib1) -- (ia1) -- (ia2) -- (n1next.west);
	\draw [->] (n1prev.west) -- (oa2) -- (oa1) -- (ob1) -- (ob2) -- (n4prev.east);
\end{tikzpicture}
\end{center}


\IFRU{Переменная}{The variable} $l$ \IFRU{всегда указывает на первый элемент}{is always points to the first node}.

\IFRU{Иметь элемент с ``мусором'' это очень популярная практика в реализации двусвязных списков}
{Having a ``garbage'' element is a very popular practice in implementing doubly-linked lists}.
\IFRU{Без него, многие операции были бы сложнее, и следовательно, медленнее}
{Without it, a lot of operations may become slightly more complex and, hence, slower}.

\IFRU{Итератор на самом деле это просто указатель на элемент}{Iterator in fact is just a pointer to a node}.
list.begin() \AndENRU list.end() \IFRU{просто возвращают указатели}{are just returning pointers}.

\begin{lstlisting}
node at .begin:
ptr=0x000349a0 _Next=0x00034988 _Prev=0x0028fe90 x=3 y=4
node at .end:
ptr=0x0028fe90 _Next=0x000349a0 _Prev=0x00034b40 x=5 y=6
\end{lstlisting}

\IFRU{Тот факт что список циркулярный, очень помогает: если иметь указатель только на первый элемент, т.е.,
тот что в переменной $l$, очень легко получить указатель на последний элемент, без необходимости
обходить все элементы списка}
{The fact the list is circular is very helpful here: having a pointer to the first list element,
i.e., that is in the $l$ variable,
it is easy to get a pointer to the last one quickly, without need to traverse whole list}.
\IFRU{Вставка элемента в конец списка также быстра благодаря этой особенности}
{Inserting element at the list end is also quick, thanks to this feature}.

\TT{operator--} \AndENRU \TT{operator++} \IFRU{просто выставляют текущее значение итератора на}
{are just set current iterator value to the} \TT{current\_node->prev}
\OrENRU \TT{current\_node->next}\IFRU{}{ values}.
\IFRU{Обратные итераторы}{Reverse iterators} (.rbegin, .rend) \IFRU{работают точно также, только наоборот}
{works just as the same, but in inverse way}.

\TT{operator*} \IFRU{на итераторе просто возвращает указатель на место в структуре, где начинается пользовательская
структура, т.е., указатель на самый первый элемент структуры}{of iterator just returns pointer to the point 
in the node structure, where user's structure is beginning, i.e., pointer to the 
very first structure element} ($x$).

\IFRU{Вставка в список и удаление очень просты: просто выделите новый элемент (или освободите) и исправьте
все указатели так, чтобы они были верны}
{List insertion and deletion is trivial: just allocate new node (or deallocate) and fix all pointers
to be valid}.

\IFRU{Вот почему итератор может стать недействительным после удаления элемента:
он может всё еще указывать на уже освобожденный элемент}{That's why iterator may become invalid after 
element deletion: it may still point to the node already deallocated}.
\IFRU{И конечно же, информация из освобожденного элемента, на который указывает итератор, не может использоваться
более}
{And of course, the information from the freed node, to which iterator still points, cannot be used anymore}.

\IFRU{В реализации GCC (по крайней мере 4.8.1) не сохраняется текущая длина списка: это выливается в медленный
метод .size(): он должен пройти по всему списку считая элементы, просто потому что нет
другого способа получить эту информацию}
{The GCC implementation (as of 4.8.1) doesn't store current list size: this resulting in slow .size() method:
it should traverse the whole list counting elements, because it doesn't have any other way to get the information}.
\IFRU{Это означает что эта операция $O(n)$, т.е., она работает тем медленнее, чем больше элементов в списке}
{This mean this operation is $O(n)$, i.e., it is as slow, as how many elements present in the list}.

\lstinputlisting[caption=GCC 4.8.1 -O3 -fno-inline-small-functions]{cpp/STL/list/GCC.asm}

\lstinputlisting[caption=\IFRU{Весь вывод}{The whole output}]{cpp/STL/list/GCC.txt}

\subsubsection{MSVC}

\IFRU{Реализация MSVC (2012) точно такая же, только еще и сохраняет текущий размер списка}
{MSVC implementation (2012) is just the same, but it also stores current list size}.
\IFRU{Это означает что метод .size() очень быстр ($O(1)$): просто прочитать одно значение из памяти}
{This mean, .size() method is very fast ($O(1)$): just read one value from memory}.
\IFRU{С другой стороны, переменная хранящая размер должна корректироваться при каждой вставке/удалении}
{On the other way, size variable must be corrected at each insertion/deletion}.

\IFRU{Реализация MSVC также немного отлична в смысле расстановки элементов}
{MSVC implementation is also slightly different in a way it arrange nodes}:

\begin{center}
\begin{tikzpicture}[thick,scale=0.85, every node/.style={scale=0.85}]
	\tikzstyle{every path}=[thick]
	\tikzstyle{node1}=[draw,rectangle,minimum height=1cm, minimum width=2.5cm, text width=2.5cm, fill=gray!20]

	\node [node1] (n1next) {Next};
	\node [node1] (n1prev) [below of=n1next] {Prev};
	\node [node1] (n1x) [below of=n1prev] {X=\IFRU{мусор}{garbage}};
	\node [node1] (n1y) [below of=n1x] {Y=\IFRU{мусор}{garbage}};

	\node [node1] (n2next) [right=0.5cm of n1next] {Next};
	\node [node1] (n2prev) [below of=n2next] {Prev};
	\node [node1] (n2x) [below of=n2prev] {X=\IFRU{1-й элемент}{1st element}};
	\node [node1] (n2y) [below of=n2x] {Y=\IFRU{1-й элемент}{1st element}};
	
	\node [node1] (n3next) [right=0.5cm of n2next] {Next};
	\node [node1] (n3prev) [below of=n3next] {Prev};
	\node [node1] (n3x) [below of=n3prev] {X=\IFRU{2-й элемент}{2nd element}};
	\node [node1] (n3y) [below of=n3x] {Y=\IFRU{2-й элемент}{2nd element}};
	
	\node [node1] (n4next) [right=0.5cm of n3next] {Next};
	\node [node1] (n4prev) [below of=n4next] {Prev};
	\node [node1] (n4x) [below of=n4prev] {X=\IFRU{3-й элемент}{3rd element}};
	\node [node1] (n4y) [below of=n4x] {Y=\IFRU{3-й элемент}{3rd element}};
	
	\node [node1] (variable) [above=3cm of n1next] {\IFRU{Переменная}{Variable} std::list};

	\node [node1] (itend) [above=2cm of n1next] {list.end()};
	\draw [->] (itend.south) -- (n1next.north);
	
	\draw [->] (variable.south) -- (itend.north);
	
	\node [node1] (itbegin) [above=2cm of n2next] {list.begin()};
	\draw [->] (itbegin.south) -- (n2next.north);

	\node (ia1) [inner sep=0pt, above left=0.5cm and 0.5cm of n1next] {};
	\node (ia2) [inner sep=0pt, left=0.5cm of n1next] {};
	\node (ib1) [inner sep=0pt, above right=0.5cm and 0.5cm of n4next] {};
	\node (ib2) [inner sep=0pt, right=0.5cm of n4next] {};
	
	\node (oa1) [inner sep=0pt, above left=1cm and 1cm of n1next] {};
	\node (oa2) [inner sep=0pt, left=1cm of n1prev] {};
	\node (ob1) [inner sep=0pt, above right=1cm and 1cm of n4next] {};
	\node (ob2) [inner sep=0pt, right=1cm of n4prev] {};

	\draw [->] (n1next.east) -- (n2next.west);
	\draw [->] (n2next.east) -- (n3next.west);
	\draw [->] (n3next.east) -- (n4next.west);

	\draw [<-] (n1prev.east) -- (n2prev.west);
	\draw [<-] (n2prev.east) -- (n3prev.west);
	\draw [<-] (n3prev.east) -- (n4prev.west);
	
	\draw [->] (n4next.east) -- (ib2) -- (ib1) -- (ia1) -- (ia2) -- (n1next.west);
	\draw [->] (n1prev.west) -- (oa2) -- (oa1) -- (ob1) -- (ob2) -- (n4prev.east);
\end{tikzpicture}
\end{center}


\IFRU{У GCC его элемент с ``мусором'' в самом конце списка, а у MSVC в самом начале}
{GCC has its ``garbage'' element at the end of the list, while MSVC at the beginning of it}.

\lstinputlisting[caption=MSVC 2012 /Fa2.asm /Ox /GS- /Ob1]{cpp/STL/list/MSVC.asm}

\IFRU{В отличие от GCC, код MSVC выделяет элемент с ``мусором'' в самом начале ф-ции при помощи
ф-ции ``Buynode'', она также используется и во время выделения остальных элементов}
{Unlike GCC, MSVC code allocates ``garbage'' element at the function start with ``Buynode'' function,
it is also used for the rest nodes allocations} (\IFRU{код GCC выделяет самый первый элемент в локальном стеке}
{GCC code allocates the very first element in the local stack}).

\lstinputlisting[caption=\IFRU{Весь вывод}{The whole output}]{cpp/STL/list/MSVC.txt}

\subsubsection{C++11 std::forward\_list}
\index{\Cpp!STL!std::forward\_list}

\IFRU{Это то же самое что и std::list, но только односвязный список, т.е., имеющий только поле ``next'' в каждом
элементе}
{The same thing as std::list, but singly-linked one, i.e., having only ``next'' field at teach node}.
\IFRU{Таким образом расход памяти меньше, но возможности идти по списку назад здесь нет}
{It require smaller memory footprint, but also don't offer a feature to traverse list back}.

