\subsection{std::map \AndENRU std::set}
\index{\Cpp!STL!std::map}
\index{\Cpp!STL!std::set}
\index{\IFRU{Двоичное дерево}{Binary tree}}

\IFRU{Двоичное дерево ~--- это еще одна фундаментальная структура данных}{Binary tree is another fundamental data structure}.
\IFRU{Как следует из названия, это дерево, но у каждого узла максимум 2 связи с другими узлами}
{As it states, this is a tree, but each node has at most 2 links to other nodes}.
\IFRU{Каждый узел имеет ключ и/или значение}{Each node have key and/or value}.

\IFRU{Обычно, именно при помощи двоичных деревьев реализуются ``словари'' пар ключ-значения 
(\ac{AKA} ``ассоциативные массивы'')}
{Binary trees are usually the structure used in ``dictionaries'' of key-values (\ac{AKA} ``associative arrays'') 
implementations}.

\IFRU{Двоичные деревья имеют по крайней мере три важных свойства}
{There are at least three important properties binary trees has}:
\begin{itemize}
\item \IFRU{Все ключи всегда хранятся в отсортированном виде}{All keys are stored in always sorted form}.
\item \IFRU{Могут хранится ключи любых типов}{Keys of any types can be stored easily}.
\IFRU{Алгоритмы для работы с двоичными деревьями не зависят от типа ключа,
для работы им нужна только ф-ция для сравнения ключей}{Binary tree algorithms are unaware of key type, 
only key comparison function is required}.
\item \IFRU{Поиск необходимого ключа относительно быстрый по сравнению со списками или массивами}
{Finding needed key is relatively fast in comparison with lists and arrays}.
\end{itemize}

\IFRU{Очень простой пример: давайте сохраним вот эти числа в двоичном дереве}
{Here is a very simple example: let's store these numbers in binary tree}:
0, 1, 2, 3, 5, 6, 9, 10, 11, 12, 20, 99, 100, 101, 107, 1001, 1010.

\tikzset
{
  treenode/.style = {align=center, inner sep=0pt, text centered},
  key/.style = {treenode, circle, white, draw=black, fill=black, minimum size=2.0em, text width=2.0em},
  empty/.style = {treenode, rectangle, draw=black, minimum width=0.5em, minimum height=0.5em}
}

\begin{center}
\begin{tikzpicture}[->,>=stealth',
		level 1/.style={sibling distance=6cm},
		level 2/.style={sibling distance=2.7cm},
		level 3/.style={sibling distance=1.7cm},
		level 4/.style={sibling distance=0.7cm}] 
	\node [key] {10}
	child { node [key] {1} 
		child{ node [key] {0} 
		}
		child{ node [key] {5}
			child{ node [key] {3}
				child{ node[key] {2}}
				child{ node[empty] {}}
			}
			child{ node [key] {6}
				child{ node[empty] {}}
				child{ node[key] {9}}
			}
		}                            
	}
	child { node [key] {100}
		child { node [key] {20}
			child{ node [key] {12}
				child{ node[key] {11}}
				child{ node[empty] {}}
			}
			child{ node [key] {99}
			}
		}
		child{ node [key] {107}
			child{ node [key] {101}}
			child{ node [key] {1001}
				child{ node[empty] {}}
				child{ node[key] {1010}}
			}
		}
	}
	; 
\end{tikzpicture}
\end{center}


\IFRU{Все ключи меньше чем значение ключа узла, сохраняются по левой стороне}
{All keys lesser than node key value is stored on the left side}.
\IFRU{Все ключи больше чем значение ключа узла, сохраняются по правой стороне}
{All keys greater than node key value is stored on the right side}.

\IFRU{Таким образом, алгоритм для поиска нужного ключа прост: если искомое значение меньше чем значение текущего узла:
двигаемся влево, если больше: двигаемся вправо, останавливаемся если они равны}
{Hence, finding algorithm is straightforward: if the value you looking for is lesser than current node's key value:
move left, if it is greater: move right, stop if the value required is equals to the node's key value}.
\IFRU{Таким образом, алгоритм может искать числа, текстовые строки, и т.д., только при помощи ф-ции сравнения ключей}
{That is why searching algorithm may search for numbers, text strings, etc, using only 
key comparison function}.

\IFRU{Все ключи имеют уникальные значения}{All keys has unique values}.

\IFRU{Учитывая это, нужно}{Having that, one need} $\approx \log_{2} n$ \IFRU{шагов для поиска ключа 
в сбалансированном дереве, содержащем $n$ ключей}{steps in order to find a key in the balanced binary tree of $n$ keys}.
\IFRU{Это}{It is} $\approx 10$ \IFRU{шагов для}{steps for} $\approx 1000$ \IFRU{ключей, или}{keys, or} $\approx 13$ 
\IFRU{шагов для}{steps for} $\approx 10000$ \IFRU{ключей}{keys}.
\IFRU{Неплохо, но для этого дерево всегда должно быть сбалансировано: т.е., ключи должны быть равномерно распределены
на всех ярусах}
{Not bad, but tree should always be balanced for this: i.e., keys should be distributed evenly on all tiers}.
\IFRU{Операции вставки и удаления проводят дополнительную работу по обслуживанию дерева и сохранения его в сбалансированном
состоянии}{Insertion and removal operations do some maintenance to keep tree in balanced state}.

\IFRU{Известно несколько популярных алгоритмом балансировки, включая AVL-деревья и красно-черные деревья}
{There are several popular balancing algorithms available, including AVL tree and red-black tree}.
\IFRU{Последний дополняет узел значением ``цвета'' для упрощения балансировки, таким образом каждый узел может быть
``красным'' или ``черным''}
{The latter extends a node by a ``color'' value for simplifying balancing process, hence, 
each node may be ``red'' or ``black''}.

\IFRU{Реализации \TT{std::map} и \TT{std::set} обоих GCC и MSVC используют красно-черные деревья}
{Both GCC and MSVC \TT{std::map} and \TT{std::set} template implementations use red-black trees}.

\TT{std::set} \IFRU{содержит только ключи}{contain only keys}.
\TT{std::map} \IFRU{это ``расширенная'' версия set: здесь имеется еще и значение (value) на каждом узле}
{is ``extended'' version of set: it also has a value at each node}.

\subsubsection{MSVC}

\lstinputlisting{cpp/STL/map_set/MSVC.cpp}

\lstinputlisting[caption=MSVC 2012]{cpp/STL/map_set/MSVC.txt}

\IFRU{Структура не запакована, так что оба значения типа \Tchar занимают по 4 байта}
{Structure is not packed, so both \Tchar type values occupy 4 bytes each}.

\IFRU{В}{As for} \TT{std::map}, \TT{first} \AndENRU \TT{second} \IFRU{могут быть представлены как одно значение
типа}{can be viewed as a single value of} \TT{std::pair}\EN{ type}.
\TT{std::set} \IFRU{имеет только одно значение в этом месте структуры}
{has only one value at this point in the structure instead}.

\IFRU{Текущий размер дерева всегда присутствует, как и в случае реализации \TT{std::list} в MSVC}
{Current size of tree is always present, as in case of \TT{std::list} MSVC implementation} (\ref{MSVC_std_list}).

\IFRU{Как и в случае с}{As in case of} \TT{std::list}, \IFRU{итераторы это просто указатели на узлы}
{iterators are just pointers to the nodes}.
\RU{Итератор }\TT{.begin()} \IFRU{указывает на минимальный ключ}{iterator pointing to the minimal key}.
\IFRU{Этот указатель нигде не сохранен (как в списках), минимальный ключ дерева нужно находить каждый раз}
{That pointer is not stored somewhere (as in lists), minimal key of tree is to be found each time}.
\TT{operator--} \AndENRU \TT{operator++} \IFRU{перемещают указатель не текущий узел на узел-предшественник
или узел-преемник, т.е., узлы содержащие предыдущий и следующий ключ}
{moves pointer to the current node to predecessor and successor respectively, i.e., nodes which has previous and next key}.
\IFRU{Алгоритмы для всех этих операций описаны в}
{The algorithms for all these operations are described in} \cite{Cormen:2009:IAT:1614191}.

\RU{Итератор }\TT{.end()} \IFRU{указывает на корневой узел, он имеет $1$ в \TT{Isnil}, что означает что у узла
нет ключа и/или значения}{iterator pointing to the root node, 
it has $1$ in \TT{Isnil}, meaning, the node has no key and/or value}.
\IFRU{Так что его можно рассматривать как}{So it can be viewed as a} ``landing zone'' \IFRU{в}{in} \ac{HDD}.

\subsubsection{GCC}

\lstinputlisting{cpp/STL/map_set/GCC.cpp}

\lstinputlisting[caption=GCC 4.8.1]{cpp/STL/map_set/GCC.txt}

\IFRU{Реализация в GCC очень похожа}{GCC implementation is very similar}
\footnote{\url{http://gcc.gnu.org/onlinedocs/libstdc++/libstdc++-html-USERS-4.1/stl__tree_8h-source.html}}.
\IFRU{Разница только в том, что здесь нет поля \TT{Isnil}}{The only difference is absence of \TT{Isnil} field},
\IFRU{так что структура занимает немного меньше места в памяти чем та что реализована в MSVC}
{so the structure occupy slightly less space in memory than as it is implemented in MSVC}.
\IFRU{Корневой узел это так же место куда указывает итератор \TT{.end()}, не имеющий ключа и/или значения}
{Root node is also used as a place \TT{.end()} iterator pointing to and also has no key and/or value}.

\subsubsection{\IFRU{Демонстрация перебалансировки}{Rebalancing demo} (GCC)}

\IFRU{Вот также демонстрация показывающая нам как дерево может перебалансироваться после вставок}
{Here is also a demo showing us how tree is rebalanced after insertions}.

\lstinputlisting[caption=GCC]{cpp/STL/map_set/GCC_rebalancing_demo.cpp}

\lstinputlisting[caption=GCC 4.8.1]{cpp/STL/map_set/GCC_rebalancing_demo.txt}

