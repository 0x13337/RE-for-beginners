\subsection{std::map \AndENRU std::set}
\index{\Cpp!STL!std::map}
\index{\Cpp!STL!std::set}
\index{Binary tree}

Binary tree is another fundamental data structure.
As it states, this is a tree, but each node has at most 2 links to other nodes.
Each node have key and/or value.

Binary trees are usually the structure used in ``dictionaries'' of key-values (\ac{AKA} ``associative arrays'') 
implementations.

There are at least three important properties binary trees has:
1) All keys are stored in always sorted form.
2) Keys of any types can be stored easily.
Binary tree algorithms are unaware of key type, as long as comparison function is present for it.
3) Finding needed key is relatively fast in comparison with other data structures.

Here is a very simple example: let's store these numbers in binary tree:
0, 1, 2, 3, 5, 6, 9, 10, 11, 12, 20, 99, 100, 101, 107, 1001, 1010.

\tikzset
{
  treenode/.style = {align=center, inner sep=0pt, text centered},
  key/.style = {treenode, circle, white, draw=black, fill=black, minimum size=2.0em, text width=2.0em},
  empty/.style = {treenode, rectangle, draw=black, minimum width=0.5em, minimum height=0.5em}
}

\begin{center}
\begin{tikzpicture}[->,>=stealth',
		level 1/.style={sibling distance=6cm},
		level 2/.style={sibling distance=2.7cm},
		level 3/.style={sibling distance=1.7cm},
		level 4/.style={sibling distance=0.7cm}] 
	\node [key] {10}
	child { node [key] {1} 
		child{ node [key] {0} 
		}
		child{ node [key] {5}
			child{ node [key] {3}
				child{ node[key] {2}}
				child{ node[empty] {}}
			}
			child{ node [key] {6}
				child{ node[empty] {}}
				child{ node[key] {9}}
			}
		}                            
	}
	child { node [key] {100}
		child { node [key] {20}
			child{ node [key] {12}
				child{ node[key] {11}}
				child{ node[empty] {}}
			}
			child{ node [key] {99}
			}
		}
		child{ node [key] {107}
			child{ node [key] {101}}
			child{ node [key] {1001}
				child{ node[empty] {}}
				child{ node[key] {1010}}
			}
		}
	}
	; 
\end{tikzpicture}
\end{center}


All keys lesser than node key value is stored on the left side.
All keys greater than node key value is stored on the right side.

Hence, finding algorithm is straightforward: if the value you looking for is lesser than current node's key value,
move left, if it is greater, move right, stop if the value required is equals to the node's key value.
One important thing is that searching algorithm may search for numbers, text strings, etc, as long as 
comparison function is supplied.

All keys has unique values.

Having that, one need $\approx \log_{2} n$ steps in order to find a key in the balanced binary tree of $n$ keys.
It is $\approx 10$ steps for $\approx 1000$ keys or $\approx 13$ steps for $\approx 10000$ keys.
Not bad, but tree should always be balanced for this: i.e., keys should be distributed evenly on all tiers.

There are several popular balancing algorithms available, including AVL tree and red-black tree.
The latter adds a ``color'' value to each node for simplifying balancing process, hence, each node may be red or black.

Both GCC and MSVC \TT{std::map} and \TT{std::set} template implementations use red-black trees.

\TT{std::set} contain only keys.
\TT{std::map} is ``extended'' version of set: it also has a value at each node.

\subsubsection{MSVC}

\lstinputlisting{cpp/STL/map_set/MSVC.cpp}

\lstinputlisting[caption=MSVC 2012]{cpp/STL/map_set/MSVC.txt}

Structure is not packed, so both char values occupy 4 bytes each.

As for \TT{std::map}, \TT{first} and \TT{second} can be viewed as a single value of \TT{std::pair} type.
\TT{std::set} has only one value at this point in the structure instead.

Current size of tree is always present, as in case of \TT{std::list} (\ref{std_list}).

As in case of \TT{std::list}, iterators are just pointers to the nodes.
\TT{.begin()} iterator pointing to the minimal key.
That pointer is not stored somewhere (as for lists), minimal value is to be found each time.
\TT{operator--} and \TT{operator++} moves current pointer to predecessor and successor respectively.
The algorithms for all these operations are described in \cite{Cormen:2009:IAT:1614191}.

\TT{.end()} iterator pointing to the root node, it has $1$ in \TT{Isnil}, meaning, it has no key and/or value.
So it can be viewed as a ``landing zone''.

\subsubsection{GCC}

\lstinputlisting{cpp/STL/map_set/GCC.cpp}

\lstinputlisting[caption=GCC 4.8.1]{cpp/STL/map_set/GCC.txt}

GCC implementation is very similar
\footnote{\url{http://gcc.gnu.org/onlinedocs/libstdc++/libstdc++-html-USERS-4.1/stl__tree_8h-source.html}}.
The only difference is absence of \TT{Isnil} field,
so the structure occupy slightly less space in memory than as it is implemented in MSVC.
Root node is also used as the place \TT{.end()} iterator pointing to and also has no key and/or value.

