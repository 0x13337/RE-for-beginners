\subsection{std::string}
\index{\Cpp!STL!std::string}
\label{std_string}

\subsubsection{\IFRU{Как устроена структура}{Internals}}

\IFRU{Многие строковые библиотеки (\cite[2.2]{CBook}) обеспечивают структуру содержащую ссылку 
на буфер собственно со строкой, переменная всегда содержащую длину строки 
(что очень удобно для массы ф-ций \cite[2.2.1]{CBook}) и переменную содержащую текущий размер буфера.}
{Many string libraries (\cite[2.2]{CBook}) implements structure containing pointer to the buffer containing string,
a variable always containing current string length 
(that is very convenient for many functions: \cite[2.2.1]{CBook}) and
a variable containing current buffer size.}
\IFRU{Строка в буфере обыкновенно оканчивается нулем: это для того чтобы указатель на буфер можно было
передавать в ф-ции требующие на вход обычную сишную \ac{ASCIIZ}-строку.}
{A string in buffer is usually terminated with zero: in order to be able to pass a pointer to a buffer
into the functions taking usual C \ac{ASCIIZ}-string.}

\IFRU{Стандарт Си++ (\cite{CPP11}) не описывает, как именно нужно реализовывать std::string,
но как правило они реализованы как описано выше, с небольшими дополнениями}
{It is not specified in the C++ standard (\cite{CPP11}) how std::string should be implemented,
however, it is usually implemented as described above}.

\IFRU{По стандарту, std::string это не класс (как, например, QString в Qt), а темплейт, 
это сделано для того чтобы поддерживать 
строки содержащие разного типа символы: как минимум char и wchar\_t.}
{By standard, std::string is not a class (as QString in Qt, for instance) but template,
this is done in order to support various character types: at least char and wchar\_t.}

\IFRU{Здесь пока не будет листингов на ассемблере, потому что проиллистрировать внутренности std::string 
в MSVC и GCC можно и без этого}
{There are no assembly listings, because std::string internals in MSVC and GCC can be illustrated without them}.

\paragraph{MSVC}

\IFRU{В реализации MSVC, вместо ссылки на буфер может содержаться сам буфер (если строка короче 16-и символов).}
{MSVC implementation may store buffer in place instead of pointer to buffer 
(if the string is shorter than 16 symbols).}

\IFRU{Это означает что каждая короткая строка будет занимать в памяти по крайней мере}{This mean
that short string will occupy at least} $16 + 4 + 4 = 24$ 
\IFRU{байт для 32-битной среды либо}{bytes in 32-bit environment or at least} $16 + 8 + 8 = 32$ 
\IFRU{байта в 64-битной, а если строка длиннее 16-и символов, то прибавьте еще длину самой строки}
{bytes in 64-bit, and if the string is longer than 16 characters, add also length of the string itself}.

\lstinputlisting[caption=\IFRU{пример для}{example for} MSVC]{cpp/STL/string/MSVC.cpp}

\IFRU{Собственно, из этого исходника почти всё ясно.}{Almost everything is clear from the source code.}

\IFRU{Несколько замечаний}{Couple notes}:

\IFRU{Если строка короче 16-и символов, то отдельный буфер для строки в куче (heap) выделяться не будет}
{If the string is shorter than 16 symbols, a buffer for the string will not be allocated in the heap}.
\IFRU{Это удобно потому что на практике, действительно немало строк короткие}
{This is convenient because
in practice, large ammount of strings are short indeed}.
\IFRU{Вероятно, разработчики в Microsoft выбрали размер в 16 символов как разумный баланс}
{Apparently, Microsoft developers chose 16 characters as a good balance}.

\IFRU{Теперь очень важный момент в конце ф-ции main(): я не пользуюсь методом c\_str(), тем не менее,
если это скомпилировать и запустить, то обе строки появятся в консоли!}
{Very important thing here is in the end of main() functions: I'm not using c\_str() method, nevertheless,
if to compile the code and run, both strings will be appeared in the console!}

\IFRU{Работает это вот почему}{This is why it works}.

\IFRU{В первом случае строка короче 16-и символов и в начале объекта std::string (его можно рассматривать
просто как структуру) расположен буфер с этой строкой}
{The string is shorter than 16 characters and buffer with the string is located in the
beginning of std::string object (it can be treated just as structure)}.
printf() \IFRU{трактует указатель как указатель на массив
символов оканчивающийся нулем и поэтому всё работает}{treats pointer as a pointer to the null-terminated 
array of characters, hence it works}.

\IFRU{Вывод второй строки (длиннее 16-и символов) даже еще опаснее: это вообще типичная программистская ошибка 
(или опечатка), забыть дописать c\_str()}
{Second string (longer than 16 characters) printing is even more dangerous: it is typical programmer's mistake
(or typo) to forget to write c\_str()}.
\IFRU{Это работает потому что в это время в начале структуры расположен указатель на буфер}
{This works because at the moment a pointer to buffer is located at the start of structure}.
\IFRU{Это может надолго остаться незамеченным: до тех пока там не появится строка короче 16-и символов, тогда процесс упадет}
{This may left unnoticed for a long span of time: until a longer string will appear there, then a process will crash}.

\paragraph{GCC}

\IFRU{В реализации GCC в структуре есть еще одна переменная --- reference count}
{GCC implementation of a structure has one more variable---reference count}.

\IFRU{Интересно, что указатель на экземпляр класса std::string в GCC указывает не на начало самой структуры, 
а на указатель на буфера}{One interesting fact is that a pointer to std::string instance of class points not to
beginning of the structure, but to the pointer to buffer}.
\IFRU{В}{In} libstdc++-v3\textbackslash{}include\textbackslash{}bits\textbackslash{}basic\_string.h 
\IFRU{мы можем прочитать что это сделано для удобства отладки}
{we may read that it was made for convenient debugging}:

\begin{lstlisting}
   *  The reason you want _M_data pointing to the character %array and
   *  not the _Rep is so that the debugger can see the string
   *  contents. (Probably we should add a non-inline member to get
   *  the _Rep for the debugger to use, so users can check the actual
   *  string length.)
\end{lstlisting}

\href{http://gcc.gnu.org/onlinedocs/libstdc++/libstdc++-html-USERS-4.4/a01068.html}
{\IFRU{исходный код }{}basic\_string.h\IFRU{}{ source code}}

\IFRU{В моем примере я учитываю это}{I considering this in my example}:

\lstinputlisting[caption=\IFRU{пример для}{example for} GCC]{cpp/STL/string/GCC.cpp}

\IFRU{Нужны еще небольшие хаки чтобы сымитировать типичную ошибку, о которой я уже написал, из-за
более ужесточенной проверки типов в GCC, тем не менее, printf() работает и здесь без c\_str()}
{A trickery should be also used to imitate mistake I already wrote above because GCC
has stronger type checking, nevertheless, printf() works here without c\_str() as well}.

\subsubsection{\IFRU{Чуть более сложный пример}{More complex example}}

\lstinputlisting{cpp/STL/string/3.cpp}

\lstinputlisting[caption=MSVC 2012]{cpp/STL/string/3_MSVC.asm}

\IFRU{Собственно, компилятор не конструирует строки статически: да в общем-то и как
это возможно, если буфер с ней нужно хранить в куче?}
{Compiler not constructing strings statically: how it is possible anyway if buffer should be located
in the heap?}
\IFRU{Вместо этого в сегменте данных хранятся обычные \ac{ASCIIZ}-строки, а позже, во время выполнения, 
при помощи метода ``assign'', конструируются строки s1 и s2}
{Usual \ac{ASCIIZ} strings are stored in the data segment instead, and later, at the moment of execution,
with the help of ``assing'' method, s1 and s2 strings are constructed}.
\IFRU{При помощи \TT{operator+}, создается строка s3}{With the help of \TT{operator+}, s3 string is constructed}.

\IFRU{Обратите внимание на то что вызов метода c\_str() отсутствует,
потому что его код достаточно короткий и компилятор вставил его прямо здесь:
если строка короче 16-и байт, то в регистре EAX остается указатель на буфер,
а если длиннее, то из этого же места достается адрес на буфер расположенный в куче}{Please note that 
there are no call to c\_str() method, because, its code is tiny enough so compiler
inlined it right here: if the string is shorter than 16 characters, a pointer to buffer is leaved
in EAX register, and an address of the string buffer located in the heap is fetched otherwise}.

\IFRU{Далее следуют вызовы трех деструкторов, причем, они вызываются только если строка длиннее 16-и байт:
тогда нужно освободить буфера в куче}{Next, we see calls to the 3 destructors, and they are called if
string is longer than 16 characters: then a buffers in the heap should be freed}.
\IFRU{В противном случае, так как все три объекта std::string хранятся в стеке,
они освобождаются автоматически после выхода из ф-ции}{Otherwise, since all three std::string objects
are stored in the stack, they are freed automatically, upon function finish}.

\IFRU{Следовательно работа с короткими строками более быстрая из-за м\`{е}ньшего обращения к куче}
{As a consequence, short strings processing is faster because of lesser heap accesses}.

\IFRU{Код на GCC даже проще (из-за того что в GCC, как я уже указывал, не реализована возможность хранить короткую
строку прямо в структуре)}{GCC code is even simpler (because GCC way, as I mentioned above, is not to store
shorter string right in the structure)}:

% TODO comment each function meaning
\lstinputlisting[caption=GCC 4.8.1]{cpp/STL/string/3_GCC.s}

\IFRU{Можно заметить что в деструкторы передается не указатель на объект,
а указатель на место за 12 байт (или 3 слова) перед ним, то есть, на настоящее начало структуры}
{It can be seen that not a pointer to object is passed to destructors, but rather a place 12 bytes (or 3 words)
before, i.e., pointer to the real start of the structure}.

\subsubsection{std::string \IFRU{как глобальная переменная}{as a global variable}}

\IFRU{Опытные программисты на Си++ могут возразить: глобальные переменные \ac{STL}-типов вполне
можно объявлять}{Experienced C++ programmers may argue: a global 
variables of \ac{STL} types are in fact can be defined}.

\IFRU{Да, действительно}{Yes, indeed}:

\lstinputlisting{cpp/STL/string/5.cpp}

\lstinputlisting[caption=MSVC 2012]{cpp/STL/string/5_MSVC.asm}

\index{\CStandardLibrary!atexit()}
\IFRU{В реальности, из \ac{CRT}, еще до вызова main(), вызывается специальная ф-ция,
в которой перечислены все конструкторы подобных переменных}
{In fact, a special function with all constructors of global variables is called from \ac{CRT}, before
main()}.
\IFRU{Более того: при помощи atexit() регистрируется ф-ция, которая будет вызвана в конце работы программы:
в этой ф-ции компилятор собирает деструкторы всех подобных глобальных переменных}
{More than that: with the help of atexit() another function is registered: which contain all destructors
of such variables}.

GCC \IFRU{работает похожим образом}{works likewise}:

\lstinputlisting[caption=GCC 4.8.1]{cpp/STL/string/5_GCC.s}

\IFRU{Он даже не выделяет отдельной ф-ции в которой будут собраны деструкторы: 
каждый деструктор передается в atexit() по одному}{It even not creates separated functions for this, each
destructor is passed to atexit(), one by one}.

