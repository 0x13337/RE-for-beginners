\subsection{std::vector}
\index{\Cpp!STL!std::vector}

\IFRU{Я бы назвал}{I would call} \TT{std::vector} ``\IFRU{безопасной оболочкой (wrapper)}{safe wrapper}'' \IFRU{}{of }\ac{PODT} \IFRU{массива в Си}{C array}.
\IFRU{Изнутри он очень похож на}{Internally, it is somewhat similar to} \TT{std::string} (\ref{std_string}):
\IFRU{он имеет указатель на буфер, указатель на конец массива и указатель на конец буфера}{it has a pointer to buffer, pointer to the end of array, and a pointer to the end of buffer}.

\IFRU{Элементы массива просто лежат в памяти впритык друг к другу, так же, как и в обычном массиве}
{Array elements are lie in memory adjacently to each other, just like in usual array} (\ref{arrays}).
\index{\Cpp!C++11}
\IFRU{В}{In} C++11 \IFRU{появился метод}{there are new method} \TT{.data()} \IFRU{возвращающий указатель на этот буфер, это похоже на}{appeared, returning a pointer to the buffer, akin to} \TT{.c\_str()} \InENRU \TT{std::string}.

\IFRU{Выделенный буфер в \glslink{heap}{куче} может быть больше чем сам массив}
{Allocated buffer in \gls{heap} may be larger than array itself}.

\IFRU{Реализации MSVC и GCC почти одинаковые, отличаются только имена полей в структуре}
{Both MSVC and GCC implementations are similar, just structure field names are slightly different}\footnote
{\IFRU{внутренности GCC}{GCC internals}: \url{http://gcc.gnu.org/onlinedocs/libstdc++/libstdc++-html-USERS-4.4/a01371.html}}, \IFRU{так что здесь один исходник работающий для обоих компиляторов}{so here is one source
code working for both compilters}.
\IFRU{И снова здесь Си-подобный код для вывода структуры}{Here is again a 
C-like code for dumping} \TT{std::vector}\IFRU{}{ structure}:

\lstinputlisting{cpp/STL/vector/2.cpp}

\IFRU{Примерный вывод программы скомпилированной в}{Here is a sample output if compiled in} MSVC:

\lstinputlisting{cpp/STL/vector/MSVC.txt}

\IFRU{Как можно заметить, выделенного буфера в самом начале ф-ции \main пока нет}
{As it can be seen, there is no allocated buffer at the \main function start yet}.
\IFRU{После первого вызова}{After first} \TT{push\_back()} \IFRU{буфер выделяется}{call buffer is allocated}.
\IFRU{И далее, после каждого вызова}{And then, after each} \TT{push\_back()} 
\IFRU{и длина массива и вместимость буфера (\IT{capacity}) увеличиваются}
{call, both array size and buffer size (\IT{capacity}) are increased}.
\IFRU{Но адрес буфера также меняется, потому что вызов ф-ции}{But buffer address is changed as well, because} \TT{push\_back()} \IFRU{перевыделяет буфер в \glslink{heap}{куче} каждый раз}
{function reallocates the buffer in the \gls{heap} each time}.
\IFRU{Это дорогая операция, вот почему очень важно предсказать размер будущего массива и зарезервировать место для него
при помощи метода}{It is costly operation, that's why it is very important to predict future array size and reserve 
a space for it with} \TT{.reserve()}\IFRU{}{ method}.
\IFRU{Самое последнее число это мусор: там нет элементов массива в этом месте, вот откуда это случайное число}
{The very last number is a garbage: there are no array elements at this point, so random number is printed}.
\IFRU{Это иллюстрация того факта что метод}{This is illustration to the fact that} \TT{operator[]} \IFRU{в}{of} 
\TT{std::vector} \IFRU{не проверяет индекс на правильность}{is not checking if the index in the array bounds}.
\IFRU{Метод }{}\TT{.at()} \IFRU{с другой стороны, проверяет, и подкидывает исключение}{method, however, does checking and 
throw} \TT{std::out\_of\_range} \IFRU{в случае ошибки}{exception in case of error}.

\IFRU{Давайте посмотрим код}{Let's see the code}:

\lstinputlisting[caption=MSVC 2012 /GS- /Ob1]{cpp/STL/vector/MSVC.asm}

\IFRU{Мы видим как метод}{We see how} \TT{.at()} \IFRU{проверяет границы и подкидывает исключение в случае ошибки}
{method check bounds and throw exception in case of error}.
\IFRU{Число, которое выводит последний вызов}{The number of the last} \printf \IFRU{берется из памяти, без всяких
проверок}{call is just to be taken from a memory, without any checks}.

\IFRU{Читатель может спросить, почему бы не использовать переменные}
{One may ask, why not to use variables like} ``size'' \AndENRU ``capacity'', 
\IFRU{как это сделано в}{like it was done in} \TT{std::string}.
\IFRU{Я подозреваю что это для более быстрой проверки границ}{I suppose, that was done for the faster bounds checking}.
\IFRU{Но я не уверен}{But I'm not sure}.

\IFRU{Код генерируемый GCC почти такой же, в целом, но метод}
{The code GCC generates is almost the same on the whole, but} \TT{.at()} \IFRU{вставлен прямо в код}{method is inlined}:

\lstinputlisting[caption=GCC 4.8.1 -fno-inline-small-functions -O1]{cpp/STL/vector/GCC.asm}

\IFRU{Метод }{}\TT{.reserve()} \IFRU{точно также вставлен прямо в код \main}{method is inlined as well}.
\IFRU{Он вызывает}{It calls} \TT{new()} \IFRU{если буфер слишком мал для нового массива}
{if buffer is too small for new size}, \IFRU{вызывает}{call} \TT{memmove()} 
\IFRU{для копирования содержимого буфера}{to copy buffer contents},
\IFRU{и вызывает}{and call} \TT{delete()} \IFRU{для освобождения старого буфера}{to free old buffer}.

\IFRU{Посмотрим, что выводит программа будучи скомпилированная GCC}
{Let's also see what the compiled program outputs if compiled by GCC}:

\lstinputlisting{cpp/STL/vector/GCC.txt}

\IFRU{Мы можем заметить, что буфер растет иначе чем в}{We can spot that buffer size grows in different way that in} MSVC.

\IFRU{При помощи простых экспериментов становится ясно, что в реализации MSVC буфер увеличивается
на}{Simple experimentation shows that MSVC implementation buffer grows by} \textasciitilde{}50\% \IFRU{каждый раз,
когда он должен был увеличен}{each time it needs to be enlarged},
\IFRU{а у GCC он увеличивается на}{while GCC code enlarges it by} 100\% \IFRU{каждый раз, т.е., удваивается}
{each time, i.e., doubles it each time}.

