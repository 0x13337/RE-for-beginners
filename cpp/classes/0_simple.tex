\section{\IFRU{Простой пример}{Simple example}}

\IFRU{Внутреннее представление классов в \Cpp почти такое же, как и представление структур.}
{Internally, \Cpp classes representation is almost the same as structures representation.}

\IFRU{Давайте попробуем простой пример с двумя переменными, двумя конструкторами и одним методом:}
{Let's try an example with two variables, two constructors and one method:}

\lstinputlisting{cpp/classes/simple.cpp}

\subsection{MSVC\EMDASH{}x86}

\IFRU{Вот как выглядит \main на ассемблере:}{Here is how \main function looks like translated 
into assembly language:}

\lstinputlisting[caption=MSVC]{cpp/classes/simple_1_msvc.asm}

\IFRU{Вот что происходит. 
Под каждый экземпляр класса \IT{c} выделяется по 8 байт, столько же, сколько нужно 
для хранения двух переменных.}
{So what's going on.
For each object (instance of class \IT{c}) 8 bytes allocated,
that is exactly size of 2 variables storage.}

\IFRU{Для \IT{c1} вызывается конструктор по умолчанию без аргументов \TT{??0c@@QAE@XZ}. 
Для \IT{c2} вызывается другой конструктор \TT{??0c@@QAE@HH@Z} и передаются два числа в качестве аргументов.}
{For \IT{c1} a default argumentless constructor \TT{??0c@@QAE@XZ} is called.
For \IT{c2} another constructor \TT{??0c@@QAE@HH@Z} is called and two numbers are passed as arguments.}

\index{thiscall}
\label{thiscall}
\index{x86!\Registers!ECX}
\IFRU{А указатель на объект (\ITthis в терминологии \Cpp) передается в регистре \ECX. 
Это называется thiscall~(\ref{thiscall}) ~--- метод передачи указателя на объект.}
{A pointer to object (\ITthis in \Cpp terminology) is passed in the \ECX register.
This is called thiscall~(\ref{thiscall})~---a pointer to object passing method.}

\IFRU{В данном случае, MSVC делает это через \ECX. Необходимо помнить, что это не стандартизированный метод, 
и другие компиляторы могут делать это иначе, например, через первый аргумент функции (как GCC).}
{MSVC doing it using the \ECX register. Needless to say, it is not a standardized method, other compilers could do it
differently, e.g., via first function argument (like GCC).}

\label{namemangling}
\index{Name mangling}
\index{\IFRU{ООП!Полиморфизм}{OOP!Polymorphism}}
\IFRU{Почему у имен функций такие странные имена? Это \gls{name mangling}.}
{Why these functions has so odd names?
That's \gls{name mangling}.}

\IFRU{В \Cpp, у класса, может иметься несколько методов с одинаковыми именами, 
но аргументами разных типов ~--- это полиморфизм. 
Ну и конечно, у разных классов могут быть методы с одинаковыми именами.}
{\Cpp class may contain several methods sharing the same name but having different 
arguments~---that is polymorphism.
And of course, different classes may own methods sharing the same name.}

\index{\IFRU{Компоновщик}{Linker}}
\IFRU{\IT{Name mangling} позволяет закодировать имя класса + имя метода + типы всех аргументов метода 
в одной ASCII-строке, которая затем используется как внутреннее имя функции. 
Это все потому что ни компоновщик\footnote{linker}, ни загрузчик DLL \ac{OS}
(мангленные имена могут быть среди экспортов/импортов в DLL), 
ничего не знают о \Cpp или \ac{OOP}.}
{\IT{Name mangling} enable us to encode class name + method name + all method argument types
in one ASCII-string, which is to be used as internal function name.
That's all because neither linker, nor DLL \ac{OS} loader (mangled names may be among 
DLL exports as well) knows nothing about \Cpp or \ac{OOP}.}

\IFRU{Далее вызывается два раза \TT{dump()}.}{\TT{dump()} function called two times after.}

\IFRU{Теперь смотрим на код в конструкторах:}{Now let's see constructors' code:}

\lstinputlisting[caption=MSVC]{cpp/classes/simple_2_msvc.asm}

\IFRU{Конструкторы ~--- это просто функции, они используют указатель на структуру в \ECX, 
перекладывают его себе в локальную переменную, хотя это и не обязательно.}
{Constructors are just functions, they use pointer to structure in the \ECX,
moving the pointer into own local variable, however, it is not necessary.}

\IFRU{Из стандарта \Cpp мы знаем}{From the \Cpp standard} \cite[12.1]{CPP11} 
\IFRU{что конструкторы не должны возвращать значение}{we know that constructors should not return
any values}.
\IFRU{В реальности, внутри, конструкторы возвращают указатель на созданный объект, т.е.}
{In fact, internally, constructors are returns pointer to the newly created object, i.e.}, \ITthis.

\IFRU{И еще метод \TT{dump()}:}{Now \TT{dump()} method:}

\lstinputlisting[caption=MSVC]{cpp/classes/simple_3_msvc.asm}

\IFRU{Все очень просто, \TT{dump()} берет указатель на структуру состоящую из двух \Tint через \ECX, 
выдергивает оттуда две переменные и передает их в \printf.}
{Simple enough: \TT{dump()} taking pointer to the structure containing two \Tint's in the \ECX,
takes two values from it and passing it into \printf.}

\IFRU{А если скомпилировать с оптимизацией (\Ox), то кода будет намного меньше:}
{The code is much shorter if compiled with optimization (\Ox):}

\lstinputlisting[caption=MSVC]{cpp/classes/simple_4_msvc_Ox.asm}

\index{x86!\Instructions!RET}
\IFRU{Вот и все. Единственное о чем еще нужно сказать, это о том что в функции \main, 
когда вызывался второй конструктор с двумя аргументами, за ним не корректировался стек при помощи 
\TT{add esp, X}. В то же время, в конце конструктора вместо \RET имеется \TT{RET 8}.}
{That's all. One more thing to say is the \gls{stack pointer} was not corrected
with \TT{add esp, X} after constructor called. 
Withal, constructor has \TT{ret 8} instead of the \RET at the end.}

\index{thiscall}
\IFRU{Это потому что здесь используется thiscall~(\ref{thiscall}), который, вместе с stdcall~(\ref{stdcall})
(все это ~--- методы передачи аргументов через стек), предлагает вызываемой функции корректировать стек. 
Инструкция \TT{ret X} сначала прибавляет \TT{X} к \ESP, затем передает управление вызывающей функции.}
{This is all because here used thiscall~(\ref{thiscall}) calling convention, the method of passing values through the
stack, which is, together with stdcall~(\ref{stdcall}) method, offers to correct stack to \gls{callee}
rather then to \gls{caller}.
\TT{ret x} instruction adding \TT{X} to the value in the \ESP, then passes control to the \gls{caller} function.}

\IFRU{См. также в соответствующем разделе о способах передачи аргументов через стек}
{See also section about calling conventions}~(\ref{sec:callingconventions}).

\IFRU{Еще, кстати, нужно отметить, что именно компилятор решает, когда вызывать конструктор и деструктор ~--- 
но это и так известно из основ языка \Cpp.}
{It is also should be noted the compiler deciding when to call constructor and 
destructor~---but that is we already know from \Cpp language basics.}

\subsection{MSVC\EMDASH{}x86-64}
\label{simple_CPP_MSVC_x64}

\IFRU{Как мы уже знаем, в x86-84 первые 4 аргумента ф-ции передаются через регистры \RCX, \RDX, \Reg{8}, 
\Reg{9}, а остальные\EMDASH{}через стек}
{As we already know, first 4 function arguments in x86-64 are passed in \RCX, \RDX, \Reg{8}, 
\Reg{9} registers, all the rest\EMDASH{}via stack}.
\IFRU{Тем не менее, указатель на объект \ITthis передается через \RCX, 
а первый аргумент метода\EMDASH{}в \RDX, итд}{Nevertheless, \ITthis pointer to the object is passed
in \RCX, first method argument\EMDASH{}in \EDX, etc}.
\IFRU{Здесь это видно во внутренностях метода}{We can see this in the} 
\TT{c(int a, int b)}\EN{ method internals}:

\lstinputlisting[caption=MSVC 2012 x64 /Ox]{cpp/classes/simple_MSVC_x64.asm}

\IFRU{Тип \Tint в x64 остается 32-битным
\footnote{Вероятно, так решили для упрощения портирования \CCpp{}-кода на x64}, 
поэтому здесь используются 32-битные части регистров}
{\Tint data type is still 32-bit in x64
\footnote{Apparently, for easier porting of \CCpp 32-bit code to x64}, 
so that is why 32-bit register's parts are used here}.

\IFRU{В методе \TT{dump()} вместо \RET мы видим \TT{JMP printf}, этот \IT{хак} мы рассматривали раннее}
{We also see \TT{JMP printf} instead of \RET in the \TT{dump()} method, that \IT{hack} we already saw
earlier}: \ref{JMP_instead_of_RET}.

\subsection{GCC\EMDASH{}x86}

\IFRU{В GCC 4.4.1 все почти так же, за исключением некоторых различий.}
{It is almost the same situation in GCC 4.4.1, with a few exceptions.}

\lstinputlisting[caption=GCC 4.4.1]{cpp/classes/simple_GCC.asm}

\IFRU{Здесь мы видим, что применяется иной \IT{name mangling} характерный для стандартов GNU
\footnote{Еще о name mangling разных компиляторов: \cite{AgnerFogCC}.}
Во-вторых, указатель на экземпляр передается как первый аргумент функции ~--- 
конечно же, скрыто от программиста.}
{Here we see another \IT{name mangling} style, specific to GNU
\footnote{One more document about different compilers name mangling types:
\cite{AgnerFogCC} standards.}
It is also can be noted the pointer to object is passed as first function 
argument~---transparently from programmer, of course.}

\IFRU{Это первый конструктор:}{First constructor:}

\begin{lstlisting}
                public _ZN1cC1Ev ; weak
_ZN1cC1Ev       proc near               ; CODE XREF: main+10

arg_0           = dword ptr  8

                push    ebp
                mov     ebp, esp
                mov     eax, [ebp+arg_0]
                mov     dword ptr [eax], 667
                mov     eax, [ebp+arg_0]
                mov     dword ptr [eax+4], 999
                pop     ebp
                retn
_ZN1cC1Ev       endp
\end{lstlisting}

\IFRU{Он просто записывает два числа по указателю переданному в первом (и единственном) аргументе.}
{What it does is just writes two numbers using pointer passed in first (and single) argument.}

\IFRU{Второй конструктор:}{Second constructor:}

\begin{lstlisting}
                public _ZN1cC1Eii
_ZN1cC1Eii      proc near

arg_0           = dword ptr  8
arg_4           = dword ptr  0Ch
arg_8           = dword ptr  10h

                push    ebp
                mov     ebp, esp
                mov     eax, [ebp+arg_0]
                mov     edx, [ebp+arg_4]
                mov     [eax], edx
                mov     eax, [ebp+arg_0]
                mov     edx, [ebp+arg_8]
                mov     [eax+4], edx
                pop     ebp
                retn
_ZN1cC1Eii      endp
\end{lstlisting}

\IFRU{Это функция, аналог которой мог бы выглядеть так:}{This is a function, analog of which could be looks like:}

\begin{lstlisting}
void ZN1cC1Eii (int *obj, int a, int b)
{
  *obj=a;
  *(obj+1)=b;
};
\end{lstlisting}

\dots \IFRU{что, в общем, предсказуемо}{and that is completely predictable}.

\IFRU{И функция \TT{dump()}:}{Now \TT{dump()} function:}

\begin{lstlisting}
                public _ZN1c4dumpEv
_ZN1c4dumpEv    proc near

var_18          = dword ptr -18h
var_14          = dword ptr -14h
var_10          = dword ptr -10h
arg_0           = dword ptr  8

                push    ebp
                mov     ebp, esp
                sub     esp, 18h
                mov     eax, [ebp+arg_0]
                mov     edx, [eax+4]
                mov     eax, [ebp+arg_0]
                mov     eax, [eax]
                mov     [esp+18h+var_10], edx
                mov     [esp+18h+var_14], eax
                mov     [esp+18h+var_18], offset aDD ; "%d; %d\n"
                call    _printf
                leave
                retn
_ZN1c4dumpEv    endp
\end{lstlisting}

\IFRU{Эта функция \IT{во внутреннем представлении} имеет один аргумент, через который передается указатель на 
объект\footnote{экземпляр класса} (\ITthis).}
{This function in its \IT{internal representation} has sole argument, 
used as pointer to the object (\ITthis).}

\IFRU{Таким образом, если брать в учет только эти простые примеры, разница между MSVC и GCC 
в способе кодирования имен функций (\IT{name mangling}) и передаче указателя на экземпляр класса 
(через \ECX или через первый аргумент).}
{Thus, if to base our judgment on these simple examples, the difference between MSVC and GCC
is style of function names encoding (\IT{name mangling}) and passing pointer to object
(via the \ECX register or via the first argument).}

\subsection{GCC\EMDASH{}x86-64}

\IFRU{Первые 6 аргументов, как мы уже знаем, передаются через 6 регистров}
{The first 6 arguments, as we already know, are passed in the} \RDI, \RSI, \RDX, \RCX, \Reg{8}, 
\Reg{9}\cite{SysVABI}\IFRU{, а указатель на \ITthis через первый (\RDI) что мы здесь и видим}{ registers,
and the pointer to \ITthis via first one (\RDI) and that is what we see here}.
\IFRU{Тип \Tint 32-битный и здесь}{\Tint data type is also 32-bit here}.
\IFRU{\IT{Хак} с \JMP вместо \RET используется и здесь}
{\JMP instead of \RET \IT{hack} is also used here}.

\lstinputlisting[caption=GCC 4.4.6 x64]{cpp/classes/simple_GCC_x64.s}

