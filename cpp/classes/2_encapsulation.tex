\section{\IFRU{Инкапсуляция}{Encapsulation}}

\IFRU{Инкапсуляция ~--- это сокрытие данных в \IT{private} секциях класса, например, чтобы разрешить доступ к ним только 
для методов этого класса, но не более.}{Encapsulation is data hiding in the \IT{private} sections of class, 
e.g. to allow access to them only from this class methods, but no more than.}

\IFRU{Однако, маркируется ли как-нибудь в коде тот факт, что некоторое поле ~--- приватное, 
а некоторое другое ~--- нет?}
{However, are there any marks in code about the fact that some field is private and
some other~---not?}

\IFRU{Нет, никак не маркируется.}{No, there are no such marks.}

\IFRU{Попробуем простой пример:}{Let's try simple example:}

\lstinputlisting{cpp/classes/classes2_0.cpp}

\IFRU{Снова скомпилируем в MSVC 2008 с опциями \Ox и \Obzero и посмотрим код метода \TT{box::dump()}:}
{Let's compile it again in MSVC 2008 with \Ox and \Obzero options and let's see \TT{box::dump()} method code:}

\lstinputlisting{cpp/classes/classes2_1.asm}

\IFRU{Разметка полей в классе выходит такой:}{Here is a memory layout of the class:}

\begin{center}
\begin{tabular}{ | l | l | }
\hline
  \tableheader{} \\
\hline
  +0x0 & int color \\
\hline
  +0x4 & int width \\
\hline
  +0x8 & int height \\
\hline
  +0xC & int depth \\
\hline
\end{tabular}
\end{center}

\IFRU{Все поля приватные и недоступные для модификации из других функций, но, зная эту разметку, 
сможем ли мы создать код модифицирующий эти поля?}{All fields are private and not allowed to access from any other
functions, but, knowing this layout, can we create a code modifying these fields?} 

\IFRU{Для этого я добавил функцию \TT{hack\_oop\_encapsulation()}, которая если обладает приведенным ниже телом, 
то просто не скомпилируется:}{So I added \TT{hack\_oop\_encapsulation()} function, which, if has the body
as follows, will not compile:}

\lstinputlisting{cpp/classes/classes2_2.cpp}

\IFRU{Тем не менее, если преобразовать тип \IT{box} к типу \IT{указатель на массив int}, 
и если модифицировать полученный массив \Tint{}-ов, тогда всё получится.}
{Nevertheless, if to cast \IT{box} type to \IT{pointer to int array},
and if to modify array of the \Tint{}-s we got, then we have success.}

\lstinputlisting{cpp/classes/classes2_3.cpp}

\IFRU{Код этой функции довольно прост ~--- можно сказать, функция берет на вход указатель на массив \Tint{}-ов и 
записывает \IT{123} во второй \Tint{}:}
{This functions' code is very simple~---it can be said, the function taking pointer to array of the \Tint{}-s on input
and writing \IT{123} to the second \Tint{}:}

\lstinputlisting{cpp/classes/classes2_5.asm}

\IFRU{Проверим, как это работает:}{Let's check, how it works:}

\lstinputlisting{cpp/classes/classes2_4.cpp}

\IFRU{Запускаем:}{Let's run:}

\begin{lstlisting}
this is box. color=1, width=10, height=20, depth=30
this is box. color=1, width=123, height=20, depth=30
\end{lstlisting}

\IFRU{Выходит, инкапсуляция ~--- это защита полей класса только на стадии компиляции.}{We see, encapsulation is
just class fields protection only on compiling stage.}
\IFRU{Компилятор Си++ не позволит сгенерировать код прямо
модифицирующий защищенные поля, тем не менее, используя \IT{грязные трюки} ~--- это вполне возможно.}
{C++ compiler will not allow to generate a code modifying protected fields straightforwardly, nevertheless,
it is possible with the help of \IT{dirty hacks}.}

