\chapter{\RU{Вызовы assert()}\EN{Calls to assert()}}
\index{\CStandardLibrary!assert()}
\RU{Может также помочь наличие \TT{assert()} в коде: обычно этот макрос оставляет название файла-исходника, 
номер строки, и условие.}
\EN{Sometimes the presence of the \TT{assert()} macro is useful too: 
commonly this macro leaves source file name, line number and condition in the code.}

\RU{Наиболее полезная информация содержится в assert-условии, по нему можно судить по именам переменных
или именам полей структур. Другая полезная информация\EMDASH{}это имена файлов, по их именам можно попытаться
предположить, что там за код. Также, по именам файлов можно опознать какую-либо очень известную опен-сорсную
библиотеку.}
\EN{The most useful information is contained in the assert's condition, we can deduce variable names or structure field
names from it. Another useful piece of information are the file names\EMDASH{}we can try to deduce what type of
code is there.
Also it is possible to recognize well-known open-source libraries by the file names.}

\lstinputlisting[caption=\RU{Пример информативных вызовов assert()}
\EN{Example of informative assert() calls}]{digging_into_code/assert_examples.lst}

\RU{Полезно \q{гуглить} и условия и имена файлов, это может вывести вас к опен-сорсной бибилотеке.
Например, если \q{погуглить} \q{sp->lzw\_nbits <= BITS\_MAX}, 
это вполне предсказуемо выводит на опенсорсный код, что-то связанное с LZW-компрессией.}
\EN{It is advisable to \q{google} both the conditions and file names, which can lead us to an open-source library.
For example, if we \q{google} \q{sp->lzw\_nbits <= BITS\_MAX}, this predictably 
gives us some open-source code that's related to the LZW compression.}
