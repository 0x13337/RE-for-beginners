\section{\IFRU{Вызовы assert()}{Calls to assert()}}
\index{\CStandardLibrary!assert()}
\IFRU{Может также помочь наличие \TT{assert()} в коде: обычно этот макрос оставляет название файла-исходника, 
номер строки, и условие.}
{Sometimes \TT{assert()} macro presence is useful too: 
commonly this macro leaves source file name, line number and condition in code.}

\IFRU{Наиболее полезная информация содержится в assert-условии, по нему можно судить по именам переменных
или именам полей структур. Другая полезная информация это имена файлов ~--- по их именам можно попытаться
предположить, что там за код. Так же, по именам файлов можно опознать какую-либо очень известную опен-сорсную
библиотеку.}
{Most useful information is contained in assert-condition, we can deduce variable names, or structure field
names from it. Another useful piece of information is file names~---we can try to deduce what type of
code is here.
Also by file names it is possible to recognize a well-known open-source libraries.}

\lstinputlisting[caption=\IFRU{Пример информативных вызовов assert()}
{Example of informative assert() calls}]{digging_into_code/assert_examples.lst}

\IFRU{Полезно ``гуглить'' и условия и имена файлов, это может вывести вас к опен-сорсной бибилотеке.
Например, если ``погуглить'' ``sp->lzw\_nbits <= BITS\_MAX'', 
это вполне предсказуемо выводит на опенсорсный код, что-то связанное с LZW-компрессией.}
{It is advisable to ``google'' both conditions and file names, that may lead us to open-source library.
For example, if to ``google'' ``sp->lzw\_nbits <= BITS\_MAX'', this predictably 
give us some open-source code, something related to LZW-compression.}
