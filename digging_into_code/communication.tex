\section{\IFRU{Связь с внешним миром}{Communication with the outer world}}

\IFRU{Первое на что нужно обратить внимание, это какие функции из API операционной 
системы и какие функции стандартных библиотек используются.}
{First what to look on is which functions from operation system API and standard libraries are used.}

\IFRU{Если программа поделена на главный исполняемый файл и группу DLL-файлов, 
то имена функций в этих DLL, бывает так, могут помочь.}
{If the program is divided into main executable file and a group of DLL-files, sometimes,
these function's names may be helpful.}

\IFRU{Если нас интересует, что именно приводит к вызову \TT{MessageBox()} с определенным текстом, 
то первое что можно попробовать сделать: найти в сегменте данных этот текст, найти ссылки на него, и найти, 
откуда может передаться управление к интересующему нас вызову \TT{MessageBox()}.}
{If we are interesting, what exactly may lead to \TT{MessageBox()} call with specific text,
first what we can try to do: find this text in data segment, find references to it and find the points
from which a control may be passed to \TT{MessageBox()} call we're interesting in.}

\IFRU{Если речь идет об игре, и нам интересно какие события в ней более-менее случайны, 
мы можем найти функцию \rand или её заменитель (как алгоритм Mersenne twister), и посмотреть, 
из каких мест эта функция вызывается и что самое главное: как используется результат этой функции.}
{If we are talking about some game and we're interesting, which events are more or less random in it,
we may try to find \rand function or its replacement (like Mersenne twister algorithm) and find a places
from which this function called and most important: how the results are used.}

\IFRU{Но если это не игра, а \rand используется, то также весьма любопытно, зачем. 
Бывают неожиданные случаи вроде использования \rand в алгоритме для сжатия данных (для имитации шифрования):}
{But if it's not a game, but \rand is used, it's also interesing, why.
There are cases of unexpected \rand usage in data compression algorithm (for encryption imitation):}
\url{http://blog.yurichev.com/node/44}.
