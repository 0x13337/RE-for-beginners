\section{\IFRU{Связь с внешним миром}{Communication with the outer world}}

\IFRU{Первое на что нужно обратить внимание, это какие функции из API \ac{OS} 
и какие функции стандартных библиотек используются.}
{First what to look on is which functions from \ac{OS} API and standard libraries are used.}

\IFRU{Если программа поделена на главный исполняемый файл и группу DLL-файлов, 
то имена функций в этих DLL, бывает так, могут помочь.}
{If the program is divided into main executable file and a group of DLL-files, sometimes,
these function's names may be helpful.}

\IFRU{Если нас интересует, что именно приводит к вызову \TT{MessageBox()} с определенным текстом, 
то первое что можно попробовать сделать: найти в сегменте данных этот текст, найти ссылки на него, и найти, 
откуда может передаться управление к интересующему нас вызову \TT{MessageBox()}.}
{If we are interesting, what exactly may lead to the \TT{MessageBox()} call with specific text,
first what we can try to do: find this text in data segment, find references to it and find the points
from which a control may be passed to the \TT{MessageBox()} call we're interesting in.}

\index{\CStandardLibrary!rand()}
\IFRU{Если речь идет о компьютерной игре, и нам интересно какие события в ней более-менее случайны, 
мы можем найти функцию \rand или её заменитель (как алгоритм Mersenne twister), и посмотреть, 
из каких мест эта функция вызывается и что самое главное: как используется результат этой функции.}
{If we are talking about a video game and we're interesting, which events are more or less random in it,
we may try to find \rand function or its replacement (like Mersenne twister algorithm) and find a places
from which this function called and most important: how the results are used.}

\IFRU{Но если это не игра, а \rand используется, то также весьма любопытно, зачем. 
Бывают неожиданные случаи вроде использования \rand в алгоритме для сжатия данных (для имитации шифрования):}
{But if it is not a game, but \rand is used, it is also interesing, why.
There are cases of unexpected \rand usage in data compression algorithm (for encryption imitation):}
\url{http://blog.yurichev.com/node/44}.

\subsection{\IFRU{Часто используемые ф-ции}{Often used functions in} Windows API}

\begin{itemize}

\item
\IFRU{Работа с реестром}{Registry access} (advapi32.dll): 
RegEnumKeyEx\footnote{\url{http://msdn.microsoft.com/en-us/library/windows/desktop/ms724862(v=vs.85).aspx}}
\footnote{
	\IFRU{Может иметь суффикс -A для ASCII-версии и -W для Unicode-версии}
	{May have -A suffix for ASCII-version and -W for Unicode-version}
	\label{note1}},
RegEnumValue\footnote{\url{http://msdn.microsoft.com/en-us/library/windows/desktop/ms724865(v=vs.85).aspx}}
\footnoteref{note1},
RegGetValue\footnote{\url{http://msdn.microsoft.com/en-us/library/windows/desktop/ms724868(v=vs.85).aspx}}
\footnoteref{note1},
RegOpenKeyEx\footnote{\url{http://msdn.microsoft.com/en-us/library/windows/desktop/ms724897(v=vs.85).aspx}}
\footnoteref{note1},
RegQueryValueEx\footnote{\url{http://msdn.microsoft.com/en-us/library/windows/desktop/ms724911(v=vs.85).aspx}}
\footnoteref{note1}.

\item
\IFRU{Работа с текстовыми .ini-файлами}{Access to text .ini-files} (kernel32.dll): 
GetPrivateProfileString
\footnote{\url{http://msdn.microsoft.com/en-us/library/windows/desktop/ms724353(v=vs.85).aspx}}
\footnoteref{note1}.

\item
\IFRU{Диалоговые окна}{Dialog boxes} (user32.dll): 
MessageBox
\footnote{\url{http://msdn.microsoft.com/en-us/library/ms645505(VS.85).aspx}}
\footnoteref{note1}, 
MessageBoxEx
\footnote{\url{http://msdn.microsoft.com/en-us/library/ms645507(v=vs.85).aspx}}
\footnoteref{note1},
SetDlgItemText
\footnote{\url{http://msdn.microsoft.com/en-us/library/ms645521(v=vs.85).aspx}}
\footnoteref{note1},
GetDlgItemText
\footnote{\url{http://msdn.microsoft.com/en-us/library/ms645489(v=vs.85).aspx}}
\footnoteref{note1}.

\item
\IFRU{Работа с ресурсами}{Resources access}\ref{PEresources}: (user32.dll): LoadMenu
\footnote{\url{http://msdn.microsoft.com/en-us/library/ms647990(v=vs.85).aspx}}
\footnoteref{note1}.
\item
\IFRU{Работа с TCP/IP-сетью}{TCP/IP-network} (ws2\_32.dll):
WSARecv
\footnote{\url{http://msdn.microsoft.com/en-us/library/windows/desktop/ms741688(v=vs.85).aspx}},
WSASend
\footnote{\url{http://msdn.microsoft.com/en-us/library/windows/desktop/ms742203(v=vs.85).aspx}}.

\item
\IFRU{Работа с файлами}{File access} (kernel32.dll):
CreateFile
\footnote{\url{http://msdn.microsoft.com/en-us/library/windows/desktop/aa363858(v=vs.85).aspx}}
\footnoteref{note1},
ReadFile
\footnote{\url{http://msdn.microsoft.com/en-us/library/windows/desktop/aa365467(v=vs.85).aspx}},
ReadFileEx
\footnote{\url{http://msdn.microsoft.com/en-us/library/windows/desktop/aa365468(v=vs.85).aspx}},
WriteFile
\footnote{\url{http://msdn.microsoft.com/en-us/library/windows/desktop/aa365747(v=vs.85).aspx}},
WriteFileEx
\footnote{\url{http://msdn.microsoft.com/en-us/library/windows/desktop/aa365748(v=vs.85).aspx}}.

\item
\IFRU{Высокоуровневая работа с}{High-level access to the} Internet
(wininet.dll):
WinHttpOpen
\footnote{\url{http://msdn.microsoft.com/en-us/library/windows/desktop/aa384098(v=vs.85).aspx}}.

\item
\IFRU{Стандартная библиотека MSVC (при случае динамического связывания)}
{Standard MSVC library (in case of dynamic linking)} (msvcr*.dll):
assert, itoa, ltoa, open, printf, read, strcmp, atol, atoi, fopen, fread, fwrite, memcmp, rand,
strlen, strstr, strchr.

\end{itemize}

\subsection{tracer: \IFRU{Перехват всех ф-ций в отдельном модуле}
{Intercepting all functions in specific module}}

\IFRU{В \tracer есть INT3-брякпоинты, хотя и срабатывающие только один раз, но зато их можно установить на все
сразу ф-ции в некоей DLL.}
{There is INT3-breakpoints in \tracer, triggering only once, however, they can be set to all functions
in specific DLL.}

\begin{lstlisting}
--one-time-INT3-bp:somedll.dll!.*
\end{lstlisting}

\IFRU{Либо, поставим INT3-прерывание на все функции, имена которых начинаются с префикса \TT{xml}:}
{Or, let's set INT3-breakpoints to all functions with \TT{xml} prefix in name:}

\begin{lstlisting}
--one-time-INT3-bp:somedll.dll!xml.*
\end{lstlisting}

\IFRU{В качестве обратной стороны медали, такие прерывания срабатывают только один раз.}
{On the other side of coin, such breakpoints are triggered only once.}

Tracer \IFRU{покажет вызов какой-либо функции, если он случится, но только один раз.}
{will show calling of a function, if it happens, but only once.}
\IFRU{Еще один недостаток --- увидеть аргументы функции также нельзя.}
{Another drawback --- it is not possible to see function's arguments.}

\IFRU{Тем не менее, эта возможность очень удобна для тех ситуаций, 
когда вы знаете что некая программа использует некую DLL,
но не знаете какие именно функции в этой DLL.}
{Nevertheless, this feature is very useful when you know the program uses a DLL,
but do not know which functions in it.}
\IFRU{И функций много.}{And there are a lot of functions.} \\
\\
\index{cygwin}
\IFRU{Например, попробуем узнать, что использует cygwin-утилита uptime}
{For example, let's see, what uptime cygwin-utility uses}:

\begin{lstlisting}
tracer -l:uptime.exe --one-time-INT3-bp:cygwin1.dll!.*
\end{lstlisting}

\IFRU{Так мы можем увидеть все ф-ции из библиотеки cygwin1.dll, которые были вызваны хотя бы один раз, и откуда}
{Thus we may see all cygwin1.dll library functions which were called at least once, and where from}:

\lstinputlisting{digging_into_code/uptime_cygwin.txt}

