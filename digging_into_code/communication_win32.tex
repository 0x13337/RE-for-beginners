\chapter{\RU{Связь с внешним миром}\EN{Communication with the outer world} (win32)}

\RU{Иногда, чтобы понять что делает та или иная ф-ция, можно её не разбирать, а просто посмотреть
на её входы и выходы.}
\EN{Sometimes it's enough to observe some function's inputs and outputs in order to understand
what it does.}
\RU{Так можно сэкономить время}\EN{That way you can save time}.

\RU{Обращения к файлам и реестру}\EN{Files and registry access}: 
\RU{для самого простого анализа может помочь утилита}\EN{for the very basic
analysis, } Process Monitor\footnote{\url{http://go.yurichev.com/17301}}
\RU{от}\EN{utility from} SysInternals\EN{ can help}. 

\RU{Для анализа обращения программы к сети, может помочь}\EN{For the basic analysis of
network accesses,} Wireshark\footnote{\url{http://go.yurichev.com/17303}}\EN{ can be useful}.

\RU{Затем всё-таки придётся смотреть внутрь}\EN{But then you will need to look inside anyway}. \\
\\
\RU{Первое на что нужно обратить внимание, это какие функции из \ac{API} \ac{OS} 
и какие функции стандартных библиотек используются.}
\EN{The first thing to look for is which functions from the \ac{OS}'s \ac{API}s and standard libraries are used.}

\RU{Если программа поделена на главный исполняемый файл и группу DLL-файлов, 
то имена функций в этих DLL, бывает так, могут помочь.}
\EN{If the program is divided into a main executable file and a group of DLL files, sometimes
the names of the functions in these DLLs can help.}

\RU{Если нас интересует, что именно приводит к вызову \TT{MessageBox()} с определенным текстом, 
то первое что можно попробовать сделать: найти в сегменте данных этот текст, найти ссылки на него, и найти, 
откуда может передаться управление к интересующему нас вызову \TT{MessageBox()}.}
\EN{If we are interested in exactly what can lead to a call to \TT{MessageBox()} with specific text, 
we can try to find this text in the data segment, find the references to it and find the points
from which the control may be passed to the \TT{MessageBox()} call we're interested in.}

\index{\CStandardLibrary!rand()}
\RU{Если речь идет о компьютерной игре, и нам интересно какие события в ней более-менее случайны, 
мы можем найти функцию \rand или её заменитель (как алгоритм Mersenne twister), и посмотреть, 
из каких мест эта функция вызывается и что самое главное: как используется результат этой функции.}
\EN{If we are talking about a video game and we're interested in which events are more or less random in it,
we may try to find the \rand function or its replacements (like the Mersenne twister algorithm) and find the places
from which those functions are called, and more importantly, how are the results used.}
\RU{Один пример}\EN{One example}: \ref{chap:color_lines}.

\RU{Но если это не игра, а \rand используется, то также весьма любопытно, зачем. 
Бывают неожиданные случаи вроде использования \rand в алгоритме для сжатия данных (для имитации шифрования):}
\EN{But if it is not a game, and \rand is still used, it is also interesting to know why.
There are cases of unexpected \rand usage in data compression algorithms (for encryption imitation):}
\href{http://go.yurichev.com/17221}{blog.yurichev.com}.

\section{\RU{Часто используемые ф-ции}\EN{Often used functions in the} Windows API}

\RU{Это ф-ции которые можно увидеть в числе импортируемых}\EN{These functions may be among the imported}.
\RU{Но также нельзя забывать, что далеко не все они были использованы в коде написанном автором}
\EN{It is worth to note that not every function might be used in the code that was written by the programmer}.
\RU{Немалая часть может вызываться из библиотечных ф-ций и}\EN{A lot of functions might
be called from library functions and} \ac{CRT}\RU{-кода}\EN{ code}.

\begin{itemize}

\item
\RU{Работа с реестром}\EN{Registry access} (advapi32.dll): 
RegEnumKeyEx\footnote{\href{http://go.yurichev.com/17228}{MSDN}}
\footnote{
	\RU{Может иметь суффикс -A для ASCII-версии и -W для Unicode-версии}
	\EN{May have the -A suffix for the ASCII version and -W for the Unicode version}
	\label{note1}},
RegEnumValue\footnote{\href{http://go.yurichev.com/17229}{MSDN}}
\footnoteref{note1},
RegGetValue\footnote{\href{http://go.yurichev.com/17230}{MSDN}}
\footnoteref{note1},
RegOpenKeyEx\footnote{\href{http://go.yurichev.com/17231}{MSDN}}
\footnoteref{note1},
RegQueryValueEx\footnote{\href{http://go.yurichev.com/17232}{MSDN}}
\footnoteref{note1}.

\item
\RU{Работа с текстовыми .ini-файлами}\EN{Access to text .ini-files} (kernel32.dll): 
GetPrivateProfileString
\footnote{\href{http://go.yurichev.com/17233}{MSDN}}
\footnoteref{note1}.

\item
\RU{Диалоговые окна}\EN{Dialog boxes} (user32.dll): 
MessageBox
\footnote{\href{http://go.yurichev.com/17234}{MSDN}}
\footnoteref{note1}, 
MessageBoxEx
\footnote{\href{http://go.yurichev.com/17235}{MSDN}}
\footnoteref{note1},
SetDlgItemText
\footnote{\href{http://go.yurichev.com/17236}{MSDN}}
\footnoteref{note1},
GetDlgItemText
\footnote{\href{http://go.yurichev.com/17237}{MSDN}}
\footnoteref{note1}.

\item
\RU{Работа с ресурсами}\EN{Resources access}(\ref{PEresources}): (user32.dll): LoadMenu
\footnote{\href{http://go.yurichev.com/17238}{MSDN}}
\footnoteref{note1}.
\item
\RU{Работа с TCP/IP-сетью}\EN{TCP/IP networking} (ws2\_32.dll):
WSARecv
\footnote{\href{http://go.yurichev.com/17239}{MSDN}},
WSASend
\footnote{\href{http://go.yurichev.com/17240}{MSDN}}.

\item
\RU{Работа с файлами}\EN{File access} (kernel32.dll):
CreateFile
\footnote{\href{http://go.yurichev.com/17241}{MSDN}}
\footnoteref{note1},
ReadFile
\footnote{\href{http://go.yurichev.com/17242}{MSDN}},
ReadFileEx
\footnote{\href{http://go.yurichev.com/17243}{MSDN}},
WriteFile
\footnote{\href{http://go.yurichev.com/17244}{MSDN}},
WriteFileEx
\footnote{\href{http://go.yurichev.com/17245}{MSDN}}.

\item
\RU{Высокоуровневая работа с}\EN{High-level access to the} Internet
(wininet.dll):
WinHttpOpen
\footnote{\href{http://go.yurichev.com/17246}{MSDN}}.

\item
\RU{Проверка цифровой подписи исполняемого файла}
\EN{Checking the digital signature of an executable file} (wintrust.dll):
WinVerifyTrust
\footnote{\href{http://go.yurichev.com/17247}{MSDN}}.

\item
\RU{Стандартная библиотека MSVC (при случае динамического связывания)}
\EN{The standard MSVC library (if it's linked dynamically)} (msvcr*.dll):
assert, itoa, ltoa, open, printf, read, strcmp, atol, atoi, fopen, fread, fwrite, memcmp, rand,
strlen, strstr, strchr.

\end{itemize}

\section{tracer: \RU{Перехват всех ф-ций в отдельном модуле}
\EN{Intercepting all functions in specific module}}
\index{tracer}

\index{x86!\Instructions!INT3}
\RU{В \tracer есть INT3-брякпойнты, хотя и срабатывающие только один раз, но зато их можно установить на все
сразу ф-ции в некоей DLL.}
\EN{There are INT3 breakpoints in the \tracer, that are triggered only once, however, they can be set for all functions
in a specific DLL.}

\begin{lstlisting}
--one-time-INT3-bp:somedll.dll!.*
\end{lstlisting}

\RU{Либо, поставим INT3-прерывание на все функции, имена которых начинаются с префикса \TT{xml}:}
\EN{Or, let's set INT3 breakpoints on all functions with the \TT{xml} prefix in their name:}

\begin{lstlisting}
--one-time-INT3-bp:somedll.dll!xml.*
\end{lstlisting}

\RU{В качестве обратной стороны медали, такие прерывания срабатывают только один раз.}
\EN{On the other side of the coin, such breakpoints are triggered only once.}

Tracer \RU{покажет вызов какой-либо функции, если он случится, но только один раз.}
\EN{will show the call of a function, if it happens, but only once.}
\RU{Еще один недостаток --- увидеть аргументы функции также нельзя.}
\EN{Another drawback~---it is impossible to see the function's arguments.}

\RU{Тем не менее, эта возможность очень удобна для тех ситуаций, 
когда вы знаете что некая программа использует некую DLL,
но не знаете какие именно функции в этой DLL.}
\EN{Nevertheless, this feature is very useful when you know that the program uses a DLL,
but you do not know which functions are actually used.}
\RU{И функций много.}\EN{And there are a lot of functions.} \\
\\
\index{Cygwin}
\RU{Например, попробуем узнать, что использует cygwin-утилита uptime:}
\EN{For example, let's see, what does the uptime utility from cygwin use:}

\begin{lstlisting}
tracer -l:uptime.exe --one-time-INT3-bp:cygwin1.dll!.*
\end{lstlisting}

\RU{Так мы можем увидеть все ф-ции из библиотеки cygwin1.dll, которые были вызваны хотя бы один раз, и откуда}
\EN{Thus we may see all that cygwin1.dll library functions that were called at least once, and where from}:

\lstinputlisting{digging_into_code/uptime_cygwin.txt}

