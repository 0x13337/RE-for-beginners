\chapter{\RU{Константы}\EN{Constants}}

\RU{Люди, включая программистов, часто используют круглые числа вроде}
\EN{Humans, including programmers, often use round numbers like} 10, 100, 1000, 
\RU{в т.ч. и в коде}\EN{as well as in the code}.

\RU{Практикующие реверсеры, обычно, хорошо знают их в шестнадцатеричном представлении}
\EN{Practicing reverse engineer, usually know them well in hexadecimal representation}:
10=0xA, 100=0x64, 1000=0x3E8, 10000=0x2710.

\RU{Иногда попадаются константы}\EN{Constants} \TT{0xAAAAAAAA} (10101010101010101010101010101010) \AndENRU 
\TT{0x55555555} (01010101010101010101010101010101) \RU{--- это чередующиеся биты}\EN{ are also popular---this
is alternating bits}.
\RU{Например, константа}\EN{For example} \TT{0x55AA} \RU{используется как минимум в бут-секторе}\EN{constant
is used at least in boot-sector}, \ac{MBR}, 
\AndENRU \InENRU \ac{ROM} \RU{плат-расширений IBM-компьютеров}\EN{of IBM-compatibles extension cards}.

\RU{Некоторые алгоритмы, особенно криптографические, используют хорошо различимые константы, 
которые при помощи \IDA легко находить в коде.}
\EN{Some algorithms, especially cryptographical, use distinct constants, which is easy to find
in code using \IDA.}

\index{MD5}
\newcommand{\URLMD}{\RU{http://ru.wikipedia.org/wiki/MD5}\EN{http://en.wikipedia.org/wiki/MD5}}

\RU{Например алгоритм MD5\footnote{\url{\URLMD}} инициализирует свои внутренние переменные так:}
\EN{For example, MD5\footnote{\url{\URLMD}} algorithm initializes its own internal variables like:}

\begin{verbatim}
var int h0 := 0x67452301
var int h1 := 0xEFCDAB89
var int h2 := 0x98BADCFE
var int h3 := 0x10325476
\end{verbatim}

\RU{Если в коде найти использование этих четырех констант подряд ~--- 
очень высокая вероятность что эта функция имеет отношение к MD5.}
\EN{If you find these four constants usage in the code in a row~---it is very high probability this function is related to MD5.} \\
\\
\RU{Еще такой пример это алгоритмы CRC16/CRC32, часто, алгоритмы вычисления контрольной суммы по CRC 
используют зараннее заполненные таблицы, вроде}\EN{Another example is CRC16/CRC32 algorithms, 
often, calculation algorithms use precomputed tables like}:

\begin{lstlisting}[caption=linux/lib/crc16.c]
/** CRC table for the CRC-16. The poly is 0x8005 (x^16 + x^15 + x^2 + 1) */
u16 const crc16_table[256] = {
	0x0000, 0xC0C1, 0xC181, 0x0140, 0xC301, 0x03C0, 0x0280, 0xC241,
	0xC601, 0x06C0, 0x0780, 0xC741, 0x0500, 0xC5C1, 0xC481, 0x0440,
	0xCC01, 0x0CC0, 0x0D80, 0xCD41, 0x0F00, 0xCFC1, 0xCE81, 0x0E40,
	...
\end{lstlisting}

\RU{См. также таблицу CRC32}\EN{See also precomputed table for CRC32}: \ref{sec:CRC32}.

\section{Magic numbers}

\newcommand{\FNURLMAGIC}{\footnote{\url{http://en.wikipedia.org/wiki/Magic_number_(programming)}}}

\RU{Немало форматов файлов определяет стандартный заголовок файла где используются \IT{magic numbers}\FNURLMAGIC{}.}
\EN{A lot of file formats defining a standard file header where \IT{magic number}\FNURLMAGIC{} is used.}

\index{MS-DOS}
\RU{Скажем, все исполняемые файлы для Win32 и MS-DOS начинаются с двух символов}
\EN{For example, all Win32 and MS-DOS executables are started with two characters} ``MZ''\footnote{\url{http://en.wikipedia.org/wiki/DOS_MZ_executable}}.

\index{MIDI}
\RU{В начале MIDI-файла должно быть ``MThd''. Если у нас есть использующая для чего-нибудь MIDI-файлы программа
очень вероятно, что она будет проверять MIDI-файлы на правильность хотя бы проверяя первые 4 байта.}
\EN{At the MIDI-file beginning ``MThd'' signature must be present. 
If we have a program which uses MIDI-files for something,
very likely, it must check MIDI-files for validity by checking at least first 4 bytes.}

\RU{Это можно сделать при помощи:}\EN{This could be done like:}

\RU{(\IT{buf} указывает на начало загруженного в память файла)}
\EN{(\IT{buf} pointing to the beginning of loaded file into memory)}

\begin{verbatim}
cmp [buf], 0x6468544D ; "MThd"
jnz _error_not_a_MIDI_file
\end{verbatim}

\index{\CStandardLibrary!memcmp()}
\index{x86!\Instructions!CMPSB}
\RU{\dots либо вызвав функцию сравнения блоков памяти \TT{memcmp()} или любой аналогичный код, 
вплоть до инструкции \TT{CMPSB} (\ref{REPE_CMPSx}).}
\EN{\dots or by calling function for comparing memory blocks \TT{memcmp()} or any other equivalent code
up to a \TT{CMPSB} (\ref{REPE_CMPSx}) instruction.}

\RU{Найдя такое место мы получаем как минимум информацию о том, где начинается загрузка MIDI-файла, во-вторых, 
мы можем увидеть где располагается буфер с содержимым файла, и что еще оттуда берется, и как используется.}
\EN{When you find such point you already may say where MIDI-file loading is starting,
also, we could see a location
of MIDI-file contents buffer and what is used from the buffer, and how.}

\subsection{DHCP}

\RU{Это касается также и сетевых протоколов. 
Например, сетевые пакеты протокола DHCP содержат так называемую \IT{magic cookie}: \TT{0x63538263}. 
Какой-либо код, генерирующий пакеты по протоколу DHCP где-то и как-то должен внедрять в пакет также и эту константу. 
Найдя её в коде мы сможем найти место где происходит это и не только это. 
\IT{Что-либо} что получает пакеты по DHCP должно где-то как-то проверять \IT{magic cookie}, 
сравнивая это поле пакета с константой.}
\EN{This applies to network protocols as well.
For example, DHCP protocol network packets contains so-called \IT{magic cookie}: \TT{0x63538263}.
Any code generating DHCP protocol packets somewhere and somehow must embed this constant into packet.
If we find it in the code we may find where it happen and not only this.
\IT{Something} received DHCP packet must check \IT{magic cookie}, comparing it with the constant.}

\RU{Например, берем файл dhcpcore.dll из Windows 7 x64 и ищем эту константу. 
И находим, два раза: оказывается, эта константа используется в функциях с красноречивыми 
названиями}
\EN{For example, let's take dhcpcore.dll file from Windows 7 x64 and search for the constant.
And we found it, two times:
it seems, the constant is used in two functions eloquently 
named as} \TT{DhcpExtractOptionsForValidation()} \AndENRU \TT{DhcpExtractFullOptions()}:

\begin{lstlisting}[caption=dhcpcore.dll (Windows 7 x64)]
.rdata:000007FF6483CBE8 dword_7FF6483CBE8 dd 63538263h          ; DATA XREF: DhcpExtractOptionsForValidation+79
.rdata:000007FF6483CBEC dword_7FF6483CBEC dd 63538263h          ; DATA XREF: DhcpExtractFullOptions+97
\end{lstlisting}

\RU{А вот те места в функциях где происходит обращение к константам:}
\EN{And the places where these constants accessed:}

\begin{lstlisting}[caption=dhcpcore.dll (Windows 7 x64)]
.text:000007FF6480875F  mov     eax, [rsi]
.text:000007FF64808761  cmp     eax, cs:dword_7FF6483CBE8
.text:000007FF64808767  jnz     loc_7FF64817179
\end{lstlisting}

\RU{И:}\EN{And:}

\begin{lstlisting}[caption=dhcpcore.dll (Windows 7 x64)]
.text:000007FF648082C7  mov     eax, [r12]
.text:000007FF648082CB  cmp     eax, cs:dword_7FF6483CBEC
.text:000007FF648082D1  jnz     loc_7FF648173AF
\end{lstlisting}

\section{\RU{Поиск констант}\EN{Constant searching}}

\RU{В \IDA это очень просто, Alt-B или Alt-I.}
\EN{It is easy in \IDA: Alt-B or Alt-I.}
\index{binary grep}
\RU{А для поиска константы в большом количестве файлов, либо для поиска их в неисполняемых файлах, 
я написал небольшую утилиту}
\EN{And for searching for constant in big pile of files, or for searching it in non-executable files,
I wrote small utility}
\IT{binary grep}\footnote{\BGREPURL}.
