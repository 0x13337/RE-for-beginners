\chapter{\RU{Идентификация исполняемых файлов}\EN{Identification of executable files}}

\section{Microsoft Visual C++}
\label{MSVC_versions}

\RU{Версии MSVC и DLL которые могут быть импортированы}\EN{MSVC versions and DLLs that can be imported}:

\begin{center}
\begin{tabular}{ | l | l | l | l | l | }
\hline
\cellcolor{blue!25} \RU{Маркетинговая версия}\EN{Marketing version} & 
\cellcolor{blue!25} \RU{Внутренняя версия}\EN{Internal version} & 
\cellcolor{blue!25} \RU{Версия }CL.EXE\EN{ version} &
\cellcolor{blue!25} \RU{Импортируемые DLL}\EN{DLLs that can be imported} &
\cellcolor{blue!25} \RU{Дата выхода}\EN{Release date} \\
\hline
% 4.0, April 1995
% 97 & 5.0 & February 1997
6		&  6.0 & 12.00 & msvcrt.dll, msvcp60.dll    & June 1998 \\
\hline
.NET (2002)	&  7.0 & 13.00 & msvcr70.dll, msvcp70.dll   & February 13, 2002 \\
\hline
.NET 2003	&  7.1 & 13.10 & msvcr71.dll, msvcp71.dll   & April 24, 2003 \\
\hline
2005		&  8.0 & 14.00 & msvcr80.dll, msvcp80.dll   & November 7, 2005 \\
\hline
2008		&  9.0 & 15.00 & msvcr90.dll, msvcp90.dll   & November 19, 2007 \\
\hline
2010		& 10.0 & 16.00 & msvcr100.dll, msvcp100.dll & April 12, 2010 \\
\hline
2012		& 11.0 & 17.00 & msvcr110.dll, msvcp110.dll & September 12, 2012 \\
\hline
2013		& 12.0 & 18.00 & msvcr120.dll, msvcp120.dll & October 17, 2013 \\
\hline
\end{tabular}
\end{center}

msvcp*.dll \RU{содержит функции связанные с \Cpp{}, так что если она импортируется, скорее всего, 
вы имеете дело с программой на \Cpp}\EN{contain \Cpp{}-related functions, so if it is imported, 
this is probably a \Cpp program}.

\subsection{Name mangling}

\RU{Имена обычно начинаются с символа}\EN{The names usually start with the} \TT{?}\EN{ symbol}.

\RU{О}\EN{You can read more about MSVC's} \gls{name mangling} \RU{в MSVC читайте также здесь}\EN{here}: \myref{namemangling}.

\section{GCC}
\index{GCC}

\RU{Кроме компиляторов под *NIX, GCC имеется так же и для win32-окружения: в виде}
\EN{Aside from *NIX targets, GCC is also present in the win32 environment, in the form of} Cygwin \AndENRU MinGW.

\subsection{Name mangling}

\RU{Имена обычно начинаются с символов}\EN{Names usually start with the} \TT{\_Z}\EN{ symbols}.

\RU{О}\EN{You can read more about GCC's} \gls{name mangling} \RU{в GCC читайте также здесь}\EN{here}: \myref{namemangling}.

\subsection{Cygwin}
\index{Cygwin}

cygwin1.dll \RU{часто импортируется}\EN{is often imported}.

\subsection{MinGW}
\index{MinGW}

msvcrt.dll \RU{может импортироваться}\EN{may be imported}.

\section{Intel FORTRAN}
\index{FORTRAN}

libifcoremd.dll, libifportmd.dll \AndENRU libiomp5md.dll (\RU{поддержка }OpenMP\EN{ support}) 
\RU{могут импортироваться}\EN{may be imported}.

\RU{В }libifcoremd.dll \RU{много функций с префиксом}\EN{has a lot of functions prefixed with} 
\TT{for\_}, \RU{что значит}\EN{which means} FORTRAN.

\section{Watcom, OpenWatcom}
\index{Watcom}
\index{OpenWatcom}

\subsection{Name mangling}

\RU{Имена обычно начинаются с символа}\EN{Names usually start with the} \TT{W}\EN{ symbol}.

\RU{Например, так кодируется метод \q{method} класса \q{class} не имеющий аргументов и возвращающий \Tvoid}
\EN{For example, that is how the method named \q{method} of the class \q{class} that does not have any arguments and returns
\Tvoid is encoded}:

\begin{lstlisting}
W?method$_class$n__v
\end{lstlisting}

\section{Borland}
\index{Borland Delphi}
\index{Borland C++Builder}

\RU{Вот пример \gls{name mangling} в Borland Delphi и C++Builder}
\EN{Here is an example of Borland Delphi's and C++Builder's \gls{name mangling}}:

\lstinputlisting{digging_into_code/identification/borland_mangling.txt}

\RU{Имена всегда начинаются с символа}\EN{The names always start with the} \TT{@} 
\RU{затем следует имя класса, имя метода
и закодированные типы аргументов}\EN{symbol, then we have the class name came, method name, and encoded the types of the arguments of the method}.

\RU{Эти имена могут присутствовать с импортах .exe, экспортах .dll, отладочной информации,}
\EN{These names can be in the .exe imports, .dll exports, debug data,}\etc{}.

Borland Visual Component Libraries (VCL) \RU{находятся в файлах .bpl вместо .dll, например}
\EN{are stored in .bpl files instead of .dll ones, for example}, vcl50.dll, rtl60.dll.

\RU{Другие DLL которые могут импортироваться}\EN{Another DLL that might be imported}: BORLNDMM.DLL.

\subsection{Delphi}

\RU{Почти все исполняемые файлы имеют текстовую строку}\EN{Almost all Delphi executables has the} \q{Boolean} 
\RU{в самом начале сегмента кода, среди остальных имен типов}
\EN{text string at the beginning of the code segment, along with other type names}.

\RU{Вот очень характерное для Delphi начало сегмента \TT{CODE}, 
этот блок следует сразу за заголовком win32 PE-файла}
\EN{This is a very typical beginning of the \TT{CODE} 
segment of a Delphi program, this block came right after the win32 PE file header}:

\lstinputlisting{digging_into_code/identification/delphi.txt}

\RU{Первые 4 байта сегмента данных (\TT{DATA}) в исполняемых файлах могут быть}
\EN{The first 4 bytes of the data segment (\TT{DATA}) can be} 
\TT{00 00 00 00}, \TT{32 13 8B C0} \OrENRU\ \TT{FF FF FF FF}. 
\RU{Эта информация может помочь при работе с запакованными/зашифрованными программами на Delphi.}
\EN{This information can be useful when dealing with packed/encrypted Delphi executables.}

\section{\RU{Другие известные DLL}\EN{Other known DLLs}}

\begin{itemize}
\index{OpenMP}
\item vcomp*.dll\EMDASH{}\RU{Реализация OpenMP от Microsoft}\EN{Microsoft's implementation of OpenMP}.
\end{itemize}

