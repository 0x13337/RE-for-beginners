\chapter{Identification of executable files}

\section{Microsoft Visual C++}
\label{MSVC_versions}

MSVC versions and DLLs that can be imported:

\begin{center}
\begin{tabular}{ | l | l | l | l | l | }
\hline
\HeaderColor Marketing version & 
\HeaderColor Internal version & 
\HeaderColor CL.EXE version &
\HeaderColor DLLs that can be imported &
\HeaderColor Release date \\
\hline
% 4.0, April 1995
% 97 & 5.0 & February 1997
6		&  6.0 & 12.00 & msvcrt.dll, msvcp60.dll    & June 1998 \\
\hline
.NET (2002)	&  7.0 & 13.00 & msvcr70.dll, msvcp70.dll   & February 13, 2002 \\
\hline
.NET 2003	&  7.1 & 13.10 & msvcr71.dll, msvcp71.dll   & April 24, 2003 \\
\hline
2005		&  8.0 & 14.00 & msvcr80.dll, msvcp80.dll   & November 7, 2005 \\
\hline
2008		&  9.0 & 15.00 & msvcr90.dll, msvcp90.dll   & November 19, 2007 \\
\hline
2010		& 10.0 & 16.00 & msvcr100.dll, msvcp100.dll & April 12, 2010 \\
\hline
2012		& 11.0 & 17.00 & msvcr110.dll, msvcp110.dll & September 12, 2012 \\
\hline
2013		& 12.0 & 18.00 & msvcr120.dll, msvcp120.dll & October 17, 2013 \\
\hline
\end{tabular}
\end{center}

msvcp*.dll contain \Cpp{}-related functions, so if it is imported, 
this is probably a \Cpp program.

\subsection{Name mangling}

The names usually start with the \TT{?} symbol.

You can read more about MSVC's \gls{name mangling} here: \myref{namemangling}.

\section{GCC}
\myindex{GCC}

Aside from *NIX targets, GCC is also present in the win32 environment, in the form of Cygwin and MinGW.

\subsection{Name mangling}

Names usually start with the \TT{\_Z} symbols.

You can read more about GCC's \gls{name mangling} here: \myref{namemangling}.

\subsection{Cygwin}
\myindex{Cygwin}

cygwin1.dll is often imported.

\subsection{MinGW}
\myindex{MinGW}

msvcrt.dll may be imported.

\section{Intel FORTRAN}
\myindex{FORTRAN}

libifcoremd.dll, libifportmd.dll and libiomp5md.dll (OpenMP support) may be imported.

libifcoremd.dll has a lot of functions prefixed with \TT{for\_}, which means FORTRAN.

\section{Watcom, OpenWatcom}
\myindex{Watcom}
\myindex{OpenWatcom}

\subsection{Name mangling}

Names usually start with the \TT{W} symbol.

For example, that is how the method named \q{method} of the class \q{class} that does not have any arguments and returns
\Tvoid is encoded:

\begin{lstlisting}
W?method$_class$n__v
\end{lstlisting}

\section{Borland}
\myindex{Borland Delphi}
\myindex{Borland C++Builder}

Here is an example of Borland Delphi's and C++Builder's \gls{name mangling}:

\lstinputlisting{digging_into_code/identification/borland_mangling.txt}

The names always start with the \TT{@} 
symbol, then we have the class name came, method name, and encoded the types of the arguments of the method.

These names can be in the .exe imports, .dll exports, debug data, etc.

Borland Visual Component Libraries (VCL) 
are stored in .bpl files instead of .dll ones, for example, vcl50.dll, rtl60.dll.

Another DLL that might be imported: BORLNDMM.DLL.

\subsection{Delphi}

Almost all Delphi executables has the \q{Boolean} text string at the beginning of the code segment, along with other type names.

This is a very typical beginning of the \TT{CODE} 
segment of a Delphi program, this block came right after the win32 PE file header:

\lstinputlisting{digging_into_code/identification/delphi.txt}

The first 4 bytes of the data segment (\TT{DATA}) can be \TT{00 00 00 00}, \TT{32 13 8B C0} or \TT{FF FF FF FF}.%

This information can be useful when dealing with packed/encrypted Delphi executables.

\section{Other known DLLs}

\myindex{OpenMP}
\begin{itemize}
\item vcomp*.dll\EMDASH{}Microsoft's implementation of OpenMP.
\end{itemize}

