\section{\IFRU{Использование magic numbers для трассировки}{Using magic numbers while tracing}}

\IFRU{Нередко бывает нужно узнать, как используется то или иное значение, прочитанное из файла либо взятое из пакета,
принятого по сети. Часто, ручное слежение за нужной переменной это трудный процесс. Один из простых методов (хотя и не
полностью надежный на 100\%) это использование вашей собственной \IT{magic number}.}
{Often, main goal is to get to know, how a value was read from file, or received via network, being used. 
Often, manual tracing of a value is very labouring task. One of the simplest techniques (although not 100\% reliable) 
is to use your own \IT{magic number}.}

\IFRU{Это чем-то напоминает компьютерную томографию: пациенту перед сканированием вводят в кровь 
рентгеноконтрастный препарат, хорошо отсвечивающий в рентгеновских лучах. Известно, как кровь нормального человека
расходится, например, по почкам, и если в этой крови будет препарат, то при томографии будет хорошо видно,
достаточно ли хорошо кровь расходится по почкам и нет ли там камней, например, и прочих образований.}
{This resembling X-ray computed tomography is some sense: radiocontrast agent is often injected into patient's blood,
which is used for improving visibility of internal structures in X-rays. For example, it is well known how blood of healthy man/woman
percolates in kidneys and if agent is in blood, it will be easily seen on tomography, how good and normal blood was percolating,
are there any stones or tumors.}

\IFRU{Мы можем взять 32-битное число вроде \TT{0x0badf00d}, либо чью-то дату рождения вроде \TT{0x11101979} 
и записать это, занимающее 4 байта число, в какое-либо место файла используемого исследуемой нами программой.}
{We can take a 32-bit number like \TT{0x0badf00d}, or someone's birth date like \TT{0x11101979}
and to write this, 4 byte holding number, to some point in file used by the program we investigate.}

\index{\GrepUsage}
\IFRU{Затем, при трассировки этой программы, в том числе, при помощи \tracer в режиме 
\IT{code coverage}, а затем при помощи
\IT{grep} или простого поиска по текстовому файлу с результатами трассировки, мы можем легко увидеть, в каких местах кода использовалось 
это значение, и как.}
{Then, while tracing this program, with \tracer in the \IT{code coverage} mode, and then, with the help of \IT{grep}
or just by searching in the text file (of tracing results), we can easily see, where the value was used and how.}

\IFRU{Пример результата работы \tracer в режиме \IT{cc}, к которому легко применить утилиту \IT{grep}}{Example 
of \IT{grepable} \tracer results in the \IT{cc} mode}:

\begin{lstlisting}
0x150bf66 (_kziaia+0x14), e=       1 [MOV EBX, [EBP+8]] [EBP+8]=0xf59c934 
0x150bf69 (_kziaia+0x17), e=       1 [MOV EDX, [69AEB08h]] [69AEB08h]=0 
0x150bf6f (_kziaia+0x1d), e=       1 [FS: MOV EAX, [2Ch]] 
0x150bf75 (_kziaia+0x23), e=       1 [MOV ECX, [EAX+EDX*4]] [EAX+EDX*4]=0xf1ac360 
0x150bf78 (_kziaia+0x26), e=       1 [MOV [EBP-4], ECX] ECX=0xf1ac360 
\end{lstlisting}

\IFRU{Это справедливо также и для сетевых пакетов.
Важно только, чтобы наш \IT{magic number} был как можно более уникален и не присутствовал в самом коде.}
{This can be used for network packets as well.
It is important to be unique for \IT{magic number} and not to be present in the program's code.}

\newcommand{\DOSBOXURL}{\url{http://blog.yurichev.com/node/55}}

\index{DosBox}
\index{MS-DOS}
\IFRU{Помимо \tracer, такой эмулятор MS-DOS как DosBox, в режиме heavydebug, может писать в отчет информацию обо всех
состояниях регистра на каждом шаге исполнения программы\footnote{См. также мой пост в блоге об этой возможности в 
DosBox: \DOSBOXURL{}}, так что этот метод может пригодиться и для исследования программ под DOS.}{Aside of 
\tracer, DosBox (MS-DOS emulator) in heavydebug mode,
is able to write information about all register's states for each executed instruction of program to plain text file\footnote{See also my 
blog post about this DosBox feature: \DOSBOXURL{}}, so this technique may be useful for DOS programs as well.}

