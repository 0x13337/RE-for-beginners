\subsection{\IFRU{Сравнение ``снимков'' памяти}{Memory ``snapshots'' comparing}}

\IFRU{Метод простого сравнения двух снимков памяти для поиска изменений часто применялся для взлома игр 
на 8-битных компьютерах и взлома файлов с записанными рекордными очками.}
{The method of simple two memory snapshots comparing in order to see changes, was often used to hack
8-bit computer games and hacking ``high score'' files.}

\IFRU{К примеру, если вы имеете загруженную игру на 8-битном компьютере (где самой памяти не очень много, но игра
занимает еще меньше), и вы знаете что сейчас у вас, условно, 100 пуль, вы можете сделать ``снимок'' всей
памяти и сохранить где-то. Затем просто стреляете куда угодно, у вас станет 99 пуль, сделать второй ``снимок'',
и затем сравнить эти два снимка: где-то наверняка должен быть байт, который в начале был 100, а затем стал 99.}
{For example, if you got some loaded game on 8-bit computer (it is not much memory on these, but game is usually
consumes even less memory) and you know that you have now, let's say, 100 bullets, you can do a ``snapshot''
of all memory and back it up to some place. Then shoot somewhere, bullet count now 99, do second ``snapshot''
and then compare both: somewhere should be a byte which was 100 in the beginning and now it is 99.}
\IFRU{Если учесть что игры на тех маломощных домашних компьютерах обычно были написанны на ассемблере и подобные
переменные там были глобальные, то можно с уверенностью сказать, какой адрес в памяти всегда отвечает за количество
пуль. Если поискать в дизассемблированном коде игры все обращения по этому адресу, несложно найти код,
отвечающий за уменьшение пуль и записать туда инструкцию \NOP\footnote{``no operation'', холостая инструкция} 
или несколько \NOP-в, так мы получим игру в которой у игрока всегда будет 100 пуль, например.}
{Considering a fact that these 8-bit games were often written in assembly language and such variables were global,
it can be said for sure, which address in memory holding bullets count. If to search all references to that
address in disassembled game code, it is not very hard to find a piece of code decrementing bullets count,
write \NOP instruction\footnote{``no operation'', idle operation} there, or couple of \NOP{}-s, 
we'll have a game with 100 (for example) bullets forever.}
\index{BASIC!POKE}
\IFRU{А так как игры на тех домашних 8-битных 
компьютерах всегда загружались по одним и тем же адресам, и версий одной игры редко когда было больше одной,
то геймеры-энтузиасты знали, по какому адресу (используя инструкцию языка BASIC \IT{POKE}\footnote{инструкция языка BASIC записывающая байт по определенному адресу}) что записать после загрузки
игры, чтобы хакнуть её. Это привело к появлению списков ``читов'' состоящих из инструкций \IT{POKE}, публикуемых
в журналах посвященным 8-битным играм. См.также:}{Games on these 8-bit computers was usually loaded on the same
address, also, there were no much different versions of each game (usually, just one version is popular),
enthusiastic gamers knew, which byte should be written (using BASIC instruction \IT{POKE}\footnote{BASIC language instruction writting byte on specific address}) to which address in
order to hack it. This led to ``cheat'' lists containing of \IT{POKE} instructions published in magazines related to
8-bit games. See also:} \url{http://en.wikipedia.org/wiki/PEEK\_and\_POKE}.

\IFRU{Точно также легко модифицировать файлы с сохраненными рекордами (кто сколько очков набрал), впрочем, это может
сработать не только с 8-битными играми. Нужно заметить, какой у вас сейчас рекорд и где-то сохранить файл
с очками. Затем, когда очков станет другое количество, просто сравнить два файла, можно даже
DOS-утилитой FC\footnote{утилита MS-DOS для сравнения двух файлов побайтово} (файлы рекордов, часто, бинарные).}
{Likewise, it is easy to modify ``high score'' files, this may work not only with 8-bit games. Let's notice 
your score count and back the file up somewhere. When ``high score'' count will be different, just compare two files,
it can be even done with DOS-utility FC\footnote{MS-DOS utility for binary files comparing} (``high score'' files
are often in binary form).}
\IFRU{Где-то будут отличаться несколько байт, и легко будет увидеть, какие именно отвечают за количество очков. 
Впрочем, разработчики игр осведомлены о таких хитростях и могут защититься от этого.}
{There will be some point where couple of bytes will be different and it will be easy to see which ones are
holding score number.
However, game developers are aware of such tricks and may protect against it.}

% FIXME: пример с какой-то простой игрушкой?

