% FIXME comparison!
\section{\RU{Сравнение \q{снимков} памяти}\EN{Memory \q{snapshots} comparing}}
\label{snapshots_comparing}

\RU{Метод простого сравнения двух снимков памяти для поиска изменений часто применялся для взлома игр 
на 8-битных компьютерах и взлома файлов с записанными рекордными очками.}
\EN{The technique of the straightforward comparison of two memory snapshots in order to see changes was often used to hack
8-bit computer games and for hacking \q{high score} files.}

\RU{К примеру, если вы имеете загруженную игру на 8-битном компьютере (где самой памяти не очень много, но игра
занимает еще меньше), и вы знаете что сейчас у вас, условно, 100 пуль, вы можете сделать \q{снимок} всей
памяти и сохранить где-то. Затем просто стреляете куда угодно, у вас станет 99 пуль, сделать второй \q{снимок},
и затем сравнить эти два снимка: где-то наверняка должен быть байт, который в начале был 100, а затем стал 99.}
\EN{For example, if you had a loaded game on an 8-bit computer (there isn't much memory on these, but the game usually
consumes even less memory) and you know that you have now, let's say, 100 bullets, you can do a \q{snapshot}
of all memory and back it up to some place. Then shoot once, the bullet count goes to 99, do a second \q{snapshot}
and then compare both: the must be must be a byte somewhere which was 100 in the beginning, and now it is 99.}
\RU{Если учесть, что игры на тех маломощных домашних компьютерах обычно были написаны на ассемблере и подобные
переменные там были глобальные, то можно с уверенностью сказать, какой адрес в памяти всегда отвечает за количество
пуль. Если поискать в дизассемблированном коде игры все обращения по этому адресу, несложно найти код,
отвечающий за уменьшение пуль и записать туда инструкцию \gls{NOP}
или несколько \gls{NOP}-в, так мы получим игру в которой у игрока всегда будет 100 пуль, например.}
\EN{Considering the fact that these 8-bit games were often written in assembly language and such variables were global,
it can be said for sure which address in memory was holding the bullet count. If you searched for all references to the
address in the disassembled game code, it was not very hard to find a piece of code \glslink{decrement}{decrementing} the bullet count,
then to write a \gls{NOP} instruction there, or a couple of \gls{NOP}-s, 
and then have a game with 100 bullets forever.}
\myindex{BASIC!POKE}
\RU{А так как игры на тех домашних 8-битных 
компьютерах всегда загружались по одним и тем же адресам, и версий одной игры редко когда было больше одной продолжительное время,
то геймеры-энтузиасты знали, по какому адресу (используя инструкцию языка BASIC \gls{POKE}) что записать после загрузки
игры, чтобы хакнуть её. Это привело к появлению списков \q{читов} состоящих из инструкций \gls{POKE}, публикуемых
в журналах посвященным 8-битным играм. См. также:}\EN{Games on these 8-bit computers were commonly loaded at the constant
address, also, there were not much different versions of each game (commonly just one version was popular for a long span of time),
so enthusiastic gamers knew which bytes must be overwritten (using the BASIC's instruction \gls{POKE}) at which address in
order to hack it. This led to \q{cheat} lists that contained \gls{POKE} instructions, published in magazines related to
8-bit games. See also:} \href{http://go.yurichev.com/17114}{wikipedia}.

\myindex{MS-DOS}
\RU{Точно так же легко модифицировать файлы с сохраненными рекордами (кто сколько очков набрал), впрочем, это может
сработать не только с 8-битными играми. Нужно заметить, какой у вас сейчас рекорд и где-то сохранить файл
с очками. Затем, когда очков станет другое количество, просто сравнить два файла, можно даже
DOS-утилитой FC\footnote{утилита MS-DOS для сравнения двух файлов побайтово} (файлы рекордов, часто, бинарные).}
\EN{Likewise, it is easy to modify \q{high score} files, this does not work with just 8-bit games. Notice 
your score count and back up the file somewhere. When the \q{high score} count gets different, just compare the two files,
it can even be done with the DOS utility FC\footnote{MS-DOS utility for comparing binary files} (\q{high score} files
are often in binary form).}
\RU{Где-то будут отличаться несколько байт, и легко будет увидеть, какие именно отвечают за количество очков. 
Впрочем, разработчики игр полностью осведомлены о таких хитростях и могут защититься от этого.}
\EN{There will be a point where a couple of bytes are different and it is easy to see which ones are
holding the score number.
However, game developers are fully aware of such tricks and may defend the program against it.}

\RU{В каком-то смысле похожий пример в этой книге здесь}
\EN{Somewhat similar example in this book is}: \myref{Millenium_DOS_game}.

% TODO: пример с какой-то простой игрушкой?

\subsection{\RU{Реестр Windows}\EN{Windows registry}}

\RU{А еще можно вспомнить сравнение реестра Windows до инсталляции программы и после}
\EN{It is also possible to compare the Windows registry before and after a program installation}.
\RU{Это также популярный метод поиска, какие элементы реестра программа использует.}
\EN{It is a very popular method of finding which registry elements are used by the program.}
\EN{Probably, this is the reason why the \q{windows registry cleaner} shareware is so popular.}
\RU{Наверное это причина, почему так популярны shareware-программы для очистки реестра в Windows.}

\subsection{\EN{Blink-comparator}\RU{Блинк-компаратор}}

\RU{Сравнение файлов или слепков памяти вообще, немного напоминает блинк-компаратор
\footnote{\url{http://go.yurichev.com/17349}}:
устройство, которое раньше использовали астрономы для поиска движущихся небесных объектов.}
\EN{Comparison of files or memory snapshots remind us blink-comparator
\footnote{\url{http://go.yurichev.com/17348}}:
a device used by astronomers in past, intended to find moving celestial objects.}
\RU{Блинк-компаратор позволял быстро переключаться между двух отснятых в разное время кадров,
и астроном мог увидеть разницу визуально.}
\EN{Blink-comparator allows to switch quickly between two photographies shot in different time,
so astronomer would spot the difference visually.}
\RU{Кстати, при помощи блинк-компаратора, в 1930 был открыт Плутон.}
\EN{By the way, Pluto was discovered by blink-comparator in 1930.}
