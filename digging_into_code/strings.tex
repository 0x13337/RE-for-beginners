\section{\IFRU{Строки}{String}}

\IFRU{Очень сильно помогают отладочные сообщения, если они имеются. В некотором смысле, отладочные сообщения, 
это отчет о том, что сейчас происходит в программе. Зачастую, это \printf-подобные функции, 
которые пишут куда-нибудь в лог, а бывает так что и не пишут ничего, но вызовы остались, так как эта сборка ~--- 
не отладочная, а release.}
{Debugging messages are often very helpful if present. In some sense, debugging messages are reporting
about what's going on in program right now. Often these are \printf-like functions,
which writes to log-files, and sometimes, not writing anything but calls are still present, because this build
is not debug build but release one.}
\index{Oracle RDBMS}
\IFRU{Если в отладочных сообщениях дампятся значения некоторых локальных или глобальных переменных, 
это тоже может помочь, как минимум, узнать их имена. 
Например, в Oracle RDBMS одна из таких функций: \TT{ksdwrt()}.}
{If local or global variables are dumped in debugging messages, it might be helpful as well because it's 
possible to get variable names at least.
For example, one of such functions in Oracle RDBMS is \TT{ksdwrt()}.}

\newcommand{\CONUSONE}{http://blog.yurichev.com/node/32}
\newcommand{\CONUSTWO}{http://blog.yurichev.com/node/43}

\IFRU{Осмысленные текстовые строки вообще очень сильно могут помочь. 
Дизассемблер \IDA может сразу указать, из какой функции и из какого её места используется эта строка. 
Попадаются и \href{\CONUSONE}{смешные случаи}.}
{Meaningful text strings are often helpful.
\IDA disassembler may show from which function and from which point this specific string is used.
Funny cases \href{\CONUSONE}{sometimes happen}.}

\IFRU{Парадоксально, но сообщения об ошибках также могут помочь найти то что нужно. 
В Oracle RDBMS сигнализация об ошибках проходит при помощи вызова некоторой группы функций. 
\href{\CONUSTWO}{Тут еще немного об этом}.}
{Paradoxically, but error messages may help us as well.
In Oracle RDBMS, errors are reporting using group of functions.
\href{\CONUSTWO}{More about it}.}

\index{Error messages}
\IFRU{Можно довольно быстро найти, какие функции сообщают о каких ошибках, и при каких условиях.}
{It's possible to find very quickly, which functions reporting about errors and in which conditions.}
\IFRU{Это, кстати, одна из причин, почему в защите софта от копирования, 
бывает так, что сообщение об ошибке заменяется 
невнятным кодом или номером ошибки. Мало кому приятно, если взломщик быстро поймет, 
из за чего именно срабатывает защита от копирования, просто по сообщению об ошибке.}
{By the way, it's often a reason why copy-protection systems has inarticulate cryptic error messages 
or just error numbers. No one happy when software cracker quickly understand why copy-protection
is triggered just by error message.}
