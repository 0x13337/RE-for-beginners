\section{\IFRU{Подозрительные паттерны кода}{Suspicious code patterns}}

\subsection{\IFRU{Вручную написанный код на ассемблере}{Hand-written assembly code}}

\index{Function prologue}
\index{Function epilogue}
\index{x86!\Instructions!LOOP}
\index{x86!\Instructions!RCL}
\IFRU{Современные компиляторы не генерируют инструкции \TT{LOOP} и \TT{RCL}. 
С другой стороны, эти инструкции хорошо знакомы кодерам предпочитающим писать прямо на ассемблере. 
Если такие инструкции встретились, можно сказать с какой-то вероятностью, что этот фрагмент кода написан вручную.
Также, пролог/эпилог функции обычно не встречается в ассемблерном коде написанном вручную.}
{Modern compilers do not emit \TT{LOOP} and \TT{RCL} instructions.
On the other hand, these instructions are well-known to coders who like to code in straight assembly language.
If you spot these, it can be said, with a high probability, this fragment of code is hand-written.
Also function prologue/epilogue is not commonly present in hand-written assembly copy.}

\IFRU{Как правило, в вручную написанном коде, нет никакого четкого метода передачи аргументов в 
функцию}{Commonly there are no fixed system in passing arguments into functions in hand-written
code}.

\IFRU{Пример из ядра}{Example from} Windows 2003\IFRU{}{ kernel} 
(\IFRU{файл }{}ntoskrnl.exe\IFRU{}{ file}):

\begin{lstlisting}
MultiplyTest    proc near               ; CODE XREF: Get386Stepping
                xor     cx, cx
loc_620555:                             ; CODE XREF: MultiplyTest+E
                push    cx
                call    Multiply
                pop     cx
                jb      short locret_620563
                loop    loc_620555
                clc
locret_620563:                          ; CODE XREF: MultiplyTest+C
                retn
MultiplyTest    endp

Multiply        proc near               ; CODE XREF: MultiplyTest+5
                mov     ecx, 81h
                mov     eax, 417A000h
                mul     ecx
                cmp     edx, 2
                stc
                jnz     short locret_62057F
                cmp     eax, 0FE7A000h
                stc
                jnz     short locret_62057F
                clc
locret_62057F:                          ; CODE XREF: Multiply+10
                                        ; Multiply+18
                retn
Multiply        endp
\end{lstlisting}

\IFRU{Действительно, если заглянуть в исходные коды}{Indeed, if we look into} 
Windows Research Kernel v1.2\IFRU{, данный код можно найти в файле}{ source code, this code
can be found easily in the file} 
\IT{WRK-v1.2\textbackslash{}base\textbackslash{}ntos\textbackslash{}ke\textbackslash{}i386\textbackslash{}cpu.asm}.
