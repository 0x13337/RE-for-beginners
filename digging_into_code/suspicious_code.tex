\chapter{\RU{Подозрительные паттерны кода}\EN{Suspicious code patterns}}

\section{\RU{Инструкции XOR}\EN{XOR instructions}}
\index{x86!\Instructions!XOR}

\RU{Инструкции вроде}\EN{Instructions like} \TT{XOR op, op} (\RU{например}\EN{for example}, \TT{XOR EAX, EAX}) 
\RU{обычно используются для обнуления регистра,
однако, если операнды разные, то применяется операция именно}\EN{are usually used for setting the register value
to zero, but if the operands are different, the} \q{\RU{исключающего или}\EN{exclusive or}}\EN{ operation
is executed}.
\RU{Эта операция очень редко применяется в обычном программировании, но применяется очень часто в криптографии,
включая любительскую.}
\EN{This operation is rare in common programming, but widespread in cryptography,
including amateur one.}
\RU{Особенно подозрительно, если второй операнд\EMDASH{}это большое число}\EN{It's especially suspicious if the
second operand is a big number}.
\RU{Это может указывать на шифрование, вычисление контрольной суммы,}
\EN{This may point to encrypting/decrypting, checksum computing,}\etc{}.\\
\\
\ifx\LITE\undefined
\RU{Одно из исключений из этого наблюдения о котором стоит сказать, то, что генерация и проверка значения \q{канарейки}
(\myref{subsec:BO_protection}) часто происходит, используя инструкцию \XOR.}
\EN{One exception to this observation worth noting is the \q{canary} (\myref{subsec:BO_protection}). 
Its generation and checking are often done using the \XOR instruction.} \\
\\
\fi
\index{AWK}
\RU{Этот AWK-скрипт можно использовать для обработки листингов (.lst) созданных \IDA{}}
\EN{This AWK script can be used for processing \IDA{} listing (.lst) files}:

\begin{lstlisting}
gawk -e '$2=="xor" { tmp=substr($3, 0, length($3)-1); if (tmp!=$4) if($4!="esp") if ($4!="ebp") { print $1, $2, tmp, ",", $4 } }' filename.lst
\end{lstlisting}

\ifx\LITE\undefined
\RU{Нельзя также забывать,
что если использовать подобный скрипт, то, возможно, он захватит и неверно дизассемблированный
код}\EN{It is also worth noting that this kind of script can also match incorrectly disassembled code} 
(\myref{sec:incorrectly_disasmed_code}).
\fi

\section{\RU{Вручную написанный код на ассемблере}\EN{Hand-written assembly code}}

\index{Function prologue}
\index{Function epilogue}
\index{x86!\Instructions!LOOP}
\index{x86!\Instructions!RCL}
\RU{Современные компиляторы не генерируют инструкции \TT{LOOP} и \TT{RCL}. 
С другой стороны, эти инструкции хорошо знакомы кодерам, предпочитающим писать прямо на ассемблере. 
\ifx\LITE\undefined
Подобные инструкции отмечены как (M) в списке инструкций в приложении: 
\myref{sec:x86_instructions}.
\fi
Если такие инструкции встретились, можно сказать с какой-то вероятностью, что этот фрагмент кода написан вручную.}
\EN{Modern compilers do not emit the \TT{LOOP} and \TT{RCL} instructions.
On the other hand, these instructions are well-known to coders who like to code directly in assembly language.
If you spot these, it can be said that there is a high probability that this fragment of code was hand-written.
\ifx\LITE\undefined
Such instructions are marked as (M) in the instructions list in this appendix: 
\myref{sec:x86_instructions}.
\fi
}\\
\\
\RU{Также, пролог/эпилог функции обычно не встречается в ассемблерном коде, написанном вручную.}
\EN{Also the function prologue/epilogue are not commonly present in hand-written assembly.}\\
\\
\RU{Как правило, в вручную написанном коде, нет никакого четкого метода передачи аргументов в 
функцию}
\EN{Commonly there is no fixed system for passing arguments to functions in the hand-written
code}.\\
\\
\RU{Пример из ядра}\EN{Example from the} Windows 2003\EN{ kernel} 
(\RU{файл }ntoskrnl.exe\EN{ file}):

\begin{lstlisting}
MultiplyTest proc near               ; CODE XREF: Get386Stepping
             xor     cx, cx
loc_620555:                          ; CODE XREF: MultiplyTest+E
             push    cx
             call    Multiply
             pop     cx
             jb      short locret_620563
             loop    loc_620555
             clc
locret_620563:                       ; CODE XREF: MultiplyTest+C
             retn
MultiplyTest endp

Multiply     proc near               ; CODE XREF: MultiplyTest+5
             mov     ecx, 81h
             mov     eax, 417A000h
             mul     ecx
             cmp     edx, 2
             stc
             jnz     short locret_62057F
             cmp     eax, 0FE7A000h
             stc
             jnz     short locret_62057F
             clc
locret_62057F:                       ; CODE XREF: Multiply+10
                                     ; Multiply+18
             retn
Multiply     endp
\end{lstlisting}

\RU{Действительно, если заглянуть в исходные коды}\EN{Indeed, if we look in the} 
\ac{WRK} v1.2\RU{, данный код можно найти в файле}\EN{ source code, this code
can be found easily in file} 
\IT{WRK-v1.2\textbackslash{}base\textbackslash{}ntos\textbackslash{}ke\textbackslash{}i386\textbackslash{}cpu.asm}.
