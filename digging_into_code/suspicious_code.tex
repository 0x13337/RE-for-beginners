\section{\IFRU{Подозрительные паттерны кода}{Suspicious code patterns}}

\subsection{\IFRU{Инструкции XOR}{XOR instructions}}
\index{x86!\Instructions!XOR}

\IFRU{Инструкции вроде}{instructions like} \TT{XOR op, op} (\IFRU{например}{for example}, \TT{XOR EAX, EAX}) 
\IFRU{обычно используются для обнуления регистра,
однако, если операнды разные, то применяется операция именно}{are usually used for setting register value
to zero, but if operands are different,} \IT{\IFRU{исключающего или}{exclusive or}}\EN{ operation
is executed}.
\IFRU{Эта операция очень редко применяется в обычном программировании, но применяется очень часто в криптографии,
включая любительскую}{This operation is rare in common programming, but used often in cryptography,
including amateur}.
\IFRU{Особенно подозрительно, если второй операнд это большое число}{Especially suspicious case if the
second operand is big number}.
\IFRU{Это может указывать на шифрование, вычисление контрольной суммы, итд}
{This may points to encrypting/decrypting, checksum computing, etc}.\\
\\
\IFRU{Одно из исключений из этого наблюдения о котором стоит сказать, то, что генерация и проверка значения ``канарейки''
(\ref{subsec:BO_protection}) часто происходит используя инструкцию \XOR}
{One exception to this observation worth to note is ``canary'' (\ref{subsec:BO_protection}) generation and checking is often done using \XOR
instruction}. \\
\\
\index{AWK}
\IFRU{Этот AWK-скрипт можно использовать для обработки листингов (.lst) созданных \IDA{}}
{This AWK script can be used for processing \IDA{} listing (.lst) files}:

\begin{lstlisting}
gawk -e '$2=="xor" { tmp=substr($3, 0, length($3)-1); if (tmp!=$4) if($4!="esp") if ($4!="ebp") { print $1, $2, tmp, ",", $4 } }' filename.lst
\end{lstlisting}

\IFRU{Нельзя также забывать,
что если использовать подобный скрипт, то, возможно, он захватит и неверно дизассемблированный
код}{It is also worth to note that such script may also capture incorrectly disassembled code} 
(\ref{sec:incorrectly_disasmed_code}).

\subsection{\IFRU{Вручную написанный код на ассемблере}{Hand-written assembly code}}

\index{Function prologue}
\index{Function epilogue}
\index{x86!\Instructions!LOOP}
\index{x86!\Instructions!RCL}
\IFRU{Современные компиляторы не генерируют инструкции \TT{LOOP} и \TT{RCL}. 
С другой стороны, эти инструкции хорошо знакомы кодерам, предпочитающим писать прямо на ассемблере. 
Подобные инструкции отмечены как (M) в списке инструкций в приложении: \ref{sec:x86_instructions}.
Если такие инструкции встретились, можно сказать с какой-то вероятностью, что этот фрагмент кода написан вручную.}
{Modern compilers do not emit \TT{LOOP} and \TT{RCL} instructions.
On the other hand, these instructions are well-known to coders who like to code in straight assembly language.
If you spot these, it can be said, with a high probability, this fragment of code is hand-written.
Such instructions are marked as (M) in the instructions list in appendix: \ref{sec:x86_instructions}.}\\
\\
\IFRU{Также, пролог/эпилог функции обычно не встречается в ассемблерном коде, написанном вручную.}
{Also function prologue/epilogue is not commonly present in hand-written assembly copy.}\\
\\
\IFRU{Как правило, в вручную написанном коде, нет никакого четкого метода передачи аргументов в 
функцию}
{Commonly there is no fixed system in passing arguments into functions in the hand-written
code}.\\
\\
\IFRU{Пример из ядра}{Example from} Windows 2003\EN{ kernel} 
(\RU{файл }ntoskrnl.exe\EN{ file}):

\begin{lstlisting}
MultiplyTest    proc near               ; CODE XREF: Get386Stepping
                xor     cx, cx
loc_620555:                             ; CODE XREF: MultiplyTest+E
                push    cx
                call    Multiply
                pop     cx
                jb      short locret_620563
                loop    loc_620555
                clc
locret_620563:                          ; CODE XREF: MultiplyTest+C
                retn
MultiplyTest    endp

Multiply        proc near               ; CODE XREF: MultiplyTest+5
                mov     ecx, 81h
                mov     eax, 417A000h
                mul     ecx
                cmp     edx, 2
                stc
                jnz     short locret_62057F
                cmp     eax, 0FE7A000h
                stc
                jnz     short locret_62057F
                clc
locret_62057F:                          ; CODE XREF: Multiply+10
                                        ; Multiply+18
                retn
Multiply        endp
\end{lstlisting}

\IFRU{Действительно, если заглянуть в исходные коды}{Indeed, if we look into} 
\ac{WRK} v1.2\IFRU{, данный код можно найти в файле}{ source code, this code
can be found easily in the file} 
\IT{WRK-v1.2\textbackslash{}base\textbackslash{}ntos\textbackslash{}ke\textbackslash{}i386\textbackslash{}cpu.asm}.
