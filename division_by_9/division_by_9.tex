\section{\DivisionByNineSectionName}
\label{sec:divisionbynine}

\IFRU{Простая функция:}{Very simple function:}

\begin{lstlisting}
int f(int a)
{
	return a/9;
};
\end{lstlisting}

\IFRU{Компилируется вполне предсказуемо:}{Is compiled in a very predictable way:}

\lstinputlisting{\IFRU{division_by_9/11_1_msvc_ru.asm}{division_by_9/11_1_msvc_en.asm}}

\IFRU{\IDIV делит 64-битное число хранящееся в паре регистров \TT{EDX:EAX} на значение в \ECX. 
В результате, \EAX будет содержать частное\footnote{результат деления}, а \EDX ~--- остаток от деления. 
Результат возвращается из функции через \EAX, так что после операции деления, 
это значение не перекладывается больше никуда, 
оно уже там где надо.}
{\IDIV divides 64-bit number stored in \TT{EDX:EAX} register pair by value in \ECX register. 
As a result, \EAX will contain quotient\footnote{result of division}, and \EDX ~--- remainder.
Result is returning from f() function in \EAX register, so, that value is not moved anymore after division 
operation, it is in right place already.}
\IFRU
{Из-за того что \IDIV требует пару регистров \TT{EDX:EAX}, то перед этим инструкция \TT{CDQ} 
расширяет \EAX до 64-битного значения учитывая знак, также как это делает \MOVSX.}
{Because \IDIV require value in \TT{EDX:EAX} register pair, \TT{CDQ} instruction (before \IDIV) extending 
\EAX value to 64-bit value taking value sign into account, just as \MOVSX does.}
\IFRU{Со включенной оптимизацией (\Ox) получается:}
{If we turn optimization on (\Ox), we got:}

\lstinputlisting{division_by_9/11_1_msvc_Ox.asm}

\newcommand{\URLMSDN}{\href{http://blogs.msdn.com/b/devdev/archive/2005/12/12/502980.aspx}
{MSDN: Integer division by constants}}
\newcommand{\URLN}{http://www.nynaeve.net/?p=115}

\IFRU{Это ~--- деление через умножение. Умножение конечно быстрее работает. 
Поэтому можно используя этот трюк
\footnote{Читайте подробнее о делении через умножение в книге 
``Генри Уоррен, мл. ~--- Алгоритмические трюки для программистов'' 
(глава 10 ~--- ``Целое деление на константы''): \URLMSDN, \url{\URLN}} 
создать код эквивалентный тому что мы хотим и работающий быстрее.}
{This is ~--- division using multiplication. Multiplication operation working much faster. 
And it is possible to use that trick
\footnote{Read more about division by multiplication in 
``Henry S. Warren Jr. ~--- Hacker's Delight'' book (chapter 10 ~--- ``Integer Division By Constants'') 
and: \URLMSDN, \url{\URLN}} 
to produce a code which is equivalent and faster.}
\IFRU
{GCC 4.4.1 даже без включенной оптимизации генерит примерно такой же код как и MSVC с оптимизацией:}
{GCC 4.4.1 even without optimization turned on, generate almost the same code as MSVC with optimization turned on:}

\lstinputlisting{division_by_9/11_2_gcc.asm}

