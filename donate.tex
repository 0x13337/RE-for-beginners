\subsection*{\IFRU{Пожертвования}{Donate}}
\label{sec:donate}

\IFRU{Как выясняется, быть (техническим) писателем требует много сил и работы}
{As it turns out, (technical) writing takes a lot of effort and work}.

\IFRU{Эта книга является свободной, находится в свободном доступе, и доступна в виде исходных кодов}
{This book is free, available freely and available in source code form}
\footnote{\url{https://github.com/dennis714/RE-for-beginners}} (LaTeX), 
\IFRU{и всегда будет оставаться таковой}{and it will be so forever}.

\IFRU{В мои текущие планы насчет этой книги входит добавление информации на эти темы}
{My current plan for this book is to add lots of information about}:
PLANS\footnote{\url{https://github.com/dennis714/RE-for-beginners/blob/master/PLANS}}.

\IFRU{Если вы хотите, чтобы я продолжал свою работу и писал на эти темы,
вы можете рассмотреть идею пожертвования}
{If you want me to continue writing on all these topics you may consider donating}.

\IFRU{Я писал эту книгу более года}{I worked more than year on this book}
\footnote{\IFRU{Самый первый коммит в git от марта 2013}{Initial git commit from March 2013}: \\
\url{https://github.com/dennis714/RE-for-beginners/tree/1e57ef540d827c7f7a92fcb3a4626af3e13c7ee4}},
\IFRU{здесь более 500 страниц}{there are more than 500 pages}.
\IFRU{Здесь $\approx 300$ \TeX-файлов, $\approx 90$ исходников на \CCpp, $\approx 350$ различных листингов}
{There are $\approx 300$ \TeX-files, $\approx 90$ \CCpp source codes, $\approx 350$ various listings}.

\IFRU{Цены на другие книги по этой же тематике на amazon.com колеблются в пределах от \$20 до \$50}
{Price of other books on the same subject varies between \$20 and \$50 on amazon.com}.

\IFRU{Со способами пожертвовать деньги можно ознакомиться на странице}
{Ways to donate are available on the page:} \url{http://yurichev.com/donate.html}

\IFRU{Имена всех жертвователей будут перечислены в книге}
{Every donor's name will be included in the book}!
\IFRU{Жертвователи также имеют право просить меня дописывать в книгу что-то раньше, чем остальное}
{Donors also have a right to ask me to rearrange items in my writing plan}.

\IFRU{Почему не попробовать издаться}{Why not try to publish}?
\IFRU{Потому что это техническая литература, которая, как мне кажется,
не может быть закончена или быть замороженной в бумажном виде}
{Because it's technical literature which, as I believe, cannot be finished or frozen in paper state}.
\IFRU{Такие технические справочники чем-то похожи на Wikipedia или библиотеку \ac{MSDN},
они могут развиваться бесконечно долго}
{Such technical references akin to Wikipedia or \ac{MSDN} library.
They can evolve and grow indefinitely}.
\IFRU{Кто-то может сесть и, не отрываясь, написать всё от начала до конца, опубликовать это и забыть}
{Someone can sit down and write everything from the begin to the end, publish it and forget about it}.
\IFRU{Как выясняется, это не я}{As it turns out, it's not me}.
\IFRU{Каждый день меня посещают мысли вроде ``это было написано плохо, можно было бы и лучше написать'',
``это плохой пример, я знаю получше'',
``ещё одна вещь, которую я могу объяснить лучше и короче'' и т.д}
{I have everyday thoughts like ``that was written badly and can be rewritten better'', 
``that was a bad example, I know a better one'', 
``that is also a thing I can explain better and shorter'', etc}.
\IFRU{Как можно увидеть в истории коммитов исходников этой книги,
я делаю много мелких изменений почти каждый день}
{As you may see in commit history of this book's source code,
I make a lot of small changes almost every day}:
\url{https://github.com/dennis714/RE-for-beginners/commits/master}.

\IFRU{Так что книга, наверное, будет в виде ``rolling release'', как говорят о дистрибутивах Linux вроде
Gentoo}
{So the book will probably be a ``rolling release'' as they say about Linux distros like Gentoo}.
\IFRU{Без релизов (и дедлайнов) вообще, а постепенная разработка}
{No fixed releases (and deadlines) at all, but continuous development}.
\IFRU{Я не знаю, сколько займет времени написать всё что я знаю. Может быть, 10 лет или больше}
{I don't know how long it will take to write all I know. Maybe 10 years or more}.
\IFRU{Конечно, это не очень удобно для читателей, желающих стабильности,
но всё что я могу им предложить ~--- это файл ChangeLog}
{Of course, it is not very convenient for readers who want something stable,
but all I can offer is a ChangeLog}
\footnote{\url{https://github.com/dennis714/RE-for-beginners/blob/master/ChangeLog}}
\IFRU{, служащий как секция ``что нового''}{ file serving as a ``what's new'' section}.
\IFRU{Те, кому интересно, могут проверять его время от времени, или мой блог/twitter
\footnote{
\url{http://blog.yurichev.com/}
\url{https://twitter.com/yurichev\_ru}
}}
{Those who are interested may check it from time to time, or my blog/twitter
\footnote{
\url{http://blog.yurichev.com/}
\url{https://twitter.com/yurichev}
}}
.

\subsubsection*{\IFRU{Жертвователи}{Donors}}

9 * \IFRU{аноним}{anonymous}, 2 * \IFRU{Олег Выговский}{Oleg Vygovsky}, Daniel Bilar, James Truscott,
Luis Rocha, Joris van de Vis, Richard S Shultz, Jang Minchang, Shade Atlas, Yao Xiao,
Pawel Szczur, Justin Simms, Shawn the R0ck, Ki Chan Ahn, Triop AB, Ange Albertini,
\IFRU{Сергей Лукьянов}{Sergey Lukianov}, Ludvig Gislason, Gérard Labadie.
