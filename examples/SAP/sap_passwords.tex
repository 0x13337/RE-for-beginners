\section{\RU{Функции проверки пароля в SAP 6.0}\EN{SAP 6.0 password checking functions}}

\index{SAP}
\RU{Когда я в очередной раз вернулся к своему SAP 6.0 IDES заинсталлированному в виртуальной машине VMware, я обнаружил что забыл пароль, впрочем, затем я вспомнил его, но теперь я получаю такую ошибку:}\EN{While returning again to my SAP 6.0 IDES installed in VMware box, I figured out I forgot the password for SAP* account, then it back to my memory, but now I got error message} 
\IT{<<Password logon no longer possible - too many failed attempts>>}, 
\RU{потому что я потратил все попытки на то, чтобы вспомнить его}
\EN{since I've spent all these attempts in trying to recall it}.

\index{Windows!PDB}
\RU{Первая очень хорошая новость состоит в том, что с SAP поставляется полный \gls{PDB}-файл \IT{disp+work.pdb}, он содержит все: имена функций, структуры, типы, локальные переменные, имена аргументов, и т.д. Какой щедрый подарок!}
\EN{First extremely good news is the full \IT{disp+work.pdb} \gls{PDB}-file is supplied with SAP, it contain almost everything: function names, structures, types, local variable and argument names, etc. What a lavish gift!}

\RU{Я нашел утилиту}\EN{I got} TYPEINFODUMP\footnote{\url{http://go.yurichev.com/17038}} \RU{для дампа содержимого \gls{PDB}-файлов во что-то более читаемое и grep-абельное}\EN{utility for converting \gls{PDB} files into something readable and grepable}.

\RU{Вот пример её работы: информация о функции + её аргументах + её локальных переменных:}\EN{Here is an example of function information + its arguments + its local variables:}

\begin{lstlisting}
FUNCTION ThVmcSysEvent 
  Address:         10143190  Size:      675 bytes  Index:    60483  TypeIndex:    60484 
  Type: int NEAR_C ThVmcSysEvent (unsigned int, unsigned char, unsigned short*)
Flags: 0 
PARAMETER events 
  Address: Reg335+288  Size:        4 bytes  Index:    60488  TypeIndex:    60489 
  Type: unsigned int
Flags: d0 
PARAMETER opcode 
  Address: Reg335+296  Size:        1 bytes  Index:    60490  TypeIndex:    60491 
  Type: unsigned char
Flags: d0 
PARAMETER serverName 
  Address: Reg335+304  Size:        8 bytes  Index:    60492  TypeIndex:    60493 
  Type: unsigned short*
Flags: d0 
STATIC_LOCAL_VAR func 
  Address:         12274af0  Size:        8 bytes  Index:    60495  TypeIndex:    60496 
  Type: wchar_t*
Flags: 80 
LOCAL_VAR admhead 
  Address: Reg335+304  Size:        8 bytes  Index:    60498  TypeIndex:    60499 
  Type: unsigned char*
Flags: 90 
LOCAL_VAR record 
  Address: Reg335+64  Size:      204 bytes  Index:    60501  TypeIndex:    60502 
  Type: AD_RECORD
Flags: 90 
LOCAL_VAR adlen 
  Address: Reg335+296  Size:        4 bytes  Index:    60508  TypeIndex:    60509 
  Type: int
Flags: 90 
\end{lstlisting}

\RU{А вот пример дампа структуры:}\EN{And here is an example of some structure:}

\begin{lstlisting}
STRUCT DBSL_STMTID 
Size: 120  Variables: 4  Functions: 0  Base classes: 0
MEMBER moduletype 
  Type:  DBSL_MODULETYPE
  Offset:        0  Index:        3  TypeIndex:    38653
MEMBER module 
  Type:  wchar_t module[40]
  Offset:        4  Index:        3  TypeIndex:      831
MEMBER stmtnum 
  Type:  long
  Offset:       84  Index:        3  TypeIndex:      440
MEMBER timestamp 
  Type:  wchar_t timestamp[15]
  Offset:       88  Index:        3  TypeIndex:     6612
\end{lstlisting}

\RU{Вау!}\EN{Wow!}

\RU{Вторая хорошая новость: \IT{отладочные} вызовы, коих здесь очень много, очень полезны.}\EN{Another good news is: \IT{debugging} calls (there are plenty of them) are very useful.} 

\RU{Здесь вы можете увидеть глобальную переменную \IT{ct\_level}}\EN{Here you can also notice \IT{ct\_level} global variable}\footnote{\RU{Еще об уровне трассировки}\EN{More about trace level}: \url{http://go.yurichev.com/17039}}, \RU{отражающую уровень трассировки}\EN{reflecting current trace level}.

\RU{В \IT{disp+work.exe} очень много таких отладочных вставок}\EN{There is a lot of such debugging inclusions in the \IT{disp+work.exe} file}:

\begin{lstlisting}
cmp     cs:ct_level, 1
jl      short loc_1400375DA
call    DpLock
lea     rcx, aDpxxtool4_c ; "dpxxtool4.c"
mov     edx, 4Eh        ; line
call    CTrcSaveLocation
mov     r8, cs:func_48
mov     rcx, cs:hdl     ; hdl
lea     rdx, aSDpreadmemvalu ; "%s: DpReadMemValue (%d)"
mov     r9d, ebx
call    DpTrcErr
call    DpUnlock
\end{lstlisting}

\RU{Если текущий уровень трассировки выше или равен заданному в этом коде порогу, отладочное сообщение будет записано в лог-файл вроде \IT{dev\_w0}, \IT{dev\_disp} и прочие файлы \IT{dev*}}\EN{If current trace level is bigger or equal to threshold defined in the code here, debugging message will be written to log files like \IT{dev\_w0}, \IT{dev\_disp}, and other \IT{dev*} files}.

\index{\GrepUsage}
\RU{Попробуем grep-ать файл полученный при помощи утилиты TYPEINFODUMP:}\EN{Let's do grepping on file I got with the help of TYPEINFODUMP utility:}

\begin{lstlisting}
cat "disp+work.pdb.d" | grep FUNCTION | grep -i password
\end{lstlisting}

\RU{Я получил:}\EN{I got:}

\begin{lstlisting}
FUNCTION rcui::AgiPassword::DiagISelection 
FUNCTION ssf_password_encrypt 
FUNCTION ssf_password_decrypt 
FUNCTION password_logon_disabled 
FUNCTION dySignSkipUserPassword 
FUNCTION migrate_password_history 
FUNCTION password_is_initial 
FUNCTION rcui::AgiPassword::IsVisible 
FUNCTION password_distance_ok 
FUNCTION get_password_downwards_compatibility 
FUNCTION dySignUnSkipUserPassword 
FUNCTION rcui::AgiPassword::GetTypeName 
FUNCTION `rcui::AgiPassword::AgiPassword'::`1'::dtor$2 
FUNCTION `rcui::AgiPassword::AgiPassword'::`1'::dtor$0 
FUNCTION `rcui::AgiPassword::AgiPassword'::`1'::dtor$1 
FUNCTION usm_set_password 
FUNCTION rcui::AgiPassword::TraceTo 
FUNCTION days_since_last_password_change 
FUNCTION rsecgrp_generate_random_password 
FUNCTION rcui::AgiPassword::`scalar deleting destructor' 
FUNCTION password_attempt_limit_exceeded 
FUNCTION handle_incorrect_password 
FUNCTION `rcui::AgiPassword::`scalar deleting destructor''::`1'::dtor$1 
FUNCTION calculate_new_password_hash 
FUNCTION shift_password_to_history 
FUNCTION rcui::AgiPassword::GetType 
FUNCTION found_password_in_history 
FUNCTION `rcui::AgiPassword::`scalar deleting destructor''::`1'::dtor$0 
FUNCTION rcui::AgiObj::IsaPassword 
FUNCTION password_idle_check 
FUNCTION SlicHwPasswordForDay 
FUNCTION rcui::AgiPassword::IsaPassword 
FUNCTION rcui::AgiPassword::AgiPassword 
FUNCTION delete_user_password 
FUNCTION usm_set_user_password 
FUNCTION Password_API 
FUNCTION get_password_change_for_SSO 
FUNCTION password_in_USR40 
FUNCTION rsec_agrp_abap_generate_random_password 
\end{lstlisting}

\RU{Попробуем так же искать отладочные сообщения содержащие слова \IT{<<password>>} и \IT{<<locked>>}.}\EN{Let's also try to search for debug messages which contain words \IT{<<password>>} and \IT{<<locked>>}.}
\RU{Одна из таких это строка}\EN{One of them is the string} \IT{<<user was locked by subsequently failed password logon attempts>>} \RU{на которую есть ссылка в}\EN{referenced in} \\
\RU{функции}\EN{function} \IT{password\_attempt\_limit\_exceeded()}.

\RU{Другие строки, которые эта найденная функция может писать в лог-файл это}\EN{Other string this function I found may write to log file are}: \IT{<<password logon attempt will be rejected immediately (preventing dictionary attacks)>>}, \IT{<<failed-logon lock: expired (but not removed due to 'read-only' operation)>>}, \IT{<<failed-logon lock: expired => removed>>}.

\RU{Немного поэкспериментировав с этой функцией, я быстро понял что проблема именно в ней}\EN{After playing for a little with this function, I quickly noticed the problem is exactly in it}. \RU{Она вызывается из функции \IT{chckpass()} ~--- одна из функций проверяющих пароль}\EN{It is called from \IT{chckpass()} function~---one of the password checking functions}.

\RU{В начале, я хочу убедиться, что я на верном пути}\EN{First, I would like to be sure I'm at the correct point}:

\RU{Запускаю свой}\EN{Run my} \tracer:
\index{tracer}

\begin{lstlisting}
tracer64.exe -a:disp+work.exe bpf=disp+work.exe!chckpass,args:3,unicode
\end{lstlisting}

\begin{lstlisting}
PID=2236|TID=2248|(0) disp+work.exe!chckpass (0x202c770, L"Brewered1                               ", 0x41) (called from 0x1402f1060 (disp+work.exe!usrexist+0x3c0))
PID=2236|TID=2248|(0) disp+work.exe!chckpass -> 0x35
\end{lstlisting}

\RU{Функции вызываются так}\EN{Call path is}: \IT{syssigni()} -> \IT{DyISigni()} -> \IT{dychkusr()} -> \IT{usrexist()} -> \IT{chckpass()}.

\RU{Число}\EN{Number} 0x35 \RU{возвращается из}\EN{is an error returning in} \IT{chckpass()} \RU{в этом месте}\EN{at that point}:

\begin{lstlisting}
.text:00000001402ED567 loc_1402ED567:                          ; CODE XREF: chckpass+B4
.text:00000001402ED567                 mov     rcx, rbx        ; usr02
.text:00000001402ED56A                 call    password_idle_check
.text:00000001402ED56F                 cmp     eax, 33h
.text:00000001402ED572                 jz      loc_1402EDB4E
.text:00000001402ED578                 cmp     eax, 36h
.text:00000001402ED57B                 jz      loc_1402EDB3D
.text:00000001402ED581                 xor     edx, edx        ; usr02_readonly
.text:00000001402ED583                 mov     rcx, rbx        ; usr02
.text:00000001402ED586                 call    password_attempt_limit_exceeded
.text:00000001402ED58B                 test    al, al
.text:00000001402ED58D                 jz      short loc_1402ED5A0
.text:00000001402ED58F                 mov     eax, 35h
.text:00000001402ED594                 add     rsp, 60h
.text:00000001402ED598                 pop     r14
.text:00000001402ED59A                 pop     r12
.text:00000001402ED59C                 pop     rdi
.text:00000001402ED59D                 pop     rsi
.text:00000001402ED59E                 pop     rbx
.text:00000001402ED59F                 retn
\end{lstlisting}

\RU{Отлично, давайте проверим}\EN{Fine, let's check}:

\begin{lstlisting}
tracer64.exe -a:disp+work.exe bpf=disp+work.exe!password_attempt_limit_exceeded,args:4,unicode,rt:0
\end{lstlisting}

\begin{lstlisting}
PID=2744|TID=360|(0) disp+work.exe!password_attempt_limit_exceeded (0x202c770, 0, 0x257758, 0) (called from 0x1402ed58b (disp+work.exe!chckpass+0xeb))
PID=2744|TID=360|(0) disp+work.exe!password_attempt_limit_exceeded -> 1
PID=2744|TID=360|We modify return value (EAX/RAX) of this function to 0
PID=2744|TID=360|(0) disp+work.exe!password_attempt_limit_exceeded (0x202c770, 0, 0, 0) (called from 0x1402e9794 (disp+work.exe!chngpass+0xe4))
PID=2744|TID=360|(0) disp+work.exe!password_attempt_limit_exceeded -> 1
PID=2744|TID=360|We modify return value (EAX/RAX) of this function to 0
\end{lstlisting}

\RU{Великолепно! Теперь я могу успешно залогиниться.}\EN{Excellent! I can successfully login now.}

\RU{Кстати, я могу сделать вид что вообще забыл пароль, заставляя \IT{chckpass()} всегда возвращать ноль, и этого достаточно для отключения проверки пароля:}
\EN{By the way, if I try to pretend I forgot the password, fixing \IT{chckpass()} function return value at $0$ is enough to bypass check:}

\begin{lstlisting}
tracer64.exe -a:disp+work.exe bpf=disp+work.exe!chckpass,args:3,unicode,rt:0
\end{lstlisting}

\begin{lstlisting}
PID=2744|TID=360|(0) disp+work.exe!chckpass (0x202c770, L"bogus                                   ", 0x41) (called from 0x1402f1060 (disp+work.exe!usrexist+0x3c0))
PID=2744|TID=360|(0) disp+work.exe!chckpass -> 0x35
PID=2744|TID=360|We modify return value (EAX/RAX) of this function to 0
\end{lstlisting}

\RU{Что еще можно сказать, бегло анализируя функцию \IT{password\_attempt\_limit\_exceeded()}, это то, что в начале
можно увидеть следующий вызов:}\EN{What also can be said while analyzing 
\IT{password\_attempt\_limit\_exceeded()} 
function is that at the very beginning of it, this call might be seen:}

\begin{lstlisting}
lea     rcx, aLoginFailed_us ; "login/failed_user_auto_unlock"
call    sapgparam
test    rax, rax
jz      short loc_1402E19DE
movzx   eax, word ptr [rax]
cmp     ax, 'N'
jz      short loc_1402E19D4
cmp     ax, 'n'
jz      short loc_1402E19D4
cmp     ax, '0'
jnz     short loc_1402E19DE
\end{lstlisting}

\RU{Очевидно, функция \IT{sapgparam()} используется чтобы узнать значение какой-либо переменной конфигурации. Эта функция может вызываться из 1768 разных мест.}
\EN{Obviously, function \IT{sapgparam()} used to query value of some configuration parameter. This function can be called from 1768 different places.}
\RU{Вероятно, при помощи этой информации, мы можем легко находить те места кода, на которые влияют определенные переменные конфигурации.}\EN{It seems, with the help of this information, we can easily find places in code, control flow of which can be affected by specific configuration parameters.}

\RU{Замечательно! Имена функций очень понятны, куда понятнее чем в \oracle.}
\EN{It is really sweet. Function names are very clear, much clearer than in the \oracle.} 
\index{\Cpp}
\RU{По всей видимости, процесс \IT{disp+work} весь написан на \Cpp. Вероятно, он был переписан не так давно?}
\EN{It seems, \IT{disp+work} process written in \Cpp. It was apparently rewritten some time ago?}

