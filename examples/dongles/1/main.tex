\section{\RU{Пример}\EN{Example} \#1: MacOS Classic \AndENRU PowerPC}

\index{PowerPC}
\index{Mac OS Classic}
\RU{Вот пример программы для}\EN{Here is an example of a program for} MacOS Classic
\footnote{\RU{MacOS перед тем как перейти на UNIX}\EN{pre-UNIX MacOS}}, \RU{для}\EN{for} PowerPC.
\RU{Компания, разработавшая этот продукт, давно исчезла, так что (легальный) пользователь
боялся того что донгла может сломаться}\EN{The company who developed the software product
has disappeared a long time ago, so the (legal) customer was afraid of physical dongle damage}.

\RU{Если запустить программу без подключенной донглы, можно увидеть окно с надписью}
\EN{While running without a dongle connected, a message box with the text}
"Invalid Security Device"\EN{ appeared}.
\RU{Мне повезло потому что этот текст можно было легко найти внутри исполняемого файла}
\EN{Luckily, this text string could easily be found in the executable binary file}.

\RU{Представим, что мы не знакомы ни с Mac OS Classic, ни с PowerPC, но всё-таки попробуем}%
\EN{Let's pretend we are not very familiar both with Mac OS Classic and PowerPC, but will try anyway}.

\ac{IDA} \RU{открывает исполняемый файл легко, показывая его тип как}
\EN{opened the executable file smoothly, reported its type as} 
"PEF (Mac OS or Be OS executable)" (\RU{действительно, это стандартный тип файлов в Mac OS Classic}
\EN{indeed, it is a standard Mac OS Classic file format}).

\RU{В поисках текстовой строки с сообщение об ошибке, мы попадаем на этот фрагмент кода}%
\EN{By searching for the text string with the error message, we've got into this code fragment}:

\lstinputlisting{examples/dongles/1/1.lst}

\index{ARM}
\index{MIPS}
\RU{Да, это код PowerPC}\EN{Yes, this is PowerPC code}.
\RU{Это очень типичный процессор для \ac{RISC} 1990-х}
\EN{The CPU is a very typical 32-bit \ac{RISC} of 1990s era}.
\RU{Каждая инструкция занимает 4 байта (как и в MIPS и ARM) и их имена немного похожи на имена 
инструкций MIPS}
\EN{Each instruction occupies 4 bytes (just as in MIPS and ARM) and the names somewhat resemble
MIPS instruction names}.

\TT{check1()} \RU{это имя которое мы дадим этой функции немного позже}\EN{is a function name we'll give to it later}.
\TT{BL} \RU{это инструкция}\EN{is} \IT{Branch Link} 
\RU{т.е. предназначенная для вызова подпрограмм}\EN{instruction, e.g., intended for calling subroutines}.
\RU{Самое важное место\EMDASH{}это инструкция \ac{BNE}, срабатывающая, если проверка наличия донглы прошла
успешно, либо не срабатывающая в случае ошибки: и тогда адрес текстовой строки с сообщением об ошибке
будет загружен в регистр r3 для последующей передачи в функцию отображения диалогового окна}
\EN{The crucial point is the \ac{BNE} instruction which jumps if the dongle protection check passes 
or not if an error occurs: 
then the address of the text string gets loaded into the r3 register for the subsequent passing into a message box routine}.

\RU{Из}\EN{From the} \cite{PPCABI} \RU{мы узнаем, что регистр r3 используется для возврата
значений (и еще r4 если значение 64-битное)}%
\EN{we will found out that the r3 register is used for return values (and r4, in case of 64-bit values)}.

\index{x86!\Instructions!MOVZX}
\RU{Еще одна, пока что неизвестная инструкция}\EN{Another yet unknown instruction is} \TT{CLRLWI}. 
\RU{Из}\EN{From} \cite{PPC} \RU{мы узнаем, что эта инструкция одновременно и очищает и загружает}%
\EN{we'll learn that this instruction does both clearing and loading}. 
\RU{В нашем случае, она очищает 24 старших бита из значения в r3 и записывает всё это в r0, 
так что это аналог}\EN{In our case, it clears the 24 high bits from the value in r3
and puts them in r0, so it is analogical to} \MOVZX \InENRU x86 (\myref{movzx}),
\RU{но также устанавливает флаги, так что}\EN{but it also sets the flags, so} \ac{BNE} 
\RU{может проверить их потом}\EN{can check them afterwards}.

\RU{Посмотрим внутрь}\EN{Let's take a look into the} \TT{check1()}\EN{ function}:

\lstinputlisting{examples/dongles/1/check1.lst}

\RU{Как можно увидеть в \ac{IDA}, эта функция вызывается из многих мест в программе, но только значение
в регистре r3 проверяется сразу после каждого вызова}
\EN{As you can see in \ac{IDA}, that function is called from many places in the program, but only the r3 register's value
is checked after each call}.
\index{thunk-\RU{функции}\EN{functions}}
\RU{Всё что эта функция делает это только вызывает другую функцию, так что это}
\EN{All this function does is to call the other function, so it is a} \gls{thunk function}: 
\RU{здесь присутствует и пролог функции и эпилог, но регистр r3 не трогается, так что}
\EN{there are function prologue and epilogue, but the r3 register is not touched, so} \TT{checkl()} 
\RU{возвращает то, что возвращает}\EN{returns what} \TT{check2()}\EN{ returns}.

\ac{BLR} \RU{это похоже возврат из функции, но так как IDA делает всю разметку функций автоматически,
наверное, мы можем пока не интересоваться этим}\EN{looks like the return from the function, but since \ac{IDA} does the function layout, we probably do not need
to care about this}.
\RU{Так как это типичный \ac{RISC}, похоже, подпрограммы вызываются, используя}
\EN{Since it is a typical \ac{RISC}, it seems that subroutines are called using a} \gls{link register},
\RU{точно как в}\EN{just like in} ARM.

\EN{The}\RU{Функция} \TT{check2()} \RU{более сложная}\EN{function is more complex}:

\lstinputlisting{examples/dongles/1/check2.lst}

\index{USB}
\RU{Снова повезло: имена некоторых функций оставлены в исполняемом файле
(в символах в отладочной секции? Трудно сказать до тех пор, пока мы не знакомы с этим форматом файлов,
может быть это что-то вроде PE-экспортов (\myref{PE_exports_imports}))?
как например \TT{.RBEFINDNEXT()} and \TT{.RBEFINDFIRST()}.}
\EN{We are lucky again: some function names are left in the executable 
(debug symbols section? Hard to say while we are not very familiar with the file format, maybe it is
some kind of PE exports? (\myref{PE_exports_imports})),
like \TT{.RBEFINDNEXT()} and \TT{.RBEFINDFIRST()}.}
\RU{В итоге, эти функции вызывают другие функции с именами вроде}
\EN{Eventually these functions call other functions with names like} \TT{.GetNextDeviceViaUSB()}, 
\TT{.USBSendPKT()},
\RU{так что они явно работают с каким-то USB-устройством}\EN{so these are clearly dealing with an USB device}.

\RU{Тут даже есть функция с названием}\EN{There is even a function named} 
\TT{.GetNextEve3Device()}\EMDASH\RU{звучит знакомо, в 1990-х годах была донгла}\EN{sounds familiar, there was a} Sentinel Eve3 
\RU{для ADB-порта (присутствующих на Макинтошах)}\EN{dongle for ADB port (present on Macs) in 1990s}.

\RU{В начале посмотрим на то как устанавливается регистр r3 одновременно игнорируя всё остальное}
\EN{Let's first take a look on how the r3 register is set before return, while ignoring everything else}.
\RU{Мы знаем, что \q{хорошее} значение в r3 должно быть не нулевым, а нулевой r3 приведет
к выводу диалогового окна с сообщением об ошибке.}
\EN{We know that a \q{good} r3 value has to be non-zero, zero r3 leads the execution
flow to the message box with an error message.}

\RU{В функции имеются две инструкции}\EN{There are two} \TT{li \%r3, 1} 
\RU{и одна}\EN{instructions present in the function and one} \TT{li \%r3, 0} 
(\IT{Load Immediate}, \RU{т.е. загрузить значение в регистр}\EN{i.e., loading a value into a register}).
\RU{Самая первая инструкция находится на}\EN{The first instruction is at} 
\TT{0x001186B0}\EMDASH\RU{и честно говоря, трудно заранее понять, что это означает}%
\EN{and frankly speaking, it's hard to say what it means}.

\RU{А вот то что мы видим дальше понять проще}\EN{What we see next is, however, easier to understand}: 
\RU{вызывается }\TT{.RBEFINDFIRST()} \RU{и в случае ошибки, 0 будет записан в r3
и мы перейдем на \IT{exit}, а иначе будет вызвана функция \TT{check3()}\EMDASH{}если и она будет
выполнена с ошибкой, будет вызвана}\EN{is called:
if it fails, 0 is written into r3 and we jump to \IT{exit}, otherwise another
function is called (\TT{check3()})\EMDASH{}if it fails too, }
\TT{.RBEFINDNEXT()} \RU{вероятно, для поиска другого USB-устройства}
\EN{is called, probably in order to look for another USB device}.

N.B.: \TT{clrlwi. \%r0, \%r3, 16} \RU{это аналог того что мы уже видели, но она очищает 16 старших бит,
т.е.}\EN{it is analogical to what we already saw, but it clears
16 bits, i.e.}, \TT{.RBEFINDFIRST()} \RU{вероятно возвращает 16-битное значение}
\EN{probably returns a 16-bit value}.

\TT{B} (\RU{означает}\EN{stands for} \IT{branch}) \RU{\EMDASH{}безусловный переход}\EN{unconditional jump}.

\ac{BEQ} \RU{это обратная инструкция от}\EN{is the inverse instruction of} \ac{BNE}.

\RU{Посмотрим на}\EN{Let's see} \TT{check3()}:

\lstinputlisting{examples/dongles/1/check3.lst}

\RU{Здесь много вызовов}\EN{There are a lot of calls to} \TT{.RBEREAD()}. 
\RU{Эта функция вероятно читает какие-то значения из донглы, которые потом сравниваются здесь при помощи}
\EN{The function probably returns some values from the dongle,
so they are compared here with some hard-coded variables using} \TT{CMPLWI}.

\RU{Мы также видим в регистр r3 записывается перед каждым вызовом}
\EN{We also see that the r3 register is also filled before each call to} \TT{.RBEREAD()} 
\RU{одно из этих значений}\EN{with one of these values}: 0, 1, 8, 0xA, 0xB, 0xC, 0xD, 4, 5.
\RU{Вероятно адрес в памяти или что-то в этом роде}\EN{Probably a memory address or something like that}?

\RU{Да, действительно, если погуглить имена этих функций, можно легко найти документацию к}
\EN{Yes, indeed, by googling these function names it is easy to find the} Sentinel Eve3\EN{ dongle manual}!

\RU{Hаверное, уже и не нужно изучать остальные инструкции PowerPC: всё что делает эта функция это просто
вызывает}
\EN{Perhaps, we don't even need to learn any other PowerPC instructions: all this function does is just
call} \TT{.RBEREAD()}, \RU{сравнивает его результаты с константами и возвращает 1 если результат сравнения положительный или 0 в другом случае}\EN{compare its results with the constants and returns 1 if the comparisons
are fine or 0 otherwise}.

\RU{Всё ясно: \TT{check1()} должна всегда возвращать 1 или иное ненулевое значение}
\EN{OK, all we've got is that \TT{check1()} has always to return 1 or any other non-zero value}.
\RU{Но так как я не очень уверен в своих знаниях инструкций PowerPC, я буду осторожен и пропатчу
переходы в \TT{check2} на адресах}\EN{But since I'm not very confident in my knowledge of PowerPC instructions,
I'm going to be careful: I'm going to patch
the jumps in \TT{check2()} at} \TT{0x001186FC} \AndENRU \TT{0x00118718}.

\RU{На}\EN{At} \TT{0x001186FC} \RU{мы записываем байты}\EN{we'll write bytes} 0x48 \AndENRU 0 
\RU{таким образом превращая инструкцию}\EN{thus converting the} \ac{BEQ} 
\RU{в инструкцию}\EN{instruction in an} 
\TT{B} (\RU{безусловный переход}\EN{unconditional jump}):
\RU{Мы можем заметить этот опкод прямо в коде даже без обращения к}%
\EN{We can spot its opcode in the code without even referring to} \cite{PPC}.

\RU{На}\EN{At} \TT{0x00118718} \RU{мы записываем байт}\EN{we'll write} 0x60 \AndENRU \RU{еще 3 нулевых байта,
таким образом превращая её в инструкцию}\EN{3 zero bytes, thus converting it to a}
\ac{NOP}\EN{ instruction}:
\RU{Этот опкод мы также можем подсмотреть прямо в коде}\EN{Its opcode we could spot in the code too}.

\RU{И всё заработало без подключенной донглы}\EN{And now it all works without a dongle connected}.

\RU{Резюмируя, такие простые модификации можно делать в \ac{IDA} даже с минимальными знаниями
ассемблера}\EN{In summary, such small modifications can be done with \ac{IDA} and minimal assembly language knowledge}.

