\section{\IFRU{Пример}{Example} \#1: MacOS Classic \AndENRU PowerPC}

\index{PowerPC}
\index{Mac OS Classic}
\IFRU{Как-то я получил программу для}{I've got a program for} MacOS Classic
\footnote{\IFRU{MacOS перед тем как перейти на UNIX}{pre-UNIX MacOS}}, \IFRU{для}{for} PowerPC.
\IFRU{Компания, разработавшая этот продукт давно исчезла, так что (легальный) пользователь
боялся того что донгла может сломаться}{The company who developed the software product
was disappeared long time ago, so the (legal) customer was afraid of physical dongle damage}.

\IFRU{Если запустить программу без подключенной донглы, можно увидеть окно с надписью}
{While running without dongle connected, a message box with a text}
"Invalid Security Device"\EN{ appeared}.
\IFRU{Мне повезло потому что этот текст можно было легко найти внутри исполняемого файла}
{Luckily, this text string can be found easily in the executable binary file}.

\IFRU{Я не был знаком ни с Mac OS Classic, ни с PowerPC, но решил попробовать}
{I was not very familiar both with Mac OS Classic and PowerPC, but I tried anyway}.

\ac{IDA} \IFRU{открывает исполняемый файл легко, показывая его тип как}
{opens the executable file smoothly, reported its type as} 
"PEF (Mac OS or Be OS executable)" (\IFRU{действительно, это стандартный тип файлов в Mac OS Classic}
{indeed, it is a standard Mac OS Classic file format}).

\IFRU{В поисках текстовой строки с сообщение об ошибке, я попал на этот фрагмент кода}
{By searching for the text string with error message, I've got into this code fragment}:

\lstinputlisting{examples/dongles/1/1.lst}

\index{ARM}
\index{MIPS}
\IFRU{Да, это код PowerPC}{Yes, this is PowerPC code}.
\IFRU{Это очень типичный процессор для \ac{RISC} 1990-х}
{The CPU is very typical 32-bit \ac{RISC} of 1990s era}.
\IFRU{Каждая инструкция занимает 4 байта (как и в MIPS и ARM) и их имена немного похожи на имена 
инструкций MIPS}
{Each instruction occupies 4 bytes (just as in MIPS and ARM) and its names are somewhat resembling
MIPS instruction names}.

\TT{check1()} \IFRU{это имя которое я дал этой ф-ции позже}{is a function name I gave it to lately}.
\TT{BL} \IFRU{это инструкция}{is} \IT{Branch Link} 
\IFRU{т.е., предназначенная для вызова подпрограмм}{instruction, e.g., intended for subroutines calling}.
\IFRU{Самое важное место ~--- это инструкция \ac{BNE} срабатывающая если проверка наличия донглы прошла
успешно, либо не срабатывающая в случае ошибки: и тогда адрес текстовой строки с сообщением об ошибке
будет загружен в регистр r3 для последующей передачи в функцию отображения диалогового окна}
{The crucial point is \ac{BNE} instruction jumping if dongle protection check is passed 
or not jumping if error is occurred: 
then the address of the text string being loaded into r3 register for the subsequent passage into message box routine}.

\IFRU{Из}{From the} \cite{PPCABI} \IFRU{я узнал, что регистр r3 используется для возврата
значений (и еще r4 если значение 64-битное)}
{I've got to know the r3 register is used for values returning (and r4, in case of 64-bit values)}.

\index{x86!\Instructions!MOVZX}
\IFRU{Еще одна пока что неизвестная инструкция}{Another yet unknown instruction is} \TT{CLRLWI}. 
\IFRU{Из}{From} \cite{PPC} \IFRU{я узнал, что эта инструкция одновременно и очищает и загружает}{I've got to know that this instruction
do both clearing and loading}. 
\IFRU{В нашем случае, она очищает 24 старших бита из значения в r3 и записывает всё это в r0, 
так что это аналог}{In our case, it clears 24 high bits from the value in r3
and put it to r0, so it is analogical to} \MOVZX \InENRU x86 (\ref{movzx}),
\IFRU{но также устанавливает флаги, так что}{but it also sets the flags, so the} \ac{BNE} 
\IFRU{может проверить их потом}{can check them after}.

\IFRU{Посмотрим внутрь}{Let's take a look into} \TT{check1()}\EN{ function}:

\lstinputlisting{examples/dongles/1/check1.lst}

\IFRU{Как можно увидеть в \ac{IDA}, эта ф-ция вызывается из многих мест в программе, но только значение
в регистре r3 проверяется сразу после каждого вызова}
{As I can see in \ac{IDA}, that function is called from many places in program, but only r3 register value
is checked right after each call}.
\index{thunk-\IFRU{функции}{functions}}
\IFRU{Всё что эта ф-ция делает это только вызывает другую ф-цию, так что это}
{All this function does is calling other function, so it is} \gls{thunk function}: 
\IFRU{здесь присутствует и пролог ф-ции и эпилог, но регистр r3 не трогается, так что}
{there is function prologue and epilogue, but r3 register is not touched, so} \TT{checkl()} 
\IFRU{возвращает то, что возвращает}{returns what} \TT{check2()}\EN{ returns}.

\ac{BLR} \IFRU{это похоже возврат из ф-ции, но так как IDA делает всю разметку ф-ций автоматически,
наверное, мы можем пока не интересоваться этим}{is seems return from function, but since \ac{IDA} does functions layout, we probably do not need
to be interesting in this}.
\IFRU{Так как это типичный \ac{RISC}, похоже, подпрограммы вызываются используя}
{It seems, since it is a typical \ac{RISC}, subroutines are called using} \gls{link register},
\IFRU{точно как в}{just like in} ARM.

\RU{Ф-ция }\TT{check2()} \IFRU{более сложная}{function is more complex}:

\lstinputlisting{examples/dongles/1/check2.lst}

\IFRU{Снова повезло: имена некоторых ф-ций оставлены в исполняемом файле
(в символах в отладочной секции? Я не уверен, т.к. я не знаком с этим форматом файлов,
может быть это что-то вроде PE-экспортов (\ref{PE_exports_imports}))?
как например \TT{.RBEFINDNEXT()} and \TT{.RBEFINDFIRST()}.}
{I'm lucky again: some function names are leaved in the executable 
(debug symbols section? I'm not sure, since I'm not very familiar with the file format, maybe it is
some kind of PE exports? (\ref{PE_exports_imports})),
like \TT{.RBEFINDNEXT()} and \TT{.RBEFINDFIRST()}.}
\IFRU{В итоге, эти ф-ции вызывают другие ф-ции с именами вроде}
{Eventually these functions are calling other functions with names like} \TT{.GetNextDeviceViaUSB()}, 
\TT{.USBSendPKT()},
\IFRU{так что они явно работают с каким-то USB-устройством}{so these are clearly dealing with USB device}.

\IFRU{Тут даже есть ф-ция с названием}{There are even a function named} 
\TT{.GetNextEve3Device()}\EMDASH\IFRU{звучит знакомо, в 1990-х годах была донгла}{sounds familiar, there was} Sentinel Eve3 
\IFRU{для ADB-порта (присутствующих на Макинтошах)}{dongle for ADB port (present on Macs) in 1990s}.

\IFRU{В начале посмотрим на то как устанавливается регистр r3 одновременно игнорируя всё остальное}
{Let's first take a look on how r3 register is set before return simultaneously ignoring all we see}.
\IFRU{Мы знаем, что ``хорошее'' значение в r3 должно быть не нулевым, а нулевой r3 приведет
к выводу диалогового окна с сообщением об ошибке}
{We know that ``good'' r3 value should be non-zero, zero r3 will lead execution
flow to the message box with an error message}.

\IFRU{В ф-ции имеются две инструкции}{There are two instructions} \TT{li \%r3, 1} 
\IFRU{и одна}{present in the function and one} \TT{li \%r3, 0} 
(\IT{Load Immediate}, \IFRU{т.е., загрузить значение в регистр}{i.e., loading value into register}).
\IFRU{Самая первая инструкция находится на}{The very first instruction at} 
\TT{0x001186B0}\EMDASH\IFRU{и честно говоря, я не знаю что это означает, нужно больше времени на изучение ассемблера PowerPC}
{frankly speaking, I don't know
what it mean, I need some more time to learn PowerPC assembly language}.

\IFRU{А вот то что мы видим дальше понять проще}{What we see next is, however, easier to understand}: 
\RU{вызывается }\TT{.RBEFINDFIRST()} \IFRU{и в случае ошибки, 0 будет записан в r3
и мы перейдем на \IT{exit}, а иначе будет вызвана ф-ция \TT{check3()} ~--- если и она будет
выполнена с ошибкой, будет вызвана}{is called:
in case of its failure, 0 is written into r3 and we jump to \IT{exit}, otherwise another
function is called (\TT{check3()})~---if it is failing too, 
the} \TT{.RBEFINDNEXT()} \IFRU{вероятно, для поиска другого USB-устройства}
{is called, probably, in order to look for another USB device}.

N.B.: \TT{clrlwi. \%r0, \%r3, 16} \IFRU{это аналог того что мы уже видели, но она очищает 16 старших бит,
т.е.}{it is analogical to what we already saw, but it clears
16 bits, i.e.}, \TT{.RBEFINDFIRST()} \IFRU{вероятно возвращает 16-битное значение}
{probably returns 16-bit value}.

\TT{B} \IFRU{означает}{meaning} \IT{branch} \IFRU{~--- безусловный переход}{is unconditional jump}.

\ac{BEQ} \IFRU{это обратная инструкция от}{is inverse instruction of} \ac{BNE}.

\IFRU{Посмотрим на}{Let's see} \TT{check3()}:

\lstinputlisting{examples/dongles/1/check3.lst}

\IFRU{Здесь много вызовов}{There are a lot of calls to} \TT{.RBEREAD()}. 
\IFRU{Эта ф-ция вероятно читает какие-то значения из донглы, которые потом сравниваются здесь при помощи}
{The function is probably return some values from the dongle,
so they are compared here with hard-coded variables using} \TT{CMPLWI}.

\IFRU{Мы также видим в регистр r3 записывается перед каждым вызовом}
{We also see that r3 register is also filled before each call to} \TT{.RBEREAD()} 
\IFRU{одно из этих значений}{by one of these values}: 0, 1, 8, 0xA, 0xB, 0xC, 0xD, 4, 5.
\IFRU{Вероятно адрес в памяти или что-то в этом роде}{Probably memory address or something like that}?

\IFRU{Да, действительно, если погуглить имена этих ф-ций, можно легко найти документацию к}
{Yes, indeed, by googling these function names it is easy to find} Sentinel Eve3\EN{ dongle manual}!

\IFRU{Мне даже, наверное, не нужно изучать остальные инструкции PowerPC: всё что делает эта ф-ция это просто
вызывает}
{I probably even do not need to learn other PowerPC instructions: all this function does is just
calls} \TT{.RBEREAD()}, \IFRU{сравнивает его результаты с константами и возвращает 1 если результат сравнения положительный или 0 в другом случае}{compare its results with constants and returns 1 if comparisons
are fine or 0 otherwise}.

\IFRU{Всё ясно: \TT{check1()} должна всегда возвращать 1 или иное ненулевое значение}
{OK, all we've got is that \TT{check1()} should return always 1 or any other non-zero value}.
\IFRU{Но так как я не очень уверен в своих знаниях инструкций PowerPC, я буду осторожен и пропатчу
переходы в \TT{check2} на адресах}{But since I'm not very confident in PowerPC instructions,
I will be careful: I will patch
jumps in \TT{check2()} at} \TT{0x001186FC} \AndENRU \TT{0x00118718}.

\IFRU{На}{At} \TT{0x001186FC} \IFRU{я записал байты}{I wrote bytes} 0x48 \AndENRU 0 
\IFRU{таким образом превращая инструкцию}{thus converting} \ac{BEQ} 
\IFRU{в инструкцию}{instruction into} 
\TT{B} (\IFRU{безусловный переход}{unconditional jump}):
\IFRU{Я заметил этот опкод прямо в коде даже без обращения к}
{I spot its opcode in the code without even referring to} \cite{PPC}.

\IFRU{На}{At} \TT{0x00118718} \IFRU{я записал байт}{I wrote} 0x60 \AndENRU \IFRU{еще 3 нулевых байта,
таким образом превращая её в инструкцию}{3 zero bytes thus converting it to}
\ac{NOP}\EN{ instruction}:
\IFRU{Этот опкод я тоже подсмотрел прямо в коде}{I spot its opcode in the code too}.

\IFRU{Резюмируя, такие простые модификации можно делать в \ac{IDA} даже с минимальными знаниями
ассемблера}{Summarizing, such small modifications can be done with \ac{IDA} and minimal assembly language knowledge}.

