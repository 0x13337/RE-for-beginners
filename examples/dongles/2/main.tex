\section{\RU{Пример}\EN{Example} \#2: SCO OpenServer}

\label{examples_SCO}
\index{SCO OpenServer}
\RU{Древняя программа для}\EN{An ancient software for} SCO OpenServer \RU{от}\EN{from} 1997 
\RU{разработанная давно исчезнувшей компанией}\EN{developed
by a company that disappeared a long time ago}.

\RU{Специальный драйвер донглы инсталлируется в системе, он содержит такие текстовые строки}
\EN{There is a special dongle driver to be installed in the system, that contains the following text strings}:
\q{Copyright 1989, Rainbow Technologies, Inc., Irvine, CA}
\AndENRU
\q{Sentinel Integrated Driver Ver. 3.0 }.

\RU{После инсталляции драйвера, в /dev появляются такие устройства}
\EN{After the installation of the driver in SCO OpenServer, these device files appear in the /dev filesystem}:

\begin{lstlisting}
/dev/rbsl8
/dev/rbsl9
/dev/rbsl10
\end{lstlisting}

\RU{Без подключенной донглы, программа сообщает об ошибке, но сообщение об ошибке не удается
найти в исполняемых файлах}
\EN{The program reports an error without dongle connected, but the error string cannot be found in the executables}.

\index{COFF}
\RU{Еще раз спасибо \ac{IDA}, она легко загружает исполняемые файлы формата COFF использующиеся в}
\EN{Thanks to \ac{IDA}, it is easy to load the COFF executable used in} SCO OpenServer.

\RU{Я попробовал также поискать строку}\EN{I tried to find} \q{rbsl} 
\RU{, и действительно, её можно найти в таком фрагменте кода}
\EN{and indeed, found it in this code fragment}:

\lstinputlisting{examples/dongles/2/1.lst}

\RU{Действительно, должна же как-то программа обмениваться информацией с драйвером}
\EN{Yes, indeed, the program needs to communicate with the driver somehow}.

\index{thunk-\RU{функции}\EN{functions}}
\RU{Единственное место где вызывается функция}\EN{The only place where the} \TT{SSQC()}
\RU{это}\EN{function is called is the} \gls{thunk function}:

\lstinputlisting{examples/dongles/2/2.lst}

\RU{А вот }SSQ() \RU{вызывается по крайней мере из двух разных функций}
\EN{is called from at least 2 functions}.

\RU{Одна из них}\EN{One of these is}:

\lstinputlisting{examples/dongles/2/check1.lst}

\q{\TT{3C}} \AndENRU \q{\TT{3E}} \RU{\EMDASH{}это звучит знакомо: когда-то была донгла}
\EN{sound familiar: there was a} Sentinel Pro \RU{от Rainbow без памяти,
предоставляющая только одну секретную крипто-хеширующую функцию}\EN{dongle by Rainbow with no memory,
providing only one crypto-hashing secret function}.

\RU{О том, что такое хэш-функция, было описано здесь}\EN{You can read a short description
of what hash function is here}: \myref{hash_func}.

\RU{Но вернемся к нашей программе}\EN{But let's get back to the program}.
\RU{Так что программа может только проверить подключена ли донгла или нет}
\EN{So the program can only check the presence or absence of a connected dongle}.
\RU{Никакой больше информации в такую донглу без памяти записать нельзя}
\EN{No other information can be written to such dongle, as it has no memory}.
\RU{Двухсимвольные коды\EMDASH{}это команды}\EN{The two-character codes are commands}
(\RU{можно увидеть, как они обрабатываются в функции}\EN{we can see how the commands are handled in the} 
\TT{SSQC()}\EN{ function}) 
\RU{а все остальные строки хешируются внутри донглы превращаясь в 16-битное число}
\EN{and all other strings are hashed inside the dongle, being transformed into a 16-bit number}.
\RU{Алгоритм был секретный, так что нельзя было написать замену драйверу или сделать
электронную копию донглы идеально эмулирующую алгоритм}\EN{The algorithm was secret,
so it was not possible to write a driver replacement or to remake the dongle hardware that would emulate it perfectly}.
\RU{С другой стороны, всегда можно было перехватить все обращения к ней и найти те константы, с которыми
сравнивается результат хеширования}
\EN{However, it is always possible to intercept all accesses to it and to find what constants
the hash function results are compared to}.
\RU{Но надо сказать, вполне возможно создать устойчивую защиту от копирования базирующуюся
на секретной хеш-функции: пусть она шифрует все файлы с которыми ваша программа работает}
\EN{But we need to say that it is possible to build a robust software copy protection scheme based on secret
cryptographic hash-function: let it encrypt/decrypt the data files your software uses}.

\RU{Но вернемся к нашему коду}\EN{But let's get back to the code}.

\RU{Коды}\EN{Codes} 51/52/53 \RU{используются для выбора номера принтеровского LPT-порта}
\EN{are used for LPT printer port selection}.
3x/4x \RU{используются для выбора}\EN{are used for} \q{family} \RU{так донглы Sentinel Pro
можно отличать друг от друга: ведь более одной донглы может быть подключено к LPT-порту}
\EN{selection (that's how Sentinel Pro dongles are differentiated from each other: more than one
dongle can be connected to a LPT port)}.

\RU{Единственная строка, передающаяся в хеш-функцию это}
\EN{The only non-2-character string passed to the hashing function is} "0123456789".
\RU{Затем результат сравнивается с несколькими правильными значениями}
\EN{Then, the result is compared against the set of valid results}.
\RU{Если результат правилен}\EN{If it is correct},
0xC \OrENRU 0xB \RU{будет записано в глобальную переменную}
\EN{is to be written into the global variable} \TT{ctl\_model}.

\RU{Еще одна строка для хеширования:}\EN{Another text string that gets passed is}
"PRESS ANY KEY TO CONTINUE: ", \RU{но результат не проверяется}\EN{but the result is not checked}.
\RU{Не знаю зачем это, может быть по ошибке}\EN{I don't know why, probably by mistake}
\footnote{\RU{Это очень странное чувство: находить ошибки в столь древнем ПО.}
\EN{What a strange feeling: to find bugs in such ancient software.}}.

\RU{Давайте посмотрим, где проверяется значение глобальной переменной}
\EN{Let's see where the value from the global variable} \TT{ctl\_mode}\EN{ is used}.

\RU{Одно из таких мест}\EN{One such place is}:

\lstinputlisting{examples/dongles/2/4.lst}

\RU{Если оно 0, шифрованное сообщение об ошибке будет передано в функцию дешифрования, и оно будет 
показано}\EN{If it is 0, an encrypted error message is passed to a decryption routine and printed}.

\index{x86!\Instructions!XOR}
\RU{Функция дешифровки сообщений об ошибке похоже применяет простой \gls{xoring}}
\EN{The error string decryption routine seems a simple \gls{xoring}}:

\lstinputlisting{examples/dongles/2/err_warn.lst}

\RU{Вот почему не получилось найти сообщение об ошибке в исполняемых файлах, потому что оно было
зашифровано, это очень популярная практика}
\EN{That's why I was unable to find the error messages in the executable files, because they are encrypted
(which is is popular practice)}.

\RU{Еще один вызов хеширующей функции передает строку}\EN{Another call to the \TT{SSQ()} hashing function passes the}
\q{offln} \RU{и сравнивает результат с константами}\EN{string to it and compares the result with}
\TT{0xFE81} \AndENRU \TT{0x12A9}.
\RU{Если результат не сходится, происходит работа с какой-то функцией}
\EN{If they don't match, it works with some} \TT{timer()} 
\RU{(может быть для ожидания плохо подключенной донглы и нового запроса?), затем дешифрует
еще одно сообщение об ошибке и выводит его}\EN{function (maybe waiting for a poorly
connected dongle to be reconnected and check again?) and then decrypts another error message to dump}.

\lstinputlisting{examples/dongles/2/check2.lst}

\RU{Заставить работать программу без донглы довольно просто: просто пропатчить все места после инструкций
\CMP где происходят соответствующие сравнения}\EN{Bypassing the dongle is pretty straightforward: just patch all jumps after the relevant \CMP instructions}.

\RU{Еще одна возможность\EMDASH{}это написать свой драйвер для SCO OpenServer}
\EN{Another option is to write our own SCO OpenServer driver}.

\subsection{\RU{Дешифровка сообщений об ошибке}\EN{Decrypting error messages}}

\RU{Кстати, мы также можем дешифровать все сообщения об ошибке}\EN{By the way, we can also 
try to decrypt all error messages}.
\RU{Алгоритм, находящийся в функции}\EN{The algorithm that is located in the} \TT{err\_warn()} 
\RU{действительно, крайне прост}\EN{function is very simple, indeed}:

\lstinputlisting[caption=\RU{Функция дешифровки}\EN{Decryption function}]
{examples/dongles/2/decrypting.lst.\LANG}

\RU{Как видно, не только сама строка поступает на вход, но также и ключ для дешифровки}\EN{As we can see, 
not just string is supplied to the decryption function, but also the key}:

\begin{lstlisting}
.text:0000DAF7 error:                                  ; CODE XREF: sync_sys+255j
.text:0000DAF7                                         ; sync_sys+274j ...
.text:0000DAF7                 mov     [ebp+var_8], offset encrypted_error_message2
.text:0000DAFE                 mov     [ebp+var_C], 17h ; decrypting key
.text:0000DB05                 jmp     decrypt_end_print_message

...

; this name I gave to label:
.text:0000D9B6 decrypt_end_print_message:              ; CODE XREF: sync_sys+29Dj
.text:0000D9B6                                         ; sync_sys+2ABj
.text:0000D9B6                 mov     eax, [ebp+var_18]
.text:0000D9B9                 test    eax, eax
.text:0000D9BB                 jnz     short loc_D9FB
.text:0000D9BD                 mov     edx, [ebp+var_C] ; key
.text:0000D9C0                 mov     ecx, [ebp+var_8] ; string
.text:0000D9C3                 push    edx
.text:0000D9C4                 push    20h
.text:0000D9C6                 push    ecx
.text:0000D9C7                 push    18h
.text:0000D9C9                 call    err_warn
\end{lstlisting}

\RU{Алгоритм это очень простой}\EN{The algorithm is a simple} \gls{xoring}: 
\RU{каждый байт XOR-ится с ключом, но ключ увеличивается на 3 после обработки каждого байта}\EN{each 
byte is xored with a key, but the key is increased by 3 after the processing of each byte}.

\RU{Я написал небольшой скрипт на Python для проверки своих догадок}\EN{I wrote a simple Python 
script to check my hypothesis}:

\lstinputlisting[caption=Python 3.x]{examples/dongles/2/decr1.py}

\RU{И он выводит}\EN{And it prints}: \q{check security device connection}.
\RU{Так что да, это дешифрованное сообщение}\EN{So yes, this is the decrypted message}.

\RU{Здесь есть также и другие сообщения, с соответствующими ключами}\EN{There are also 
other encrypted messages with their corresponding keys}.
\RU{Но надо сказать, их можно дешифровать и без ключей}\EN{But needless to say, it is possible 
to decrypt them without their keys}.
\RU{В начале, мы можем заметить, что ключ\EMDASH{}это просто байт}\EN{First, we can see that the key 
is in fact a byte}.
\RU{Это потому что самая важная часть функции дешифровки}\EN{It is because the core decryption instruction}
(\XOR) \RU{оперирует байтами}\EN{works on byte level}. 
\RU{Ключ находится в регистре}\EN{The key is located in the} \ESI\RU{, но только младшие 8 бит
(т.е. байт) регистра используются}\EN{ register, but only one byte part of \ESI is used}.
\RU{Следовательно, ключ может быть больше}\EN{Hence, a key may be greater than} 255, 
\RU{но его значение будет округляться}\EN{but its value is always to be rounded}.

\RU{И как следствие, мы можем попробовать обычный перебор всех ключей в диапазоне}
\EN{As a consequence, we can just try brute-force, trying all possible keys in the} 0..255\EN{ range}.
\RU{Мы также можем пропускать сообщения содержащие непечатные символы}\EN{We are also going to skip 
the messages that contain unprintable characters}.

\lstinputlisting[caption=Python 3.x]{examples/dongles/2/decr2.py}

\RU{И мы получим}\EN{And we get}:

\lstinputlisting[caption=Results]{examples/dongles/2/decr2_result.txt}

\RU{Тут есть какой-то мусор, но мы можем быстро отыскать сообщения на английском языке}\EN{There 
is some garbage, but we can quickly find the English-language messages}!

\RU{Кстати, так как алгоритм использует простой \XOR, та же функция может использоваться и для шифрования
сообщения}\EN{By the way, since the algorithm is a simple xoring encryption, the very same function can be used
to encrypt messages}.
\RU{Если нужно, мы можем зашифровать наши собственные сообщения, и пропатчить программу вставив
их}\EN{If needed, we can encrypt our own messages, and patch the program by inserting them}.

