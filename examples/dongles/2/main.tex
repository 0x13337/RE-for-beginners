\subsection{\IFRU{Пример}{Example} \#2: SCO OpenServer}

\index{SCO OpenServer}
\IFRU{Древняя программа для}{An ancient software for} SCO OpenServer \IFRU{от}{from} 1997 
\IFRU{разработанная давно исчезнувшей компанией}{developed
by a company disappeared long time ago}.

\IFRU{Специальный драйвер донглы инсталлируется в системе, он содержит такие текстовые строки}
{There is a special dongle driver to be installed in the system, containing text strings}:
``Copyright 1989, Rainbow Technologies, Inc., Irvine, CA''
\AndENRU
``Sentinel Integrated Driver Ver. 3.0 ''.

\IFRU{После инсталляции драйвера, в /dev появляются такие устройства}
{After driver installation in SCO OpenServer, these device files are appeared in /dev filesystem}:

\begin{lstlisting}
/dev/rbsl8
/dev/rbsl9
/dev/rbsl10
\end{lstlisting}

\IFRU{Без подключенной донглы, программа сообщает об ошибке, но сообщение об ошибке не удается
найти в исполняемых файлах}
{The program without dongle connected reports error, but the error string cannot be found in the executables}.

\index{COFF}
\IFRU{Еще раз спасибо \ac{IDA}, она легко загружает исполняемые файлы формата COFF использующиеся в}
{Thanks to \ac{IDA}, it does its job perfectly working out COFF executable used in} SCO OpenServer.

\IFRU{Я попробовал также поискать строку}{I've tried to find} ``rbsl'' 
\IFRU{, и действительно, её можно найти в таком фрагменте кода}
{and indeed, found it in this code fragment}:

\lstinputlisting{examples/dongles/2/1.lst}

\IFRU{Действительно, должна же как-то программа обмениваться информацией с драйвером}
{Yes, indeed, the program should comminicate with driver somehow and that is how it is}.

\index{thunk-\IFRU{функции}{functions}}
\IFRU{Единственное место где вызывается ф-ция}{The only place} \TT{SSQC()}
\IFRU{это}{function called is the} \gls{thunk function}:

\lstinputlisting{examples/dongles/2/2.lst}

\IFRU{А вот }{}SSQ() \IFRU{вызывается по крайней мерез из двух разных ф-ций}
{is called at least from 2 functions}.

\IFRU{Одна из них}{One of these is}:

\lstinputlisting{examples/dongles/2/check1.lst}

``\TT{3C}'' \AndENRU ``\TT{3E}'' \IFRU{~--- это звучит знакомо: когда-то была донгла}
{are sounds familiar: there was a} Sentinel Pro \IFRU{от Rainbow без памяти,
предоставляющая только одну секретную крипто-хеширующую ф-цию}{dongle by Rainbow with no memory,
providing only one crypto-hashing secret function}.

\begin{quotation}
\index{\IFRU{Хеш-функции}{Hash functions}}
\index{CRC32}
\IFRU{Но что такое хеш-функция}{But what is hash-function}?
\IFRU{Простейший пример это CRC32, алгоритм ``более мощный'' чем простая контрольная сумма,
для проверки целостности данных}
{Simplest example is CRC32, an algorithm providing ``stronger'' checksum for integrity checking purposes}.
\IFRU{Невозмжоно восстановить оригинальный текст из хеша, там просто меньше информации: ведь текст
может быть очень длинным, но результат CRC32 всегда ограничен 32 битами}
{It is not possible to restore original text from the hash value, it just has much less information:
there can be long text, but CRC32 result is always limited to 32 bits}.
\IFRU{Но CRC32 не надежна в криптографическом смысле: известны методы как изменить текст таким образом,
чтобы получить нужный результат}
{But CRC32 is not cryptographically secure: it is known how to alter a text in that way so the resulting
CRC32 hash value will be one we need}.
\IFRU{Криптографические хеш-функции защищены от этого}
{Cryptographical hash functions are protected from this}.
\index{MD5}
\index{SHA1}
\IFRU{Такие ф-ции как MD5, SHA1, итд, широко используются для хеширования паролей
для хранения их в базе}
{They are widely used to hash user passwords in order to store them in the database, 
like MD5, SHA1, etc}.
\IFRU{Действительно: БД форума в интернете может и не хранить пароли 
(иначе злоумышленник получивший доступ к БД сможет узнать все пароли), а только хеши}
{Indeed: an internet forum database may not contain user passwords (stolen database will compromise
all user's passwords) but only hashes (a cracker will not be able to reveal passwords)}.
\IFRU{К тому же, скрипту интернет-форума вовсе не обязательно знать ваш пароль, он только должен
cверить его хеш с тем что лежит в БД, и дать вам доступ если cверка проходит}
{Besides, an internet forum engine is not aware of your password, it should only check if its hash
is the same as in the database, then it will give you access in this case}.
\IFRU{Один из самых простых способов взлома это просто перебирать все пароли и ждать пока
результат будет такой же как тот что нам нужен}
{One of the simplest passwords cracking methods is just to brute-force all passwords in order to wait
when resulting value will be the same as we need}.
\IFRU{Другие методы намного сложнее}{Other methods are much more complex}. \\
\end{quotation}

\IFRU{Но вернемся к нашей программе}{But let's back to the program}.
\IFRU{Так что программа может только проверить подключена ли донгла или нет}
{So the program can only check the presence or absence dongle connected}.
\IFRU{Никакой больше информации в такую донглу без памяти записать нельзя}
{No other information can be written to such dongle with no memory}.
\IFRU{Двухсимвольные коды это команды}{Two-character codes are commands}
(\IFRU{можно увидеть как они обрабатывюатся в ф-ции}{we can see how commands are handled in} 
\TT{SSQC()}\IFRU{}{ function}) 
\IFRU{а все остальные строки хешируются внутри донглы превращаясь в 16-битное число}
{and all other strings are hashed inside the dongle transforming into 16-bit number}.
\IFRU{Алгоритм был секретный, так что нельзя было написать замену драйверу или сделать
электронную копию донглы идеально эмулирующую алгоритм}{The algorithm was secret,
so it was not possible to write driver replacement or to remake dongle hardware emulating it perfectly}.
\IFRU{С другой стороны, всегда можно было перехватить все обращения к ней и найти те константы, с которыми
сравнивается результат хеширования}
{However, it was always possible to intercept all accesses to it and to find what constants
the hash function results compared to}.
\IFRU{Но надо сказать, вполне возможно создать устойчивую защиту от копирования базирующуюся
на секретной хеш-функции: пусть она шифрует все файлы с которыми ваша программа работает}
{Needless to say it is possible to build a robust software copy protection scheme based on secret
cryptographical hash-function: let it to encrypt/decrypt data files your software dealing with}.

\IFRU{Но вернемся к нашему коду}{But let's back to the code}.

\IFRU{Коды}{Codes} 51/52/53 \IFRU{используются для выбора номера принтеровского LPT-порта}
{are used for LPT printer port selection}.
3x/4x \IFRU{используются для выбора}{is for} ``family'' \IFRU{так донглы Sentinel Pro
можно отличать друг от друга: ведь более одной донглы может быть подключено к LPT-порту}
{selection (that's how Sentinel Pro dongles are differentiated from each other: more than one
dongle can be connected to LPT port)}.

\IFRU{Единственная строка передающаяся в хеш-функцию это}
{The only non-2-character string passed to the hashing function is} "0123456789".
\IFRU{Затем результат сравнивается с несколькими правильными значениями}
{Then, the result is compared against the set of valid results}.
\IFRU{Если результат правилен}{If it is correct},
0xC \OrENRU 0xB \IFRU{будет записано в глобальную переменную}
{is to be written into global variable} \TT{ctl\_model}.

\IFRU{Еще одна строка для хеширования:}{Another text string to be passed is}
"PRESS ANY KEY TO CONTINUE: ", \IFRU{но результат не проверяется}{but the result is not checked}.
\IFRU{Не знаю зачем это, может быть по ошибке}{I don't know why, probably by mistake}.
(\IFRU{Это очень странное чувство: находить ошибки в столь древнем ПО}
{What a strange feeling: to reveal bugs in such ancient software}.)

\IFRU{Давайте посмотрим, где проверяется значение глобальной переменной}
{Let's see where the value from the global variable} \TT{ctl\_mode}\IFRU{}{ is used}.

\IFRU{Одно из таких мест}{One of such places is}:

\lstinputlisting{examples/dongles/2/4.lst}

\IFRU{Если оно 0, шифрованное сообщение об ошибке будет передано в ф-цию дешифрования и оно будет 
показано}{If it is 0, an encrypted error message is passed into decryption routine and printed}.

\index{x86!\Instructions!XOR}
\IFRU{Ф-ция дешифровки сообщений об ошибке похоже применяет простой \ac{XOR}}
{Error strings decryption routine is seems simple xoring}:

\lstinputlisting{examples/dongles/2/err_warn.lst}

% TODO: reverse the function with examples

\IFRU{Вот почему не получилось найти сообщение об ошибке в исполняемых файлах, потому что оно было
зашифровано, это очень популярная практика}
{That's why I was unable to find error messages in the executable files, because they are enrcypted,
this is popular practice}.

\IFRU{Еще один вызов хеширующей ф-ции передает строку}{Another call to \TT{SSQ()} hashing function passes}
``offln'' \IFRU{и сравнивает результат с константами}{string to it and comparing result with}
\TT{0xFE81} \AndENRU \TT{0x12A9}.
\IFRU{Если результат не сходится, происходит работа с какой-то ф-цией}
{If it not so, it deals with some} \TT{timer()} 
\IFRU{(может быть для ожидания плохо подключенной донглы и нового запроса?), затем дешифрует
еще одно сообщение об ошибке и выводит его}{function (maybe waiting for poorly
connected dongle to be reconnected and check again?) and then decrypt another error message to dump}.

\lstinputlisting{examples/dongles/2/check2.lst}

\IFRU{Заставить работать программу без донглы довольно просто: просто пропатчить все места после инструкций
\CMP где происходят соответствующие сравнения}{Dongle bypassing is pretty straightforward: just patch all jumps after \CMP the relevant instructions}.

\IFRU{Еще одна возможность это написать свой драйвер для SCO OpenServer}
{Another option is to write our own SCO OpenServer driver}.

