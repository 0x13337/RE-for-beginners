\subsection{Example \#2: SCO OpenServer}

\label{examples_SCO}
\myindex{SCO OpenServer}
An ancient software for SCO OpenServer from 1997 
developed
by a company that disappeared a long time ago.


There is a special dongle driver to be installed in the system, that contains the following text strings:
\q{Copyright 1989, Rainbow Technologies, Inc., Irvine, CA}
and
\q{Sentinel Integrated Driver Ver. 3.0 }.

After the installation of the driver in SCO OpenServer, these device files appear in the /dev filesystem:

\begin{lstlisting}
/dev/rbsl8
/dev/rbsl9
/dev/rbsl10
\end{lstlisting}

The program reports an error without dongle connected, but the error string cannot be found in the executables.

\myindex{COFF}

Thanks to \ac{IDA}, it is easy to load the COFF executable used in SCO OpenServer.

Let's also try to find \q{rbsl} string and indeed, found it in this code fragment:

\lstinputlisting{examples/dongles/2/1.lst}

Yes, indeed, the program needs to communicate with the driver somehow.

\myindex{thunk-functions}
The only place where the \TT{SSQC()}
function is called is the \gls{thunk function}:

\lstinputlisting{examples/dongles/2/2.lst}

SSQ() can be called from at least 2 functions.

One of these is:

\lstinputlisting{examples/dongles/2/check1.lst}

\q{\TT{3C}} and \q{\TT{3E}} sound familiar: there was a Sentinel Pro dongle by Rainbow with no memory,
providing only one crypto-hashing secret function.

You can read a short description
of what hash function is here: \myref{hash_func}.

But let's get back to the program.

So the program can only check the presence or absence of a connected dongle.

No other information can be written to such dongle, as it has no memory.
The two-character codes are commands
(we can see how the commands are handled in the 
\TT{SSQC()} function) 

and all other strings are hashed inside the dongle, being transformed into a 16-bit number.
The algorithm was secret,
so it was not possible to write a driver replacement or to remake the dongle hardware that would emulate it perfectly.

However, it is always possible to intercept all accesses to it and to find what constants
the hash function results are compared to.

But we need to say that it is possible to build a robust software copy protection scheme based on secret
cryptographic hash-function: let it encrypt/decrypt the data files your software uses.

But let's get back to the code.

Codes 51/52/53 
are used for LPT printer port selection.
3x/4x are used for \q{family} 
selection (that's how Sentinel Pro dongles are differentiated from each other: more than one
dongle can be connected to a LPT port).

The only non-2-character string passed to the hashing function is "0123456789".

Then, the result is compared against the set of valid results.

If it is correct, 0xC or 0xB is to be written into the global variable \TT{ctl\_model}.%

Another text string that gets passed is
"PRESS ANY KEY TO CONTINUE: ", but the result is not checked.
Hard to say why, probably by mistake
\footnote{What a strange feeling: to find bugs in such ancient software.}.

Let's see where the value from the global variable \TT{ctl\_mode} is used.

One such place is:

\lstinputlisting{examples/dongles/2/4.lst}

If it is 0, an encrypted error message is passed to a decryption routine and printed.

\myindex{x86!\Instructions!XOR}

The error string decryption routine seems a simple \gls{xoring}:

\lstinputlisting{examples/dongles/2/err_warn.lst}

That's why we were unable to find the error messages in the executable files, because they are encrypted
(which is is popular practice).

Another call to the \TT{SSQ()} hashing function passes the
\q{offln} string to it and compares the result with
\TT{0xFE81} and \TT{0x12A9}.

If they don't match, it works with some \TT{timer()} 
function (maybe waiting for a poorly
connected dongle to be reconnected and check again?) and then decrypts another error message to dump.

\lstinputlisting{examples/dongles/2/check2.lst}

Bypassing the dongle is pretty straightforward: just patch all jumps after the relevant \CMP instructions.

Another option is to write our own SCO OpenServer driver, containing a table of questions and answers, all of those which present in the program.

\subsubsection{Decrypting error messages}

By the way, we can also try to decrypt all error messages.
The algorithm that is located in the \TT{err\_warn()} function is very simple, indeed:

\lstinputlisting[caption=Decryption function]{examples/dongles/2/decrypting_EN.lst}

As we can see, 
not just string is supplied to the decryption function, but also the key:

\begin{lstlisting}
.text:0000DAF7 error:                                  ; CODE XREF: sync_sys+255j
.text:0000DAF7                                         ; sync_sys+274j ...
.text:0000DAF7                 mov     [ebp+var_8], offset encrypted_error_message2
.text:0000DAFE                 mov     [ebp+var_C], 17h ; decrypting key
.text:0000DB05                 jmp     decrypt_end_print_message

...

; this name we gave to label manually:
.text:0000D9B6 decrypt_end_print_message:              ; CODE XREF: sync_sys+29Dj
.text:0000D9B6                                         ; sync_sys+2ABj
.text:0000D9B6                 mov     eax, [ebp+var_18]
.text:0000D9B9                 test    eax, eax
.text:0000D9BB                 jnz     short loc_D9FB
.text:0000D9BD                 mov     edx, [ebp+var_C] ; key
.text:0000D9C0                 mov     ecx, [ebp+var_8] ; string
.text:0000D9C3                 push    edx
.text:0000D9C4                 push    20h
.text:0000D9C6                 push    ecx
.text:0000D9C7                 push    18h
.text:0000D9C9                 call    err_warn
\end{lstlisting}

The algorithm is a simple \gls{xoring}: each 
byte is xored with a key, but the key is increased by 3 after the processing of each byte.

We can write a simple Python script to check our hypothesis:

\lstinputlisting[caption=Python 3.x]{examples/dongles/2/decr1.py}

And it prints: \q{check security device connection}.
So yes, this is the decrypted message.

There are also 
other encrypted messages with their corresponding keys.
But needless to say, it is possible 
to decrypt them without their keys.
First, we can see that the key 
is in fact a byte.
It is because the core decryption instruction
(\XOR) works on byte level. 
The key is located in the \ESI register, but only one byte part of \ESI is used.
Hence, a key may be greater than 255, 
but its value is always to be rounded.

As a consequence, we can just try brute-force, trying all possible keys in the 0..255 range.
We are also going to skip 
the messages that has unprintable characters.

\lstinputlisting[caption=Python 3.x]{examples/dongles/2/decr2.py}

And we get:

\lstinputlisting[caption=Results]{examples/dongles/2/decr2_result.txt}

There 
is some garbage, but we can quickly find the English-language messages!

By the way, since the algorithm is a simple xoring encryption, the very same function can be used
to encrypt messages.
If needed, we can encrypt our own messages, and patch the program by inserting them.

