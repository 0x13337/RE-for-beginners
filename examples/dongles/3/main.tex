\section{\RU{Пример}\EN{Example} \#3: MS-DOS}
\label{dongle_16bit_dos}

\myindex{MS-DOS}
\RU{Еще одна очень старая программа для}\EN{Another very old software for} MS-DOS \RU{от}\EN{from} 1995 
\RU{также разработанная давно исчезнувшей компанией}
\EN{also developed by a company that disappeared a long time ago}.

\myindex{Intel!8086}
\myindex{Intel!80286}
\RU{Во времена перед DOS-экстендерами, всё ПО для MS-DOS рассчитывалось на процессоры 8086 или 80286,
так что в своей массе весь код был 16-битным}
\EN{In the pre-DOS extenders era, all the software for MS-DOS mostly relied on 16-bit 8086 or 80286 CPUs,
so en masse the code was 16-bit}.
\RU{16-битный код в основном такой же, какой вы уже видели в этой книге, но все регистры 16-битные,
и доступно меньше инструкций}
\EN{The 16-bit code is mostly same as you already saw in this book, but all registers
are 16-bit and there are less instructions available}.

\label{IN_example}
\label{OUT_example}
\myindex{x86!\Instructions!IN}
\myindex{x86!\Instructions!OUT}
\RU{Среда MS-DOS не могла иметь никаких драйверов, и ПО работало с \q{голым} железом через порты,
так что здесь вы можете увидеть инструкции \TT{OUT}/\TT{IN}, 
которые в наше время присутствуют в основном только
в драйверах (в современных OS нельзя обращаться на прямую к портам из \gls{user mode})}
\EN{The MS-DOS environment has no system drivers, and any program can deal with the bare hardware via ports,
so here you can see the \TT{OUT}/\TT{IN} instructions, which are present in mostly in drivers in our times
(it is impossible to access ports directly in \gls{user mode} on all modern \ac{OS}es)}.

\RU{Учитывая это, ПО для MS-DOS должно работать с донглой обращаясь к принтерному LPT-порту
напрямую}
\EN{Given that, the MS-DOS program which works with a dongle has to access the LPT printer port directly}.
\RU{Так что мы можем просто поискать эти инструкции. И да, вот они}
\EN{So we can just search for such instructions. And yes, here they are}:

\lstinputlisting{examples/dongles/3/1.lst}

(\RU{Все имена меток в этом примере даны мною}\EN{All label names in this example were given by me}).

\RU{Функция }\TT{out\_port()} \RU{вызывается только из одной функции}
\EN{is referenced only in one function}:

\lstinputlisting{examples/dongles/3/2.lst}

\RU{Это также \q{хеширующая} донгла Sentinel Pro как и в предыдущем примере}
\EN{This is again a Sentinel Pro \q{hashing} dongle as in the previous example}.
\RU{Это заметно по тому что текстовые строки передаются и здесь, 16-битные значения также возвращаются и сравниваются с другими}%
\EN{It is noticeably because text strings are passed here, too, and 16 bit values are returned and compared with others}.

\RU{Так вот как происходит работа с Sentinel Pro через порты}
\EN{So that is how Sentinel Pro is accessed via ports}.
\RU{Адрес выходного порта обычно 0x378, т.е. принтерного порта, данные для него во времена
перед USB отправлялись прямо сюда}
\EN{The output port address is usually 0x378, i.e.,
the printer port, where the data to the old printers in pre-USB era was passed to}.
\RU{Порт однонаправленный, потому что когда его разрабатывали, никто не мог предположить,
что кому-то понадобится получать информацию из принтера}
\EN{The port is uni-directional, because when it was developed, no one imagined that someone
will need to transfer information from the printer}
\footnote{\RU{Если учитывать только Centronics и не учитывать последующий стандарт IEEE 1284\EMDASH{}
в нем из принтера можно получать информацию}
\EN{If we consider Centronics only. The following IEEE 1284 standard allows the transfer of information from
the printer}.}.
\RU{Единственный способ получить информацию из принтера это регистр статуса на порту 0x379,
он содержит такие биты как \q{paper out}, \q{ack}, \q{busy}\EMDASH{}так принтер может сигнализировать
о том, что он готов или нет, и о том, есть ли в нем бумага}
\EN{The only way to get information from the printer is a status register on port 0x379, which contains
such bits as \q{paper out}, \q{ack}, \q{busy}\EMDASH{}thus the printer may signal to the host computer
if it is ready or not and if paper is present in it}.
\RU{Так что донгла возвращает информацию через какой-то из этих бит, по одному биту на каждой
итерации}
\EN{So the dongle returns information from one of these bits, one bit at each iteration}.

\TT{\_in\_port\_2} \RU{содержит адрес статуса}\EN{contains the address of the status word} (0x379) \AndENRU 
\TT{\_in\_port\_1} \RU{содержит адрес управляющего регистра}\EN{contains the control register address} (0x37A).

\RU{Судя по всему, донгла возвращает информацию только через флаг \q{busy} на}
\EN{It seems that the dongle returns information via the \q{busy} flag at} \TT{seg030:00B9}: 
\RU{каждый бит записывается в регистре \TT{DI} позже возвращаемый в самом конце функции}
\EN{each bit is stored in the \TT{DI} register, which is returned at the end of the function}.

\RU{Что означают все эти отсылаемые в выходной порт байты}%
\EN{What do all these bytes sent to output port mean}?
\RU{Трудно сказать. Возможно, команды донглы.}\EN{Hard to say. Probably commands to the dongle.}
\RU{Но честно говоря, нам и не обязательно знать: нашу задачу можно легко решить и без этих знаний}
\EN{But generally speaking, it is not necessary to know: it is easy to solve our task without that knowledge}.

\RU{Вот функция проверки донглы}\EN{Here is the dongle checking routine}:

\lstinputlisting{examples/dongles/3/3.lst}

\RU{А так как эта функция может вызываться слишком часто, например, 
перед выполнением каждой важной возможности ПО,
а обращение к донгле вообще-то медленное (и из-за медленного принтерного порта, и из-за медленного
\ac{MCU} в донгле), так что они, наверное, добавили возможность пропускать проверку донглы слишком часто,
используя текущее время в функции \TT{biostime()}}
\EN{Since the routine can be called very frequently, e.g., before the execution of each important software feature, 
and accessing the dongle is generally slow (because of the slow printer port and also slow
\ac{MCU} in the dongle), they probably added a way to skip some dongle checks,
by checking the current time in the \TT{biostime()} function}.

\myindex{\CStandardLibrary!rand()}
\EN{The}\RU{Функция} \TT{get\_rand()} \RU{использует стандартную функцию Си}
\EN{function uses the standard C function}:

\lstinputlisting{examples/dongles/3/4.lst}

\RU{Так что текстовая строка выбирается случайно, отправляется в донглу и результат
хеширования сверяется с корректным значением}
\EN{So the text string is selected randomly, passed into the dongle, and then the result of the hashing 
is compared with the correct value}.

\RU{Текстовые строки, похоже, составлялись так же случайно, во время разработки ПО.}%
\EN{The text strings seem to be constructed randomly as well, during software development.}

\RU{И вот как вызывается главная процедура проверки донглы}
\EN{And this is how the main dongle checking function is called}:

\lstinputlisting{examples/dongles/3/5.lst}

\RU{Заставить работать программу без донглы очень просто: просто заставить функцию
\TT{check\_dongle()} возвращать всегда 0}
\EN{Bypassing the dongle is easy, just force the \TT{check\_dongle()} function to always return 0}.

\RU{Например, вставив такой код в самом её начале}\EN{For example, by inserting this code at its beginning}:

\begin{lstlisting}
mov ax,0
retf
\end{lstlisting}

\myindex{\CStandardLibrary!strcpy()}
\RU{Наблюдательный читатель может заметить, что функция Си \TT{strcpy()} имеет 2 аргумента, но здесь
мы видим, что передается 4}
\EN{The observant reader might recall that the \TT{strcpy()} C function usually requires two pointers in its arguments,
but we see that 4 values are passed}:

\begin{lstlisting}
seg033:088F 1E                          push    ds
seg033:0890 68 22 44                    push    offset aTrupcRequiresA ; "This Software Requires a Software Lock\n"
seg033:0893 1E                          push    ds
seg033:0894 68 60 E9                    push    offset byte_6C7E0 ; dest
seg033:0897 9A 79 65 00+                call    _strcpy
seg033:089C 83 C4 08                    add     sp, 8
\end{lstlisting}

\RU{Это связано с моделью памяти в MS-DOS. Об этом больше читайте здесь}
\EN{This is related to MS-DOS' memory model. You can read more about it here}: 
\myref{8086_memory_model}.

\RU{Так что, \TT{strcpy()}, как и любая другая функция принимающая указатель (-и) в аргументах,
работает с 16-битными парами}
\EN{So as you may see, \TT{strcpy()} and any other function that take pointer(s) in arguments
work with 16-bit pairs}.

\RU{Вернемся к нашему примеру}\EN{Let's get back to our example}.
\TT{DS} \RU{сейчас указывает на сегмент данных размещенный в исполняемом файле, там, где хранится текстовая
строка.}
\EN{is currently set to the data segment located in the executable,
that is where the text string is stored.}

\myindex{x86!\Instructions!LES}
\RU{В функции \TT{sent\_pro()} каждый байт строки загружается на \TT{seg030:00EF}: инструкция
\TT{LES} загружает из переданного аргумента пару ES:BX одновременно}
\EN{In the \TT{sent\_pro()} function, each byte of the string is loaded at \TT{seg030:00EF}: the \TT{LES} instruction
loads the ES:BX pair simultaneously from the passed argument}.
\RU{\MOV на \TT{seg030:00F5} загружает байт из памяти, на который указывает пара ES:BX.}
\EN{The \MOV at \TT{seg030:00F5} loads the byte from the memory at which the ES:BX pair points.}

% TODO rewrite
%\RU{На \TT{seg030:00F2} \glslink{increment}{инкрементируется} только вторая 16-битная пара адреса.}
%\EN{At \TT{seg030:00F2} only a second 16-bit part of address is \glslink{increment}{incremented}.}
%\RU{Это значит, что переданная в функцию строка не может находиться на границе двух сегментов.}
%\EN{This implies that the string passed to the function cannot be located on the boundary between two data segments.}

