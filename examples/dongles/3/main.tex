\subsection{\IFRU{Пример}{Example} \#3: MS-DOS}

\index{MS-DOS}
\IFRU{Еще одна очень старая программа для}{Another very old software for} MS-DOS \IFRU{от}{from} 1995 
\IFRU{также разработанная давно исчезнувшей компанией}
{also developed by a company disappeared long time ago}.

\index{8086}
\index{80286}
\IFRU{Во времена перед DOS-экстендерами, всё ПО для MS-DOS расчитывалось на процессоры 8086 или 80286,
так что в своей массе весь код был 16-битным}
{In the pre-DOS extenders era, all the software for MS-DOS were mostly rely on 16-bit 8086 or 80286 CPUs,
so en masse code was 16-bit}.
\IFRU{16-битный код в основном такой же, какой вы уже видели в этой книге, но все регистры 16-битные,
и доступно меньше инструкций}
{16-bit code is mostly same as you already saw in this book, but all registers
are 16-bit and there are less number of instructions available}.

\label{IN_example}
\label{OUT_example}
\index{x86!\Instructions!IN}
\index{x86!\Instructions!OUT}
\IFRU{Среда MS-DOS не могла иметь никаких драйверов, и ПО работало с ``голым'' железом через порты,
так что здесь вы можете увидеть инструкции \TT{OUT}/\TT{IN}, 
которые в наше время присутствуют в основном только
в драйверах (в современных OS нельзя обращаться на прямую к портам из \gls{user mode})}
{MS-DOS environment has no any system drivers, any program may deal with bare hardware via ports,
so here you may see \TT{OUT}/\TT{IN} instructions, which are mostly present in drivers in our times
(it is not possible to access ports directly in \gls{user mode} in all modern OS)}.

\IFRU{Учитывая это, ПО для MS-DOS должно работать с донглой обращаясь к принтерному LPT-порту
напрямую}
{Given that, the MS-DOS program working with a dongle should access LPT printer port directly}.
\IFRU{Так что мы можем просто поискать эти инструкции. И да, вот они}
{So we can just search for such instructions. And yes, here it is}:

\lstinputlisting{examples/dongles/3/1.lst}

(\IFRU{Все имена меток в этом примере даны мною}{All label names in this example were given by me}).

\IFRU{Функция }{}\TT{out\_port()} \IFRU{вызывается только из одной ф-ции}
{is referenced only in one function}:

\lstinputlisting{examples/dongles/3/2.lst}

\IFRU{Это также ``хеширующая'' донгла Sentinel Pro как и в предыдущем примере}
{It is also Sentinel Pro ``hashing'' dongle as in the previous example}.
\IFRU{Я заметил это по тому что текстовые строки передаются и здесь, 
16-битные значения также возвращаются и сравниваются с другими}
{I figured out its type by noticing that a text strings are also passed here and 16 bit values
are also returned and compared with others}.

\IFRU{Так вот как происходит работа с Sentinel Pro через порты}
{So that is how Sentinel Pro is accessed via ports}.
\IFRU{Адрес выходного порта обычно 0x378, т.е., принтерного порта, данные для него во времена
перед USB отправлялись прямо сюда}
{Output port address is usually 0x378, i.e.,
printer port, the data to the old printers in pre-USB era were passed to it}.
\IFRU{Порт однонаправленный, потому что когда его разрабатывали, никто не мог предположить
что кому-то понадобится получать информацию из принтера}
{The port is one-directional, because when it was developed, no one can imagined someone
will need to transfer information from the printer}
\footnote{\IFRU{Если учитывать только Centronics и не учитывать последующий стандарт IEEE 1284 ~---
в нем из принтера можно получать информацию}
{If to consider Centronics only. Following IEEE 1284 standard allows to transfer information from
the printer}.}.
\IFRU{Единственный способ получить информацию из принтера это регистр статуса на порту 0x379,
он содержит такие биты как ``paper out'', ``ack'', ``busy'' ~--- так принтер может сигнализировать
о том что он готов или нет, и о том, есть ли в нем бумага}
{The only way to get information from the printer, is a status register on port 0x379, it contain
such bits as ``paper out'', ``ack'', ``busy''~---thus printer may signal to the host computer
that it is ready or not and if a paper present in it}.
\IFRU{Так что донгла возвращает информацию через какой-то из этих бит, по одному биту на каждой
итерации}
{So the dongle return information from one of these bits, by one bit at each iteration}.

\TT{\_in\_port\_2} \IFRU{содержит адрес статуса}{has address of status word} (0x379) \AndENRU 
\TT{\_in\_port\_1} \IFRU{содержит адрес управляющего регистра}{has control register address} (0x37A).

\IFRU{Судя по всему, донгла возвращает информацию только через флаг ``busy'' на}
{It seems, the dongle return information via ``busy'' flag at} \TT{seg030:00B9}: 
\IFRU{каждый бит записывается в регистре \TT{DI} позже возвращаемый в самом конце ф-ции}
{each bit is stored in the \TT{DI} register, later returned at the function end}.

\IFRU{Что означают все эти отсылаемые в выходной порт байты}
{What all these bytes sent to output port mean}?
\IFRU{Я не знаю. Возможно, команды донглы.}
{I don't know. Probably commands to the dongle.}
\IFRU{Но честно говоря, нам и не обязательно знать: нашу задачу можно легко решить и без этих знаний}
{But generally speaking, it is not necessary to know: it is easy to solve our task without that knowledge}.

\IFRU{Вот ф-ция проверки донглы}{Here is a dongle checking routine}:

\lstinputlisting{examples/dongles/3/3.lst}

\IFRU{А так как эта ф-ция может вызываться слишком часто, например, 
перед выполнением каждой важной фичи ПО,
а обращение к донгле вообще-то медленное (и из-за медленного принтерного порта и из-за медленного
\ac{MCU} в донгле), так что они наверное добавили возможность пропускать проверку донглы слишком часто,
используя текущее время в ф-ции \TT{biostime()}}
{Since the routine may be called too frequently, e.g., before each important software feature executing, 
and the dongle accessing process is generally slow (because of slow printer port and also slow
\ac{MCU} in the dongle), so they probably added a way to skip dongle checking too often,
using checking current time in \TT{biostime()} function}.

\index{\CStandardLibrary!rand()}
\IFRU{Ф-ция }{}\TT{get\_rand()} \IFRU{использует стандартную ф-цию Си}
{function uses standard C function}:

\lstinputlisting{examples/dongles/3/4.lst}

\IFRU{Так что текстовая строка выбирается случайно, отправляется в донглу и результат
хеширования сверяется с корректным значением}
{So the text string is selected randomly, passed into dongle, and then the result of hashing 
is compared with correct value}.

\IFRU{Текстовые строки, похоже, выбирались так же случайно}
{Text strings are seems to be chosen randomly as well}.

\IFRU{И вот как вызывается главная процедура проверки донглы}
{And that is how the main dongle checking function is called}:

\lstinputlisting{examples/dongles/3/5.lst}

\IFRU{Заставить работать программу без донглы очень просто: просто заставить ф-цию
\TT{check\_dongle()} возвращать всегда 0}
{Dongle bypassing is easy, just force the \TT{check\_dongle()} function to always return 0}.

\IFRU{Например, вставив такой код в самом её начале}{For example, by inserting this code at its beginning}:

\begin{lstlisting}
mov ax,0
retf
\end{lstlisting}

\subsubsection{\IFRU{Модель памяти в MS-DOS}{MS-DOS memory model}}

\index{MS-DOS\IFRU{Модель памяти}{Memory model}}
\label{dos_memory_model}
\index{\CStandardLibrary!strcpy()}
\IFRU{Наблюдательный читатель может заметить что ф-ция Си \TT{strcpy()} имеет 2 аргумента, но здесь
мы видим что передается 4}
{Observant reader might recall that \TT{strcpy()} C function usually requires two pointers in arguments,
but we saw how 4 values are passed}:

\begin{lstlisting}
seg033:088F 1E                          push    ds
seg033:0890 68 22 44                    push    offset aTrupcRequiresA ; "This Software Requires a Software Lock\n"
seg033:0893 1E                          push    ds
seg033:0894 68 60 E9                    push    offset byte_6C7E0 ; dest
seg033:0897 9A 79 65 00+                call    _strcpy
seg033:089C 83 C4 08                    add     sp, 8
\end{lstlisting}

\IFRU{Что это означает? О да, еще один дивный артефакт MS-DOS и 8086}
{What it means? Oh yes, that is another MS-DOS and 8086 weird artefact}.

8086/8088 \IFRU{был 16-битным процессором, но мог адресовать 20-битное адресное
пространство (таким образом мог адресовать 1MB внешней памяти)}
{was a 16-bit CPU, but was able to address 20-bit address RAM 
(thus resulting 1MB external memory)}.
\IFRU{Внешняя адресное пространство было разделено между \ac{RAM} (максимум 640KB),
\ac{ROM}, окна для видеопамяти, EMS-карт, итд}
{External memory address space was divided between \ac{RAM} (640KB max),
\ac{ROM}, windows for video memory, EMS cards, etc}.

\IFRU{Припомним также что 8086/8088 был на самом деле наследником 8-битного процессора 8080}
{Let's also recall that 8086/8088 was in fact inheritor of 8-bit 8080 CPU}.
\IFRU{Процессор 8080 имел 16-битное адресное пространство, т.е., мог адресовать только}
{The 8080 has 16-bit memory spaces, i.e., it was able to address only} 64KB.
\IFRU{И возможно в расчете на портирование старого ПО\footnote{Хотя я и не уверен на 100\%},
8086 может поддерживать 64-килобайтные
окна, одновременно много таких, расположенных внутри одномегабайтного адресного пространства}
{And probably of old software porting reason\footnote{I'm not 100\% sure here},
8086 can support 64KB windows, many of them placed simultaneously
within 1MB address space}.
\IFRU{Это, в каком-то смысле, игрушечная виртуализация}
{This is some kind of toy-level virtualization}.
\IFRU{Все регистры 8086 16-битные, так что, чтобы адресовать больше, специальные сегментные
регистры (CS, DS, ES, SS) были введены}
{All 8086 registers are 16-bit, so to address more, a special segment registers (CS, DS, ES, SS) were
introduced}.
\IFRU{Каждый 20-битный указатель вычисляется используя значения из пары состоящей из сегментного регистра
и адресного регистра (например DS:BX) вот так}
{Each 20-bit pointer is calculated using values from a segment register and 
an address register pair (e.g. DS:BX) as follows:}

\begin{center}
$real\_address = (segment\_register \ll 4) + address\_register$
\end{center}

\IFRU{Например, окно памяти для графики (\ac{EGA}, \ac{VGA}) на старых IBM PC-совместимых компьютерах
имело размер 64KB.}
{For example, graphics (\ac{EGA}, \ac{VGA}) video \ac{RAM} window on old IBM PC-compatibles 
has size of 64KB.}
\IFRU{Для доступа к нему, значение 0xA000 должно быть записано в один из сегментных регистров,
например, в DS.}
{For accessing it, a 0xA000 value should be stored in one of segment registers, e.g. into DS.}
\IFRU{Тогда DS:0 будет адресовать самый первый байт видеопамяти, а DS:0xFFFF ~--- самый последний байт.}
{Then DS:0 will address the very first byte of video \ac{RAM} and DS:0xFFFF is the very last byte of RAM.}
\IFRU{А реальный адрес на 20-битной адресной шине, на самом деле будет от 0xA0000 до 0xAFFFF.}
{The real address on 20-bit address bus, however, will range from 0xA0000 to 0xAFFFF.}

\IFRU{Программа может содержать жесткопривязанные адреса вроде 0x1234, но \ac{OS} может иметь необходимость
загрузить программу по другим адресам, так что она пересчитает значения для сегментных регистров так,
что программа будет нормально работать не обращая внимания на то,
в каком месте памяти она была расположена.}
{The program may contain hardcoded addresses like 0x1234, but \ac{OS} may need to load program on arbitrary
addresses, so it recalculate segment register values in such way, so the program will not care about
where in the RAM it is placed.}

\IFRU{Так что, любой указатель в окружении старой MS-DOS на самом деле состоял из адреса сегмента
и адреса внутри сегента, т.е., из двух 16-битных значений. 20-битного значения было бы достаточно для
этого, хотя, тогда пришлось бы вычислять адреса слишком часто: так что передача большего количества
информации в стеке это более хороший баланс между экономией места и удобством}
{So, any pointer it old MS-DOS environment was in fact consisted of segment address and the address inside
segment, i.e., two 16-bit values. 20-bit was enough for that, though, but one will need to recalculate
the addresses very often: passing more information on stack is seems better space/convenience balance}.

\IFRU{Кстати, из-за всего этого, не было возможным выделить блок памяти больше чем}
{By the way, because of all this, it was not possible to allocate the memory block larger than} 64KB.

\IFRU{Так что, \TT{strcpy()}, как и любая другая ф-ция принимающая указатель (-и) в аргументах,
работает с 16-битными парами}
{So as you may see, \TT{strcpy()} and any other function taking pointer(s) in arguments,
works with 16-bit pairs}.

\IFRU{Вернемся к нашему примеру}{Let's back to our example}.
\TT{DS} \IFRU{сейчас указывает на сегмент данных размещенный в исполняемом файле, там где хранится текстовая
строка.}
{is currently set to data segment located in the executable,
that is where the text string is stored.}

\index{x86!\Instructions!LES}
\IFRU{В ф-ции \TT{sent\_pro()} каждый байт строки загружается на \TT{seg030:00EF}: инструкция
\TT{LES} загружает из переданного аргумента пару ES:BX одновременно}
{In the \TT{sent\_pro()} function, each byte of string is loaded at \TT{seg030:00EF}: the \TT{LES} instruction
loads ES:BX pair simultaneously from the passed argument}.
\IFRU{\MOV на \TT{seg030:00F5} загружает байт из памяти, на который указывает пара ES:BX.}
{The \MOV at \TT{seg030:00F5} loads the byte from the memory to which ES:BX pair points.}

\IFRU{На \TT{seg030:00F2} \glslink{increment}{инкрементируется} только 16-битное слово,
но не значение сегмента.}
{At \TT{seg030:00F2} only 16-bit word is \glslink{increment}{incremented}, not segment value.}
\IFRU{Это значит что переданная в ф-цию строка не может находится на границе двух сегментов.}
{This means, the string passed to the function cannot be located on two data segments boundaries.}

\index{80286}
\index{80386}
\IFRU{В 80286 сегментные регистры получили новую роль селекторов, имеющих немного другую ф-цию.}
{Segment registers were reused at 80286 as selectors, serving different function.}

\index{MS-DOS!DOS extenders}
\IFRU{Когда появился процессор 80386 и компьютеры с большей памятью,
MS-DOS была всё еще популярна, так что появились DOS-экстендеры: на самом деле это уже был шаг
к ``серьезным'' \ac{OS}, они переключали \ac{CPU} в защищенный режим и предлагали куда лучшее \ac{API} для
программ, которые всё еще предполагалось запускать в MS-DOS.}
{When 80386 CPU and computers with bigger \ac{RAM} were introduced,
MS-DOS was still popular, so the DOS extenders are emerged: these were in fact
a step toward ``serious'' \ac{OS},
switching CPU into protected mode and providing much better memory \ac{API}s for the programs 
which still needs to be runned from MS-DOS.}
\IFRU{Широко известные примеры это DOS/4GW (игра DOOM была скопилирована под него), Phar Lap, PMODE}
{Widely popular examples include DOS/4GW (DOOM video game was compiled for it), Phar Lap, PMODE.} \\
\\
\index{Windows 3.x}
\index{Win32}
\IFRU{Кстати, точно такой же способ адресации памяти был и в 16-битной линейке Windows 3.x, перед Win32}
{By the way, the same was of addressing memory was in 16-bit line of Windows 3.x, before Win32}.
