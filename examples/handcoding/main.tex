\chapter{\RU{Вручную написанный на ассемблере код}\EN{Handwritten assembly code}}

\section{\RU{Тестовый файл} EICAR\EN{ test file}}
\label{subsec:EICAR}

\index{MS-DOS}
\index{EICAR}
\RU{Этот .COM-файл предназначен для тестирования антивирусов, его можно запустить в MS-DOS
и он выведет такую строку}\EN{This .COM-file is intended for antivirus testing, it is possible to run in
in MS-DOS and it will print string}: ``EICAR-STANDARD-ANTIVIRUS-TEST-FILE!''
\footnote{\url{\RU{https://ru.wikipedia.org/wiki/EICAR-Test-File}
\EN{https://en.wikipedia.org/wiki/EICAR_test_file}}}.
% TODO \ref{} -> about .COM files

\RU{Он примечателен тем, что он полностью состоит только из печатных ASCII-символов, следовательно, его можно
набрать в любом текстовом редакторе}\EN{Its important property is that it's entirely consisting of printable 
ASCII-symbols, which, in turn, makes possible to create it in any text editor}:

\begin{lstlisting}
X5O!P%@AP[4\PZX54(P^)7CC)7}$EICAR-STANDARD-ANTIVIRUS-TEST-FILE!$H+H*
\end{lstlisting}

\RU{Попробуем его разобрать}\EN{Let's decompile it}:

\lstinputlisting{examples/handcoding/EICAR_\LANG.lst}

\RU{Я добавил везде комментарии, показывающие состояние регистров и стека после каждой инструкции}\EN{I added 
comments about registers and stack after each instruction}.

\RU{Собственно, все эти инструкции нужны только для того чтобы исполнить следующий код}\EN{Essentially, all these
instructions are here only to execute this code}:

\begin{lstlisting}
B4 09     MOV AH, 9
BA 1C 01  MOV DX, 11Ch
CD 21     INT 21h
CD 20     INT 20h
\end{lstlisting}

\index{x86!\Instructions!INT}
\TT{INT 21h} \RU{с ф-цией 9 (переданной в \TT{AH}) просто выводит строку, адрес которой передан в}\EN{with 9th
function (passed in \TT{AH}) just prints a string, address of which is passed in} \TT{DS:DX}.
\RU{Кстати, строка должна быть завершена символом '\$'}\EN{By the way, the string should be terminated
with '\$' sign}.
\RU{Вероятно, это наследие}\EN{Apparently, it's inherited from} \gls{CP/M} 
\RU{и эта ф-ция в DOS осталась для совместимости}\EN{and this function was left in DOS for compatibility}.
\TT{INT 20h} \RU{возвращает управление в}\EN{exits to} DOS.

\RU{Но, как видно, далеко не все опкоды этих инструкций печатные}\EN{But as we can see, these instruction's
opcodes are not strictly printable}.
\RU{Так что основная часть EICAR-файла это}\EN{So the main part of EICAR-file is}:

\begin{itemize}
\item \RU{подготовка нужных значений регистров (AH и DX)}\EN{preparing register (AH and DX) values we need};
\item \RU{подготовка в памяти опкодов для INT 21 и INT 20}\EN{preparing INT 21 and INT 20 opcodes in memory};
\item \RU{исполнение}\EN{executing} INT 21 \AndENRU INT 20.
\end{itemize}

\index{Shellcode}
\RU{Кстати, подобная техника широко используется для создания шеллкодов, 
где нужно создать x86-код, который будет нужно передать в виде текстовой строки}
\EN{By the way, this technique is widely used in shellcode constructing, when one need to pass x86-code
in the string form}.

\RU{Здесь также список всех x86-инструкций с печатаемыми опкодоами}\EN{Here is also a list of all 
x86 instructions which has printable opcodes}: \ref{printable_x86_opcodes}.

