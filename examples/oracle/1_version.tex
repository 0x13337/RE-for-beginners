\subsection{\RU{Таблица \TT{V\$VERSION} в \oracle}\EN{\TT{V\$VERSION} table in the \oracle}}

\myindex{\oracle}
\myindex{Linux}
\myindex{Windows!ntoskrnl.exe}
\RU{\oracle 11.2 это очень большая программа, основной модуль \TT{oracle.exe} содержит около 124 тысячи функций.}\EN{\oracle 11.2 is a huge program, its main module \TT{oracle.exe} contain approx. 124,000 functions.} \RU{Для сравнения, ядро Windows 7 x64 (ntoskrnl.exe)\EMDASH{}около 11 тысяч функций, а ядро Linux 3.9.8 (с драйверами по умолчанию)\EMDASH{}31 тысяч функций.}\EN{For comparison, the Windows 7 x86 kernel (ntoskrnl.exe) contains approx. 11,000 functions and the Linux 3.9.8 kernel (with default drivers compiled)\EMDASH{}31,000 functions.}

\RU{Начнем с одного простого вопроса. Откуда \oracle берет информацию, когда мы в SQL*Plus пишем вот такой вот простой запрос:}\EN{Let's start with an easy question. Where does \oracle get all this information, when we execute this simple statement in SQL*Plus:}

\begin{lstlisting}
SQL> select * from V$VERSION;
\end{lstlisting}

\RU{И получаем}\EN{And we get}:

\begin{lstlisting}
BANNER
--------------------------------------------------------

Oracle Database 11g Enterprise Edition Release 11.2.0.1.0 - Production
PL/SQL Release 11.2.0.1.0 - Production
CORE    11.2.0.1.0      Production
TNS for 32-bit Windows: Version 11.2.0.1.0 - Production
NLSRTL Version 11.2.0.1.0 - Production
\end{lstlisting}

\RU{Начнем. Где в самом \oracle мы можем найти строку}\EN{Let's start. Where in the \oracle can we find the string} \TT{V\$VERSION}?

\RU{Для win32-версии, эта строка имеется в файле \TT{oracle.exe}, это легко увидеть.}
\EN{In the win32-version, \TT{oracle.exe} file contains the string,
it's easy to see.}
\RU{Но мы также можем использовать объектные (.o) файлы от версии \oracle для Linux, потому что в них сохраняются имена функций и глобальных переменных, а в \TT{oracle.exe} для win32 этого нет.}\EN{But we can also use the object (.o) files from the Linux version of \oracle since, unlike the win32 version \TT{oracle.exe}, the function names (and global variables as well) are preserved there.}

\RU{Итак, строка \TT{V\$VERSION} имеется в файле \TT{kqf.o}, в самой главной Oracle-библиотеке \TT{libserver11.a}.}\EN{So, the \TT{kqf.o} file contains the \TT{V\$VERSION} string.
The object file is in the main Oracle-library \TT{libserver11.a}.}

\RU{Ссылка на эту текстовую строку имеется в таблице \TT{kqfviw}, размещенной в этом же файле \TT{kqf.o}}\EN{A reference to this text string can find in the \TT{kqfviw} table stored in the same file, \TT{kqf.o}}:

\begin{lstlisting}[caption=kqf.o]
.rodata:0800C4A0 kqfviw dd 0Bh    ; DATA XREF: kqfchk:loc_8003A6D
.rodata:0800C4A0                  ; kqfgbn+34
.rodata:0800C4A4        dd offset _2__STRING_10102_0 ; "GV$WAITSTAT"
.rodata:0800C4A8        dd 4
.rodata:0800C4AC        dd offset _2__STRING_10103_0 ; "NULL"
.rodata:0800C4B0        dd 3
.rodata:0800C4B4        dd 0
.rodata:0800C4B8        dd 195h
.rodata:0800C4BC        dd 4
.rodata:0800C4C0        dd 0
.rodata:0800C4C4        dd 0FFFFC1CBh
.rodata:0800C4C8        dd 3
.rodata:0800C4CC        dd 0
.rodata:0800C4D0        dd 0Ah
.rodata:0800C4D4        dd offset _2__STRING_10104_0 ; "V$WAITSTAT"
.rodata:0800C4D8        dd 4
.rodata:0800C4DC        dd offset _2__STRING_10103_0 ; "NULL"
.rodata:0800C4E0        dd 3
.rodata:0800C4E4        dd 0
.rodata:0800C4E8        dd 4Eh
.rodata:0800C4EC        dd 3
.rodata:0800C4F0        dd 0
.rodata:0800C4F4        dd 0FFFFC003h
.rodata:0800C4F8        dd 4
.rodata:0800C4FC        dd 0
.rodata:0800C500        dd 5
.rodata:0800C504        dd offset _2__STRING_10105_0 ; "GV$BH"
.rodata:0800C508        dd 4
.rodata:0800C50C        dd offset _2__STRING_10103_0 ; "NULL"
.rodata:0800C510        dd 3
.rodata:0800C514        dd 0
.rodata:0800C518        dd 269h
.rodata:0800C51C        dd 15h
.rodata:0800C520        dd 0
.rodata:0800C524        dd 0FFFFC1EDh
.rodata:0800C528        dd 8
.rodata:0800C52C        dd 0
.rodata:0800C530        dd 4
.rodata:0800C534        dd offset _2__STRING_10106_0 ; "V$BH"
.rodata:0800C538        dd 4
.rodata:0800C53C        dd offset _2__STRING_10103_0 ; "NULL"
.rodata:0800C540        dd 3
.rodata:0800C544        dd 0
.rodata:0800C548        dd 0F5h
.rodata:0800C54C        dd 14h
.rodata:0800C550        dd 0
.rodata:0800C554        dd 0FFFFC1EEh
.rodata:0800C558        dd 5
.rodata:0800C55C        dd 0
\end{lstlisting}

\RU{Кстати, нередко, при изучении внутренностей \oracle, появляется вопрос, почему имена функций и глобальных переменных такие странные}\EN{By the way, often, while analysing \oracle's internals, you may ask yourself, why are the names of the functions and global variable so weird}. \RU{Вероятно, дело в том, что \oracle очень старый продукт сам по себе и писался на Си еще в 1980-х}
\EN{Probably, because \oracle is a very old product and was developed in C in the 1980s}. \RU{А в те времена стандарт Си гарантировал поддержку имен переменных длиной только до шести символов включительно}
\EN{And that was a time when the C standard guaranteed that the function names/variables can support only up to 6 characters inclusive}: <<6 significant initial characters in an external identifier>>\footnote{\href{http://go.yurichev.com/17142}{Draft ANSI C Standard (ANSI X3J11/88-090) (May 13, 1988) (yurichev.com)}}

\RU{Вероятно, таблица \TT{kqfviw} содержащая в себе многие (а может даже и все) view с префиксом V\$, это служебные view (fixed views), присутствующие всегда.}\EN{Probably, the table \TT{kqfviw} contains most (maybe even all) views prefixed with V\$, these are \IT{fixed views}, present all the time.}
\RU{Бегло оценив цикличность данных, мы легко видим, что в каждом элементе таблицы \TT{kqfviw} 12 полей 32-битных полей.}\EN{Superficially, by noticing the cyclic recurrence of data, we can easily see that each \TT{kqfviw} table element has 12 32-bit fields.}
\RU{В \IDA легко создать структуру из 12-и элементов и применить её ко всем элементам таблицы.}\EN{It is very simple to create a 12-elements structure in \IDA and apply it to all table elements.}
\RU{Для версии \oracle 11.2, здесь 1023 элемента в таблице, то есть, здесь описываются 1023 всех возможных \IT{fixed view}.}\EN{As of \oracle version 11.2, there are 1023 table elements, i.e., in it are described 1023 of all possible \IT{fixed views}.}
\RU{Позже, мы еще вернемся к этому числу.}
\EN{We are going to return to this number later.}

\RU{Как видно, мы не очень много можем узнать чисел в этих полях. Самое первое число всегда равно длине строки-названия view (без терминирующего ноля).}
\EN{As we can see, there is not much information in these numbers in the fields. The first number is always equals to the name of the view (without the terminating zero.}
\RU{Это справедливо для всех элементов. Но эта информация не очень полезна.}\EN{This is correct for each element. But this information is not very useful.}

\RU{Мы также знаем, что информацию обо всех fixed views можно получить из \IT{fixed view} под названием}
\EN{We also know that the information about all fixed views can be retrieved from a \IT{fixed view} named} \TT{V\$FIXED\_VIEW\_DEFINITION}
\RU{(кстати, информация для этого view также берется из таблиц \TT{kqfviw} и \TT{kqfvip}).}\EN{(by the way, the information for this view is also taken from the \TT{kqfviw} and \TT{kqfvip} tables.)}
\RU{Кстати, там тоже 1023 элемента. Совпадение? Нет.}\EN{By the way, there are 1023 elements in those too. Coincidence? No.}

\begin{lstlisting}
SQL> select * from V$FIXED_VIEW_DEFINITION where view_name='V$VERSION';

VIEW_NAME
------------------------------
VIEW_DEFINITION
------------------------------

V$VERSION
select  BANNER from GV$VERSION where inst_id = USERENV('Instance')
\end{lstlisting}

\RU{Итак, \TT{V\$VERSION} это как бы \IT{thunk view} для другого, с названием \TT{GV\$VERSION}, который, в свою очередь:}
\EN{So, \TT{V\$VERSION} is some kind of a \IT{thunk view} for another view, named \TT{GV\$VERSION}, which is, in turn:}

\begin{lstlisting}
SQL> select * from V$FIXED_VIEW_DEFINITION where view_name='GV$VERSION';

VIEW_NAME
------------------------------
VIEW_DEFINITION
------------------------------

GV$VERSION
select inst_id, banner from x$version
\end{lstlisting}

\RU{Таблицы с префиксом X\$ в \oracle\EMDASH{}это также служебные таблицы, они не документированы, не могут изменятся пользователем, и обновляются динамически.}\EN{The tables prefixed with X\$ in the \oracle are service tables too, undocumented, cannot be changed by the user and are refreshed dynamically.}

\RU{Попробуем поискать текст}\EN{If we search for the text} \\
\begin{lstlisting}
select BANNER from GV\$VERSION where inst\_id = 
USERENV('Instance')
\end{lstlisting}
... \RU{в файле \TT{kqf.o} и находим ссылку на него в таблице \TT{kqfvip}}
\EN{in the \TT{kqf.o} file, we find it in the \TT{kqfvip} table}:

\begin{lstlisting}[caption=kqf.o]
.rodata:080185A0 kqfvip dd offset _2__STRING_11126_0 ; DATA XREF: kqfgvcn+18
.rodata:080185A0                                ; kqfgvt+F
.rodata:080185A0                                ; "select inst_id,decode(indx,1,'data bloc"...
.rodata:080185A4        dd offset kqfv459_c_0
.rodata:080185A8        dd 0
.rodata:080185AC        dd 0

...

.rodata:08019570        dd offset _2__STRING_11378_0 ; "select  BANNER from GV$VERSION where in"...
.rodata:08019574        dd offset kqfv133_c_0
.rodata:08019578        dd 0
.rodata:0801957C        dd 0
.rodata:08019580        dd offset _2__STRING_11379_0 ; "select inst_id,decode(bitand(cfflg,1),0"...
.rodata:08019584        dd offset kqfv403_c_0
.rodata:08019588        dd 0
.rodata:0801958C        dd 0
.rodata:08019590        dd offset _2__STRING_11380_0 ; "select  STATUS , NAME, IS_RECOVERY_DEST"...
.rodata:08019594        dd offset kqfv199_c_0
\end{lstlisting}

\RU{Таблица, по всей видимости, имеет 4 поля в каждом элементе. Кстати, здесь так же 1023 элемента, уже знакомое нам число.}
\EN{The table appear to have 4 fields in each element. By the way, there are 1023 elements in it, again, the number we already know.}
\RU{Второе поле указывает на другую таблицу, содержащую поля этого \IT{fixed view}.}
\EN{The second field points to another table that contains the table fields for this \IT{fixed view}.}
\RU{Для \TT{V\$VERSION}, эта таблица только из двух элементов, первый это 6 и второй это строка 
\TT{BANNER} (число 6 это длина строки) и далее \IT{терминирующий} элемент содержащий 0 и \IT{нулевую} 
Си-строку:}\EN{As for \TT{V\$VERSION}, this table contains only two elements, the first is 6 and the second is 
the \TT{BANNER} string (the number 6 is this string's length) and after, a \IT{terminating} element that contains 
0 and a \IT{null} C string:}

\begin{lstlisting}[caption=kqf.o]
.rodata:080BBAC4 kqfv133_c_0 dd 6     ; DATA XREF: .rodata:08019574
.rodata:080BBAC8             dd offset _2__STRING_5017_0 ; "BANNER"
.rodata:080BBACC             dd 0
.rodata:080BBAD0             dd offset _2__STRING_0_0
\end{lstlisting}

\RU{Объединив данные из таблиц \TT{kqfviw} и \TT{kqfvip}, мы получим SQL-запросы, которые исполняются, когда пользователь хочет получить информацию из какого-либо \IT{fixed view}.}\EN{By joining data from both \TT{kqfviw} and \TT{kqfvip} tables, we can get the SQL statements which are executed when the user wants to query information from a specific \IT{fixed view}.}

\RU{Напишем программу \oracletables, которая собирает всю эту информацию из объектных файлов от \oracle под Linux.}%
\EN{So we can write an \oracletables program, to gather all this information from \oracle for Linux's object files.}
\RU{Для \TT{V\$VERSION}, мы можем найти следующее:}\EN{For \TT{V\$VERSION}, we find this:}

\begin{lstlisting}[caption=\RU{Результат работы}\EN{Result of} \OracleTablesName]
kqfviw_element.viewname: [V$VERSION] ?: 0x3 0x43 0x1 0xffffc085 0x4
kqfvip_element.statement: [select  BANNER from GV$VERSION where inst_id = USERENV('Instance')]
kqfvip_element.params:
[BANNER] 
\end{lstlisting}

\AndENRU:

\begin{lstlisting}[caption=\RU{Результат работы}\EN{Result of} \OracleTablesName]
kqfviw_element.viewname: [GV$VERSION] ?: 0x3 0x26 0x2 0xffffc192 0x1
kqfvip_element.statement: [select inst_id, banner from x$version]
kqfvip_element.params:
[INST_ID] [BANNER] 
\end{lstlisting}

\RU{\IT{Fixed view} \TT{GV\$VERSION} отличается от \TT{V\$VERSION} тем, что содержит еще и поле отражающее идентификатор \IT{instance}.}
\EN{The \TT{GV\$VERSION} \IT{fixed view} is different from \TT{V\$VERSION} only in that it contains one more field with the identifier \IT{instance}.}
\RU{Но так или иначе, мы теперь упираемся в таблицу \TT{X\$VERSION}. Как и прочие X\$-таблицы, она не документирована, однако, мы можем оттуда что-то прочитать}
\EN{Anyway, we are going to stick with the \TT{X\$VERSION} table. Just like any other X\$-table, it is undocumented, however, we can query it}:

\begin{lstlisting}
SQL> select * from x$version;

ADDR           INDX    INST_ID
-------- ---------- ----------
BANNER
------------------------------

0DBAF574          0          1
Oracle Database 11g Enterprise Edition Release 11.2.0.1.0 - Production

...
\end{lstlisting}

\RU{Эта таблица содержит дополнительные поля вроде \TT{ADDR} и \TT{INDX}.}\EN{This table has some additional fields, like \TT{ADDR} and \TT{INDX}.}

\RU{Бегло листая содержимое файла \TT{kqf.o} в \IDA мы можем увидеть еще одну таблицу где есть ссылка на строку \TT{X\$VERSION}, это \TT{kqftab}:}\EN{While scrolling \TT{kqf.o} in \IDA we can spot another table that contains a pointer to the \TT{X\$VERSION} string, this is \TT{kqftab}:}

\begin{lstlisting}[caption=kqf.o]
.rodata:0803CAC0      dd 9                    ; element number 0x1f6
.rodata:0803CAC4      dd offset _2__STRING_13113_0 ; "X$VERSION"
.rodata:0803CAC8      dd 4
.rodata:0803CACC      dd offset _2__STRING_13114_0 ; "kqvt"
.rodata:0803CAD0      dd 4
.rodata:0803CAD4      dd 4
.rodata:0803CAD8      dd 0
.rodata:0803CADC      dd 4
.rodata:0803CAE0      dd 0Ch
.rodata:0803CAE4      dd 0FFFFC075h
.rodata:0803CAE8      dd 3
.rodata:0803CAEC      dd 0
.rodata:0803CAF0      dd 7
.rodata:0803CAF4      dd offset _2__STRING_13115_0 ; "X$KQFSZ"
.rodata:0803CAF8      dd 5
.rodata:0803CAFC      dd offset _2__STRING_13116_0 ; "kqfsz"
.rodata:0803CB00      dd 1
.rodata:0803CB04      dd 38h
.rodata:0803CB08      dd 0
.rodata:0803CB0C      dd 7
.rodata:0803CB10      dd 0
.rodata:0803CB14      dd 0FFFFC09Dh
.rodata:0803CB18      dd 2
.rodata:0803CB1C      dd 0
\end{lstlisting}

\RU{Здесь очень много ссылок на названия X\$-таблиц, вероятно, на все те что имеются в \oracle этой версии.}
\EN{There are a lot of references to the X\$-table names, apparently, to all \oracle 11.2 X\$-tables.}
\RU{Но мы снова упираемся в то что не имеем достаточно информации.}\EN{But again, we don't have enough information.}
\RU{Не ясно, что означает строка \TT{kqvt}.}
\EN{It's not clear what does the \TT{kqvt} string stands for.} 
\RU{Вообще, префикс \TT{kq} может означать \IT{kernel} и \IT{query}.}
\EN{The \TT{kq} prefix may mean \IT{kernel} or \IT{query}.} 
\RU{\TT{v}, может быть, \IT{version}, а \TT{t}\EMDASH{}\IT{type}?}
\EN{\TT{v} apparently stands for \IT{version} and \TT{t}\EMDASH{}\IT{type}?} 
\RU{Сказать трудно.}\EN{Hard to say.}

\RU{Таблицу с очень похожим названием мы можем найти в}\EN{A table with a similar name can be found in} \TT{kqf.o}:

\begin{lstlisting}[caption=kqf.o]
.rodata:0808C360 kqvt_c_0 kqftap_param <4, offset _2__STRING_19_0, 917h, 0, 0, 0, 4, 0, 0>
.rodata:0808C360                                  ; DATA XREF: .rodata:08042680
.rodata:0808C360                                  ; "ADDR"
.rodata:0808C384          kqftap_param <4, offset _2__STRING_20_0, 0B02h, 0, 0, 0, 4, 0, 0> ; "INDX"
.rodata:0808C3A8          kqftap_param <7, offset _2__STRING_21_0, 0B02h, 0, 0, 0, 4, 0, 0> ; "INST_ID"
.rodata:0808C3CC          kqftap_param <6, offset _2__STRING_5017_0, 601h, 0, 0, 0, 50h, 0, 0> ; "BANNER"
.rodata:0808C3F0          kqftap_param <0, offset _2__STRING_0_0, 0, 0, 0, 0, 0, 0, 0>
\end{lstlisting}

\RU{Она содержит информацию об именах полей в таблице \TT{X\$VERSION}.}\EN{It contains information about all fields in the \TT{X\$VERSION} table.}
\RU{Единственная ссылка на эту таблицу имеется в таблице \TT{kqftap}:}\EN{The only reference to this table is in the \TT{kqftap} table:}

\begin{lstlisting}[caption=kqf.o]
.rodata:08042680                 kqftap_element <0, offset kqvt_c_0, offset kqvrow, 0> ; element 0x1f6
\end{lstlisting}

\RU{Интересно что здесь этот элемент проходит так же под номером \TT{0x1f6} (502-й), как и ссылка на строку 
\TT{X\$VERSION} в таблице \TT{kqftab}.}
\EN{It is interesting that this element here is \TT{0x1f6th} (502nd), just like the pointer to the \TT{X\$VERSION} string in 
the \TT{kqftab} table.}
\RU{Вероятно, таблицы \TT{kqftap} и \TT{kqftab} дополняют друг друга, как и \TT{kqfvip} и \TT{kqfviw}.}
\EN{Probably, the \TT{kqftap} and \TT{kqftab} tables complement each other, just like \TT{kqfvip} and \TT{kqfviw}.}
\RU{Мы также видим здесь ссылку на функцию с названием \TT{kqvrow()}. А вот это уже кое-что!}
\EN{We also see a pointer to the \TT{kqvrow()} function. Finally, we got something useful!}

\RU{Сделаем так чтобы наша программа \oracletables могла дампить и эти таблицы. Для \TT{X\$VERSION} получается:}
\EN{So we will add these tables to our \oracletables utility too. For \TT{X\$VERSION} we get:}

\begin{lstlisting}[caption=\RU{Результат работы}\EN{Result of} \OracleTablesName]
kqftab_element.name: [X$VERSION] ?: [kqvt] 0x4 0x4 0x4 0xc 0xffffc075 0x3
kqftap_param.name=[ADDR] ?: 0x917 0x0 0x0 0x0 0x4 0x0 0x0
kqftap_param.name=[INDX] ?: 0xb02 0x0 0x0 0x0 0x4 0x0 0x0
kqftap_param.name=[INST_ID] ?: 0xb02 0x0 0x0 0x0 0x4 0x0 0x0
kqftap_param.name=[BANNER] ?: 0x601 0x0 0x0 0x0 0x50 0x0 0x0
kqftap_element.fn1=kqvrow
kqftap_element.fn2=NULL
\end{lstlisting}

\myindex{tracer}
\RU{При помощи \tracer, можно легко проверить, что эта функция вызывается 6 раз кряду (из функции \TT{qerfxFetch()}) при получении строк из \TT{X\$VERSION}.}\EN{With the help of \tracer, it is easy to check that this function is called 6 times in row (from the \TT{qerfxFetch()} function) while querying the \TT{X\$VERSION} table.}

\RU{Запустим \tracer в режиме \TT{cc} (он добавит комментарий к каждой исполненной инструкции):}
\EN{Let's run \tracer in \TT{cc} mode (it comments each executed instruction):}

\begin{lstlisting}
tracer -a:oracle.exe bpf=oracle.exe!_kqvrow,trace:cc
\end{lstlisting}

\lstinputlisting{examples/oracle/VERSION_kqvrow.asm}

\RU{Так можно легко увидеть, что номер строки таблицы задается извне. Сама функция возвращает строку, формируя её так:}
\EN{Now it is easy to see that the row number is passed from outside. The function returns the string, constructing it as follows:}

\begin{center}
\begin{tabular}{ | l | l | }
\hline                        
\RU{Строка}\EN{String} 1	& \RU{Использует глобальные переменные \TT{vsnstr}, \TT{vsnnum}, \TT{vsnban}.}\EN{Using \TT{vsnstr}, \TT{vsnnum}, \TT{vsnban} global variables.} \\
                                & \RU{Вызывает \TT{sprintf()}.}\EN{Calls \TT{sprintf()}.} \\
\RU{Строка}\EN{String} 2	& \RU{Вызывает}\EN{Calls} \TT{kkxvsn()}. \\
\RU{Строка}\EN{String} 3	& \RU{Вызывает}\EN{Calls} \TT{lmxver()}. \\
\RU{Строка}\EN{String} 4	& \RU{Вызывает}\EN{Calls} \TT{npinli()}, \TT{nrtnsvrs()}. \\
\RU{Строка}\EN{String} 5	& \RU{Вызывает}\EN{Calls} \TT{lxvers()}. \\
\hline  
\end{tabular}
\end{center}

\RU{Так вызываются соответствующие функции для определения номеров версий отдельных модулей.}
\EN{That's how the corresponding functions are called for determining each module's version.}

