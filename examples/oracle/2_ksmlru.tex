\sectionold{\RU{Таблица \TT{X\$KSMLRU} в}\EN{\TT{X\$KSMLRU} table in} \oracle}
\myindex{\oracle}

\RU{В заметке}\EN{There is a mention of a special table in the} \IT{Diagnosing and Resolving Error ORA-04031 
on the Shared Pool or Other Memory Pools [Video] [ID 146599.1]} \RU{упоминается некая служебная таблица}\EN{note}:

\begin{framed}
\begin{quotation}
There is a fixed table called X\$KSMLRU that tracks allocations in the shared pool that cause other objects 
in the shared pool to be aged out. This fixed table can be used to identify what is causing the large allocation.

If many objects are being periodically flushed from the shared pool then this will cause response time problems 
and will likely cause library cache latch contention problems when the objects are reloaded into the shared pool.

One unusual thing about the X\$KSMLRU fixed table is that the contents of the fixed table are erased whenever 
someone selects from the fixed table. This is done since the fixed table stores only the largest allocations 
that have occurred. The values are reset after being selected so that subsequent large allocations can be noted 
even if they were not quite as large as others that occurred previously. Because of this resetting, the output 
of selecting from this table should be carefully kept since it cannot be retrieved back after the query is issued.
\end{quotation}
\end{framed}

\RU{Однако, как можно легко убедиться, эта системная таблица очищается всякий раз, когда кто-то делает запрос 
к ней.}\EN{However, as it can be easily checked, the contents of this table are cleared each time it's queried.}
\RU{Сможем ли мы найти причину, почему это происходит?}\EN{Are we able to find why?}
\RU{Если вернуться к уже рассмотренным таблицам \TT{kqftab} и \TT{kqftap} полученных при помощи \oracletables, 
содержащим информацию о X\$-таблицах, мы узнаем что для того чтобы подготовить строки этой таблицы, 
вызывается функция \TT{ksmlrs()}:}\EN{Let's get back to tables we already know: \TT{kqftab} and \TT{kqftap} 
which were generated with \oracletables's help, that contain all information about the X\$-tables. We can see here 
that the \TT{ksmlrs()} function is called to prepare this table's elements:}

\begin{lstlisting}[caption=\RU{Результат работы}\EN{Result of} \OracleTablesName]
kqftab_element.name: [X$KSMLRU] ?: [ksmlr] 0x4 0x64 0x11 0xc 0xffffc0bb 0x5
kqftap_param.name=[ADDR] ?: 0x917 0x0 0x0 0x0 0x4 0x0 0x0
kqftap_param.name=[INDX] ?: 0xb02 0x0 0x0 0x0 0x4 0x0 0x0
kqftap_param.name=[INST_ID] ?: 0xb02 0x0 0x0 0x0 0x4 0x0 0x0
kqftap_param.name=[KSMLRIDX] ?: 0xb02 0x0 0x0 0x0 0x4 0x0 0x0
kqftap_param.name=[KSMLRDUR] ?: 0xb02 0x0 0x0 0x0 0x4 0x4 0x0
kqftap_param.name=[KSMLRSHRPOOL] ?: 0xb02 0x0 0x0 0x0 0x4 0x8 0x0
kqftap_param.name=[KSMLRCOM] ?: 0x501 0x0 0x0 0x0 0x14 0xc 0x0
kqftap_param.name=[KSMLRSIZ] ?: 0x2 0x0 0x0 0x0 0x4 0x20 0x0
kqftap_param.name=[KSMLRNUM] ?: 0x2 0x0 0x0 0x0 0x4 0x24 0x0
kqftap_param.name=[KSMLRHON] ?: 0x501 0x0 0x0 0x0 0x20 0x28 0x0
kqftap_param.name=[KSMLROHV] ?: 0xb02 0x0 0x0 0x0 0x4 0x48 0x0
kqftap_param.name=[KSMLRSES] ?: 0x17 0x0 0x0 0x0 0x4 0x4c 0x0
kqftap_param.name=[KSMLRADU] ?: 0x2 0x0 0x0 0x0 0x4 0x50 0x0
kqftap_param.name=[KSMLRNID] ?: 0x2 0x0 0x0 0x0 0x4 0x54 0x0
kqftap_param.name=[KSMLRNSD] ?: 0x2 0x0 0x0 0x0 0x4 0x58 0x0
kqftap_param.name=[KSMLRNCD] ?: 0x2 0x0 0x0 0x0 0x4 0x5c 0x0
kqftap_param.name=[KSMLRNED] ?: 0x2 0x0 0x0 0x0 0x4 0x60 0x0
kqftap_element.fn1=ksmlrs
kqftap_element.fn2=NULL
\end{lstlisting}

\myindex{tracer}
\RU{Действительно, при помощи \tracer легко убедиться, что эта функция вызывается каждый раз, когда мы обращаемся 
к таблице \TT{X\$KSMLRU}.}\EN{Indeed, with \tracer's help it is easy to see that this function is called each 
time we query the \TT{X\$KSMLRU} table.}

\myindex{\CStandardLibrary!memset()}
\RU{Здесь есть ссылки на функции \TT{ksmsplu\_sp()} и \TT{ksmsplu\_jp()}, каждая из которых в итоге вызывает 
\TT{ksmsplu()}.}\EN{Here we see a references to the \TT{ksmsplu\_sp()} and \TT{ksmsplu\_jp()} functions, each of 
them calls the \TT{ksmsplu()} in the end.}
\RU{В конце функции \TT{ksmsplu()} мы видим вызов \TT{memset()}:}\EN{At the end of the \TT{ksmsplu()} function we see 
a call to \TT{memset()}:}

\begin{lstlisting}[caption=ksm.o]
...

.text:00434C50 loc_434C50:                             ; DATA XREF: .rdata:off_5E50EA8
.text:00434C50                 mov     edx, [ebp-4]
.text:00434C53                 mov     [eax], esi
.text:00434C55                 mov     esi, [edi]
.text:00434C57                 mov     [eax+4], esi
.text:00434C5A                 mov     [edi], eax
.text:00434C5C                 add     edx, 1
.text:00434C5F                 mov     [ebp-4], edx
.text:00434C62                 jnz     loc_434B7D
.text:00434C68                 mov     ecx, [ebp+14h]
.text:00434C6B                 mov     ebx, [ebp-10h]
.text:00434C6E                 mov     esi, [ebp-0Ch]
.text:00434C71                 mov     edi, [ebp-8]
.text:00434C74                 lea     eax, [ecx+8Ch]
.text:00434C7A                 push    370h            ; Size
.text:00434C7F                 push    0               ; Val
.text:00434C81                 push    eax             ; Dst
.text:00434C82                 call    __intel_fast_memset
.text:00434C87                 add     esp, 0Ch
.text:00434C8A                 mov     esp, ebp
.text:00434C8C                 pop     ebp
.text:00434C8D                 retn
.text:00434C8D _ksmsplu        endp
\end{lstlisting}

\RU{Такие конструкции (\TT{memset (block, 0, size)}) очень часто используются для простого обнуления блока памяти.}\EN{Constructions like \TT{memset (block, 0, size)} are often used just to zero memory block.}
\RU{Мы можем попробовать рискнуть, заблокировав вызов \TT{memset()} и посмотреть, что будет?}\EN{What if we take a risk, block the \TT{memset()} call and see what happens?}

\myindex{tracer}
\RU{Запускаем \tracer со следующей опцией: поставить точку останова на \TT{0x434C7A} 
(там, где начинается передача параметров для функции \TT{memset()}) так, 
чтобы \tracer в этом месте установил указатель инструкций процессора (\TT{EIP}) на место, где уже произошла очистка переданных параметров в \TT{memset()} (по адресу \TT{0x434C8A}):}
\EN{Let's run \tracer with the following options: set breakpoint at \TT{0x434C7A} 
(the point where the arguments to \TT{memset()} are to be passed), 
so that \tracer will set program counter \TT{EIP} to the point where the arguments passed to \TT{memset()} are to be cleared (at \TT{0x434C8A})}
\RU{Можно сказать, при помощи этого, мы симулируем безусловный переход с адреса}\EN{It can be said that we just simulate an unconditional jump from address} \TT{0x434C7A} \RU{на}\EN{to} \TT{0x434C8A}.

\begin{lstlisting}
tracer -a:oracle.exe bpx=oracle.exe!0x00434C7A,set(eip,0x00434C8A)
\end{lstlisting}

\RU{(Важно: все эти адреса справедливы только для win32-версии \oracle 11.2)}\EN{(Important: all these addresses are valid only for the win32 version of \oracle 11.2)}

\RU{Действительно, после этого мы можем обращаться к таблице \TT{X\$KSMLRU} сколько угодно, и она уже не очищается!}\EN{Indeed, now we can query the \TT{X\$KSMLRU} table as many times as we want and it is not being cleared anymore!}

\RU{\sout{Не делайте этого дома ("Разрушители легенд")} Не делайте этого на своих production-серверах.}
\EN{\sout{Do not try this at home ("MythBusters")} Do not try this on your production servers.}

\RU{Впрочем, это не обязательно полезное или желаемое поведение системы, но как эксперимент по поиску нужного кода, нам это подошло!}
\EN{It is probably not a very useful or desired system behaviour, but as an experiment for locating a piece of code that we need, it perfectly suits our needs!}

