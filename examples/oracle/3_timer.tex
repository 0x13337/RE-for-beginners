\subsection{\IFRU{Таблица \TT{V\$TIMER} в}{\TT{V\$TIMER} table in} \oracle}
\index{Oracle RDBMS}

\TT{V\$TIMER} \IFRU{это еще один служебный \IT{fixed view}, отражающий какое-то часто меняющееся значение:}
{is another \IT{fixed view}, reflecting some rapidly changing value:}

\begin{framed}
\begin{quotation}
V\$TIMER displays the elapsed time in hundredths of a second. Time is measured since the beginning of the epoch, 
which is operating system specific, and wraps around to 0 again whenever the value overflows four bytes 
(roughly 497 days).
\end{quotation}
\end{framed}(\IFRU{Из документации \oracle}{From \oracle documentation}
\footnote{\url{http://docs.oracle.com/cd/B28359_01/server.111/b28320/dynviews_3104.htm}})

\IFRU{Интересно что периоды разные в Oracle для Win32 и для Linux. Сможем ли мы найти функцию, отвечающую 
за генерирование этого значения?}{It is interesting that periods are different for Oracle for win32 and for Linux. 
Will we able to find a function generating this value?}

\IFRU{Как видно, эта информация, в итоге, берется из системной таблицы \TT{X\$KSUTM}.}{As we can see, 
this information is finally taken from \TT{X\$KSUTM} table.}

\begin{lstlisting}
SQL> select * from V$FIXED_VIEW_DEFINITION where view_name='V$TIMER';

VIEW_NAME
------------------------------
VIEW_DEFINITION
--------------------------------------------------------------------------------

V$TIMER
select  HSECS from GV$TIMER where inst_id = USERENV('Instance')

SQL> select * from V$FIXED_VIEW_DEFINITION where view_name='GV$TIMER';

VIEW_NAME
------------------------------
VIEW_DEFINITION
--------------------------------------------------------------------------------

GV$TIMER
select inst_id,ksutmtim from x$ksutm
\end{lstlisting}

\IFRU{Здесь мы упираемся в небольшую проблему, в таблицах \TT{kqftab}/\TT{kqftap} нет указателей на функцию, 
которая бы генерировала значение}
{Now we stuck in a small problem, there are no references to value generating function(s) 
in the tables \TT{kqftab}/\TT{kqftap}}:

\begin{lstlisting}[caption=\IFRU{Результат работы}{Result of} \OracleTablesName]
kqftab_element.name: [X$KSUTM] ?: [ksutm] 0x1 0x4 0x4 0x0 0xffffc09b 0x3
kqftap_param.name=[ADDR] ?: 0x10917 0x0 0x0 0x0 0x4 0x0 0x0
kqftap_param.name=[INDX] ?: 0x20b02 0x0 0x0 0x0 0x4 0x0 0x0
kqftap_param.name=[INST_ID] ?: 0xb02 0x0 0x0 0x0 0x4 0x0 0x0
kqftap_param.name=[KSUTMTIM] ?: 0x1302 0x0 0x0 0x0 0x4 0x0 0x1e
kqftap_element.fn1=NULL
kqftap_element.fn2=NULL
\end{lstlisting}

\IFRU{Попробуем в таком случае просто поискать строку \TT{KSUTMTIM}, и находим ссылку на нее в такой функции:}
{Let's try to find a string \TT{KSUTMTIM}, and we find it in this function:}

\begin{lstlisting}
kqfd_DRN_ksutm_c proc near              ; DATA XREF: .rodata:0805B4E8

arg_0           = dword ptr  8
arg_8           = dword ptr  10h
arg_C           = dword ptr  14h

                push    ebp
                mov     ebp, esp
                push    [ebp+arg_C]
                push    offset ksugtm
                push    offset _2__STRING_1263_0 ; "KSUTMTIM"
                push    [ebp+arg_8]
                push    [ebp+arg_0]
                call    kqfd_cfui_drain
                add     esp, 14h
                mov     esp, ebp
                pop     ebp
                retn
kqfd_DRN_ksutm_c endp
\end{lstlisting}

\IFRU{Сама ф-ция}{The function} \TT{kqfd\_DRN\_ksutm\_c()} \IFRU{упоминается в таблице}{is mentioned in} 
\TT{kqfd\_tab\_registry\_0} \IFRU{вот так}{table}:

\begin{lstlisting}
dd offset _2__STRING_62_0 ; "X$KSUTM"
dd offset kqfd_OPN_ksutm_c
dd offset kqfd_tabl_fetch
dd 0
dd 0
dd offset kqfd_DRN_ksutm_c
\end{lstlisting}

\IFRU{Упоминается также некая ф-ция \TT{ksugtm()}}
{There are is a function \TT{ksugtm()} referenced here}.
\IFRU{Посмотрим, что там (в Linux x86)}{Let's see what's in it (Linux x86)}:

\begin{lstlisting}[caption=ksu.o]
ksugtm          proc near

var_1C          = byte ptr -1Ch
arg_4           = dword ptr  0Ch

                push    ebp
                mov     ebp, esp
                sub     esp, 1Ch
                lea     eax, [ebp+var_1C]
                push    eax
                call    slgcs
                pop     ecx
                mov     edx, [ebp+arg_4]
                mov     [edx], eax
                mov     eax, 4
                mov     esp, ebp
                pop     ebp
                retn
ksugtm          endp
\end{lstlisting}

\IFRU{В win32-версии тоже самое}{Almost the same code in win32-version}.

\IFRU{Искомая ли эта функция? Попробуем узнать}{Is this the function we are looking for? Let's see}:

\begin{lstlisting}
tracer -a:oracle.exe bpf=oracle.exe!_ksugtm,args:2,dump_args:0x4
\end{lstlisting}

\IFRU{Пробуем несколько раз}{Let's try again}:

\begin{lstlisting}
SQL> select * from V$TIMER;

     HSECS
----------
  27294929

SQL> select * from V$TIMER;

     HSECS
----------
  27295006

SQL> select * from V$TIMER;

     HSECS
----------
  27295167
\end{lstlisting}

\begin{lstlisting}[caption=\IFRU{вывод \tracer}{\tracer output}]
TID=2428|(0) oracle.exe!_ksugtm (0x0, 0xd76c5f0) (called from oracle.exe!__VInfreq__qerfxFetch+0xfad (0x56bb6d5))
Argument 2/2
0D76C5F0: 38 C9                                           "8.              "
TID=2428|(0) oracle.exe!_ksugtm () -> 0x4 (0x4)
Argument 2/2 difference
00000000: D1 7C A0 01                                     ".|..            "
TID=2428|(0) oracle.exe!_ksugtm (0x0, 0xd76c5f0) (called from oracle.exe!__VInfreq__qerfxFetch+0xfad (0x56bb6d5))
Argument 2/2
0D76C5F0: 38 C9                                           "8.              "
TID=2428|(0) oracle.exe!_ksugtm () -> 0x4 (0x4)
Argument 2/2 difference
00000000: 1E 7D A0 01                                     ".}..            "
TID=2428|(0) oracle.exe!_ksugtm (0x0, 0xd76c5f0) (called from oracle.exe!__VInfreq__qerfxFetch+0xfad (0x56bb6d5))
Argument 2/2
0D76C5F0: 38 C9                                           "8.              "
TID=2428|(0) oracle.exe!_ksugtm () -> 0x4 (0x4)
Argument 2/2 difference
00000000: BF 7D A0 01                                     ".}..            "
\end{lstlisting}

\IFRU{Действительно ~--- значение то, что мы видим в SQL*Plus, и оно возвращается через второй аргумент}
{Indeed ~--- the value is the same we see in SQL*Plus and it is returning via second argument}.

\IFRU{Посмотрим что в ф-ции}{Let's see what is in} \TT{slgcs()} (Linux x86):

\begin{lstlisting}
slgcs           proc near

var_4           = dword ptr -4
arg_0           = dword ptr  8

                push    ebp
                mov     ebp, esp
                push    esi
                mov     [ebp+var_4], ebx
                mov     eax, [ebp+arg_0]
                call    $+5
                pop     ebx
                nop                     ; PIC mode
                mov     ebx, offset _GLOBAL_OFFSET_TABLE_
                mov     dword ptr [eax], 0
                call    sltrgatime64    ; PIC mode
                push    0
                push    0Ah
                push    edx
                push    eax
                call    __udivdi3       ; PIC mode
                mov     ebx, [ebp+var_4]
                add     esp, 10h
                mov     esp, ebp
                pop     ebp
                retn
slgcs           endp
\end{lstlisting}

(\IFRU{это просто вызов}{it is just a call to} \TT{sltrgatime64()} \IFRU{и деление его результата на}
{and division of its result by} 10~\ref{sec:divisionbynine})

\IFRU{И в win32-версии}{And win32-version}:

\begin{lstlisting}
_slgcs          proc near               ; CODE XREF: _dbgefgHtElResetCount+15
                                        ; _dbgerRunActions+1528
                db      66h
                nop
                push    ebp
                mov     ebp, esp
                mov     eax, [ebp+8]
                mov     dword ptr [eax], 0
                call    ds:__imp__GetTickCount@0 ; GetTickCount()
                mov     edx, eax
                mov     eax, 0CCCCCCCDh
                mul     edx
                shr     edx, 3
                mov     eax, edx
                mov     esp, ebp
                pop     ebp
                retn
_slgcs          endp
\end{lstlisting}

\IFRU{Это просто результат}{It is just result of} \TT{GetTickCount()
\footnote{\url{http://msdn.microsoft.com/en-us/library/windows/desktop/ms724408(v=vs.85).aspx}}} 
\IFRU{поделенный на}{divided by} 10~\ref{sec:divisionbynine}.

\IFRU{Вуаля! Вот почему в win32-версии и версии Linux x86 разные результаты, потому что они получаются разными 
системными ф-циями \ac{OS}.}{Voilà! That's why win32-version and Linux x86 version show different results, 
just because they are generated by different \ac{OS} functions.}

\IFRU{\IT{Drain} по английски дренаж, отток, водосток. Таким образом, возможно имеется ввиду \IT{подключение} 
определенного столбца системной таблице к функции.}{\IT{Drain} apparently means \IT{connecting} specific 
table column to specific function.}

\IFRU{Я добавил поддержку таблицы}{I added the table} \TT{kqfd\_tab\_registry\_0} \IFRU{в}{to} \oracletables, 
\IFRU{теперь мы можем видеть, при помощи каких ф-ций, столбцы в системных таблицах \IT{подключаются} к значениям, 
например}{now we can see, how table column's variables are \IT{connected} to specific functions}:

\begin{lstlisting}
[X$KSUTM] [kqfd_OPN_ksutm_c] [kqfd_tabl_fetch] [NULL] [NULL] [kqfd_DRN_ksutm_c]
[X$KSUSGIF] [kqfd_OPN_ksusg_c] [kqfd_tabl_fetch] [NULL] [NULL] [kqfd_DRN_ksusg_c]
\end{lstlisting}

\IT{OPN}, \IFRU{возможно}{apparently}, \IT{open}, \IFRU{а}{and} \IT{DRN}, \IFRU{вероятно, означает}
{apparently, meaning} \IT{drain}.

