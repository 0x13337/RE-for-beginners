\section{Task manager practical joke (Windows Vista)}
\myindex{Windows!Windows Vista}

Let's see if it's possible to hack Task Manager slightly so it would detect more \ac{CPU} cores.

\myindex{Windows!NTAPI}

Let us first think, how does the Task Manager know the number of cores?

There is the \TT{GetSystemInfo()} win32 function present in win32 userspace which can tell us this.
But it's not imported in \TT{taskmgr.exe}.

There is, however, another one in \gls{NTAPI}, \TT{NtQuerySystemInformation()}, 
which is used in \TT{taskmgr.exe} in several places.

To get the number of cores, one has to call this function with the \TT{SystemBasicInformation} constant
as a first argument (which is zero
\footnote{\href{http://go.yurichev.com/17251}{MSDN}}).

The second argument has to point to the buffer which is getting all the information.

So we have to find all calls to the \\
\TT{NtQuerySystemInformation(0, ?, ?, ?)} function.
Let's open \TT{taskmgr.exe} in IDA. 
\myindex{Windows!PDB}

What is always good about Microsoft executables is that IDA can download the corresponding \gls{PDB} 
file for this executable and show all function names.

It is visible that Task Manager is written in \Cpp and some of the function names and classes are really 
speaking for themselves.
There are classes CAdapter, CNetPage, CPerfPage, CProcInfo, CProcPage, CSvcPage, 
CTaskPage, CUserPage.

Apparently, each class corresponds to each tab in Task Manager.

Let's visit each call and add comment with the value which is passed as the first function argument.
We will write \q{not zero} at some places, because the value there was clearly not zero, 
but something really different (more about this in the second part of this chapter).

And we are looking for zero passed as argument, after all.

\begin{figure}[H]
\centering
\myincludegraphics{examples/taskmgr/IDA_xrefs.png}
\caption{IDA: cross references to NtQuerySystemInformation()}
\end{figure}

Yes, the names are really speaking for themselves.

When we closely investigate each place where\\
\TT{NtQuerySystemInformation(0, ?, ?, ?)} is called,
we quickly find what we need in the \TT{InitPerfInfo()} function:

\lstinputlisting[caption=taskmgr.exe (Windows Vista)]{examples/taskmgr/taskmgr.lst}

\TT{g\_cProcessors} is a global variable, and this name was assigned by 
IDA according to the \gls{PDB} loaded from Microsoft's symbol server.

The byte is taken from \TT{var\_C20}. 
And \TT{var\_C58} is passed to\\
\TT{NtQuerySystemInformation()} 
as a pointer to the receiving buffer.
The difference between 0xC20 and 0xC58 is 0x38 (56).

Let's take a look at format of the return structure, which we can find in MSDN:

\begin{lstlisting}
typedef struct _SYSTEM_BASIC_INFORMATION {
    BYTE Reserved1[24];
    PVOID Reserved2[4];
    CCHAR NumberOfProcessors;
} SYSTEM_BASIC_INFORMATION;
\end{lstlisting}

This is a x64 system, so each PVOID takes 8 byte.

All \IT{reserved} fields in the structure take $24+4*8=56$ bytes.

Oh yes, this implies that \TT{var\_C20} is the local stack is exactly the
\TT{NumberOfProcessors} field of the \TT{SYSTEM\_BASIC\_INFORMATION} structure.

Let's check our guess.
Copy \TT{taskmgr.exe} from \TT{C:\textbackslash{}Windows\textbackslash{}System32} 
to some other folder 
(so the \IT{Windows Resource Protection} 
will not try to restore the patched \TT{taskmgr.exe}).

Let's open it in Hiew and find the place:

\begin{figure}[H]
\centering
\myincludegraphics{examples/taskmgr/hiew2.png}
\caption{Hiew: find the place to be patched}
\end{figure}

Let's replace the \TT{MOVZX} instruction with ours.
Let's pretend we've got 64 CPU cores.

Add one additional \ac{NOP} (because our instruction is shorter than the original one):

\begin{figure}[H]
\centering
\myincludegraphics{examples/taskmgr/hiew1.png}
\caption{Hiew: patch it}
\end{figure}

And it works!
Of course, the data in the graphs is not correct.

At times, Task Manager even shows an overall CPU load of more than 100\%.

\begin{figure}[H]
\centering
\myincludegraphics{examples/taskmgr/taskmgr_64cpu_crop.png}
\caption{Fooled Windows Task Manager}
\end{figure}

The biggest number Task Manager does not crash with is 64.

Apparently, Task Manager in Windows Vista was not tested on computers with a large number of cores.

So there are probably some static data structure(s) inside it limited to 64 cores.

\subsection{Using LEA to load values}
\label{TaskMgr_LEA}

Sometimes, \TT{LEA} is used in \TT{taskmgr.exe} instead of \TT{MOV} to set the first argument of \\
\TT{NtQuerySystemInformation()}:

\lstinputlisting[caption=taskmgr.exe (Windows Vista)]{examples/taskmgr/taskmgr2.lst}

\myindex{x86!\Instructions!LEA}

Perhaps \ac{MSVC} did so because machine code of \INS{LEA} is shorter than \INS{MOV REG, 5} (would be 5 instead of 4).

\INS{LEA} with offset in $-128..127$ range (offset will occupy 1 byte in opcode) with 32-bit registers is even shorter (for lack of REX pregix)---3 bytes.

Another example of such thing is: \myref{using_MOV_and_pack_of_LEA_to_load_values}.
