\chapter{\RU{Ручная декомпиляция + использование SMT-солвера Z3}
\EN{Hand decompiling + Z3 SMT solver}}
\index{Z3}

\RU{Любительская криптография обычно (непреднамеренно) очень слабая и может быть легко сломана ---
для криптографов, конечно}\EN{Amateur cryptography is usually (unintentionally) 
very weak and can be broken easily---for cryptographers, of course}.

\RU{Но представим на время что мы не в числе этих профессионалов}
\EN{But let's pretend we are not among these crypto-professionals}.

\RU{Я нашел эту необратимую хэш-функцию (читайте больше о них: \myref{hash_func}), 
которая конвертирует одно 64-битное значение в другое,
и нам нужно попытаться развернуть её работу назад.}
\EN{I once found this one-way hash function (read more about them: \myref{hash_func}),
that converted a 64-bit value to another and we need to try
to reverse its flow back.}

\section{\RU{Ручная декомпиляция}\EN{Hand decompiling}}

\RU{Вот листинг на ассемблере в}\EN{Here its assembly language listing in} \IDA:

\lstinputlisting{examples/z3/algo_1.asm}

\RU{Пример был скомпилирован в}\EN{The example was compiled by} GCC, \RU{так что первый аргумент
передается в}\EN{so the first argument is passed in} \ECX.

\index{Hex-Rays}
\RU{Если вы не имеете}\EN{If you don't have} Hex-Rays\RU{, либо вы не доверяете его результатам, мы можем попробовать
переписать всё это на Си вручную}\EN{  or if you distrust to it, 
you can try to reverse this code manually}.
\RU{Один из методов, это представить регистры \ac{CPU} в виде локальных переменных Си и заменить каждую инструкцию
эквивалентным выражением, например}\EN{One method is to represent the \ac{CPU} registers as local C variables and 
replace each instruction by a one-line equivalent expression, like}:

\lstinputlisting{examples/z3/algo_2.c}

\RU{Если быть очень аккуратным, этот код можно скомпилировать, и он даже будет работать, 
точно так же, как оригинальный.}
\EN{If you are careful enough, 
this code can be compiled and will even work in the same way as the original.}

\RU{Затем, будем переписывать его постепенно, не забывая об использовании регистров.}
\EN{Then, we are going to rewrite it gradually, keeping in mind all registers usage.}
\RU{Внимание и фокусирование здесь крайне важно --- любая самая мелкая опечатка может испортить всю работу}
\EN{Attention and focus is very important here---any tiny typo may ruin all your work}!

\RU{Первый шаг}\EN{Here is the first step}:

\lstinputlisting{examples/z3/algo_3.c}

\RU{Следующий шаг}\EN{Next step}:

\lstinputlisting{examples/z3/algo_4.c}

\RU{Мы находим деление через умножение}\EN{We can spot the division using multiplication} (\myref{sec:divisionbynine}).
\index{Wolfram Mathematica}
\RU{Действительно, найдем делитель в}\EN{Indeed, let's calculate the divider in} Wolfram Mathematica:

\begin{lstlisting}[caption=Wolfram Mathematica]
In[1]:=N[2^(64 + 5)/16^^8888888888888889]
Out[1]:=60.
\end{lstlisting}

\RU{Получаем}\EN{We get this}:

\lstinputlisting{examples/z3/algo_5.c}

\RU{Еще один шаг}\EN{One more step}:

\lstinputlisting{examples/z3/algo_6.c}

\RU{Простым сокращением, мы видим, что вычислялось вовсе не \glslink{quotient}{частное}, а остаток от деления}
\EN{By simple reducing, we finally see that it's calculating the remainder, not the \gls{quotient}}:

\lstinputlisting{examples/z3/algo_7.c}

\RU{Заканчиваем на приятно отформатированном исходном коде}\EN{We end up with this fancy formatted source-code}:

\lstinputlisting{examples/z3/algo_src.c}

\RU{Так как мы не криптоаналитики, мы не можем найти простой способ найти входное значение
для определенного выходного значения}\EN{Since we are not cryptoanalysts we can't find an easy way to 
generate the input value for some specific output value}.
\RU{Коэффициенты инструкций сдвигов выглядят очень пугающе --- это гарантия что функция не биективная,
она имеет коллизии, или, говоря проще, возможны несколько значений на входе для одного на выходе}
\EN{The rotate instruction's coefficients look frightening---it's a warranty that the function is not bijective,
it has collisions, or, speaking more simply, many inputs may be possible for one output}.

\RU{Брут-форс это тоже не решение, т.к., значения 64-битные, и это совершенно нереально}
\EN{Brute-force is not solution because values are 64-bit ones, that's beyond reality}.

\section{\RU{Попробуем Z3 SMT-солвер}\EN{Now let's use the Z3 SMT solver}}
\index{Z3}

\RU{Но все же, без всяких специальных знаний из криптографии, мы можем попытаться взломать алгоритм при помощи
великолепного SMT-солвера от}\EN{Still, without any special cryptographic knowledge, we may try to break this 
algorithm using the excellent SMT solver from} Microsoft Research \RU{под названием}\EN{named} 
Z3\footnote{\url{http://go.yurichev.com/17314}}.
\RU{На самом деле, это автоматический доказыватель теорем, 
но мы будем использовать его как SMT-солвер.}
\EN{It is in fact theorem prover, but we are going to use it as SMT solver.}
\RU{Упрощенно говоря, мы можем думать о нем как о системе, способной решать очень большие системы уравнений}
\EN{Simply said, we can think about it as a system capable of solving huge equation systems}.

\RU{Вот исходный код на Питоне}\EN{Here is the Python source code}:

\lstinputlisting[numbers=left]{examples/z3/1.py}

\RU{Это будет наш первый солвер}\EN{This is going to be our first solver}.

\RU{На строке 7 мы видим объявление переменных}\EN{We see the variable definitions on line 7}.
\RU{Это просто 64-битные переменные}\EN{These are just 64-bit variables}.
\TT{i1..i6} \RU{это промежуточные переменные, отражающие значения в регистрах между исполнениями инструкций}
\EN{are intermediate variables, representing the values in the registers between instruction executions}.

\RU{Потом добавляем т.н. констрайнты, в строках}\EN{Then we add the so-called constraints on lines} 10..15.
\RU{Самый последний констрайнт в строке 17 это наиболее важный: мы будем искать входное значение для
нашего алгоритма, при котором он выдаст на выходе}
\EN{The last constraint at 17 is the most important one: 
we are going to try to find an input value for which our algorithm will produce} $10816636949158156260$.

\RU{Собственно, SMT-солвер ищет (любые) значения, удовлетворяющие всем констрайнтам}
\EN{Essentially, the SMT-solver searches for (any) values that satisfies all constraints}.

RotateRight, RotateLeft, URem\EMDASH{}\RU{это функции из Питоновского Z3 \ac{API} для описания выражений, 
они не связаны с ЯП Python}
\EN{are functions from the Z3 Python \ac{API}, not related to Python \ac{PL}}.

\RU{Запускаем}\EN{Then we run it}:

\begin{lstlisting}
...>python.exe 1.py
sat
[i1 = 3959740824832824396,
 i3 = 8957124831728646493,
 i5 = 10816636949158156260,
 inp = 1364123924608584563,
 outp = 10816636949158156260,
 i4 = 14065440378185297801,
 i2 = 4954926323707358301]
 inp=0x12EE577B63E80B73
outp=0x961C69FF0AEFD7E4
\end{lstlisting}

``sat'' \RU{означает}\EN{mean} ``satisfiable'', \RU{т.е., солвер нашел по крайней мере одно решение}
\EN{i.e., the solver was able to found at least one solution}.
\RU{Решение выведено внутри квадратных скобок}\EN{The solution is printed in the square brackets}.
\RU{Две последние строки это пара входного/выходного значения в шестнадцатеричном виде}
\EN{The last two lines are the input/output pair in hexadecimal form}.
\RU{Да, действительно, если мы запустим нашу функцию с}\EN{Yes, indeed, if we run our function with} 
\TT{0x12EE577B63E80B73} \RU{на входе, алгоритм выдаст искомое значение}
\EN{as input, the algorithm will produce the value we were looking for}.

\RU{Но, как мы заметили раннее, функция не биективная, так что тут могут быть и другие корректные входные значения}
\EN{But, as we noticed before, the function we work with is not bijective, so there may be other correct
input values}.
\RU{Z3 SMT-солвер не выдает результаты больше одного, но мы можем хакнуть наш пример немного, 
добавив констрайнт в строке 19, означая, что мы ищем какие угодно другие результаты кроме этого}
\EN{The Z3 SMT solver is not capable of producing more than one result, but let's hack our example slightly, 
by adding line 19, which implies ``look for any other results than this''}:

\lstinputlisting[numbers=left]{examples/z3/2.py}

\RU{Действительно, получаем еще один верный результат}\EN{Indeed, it finds another correct result}:

\begin{lstlisting}
...>python.exe 2.py
sat
[i1 = 3959740824832824396,
 i3 = 8957124831728646493,
 i5 = 10816636949158156260,
 inp = 10587495961463360371,
 outp = 10816636949158156260,
 i4 = 14065440378185297801,
 i2 = 4954926323707358301]
 inp=0x92EE577B63E80B73
outp=0x961C69FF0AEFD7E4
\end{lstlisting}

\RU{Это можно автоматизировать}\EN{This can be automated}.
\RU{Каждый найденный результат можно добавлять в качестве констрайнта и искать следующий.}
\EN{Each found result can be added as a constraint and then the next result will be searched for.}
\RU{Пример немного сложнее}\EN{Here is a slightly more sophisticated example}:

\lstinputlisting[numbers=left]{examples/z3/3.py}

\RU{Получаем}\EN{We got}:

\begin{lstlisting}
1364123924608584563
1234567890
9223372038089343698
4611686019661955794
13835058056516731602
3096040143925676201
12319412180780452009
7707726162353064105
16931098199207839913
1906652839273745429
11130024876128521237
15741710894555909141
6518338857701133333
5975809943035972467
15199181979890748275
10587495961463360371
results total= 16
\end{lstlisting}

\RU{Так что имеется 16 верных входных значений для}\EN{So there are 16 correct input values for} 
\TT{0x92EE577B63E80B73} \RU{на выходе}\EN{as a result}.

\RU{Второй это}\EN{The second is} $1234567890$\EMDASH{}\RU{действительно, я это и использовал изначально,
когда готовил этот пример}
\EN{it is indeed the value I used originally while preparing this example}.

\RU{Попробуем изучить алгоритм немного больше}\EN{Let's also try to research our algorithm a bit more}.
\RU{В порыве садистских желаний, попробуем найти, есть ли здесь какая-нибудь возможная пара входов/выходов,
в которых младшие 32-битные части равны друг другу}
\EN{Acting on a sadistic whim, let's find if the there are any possible input/output pairs in 
which the lower 32-bit parts are equal to each other}?

\RU{Уберем констрайнт}\EN{Let's remove the} \IT{outp} \RU{и добавим другой, в строке 17}
\EN{constraint and add another, at line 17}:

\lstinputlisting[numbers=left]{examples/z3/4.py}

\RU{И действительно}\EN{It is indeed so}:

\begin{lstlisting}
sat
[i1 = 14869545517796235860,
 i3 = 8388171335828825253,
 i5 = 6918262285561543945,
 inp = 1370377541658871093,
 outp = 14543180351754208565,
 i4 = 10167065714588685486,
 i2 = 5541032613289652645]
 inp=0x13048F1D12C00535
outp=0xC9D3C17A12C00535
\end{lstlisting}

\RU{Можем упражняться в садизме и далее: пусть последние 16-бит всегда будут}
\EN{Let's be more sadistic and add another constraint: last the 16 bits must be} \TT{0x1234}:

\lstinputlisting[numbers=left]{examples/z3/5.py}

\RU{Это так же возможно}\EN{Oh yes, this possible as well}:

\begin{lstlisting}
sat
[i1 = 2834222860503985872,
 i3 = 2294680776671411152,
 i5 = 17492621421353821227,
 inp = 461881484695179828,
 outp = 419247225543463476,
 i4 = 2294680776671411152,
 i2 = 2834222860503985872]
 inp=0x668EEC35F961234
outp=0x5D177215F961234
\end{lstlisting}

\RU{Z3 работает крайне быстро и это означает что алгоритм слаб, и вообще не относится к криптографическим 
(как и почти вся любительская криптография)}
\EN{Z3 works very fast and it implies that the algorithm is weak, it is not cryptographic at all
(like the most of the amateur cryptography)}.

\RU{Можно ли попытаться сделать что-то подобное с настоящими криптоалгоритмами этими методами}
\EN{Is it possible to tackle real cryptography by these methods}? 
\RU{Настоящие алгоритмы, такие как}\EN{Real algorithms like} AES, RSA, \RU{\etc{}, так же могут быть представлены
в виде огромных систем уравнений, но они большие настолько, что с ними нельзя работать на компьютерах,
ни сейчас, ни в обозримом будущем}\EN{\etc{}, can also be represented as huge system of equations, 
but these are so huge that they are impossible to work with on computers, now or in the near future}.
\RU{Разумеется, криптографы об этом всем прекрасно знают}\EN{Of course, cryptographers are aware of this}.

\EN{Summarizing, when dealing with amateur crypto, 
it's a very good idea to try a SMT/SAT solver (like Z3).}
\RU{Подводя итоги, нужно сказать, что работая с любительской криптографией, 
попробовать SMT/SAT-солвер (как Z3) это всегда хорошая идея.}

\RU{Еще одна статья которую я написал о Z3:}\EN{Another article I wrote about Z3 is} \cite{Rockey}.
