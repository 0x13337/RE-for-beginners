\chapter{\RU{Ручная декомпиляция + использование SMT-солвера Z3 для взлома любительской криптографии}
\EN{Hand decompiling + using Z3 SMT solver for defeating amateur cryptography}}

\RU{Любительская криптография обычно (непреднамеренно) очень слабая и может быть легко сломана ---
для криптографов, конечно}\EN{Amateur cryptography is usually (unintentionally) 
very weak and can be breaked easily---for cryptographers, of course}.

\RU{Но представим на время что мы не в числе этих профессионалов}
\EN{But let's pretend we are not among these crypto-professionals}.

\RU{Я нашел эту необратимую хэш-ф-цию, которая конвертирует одно 64-битное значение в другое,
и нам нужно попытаться развернуть её работу назад}
\EN{I once found this one-way hash function, converting 64-bit value to another one and we need to try
to reverse its flow back}.

\label{hash_func}
\begin{quotation}
\index{\RU{Хеш-функции}\EN{Hash functions}}
\index{CRC32}
\RU{Но что такое хеш-функция}\EN{But what is hash-function}?
\RU{Простейший пример это CRC32, алгоритм ``более мощный'' чем простая контрольная сумма,
для проверки целостности данных}
\EN{Simplest example is CRC32, an algorithm providing ``stronger'' checksum for integrity checking purposes}.
\RU{Невозможно восстановить оригинальный текст из хеша, там просто меньше информации: ведь текст
может быть очень длинным, но результат CRC32 всегда ограничен 32 битами}
\EN{it is impossible to restore original text from the hash value, it just has much less information:
there can be long text, but CRC32 result is always limited to 32 bits}.
\RU{Но CRC32 не надежна в криптографическом смысле: известны методы как изменить текст таким образом,
чтобы получить нужный результат}
\EN{But CRC32 is not cryptographically secure: it is known how to alter a text in that way so the resulting
CRC32 hash value will be one we need}.
\RU{Криптографические хеш-функции защищены от этого}
\EN{Cryptographical hash functions are protected from this}.
\index{MD5}
\index{SHA1}
\RU{Такие ф-ции как MD5, SHA1, и т.д, широко используются для хеширования паролей
для хранения их в базе}
\EN{They are widely used to hash user passwords in order to store them in the database, 
like MD5, SHA1, etc}.
\RU{Действительно: БД форума в интернете может и не хранить пароли 
(иначе злоумышленник получивший доступ к БД сможет узнать все пароли), а только хеши}
\EN{Indeed: an internet forum database may not contain user passwords (stolen database will compromise
all user's passwords) but only hashes (a cracker will not be able to reveal passwords)}.
\RU{К тому же, скрипту интернет-форума вовсе не обязательно знать ваш пароль, он только должен
cверить его хеш с тем что лежит в БД, и дать вам доступ если cверка проходит}
\EN{Besides, an internet forum engine is not aware of your password, it should only check if its hash
is the same as in the database, then it will give you access in this case}.
\RU{Один из самых простых способов взлома это просто перебирать все пароли и ждать пока
результат будет такой же как тот что нам нужен}
\EN{One of the simplest passwords cracking methods is just to brute-force all passwords in order to wait
when resulting value will be the same as we need}.
\RU{Другие методы намного сложнее}\EN{Other methods are much more complex}. \\
\end{quotation}

\section{\RU{Ручная декомпиляция}\EN{Hand decompiling}}

\RU{Вот листинг на ассемблере в}\EN{Here its assembly language listing in} \IDA:

\lstinputlisting{examples/z3/algo_1.asm}

\RU{Пример был скомпилирован в}\EN{The example was compiled by} GCC, \RU{так что первый аргумент
передается в}\EN{so the first argument is passed in} \ECX.

\index{Hex-Rays}
\RU{Если вы не имеете}\EN{If} Hex-Rays\RU{, либо вы не доверяете его результатам, мы можем попробовать
переписать всё это на Си вручную}\EN{ is not in list of our possessions, or we distrust to it, 
we may try to reverse this code manually}.
\RU{Один из методов, это представить регистры \ac{CPU} в виде локальных переменных Си и заменить каждую инструкцию
эквивалентным выражением, например}\EN{One method is to represent \ac{CPU} registers as local C variables and 
replace each instruction by one-line equivalent expression, like}:

\lstinputlisting{examples/z3/algo_2.c}

\RU{Если быть очень аккуратным, этот код можно скомпилировать и он даже будет работать, 
точно так же как оригинальный}
\EN{If to be careful enough, this code can be compiled and will even work in the same way as original one}.

\RU{Затем, будем переписывать его постепенно, не забывая об использовании регистров}
\EN{Then, we will rewrite it gradually, keeping in mind all registers usage}.
\RU{Внимание и фокусирование здесь крайне важно --- любая самая мелкая опечатка может испортить всю работу}
\EN{Attention and focusing is very important here---any tiny typo may ruin all your work}!

\RU{Первый шаг}\EN{Here is a first step}:

\lstinputlisting{examples/z3/algo_3.c}

\RU{Следующий шаг}\EN{Next step}:

\lstinputlisting{examples/z3/algo_4.c}

\RU{Мы находим деление через умножение}\EN{We may spot division using multiplication} (\ref{sec:divisionbynine}).
\index{Wolfram Mathematica}
\RU{Действительно, найдем делитель в}\EN{Indeed, let's calculate divider in} Wolfram Mathematica:

\begin{lstlisting}[caption=Wolfram Mathematica]
In[1]:=N[2^(64 + 5)/16^^8888888888888889]
Out[1]:=60.
\end{lstlisting}

\RU{Получаем}\EN{We get this}:

\lstinputlisting{examples/z3/algo_5.c}

\RU{Еще один шаг}\EN{Another step}:

\lstinputlisting{examples/z3/algo_6.c}

\RU{Простым сокращением, мы видим, что вычислялось вовсе не \glslink{quotient}{частное}, а остаток от деления}
\EN{By simple reducing, we finally see that it's not \gls{quotient} calculated, but division remainder}:

\lstinputlisting{examples/z3/algo_7.c}

\RU{Заканчиваем на приятно отформатированном исходном коде}\EN{We end up on something fancy formatted source-code}:

\lstinputlisting{examples/z3/algo_src.c}

\RU{Так как мы не криптоаналитики, мы не можем найти простой способ найти входное значение
для определенного выходного значения}\EN{Since we are not cryptoanalysts we can't find an easy way to 
generate input value for some specific output value}.
\RU{Коэффициенты инструкций сдвигов выглядят очень пугающе --- это гарантия что ф-ция не биективная,
она имеет коллизии, или, говоря проще, возможны несколько значений на входе для одного на выходе}
\EN{Rotate instruction coefficients are look frightening---it's a warranty that the function is not bijective,
it has collisions, or, speaking more simply, many inputs may be possible for one output}.

\RU{Брут-форс это тоже не решение, т.к., значения 64-битные, и это совершенно нереально}
\EN{Brute-force is not solution because values are 64-bit ones, that's beyond reality}.

\section{\RU{Попробуем Z3 SMT-солвер}\EN{Now let's use Z3 SMT solver}}
\index{Z3}

\RU{Но все же, без всяких специальных знаний из криптографии, мы можем попытаться взломать алгоритм при помощи
великолепного SMT-солвера от}\EN{Still, without any special cryptographical knowledge, we may try to break this 
algorithm using excellent SMT solver from} Microsoft Research \RU{под названием}\EN{named} 
Z3\footnote{\url{http://z3.codeplex.com/}}.
\RU{На самом деле, это автоматический доказыватель теорем, но мы будем использовать его как SMT-солвер}
\EN{It is in fact theorem prover, but we will use it as SMT solver}.
\RU{Упрощенно говоря, мы можем думать о нем как о системе, способной решать очень большие системы уравнений}
\EN{In terms of simplicity, we may think about it as a system capable of solving huge equation systems}.

\RU{Вот исходный код на Питоне}\EN{Here is a Python source code}:

\lstinputlisting[numbers=left]{examples/z3/1.py}

\RU{Это будет наш первый солвер}\EN{This will be our first solver}.

\RU{На строке 7 мы видим объявление переменных}\EN{We see variable definitions on line 7}.
\RU{Это просто 64-битные переменные}\EN{These are just 64-bit variables}.
\TT{i1..i6} \RU{это промежуточные переменные, отражающие значения в регистрах между исполнениями инструкций}
\EN{are intermediate variables, representing values in registers between instruction executions}.

\RU{Потом добавляем т.н. констрайнты, в строках}\EN{Then we add so called constraints on lines} 10..15.
\RU{Самый последний констрайнт в строке 17 это наиболее важный: мы будем искать входное значение для
нашего алгоритма, при котором он выдаст на выходе}\EN{The very last constraint at 17 is most important: 
we will try to find input value for which our algorithm will produce} $10816636949158156260$.

\RU{Собственно, SMT-солвер ищет (любые) значения, удовлетворяющие всем констрайнтам}
\EN{Essentially, SMT-solver searches for (any) values that satisfy all constraints}.

RotateRight, RotateLeft, URem\EMDASH{}\RU{это ф-ции из Питоновского Z3 \ac{API} для описания выражений, 
они не связаны с ЯП Python}
\EN{are functions from Z3 Python \ac{API}, they are not related to Python \ac{PL}}.

\RU{Запускаем}\EN{Then we run it}:

\begin{lstlisting}
...>python.exe 1.py
sat
[i1 = 3959740824832824396,
 i3 = 8957124831728646493,
 i5 = 10816636949158156260,
 inp = 1364123924608584563,
 outp = 10816636949158156260,
 i4 = 14065440378185297801,
 i2 = 4954926323707358301]
 inp=0x12EE577B63E80B73
outp=0x961C69FF0AEFD7E4
\end{lstlisting}

``sat'' \RU{означает}\EN{mean} ``satisfiable'', \RU{т.е., солвер нашел по крайней мере одно решение}
\EN{i.e., solver was able to found at least one solution}.
\RU{Решение выведено внутри квадратных скобок}\EN{The solution is printed inside square brackets}.
\RU{Две последние строки это пара входного/выходного значения в шестнадцатеричном виде}
\EN{Two last lines are input/output pair in hexadecimal form}.
\RU{Да, действительно, если мы запустим нашу ф-цию с}\EN{Yes, indeed, if we run our function with} 
\TT{0x12EE577B63E80B73} \RU{на входе, алгоритм выдаст искомое значение}
\EN{on input, the algorithm will produce the value we were looking for}.

\RU{Но, как мы заметили раннее, ф-ция не биективная, так что тут могут быть и другие корректные входные значения}
\EN{But, as we are noticed before, the function we work with is not bijective, so there are may be other correct
input values}.
\RU{Z3 SMT-солвер не выдает результаты больше одного, но мы можем хакнуть наш пример немного, 
добавив констрайнт в строке 19, означая, что мы ищем какие угодно другие результаты кроме этого}
\EN{Z3 SMT solver is not capable of producing more than one result, but let's hack our example slightly, 
by adding line 19, meaning, look for any other results than this}:

\lstinputlisting[numbers=left]{examples/z3/2.py}

\RU{Действительно, получаем еще один верный результат}\EN{Indeed, it found other correct result}:

\begin{lstlisting}
...>python.exe 2.py
sat
[i1 = 3959740824832824396,
 i3 = 8957124831728646493,
 i5 = 10816636949158156260,
 inp = 10587495961463360371,
 outp = 10816636949158156260,
 i4 = 14065440378185297801,
 i2 = 4954926323707358301]
 inp=0x92EE577B63E80B73
outp=0x961C69FF0AEFD7E4
\end{lstlisting}

\RU{Это можно автоматизировать}\EN{This can be automated}.
\RU{Каждый найденный результат можно добавлять в качестве констрайнта и искать следующий}
\EN{Each found result may be added as constraint and the next result will be searched for}.
\RU{Пример немного сложнее}\EN{Here is slightly sophisticated example}:

\lstinputlisting[numbers=left]{examples/z3/3.py}

\RU{Получаем}\EN{We got}:

\begin{lstlisting}
1364123924608584563
1234567890
9223372038089343698
4611686019661955794
13835058056516731602
3096040143925676201
12319412180780452009
7707726162353064105
16931098199207839913
1906652839273745429
11130024876128521237
15741710894555909141
6518338857701133333
5975809943035972467
15199181979890748275
10587495961463360371
results total= 16
\end{lstlisting}

\RU{Так что имеется 16 верных входных значений для}\EN{So there are 16 correct input values are possible for} 
\TT{0x92EE577B63E80B73} \RU{на выходе}\EN{as a result}.

\RU{Второй это}\EN{The second is} $1234567890$\EMDASH{}\RU{действительно, я это и использовал изначально,
когда готовил этот пример}
\EN{it is indeed a value I used originally while preparing this example}.

\RU{Попробуем изучить алгоритм немного больше}\EN{Let's also try to research our algorithm more}.
\RU{В порыве садистских желаний, попробуем найти, есть ли здесь какая-нибудь возможная пара входов/выходов,
в которых младшие 32-битные части равны друг другу}
\EN{By some sadistic purposes, let's find, are there any possible input/output pair in 
which lower 32-bit parts are equal to each other}?

\RU{Уберем констрайнт}\EN{Let's remove} \IT{outp} \RU{и добавим другой, в строке 17}
\EN{constraint and add another, at line 17}:

\lstinputlisting[numbers=left]{examples/z3/4.py}

\RU{И действительно}\EN{It is indeed so}:

\begin{lstlisting}
sat
[i1 = 14869545517796235860,
 i3 = 8388171335828825253,
 i5 = 6918262285561543945,
 inp = 1370377541658871093,
 outp = 14543180351754208565,
 i4 = 10167065714588685486,
 i2 = 5541032613289652645]
 inp=0x13048F1D12C00535
outp=0xC9D3C17A12C00535
\end{lstlisting}

\RU{Можем упражняться в садизме и далее: пусть последние 16-бит всегда будут}
\EN{Let's be more sadistic and add another constraint: last 16-bit should be} \TT{0x1234}:

\lstinputlisting[numbers=left]{examples/z3/5.py}

\RU{Это так же возможно}\EN{Oh yes, this possible as well}:

\begin{lstlisting}
sat
[i1 = 2834222860503985872,
 i3 = 2294680776671411152,
 i5 = 17492621421353821227,
 inp = 461881484695179828,
 outp = 419247225543463476,
 i4 = 2294680776671411152,
 i2 = 2834222860503985872]
 inp=0x668EEC35F961234
outp=0x5D177215F961234
\end{lstlisting}

\RU{Z3 работает крайне быстро и это означает что алгоритм слаб, и вообще не относится к криптографическим 
(как и почти вся любительская криптография)}
\EN{Z3 works very fast and it means that algorithm is weak, it is not cryptographical at all
(like the most of amateur cryptography)}.

\RU{Можно ли попытаться сделать что-то подобное с настоящими криптоалгоритмами этими методами}
\EN{Will it be possible to tackle real cryptography by these methods}? 
\RU{Настоящие алгоритмы, такие как}\EN{Real algorithms like} AES, RSA, \RU{итд, так же могут быть представлены
в виде огромных систем уравнений, но они большие настолько, что с ними нельзя работать на компьютерах,
ни сейчас, ни в обозримом будущем}\EN{etc, can also be represented as huge system of equations, 
but these are that huge that are impossible to work with on computers, now or in near future}.
\RU{Разумеется, криптографы об этом всем прекрасно знают}\EN{Of course, cryptographers are aware of this}.

\RU{Еще одна статья которую я написал о Z3:}\EN{Another article I wrote about Z3 is} \cite{Rockey}.

