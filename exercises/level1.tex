\chapter{\IFRU{Уровень}{Level} 1}

\IFRU{Задачи первого уровня, это те, которые можно решать голове}{Level 1 exercises are ones you
may try to solve in mind}.

\section{\Exercise 1.1}
% max()

\subsection{MSVC 2012 x64 + \Ox}

\index{x86!\Instructions!CMOVcc}
\begin{lstlisting}
a$ = 8
b$ = 16
f	PROC
	cmp	ecx, edx
	cmovg	edx, ecx
	mov	eax, edx
	ret	0
f	ENDP
\end{lstlisting}

\subsection{Keil (ARM)}

\begin{lstlisting}
        CMP      r0,r1
        MOVLE    r0,r1
        BX       lr
\end{lstlisting}

\subsection{Keil (thumb)}
        
\begin{lstlisting}
	CMP      r0,r1
        BGT      |L0.6|
        MOVS     r0,r1
|L0.6|
        BX       lr
\end{lstlisting}

\section{\Exercise 1.2}

\index{x86!\Instructions!LOOP}
\IFRU{Почему инструкция}{Why} \LOOP \IFRU{больше не используется компиляторами}{instruction is 
not used by compilers anymore}?

\section{\Exercise 1.3}

\IFRU{Возьмите пример из секции}{Take an loop example from} ``\Loops''\EN{ section} (\ref{sec:loops}), 
\IFRU{скомпилируйте его в вашей любимой}{compile it in your favorite} \ac{OS}
\IFRU{и компиляторе, и модифицируйте исполняемый файл так, чтобы цикл был в пределах}{and compiler 
and modify (patch) executable file, so the loop range will be} [6..20].

\section{\Exercise 1.4}

\IFRU{Эта программа запрашивает пароль}{This program requires password}. \IFRU{Найдите его}{Find it}.

\begin{itemize}
\item win32: \url{http://yurichev.com/RE-exercises/1/4/password1.exe}
\item Linux x86: \url{http://yurichev.com/RE-exercises/1/4/password1_Linux_x86.tar}
\item \MacOSX: \url{http://yurichev.com/RE-exercises/1/4/password1_MacOSX64.tar}
\end{itemize}

