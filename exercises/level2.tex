\chapter{\RU{Уровень}\EN{Level} 2}

\RU{Для решения задач второго уровня, вам вероятно понадобится текстовый редактор или тетрадка с ручкой.}
\EN{For solving exercises of level 2, you may need a text editor or paper and pencil.}

\section{\Exercise 2.1}
% sqrt

\subsection{\Optimizing MSVC 2010 x86}

\lstinputlisting{exercises/2/sqrt_MSVC2010_Ox.asm}

\subsection{\Optimizing MSVC 2012 x64}

\lstinputlisting{exercises/2/sqrt_MSVC2012_Ox_x64.asm}

% 2.2
% 2.3

\section{\Exercise 2.4}
% strstr()

\RU{Это стандартная функция из библиотек Си. Исходник взят из MSVC 2010.}
\EN{This is a standard C library function. The source code is taken from MSVC 2010.}

\subsection{\Optimizing MSVC 2010}

\lstinputlisting{exercises/1_4_msvc.asm}

\subsection{GCC 4.4.1}

\lstinputlisting{exercises/1_4_gcc.asm}

\subsection{\Optimizing Keil (\ARMMode)}

\lstinputlisting{exercises/1_4_ARM.s}

\subsection{\Optimizing Keil (\ThumbMode)}

\lstinputlisting{exercises/1_4_thumb.s}

\subsection{\Optimizing GCC 4.9.1 (ARM64)}

\lstinputlisting[caption=\Optimizing GCC 4.9.1 (ARM64)]{exercises/1_4_ARM64_O3.s}

\subsection{\Optimizing GCC 4.4.5 (MIPS)}

\lstinputlisting[caption=\Optimizing GCC 4.4.5 (MIPS) (IDA)]{exercises/1_4_MIPS_O3_IDA.lst}

% 2.5

\section{\Exercise 2.6}
% TEA

\subsection{\Optimizing MSVC 2010}

\lstinputlisting{exercises/1_6_msvc.asm}

\subsection{\Optimizing Keil (\ARMMode)}

\lstinputlisting{exercises/1_6_ARM.s}

\subsection{\Optimizing Keil (\ThumbMode)}

\lstinputlisting{exercises/1_6_thumb.s}

\subsection{\Optimizing GCC 4.9.1 (ARM64)}

\lstinputlisting[caption=\Optimizing GCC 4.9.1 (ARM64)]{exercises/1_6_ARM64_O3.s}

\subsection{\Optimizing GCC 4.4.5 (MIPS)}

\lstinputlisting[caption=\Optimizing GCC 4.4.5 (MIPS) (IDA)]{exercises/1_6_MIPS_O3_IDA.lst}

% 2.7
% 2.8
% 2.9
% 2.10
% 2.11
% 2.12

\section{\Exercise 2.13}
% LFSR

\RU{Это довольно известный криптоалгоритм прошлого}\EN{This is a well-known cryptographic algorithm from the past}.
\RU{Как он называется}\EN{How is it called}?

\subsection{\Optimizing MSVC 2012}

\begin{lstlisting}
_in$ = 8						; size = 2
_f	PROC
	movzx	ecx, WORD PTR _in$[esp-4]
	lea	eax, DWORD PTR [ecx*4]
	xor	eax, ecx
	add	eax, eax
	xor	eax, ecx
	shl	eax, 2
	xor	eax, ecx
	and	eax, 32					; 00000020H
	shl	eax, 10					; 0000000aH
	shr	ecx, 1
	or	eax, ecx
	ret	0
_f	ENDP
\end{lstlisting}

\subsection{Keil (\ARMMode)}

\begin{lstlisting}
f PROC
        EOR      r1,r0,r0,LSR #2
        EOR      r1,r1,r0,LSR #3
        EOR      r1,r1,r0,LSR #5
        AND      r1,r1,#1
        LSR      r0,r0,#1
        ORR      r0,r0,r1,LSL #15
        BX       lr
        ENDP
\end{lstlisting}

\subsection{Keil (\ThumbMode)}

\begin{lstlisting}
f PROC
        LSRS     r1,r0,#2
        EORS     r1,r1,r0
        LSRS     r2,r0,#3
        EORS     r1,r1,r2
        LSRS     r2,r0,#5
        EORS     r1,r1,r2
        LSLS     r1,r1,#31
        LSRS     r0,r0,#1
        LSRS     r1,r1,#16
        ORRS     r0,r0,r1
        BX       lr
        ENDP
\end{lstlisting}

\subsection{\Optimizing GCC 4.9.1 (ARM64)}

\begin{lstlisting}
f:
	uxth	w1, w0
	lsr	w2, w1, 3
	lsr	w0, w1, 1
	eor	w2, w2, w1, lsr 2
	eor	w2, w1, w2
	eor	w1, w2, w1, lsr 5
	and	w1, w1, 1
	orr	w0, w0, w1, lsl 15
	ret
\end{lstlisting}

\subsection{\Optimizing GCC 4.4.5 (MIPS)}

\lstinputlisting[caption=\Optimizing GCC 4.4.5 (MIPS) (IDA)]{exercises/2_13_MIPS_O3_IDA.lst}

\section{\Exercise 2.14}
% GCD

\RU{Еще один хорошо известный алгоритм. Функция берет на вход 2 значения и возвращает одно.}
\EN{Another well-known algorithm. The function takes two variables and returns one.}

\subsection{MSVC 2012}

\index{x86!\Instructions!BSF}
\lstinputlisting{exercises/2/GCD_MSVC_2012_Ox.asm}

\subsection{Keil (\ARMMode)}

\index{ARM!\Instructions!CLZ}
\lstinputlisting{exercises/2/GCD_Keil_ARM_O3.s}

\subsection{GCC 4.6.3 for Raspberry Pi (\ARMMode)}

\index{ARM!\Instructions!CLZ}
\lstinputlisting{exercises/2/GCD_ARM_pi_GCC_4.6.3_O3.s}

\subsection{\Optimizing GCC 4.9.1 (ARM64)}

\lstinputlisting[caption=\Optimizing GCC 4.9.1 (ARM64)]{exercises/2/GCD_ARM64_GCC491_O3.s}

\subsection{\Optimizing GCC 4.4.5 (MIPS)}

\lstinputlisting[caption=\Optimizing GCC 4.4.5 (MIPS) (IDA)]{exercises/2/GCD_MIPS_O3_IDA.lst}

\section{\Exercise 2.15}
% Monte Carlo

\RU{И снова известный алгоритм. Что он делает?}\EN{A well-known algorithm again. What does it do?}

\RU{Обратите внимание, что код для x86 использует FPU, а для x64 --- SIMD-инструкции. Это нормально}
\EN{Take also notice that the code for x86 uses FPU, but SIMD instructions are used instead in x64 code.
That's OK}: \myref{floating_SIMD}.

\subsection{\Optimizing MSVC 2012 x64}

\lstinputlisting{exercises/2/monte_MSVC_2012_Ox_x64.asm}

\subsection{\Optimizing GCC 4.4.6 x64}

\lstinputlisting{exercises/2/monte_GCC_4.4.6_O3_x64.s}

\subsection{\Optimizing GCC 4.8.1 x86}

\lstinputlisting{exercises/2/monte_GCC_4.8.1_O3_x86.s}

\subsection{Keil (\ARMMode): \RU{для процессора Cortex-R4F}\EN{Cortex-R4F CPU as target}}

\lstinputlisting{exercises/2/monte_Keil_ARM_Cortex.s}

\subsection{\Optimizing GCC 4.9.1 (ARM64)}

\lstinputlisting[caption=\Optimizing GCC 4.9.1 (ARM64)]{exercises/2/monte_GCC_491_ARM64_O3.s}

\subsection{\Optimizing GCC 4.4.5 (MIPS)}

\lstinputlisting[caption=\Optimizing GCC 4.4.5 (MIPS) (IDA)]{exercises/2/monte_GCC_4.4.6_O3_x64.s}

\section{\Exercise 2.16}
% Ackermann function

\RU{Известная функция. Что она вычисляет? Почему стек переполняется если на вход подать
числа 4 и 2? Есть ли здесь какая-то ошибка?}\EN{Well-known function. What does it compute? 
Why does the stack overflow if 4 and 2 are supplied at input? Is there any error?}

\subsection{\Optimizing MSVC 2012 x64}

\lstinputlisting{exercises/2/ack_MSVC_Ox_x64.asm}

\subsection{\Optimizing Keil (\ARMMode)}

\lstinputlisting{exercises/2/ack_ARM_O3.s}

\subsection{\Optimizing Keil (\ThumbMode)}

\lstinputlisting{exercises/2/ack_thumb_O3.s}

\subsection{\NonOptimizing GCC 4.9.1 (ARM64)}

\lstinputlisting[caption=\NonOptimizing GCC 4.9.1 (ARM64)]{exercises/2/ack_ARM64_GCC491_O0.s}

\subsection{\Optimizing GCC 4.9.1 (ARM64)}

\RU{Оптимизирующий GCC генерирует куда больше кода. Почему?}
\EN{Optimizing GCC generates a lot more code. Why?}

\lstinputlisting[caption=\Optimizing GCC 4.9.1 (ARM64)]{exercises/2/ack_ARM64_GCC491_O3.s}

\subsection{\NonOptimizing GCC 4.4.5 (MIPS)}

\lstinputlisting[caption=\NonOptimizing GCC 4.4.5 (MIPS) (IDA)]{exercises/2/ack_MIPS_O0_IDA.lst}

\section{\Exercise 2.17}
% Rule 110

\RU{Эта программа выдает в \gls{stdout} какую-то информацию, каждый раз --- разную}\EN{This program
prints some information to \gls{stdout}, each time different}.
\RU{Что это}\EN{What is that}?

\RU{Скомпилированные бинарные файлы}\EN{Compiled binaries}:

\begin{itemize}
\item \href{http://go.yurichev.com/17170}{Linux x64 (beginners.re)}
\item \href{http://go.yurichev.com/17171}{\MacOSX (beginners.re)}
\item \href{http://go.yurichev.com/17172}{Linux MIPS (beginners.re)}
\item \href{http://go.yurichev.com/17173}{Win32 (beginners.re)}
\item \href{http://go.yurichev.com/17174}{Win64 (beginners.re)}
\end{itemize}

\RU{Для версий под Windows, возможно, нужно будет установить}
\EN{As for the Windows versions, you may need to install} 
\href{http://go.yurichev.com/17302}{MSVC 2012 redist}.

\section{\Exercise 2.18}

\RU{Эта программа запрашивает пароль}\EN{This program requires a password}.
\RU{Найдите его}\EN{Find it}.

\RU{Кстати, не только один пароль может подойти}\EN{By the way, multiple passwords may work}. 
\RU{Попробуйте найти еще}\EN{Try to find them}.

\RU{Как дополнительное упражнение, попробуйте изменить пароль модифицируя исполняемый файл}
\EN{As an additional exercise, try to change the password by patching the executable file}.

\begin{itemize}
\item \href{http://go.yurichev.com/17175}{Win32 (beginners.re)}
\item \href{http://go.yurichev.com/17176}{Linux x86 (beginners.re)}
\item \href{http://go.yurichev.com/17177}{\MacOSX (beginners.re)}
\item \href{http://go.yurichev.com/17178}{Linux MIPS (beginners.re)}
\end{itemize}

\section{\Exercise 2.19}

\RU{То же что и в упражнении}\EN{The same as in exercise} 2.18.

\begin{itemize}
\item \href{http://go.yurichev.com/17179}{Win32 (beginners.re)}
\item \href{http://go.yurichev.com/17180}{Linux x86 (beginners.re)}
\item \href{http://go.yurichev.com/17181}{\MacOSX (beginners.re)}
\item \href{http://go.yurichev.com/17182}{Linux MIPS (beginners.re)}
\end{itemize}

\section{\Exercise 2.20}
% Collatz conjecture

\RU{Эта программа выдает в \gls{stdout} какие-то числа.}
\EN{This program prints some numbers to \gls{stdout}.}
\RU{Что это}\EN{What is it}?

\RU{Скомпилированные бинарные файлы}\EN{Compiled binaries}:

\begin{itemize}
\item \href{http://go.yurichev.com/17183}{Linux x64 (beginners.re)}
\item \href{http://go.yurichev.com/17184}{\MacOSX (beginners.re)}
\item \href{http://go.yurichev.com/17185}{Linux ARM Raspberry Pi (beginners.re)}
\item \href{http://go.yurichev.com/17186}{Linux MIPS (beginners.re)}
\item \href{http://go.yurichev.com/17187}{Win64 (beginners.re)}
\end{itemize}
