\chapter{\RU{Уровень}\EN{Level} 3}

\RU{Для решения задач третьего уровня вам придется потратить какое-то ощутимое время, 
вплоть до одного дня}
\EN{For solving level 3 tasks, you'll probably need considerable ammount of time, maybe up to one day}.

\section{\Exercise 3.1}

\RU{Довольно известный алгоритм, так же включен в стандартную библиотеку Си. Исходник взят из glibc 2.11.1. 
Скомпилирован в GCC 4.4.1 с ключом \TT{-Os} (оптимизация по размеру кода). 
Листинг сделан дизассемблером IDA 4.9 из ELF-файла созданным GCC и линкером.}
\EN{Well-known algorithm, also included in standard C library. Source code was taken from glibc 2.11.1.
Compiled in GCC 4.4.1 with \TT{-Os} option (code size optimization).
Listing was done by IDA 4.9 disassembler from ELF-file generated by GCC and linker.}

\RU{Для тех кто хочет использовать IDA в процессе изучения, вот здесь лежат .elf и .idb файлы, 
.idb можно открыть при помощи бесплатной IDA 4.9:}
\EN{For those who wants use IDA while learning, here you may find .elf and .idb files,
.idb can be opened with freeware IDA 4.9:}

\url{http://yurichev.com/RE-exercises/3/1/}

\lstinputlisting{exercises/2_1_gcc.asm}

\section{\Exercise 3.2}

\RU{Имеется небольшой исполняемый файл, внутри которого находится довольно известная криптосистема}
\EN{There is a small executable file with a well-known cryptosystem inside}.
\RU{Попробуйте её идентифицировать}\EN{Try to identify it}.

\begin{itemize}
\item Windows x86: \url{http://yurichev.com/RE-exercises/3/2/unknown_cryptosystem.exe}
\item Linux x86: \url{http://yurichev.com/RE-exercises/3/2/unknown_encryption_linux86.tar}
\item \MacOSX (x64): \url{http://yurichev.com/RE-exercises/3/2/unknown_encryption_MacOSX.tar}
\end{itemize}

\section{\Exercise 3.3}

\RU{Имеется небольшой исполняемый файл, некая утилита}
\EN{There is a small executable file, some utility}.
\RU{Она открывает другой файл, читает его, что-то вычисляет и показывает число с плавающей точкой}
\EN{It opens another file, reads it, calculate something and prints a float number}.
\RU{Попробуйте разобраться, что она делает}\EN{Try to understand what it do}.

\begin{itemize}
\item Windows x86: \url{http://yurichev.com/RE-exercises/3/3/unknown_utility_2_3.exe}
\item Linux x86: \url{http://yurichev.com/RE-exercises/3/3/unknown_utility_2_3_Linux86.tar}
\item \MacOSX (x64): \url{http://yurichev.com/RE-exercises/3/3/unknown_utility_2_3_MacOSX.tar}
\end{itemize}

\section{\Exercise 3.4}

\RU{Утилита, шифрующая и дешифрующая файлы, по паролю}
\EN{There is an utility which encrypts/decrypts files, by password}.
\RU{Есть зашифрованный текстовый файл, пароль неизвестен}\EN{There is an encrypted text file,
password is unknown}.
\RU{Зашифрованный файл ~--- это текст на английском языке}\EN{Encrypted file is a text in English language}.
\RU{Утилита использует сравнительно мощный алгоритм шифрования, тем не менее,
он был применен с очень грубой ошибкой. И из-за ошибки расшифровать файл вполне возможно с минимумом затрат}
\EN{The utility uses relatively strong cryptosystem, nevertheless, it was implemented with a serious blunder.
Since the mistake present, it is possible to decrypt the file with a little effort.}.

\RU{Попробуйте найти ошибку и расшифровать файл}\EN{Try to find the mistake and decrypt the file}.

\begin{itemize}
\item Windows x86: \url{http://yurichev.com/RE-exercises/3/4/amateur_cryptor.exe}

\item {\RU{Текстовый файл}\EN{Text file}}: \url{http://yurichev.com/RE-exercises/3/4/text_encrypted}
\end{itemize}

\section{\Exercise 3.5}

\RU{Это имитация защиты от копирования использующей ключевой файл}
\EN{This is software copy protection imitation, which uses key file}.
\RU{В ключевом файле имя пользователя и серийный номер}
\EN{The key file contain user (or customer) name and serial number}.

\RU{Задачи две}\EN{There are two tasks}:

\index{tracer}
\begin{itemize}
\item
\RU{(Простая) при помощи \tracer либо иного отладчика, 
заставьте эту программу принимать измененный ключевой файл}\EN{(Easy) with the help of \tracer
or any other debugger, force the program to accept changed key file}.

\item
\RU{(Средняя) ваша задача заключается в том, чтобы изменить в файле имя пользователя на другое, 
но при этом, модифицировать саму программу нельзя}
\EN{(Medium) your goal is to modify user name to another, however, it is not allowed to patch the program}.
\end{itemize}

\begin{itemize}
\item Windows x86: \url{http://yurichev.com/RE-exercises/3/5/super_mega_protection.exe}
\item Linux x86: \url{http://yurichev.com/RE-exercises/3/5/super_mega_protection.tar}
\item \MacOSX (x64) \url{http://yurichev.com/RE-exercises/3/5/super_mega_protection_MacOSX.tar}
\item \RU{Ключевой файл}\EN{Key file}: \url{http://yurichev.com/RE-exercises/3/5/sample.key}
\end{itemize}

\section{\Exercise 3.6}

\RU{Это очень примитивный игрушечный веб-сервер, поддерживающий только статические файлы, без \ac{CGI}, и т.д}
\EN{Here is a very primitive toy web-server, supporting only static files, without \ac{CGI}, etc}.
\RU{В нем сознательно оставлено по крайней мере 4 уязвимости}
\EN{At least 4 vulnerabilities are left here intentionally}.
\RU{Постарайтесь найти их все и использовать для взлома удаленной машины}
\EN{Try to find them all and exploit them in order for breaking into a remote host}.

\begin{itemize}
\item Windows x86: \url{http://yurichev.com/RE-exercises/3/6/webserv_win32.rar}
\item Linux x86: \url{http://yurichev.com/RE-exercises/3/6/webserv_Linux_x86.tar}
\item \MacOSX (x64): \url{http://yurichev.com/RE-exercises/3/6/webserv_MacOSX_x64.tar}
\end{itemize}

\section{\Exercise 3.7}

\index{tracer}
\RU{При помощи \tracer или любого другого win32-отладчика, найдите скрытые мины во время игры,
в стандартной игре Windows MineSweeper}
\EN{With the help of \tracer or any other win32 debugger, reveal hidden mines in the MineSweeper standard Widnows game
during play}.

\RU{Подсказка: в}\EN{Hint:} \cite{trew} \RU{имеются некоторые описания внутренностей игры MineSweeper}
\EN{have some insights about MineSweeper's internals}.

\section{\Exercise 3.8}

\RU{Это достаточно известный алгоритм компрессии данных}\EN{It's a well known data compression algorithm}.
\RU{Но из-за ошибки (или даже опечатки) он разжимает неверно}\EN{However, due to mistake (or typo), 
it decompress incorrectly}.
\RU{В этом можно убедиться на этих примерах}\EN{Here we can see this bug in these examples}.\\
\RU{Это исходный текст}\EN{This is a text used as a source}: 
\url{http://yurichev.com/RE-exercises/3/8/test.txt}\\
\RU{Это корректно сжатый текст}\EN{This is a text compressed correctly}: 
\url{http://yurichev.com/RE-exercises/3/8/test.compressed}\\
\RU{Это некорректно разжатый текст}\EN{This is incorrectly uncompressed text}:
\url{http://yurichev.com/RE-exercises/3/8/test.uncompressed_incorrectly}.\\
\\
\RU{Попробуйте найти и исправить ошибку}\EN{Try to find and fix bug}.
\RU{При некотором упорстве, это можно сделать при помощи модификации исполняемого файла}
\EN{With some effort, it can be done even by patching}.

\begin{itemize}
\item Windows x86: \url{http://yurichev.com/RE-exercises/3/8/compressor_win32.exe}
\item Linux x86: \url{http://yurichev.com/RE-exercises/3/8/compressor_linux86.tar}
\item \MacOSX (x64): \url{http://yurichev.com/RE-exercises/3/8/compressor_MacOSX64.tar}
\end{itemize}

