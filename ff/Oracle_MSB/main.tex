\chapterold{\oracle: \EN{.MSB-files}\RU{.MSB-файлы}\ESph{}\PTBRph{}\PLph{}\ITAph{}\DEph{}\NLph{}}
\myindex{\oracle}
\EN{\epigraph{When working toward the solution of a problem, it always helps if you know the answer.}{Murphy's Laws, Rule of Accuracy}}
\RU{\epigraph{Работая над решением задачи, всегда полезно знать ответ.}{Законы Мерфи, правило точности}}

\RU{Это бинарный файл, содержащий сообщения об ошибках вместе с их номерами.}
\EN{This is a binary file that contains error messages with their corresponding numbers.}
\RU{Давайте попробуем понять его формат и найти способ распаковать его}\EN{Let's try to understand 
its format and find a way to unpack it}.

\RU{В \oracle имеются файлы с сообщениями об ошибках в текстовом виде, так что мы можем сравнивать файлы:
текстовый и запакованный бинарный}\EN{There are \oracle error message files in text form, 
so we can compare the text and packed binary files}
\footnote{\EN{Open-source text files don't exist in \oracle for every .MSB file, so that's why we will work on their file format}
\RU{Текстовые файлы с открытым кодом в \oracle имеются не для каждого .MSB-файла, вот почему мы будем работать над его форматом}}.

\RU{Это начало файла}\EN{This is the beginning of the} ORAUS.MSG \RU{без ненужных комментариев}\EN{text file 
with some irrelevant comments stripped}:

\begin{lstlisting}[caption=\RU{Начало файла}\EN{Beginning of} ORAUS.MSG \RU{без комментариев}\EN{file without comments}]
00000, 00000, "normal, successful completion"
00001, 00000, "unique constraint (%s.%s) violated"
00017, 00000, "session requested to set trace event"
00018, 00000, "maximum number of sessions exceeded"
00019, 00000, "maximum number of session licenses exceeded"
00020, 00000, "maximum number of processes (%s) exceeded"
00021, 00000, "session attached to some other process; cannot switch session"
00022, 00000, "invalid session ID; access denied"
00023, 00000, "session references process private memory; cannot detach session"
00024, 00000, "logins from more than one process not allowed in single-process mode"
00025, 00000, "failed to allocate %s"
00026, 00000, "missing or invalid session ID"
00027, 00000, "cannot kill current session"
00028, 00000, "your session has been killed"
00029, 00000, "session is not a user session"
00030, 00000, "User session ID does not exist."
00031, 00000, "session marked for kill"
...
\end{lstlisting}

\RU{Первое число\EMDASH{}это код ошибки}\EN{The first number is the error code}.
\RU{Второе это, вероятно, могут быть дополнительные флаги}\EN{The second is perhaps maybe some additional flags}.

\clearpage
\RU{Давайте откроем бинарный файл}\EN{Now let's open the} ORAUS.MSB 
\RU{и найдем эти текстовые строки}\EN{binary file and find these text strings}. 
\RU{И вот они}\EN{And there are}:

\begin{figure}[H]
\centering
\myincludegraphics{ff/Oracle_MSB/1.png}
\caption{Hiew: \RU{первый блок}\EN{first block}}
\label{fig:oracle_MSB_1}
\end{figure}

\RU{Мы видим текстовые строки (включая те, с которых начинается файл ORAUS.MSG) перемежаемые с какими-то
бинарными значениями}\EN{We see the text strings (including those from the beginning of the ORAUS.MSG file) 
interleaved with some binary values}.
\RU{Мы можем довольно быстро обнаружить что главная часть бинарного файла поделена на блоки размером 0x200 (512)
байт}\EN{By quick investigation, we can see that main part of the binary file is divided by blocks of 
size 0x200 (512) bytes}.

\clearpage
\RU{Посмотрим содержимое первого блока}\EN{Let's see the contents of the first block}:

\begin{figure}[H]
\centering
\myincludegraphics{ff/Oracle_MSB/2.png}
\caption{Hiew: \RU{первый блок}\EN{first block}}
\label{fig:oracle_MSB_2}
\end{figure}

\RU{Мы видим тексты первых сообщений об ошибках}\EN{Here we see the texts of the first messages errors}.
\RU{Что мы видим еще, так это то, что здесь нет нулевых байтов между сообщениями}\EN{What we also see is 
that there are no zero bytes between the error messages}.
\RU{Это значит, что это не оканчивающиеся нулем Си-строки}\EN{This implies that these are not null-terminated C strings}.
\RU{Как следствие, длина каждого сообщения об ошибке должна быть как-то закодирована}\EN{As a consequence, 
the length of each error message must be encoded somehow}.
\RU{Попробуем также найти номера ошибок}\EN{Let's also try to find the error numbers}.
\RU{Файл}\EN{The} ORAUS.MSG \RU{начинается с таких}\EN{files starts with these}: 
0, 1, 17 (0x11), 18 (0x12), 19 (0x13), 20 (0x14), 21 (0x15), 22 (0x16), 23 (0x17), 24 (0x18)...
\RU{Найдем эти числа в начале блока и отметим их красными линиями}\EN{We will find these numbers in the beginning 
of the block and mark them with red lines}.
\RU{Период между кодами ошибок 6 байт}\EN{The period between error codes is 6 bytes}.
\RU{Это значит, здесь, наверное, 6 байт информации выделено для каждого сообщения об ошибке.}
\EN{This implies that there are probably 6 bytes of information allocated for each error message.}

\RU{Первое 16-битное значение (здесь 0xA или 10) означает количество сообщений в блоке: это можно проверить глядя на другие блоки.}%
\EN{The first 16-bit value (0xA here or 10) mean the number of messages in each block: this can be checked by investigating other blocks.}
\RU{Действительно: сообщения об ошибках имеют произвольный размер}\EN{Indeed: the error messages have arbitrary size}. 
\RU{Некоторые длиннее, некоторые короче}\EN{Some are longer, some are shorter}. 
\RU{Но длина блока всегда фиксирована, следовательно, никогда не знаешь, сколько сообщений можно запаковать
в каждый блок}\EN{But block size is always fixed, hence,
you never know how many text messages can be packed in each block}.

\RU{Как мы уже отметили, так как это не оканчивающиеся нулем Си-строки, длина строки должна быть закодирована где-то.}%
\EN{As we already noted, since these are not null-terminating C strings, their size must be encoded somewhere.}
\RU{Длина первой строки}\EN{The size of the first string} \q{normal, successful completion} \RU{это}\EN{is} 
29 (0x1D) \RU{байт}\EN{bytes}.
\RU{Длина второй строки}\EN{The size of the second string} \q{unique constraint (\%s.\%s) violated} 
\RU{это}\EN{is} 34 (0x22) \RU{байт}\EN{bytes}.
\EN{We can't find these values (0x1D or/and 0x22) in the block.}%
\RU{Мы не можем отыскать этих значений (0x1D или/и 0x22) в блоке.}

\RU{А вот еще кое-что}\EN{There is also another thing}.
\oracle \RU{должен как-то определять позицию строки, которую он должен загрузить, верно}
\EN{has to determine the position of the string it needs to load in the block, right}?
\RU{Первая строка}\EN{The first string} \q{normal, successful completion} \RU{начинается с позиции}\EN{starts 
at  position} 0x1444 (\RU{если считать с начала бинарного файла}\EN{if we count starting at the beginning of the file}) \RU{или с}\EN{or at} 0x44 (\RU{от начала блока}\EN{from the block's start}).
\RU{Вторая строка}\EN{The second string} \q{unique constraint (\%s.\%s) violated} 
\RU{начинается с позиции}\EN{starts at position} 0x1461 (\RU{от начала файла}\EN{from the
file's start}) \RU{или с}\EN{or at} 0x61 (\RU{считая с начала блока}\EN{from the at the block's start}).
\RU{Эти числа}\EN{These numbers} (0x44 \AndENRU 0x61) \RU{нам знакомы}\EN{are familiar somehow}! 
\RU{Мы их можем легко отыскать в начале блока}\EN{We can clearly see them at the start of the block}.

\RU{Так что, каждый 6-байтный блок это}\EN{So, each 6-byte block is}:

\begin{itemize}
\item 16-\RU{битный номер ошибки}\EN{bit error number}; 
\item 16-\RU{битный ноль (может быть, дополнительные флаги}\EN{bit zero (maybe additional flags)}; 
\item 16-\RU{битная начальная позиция текстовой строки внутри текущего блока}\EN{bit starting position of 
the text string within the current block}.
\end{itemize}

\RU{Мы можем быстро проверить остальные значения чтобы удостовериться в своей правоте}%
\EN{We can quickly check the other values and be sure our guess is correct}.
\RU{И здесь еще последний \q{пустой} 6-байтный блок с нулевым номером ошибки и начальной позицией за последним
символом последнего сообщения об ошибке.}\EN{And there is also the last \q{dummy} 6-byte block 
with an error number of zero and starting position beyond the last error message's last character.}
\RU{Может быть именно так и определяется длина сообщения}\EN{Probably that's how text message length is 
determined}?
\RU{Мы просто перебираем 6-байтные блоки в поисках нужного номера ошибки, затем
мы узнаем позицию текстовой строки, затем мы узнаем позицию следующей текстовой строки глядя на
следующий 6-байтный блок!}\EN{We just enumerate 6-byte blocks to find the error number
we need, then we get the text string's position, then we get the position of the text string by looking at the next
6-byte block!}
\RU{Так мы определяем границы строки}\EN{This way we determine the string's boundaries}!
\RU{Этот метод позволяет сэкономить место в файле не записывая длину строки}\EN{This method allows to 
save some space by not saving the text string's size in the file}!
\RU{Нельзя сказать, что экономия памяти большая, но это интересный трюк.}%
\EN{It's not possible to say it saves a lot of space, but it's a clever trick}.

\clearpage
\RU{Вернемся к заголовку .MSB-файла}\EN{Let's back to the header of .MSB-file}:

\begin{figure}[H]
\centering
\myincludegraphics{ff/Oracle_MSB/3.png}
\caption{Hiew: \RU{заголовок файла}\EN{file header}}
\label{fig:oracle_MSB_3}
\end{figure}

\RU{Теперь мы можем быстро найти количество блоков (отмечено красным)}\EN{Now we can quickly find the number of blocks in the file 
(marked by red)}.
\RU{Проверяем другие .MSB-файлы и оказывается что это справедливо для всех}\EN{We can checked other .MSB-files and we see that it's true 
for all of them}.
\RU{Здесь есть много других значений, но мы не будем разбираться с ними, так как наша задача (утилита для распаковки) уже решена.}
\EN{There are a lot of other values, but we will not investigate them, since our job (an unpacking utility) was done.}
\RU{А если бы мы писали запаковщик .MSB-файлов, тогда нам наверное пришлось бы понять, зачем нужны остальные.}
\EN{If we have to write a .MSB file packer, we would probably need to understand the meaning of the other values.}

\clearpage
\RU{Тут еще есть таблица после заголовка, вероятно, содержащая 16-битные значения}\EN{There is also a 
table that came after the header which probably contains 16-bit values}:

\begin{figure}[H]
\centering
\myincludegraphics{ff/Oracle_MSB/4.png}
\caption{Hiew: \RU{таблица }last\_errnos\EN{ table}}
\label{fig:oracle_MSB_4}
\end{figure}

\RU{Их длина может быть определена визуально (здесь нарисованы красные линии).}%
\EN{Their size can be determined visually (red lines are drawn here).}
\RU{Когда мы сдампили эти значения, мы обнаружили, что каждое 16-битное число\EMDASH{}это последний код ошибки для каждого блока.}%
\EN{While dumping these values, we have found that each 16-bit number is the last error code for each block.}

\RU{Так вот как \oracle быстро находит сообщение об ошибке}\EN{So that's how \oracle quickly finds the error message}:

\begin{itemize}
\item \RU{загружает таблицу, которую мы назовем}\EN{load a table we will call} last\_errnos 
(\RU{содержащую последний номер ошибки для каждого блока}\EN{that contains the last error number for each block});
\item \RU{находит блок содержащий код ошибки, полагая что все коды ошибок увеличиваются и внутри каждого блока
и также в файле}\EN{find a block that contains the error code we need, assuming all error codes 
increase across each block and across the file as well};
\item \RU{загружает соответствующий блок}\EN{load the specific block};
\item \RU{перебирает 6-байтные структуры, пока не найдется соответствующий номер ошибки}\EN{enumerate the 6-byte 
structures until the specific error number is found};
\item \RU{находит позицию первого символа из текущего 6-байтного блока}\EN{get the position of the first 
character from the current 6-byte block};
\item \RU{находит позицию последнего символа из следующего 6-байтного блока}\EN{get the position of the last 
character from the next 6-byte block};
\item \RU{загружает все символы сообщения в этих пределах}\EN{load all characters of the message in this range}.
\end{itemize}

\RU{Это программа на Си которую мы написали для распаковки .MSB-файлов}
\EN{This is C program that we wrote which unpacks .MSB-files}:
\href{http://go.yurichev.com/17213}{beginners.re}.

\RU{И еще два файла которые были использованы в этом примере}
\EN{There are also the two files which were used in the example} 
(\oracle 11.1.0.6):
\href{http://go.yurichev.com/17214}{beginners.re},
\href{http://go.yurichev.com/17215}{beginners.re}.

\sectionold{\RU{Вывод}\EN{Summary}}

\RU{Этот метод, наверное, слишком олд-скульный для современных компьютеров}\EN{The method is probably too 
old-school for modern computers}.
\RU{Возможно, формат этого файла был разработан в середине 1980-х кем-то, кто программировал для мейнфреймов,
учитывая экономию памяти и места на дисках}\EN{Supposedly, this file format was developed in the mid-80's by 
someone who also coded for \IT{big iron} with
memory/disk space economy in mind}.
\RU{Тем не менее, это интересная (хотя и простая) задача на разбор проприетарного формата файла без
заглядывания в код \oracle}\EN{Nevertheless, it was an interesting and yet easy task 
to understand a proprietary file format without looking into \oracle's code}.
