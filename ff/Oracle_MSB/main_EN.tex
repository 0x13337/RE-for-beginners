\section{\oracle: .MSB-files\ESph{}\PTBRph{}\PLph{}\ITAph{}\DEph{}\NLph{}}
\myindex{\oracle}
\epigraph{When working toward the solution of a problem, it always helps if you know the answer.}{Murphy's Laws, Rule of Accuracy}

This is a binary file that contains error messages with their corresponding numbers.
Let's try to understand 
its format and find a way to unpack it.

There are \oracle error message files in text form, 
so we can compare the text and packed binary files
\footnote{Open-source text files don't exist in \oracle for every .MSB file, so that's why we will work on their file format}.

This is the beginning of the ORAUS.MSG text file with some irrelevant comments stripped:

\begin{lstlisting}[caption=Beginning of ORAUS.MSG file without comments]
00000, 00000, "normal, successful completion"
00001, 00000, "unique constraint (%s.%s) violated"
00017, 00000, "session requested to set trace event"
00018, 00000, "maximum number of sessions exceeded"
00019, 00000, "maximum number of session licenses exceeded"
00020, 00000, "maximum number of processes (%s) exceeded"
00021, 00000, "session attached to some other process; cannot switch session"
00022, 00000, "invalid session ID; access denied"
00023, 00000, "session references process private memory; cannot detach session"
00024, 00000, "logins from more than one process not allowed in single-process mode"
00025, 00000, "failed to allocate %s"
00026, 00000, "missing or invalid session ID"
00027, 00000, "cannot kill current session"
00028, 00000, "your session has been killed"
00029, 00000, "session is not a user session"
00030, 00000, "User session ID does not exist."
00031, 00000, "session marked for kill"
...
\end{lstlisting}

The first number is the error code.
The second is perhaps maybe some additional flags.

\clearpage
Now let's open the ORAUS.MSB 
binary file and find these text strings. 
And there are:

\begin{figure}[H]
\centering
\myincludegraphics{ff/Oracle_MSB/1.png}
\caption{Hiew: first block}
\label{fig:oracle_MSB_1}
\end{figure}

We see the text strings (including those from the beginning of the ORAUS.MSG file) 
interleaved with some binary values.
By quick investigation, we can see that main part of the binary file is divided by blocks of 
size 0x200 (512) bytes.

\clearpage
Let's see the contents of the first block:

\begin{figure}[H]
\centering
\myincludegraphics{ff/Oracle_MSB/2.png}
\caption{Hiew: first block}
\label{fig:oracle_MSB_2}
\end{figure}

Here we see the texts of the first messages errors.
What we also see is 
that there are no zero bytes between the error messages.
This implies that these are not null-terminated C strings.
As a consequence, 
the length of each error message must be encoded somehow.
Let's also try to find the error numbers.
The ORAUS.MSG files starts with these: 
0, 1, 17 (0x11), 18 (0x12), 19 (0x13), 20 (0x14), 21 (0x15), 22 (0x16), 23 (0x17), 24 (0x18)...
We will find these numbers at the beginning 
of the block and mark them with red lines.
The period between error codes is 6 bytes.

This implies that there are probably 6 bytes of information allocated for each error message.

The first 16-bit value (0xA here or 10) mean the number of messages in each block: this can be checked by investigating other blocks.
Indeed: the error messages have arbitrary size. 
Some are longer, some are shorter. 
But block size is always fixed, hence,
you never know how many text messages can be packed in each block.

As we already noted, since these are not null-terminating C strings, their size must be encoded somewhere.
The size of the first string \q{normal, successful completion} is 
29 (0x1D) bytes.
The size of the second string \q{unique constraint (\%s.\%s) violated} 
is 34 (0x22) bytes.
We can't find these values (0x1D or/and 0x22) in the block.%

There is also another thing.
\oracle 
has to determine the position of the string it needs to load in the block, right?
The first string \q{normal, successful completion} starts 
at  position 0x1444 (if we count starting at the beginning of the file) or at 0x44 (from the block's start).
The second string \q{unique constraint (\%s.\%s) violated} 
starts at position 0x1461 (from the
file's start) or at 0x61 (from the at the block's start).
These numbers (0x44 and 0x61) are familiar somehow! 
We can clearly see them at the start of the block.

So, each 6-byte block is:

\begin{itemize}
\item 16-bit error number; 
\item 16-bit zero (maybe additional flags); 
\item 16-bit starting position of 
the text string within the current block.
\end{itemize}

We can quickly check the other values and be sure our guess is correct.
And there is also the last \q{dummy} 6-byte block 
with an error number of zero and starting position beyond the last error message's last character.
Probably that's how text message length is 
determined?
We just enumerate 6-byte blocks to find the error number
we need, then we get the text string's position, then we get the position of the text string by looking at the next
6-byte block!
This way we determine the string's boundaries!
This method allows to 
save some space by not saving the text string's size in the file!

It's not possible to say it saves a lot of space, but it's a clever trick.

\clearpage
Let's back to the header of .MSB-file:

\begin{figure}[H]
\centering
\myincludegraphics{ff/Oracle_MSB/3.png}
\caption{Hiew: file header}
\label{fig:oracle_MSB_3}
\end{figure}

Now we can quickly find the number of blocks in the file (marked by red).
We can checked other .MSB-files and we see that it's true for all of them.

There are a lot of other values, but we will not investigate them, since our job (an unpacking utility) is done.

If we have to write a .MSB file packer, we would probably have to understand the meaning of the other values.

\clearpage
There is also a table that came after the header which probably contains 16-bit values:

\begin{figure}[H]
\centering
\myincludegraphics{ff/Oracle_MSB/4.png}
\caption{Hiew: last\_errnos table}
\label{fig:oracle_MSB_4}
\end{figure}

Their size can be determined visually (red lines are drawn here).

While dumping these values, we have found that each 16-bit number is the last error code for each block.

So that's how \oracle quickly finds the error message:

\begin{itemize}
\item load a table we will call last\_errnos (that contains the last error number for each block);

\item find a block that contains the error code we need, assuming all error codes 
increase across each block and across the file as well;

\item load the specific block;

\item enumerate the 6-byte structures until the specific error number is found;

\item get the position of the first character from the current 6-byte block;

\item get the position of the last character from the next 6-byte block;

\item load all characters of the message in this range.
\end{itemize}

This is C program that we wrote which unpacks .MSB-files:
\href{http://go.yurichev.com/17213}{beginners.re}.

There are also the two files which were used in the example 
(\oracle 11.1.0.6):
\href{http://go.yurichev.com/17214}{beginners.re},
\href{http://go.yurichev.com/17215}{beginners.re}.

\subsection{Summary}

The method is probably too 
old-school for modern computers.
Supposedly, this file format was developed in the mid-80's by 
someone who also coded for \IT{big iron} with
memory/disk space economy in mind.
Nevertheless, it has been an interesting and yet easy task 
to understand a proprietary file format without looking into \oracle's code.
