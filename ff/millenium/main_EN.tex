\section{Millenium game save file}
\label{Millenium_DOS_game}
\myindex{MS-DOS}

The \q{Millenium Return to Earth} 
is an ancient DOS game (1991), that allows you to mine resources, build ships,
equip them on other planets, and so on\footnote{It can be downloaded for free
\href{http://go.yurichev.com/17316}{here}}.

Like many other games, it allows you to save all game state into a file.

Let's see if we can find something in it.

\clearpage
So there is a mine in the game.
Mines at some planets 
work faster, or slower on others. 
The set of resources is also different.

Here we can see what resources are mined at the time: 

\begin{figure}[H]
\centering
\myincludegraphics{ff/millenium/1.png}
\caption{Mine: state 1}
\label{fig:mill_1}
\end{figure}

Let's save a game state.
This is a file of size 9538 bytes.

Let's wait some \q{days} here in the game, and now we've got more resources from the mine:

\begin{figure}[H]
\centering
\myincludegraphics{ff/millenium/2.png}
\caption{Mine: state 2}
\label{fig:mill_2}
\end{figure}

Let's sav game state again.

Now let's try to just do binary comparison of the save files using the simple DOS/Windows FC utility:

\lstinputlisting{ff/millenium/fc_result.txt}

The output is incomplete here, there are more differences, but we will cut result to show the most interesting.

In the first state, we have 14 \q{units} of hydrogen and 102 \q{units} of oxygen.

We have 22 and 155 \q{units} respectively in the second state.
If these values are saved into 
the save file, we would see this in the difference.
And indeed we do. 
There is 0x0E (14) at position 0xBDA and this value is 
0x16 (22) in the new version of the file.
This is probably hydrogen.
There is 0x66 (102) at position 0xBDC in the old 
version and 0x9B (155) in the new version of the file. 
This seems to be the oxygen.

Both files are available on the website for those who wants to inspect them (or experiment) more: 
\href{http://go.yurichev.com/17212}{beginners.re}.

\clearpage
Here is the new version of file opened in Hiew, we marked the values related to the resources mined in the game: 

\begin{figure}[H]
\centering
\myincludegraphics{ff/millenium/hiew3.png}
\caption{Hiew: state 1}
\label{fig:mill_hiew3}
\end{figure}

Let's check each, and these are.

These are clearly 16-bit values: not a strange thing for 16-bit DOS software where the \Tint type has 16-bit width.

\clearpage
Let's check our assumptions.
We will write the 1234 (0x4D2) value at the first position (this must be hydrogen):

\begin{figure}[H]
\centering
\myincludegraphics{ff/millenium/hiew4.png}
\caption{Hiew: let's write 1234 (0x4D2) there}
\label{fig:mill_hiew4}
\end{figure}

Then we will load the changed file in the game and took a look at mine statistics:

\begin{figure}[H]
\centering
\myincludegraphics{ff/millenium/5.png}
\caption{Let's check for hydrogen value}
\label{fig:mill_5}
\end{figure}

So yes, this is it.

\clearpage
Now let's try to 
finish the game as soon as possible, set the maximal values everywhere:

\begin{figure}[H]
\centering
\myincludegraphics{ff/millenium/hiew7.png}
\caption{Hiew: let's set maximal values}
\label{fig:mill_hiew7}
\end{figure}

0xFFFF is 65535, so yes, we now have a 
lot of resources:

\begin{figure}[H]
\centering
\myincludegraphics{ff/millenium/6.png}
\caption{All resources are 65535 (0xFFFF) indeed}
\label{fig:mill_6}
\end{figure}

\clearpage
Let's skip some \q{days} in the game and oops! 
We have a lower amount of some resources:

\begin{figure}[H]
\centering
\myincludegraphics{ff/millenium/8.png}
\caption{Resource variables overflow}
\label{fig:mill_8}
\end{figure}

That's just overflow. 

The game's developer supposedly didn't think about such high amounts of resources,
so there are probably no overflow checks, but the mine is \q{working} in the game, resources are added,
hence the overflows.
Apparently, it is a bad idea to be that greedy.

There are probably a lot of more values 
saved in this file.

So this is very simple method of cheating in games.
High score files often can be easily 
patched like that.

More about files and memory snapshots comparing: 
\myref{snapshots_comparing}.
