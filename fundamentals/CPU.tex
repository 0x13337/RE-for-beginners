\chapter{CPU}

\section{%
	\RU{Предсказатели переходов}%
	\EN{Branch predictors}%
	\ES{Predictores del saltos}%
	\PTBRph{}%
	\PLph{}%
	\ITAph{}%
}
\label{branch_predictors}

\RU{Некоторые современные компиляторы пытаются избавиться от инструкций условных переходов.}%
\EN{Some modern compilers try to get rid of conditional jump instructions.}%
\ES{Algunos compiladores modernos intentan deshacerse de las instrucciones de saltos condicionales.}%
\PTBRph{}%
\PLph{}%
\ITAph{}
\RU{Примеры в этой книге:}%
\EN{Examples in this book are:}%
\ES{Ejemplos en este libro son:}%
\PTBRph{}%
\PLph{}%
\ITAph{}
\myref{subsec:jcc_ARM}, \myref{chap:cond}, \myref{subsec:popcnt}.

\RU{Это потому что предсказатель переходов далеко не всегда работает идеально, поэтому, компиляторы и стараются
реже использовать переходы, если возможно.}%
\EN{This is because the branch predictor is not always perfect, so the compilers try to do 
without conditional jumps, if possible.}%
\ES{Esto se debe a que el predictor de saltos no siempre es perfecto, por lo tanto los compiladores
tratan de evitar los saltos condicionales, de ser posible.}%
\PTBRph{}%
\PLph{}%
\ITAph{}

\index{x86!\Instructions!CMOVcc}
\index{ARM!\Instructions!ADRcc}
\RU{Одна из возможностей --- это условные инструкции в ARM (как ADRcc), а еще инструкция CMOVcc в x86.}%
\EN{Conditional instructions in ARM (like ADRcc) are one way, another one is the CMOVcc x86 instruction.}%
\ES{Las instrucciones condicionales en ARM (como ADRcc) son una forma de hacerlo, otra es el conjunto de instrucciones x86 CMOVcc.}%
\PTBRph{}%
\PLph{}%
\ITAph{}

\section{%
	\RU{Зависимости между данными}%
	\EN{Data dependencies}%
	\ES{Dependencias de datos}%
	\PTBRph{}%
	\PLph{}%
	\ITAph{}%
}

\RU{Современные процессоры способны исполнять инструкции одновременно (\ac{OOE}), но для этого,
внутри такой группы, результат одних не должен влиять на работу других.}%
\EN{Modern CPUs are able to execute instructions simultaneously (\ac{OOE}), but in order to do so,
the results of one instruction in a group must not influence the execution of others.}%
\ES{Los CPUs modernos son capaces de ejecutar instrucciones de manera simultanea (\ac{OOE}), pero para
poder lograrlo, los resultados de una instrucci\'on en un grupo no debe influenciar la ejecuci\'on de otras.}%
\PTBRph{}%
\PLph{}%
\ITAph{}
\RU{Следовательно, компилятор старается использовать инструкции с наименьшим влиянием на состояние процессора.}%
\EN{Hence, the compiler endeavors to use instructions with minimal influence on the CPU state.}%
\ES{Como consecuencia, el compilador se esfuerza en hacer uso de instrucciones que tengan una influencia m\'inima en el estado del CPU}%
\PTBRph{}%
\PLph{}%
\ITAph{}

\RU{Вот почему инструкция \LEA в x86 такая популярная --- 
потому что она не модифицирует флаги процессора,
а прочие арифметические инструкции --- модифицируют.}%
\EN{That's why the \LEA instruction is so popular, because it does not modify CPU flags, while
other arithmetic instructions does.}%
\ES{Por eso la instrucci\'on \LEA es tan popular, porque no modifica las banderas del CPU,
mientras que otras instrucciones aritm\'eticas s\'i lo hacen.}%
\PTBRph{}%
\PLph{}%
\ITAph{}
