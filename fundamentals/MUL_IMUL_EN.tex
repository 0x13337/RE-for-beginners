% TODO translate
\section{Using IMUL over MUL}
\label{IMUL_over_MUL}

\myindex{x86!\Instructions!MUL}
\myindex{x86!\Instructions!IMUL}
Example like \lstref{unsigned_multiply_C} where two unsigned values are multiplied compiles into \lstref{unsigned_multiply_lst} where \IMUL is used instead of \MUL.

This is important property of both \MUL and \IMUL instructions.
First of all, they both produce 64-bit value if two 32-bit values are multiplied, or 128-bit value if two 64-bit values are multiplied (biggest possible \gls{product} in 32-bit environment is \GTT{0xffffffff*0xffffffff=0xfffffffe00000001}).
But \CCpp standards have no way to access higher half of result, and a \gls{product} always has the same size as multiplicands. % TODO \gls{}?
And both \MUL and \IMUL instructions works in the same way if higher half is ignored, i.e., lower half is the same.
This is important property of \q{two's complement} way of representing signed numbers.

So \CCpp compiler can use any of these instructions.

But \IMUL is more versatile than MUL because it can take any register(s) as source, while \MUL requires one of multiplicands stored in \AX/\EAX/\RAX register.
Even more than that: \MUL stores result in \GTT{EDX:EAX} pair in 32-bit environment, or \GTT{RDX:RAX} in 64-bit one, so it always calculates the whole result.
On contrary, it's possible to set a single destination register while using \IMUL instead of pair, and then \ac{CPU} will calculate only lower half, which works faster [see Torborn Granlund, \IT{Instruction latencies and throughput for AMD and Intel x86 processors}\footnote{\url{http://yurichev.com/mirrors/x86-timing.pdf}]}).

Given than, \CCpp compilers may generate \IMUL instruction more often then \MUL.

\myindex{Compiler intrinsic}
Nevertheless, using compiler intrinsic, it's still possible to do unsigned multiplication and get \IT{full} result.
This is sometimes called \IT{extended multiplication}.
MSVC has intrinsic for this called \IT{\_\_emul}\footnote{\url{https://msdn.microsoft.com/en-us/library/d2s81xt0(v=vs.80).aspx}} and another one: \IT{\_umul128}\footnote{\url{https://msdn.microsoft.com/library/3dayytw9%28v=vs.100%29.aspx}}.
GCC offer \IT{\_\_int128} data type, and if 64-bit multiplicands are first promoted to 128-bit ones,
then a \gls{product} is stored into another \IT{\_\_int128}, then result is shifted by 64-bit right, you'll get higher half of result\footnote{Example: \url{http://stackoverflow.com/a/13187798}}.

\subsection{MulDiv() function in Windows}
\myindex{Windows!Win32!MulDiv()}

Windows has MulDiv() function
\footnote{\url{https://msdn.microsoft.com/en-us/library/windows/desktop/aa383718(v=vs.85).aspx}},
fused multiply/divide function, it multiplies two 32-bit integers into intermediate 64-bit value
and then divides it by a third 32-bit integer.
It is easier than to use two compiler intrinsic, so Microsoft developers made a special function for it.
And it seems, this is busy function, judging by its usage.

