\section{Population count}
\label{POPCNT}

\INS{POPCNT} instruction is population count (AKA Hamming weight).
It just counts number of bits set in an input value.

As a side effect, \INS{POPCNT} instruction (or operation) can be used to determine, if the value has $2^n$ form.
Since, $2^n$ number always has just one single bit, \INS{POPCNT}'s result will always be just 1.

For example, I once wrote a base64 strings scanner for hunting something interesting in binary files. % TODO URL
And there is a lot of garbage and false positives, so I add an option to filter out data blocks which has size of $2^n$ bytes
(i.e., 256 bytes, 512, 1024, etc).
The size of block is checked just like this:

\begin{lstlisting}
if (popcnt(size)==1)
	// OK
...
\end{lstlisting}

The instruction is also known as \q{\ac{NSA} instruction} due to rumors:

\begin{framed}
\begin{quotation}
  This branch of cryptography is fast-paced and very politically charged.
  Most designs are secret; a majority of military encryptions systems in use today are 
  based on LFSRs. 
  In fact, most Cray computers (Cray 1, Cray X-MP, Cray Y-MP) have a rather curious 
  instruction generally known as “population count.” It counts the 1 bits in a register 
  and can be used both to efficiently calculate the Hamming distance between two binary 
  words and to implement a vectorized version of a LFSR. I’ve heard this called the canonical 
  NSA instruction, demanded by almost all computer contracts.
\end{quotation}
\end{framed}
\InSqBrackets{\Schneier{}}

