\chapter{Endianness\RU{ (порядок байт)}}
\label{sec:endianness}

\RU{Endianness (порядок байт) это способ представления чисел в памяти.}
\EN{The endianness is a way of representing values in memory.}%
\ES{Endianness es una forma de representar valores en la memoria.}%
\PTBRph{}%
\DEph{}\PLph{}%
\ITAph{}

\section{Big-endian\RU{ (от старшего к младшему)}}

\RU{Число}%
\EN{The}%
\ES{El valor}%
\PTBRph{}%
\DEph{}\PLph{}%
\ITAph{} 
\TT{0x12345678} 
\RU{представляется в памяти так:}%
\EN{value is represented in memory as:}%
\ES{es representado en la memoria como:}%
\PTBRph{}%
\DEph{}\PLph{}%
\ITAph{}

\begin{center}
\begin{tabular}{ | l | l | }
\hline
\cellcolor{blue!25}
\RU{адрес в памяти}%
\EN{address in memory}%
\ES{direcci\'on en memoria}%
\PTBRph{}%
\DEph{}\PLph{}%
\ITAph{}
&
\cellcolor{blue!25}
\RU{значение байта}%
\EN{byte value}%
\ES{valor del byte}%
\PTBRph{}%
\DEph{}\PLph{}%
\ITAph{}
\\
\hline
+0 & 0x12 \\
\hline
+1 & 0x34 \\
\hline
+2 & 0x56 \\
\hline
+3 & 0x78 \\
\hline
\end{tabular}
\end{center}

\RU{CPU с таким порядком включают в себя}%
\EN{Big-endian CPUs include}%
\ES{Entre los CPU big-endian se encuentran}%
\PTBRph{}%
\DEph{}\PLph{}%
\ITAph{}
Motorola 68k, IBM POWER.

\section{Little-endian\RU{ (от младшего к старшему)}}

\RU{Число}%
\EN{The}%
\ES{El valor}%
\PTBRph{}%
\DEph{}\PLph{}%
\ITAph{}
\TT{0x12345678} 
\RU{представляется в памяти так:}%
\EN{value is represented in memory as:}%
\ES{es representado en la memoria como:}%
\PTBRph{}%
\DEph{}\PLph{}%
\ITAph{}

\begin{center}
\begin{tabular}{ | l | l | }
\hline
\cellcolor{blue!25}
\RU{адрес в памяти}%
\EN{address in memory}%
\ES{Direcci\'on en memoria}%
\PTBRph{}%
\DEph{}\PLph{}%
\ITAph{}
&
\cellcolor{blue!25}
\RU{значение байта}%
\EN{byte value}%
\ES{valor del byte}%
\PTBRph{}%
\DEph{}\PLph{}%
\ITAph{}
\\
\hline
+0 & 0x78 \\
\hline
+1 & 0x56 \\
\hline
+2 & 0x34 \\
\hline
+3 & 0x12 \\
\hline
\end{tabular}
\end{center}

\RU{CPU с таким порядком байт включают в себя}%
\EN{Little-endian CPUs include}%
\ES{Entre los CPU little-endian tenemos el}%
\PTBRph{}%
\DEph{}\PLph{}%
\ITAph{}
Intel x86.

\section{\Example}

\RU{Возьмем big-endian Linux для MIPS заинсталированный в QEMU}%
\EN{Let's take big-endian MIPS Linux installed and ready in QEMU}%
\ES{Tomemos el MIPS big-endian para Linux instalado y listo en QEMU}%
\PTBRph{}%
\DEph{}\PLph{}%
\ITAph{}
\footnote{%
	\RU{Доступен для скачивания здесь:}%
	\EN{Available for download here:}%
	\ES{Disponible para descargar aqu\'i:}%
	\PTBRph{}%
	\DEph{}\PLph{}%
	\ITAph{}
	\url{http://go.yurichev.com/17008}%
}.

\RU{И скомпилируем этот простой пример:}%
\EN{And let's compile this simple example:}%
\ES{Y compilemos este ejemplo sencillo:}%
\PTBRph{}%
\DEph{}\PLph{}%
\ITAph{}

\begin{lstlisting}
#include <stdio.h>

int main()
{
	int v, i;

	v=123;

	printf ("%02X %02X %02X %02X\n", 
		*(char*)&v,
		*(((char*)&v)+1),
		*(((char*)&v)+2),
		*(((char*)&v)+3));
};
\end{lstlisting}

\RU{И запустим его:}%
\EN{After running it we get:}%
\ES{Despu\'es de correrlo obtenemos:}%
\PTBRph{}%
\DEph{}\PLph{}%
\ITAph{}

\begin{lstlisting}
root@debian-mips:~# ./a.out 
00 00 00 7B
\end{lstlisting}

\RU{Это оно и есть.}%
\EN{That is it.}%
\ES{Eso es todo.}%
\PTBRph{}%
\DEph{}\PLph{}%
\ITAph{}
0x7B
\RU{это}%
\EN{is}%
\ES{es}%
\PTBRph{}%
\DEph{}\PLph{}%
\ITAph{}
123
\RU{в десятичном виде.}%
\EN{in decimal.}%
\ES{en decimal.}%
\PTBRph{}%
\DEph{}\PLph{}%
\ITAph{}
\RU{В little-endian-архитектуре, 7B это первый байт (вы можете это проверить в x86 или x86-64),
но здесь он последний, потому что старший байт идет первым.}%
\EN{In little-endian architectures, 7B is the first byte (you can check on x86 or x86-64), 
but here it is the last one, because the highest byte goes first.}%
\ES{En las arquitecturas little-endian, 7B es el primer byte (puedes checarlo en x86 o x86-64),
pero aqu\'i es el \'ultimo, porque el byte m\'as alto va primero.}%
\PTBRph{}%
\DEph{}\PLph{}%
\ITAph{}

\RU{Вот почему имеются разные дистрибутивы Linux для MIPS}%
\EN{That's why there are separate Linux distributions for MIPS}%
\ES{Por eso hay distribuciones separadas de Linux para MIPS}%
\PTBRph{}%
\DEph{}\PLph{}%
\ITAph{}
(\q{mips} (big-endian) \AndENRU \q{mipsel} (little-endian)).
\RU{Программа скомпилированная для одного соглашения об endiannes, не сможет работать в OS использующей
другое соглашение.}%
\EN{It is impossible for a binary compiled for one endianness to work on an OS with different endianness.}%
\ES{Es imposible que un binario compilado para un endianness trabaje en un SO con diferente endianness.}%
\PTBRph{}%
\DEph{}\PLph{}%
\ITAph{}
\\
\\
\RU{Еще один пример связанный с big-endian в MIPS в этой книге:}%
\EN{There is another example of MIPS big-endiannes in this book:}%
\ES{En este libro hay otro ejemplo de big-endianness en MIPS:}%
\PTBRph{}%
\DEph{}\PLph{}%
\ITAph{}
\myref{MIPS_structure_big_endian}.

\section{Bi-endian\RU{ (переключаемый порядок)}}

\RU{CPU поддерживающие оба порядка, и его можно переключать, включают в себя}%
\EN{CPUs that may switch between endianness are}%
\ES{CPUs que puede cambiar entre endianness son}%
\PTBRph{}%
\DEph{}\PLph{}%
\ITAph{}
ARM, PowerPC, SPARC, MIPS, \ac{IA64}, \etc{}.

\section{%
	\RU{Конвертирование}%
	\EN{Converting data}%
	\ES{Convirtiendo datos}%
	\PTBRph{}%
	\DEph{}\PLph{}%
	\ITAph{}%
}

\index{x86!\Instructions!BSWAP}
\RU{Инструкция }%
\EN{The }%
\ES{La instrucci\'on}%
\PTBRph{}%
\DEph{}\PLph{}%
\ITAph{}
\TT{BSWAP}
\RU{может использоваться для конвертирования.}%
\EN{instruction can be used for conversion.}%
\ES{puede ser utilizada para conversiones.}%
\PTBRph{}%
\DEph{}\PLph{}%
\ITAph{}

\index{TCP/IP}
\RU{Сетевые пакеты TCP/IP используют соглашение big-endian, вот почему программа, работающая на little-endian архитектуре
должна конвертировать значения.}%
\EN{TCP/IP network data packets use the big-endian conventions, so that is why a program working on a little-endian architecture
has to convert the values.}%
\ES{Los paquetes de datos en redes TCP/IP utilizan convenciones big-endian, por eso un programa corriendo en una arquitectura
little-endian tiene que convertir los valores.}%
\PTBRph{}%
\DEph{}\PLph{}%
\ITAph{}

\RU{Обычно, используются функции}%
\EN{The}%
\ES{Las funciones}%
\PTBRph{}%
\DEph{}\PLph{}%
\ITAph{}
\TT{htonl()} \AndENRU \TT{htons()}
\EN{ functions are usually used.}%
\ES{son usadas generalmente.}%
\PTBRph{}%
\DEph{}\PLph{}%
\ITAph{}

\RU{Порядок байт big-endian в среде TCP/IP также называется, \q{network byte order},}%
\EN{In TCP/IP, big-endian is also called \q{network byte order},}%
\ES{En TCP/IP, big-endian tambi\'en es llamado \q{orden de bytes de red},}%
\PTBRph{}%
\DEph{}\PLph{}%
\ITAph{}
\RU{а порядок байт на компьютере \q{host byte order}.}%
\EN{while byte order on the computer \q{host byte order}.}%
\ES{mientras que el orden de bytes en la computadora se conoce como \q{orden de bytes de host}.}%
\PTBRph{}%
\DEph{}\PLph{}%
\ITAph{}
\RU{На архитектуре Intel x86, и других little-endian архитектурах, \q{host byte order} это little-endian, 
а вот на IBM POWER это может быть big-endian, так что на последней, 
\TT{htonl()} и \TT{htons()} не меняют порядок байт.}%
\EN{\q{host byte order} is little-endian on Intel x86 and other little-endian architectures,
but it is big-endian on IBM POWER, so \TT{htonl()} and \TT{htons()} don't shuffle any bytes
on the latter.}%
\ES{\q{orden de bytes de host} es little-endian en Intel x86 y otras arquitecturas little-endian,
pero es big-endian en IBM POWER, asi que \TT{htonl()} y \TT{htons()} no cambian el orden de los bytes
en \'esta \'ultima.}%
\PTBRph{}%
\DEph{}\PLph{}%
\ITAph{}
