\chapter{Endianness\RU{ (порядок байт)}}
\label{sec:endianness}

\RU{Endianness (порядок байт) это способ представления чисел в памяти}
\EN{The endianness is a way of representing values in memory}.

\section{Big-endian\RU{ (от старшего к младшему)}}

\RU{Число}\EN{The} \TT{0x12345678} \RU{представляется в памяти так}
\EN{value is represented in memory as}:

\begin{center}
\begin{tabular}{ | l | l | }
\hline
\cellcolor{blue!25} \RU{адрес в памяти}\EN{address in memory} & \cellcolor{blue!25} \RU{значение байта}\EN{byte value} \\
\hline
+0 & 0x12 \\
\hline
+1 & 0x34 \\
\hline
+2 & 0x56 \\
\hline
+3 & 0x78 \\
\hline
\end{tabular}
\end{center}

\RU{CPU с таким порядком включают в себя}\EN{Big-endian CPUs include} Motorola 68k, IBM POWER.

\section{Little-endian\RU{ (от младшего к старшему)}}

\RU{Число}\EN{The} \TT{0x12345678} 
\RU{представляется в памяти так}
\EN{value is represented in memory as}:

\begin{center}
\begin{tabular}{ | l | l | }
\hline
\cellcolor{blue!25} \RU{адрес в памяти}\EN{address in memory} & \cellcolor{blue!25} \RU{значение байта}\EN{byte value} \\
\hline
+0 & 0x78 \\
\hline
+1 & 0x56 \\
\hline
+2 & 0x34 \\
\hline
+3 & 0x12 \\
\hline
\end{tabular}
\end{center}

\RU{CPU с таким порядком байт включают в себя}\EN{Little-endian CPUs include} Intel x86.

\section{\Example}

\RU{Возьмем big-endian Linux для MIPS заинсталированный в QEMU}%
\EN{Let's take big-endian MIPS Linux installed and ready in QEMU}
\footnote{\RU{Доступен для скачивания здесь}\EN{Available for download here}: \url{http://go.yurichev.com/17008}}.

\RU{И скомпилируем этот простой пример}\EN{And let's compile this simple example}:

\begin{lstlisting}
#include <stdio.h>

int main()
{
	int v, i;

	v=123;

	printf ("%02X %02X %02X %02X\n", 
		*(char*)&v,
		*(((char*)&v)+1),
		*(((char*)&v)+2),
		*(((char*)&v)+3));
};
\end{lstlisting}

\RU{И запустим его}\EN{And after running it we get}:

\begin{lstlisting}
root@debian-mips:~# ./a.out 
00 00 00 7B
\end{lstlisting}

\RU{Это оно и есть}\EN{That is it}.
0x7B \RU{это}\EN{is} 123 \RU{в десятичном виде}\EN{in decimal}.
\RU{В little-endian-архитектуре, 7B это первый байт (вы можете это проверить в x86 или x86-64),
но здесь он последний, потому что старший байт идет первым.}
\EN{In little-endian architectures, 7B is the first byte (you can check on x86 or x86-64), 
but here it is the last one, because the highest byte goes first.}

\RU{Вот почему имеются разные дистрибутивы Linux для MIPS}
\EN{That's why there are separate Linux distributions for MIPS} 
(\q{mips} (big-endian) \AndENRU \q{mipsel} (little-endian)).
\RU{Программа скомпилированная для одного соглашения об endiannes, не сможет работать в OS использующей
другое соглашение.}
\EN{It is impossible for a binary compiled for one endianness to work on an OS with different endianness.}\\
\\
\RU{Еще один пример связанный с big-endian в MIPS в этой книге:}
\EN{There is another example of MIPS big-endiannes in this book:} \myref{MIPS_structure_big_endian}.

\section{Bi-endian\RU{ (переключаемый порядок)}}

\RU{CPU поддерживающие оба порядка, и его можно переключать, включают в себя}
\EN{CPUs that may switch between endianness are} ARM, PowerPC, SPARC, MIPS, \ac{IA64}, \etc{}.

\section{\RU{Конвертирование}\EN{Converting data}}

\index{x86!\Instructions!BSWAP}
\RU{Инструкция }{The }\TT{BSWAP} \RU{может использоваться для конвертирования}
\EN{instruction can be used for conversion}.

\index{TCP/IP}
\RU{Сетевые пакеты TCP/IP используют соглашение big-endian, вот почему программа, работающая на little-endian архитектуре
должна конвертировать значения.}
\EN{TCP/IP network data packets use the big-endian conventions, so that is why a program working on a little-endian architecture
has to convert the values.} 

\RU{Обычно, используются функции}\EN{The} \TT{htonl()} \AndENRU \TT{htons()}\EN{ functions are usually used}.

\RU{Порядок байт big-endian в среде TCP/IP также называется}
\EN{In TCP/IP, big-endian is also called} \q{network byte order},
\RU{а порядок байт на компьютере}\EN{while byte order on the computer}\EMDASH{}\q{host byte order}.
\RU{На архитектуре Intel x86, и других little-endian архитектурах, \q{host byte order} это little-endian, 
а вот на IBM POWER это может быть big-endian, так что на последней, 
\TT{htonl()} и \TT{htons()} не меняют порядок байт.}
\EN{\q{host byte order} is little-endian on Intel x86 and other little-endian architectures,
but it is big-endian on IBM POWER, so \TT{htonl()} and \TT{htons()} are not shuffle any bytes
on latter.}
