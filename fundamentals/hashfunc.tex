\newcommand{\HashFuncChapterName}{\RU{Хеш-функции}\EN{Hash functions}}
\chapter{\HashFuncChapterName}
\label{hash_func}

\index{\HashFuncChapterName}
\index{CRC32}
\RU{Простейший пример это CRC32, алгоритм \q{более мощный} чем простая контрольная сумма,
для проверки целостности данных}
\EN{A very simple example is CRC32, an algorithm that provides \q{stronger} checksum for integrity checking purposes}.
\RU{Невозможно восстановить оригинальный текст из хеша, там просто меньше информации: ведь текст
может быть очень длинным, но результат CRC32 всегда ограничен 32 битами}
\EN{it is impossible to restore the original text from the hash value, it has much less information:
the input can be long, but CRC32's result is always limited to 32 bits}.
\RU{Но CRC32 не надежна в криптографическом смысле: известны методы как изменить текст таким образом,
чтобы получить нужный результат.}
\EN{But CRC32 is not cryptographically secure: it is known how to alter a text in a way that the resulting
CRC32 hash value will be the one we need.}
\RU{Криптографические хеш-функции защищены от этого}
\EN{Cryptographic hash functions are protected from this}.\\
\\
\index{MD5}
\index{SHA1}
\RU{Такие функции как MD5, SHA1, \etc{}., широко используются для хеширования паролей
для хранения их в базе}
\EN{Such functions are MD5, SHA1, \etc{}, and they are widely used to hash user passwords in order to store them in a database}.
\RU{Действительно: БД форума в интернете может и не хранить пароли 
(иначе злоумышленник получивший доступ к БД сможет узнать все пароли), а только хеши}
\EN{Indeed: an internet forum database may not contain user passwords 
(a stolen database can compromise all user's passwords) but only hashes 
(a cracker can't reveal passwords)}.
\RU{К тому же, скрипту интернет-форума вовсе не обязательно знать ваш пароль, он только должен
сверить его хеш с тем что лежит в БД, и дать вам доступ если cверка проходит}
\EN{Besides, an internet forum engine is not aware of your password, it has only to check if its hash
is the same as the one in the database, and give you access if they match}.
\RU{Один из самых простых способов взлома --- это просто перебирать все пароли и ждать пока
результат будет такой же как тот что нам нужен.}
\EN{One of the simplest password cracking methods is just to try hashing all possible passwords in order
to see which is matching the resulting value that we need.}
\RU{Другие методы намного сложнее}\EN{Other methods are much more complex}.
% TODO1 add about Rainbow tables

\section{\RU{Как работает односторонняя функция}\EN{How one-way function works}?}

\RU{Односторонняя функция, это функция, которая способна превратить из одного значения другое,
при этом невозможно (или трудно) проделать обратную операцию.}
\EN{One-way function is a function which is able to transform one value into another,
while it is impossible (or very hard) to do reverse it back.}
\RU{Некоторые люди имеют трудности с пониманием, как это возможно.}
\EN{Some people have difficulties while understanding how it's possible at all.}
\RU{Приведу очень простой пример.}
\EN{I'll try to demonstrate it.}

\RU{У нас есть ряд из 10-и чисел в пределах 0..9, каждое встречается один раз, например:}
\EN{We've got vector of 10 numbers in range 0..9, each encounters only once, for example:}

\begin{lstlisting}
4 6 0 1 3 5 7 8 9 2
\end{lstlisting}

\RU{Алгоритм простейшей односторонней функции выглядит так}\EN{Algorithm of simplest possible one-way function is}:

\begin{itemize}
\item \RU{возьми число на нулевой позиции (у нас это 4)}\EN{take a number at zeroth position (4 in our case)};
\item \RU{возьми число на первой позиции (у нас это 6)}\EN{take a number at first position (6 in our case)};
\item \RU{обменяй местами числа на позициях 4 и 6}\EN{swap numbers at positions of  4 and 6}.
\end{itemize}

\RU{Отметим числа на позициях 4 и 6}\EN{Let's mark numbers on positions of 4 and 6}:

\begin{lstlisting}
4 6 0 1 3 5 7 8 9 2
        ^   ^
\end{lstlisting}

\RU{Меняем их местами и получаем результат}\EN{Let's swap them and we've got a result}:

\begin{lstlisting}
4 6 0 1 7 5 3 8 9 2
\end{lstlisting}

\RU{Глядя на результат, и даже зная алгоритм функции, мы не можем однозначно восстановить изначальное
положение чисел.
Ведь первые два числа могли быть 0 и/или 1, и тогда именно они могли бы участвовать в обмене.}
\EN{While looking at the result and even if we know algorithm, we can't say unambiguously initial
numbers set.
Because first two numbers could be 0 and/or 1, and then they could be participate in swapping
procedure.}

\RU{Это крайне упрощенный пример для демонстрации, настоящие односторонние функции могут быть значительно
сложнее.}
\EN{This is utterly simplified example for demonstration, real one-way functions may be much more
complex.}
