\chapter{\RU{Память}\EN{Memory}}

\RU{Есть три основных типа памяти:}
\EN{There are 3 main types of memory:}

\begin{itemize}
\item \RU{Глобальная память}\EN{Global memory}.
\ac{AKA} ``static memory allocation''.
\RU{Нет нужды явно выделять, выделение происходит просто при объявлении переменных/массивов 
глобально.}
\EN{No need to allocate explicitly, the allocation is done just by declaring variables/arrays 
globally.}
\RU{Это глобальные переменные расположенные в сегменте данных или констант.}
\EN{These are global variables, residing in the data or constant segments.}
\RU{Доступны глобально (поэтому считаются \glslink{anti-pattern}{анти-паттерном}).}
\EN{The are available globally (hence, considered as an \gls{anti-pattern}).}
\RU{Не удобны для буферов/массивов, потому что должны иметь фиксированный размер.}
\EN{Not convenient for buffers/arrays, because they must have a fixed size.}
\RU{Переполнения буфера, случающиеся здесь, обычно перезаписывают переменные или буферы
расположенные рядом в памяти.}
\EN{Buffer overflows that occur here usually overwrite variables or buffers reside next to them in memory.}
\RU{Пример в этой книге}\EN{There's an example in this book}: \myref{scanf_global_variable}.

\item \RU{Стек}\EN{Stack}.
\ac{AKA} ``allocate on stack''\RU{, ``выделить память в/на стеке''}.
\RU{Выделение происходит просто при объявлении переменных/массивов локально в функции.}
\EN{The allocation is done just by declaring variables/arrays locally in the function.}
\RU{Обычно это локальные для функции переменные}\EN{These are usually local variables for the function}.
\RU{Иногда эти локальные переменные также доступны и для нисходящих функций (если функция передает
указатель на переменную в следующую функцию).}
\EN{Sometimes these local variable are also available to descending functions (if one passes
a pointer to a variable to the function to be executed).}
\RU{Выделение и освобождение очень быстрое, достаточно просто сдвига \ac{SP}.}
\EN{Allocation and deallocation are very fast, only \ac{SP} needs to be shifted.}
\index{\CStandardLibrary!alloca()}
\EN{But they're also not convenient for buffers/arrays, because the buffer size has to be fixed,
unless \TT{alloca()} (\myref{alloca}) (or a variable-length array) is used.}
\RU{Но также не удобно для буферов/массивов, потому что размер буфера фиксирован,
если только не используется \TT{alloca()} (\myref{alloca}) (или массив с переменной длиной).}
\RU{Переполнение буфера обычно перезаписывает важные структуры стека}\EN{Buffer overflows usually 
overwrite important stack structures}: \myref{subsec:bufferoverflow}.

\index{\CStandardLibrary!malloc()}
\index{\CStandardLibrary!free()}
\item \RU{Куча (\IT{heap})}\EN{Heap}. 
\ac{AKA} ``dynamic memory allocation''\RU{, ``выделить память в куче''}.
\RU{Выделение происходит при помощи вызова}\EN{Allocation is done by calling} 
\TT{malloc()/free()} \OrENRU \TT{new/delete} \InENRU \Cpp.
\RU{Самый удобный метод: размер блока может быть задан во время исполнения}\EN{This is the most convenient 
method: the block size may be set at runtime}.
\index{\CStandardLibrary!realloc()}
\RU{Изменение размера возможно (при помощи \TT{realloc()}), но может быть медленным.}
\EN{Resizing is possible (using \TT{realloc()}), but can be slow.}
\RU{Это самый медленный метод выделения памяти: аллокатор памяти должен поддерживать и обновлять
все управляющие структуры во время выделения и освобождения.}
\EN{This is the slowest way to allocate memory: 
the memory allocator must support and update all control structures while
allocating and deallocating.}
\RU{Переполнение буфера обычно перезаписывает все эти структуры}\EN{Buffer overflows usually 
overwrite these structures}.
\RU{Выделения в куче также ведут к проблеме утечек памяти: каждый выделенный блок должен быть
явно освобожден, но кто-то может забыть об этом, или делать это неправильно.}
\EN{Heap allocations are also source of memory leak problems: each memory block has to be deallocated
explicitly, but one may forget about it, or do it incorrectly.}
\index{\CStandardLibrary!free()}
\RU{Еще одна проблема --- это ``использовать после освобождения'' --- использовать блок памяти после
того как \TT{free()} был вызван на нем, это тоже очень опасно.}
\EN{Another problem is the ``use after free''---using a memory block after \TT{free()} was called on it,
which is very dangerous.}
\RU{Пример в этой книге}\EN{Example in this book}: \myref{struct_malloc_example}.

\end{itemize}
