\chapter{\SignedNumbersSectionName}
\label{sec:signednumbers}
\index{Signed numbers}

\newcommand{\URLS}{\href{http://go.yurichev.com/17117}{wikipedia}}

\RU{Методов представления чисел с знаком ``плюс'' или ``минус'' несколько\footnote{\URLS}, 
но в компьютерах обычно применяется метод ``дополнительный код'' или ``two's complement''.}
\EN{There are several methods of representing signed numbers\footnote{\URLS}, 
but ``two's complement'' is most popular in computers.}

\begin{center}
\begin{tabular}{ | l | l | l | l | }
\hline
\cellcolor{blue!25} \RU{двоичное}\EN{binary} & 
\cellcolor{blue!25} \RU{шестнадцатеричное}\EN{hexadecimal} & 
\cellcolor{blue!25} \RU{беззнаковое}\EN{unsigned} &
\cellcolor{blue!25} \RU{знаковое}\EN{signed} (\RU{дополнительный код}\EN{2's complement}) \\
\hline
01111111 & 0x7f & 127 & 127 \\
\hline
01111110 & 0x7e & 126 & 126 \\
\hline
\multicolumn{4}{ |c| }{...} \\
\hline
00000110 & 0x6 & 6 & 6 \\
\hline
00000101 & 0x5 & 5 & 5 \\
\hline
00000100 & 0x4 & 4 & 4 \\
\hline
00000011 & 0x3 & 3 & 3 \\
\hline
00000010 & 0x2 & 2 & 2 \\
\hline
00000001 & 0x1 & 1 & 1 \\
\hline
00000000 & 0x0 & 0 & 0 \\
\hline
11111111 & 0xff & 255 & -1 \\
\hline
11111110 & 0xfe & 254 & -2 \\
\hline
11111101 & 0xfd & 253 & -3 \\
\hline
11111100 & 0xfc & 252 & -4 \\
\hline
11111011 & 0xfb & 251 & -5 \\
\hline
11111010 & 0xfa & 250 & -6 \\
\hline
\multicolumn{4}{ |c| }{...} \\
\hline
10000010 & 0x82 & 130 & -126 \\
\hline
10000001 & 0x81 & 129 & -127 \\
\hline
10000000 & 0x80 & 128 & -128 \\
\hline
\end{tabular}
\end{center}

\index{x86!\Instructions!JA}
\index{x86!\Instructions!JB}
\index{x86!\Instructions!JL}
\index{x86!\Instructions!JG}
\RU{Разница в подходе к знаковым/беззнаковым числам, собственно, нужна потому что, например, 
если представить \TT{0xFFFFFFFE} и \TT{0x0000002} как беззнаковое, то первое число ($4294967294$) больше второго ($2$). 
Если их оба представить как знаковые, то первое будет $-2$, которое, разумеется, меньше чем второе ($2$).
Вот почему инструкции для условных переходов~(\ref{sec:Jcc}) представлены в обоих версиях ~--- 
и для знаковых сравнений (например, \JG, \JL) и для беззнаковых (\JA, \JB).}
\EN{The difference between signed and unsigned numbers is that if we represent \TT{0xFFFFFFFE} and \TT{0x0000002} 
as unsigned, then first number ($4294967294$) is bigger than second ($2$). 
If to represent them both as signed, first will be $-2$, and it is lesser than second ($2$). 
That is the reason why conditional jumps~(\ref{sec:Jcc}) are present both for signed (e.g. \JG, \JL) 
and unsigned (\JA, \JB) operations.} \\
\\
\RU{Для простоты, вот что нужно знать}\EN{For the sake of simplicity, that is what one need to know}:
\begin{itemize}
\item \RU{Числа бывают знаковые и беззнаковые}\EN{Number can be signed or unsigned}.

\item \RU{Знаковые типы в \CCpp}\EN{\CCpp signed types}:
  \begin{itemize}
    \item \TT{int64\_t} (-9223372036854775806..9223372036854775807) \OrENRU \\
                \TT{0x8000000000000000..0x7FFFFFFFFFFFFFFF}),
    \item \Tint (-2147483646..2147483647 \OrENRU \TT{0x80000000..0x7FFFFFFF}),
    \item \Tchar (-127..128 \OrENRU \TT{0x7F..0x80}),
    \item \TT{ssize\_t}.
   \end{itemize}

	\RU{Беззнаковые}\EN{Unsigned}: 
  \begin{itemize}
   \item \TT{uint64\_t} (0..18446744073709551615 \OrENRU \TT{0..0xFFFFFFFFFFFFFFFF}),
   \item \TT{unsigned int} (0..4294967295 \OrENRU \TT{0..0xFFFFFFFF}),
   \item \TT{unsigned char} (0..255 \OrENRU \TT{0..0xFF}), 
   \item \TT{size\_t}.
  \end{itemize}

\item \RU{У знаковых чисел знак определяется самым старшим битом: 1 означает ``минус'', 0 означает ``плюс''}
	\EN{Signed types has sign in the most significant bit: 1 mean ``minus'', 0 mean ``plus''}.

\item \EN{Promoting to larger data types is simple:}
      \RU{Преобразование в б\'{о}льшие типы данных обходится легко:} \ref{subsec:sign_extending_32_to_64}.

\item \EN{Negation is simple: just invert all bits and add 1.}
\RU{Изменить знак легко: просто инвертируйте все биты и прибавьте 1.}
\EN{We may observe that a number of different sign is located 
on the opposite side at the same proximity from zero.}
\RU{Мы можем заметить, что число другого знака находится на другой стороне на том же расстоянии от нуля.}
\RU{Прибавление необходимо из-за присутствия нуля посредине.}
\EN{Addition is needed because zero is present in the middle.}

\index{x86!\Instructions!IDIV}
\index{x86!\Instructions!DIV}
\index{x86!\Instructions!IMUL}
\index{x86!\Instructions!MUL}
\index{x86!\Instructions!CBW}
\index{x86!\Instructions!CWD}
\index{x86!\Instructions!CWDE}
\index{x86!\Instructions!CDQ}
\index{x86!\Instructions!CDQE}
\index{x86!\Instructions!MOVSX}
\index{x86!\Instructions!SAR}
\item \RU{Инструкции сложения и вычитания работают одинаково хорошо и для знаковых и для беззнаковых значений}
	\EN{Addition and subtraction operations are working well for both signed and unsigned values}.
	\RU{Но для операций умножения и деления, в x86 имеются разные инструкции}
	\EN{But for multiplication and division operations, x86 has different instructions}:
	\TT{IDIV}/\TT{IMUL} \RU{для знаковых}\EN{for signed}
	\AndENRU \TT{DIV}/\TT{MUL} \RU{для беззнаковых}\EN{for unsigned}.
\item \RU{Еще инструкции работающие с знаковыми числами}\EN{More instructions working with signed numbers}:
	\TT{CBW/CWD/CWDE/CDQ/CDQE} (\ref{ins:CBW_CWD_etc}), \TT{MOVSX} (\ref{MOVSX}), \TT{SAR} (\ref{ins:SAR}).
\end{itemize}

\iffalse
\section{\RU{Переполнение integer}\EN{Integer overflow}}

\RU{Бывает так, что ошибки представления знаковых/беззнаковых могут привести к уязвимости 
\IT{переполнение integer}.}
\EN{It is worth noting that incorrect representation of number can lead integer overflow vulnerability.}

\RU{Например, есть некий сервис, который принимает по сети некие пакеты. 
В пакете есть заголовок где указана длина пакета. Это 32-битное значение. 
В процессе приема пакета, 
сервис проверяет это значение и сверяет, больше ли оно чем максимальный размер пакета, скажем, константа
\TT{MAX\_PACKET\_SIZE} (например, 10 килобайт), и если да, то пакет отвергается как некорректный. 
Сравнение знаковое. Злоумышленник подставляет значение \TT{0xFFFFFFFF}. Это число трактуется как знаковое $-1$ 
и оно меньше чем $10000$. Проверка проходит. Продолжаем дальше и копируем этот пакет куда-нибудь себе 
в сегмент данных\dots вызов функции \TT{memcpy (dst, src, 0xFFFFFFFF)} скорее всего, 
затрет много чего внутри процесса.}
\EN{For example, we have a network service, it receives network packets. 
In the packets there is also a field where subpacket length is coded. 
It is 32-bit value. 
After network packet received, service checking the field, and if it is larger than, 
e.g. some \TT{MAX\_PACKET\_SIZE} (let's say, 10 kilobytes), the packet is rejected as incorrect.
Comparison is signed. Intruder set this value to the \TT{0xFFFFFFFF}.
While comparison, this number is considered as signed $-1$ and it is lesser than 10 kilobytes. 
No error here. 
Service would like to copy the subpacket to another place in memory and call 
\TT{memcpy (dst, src, 0xFFFFFFFF)} function: this operation, rapidly garbling a lot of 
inside of process memory.}

\RU{Немного подробнее}\EN{More about it}: \cite{Phrack3C0A}.
% TODO example!
\fi
