\section{\SignedNumbersSectionName}
\label{sec:signednumbers}
\myindex{Signed numbers}

\newcommand{\URLS}{\href{http://go.yurichev.com/17117}{wikipedia}}

There are several methods for representing signed numbers\footnote{\URLS}, 
but \q{two's complement} is the most popular one in computers.

Here is a table for some byte values:

\begin{center}
\begin{tabular}{ | l | l | l | l | }
\hline
\HeaderColor binary & \HeaderColor hexadecimal & \HeaderColor unsigned & \HeaderColor signed (2's complement) \\
\hline
01111111 & 0x7f & 127 & 127 \\
\hline
01111110 & 0x7e & 126 & 126 \\
\hline
\multicolumn{4}{ |c| }{...} \\
\hline
00000110 & 0x6 & 6 & 6 \\
\hline
00000101 & 0x5 & 5 & 5 \\
\hline
00000100 & 0x4 & 4 & 4 \\
\hline
00000011 & 0x3 & 3 & 3 \\
\hline
00000010 & 0x2 & 2 & 2 \\
\hline
00000001 & 0x1 & 1 & 1 \\
\hline
00000000 & 0x0 & 0 & 0 \\
\hline
11111111 & 0xff & 255 & -1 \\
\hline
11111110 & 0xfe & 254 & -2 \\
\hline
11111101 & 0xfd & 253 & -3 \\
\hline
11111100 & 0xfc & 252 & -4 \\
\hline
11111011 & 0xfb & 251 & -5 \\
\hline
11111010 & 0xfa & 250 & -6 \\
\hline
\multicolumn{4}{ |c| }{...} \\
\hline
10000010 & 0x82 & 130 & -126 \\
\hline
10000001 & 0x81 & 129 & -127 \\
\hline
10000000 & 0x80 & 128 & -128 \\
\hline
\end{tabular}
\end{center}

\myindex{x86!\Instructions!JA}
\myindex{x86!\Instructions!JB}
\myindex{x86!\Instructions!JL}
\myindex{x86!\Instructions!JG}
The difference between signed and unsigned numbers is that if we represent \TT{0xFFFFFFFE} and \TT{0x00000002} 
as unsigned, then the first number (4294967294) is bigger than the second one (2). 
If we represent them both as signed, the first one becomes $-2$, and it is smaller than the second (2). 
That is the reason why conditional jumps~(\myref{sec:Jcc}) are present both for signed (e.g. \JG, \JL) 
and unsigned (\JA, \JB) operations.

For the sake of simplicity, this is what one needs to know:

\begin{itemize}
\item Numbers can be signed or unsigned.

\item \CCpp signed types:

  \begin{itemize}
    \item \TT{int64\_t} (-9,223,372,036,854,775,808..9,223,372,036,854,775,807) 
	  (-~9.2..~9.2 quintillions) or \\
                \TT{0x8000000000000000..0x7FFFFFFFFFFFFFFF}),
    \item \Tint (-2,147,483,648..2,147,483,647 (-~2.15..~2.15Gb) or \TT{0x80000000..0x7FFFFFFF}),
    \item \Tchar (-128..127 or \TT{0x80..0x7F}),
    \item \TT{ssize\_t}.
   \end{itemize}

	Unsigned:
  \begin{itemize}
	  \item \TT{uint64\_t} (0..18,446,744,073,709,551,615 
		  (~18 quintillions) or \TT{0..0xFFFFFFFFFFFFFFFF}),
   \item \TT{unsigned int} (0..4,294,967,295 (~4.3Gb) or \TT{0..0xFFFFFFFF}),
   \item \TT{unsigned char} (0..255 or \TT{0..0xFF}), 
   \item \TT{size\_t}.
  \end{itemize}

\item Signed types have the sign in the most significant bit: 1 mean \q{minus}, 0 mean \q{plus}.

\item Promoting to a larger data types is simple:
\myref{subsec:sign_extending_32_to_64}.

\label{sec:signednumbers:negation}
\item Negation is simple: just invert all bits and add 1.

We can remember that a number of inverse sign is located on the opposite side at the same proximity from zero.
The addition of one is needed because zero is present in the middle.

\myindex{x86!\Instructions!IDIV}
\myindex{x86!\Instructions!DIV}
\myindex{x86!\Instructions!IMUL}
\myindex{x86!\Instructions!MUL}
\myindex{x86!\Instructions!CBW}
\myindex{x86!\Instructions!CWD}
\myindex{x86!\Instructions!CWDE}
\myindex{x86!\Instructions!CDQ}
\myindex{x86!\Instructions!CDQE}
\myindex{x86!\Instructions!MOVSX}
\myindex{x86!\Instructions!SAR}
\item 
	The addition and subtraction operations work well for both signed and unsigned values.
	But for multiplication and division operations, x86 has different instructions:
	\TT{IDIV}/\TT{IMUL} for signed and \TT{DIV}/\TT{MUL} for unsigned.
\item
	Here are some more instructions that work with signed numbers:
	\TT{CBW/CWD/CWDE/CDQ/CDQE} (\myref{ins:CBW_CWD_etc}), \TT{MOVSX} (\myref{MOVSX}), \TT{SAR} (\myref{ins:SAR}).
\end{itemize}

% section
% TODO translate
\section{Using IMUL over MUL}
\label{IMUL_over_MUL}

\myindex{x86!\Instructions!MUL}
\myindex{x86!\Instructions!IMUL}
Example like \lstref{unsigned_multiply_C} where two unsigned values are multiplied compiles into \lstref{unsigned_multiply_lst} where \IMUL is used instead of \MUL.

This is important property of both \MUL and \IMUL instructions.
First of all, they both produce 64-bit value if two 32-bit values are multiplied, or 128-bit value if two 64-bit values are multiplied (biggest possible \gls{product} in 32-bit environment is \GTT{0xffffffff*0xffffffff=0xfffffffe00000001}).
But \CCpp standards have no way to access higher half of result, and a \gls{product} always has the same size as multiplicands. % TODO \gls{}?
And both \MUL and \IMUL instructions works in the same way if higher half is ignored, i.e., lower half is the same.
This is important property of \q{two's complement} way of representing signed numbers.

So \CCpp compiler can use any of these instructions.

But \IMUL is more versatile than MUL because it can take any register(s) as source, while \MUL requires one of multiplicands stored in \AX/\EAX/\RAX register.
Even more than that: \MUL stores result in \GTT{EDX:EAX} pair in 32-bit environment, or \GTT{RDX:RAX} in 64-bit one, so it always calculates the whole result.
On contrary, it's possible to set a single destination register while using \IMUL instead of pair, and then \ac{CPU} will calculate only lower half, which works faster (see \IT{Instruction latencies and throughput for AMD and Intel x86 processors} by Torborn Granlund\footnote{\url{http://yurichev.com/mirrors/x86-timing.pdf}}).

Given than, \CCpp compilers may generate \IMUL instruction more often then \MUL.

\myindex{Compiler intrinsic}
Nevertheless, using compiler intrinsic, it's still possible to do unsigned multiplication and get \IT{full} result.
This is sometimes called \IT{extended multiplication}.
MSVC has intrinsic for this called \IT{\_\_emul}\footnote{\url{https://msdn.microsoft.com/en-us/library/d2s81xt0(v=vs.80).aspx}} and another one: \IT{\_umul128}\footnote{\url{https://msdn.microsoft.com/library/3dayytw9%28v=vs.100%29.aspx}}.
GCC offer \IT{\_\_int128} data type, and if 64-bit multiplicands are first promoted to 128-bit ones,
then a \gls{product} is stored into another \IT{\_\_int128}, then result is shifted by 64-bit right, you'll get higher half of result\footnote{Example: \url{http://stackoverflow.com/a/13187798}}.

\subsection{MulDiv() function in Windows}
\myindex{Windows!Win32!MulDiv()}

Windows has MulDiv() function
\footnote{\url{https://msdn.microsoft.com/en-us/library/windows/desktop/aa383718(v=vs.85).aspx}},
fused multiply/divide function, it multiplies two 32-bit integers into intermediate 64-bit value
and then divides it by a third 32-bit integer.
It is easier than to use two compiler intrinsic, so Microsoft developers made a special function for it.
And it seems, this is busy function, judging by its usage.



