\newglossaryentry{tail call}
{
  name=\RU{хвостовая рекурсия}\EN{tail call},
  description={\RU{Это когда компилятор или интерпретатор превращает рекурсию 
  (с которой возможно это проделать, т.е., \IT{хвостовую}) в итерацию для эффективности}
  \EN{It is when compiler (or interpreter) transforms recursion (with which it is possible: \IT{tail recursion}) 
  into iteration for efficiency}: \url{http://en.wikipedia.org/wiki/Tail_call}}
}

\newglossaryentry{endianness}
{
  name=endianness,
  description={\RU{Порядок байт}\EN{Byte order}: \ref{sec:endianness}}
}

\newglossaryentry{caller}
{
  name=caller,
  description={\RU{Ф-ция вызывающая другую ф-цию}\EN{A function calling another}}
}

\newglossaryentry{callee}
{
  name=callee,
  description={\RU{Вызываемая ф-ция}\EN{A function being called by another}}
}

\newglossaryentry{debuggee}
{
  name=debuggee,
  description={\RU{Отлаживаемая программа}\EN{A program being debugged}}
}

\newglossaryentry{leaf function}
{
  name=leaf function,
  description={\RU{Ф-ция не вызывающая больше никаких ф-ций}
  \EN{A function which is not calling any other function}}
}

\newglossaryentry{link register}
{
  name=link register,
  description=(RISC) {\RU{Регистр в котором обычно записан адрес возврата.
  Это позволяет вызывать leaf-функции без использования стека, т.е., быстрее.}
  \EN{A register where return address is usually stored.
  This makes calling leaf functions without stack usage, i.e., faster.}}
}

\newglossaryentry{anti-pattern}
{
  name=anti-pattern,
  description={\RU{Нечто широко известное как плохое решение}
  \EN{Generally considered as bad practice}}
}

\newglossaryentry{stack pointer}
{
  name=\RU{указатель стека}\EN{stack pointer},
  description={\RU{Регистр указывающий на место в стеке.}
  \EN{A register pointing to the place in the stack.}}
}

\newglossaryentry{decrement}
{
  name=\RU{декремент}\EN{decrement},
  description={\RU{Уменьшение на $1$}\EN{Decrease by $1$}}
}

\newglossaryentry{increment}
{
  name=\RU{инкремент}\EN{increment},
  description={\RU{Увеличение на $1$}\EN{Increase by $1$}}
}

\newglossaryentry{loop unwinding}
{
  name=loop unwinding,
  description={\RU{Это когда вместо организации цикла на $n$ итераций, компилятор генерирует $n$ копий тела
  цикла, для экономии на инструкциях, обеспечивающих сам цикл}
  \EN{It is when a compiler instead of generation loop code of $n$ iteration, generates just $n$ copies of the
  loop body, in order to get rid of loop maintenance instructions}}
}

\newglossaryentry{register allocator}
{
  name=register allocator,
  description={\RU{Ф-ция компилятора распределяющая локальные переменные по регистрам процессора}
  \EN{Compiler's function assigning local variables to CPU registers}}
}

\newglossaryentry{quotient}
{
  name=\RU{частное}\EN{quotient},
  description={\RU{Результат деления}\EN{Division result}}
}

\newglossaryentry{product}
{
  name=\RU{произведение}\EN{product},
  description={\RU{Результат умножения}\EN{Multiplication result}}
}

\newglossaryentry{NOP}
{
  name=NOP,
  description={``no operation'', \RU{холостая инструкция}\EN{idle instruction}}
}

\newglossaryentry{POKE}
{
  name=POKE,
  description={\RU{Инструкция языка BASIC записывающая байт по определенному адресу}
  	\EN{BASIC language instruction writing byte on specific address}}
}

\newglossaryentry{keygenme}
{
  name=keygenme,
  description={\RU{Программа, имитирующая защиту вымышленной программы, для которой нужно сделать 
  генератор ключей/лицензий}\EN{A program which imitates fictional software protection,
  for which one needs to make a keys/licenses generator}}
}

\newglossaryentry{dongle}
{
  name=dongle,
  description={\RU{Небольшое устройство подключаемое к LPT-порту для принтера (в прошлом) или к USB}
  \EN{Dongle is a small piece of hardware connected to LPT printer port (in past) or to USB}.
  \RU{Исполняло функции security token-а, имела память и, иногда,
  секретную (крипто-)хеширующую функцию}
  \EN{Its function was akin to security token, it has some memory and, sometimes,
  secret (crypto-)hashing algorithm}.}
}

\newglossaryentry{thunk function}
{
  name=thunk function,
  description={\RU{Крохотная функция делающая только одно: вызывающая другую функцию.}
  \EN{Tiny function with a single role: call another function.}}
}

\newglossaryentry{user mode}
{
  name=user mode,
  description={\RU{Режим CPU с ограниченными возможностями в котором он исполняет прикладное ПО. ср.}
  \EN{A restricted CPU mode in which it executes all applied software code. cf.} \gls{kernel mode}.}
}

\newglossaryentry{kernel mode}
{
  name=kernel mode,
  description={\RU{Режим CPU с неограниченными возможностями в котором он исполняет ядро OS и драйвера. ср.}
  \EN{A restrictions-free CPU mode in which it executes OS kernel and drivers. cf.} \gls{user mode}.}
}

\newglossaryentry{Windows NT}
{
  name=Windows NT,
  description={Windows NT, 2000, XP, Vista, 7, 8}
}

\newglossaryentry{atomic operation}
{
  name=atomic operation,
  description={
  ``$\alpha{}\tau{}o\mu{}o\varsigma{}$''
  %``atomic''
  \RU{означает ``неделимый'' в греческом языке, так что атомарная операция
  это операция которая гарантированно не будет прервана другими тредами}
  \EN{mean ``indivisible'' in Greek, so atomic operation is what guaranteed not
  to be broke up during operation by other threads}}
}

% to be proofreaded (begin)
\newglossaryentry{NaN}
{
  name=NaN,
  description={
  	\RU{не число: специальные случаи чисел с плавающей запятой, обычно сигнализирующие об ошибках}
	\EN{not a number: special cases of floating point numbers, usually signaling about errors}
  }
}

\newglossaryentry{basic block}
{
  name=basic block,
  description={
  	\RU{группа инструкций не имеющая инструкций переходов,
	а также не имеющая переходов в середину блока извне.
	В IDA он выглядит как просто список инструкций без строк-разрывов}
	\EN{a group of instructions not having jump/branch instructions, and also not having
	jumps inside block from the outside.
	In IDA it looks just like as a list of instructions without breaking empty lines}
  }
}

\newglossaryentry{NEON}
{
  name=NEON,
  description={\ac{AKA} ``Advanced SIMD''\EMDASH\ac{SIMD} \RU{от}\EN{from} ARM}
}

\newglossaryentry{reverse engineering}
{
  name=reverse engineering,
  description={\RU{процесс понимания как устроена некая вещь, иногда, с целью клонирования оной}
  \EN{act of understanding, how the thing works, sometimes, in order to clone it}}
}

\newglossaryentry{compiler intrinsic}
{
  name=compiler intrinsic,
  description={\RU{Специфичная для компилятора ф-ция не являющаяся обычной библиотечной ф-цией.
	Компилятор вместо её вызова генерирует определенный машинный код.
	Нередко, это псевдофункции для определенной инструкции \ac{CPU}. Читайте больше:}
	\EN{A function specific to a compiler which is not usual library function.
	Compiler generate a specific machine code instead of call to it.
	It is often a pseudofunction for specific \ac{CPU} instruction. Read more:} (\ref{sec:compiler_intrinsic})}
}

\newglossaryentry{heap}
{
  name=heap,
  description={\RU{(куча) обычно, большой кусок памяти предоставляемый \ac{OS}, так что прикладное ПО может делить его
  как захочет. malloc()/free() работают с кучей.}
  \EN{usually, a big chunk of memory provided by \ac{OS} so that applications can divide it by themselves as they wish.
  malloc()/free() works with heap.}}
}

\newglossaryentry{name mangling}
{
  name=name mangling,
  description={\RU{применяется как минимум в Си++, где компилятору нужно закодировать имя класса,
  метода и типы аргументов в одной
  строке, которая будет внутренним именем ф-ции. читайте также здесь}
  \EN{used at least in C++, where compiler need to encode name of class, method and argument types in the one string,
  which will become internal name of the function. read more here}: \ref{namemangling}}
}

\newglossaryentry{xoring}
{
  name=xoring,
  description={\RU{нередко применяемое в английском языке, означает применение операции 
  \ac{XOR}}
  \EN{often used in English language, meaning applying \ac{XOR} operation}}
}

\newglossaryentry{security cookie}
{
  name=security cookie,
  description={\RU{Случайное значение, разное при каждом исполнении. Читайте больше об этом тут}
  \EN{A random value, different at each execution. Read more about it}: \ref{subsec:BO_protection}}
}

\newglossaryentry{tracer}
{
  name=tracer,
  description={\RU{Моя простейшая утилита для отладки. Читайте больше об этом тут}
  \EN{My own simple debugging tool. Read more about it}: \ref{tracer}}
}

\newglossaryentry{GiB}
{
  name=GiB,
  description={\RU{Гибибайт: $2^{30}$ или 1024 мебибайт или 1073741824 байт}
  \EN{Gibibyte: $2^{30}$ or 1024 mebibytes or 1073741824 bytes}}
}

\newglossaryentry{CP/M}
{
  name=CP/M,
  description={Control Program for Microcomputers: \RU{очень простая дисковая \ac{OS} использовавшаяся перед}
  \EN{a very basic disk \ac{OS} used before} MS-DOS}
}

\newglossaryentry{stack frame}
{
  name=stack frame,
  description={\RU{Часть стека, в которой хранится информация связанная с текущей ф-цией: локальные переменные,
  аргументы ф-ции, \ac{RA}, итд}\EN{Part of stack containing information specific to the current functions:
  local variables, function arguments, \ac{RA}, etc}}
}

\newglossaryentry{jump offset}
{
  name=jump offset,
  description={\RU{Часть опкода JMP или Jcc инструкции, просто прибавляется к адресу следующей инструкции,
  и так вычисляется новый \ac{PC}. Может быть отрицательным.}\EN{a part of JMP or Jcc instruction opcode, 
  it just to be added to the address
  of the next instruction, and thus is how new \ac{PC} is calculated. May be negative as well.}}
}

\newglossaryentry{integral type}
{
  name=\RU{интегральный тип данных}\EN{integral data type},
  description={\RU{обычные числа, но не с плавающей точкой}
  \EN{usual numbers, but not floating point ones}}
}

\newglossaryentry{PDB}
{
  name=PDB,
  description={(Win32) \RU{Файл с отладочной информацией, обычно просто имена ф-ций, 
  но иногда имена аргументов ф-ций и локальных переменных}
  \EN{Debugging information file, usually just function names, but sometimes also function
  arguments and local variables names}}
}

\newglossaryentry{NTAPI}
{
  name=NTAPI,
  description={\RU{\ac{API} доступное только в линии Windows NT. 
  Большей частью не документировано Microsoft-ом.}\EN{\ac{API} available only in Windows NT line. 
  Largely, not documented by Microsoft.}}
}
