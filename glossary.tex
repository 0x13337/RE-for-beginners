\newglossaryentry{tail call}
{
  name=\IFRU{хвостовая рекурсия}{tail call},
  description={\IFRU{Это когда компилятор или интерпретатор превращает рекурсию 
  (с которой возможно это проделать, т.е., \IT{хвостовую}) в итерацию для эффективности}
  {It is when compiler (or interpreter) transforms recursion (with which it is possible: \IT{tail recursion}) 
  into iteration for efficiency}: \url{http://en.wikipedia.org/wiki/Tail_call}}
}

\newglossaryentry{endianness}
{
  name=endianness,
  description={\IFRU{Порядок байт}
  {Byte order}: \url{https://en.wikipedia.org/wiki/Endianness}}
}

\newglossaryentry{caller}
{
  name=caller,
  description={\IFRU{Ф-ция вызывающая другую ф-цию}{A function calling another}}
}

\newglossaryentry{callee}
{
  name=callee,
  description={\IFRU{Вызываемая ф-ция}{A function being called by another}}
}

\newglossaryentry{debuggee}
{
  name=debuggee,
  description={\IFRU{Отлаживаемая программа}{A program being debugged}}
}

\newglossaryentry{leaf function}
{
  name=leaf function,
  description={\IFRU{Ф-ция не вызывающая больше никаких ф-ций}
  {A function which is not calling any other function}}
}

\newglossaryentry{link register}
{
  name=link register,
  description=(RISC) {\IFRU{Регистр в котором обычно записан адрес возврата.
  Это позволяет вызывать leaf-функции без использования стека, т.е., быстрее.}
  {A register where return address is usually stored.
  This makes calling leaf functions without stack usage, i.e., faster.}}
}

\newglossaryentry{anti-pattern}
{
  name=anti-pattern,
  description={\IFRU{Нечто широко известное как плохое решение}
  {Generally considered as bad practice}}
}

\newglossaryentry{stack pointer}
{
  name=\IFRU{указатель стека}{stack pointer},
  description={\IFRU{Регистр указывающий на место в стеке.}
  {A register pointing to the place in the stack.}
  \SP/\ESP/\RSP \InENRU x86}
}

\newglossaryentry{decrement}
{
  name=\IFRU{декремент}{decrement},
  description={\IFRU{Уменьшение на $1$}
  {Decrease by $1$}}
}

\newglossaryentry{increment}
{
  name=\IFRU{инкремент}{increment},
  description={\IFRU{Увеличение на $1$}
  {Increase by $1$}}
}

\newglossaryentry{loop unwinding}
{
  name=loop unwinding,
  description={\IFRU{Это когда вместо организации цикла на $n$ итераций, компилятор генерирует $n$ копий тела
  цикла, для экономии на инструкциях, обеспечивающих сам цикл}
  {It is when a compiler instead of generation loop code of $n$ iteration, generates just $n$ copies of the
  loop body, in order to get rid of loop maintenance instructions}}
}

\newglossaryentry{register allocator}
{
  name=register allocator,
  description={\IFRU{Ф-ция компилятора распределяющая локальные переменные по регистрам процессора}
  {Compiler's function assigning local variables to CPU registers}}
}

\newglossaryentry{quotient}
{
  name=\IFRU{частное}{quotient},
  description={\IFRU{Результат деления}{Division result}}
}

\newglossaryentry{product}
{
  name=\IFRU{произведение}{product},
  description={\IFRU{Результат умножения}{Multiplication result}}
}

\newglossaryentry{NOP}
{
  name=NOP,
  description={``no operation'', \IFRU{холостая инструкция}{idle instruction}}
}

\newglossaryentry{POKE}
{
  name=POKE,
  description={\IFRU{Инструкция языка BASIC записывающая байт по определенному адресу}
  	{BASIC language instruction writting byte on specific address}}
}

\newglossaryentry{keygenme}
{
  name=keygenme,
  description={\IFRU{Программа, имитирующая защиту вымышленной программы, для которой нужно сделать 
  генератор ключей/лицензий}{A program which imitates fictional software protection,
  for which one needs to make a keys/licenses generator}}
}

\newglossaryentry{dongle}
{
  name=dongle,
  description={\IFRU{Небольшое устройство подключаемое к LPT-порту для принтера (в прошлом) или к USB}
  {Dongle is a small piece of hardware connected to LPT printer port (in past) or to USB}.
  \IFRU{Исполняло функции security token-а, имела память и, иногда,
  секретную (крипто-)хеширующую функцию}
  {Its function was akin to security token, it has some memory and, sometimes,
  secret (crypto-)hashing algorithm}.}
}

\newglossaryentry{thunk function}
{
  name=thunk function,
  description={\IFRU{Крохотная функция делающая только одно: вызывающая другую функцию.}
  {Tiny function with a single role: call another function.}}
}

\newglossaryentry{user mode}
{
  name=user mode,
  description={\IFRU{Режим CPU с ограниченными возможностями в котором он исполняет прикладное ПО. ср.}
  {A restricted CPU mode in which it executes all applied software code. cf.} \gls{kernel mode}.}
}

\newglossaryentry{kernel mode}
{
  name=kernel mode,
  description={\IFRU{Режим CPU с неограниченными возможностями в котором он исполняет ядро OS и драйвера. ср.}
  {A restrictions-free CPU mode in which it executes OS kernel and drivers. cf.} \gls{user mode}.}
}

\newglossaryentry{Windows NT}
{
  name=Windows NT,
  description={Windows NT, 2000, XP, Vista, 7, 8}
}

\newglossaryentry{atomic operation}
{
  name=atomic operation,
  description={
  ``$\alpha{}\tau{}o\mu{}o\varsigma{}$''
  %``atomic''
  \IFRU{означает ``неделимый'' в греческом языке, так что атомарная операция
  это операция которая гарантированно не будет прервана другими тредами}
  {mean ``indivisible'' in Greek, so atomic operation is what guaranteed not
  to be breaked up during operation by other threads}}
}

% to be proofreaded (begin)
\newglossaryentry{NaN}
{
  name=NaN,
  description={
  	\IFRU{не число: специальные случаи чисел с плавающей запятой, обычно сигнализирующие об ошибках}
	{not a number: special cases of floating point numbers, usually signalling about errors}
  }
}

\newglossaryentry{basic block}
{
  name=basic block,
  description={
  	\IFRU{группа инструкций не имеющая инструкций переходов,
	а также не имеющая переходов в середину блока извне.
	В IDA он выглядит как просто список инструкций без строк-разрывов}
	{a group of instructions not having jump/branch instructions, and also not having
	jumps inside block from the outside.
	In IDA it looks just like as a list of instructions without breaking empty lines}
  }
}

\newglossaryentry{NEON}
{
  name=NEON,
  description={\ac{AKA} ``Advanced SIMD''\EMDASH\ac{SIMD} \IFRU{от}{from} ARM}
}

\newglossaryentry{reverse engineering}
{
  name=reverse engineering,
  description={\IFRU{процесс понимания как устроена некая вещь, иногда, с целью клонирования оной}
  {act of understanding, how the thing works, sometimes, in order to clone it}}
}

\newglossaryentry{compiler intrinsic}
{
  name=compiler intrinsic,
  description={\IFRU{Специфичная для компилятора ф-ция не являющаяся обычной библиотечной ф-цией.
	Компилятор вместо её вызова генерирует определенный машинный код.
	Нередко, это псевдофункции для определенной инструкции \ac{CPU}. Читайте больше:}
	{A function specific to a compiler which is not usual library function.
	Compiler generate a specific machine code instead of call to it.
	It is often a pseudofunction for specific \ac{CPU} insruction. Read more:} (\ref{sec:compiler_intrinsic})}
}

\newglossaryentry{heap}
{
  name=heap,
  description={\IFRU{(куча) обычно, большой кусок памяти предоставляемый \ac{OS}, так что прикладное ПО может делить его
  как захочет. malloc()/free() работают с кучей.}
  {usually, a big chunk of memory provided by \ac{OS} so that applications can divide it by themselves as they wish.
  malloc()/free() works with heap.}}
}

\newglossaryentry{name mangling}
{
  name=name mangling,
  description={\IFRU{применяется как минимум в Си++, где компилятору нужно закодировать имя класса,
  метода и типы аргументов в одной
  строке, которая будет внутренним именем ф-ции. читайте также здесь}
  {used at least in C++, where compiler need to encode name of class, method and argument types in the one string,
  which will become internal name of the function. read more here}: \ref{namemangling}}
}


