% done

\section{Hello, world!}
\label{sec:helloworld}

\IFRU{Начнем с знаменитого примера из книги}{Let's start with famous example from the book} "The C programming Language"\footnote{\url{http://en.wikipedia.org/wiki/The_C_Programming_Language}}:

\lstinputlisting{helloworld/1_1.c}

\IFRU{Компилируем в}{Let's compile it in} MSVC 2010: \TT{cl 1.cpp /Fa1.asm}

\IFRU
{(Ключ /Fa означает сгенерировать листинг на ассемблере)}
{(/Fa option mean generate assembly listing file)}

\begin{lstlisting}
CONST	SEGMENT
$SG3830	DB	'hello, world', 00H
CONST	ENDS
PUBLIC	_main
EXTRN	_printf:PROC
; Function compile flags: /Odtp
_TEXT	SEGMENT
_main	PROC
	push	ebp
	mov	ebp, esp
	push	OFFSET $SG3830
	call	_printf
	add	esp, 4
	xor	eax, eax
	pop	ebp
	ret	0
_main	ENDP
_TEXT	ENDS
\end{lstlisting}

\IFRU{Компилятор сгенерировал файл \TT{1.obj}, который впоследствии будет слинкован линкером в \TT{1.exe}.} 
{Compiler generated \TT{1.obj} file which will be linked into \TT{1.exe}.}

\IFRU{В нашем случае, этот файл состоит из двух сегментов: \TT{CONST} (для данных-констант) и \TT{\_TEXT} (для кода).}
{In our case, the file contain two segments: \TT{CONST} (for data constants) and \TT{\_TEXT} (for code).} 

\IFRU{Строка \TT{"hello, world"} в \CCpp имеет тип \TT{const char*}, однако не имеет имени.}
{The string \TT{"hello, world"} in \CCpp has type \TT{const char*}, however hasn't its own name.}

\IFRU{Но компилятору нужно как-то с ней работать, так что он дает ей внутреннее имя \TT{\$SG3830}.}
{But compiler need to work with this string somehow, so it define internal name \TT{\$SG3830} for it.}

\IFRU{Как видно, строка заканчивается нулевым байтом ~--- это требования стандарта \CCpp насчет строк.}
{As we can see, the string is terminated by zero byte ~--- this is \CCpp standard of strings.}

\IFRU{В сегменте кода \TT{\_TEXT} находится пока только одна функция ~--- \TT{\_main}.}
{In the code segment \TT{\_TEXT} there are only one function so far ~--- \TT{\_main}.}

\IFRU{Функция \TT{\_main}, как и практически все функции, начинается с пролога и заканчивается эпилогом.}
{Function \TT{\_main} starting with prologue code and ending with epilogue code, like almost any function.}

\IFRU{Об этом смотрите подробнее в разделе о прологе и эпилоге функции}
{Read more about it in section about function prolog and epilog}
~\ref{sec:prologepilog}.

\IFRU{Далее следует вызов функции \printf}
{After function prologue we see a function \printf call}: \TT{CALL \_printf}. 

\IFRU
{Перед этим вызовом, адрес строки (или указатель на нее) с нашим приветствием при помощи инструкции \PUSH помещается в стек.}
{Before the call, string address (or pointer to it) containing our greeting is placed into stack with help of \PUSH instruction.}

\IFRU{После того как функция \printf возвращает управление в функцию \main, адрес строки (или указатель на нее) все еще лежит в стеке.}
{When \printf function returning control flow to \main function, string address (or pointer to it) is still in stack.}

\IFRU{Так как он больше не нужен, то указатель стека (регистр \ESP) корректируется.} 
{Because we do not need it anymore, stack pointer (\ESP register) is to be corrected.}

\TT{ADD ESP, 4} \IFRU{означает прибавить 4 к значению в регистре \ESP.}
{mean add 4 to the value in \ESP register.}

\IFRU
{Так как, это 32-битный код, для передачи адреса нужно аккурат 4 байта. В x64-коде это 8 байт.}
{Since it is 32-bit code, we need exactly 4 bytes for address passing through the stack. 
It's 8 bytes in x64-code}

\IFRU
{Некоторые компиляторы, например Intel C++ Compiler, в этой же ситуации, могут вместо 
\ADD сгенерировать \TT{POP ECX} (это можно встретить например в коде Oracle RDBMS, им скомпилированном), 
что почти то же самое, только портится значение в регистре \ECX.}
{Some compilers like Intel C++ Compiler, at the same point, could emit \TT{POP ECX} 
instead of \ADD (for example, this can be observed in Oracle RDBMS code, compiled by Intel compiler), 
and this instruction has almost the same effect, but \ECX register contents will be rewritten.}

\IFRU
{Возможно, компилятор применяет \TT{POP ECX} потому что эта инструкция короче (1 байт против 3).}
{Probably, Intel compiler using \TT{POP ECX} because this instruction's opcode is shorter then 
\TT{ADD ESP, x} (1 byte against 3).}

\IFRU{О стеке можно прочитать в соответствующем разделе}{More about stack in relevant section}~\ref{sec:stack}.

\IFRU{После вызова \printf, в оригинальном коде на \CCpp было \TT{return 0} 
- вернуть 0 в качестве результата функции \main.} 
{After \printf call, in original \CCpp code was \TT{return 0} ~--- return zero as a \main function result.} 

\IFRU{В сгенерированном коде это обеспечивается инструкцией}
{In the generated code this is implemented by instruction} \TT{XOR EAX, EAX} 

\IFRU
{\XOR, на самом деле, как легко догадаться, "исключающее ИЛИ"}
{\XOR, in fact, just "eXclusive OR"}
\footnote{\url{http://en.wikipedia.org/wiki/Exclusive_or}}, 
\IFRU
{но компиляторы часто используют его вместо простого}
{but compilers using it often instead of}
\TT{MOV EAX, 0} ~--- 
\IFRU
{снова опкод немного короче (2 байта против 5).}
{slightly shorter opcode again (2 bytes against 5).}

\IFRU{Бывает так, что некоторые компиляторы генерируют}{Some compilers emitting} 
\TT{SUB EAX, EAX}, 
\IFRU
{что значит, \IT{отнять значение \EAX от \EAX}, в любом случае это даст 0 в результате.}
{which mean \IT{SUBtract \EAX value from \EAX}, which is in any case will result zero.}

\IFRU{Самая последняя инструкция \RET возвращает управление в вызывающую функцию.
Обычно, это код \CCpp CRT\footnote{C Run-Time Code}, который, в свою очередь, 
вернет управление операционной системе.}
{Last instruction \RET returning control flow to calling function.
Usually, it's \CCpp CRT\footnote{C Run-Time Code} code, which, in turn, 
return control to operation system.}

\IFRU{Теперь скомпилируем то же самое компилятором GCC 4.4.1 в Linux}
{Now let's try to compile the same \CCpp code in GCC 4.4.1 compiler in Linux}: \TT{gcc 1.c -o 1}

\IFRU{Затем при помощи \IDA. посмотрим как создалась функция \main.}
{After, with the \IDA disassembler assistance, let's see how \main function was created.} 

\IFRU{С другой стороны, мы можем посмотреть результат работы GCC при помощи ключа}
{Note: we could also see GCC assembler result applying option} \TT{-S -masm=intel})

\begin{lstlisting}
main            proc near

var_10          = dword ptr -10h

                push    ebp
                mov     ebp, esp
                and     esp, 0FFFFFFF0h
                sub     esp, 10h
                mov     eax, offset aHelloWorld ; "hello, world"
                mov     [esp+10h+var_10], eax
                call    _printf
                mov     eax, 0
                leave
                retn
main            endp
\end{lstlisting}

\IFRU{Почти то же самое, за исключением того что в прологе функции мы видим \TT{AND ESP, 0FFFFFFF0h} ~--- эта инструкция выравнивает значение в \ESP по 16-байтной границе, делая некоторые значения 
в стеке также выровненными по этой границе.}
{Almost the same, only sole exception that in function prologue we see \TT{AND ESP, 0FFFFFFF0h} ~--- 
this instruction aligning \ESP value on 16-byte border, resulting some values in stack aligned too.}

\TT{SUB ESP, 10h} \IFRU{выделяет в стеке 16 байт, хотя, как будет видно далее, нам достаточно только 4.}{allocate 16 bytes in stack, although, as we could see below, we need only 4.} 

\IFRU{Это происходит потому что количество выделяемого места в локальном стеке тоже выровнено по 16-байтной границе.}{This is because the size of allocated stack is also aligned on 16-byte border.}

\IFRU{Адрес строки (или указатель на строку) затем записывается прямо в стек без помощи инструкции \PUSH.}
{String address (or pointer to string) is then writing directly into stack space without \PUSH instruction use.}

\IFRU{Затем вызывается \printf.}{Then \printf function is called.}

\IFRU{В отличие от MSVC, GCC в компиляции без включенной оптимизации генерирует \TT{MOV EAX, 0} вместо более короткого опкода.}{Unlike MSVC, GCC while compiling without optimization turned on, emitting \TT{MOV EAX, 0} instead of shorter opcode.}

\IFRU{Последняя инструкция \LEAVE ~--- это аналог команд \TT{MOV ESP, EBP} и \TT{POP EBP} ~--- то есть возврат указателя стека и регистра \EBP в первоначальное состояние.} 
{The last instruction \LEAVE ~--- is \TT{MOV ESP, EBP} and \TT{POP EBP} instructions pair equivalent ~--- in other words, this instruction setting back stack pointer (\ESP) and \EBP register to its initial state.} 

\IFRU{Это необходимо, т.к., в начале функции мы модифицировали регистры \ESP и \EBP (при помощи}
{This is necessary because we modified these register values (\ESP and \EBP) at the function start (executing}
\TT{\MOV EBP, ESP} / \TT{AND ESP, ...}).

