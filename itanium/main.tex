% to be proofreaded
\section{Itanium}
\label{itanium}
\index{Itanium}
\IFRU{Еще одна очень интересная архитектура (хотя и почти провальная) это Intel Itanium (\ac{IA64})}
{Although almost failed, another very interesting architecture is Intel Itanium (\ac{IA64})}.
\IFRU{Другие \ac{OOE}-процессоры сами решают, как переставлять инструкции и исполнять их параллельно,
\ac{EPIC} это была попытка сдвинуть эти решения на компилятор: дать ему возможность самому 
группировать инструкции во время компиляции.}
{While \ac{OOE} CPUs decides how to rearrange instructrions and execute them in parallel,
\ac{EPIC} was an attempt to shift these decisions to the compiler:
to let it group instructions at the compile stage.}

\IFRU{Это вылилось в очень сложные компиляторы}
{This result in notoriously complex compilers.}

\IFRU{Вот один пример \ac{IA64}-кода: простой криптоалгоритм из ядра Linux}
{Here is one sample of \ac{IA64} code: simple cryptoalgorithm from Linux kernel}:

\lstinputlisting[caption=Linux kernel 3.2.0.4]{itanium/tea_from_linux.c}

\IFRU{И вот как он был скомпилирован}{Here is how it was compiled}:

\lstinputlisting[caption=Linux Kernel 3.2.0.4 \IFRU{для}{for} Itanium 2 (McKinley)]{itanium/ia64_linux_3.2.0.4_mckinley.lst}

\IFRU{Прежде всего, все инструкции \ac{IA64} сгруппированы в пачки (bundle) из трех инструкций}
{First of all, all \ac{IA64} instructions are grouped into 3-instruction bundles}.
\IFRU{Каждая пачка имеет размер 16 байт и состоит из template-кода и трех инструкций}
{Each bundle has size of 16 bytes and consists of template code + 3 instructions}.
\IFRU{\IDA показывает пачки как 6+6+4 байт --- вы можете легко заметить эту повторяющуюся структуру}
{\IDA shows bundles into 6+6+4 bytes~---you may easily spot the pattern}.

\IFRU{Все 3 инструкции каждой пачки обычно исполняются одновременно, если только у какой-то инструкции
нет ``стоп-бита''}
{All 3 instructions from each bundle usually executes simultaneously, unless one of instructions have
``stop bit''}.

\IFRU{Вероятно, инженеры Intel и HP собрали статистику наиболее встречающихся шаблонных сочетаний
инструкций и решили ввести типы пачек (\ac{AKA} ``templates''): код пачки определяет типы инструкций
в пачке}
{Supposedly, Intel and HP engineers gathered statistics of most occurred instruction patterns and decided to bring
bundle types (\ac{AKA} ``templates''): a bundle code defines instruction types in the bundle}.
\IFRU{Их всего 12}{There are 12 of them}.
\IFRU{Например, нулевой тип это \TT{MII}, что означает: первая инструкция это Memory (загрузка
или запись в память), вторая и третья это I (инструкция, работающая с целочисленными значениями).}
{For example, zeroth bundle type is \TT{MII}, meaning: 
first instruction is Memory (load or store), second and third are I (integer instructions).}
\IFRU{Еще один пример, тип 0x1d: \TT{MFB}: первая инструкция это Memory (загрузка или запись
в память), вторая это Float (инструкция, работающая с \ac{FPU}), третья это Branch (инструкция
перехода).}
{Another example is bundle type 0x1d: \TT{MFB}:
first instruction is Memory (load or store), second is Float 
(\ac{FPU} instruction), third is Branch (branch instruction).}

\IFRU{Если компилятор не может подобрать подходящую инструкцию в соответствующее место пачки,
он может вставить \ac{NOP}:
вы можете здесь увидеть инструкции \TT{nop.i} (\ac{NOP} на том месте где должна была бы находиться
целочисленная инструкция) или \TT{nop.m} (инструкция обращения к памяти должна была находиться
здесь).}
{If compiler cannot pick suitable instruction to relevant bundle slot, it may insert \ac{NOP}:
you may see here
\TT{nop.i} instructions (\ac{NOP} at the place where integer instructrion might be) or \TT{nop.m} 
(a memory instruction might be at this slot).}
\IFRU{Если вручную писать на ассемблере, \ac{NOP}-ы могут вставляться автоматически}
{\ac{NOP}s are inserted automatically when one use assembly language manually}.

\IFRU{И это еще не все. Пачки тоже могут быть объединены в группы}
{And that is not all. Bundles are also grouped}.
\IFRU{Каждая пачка может иметь ``стоп-бит'', так что все следующие друг за другом пачки вплоть до той,
что имеет стоп-бит, могут быть исполнены одновременно}
{Each bundle may have ``stop bit'',
so all the consequtive bundles with terminating bundle which have ``stop bit'' 
may be executed simultaneously}.
\IFRU{На практике, Itanium 2 может исполнять 2 пачки одновременно, таким образом, исполнять
6 инструкций одновременно}
{In practice, Itanium 2 may execute 2 bundles at once, resulting execution of 6 instructrions at once}.

\IFRU{Так что все инструкции внутри пачки и группы не могут мешать друг другу (т.е., не должны
иметь data hazard-ов)}
{So all instructions inside bundle and bundle group cannot interfere with each other 
(i.e., should not have data hazards)}.
\IFRU{А если это так, то результаты будут непредсказуемые}{If they do, results will be undefined}.

\IFRU{На ассемблере, каждый стоп-бит маркируется как \TT{;;} (две точки с запятой) после инструкции}
{Each stop bit is marked in assembly language as \TT{;;} (two semicolons) after instruction}.
\IFRU{Так, инструкции на [180-19c] могут быть исполнены одновременно: они не мешают друг другу}
{So, instructions at [180-19c] may be executed simultaneously:
they do not interfere}. \IFRU{Следующая группа:}{Next group is} [1a0-1bc].

\IFRU{Мы также видим стоп-бит на 22c}
{We also see a stop bit at 22c}.
\IFRU{Следующая инструкция на 230 также имеет стоп-бит}{The next instruction at 230 have stop bit too}.
\IFRU{Это значит, что эта инструкция должна исполняться изолированно от всех остальных (как в \ac{CISC})}
{This mean, this instruction is to be executed as isolated from all others (as in \ac{CISC})}.
\IFRU{Действительно: следующая инструкция на 236 использует результат полученный от нее (значение
в регистре r10), так что они не могут исполняться одновременно}
{Indeed: the next instructrion at 236 use result from it (value in register r10),
so they cannot be executed at the same time}.
\IFRU{Должно быть, компилятор не смог найти лучший способ распараллелить инструкции, или, иными
словами, загрузить \ac{CPU} насколько это возможно, отсюда так много стоп-битов и \ac{NOP}-ов}
{Apparently, compiler was not able to find a better way to parallelize instructions, which is,
in other words, to load \ac{CPU} as much as possible, hence too much stop bits and \ac{NOP}s}.
\IFRU{Писать на ассемблере вручную это также очень трудная задача: программист должен группировать
инструкции вручную}
{Manual assembly programming is tedious job as well: programmer should group instructrions manually}.

\IFRU{У программиста остается возможность добавлять стоп-биты к каждой инструкции, но это
сведет на нет всю мощность Itanium, ради которой он создавался}
{Programmer is still able to add stop-bits to each instructions, but this will degrade
all performance Itanium was made for}.

\IFRU{Интересные примеры написания \ac{IA64}-кода вручную можно найти в исходниках ядра Linux}
{Interesting examples of manual \ac{IA64} assembly code can be found in Linux kernel sources}:

\url{http://lxr.free-electrons.com/source/arch/ia64/lib/}.

\IFRU{Еще одна вводная статья об ассемблере Itanium}
{Another introductory Itanium assembly paper}: \cite{Itanium}.

\IFRU{Еще одна интересная особенность Itanium это \IT{speculative execution} (исполнение инструкций
заранее, когда еще не известно, нужно ли это) и бит NaT (``not a thing''), отдаленно напоминающий
\gls{NaN}-числа}
{Another very interesting Itanium feature is \IT{speculative execution} and NaT (``not a thing'') bit,
somewhat resembling \gls{NaN} numbers}:

\url{http://blogs.msdn.com/b/oldnewthing/archive/2004/01/19/60162.aspx}.

