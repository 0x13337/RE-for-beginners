\newcommand{\myhrule}{\begin{center}\rule{0.5\textwidth}{.4pt}\end{center}}

% TODO: find better name:
\newcommand{\myincludegraphics}[1]{\includegraphics[width=\textwidth]{#1}}
\newcommand{\myincludegraphicsSmall}[1]{\includegraphics[width=0.3\textwidth]{#1}}

\ifdefined\ebook
\newcommand{\myincludegraphicsSmallOrNormalForEbook}{\myincludegraphics}
\else
\newcommand{\myincludegraphicsSmallOrNormalForEbook}{\myincludegraphicsSmall}
\fi

%\newcommand*{\TT}[1]{\texttt{#1}}
% synonyms, so far
\newcommand*{\InSqBrackets}[1]{\lbrack{}#1\rbrack{}}
\newcommand*{\TT}[1]{\colorbox{light-gray}{\texttt{#1}}}
\newcommand*{\GTT}[1]{\colorbox{light-gray}{\texttt{#1}}}
\newcommand*{\IT}[1]{\textit{#1}}
\newcommand*{\EN}[1]{\iflanguage{english}{#1}{}}

\newcommand{\HeaderColor}{\cellcolor{blue!25}}

\ifdefined\RUSSIAN{}
\newcommand*{\RU}[1]{\iflanguage{russian}{#1}{}}
\else
\newcommand*{\RU}[1]{}
\fi

\ifdefined\SPANISH{}
\newcommand*{\ES}[1]{\iflanguage{spanish}{#1}{}}
\else
\newcommand*{\ES}[1]{}
\fi

\ifdefined\ITALIAN{}
\newcommand*{\ITA}[1]{\iflanguage{italian}{#1}{}}
\else
\newcommand*{\ITA}[1]{}
\fi

\ifdefined\BRAZILIAN{}
\newcommand*{\PTBR}[1]{\iflanguage{portuges}{#1}{}}
\else
\newcommand*{\PTBR}[1]{}
\fi

\ifdefined\POLISH{}
\newcommand*{\PL}[1]{\iflanguage{polish}{#1}{}}
\else
\newcommand*{\PL}[1]{}
\fi

\ifdefined\GERMAN{}
%\newcommand*{\DE}[1]{\iflanguage{german}{#1}{}}
\newcommand*{\DE}[1]{#1}
\else
\newcommand*{\DE}[1]{}
\fi

\ifdefined\THAI{}
\newcommand*{\THA}[1]{\iflanguage{thai}{#1}{}}
\else
\newcommand*{\THA}[1]{}
\fi

\ifdefined\DUTCH{}
\newcommand*{\NL}[1]{\iflanguage{dutch}{#1}{}}
\else
\newcommand*{\NL}[1]{}
\fi

\ifdefined\FRENCH{}
\newcommand*{\FR}[1]{\iflanguage{french}{#1}{}}
\else
\newcommand*{\FR}[1]{}
\fi

\newcommand{\ESph}{\ES{Spanish text placeholder}}
\newcommand{\PTBRph}{\PTBR{Brazilian Portuguese text placeholder}}
\newcommand{\PLph}{\PL{Polish text placeholder}}
\newcommand{\ITAph}{\ITA{Italian text placeholder}}
\newcommand{\DEph}{\DE{German text placeholder}}
\newcommand{\THAph}{\THA{Thai text placeholder}}
\newcommand{\NLph}{\NL{Dutch text placeholder}}
\newcommand{\FRph}{\FR{French text placeholder}}

\newcommand{\TITLE}{%
	\RU{Reverse Engineering для начинающих}%
	\EN{Reverse Engineering for Beginners}%
	\ES{Ingenier\'ia Inversa para Principiantes}%
	\PTBRph{}%
	\DEph{}\PLph{}%
	\ITAph{}%
	\THAph{}\NLph{}%
	\FR{Rétro-ingénierie pour les Débutants}%
}

\ifdefined\RUSSIAN
\newcommand{\AUTHOR}{Денис Юричев}
\else
\newcommand{\AUTHOR}{Dennis Yurichev}
\fi

\newcommand{\EMAIL}{dennis(a)yurichev.com}

\newcommand*{\dittoclosing}{---''---}
\newcommand*{\EMDASH}{\RU{~--- }\EN{---}\ES{---}\PTBR{---}\PL{---}\ITA{---}\DE{---}\THA{---}\NL{---}}
\newcommand*{\AsteriskOne}{${}^{*}$}
\newcommand*{\AsteriskTwo}{${}^{**}$}
\newcommand*{\AsteriskThree}{${}^{***}$}
\newcommand{\q}[1]{\enquote{#1}}
\newcommand{\var}[1]{\textit{#1}}

\newcommand{\ttf}{\GTT{f()}\xspace}
\newcommand{\ttfone}{\GTT{f1()}\xspace}

% FIXME get rid of
% without dot!
\newcommand{\etc}{%
	\RU{и~т.д}%
	\EN{etc}%
	\ES{etc}%
	\PTBRph{}%
	\PLph{}%
	\ITAph{}%
	\DEph{}%
	\THAph{}%
	\NLph{}%
	\FR{etc}%
}

% http://tex.stackexchange.com/questions/32160/new-line-after-paragraph
\newcommand{\myparagraph}[1]{\paragraph{#1}\mbox{}\\}
\newcommand{\mysubparagraph}[1]{\subparagraph{#1}\mbox{}\\}

\newcommand{\figname}{%
	\RU{илл.}%
	\EN{fig.}%
	\ES{fig.}%
	\PTBRph{}%
	\PLph{}%
	\ITAph{}%
	\DEph{}%
	\THAph{}%
	\NLph{}%
	\FR{fig.}%
\xspace}
\newcommand{\figref}[1]{\figname{}\ref{#1}\xspace}
\newcommand{\listingname}{%
	\RU{листинг.}%
	\EN{listing.}%
	\ES{listado.}%
	\PTBRph{}%
	\PLph{}%
	\ITAph{}%
	\DEph{}%
	\THAph{}%
	\NLph{}%
	\FR{liste.}%
\xspace}
\newcommand{\lstref}[1]{\listingname{}\ref{#1}\xspace}
\newcommand{\bitENRU}{%
	\RU{бит}%
	\EN{bit}%
	\ES{bit}%
	\PTBRph{}%
	\PLph{}%
	\ITAph{}%
	\DEph{}%
	\THAph{}%
	\NLph{}%
	\FR{bit}%
\xspace}
% FIXME get rid of:
\newcommand{\bitsENRU}{%
	\RU{бита}%
	\EN{bits}%
	\ES{bits}%
	\PTBRph{}%
	\PLph{}%
	\ITAph{}%
	\DEph{}%
	\THAph{}%
	\NLph{}%
	\FR{bits}%
\xspace}
\newcommand{\Sourcecode}{%
	\RU{Исходный код}%
	\EN{Source code}%
	\ES{C\'odigo fuente}%
	\PTBRph{}%
	\PLph{}%
	\ITAph{}%
	\DEph{}%
	\THAph{}%
	\NLph{}%
	\FR{Code source}%
\xspace}
\newcommand{\Seealso}{%
	\RU{См. также}%
	\EN{See also}%
	\ES{V\'ease tambi\'en}%
	\PTBRph{}%
	\PLph{}%
	\ITAph{}%
	\DEph{}%
	\THAph{}%
	\NLph{}%
	\FR{Voir également}%
\xspace}
\newcommand{\MacOSX}{Mac OS X\xspace}

% FIXME TODO non-overlapping color!
% \newcommand{\headercolor}{\cellcolor{blue!25}}
\newcommand{\headercolor}{}

\newcommand{\tableheader}{\headercolor{}%
	\RU{смещение}%
	\EN{offset}%
	\ES{offset}%
	\PLph{}%
	\ITAph{}%
	\DEph{}%
	\THAph{}%
	\NLph{}%
	\FR{offset}%
& \headercolor{}%
	\RU{описание}%
	\EN{description}%
	\ES{descripci\'on}%
	\PTBRph{}%
	\PLph{}%
	\ITAph{}%
	\DEph{}%
	\THAph{}%
	\NLph{}%
	\FR{description}%
}

\newcommand{\IDA}{\ac{IDA}\xspace}

\newcommand{\tracer}{\protect\gls{tracer}\xspace}

\newcommand{\Tchar}{\IT{char}\xspace} 
\newcommand{\Tint}{\IT{int}\xspace}
\newcommand{\Tbool}{\IT{bool}\xspace}
\newcommand{\Tfloat}{\IT{float}\xspace}
\newcommand{\Tdouble}{\IT{double}\xspace}
\newcommand{\Tvoid}{\IT{void}\xspace}
\newcommand{\ITthis}{\IT{this}\xspace}

\newcommand{\Ox}{\GTT{/Ox}\xspace}
\newcommand{\Obzero}{\GTT{/Ob0}\xspace}
\newcommand{\Othree}{\GTT{-O3}\xspace}

\newcommand{\oracle}{Oracle RDBMS\xspace}

\newcommand{\idevices}{iPod/iPhone/iPad\xspace}
\newcommand{\olly}{OllyDbg\xspace}

% common C functions
\newcommand{\printf}{\GTT{printf()}\xspace} 
\newcommand{\puts}{\GTT{puts()}\xspace} 
\newcommand{\main}{\GTT{main()}\xspace} 
\newcommand{\qsort}{\GTT{qsort()}\xspace} 
\newcommand{\strlen}{\GTT{strlen()}\xspace} 
\newcommand{\scanf}{\GTT{scanf()}\xspace} 
\newcommand{\rand}{\GTT{rand()}\xspace} 


% for easier fiddling with formatting of all instructions together
\newcommand{\INS}[1]{\GTT{#1}\xspace}

% x86 instructions
\newcommand{\ADD}{\INS{ADD}}
\newcommand{\ADRP}{\INS{ADRP}}
\newcommand{\AND}{\INS{AND}}
\newcommand{\CALL}{\INS{CALL}}
\newcommand{\CPUID}{\INS{CPUID}}
\newcommand{\CMP}{\INS{CMP}}
\newcommand{\DEC}{\INS{DEC}}
\newcommand{\FADDP}{\INS{FADDP}}
\newcommand{\FCOM}{\INS{FCOM}}
\newcommand{\FCOMP}{\INS{FCOMP}}
\newcommand{\FCOMI}{\INS{FCOMI}}
\newcommand{\FCOMIP}{\INS{FCOMIP}}
\newcommand{\FUCOM}{\INS{FUCOM}}
\newcommand{\FUCOMI}{\INS{FUCOMI}}
\newcommand{\FUCOMIP}{\INS{FUCOMIP}}
\newcommand{\FUCOMPP}{\INS{FUCOMPP}}
\newcommand{\FDIVR}{\INS{FDIVR}}
\newcommand{\FDIV}{\INS{FDIV}}
\newcommand{\FLD}{\INS{FLD}}
\newcommand{\FMUL}{\INS{FMUL}}
\newcommand{\MUL}{\INS{MUL}}
\newcommand{\FSTP}{\INS{FSTP}}
\newcommand{\FDIVP}{\INS{FDIVP}}
\newcommand{\IDIV}{\INS{IDIV}}
\newcommand{\IMUL}{\INS{IMUL}}
\newcommand{\INC}{\INS{INC}}
\newcommand{\JAE}{\INS{JAE}}
\newcommand{\JA}{\INS{JA}}
\newcommand{\JBE}{\INS{JBE}}
\newcommand{\JB}{\INS{JB}}
\newcommand{\JE}{\INS{JE}}
\newcommand{\JGE}{\INS{JGE}}
\newcommand{\JG}{\INS{JG}}
\newcommand{\JLE}{\INS{JLE}}
\newcommand{\JL}{\INS{JL}}
\newcommand{\JMP}{\INS{JMP}}
\newcommand{\JNE}{\INS{JNE}}
\newcommand{\JNZ}{\INS{JNZ}}
\newcommand{\JNA}{\INS{JNA}}
\newcommand{\JNAE}{\INS{JNAE}}
\newcommand{\JNB}{\INS{JNB}}
\newcommand{\JNBE}{\INS{JNBE}}
\newcommand{\JZ}{\INS{JZ}}
\newcommand{\JP}{\INS{JP}}
\newcommand{\Jcc}{\INS{Jcc}}
\newcommand{\SETcc}{\INS{SETcc}}
\newcommand{\LEA}{\INS{LEA}}
\newcommand{\LOOP}{\INS{LOOP}}
\newcommand{\MOVSX}{\INS{MOVSX}}
\newcommand{\MOVZX}{\INS{MOVZX}}
\newcommand{\MOV}{\INS{MOV}}
\newcommand{\NOP}{\INS{NOP}}
\newcommand{\POP}{\INS{POP}}
\newcommand{\PUSH}{\INS{PUSH}}
\newcommand{\NOT}{\INS{NOT}}
\newcommand{\NOR}{\INS{NOR}}
\newcommand{\RET}{\INS{RET}}
\newcommand{\RETN}{\INS{RETN}}
\newcommand{\SETNZ}{\INS{SETNZ}}
\newcommand{\SETBE}{\INS{SETBE}}
\newcommand{\SETNBE}{\INS{SETNBE}}
\newcommand{\SUB}{\INS{SUB}}
\newcommand{\TEST}{\INS{TEST}}
\newcommand{\TST}{\INS{TST}}
\newcommand{\FNSTSW}{\INS{FNSTSW}}
\newcommand{\SAHF}{\INS{SAHF}}
\newcommand{\XOR}{\INS{XOR}}
\newcommand{\OR}{\INS{OR}}
\newcommand{\SHL}{\INS{SHL}}
\newcommand{\SHR}{\INS{SHR}}
\newcommand{\SAR}{\INS{SAR}}
\newcommand{\LEAVE}{\INS{LEAVE}}
\newcommand{\MOVDQA}{\INS{MOVDQA}}
\newcommand{\MOVDQU}{\INS{MOVDQU}}
\newcommand{\PADDD}{\INS{PADDD}}
\newcommand{\PCMPEQB}{\INS{PCMPEQB}}
\newcommand{\LDR}{\INS{LDR}}
\newcommand{\LSL}{\INS{LSL}}
\newcommand{\LSR}{\INS{LSR}}
\newcommand{\ASR}{\INS{ASR}}
\newcommand{\RSB}{\INS{RSB}}
\newcommand{\BTR}{\INS{BTR}}
\newcommand{\BTS}{\INS{BTS}}
\newcommand{\BTC}{\INS{BTC}}
\newcommand{\LUI}{\INS{LUI}}
\newcommand{\ORI}{\INS{ORI}}
\newcommand{\BIC}{\INS{BIC}}
\newcommand{\EOR}{\INS{EOR}}
\newcommand{\MOVS}{\INS{MOVS}}
\newcommand{\LSLS}{\INS{LSLS}}
\newcommand{\LSRS}{\INS{LSRS}}
\newcommand{\FMRS}{\INS{FMRS}}
\newcommand{\CMOVNE}{\INS{CMOVNE}}
\newcommand{\CMOVNZ}{\INS{CMOVNZ}}
\newcommand{\ROL}{\INS{ROL}}
\newcommand{\CSEL}{\INS{CSEL}}
\newcommand{\SLL}{\INS{SLL}}
\newcommand{\SLLV}{\INS{SLLV}}
\newcommand{\SW}{\INS{SW}}
\newcommand{\LW}{\INS{LW}}

% x86 flags

\newcommand{\ZF}{\GTT{ZF}\xspace} 
\newcommand{\CF}{\GTT{CF}\xspace} 
\newcommand{\PF}{\GTT{PF}\xspace} 

% x86 registers

\newcommand{\AL}{\GTT{AL}\xspace} 
\newcommand{\AH}{\GTT{AH}\xspace} 
\newcommand{\AX}{\GTT{AX}\xspace} 
\newcommand{\EAX}{\GTT{EAX}\xspace} 
\newcommand{\EBX}{\GTT{EBX}\xspace} 
\newcommand{\ECX}{\GTT{ECX}\xspace} 
\newcommand{\EDX}{\GTT{EDX}\xspace} 
\newcommand{\DL}{\GTT{DL}\xspace} 
\newcommand{\ESI}{\GTT{ESI}\xspace} 
\newcommand{\EDI}{\GTT{EDI}\xspace} 
\newcommand{\EBP}{\GTT{EBP}\xspace} 
\newcommand{\ESP}{\GTT{ESP}\xspace} 
\newcommand{\RSP}{\GTT{RSP}\xspace} 
\newcommand{\EIP}{\GTT{EIP}\xspace} 
\newcommand{\RIP}{\GTT{RIP}\xspace} 
\newcommand{\RAX}{\GTT{RAX}\xspace} 
\newcommand{\RBX}{\GTT{RBX}\xspace} 
\newcommand{\RCX}{\GTT{RCX}\xspace} 
\newcommand{\RDX}{\GTT{RDX}\xspace} 
\newcommand{\RBP}{\GTT{RBP}\xspace} 
\newcommand{\RSI}{\GTT{RSI}\xspace} 
\newcommand{\RDI}{\GTT{RDI}\xspace} 
\newcommand*{\ST}[1]{\GTT{ST(#1)}\xspace}
\newcommand*{\XMM}[1]{\GTT{XMM#1}\xspace}

% ARM
\newcommand*{\Reg}[1]{\GTT{R#1}\xspace}
\newcommand*{\RegX}[1]{\GTT{X#1}\xspace}
\newcommand*{\RegW}[1]{\GTT{W#1}\xspace}
\newcommand*{\RegD}[1]{\GTT{D#1}\xspace}
\newcommand{\ADREQ}{\GTT{ADREQ}\xspace}
\newcommand{\ADRNE}{\GTT{ADRNE}\xspace}
\newcommand{\BEQ}{\GTT{BEQ}\xspace}

% instructions descriptions
\newcommand{\ASRdesc}{%
	\RU{арифметический сдвиг вправо}%
	\EN{arithmetic shift right}%
	\ES{desplazamiento aritm\'etico a la derecha}%
	\PTBRph{}%
	\PLph{}%
	\ITAph{}%
	\DEph{}%
	\THAph{}\NLph{}%
	\FR{décalage arithmétique vers la gauche}%
}

% x86 registers tables
\newcommand{\RegHeaderTop}{
	\multicolumn{8}{ | c | }{
	\RU{Номер байта:}
	\EN{Byte number:}
	\ES{}%
	\PTBRph{}%
	\PLph{}%
	\ITAph{}%
	\DEph{}%
	\THAph{}%
	\NLph{}%
	\FR{Nombre d'octets}%
	}
}

% TODO: non-overlapping color!
\newcommand{\RegHeader}{
\RU{ 7-й & 6-й & 5-й & 4-й & 3-й & 2-й & 1-й & 0-й }%
\EN{ 7th & 6th & 5th & 4th & 3rd & 2nd & 1st & 0th }%
\ES{ 7mo & 6to & 5to & 4to & 3ro & 2do & 1ro & 0 }%
\PTBRph{}%
\PLph{}%
\ITAph{}%
\DEph{}%
\THAph{}%
\NLph{}%
\FR{7ème & 6ème & 5ème & 4ème & 3ème & 2ème & 1er & 0ème}%
}

% FIXME навести порядок тут...
\newcommand{\RegTableThree}[5]{
\begin{center}
\begin{tabular}{ | l | l | l | l | l | l | l | l | l |}
\hline
\RegHeaderTop \\
\hline
\RegHeader \\
\hline
\multicolumn{8}{ | c | }{#1} \\
\hline
\multicolumn{4}{ | c | }{} & \multicolumn{4}{ c | }{#2} \\
\hline
\multicolumn{6}{ | c | }{} & \multicolumn{2}{ c | }{#3} \\
\hline
\multicolumn{6}{ | c | }{} & #4 & #5 \\
\hline
\end{tabular}
\end{center}
}

\newcommand{\RegTableOne}[5]{\RegTableThree{#1\textsuperscript{x64}}{#2}{#3}{#4}{#5}}

\newcommand{\RegTableTwo}[4]{
\begin{center}
\begin{tabular}{ | l | l | l | l | l | l | l | l | l |}
\hline
\RegHeaderTop \\
\hline
\RegHeader \\
\hline
\multicolumn{8}{ | c | }{#1\textsuperscript{x64}} \\
\hline
\multicolumn{4}{ | c | }{} & \multicolumn{4}{ c | }{#2} \\
\hline
\multicolumn{6}{ | c | }{} & \multicolumn{2}{ c | }{#3} \\
\hline
\multicolumn{7}{ | c | }{} & #4\textsuperscript{x64} \\
\hline
\end{tabular}
\end{center}
}

\newcommand{\RegTableFour}[4]{
\begin{center}
\begin{tabular}{ | l | l | l | l | l | l | l | l | l |}
\hline
\RegHeaderTop \\
\hline
\RegHeader \\
\hline
\multicolumn{8}{ | c | }{#1} \\
\hline
\multicolumn{4}{ | c | }{} & \multicolumn{4}{ c | }{#2} \\
\hline
\multicolumn{6}{ | c | }{} & \multicolumn{2}{ c | }{#3} \\
\hline
\multicolumn{7}{ | c | }{} & #4 \\
\hline
\end{tabular}
\end{center}
}

\newcommand{\ReturnAddress}{%
	\RU{Адрес возврата}%
	\EN{Return Address}%
	\ES{Direcci\'on de Retorno}%
	\NL{Return Adres}%
	\PTBR{Endereço de retorno}%
	\ITAph{}
	\FR{Adresse de retour}%
}

\newcommand{\localVariable}{%
	\RU{локальная переменная}%
	\EN{local variable}%
	\PTBR{Variável local}%
	\FR{variable locale}%
}
\newcommand{\savedValueOf}{%
	\RU{сохраненное значение}%
	\EN{saved value of}%
	\PTBR{Valor salvo de}%
	\FR{valeur enregistrée de}%
}

% was in common_phrases.tex
% for index
\newcommand{\GrepUsage}{\RU{Использование grep}\EN{grep usage}\PTBR{Uso do grep}\ES{Uso de grep}\FR{Utilisation de grep}}
\newcommand{\SyntacticSugar}{\RU{Синтаксический сахар}\EN{Syntactic Sugar}\PTBR{Açúcar sintático}\ES{Azúcar sintáctica}\FR{Sucre syntaxique}}
\newcommand{\CompilerAnomaly}{\RU{Аномалии компиляторов}\EN{Compiler's anomalies}\PTBR{Anomalias do compilador}\ES{Anomalías del compilador}\FR{Anomalies du compilateur}}
\newcommand{\CLanguageElements}{\RU{Элементы языка Си}\EN{C language elements}\PTBR{Elementos da linguagem C}\ES{Elementos del lenguaje C}\FR{Eléments du langage C}}
\newcommand{\CStandardLibrary}{\RU{Стандартная библиотека Си}\EN{C standard library}\PTBR{Biblioteca padrão C}\ES{Librería estándar C}\FR{Librairie standard C}}
\newcommand{\Instructions}{\RU{Инструкции}\EN{Instructions}\PTBR{Instruções}\ES{Instrucciones}\FR{Instructions}}
\newcommand{\Pseudoinstructions}{\RU{Псевдоинструкции}\EN{Pseudoinstructions}\PTBR{Pseudo-instruções}\ES{Pseudo-instrucciones}\FR{Pseudo-instructions}}
\newcommand{\Prefixes}{\RU{Префиксы}\EN{Prefixes}\PTBR{Prefixos}\ES{Prefijos}\FR{Préfixes}}

\newcommand{\Flags}{\RU{Флаги}\EN{Flags}\PTBR{Flags}\ES{Flags}\FR{Flags}}
\newcommand{\Registers}{\RU{Регистры}\EN{Registers}\PTBR{Registradores}\ES{Registros}\FR{Registres}}
\newcommand{\registers}{\RU{регистры}\EN{registers}\PTBR{registradores}\ES{registros}\FR{registres}}
\newcommand{\Stack}{\RU{Стек}\EN{Stack}\PTBR{Pilha}\ES{Pila}\FR{Pile}}
\newcommand{\Recursion}{\RU{Рекурсия}\EN{Recursion}\PTBR{Recursividade}\ES{Recursión}\FR{Récursivité}}
\newcommand{\RAM}{\RU{ОЗУ}\EN{RAM}\PTBR{RAM}\ES{RAM}\FR{RAM}}
\newcommand{\ROM}{\RU{ПЗУ}\EN{ROM}\PTBR{ROM}\ES{ROM}\FR{ROM}}
\newcommand{\Pointers}{\RU{Указатели}\EN{Pointers}\PTBR{Ponteiros}\ES{Apuntadores}\FR{Pointeurs}}
\newcommand{\BufferOverflow}{\RU{Переполнение буфера}\EN{Buffer Overflow}\PTBR{Buffer Overflow}\ES{Desbordamiento de buffer}\FR{Débordement de tampon}}
\newcommand{\Conclusion}{\RU{Вывод}\EN{Conclusion}\PTBR{Conclusão}\ES{Conclusión}\FR{Conclusion}}

\newcommand{\Exercise}{\RU{Упражнение}\EN{Exercise}\PTBR{Exercício}\ES{Ejercicio}\FR{Exercice}\xspace}
\newcommand{\Exercises}{\RU{Упражнения}\EN{Exercises}\PTBR{Exercícios}\ES{Ejercicios}\FR{Exercices}\xspace}
\newcommand{\Arrays}{\RU{Массивы}\EN{Arrays}\PTBR{Matriz}\ES{Matriz}\FR{Tableaux}}
\newcommand{\Cpp}{\RU{Си++}\EN{C++}\PTBR{C++}\ES{C++}\FR{C++}\xspace}
\newcommand{\CCpp}{\RU{Си/Си++}\EN{C/C++}\PTBR{C/C++}\ES{C/C++}\FR{C/C++}\xspace}
\newcommand{\NonOptimizing}{\RU{Неоптимизирующий}\EN{Non-optimizing}\PTBR{Sem otimização}\ES{Sin optimización}\FR{Non optimisé}\xspace}
\newcommand{\Optimizing}{\RU{Оптимизирующий}\EN{Optimizing}\PTBR{Com otimização}\ES{Con optimización}\FR{Optimisant}\xspace}
\newcommand{\NonOptimizingKeilVI}{\NonOptimizing Keil 6/2013\xspace}
\newcommand{\OptimizingKeilVI}{\Optimizing Keil 6/2013\xspace}
\newcommand{\NonOptimizingXcodeIV}{\NonOptimizing Xcode 4.6.3 (LLVM)\xspace}
\newcommand{\OptimizingXcodeIV}{\Optimizing Xcode 4.6.3 (LLVM)\xspace}
\newcommand{\ARMMode}{\RU{Режим ARM}\EN{ARM mode}\PTBR{Modo ARM}\ES{Modo ARM}\FR{Mode ARM}\xspace}
\newcommand{\ThumbMode}{\RU{Режим Thumb}\EN{Thumb mode}\PTBR{Modo Thumb}\ES{Modo Thumb}\FR{Mode Thumb}\xspace}
\newcommand{\ThumbTwoMode}{\RU{Режим Thumb-2}\EN{Thumb-2 mode}\PTBR{Modo Thumb-2}\ES{Modo Thumb-2}\FR{Mode Thumb-2}\xspace}
\newcommand{\AndENRU}{\RU{и}\EN{and}\PTBR{e}\ES{y}\FR{et}\xspace}
\newcommand{\OrENRU}{\RU{или}\EN{or}\PTBR{ou}\ES{o}\FR{ou}\xspace}
\newcommand{\InENRU}{\RU{в}\EN{in}\PTBR{em}\ES{en}\FR{dans}\xspace}
\newcommand{\ForENRU}{\RU{для}\EN{for}\PTBR{para}\ES{para}\FR{pour}\xspace}
\newcommand{\LineENRU}{\RU{строка}\EN{line}\PTBR{linha}\ES{línea}\FR{ligne}\xspace}

\newcommand{\DataProcessingInstructionsFootNote}{%
	\RU{Эти инструкции также называются \q{data processing instructions}}%
	\EN{These instructions are also called \q{data processing instructions}}%
	\PTBR{Estas intruções também são chamadas \q{data processing instructions}}%
	\ES{Estas instrucciones también son llamadas \q{instrucciones de procesamiento de datos}}%
	\FR{Ces instructions sont également appelées \q{instructions de traitement de données}}%
}

% for .bib files
\newcommand{\AlsoAvailableAs}{\RU{Также доступно здесь:}\EN{Also available as}\PTBR{Também disponível como}\ES{También disponible como}\FR{Aussi disponible en tant que}\xspace}

% section names
\newcommand{\ShiftsSectionName}{\RU{Сдвиги}\EN{Shifts}\PTBR{Shifts}\ES{Desplazamientos}\FR{Décalages}}
\newcommand{\SignedNumbersSectionName}{\RU{Представление знака в числах}\EN{Signed number representations}\ES{Representaci\'on de n\'umeros con signo}\FR{Représentations en nombre signé}}
\newcommand{\HelloWorldSectionName}{\RU{Hello, world!}\EN{Hello, world!}\ES{!`Hola, mundo!}\FR{Hello, world!}}
\newcommand{\SwitchCaseDefaultSectionName}{switch()/case/default}
\newcommand{\PrintfSeveralArgumentsSectionName}{printf() \RU{с несколькими аргументами}\EN{with several arguments}\PTBR{com vários argumentos}\ES{con varios argumentos}\ITAph{}\FR{avec plusieurs arguments}}
\newcommand{\BitfieldsChapter}{\RU{Работа с отдельными битами}\EN{Manipulating specific bit(s)}\PTBR{Manipulando bit(s) específicos}\ES{Manipulando bit(s) específicos}\FR{En manipulant des bits spécifiques}}
\newcommand{\ArithOptimizations}{%
	\RU{Замена одних арифметических инструкций на другие}%
	\EN{Replacing arithmetic instructions to other ones}%
	\PTBR{Substituição de instruções aritiméticas por outras}%
	\ES{Substituición de instrucciones aritméticas por otras}
	\FR{En remplaçant certains instructions arithmétiques par d'autres}
	}
\newcommand{\FPUChapterName}{\RU{Работа с FPU}\EN{Floating-point unit}\PTBR{Unidade de Ponto flutuante}\ES{Unidad de punto flotante}\FR{Unité à virgule flottante}}
\newcommand{\SimpleStringsProcessings}{\RU{Простая работа с Си-строками}\EN{Simple C-strings processing}\PTBR{Processamento de strings C simples}\ES{Procesamiento simple de cadenas en C}\FR{Traitement de strings C simples}}
\newcommand{\DivisionByNineSectionName}{\RU{Деление используя умножение}\EN{Division using multiplication}\PTBR{Divisão por 9}\ES{División entre 9}\FR{Division par 9}}
\newcommand{\Answer}{\RU{Ответ}\EN{Answer}\PTBR{Responda}\ES{Respuesta}\FR{Réponse}}
\newcommand{\WhatThisCodeDoes}{\RU{Что делает этот код}\EN{What does this code do}\PTBR{O que este código faz}\ES{?`Qu\'e hace este código}?\FR{Que fait ce code ?}}
\newcommand{\WorkingWithFloatAsWithStructSubSubSectionName}{%
\RU{Работа с типом float как со структурой}\EN{Handling float data type as a structure}\PTBR{Trabalhando com o tipo float como uma estrutura}\ES{Trabajando con el tipo float como una estructura}\FR{Travailler avec le type float comme une structure}}

\newcommand{\MinesweeperWinXPExampleChapterName}{\RU{Сапёр}\EN{Minesweeper}\PTBR{Campo minado}\ES{Buscaminas}\FR{Démineur} (Windows XP)}

\newcommand{\StructurePackingSectionName}{\RU{Упаковка полей в структуре}\EN{Fields packing in structure}\PTBR{Organização de campos na estrutura}\ES{Organización de campos en la estructura}\FR{Organisation des champs dans la structure}}
\newcommand{\StructuresChapterName}{\RU{Структуры}\EN{Structures}\PTBR{Estruturas}\ES{Estructuras}\FR{Structures}}
\newcommand{\PICcode}{\RU{адресно-независимый код}\EN{position-independent code}\PTBR{código independente de posição}\ES{código independiente de la posición}\FR{code indépendant de la position}}
\newcommand{\CapitalPICcode}{\RU{Адресно-независимый код}\EN{Position-independent code}\PTBR{Código independente de posição}\ES{Código independiente de lá posición}\FR{Code indépendant de la position}}
\newcommand{\Loops}{\RU{Циклы}\EN{Loops}\PTBR{Laços}\ES{Bucles}\FR{Boucles}}

% C
\newcommand{\PostIncrement}{\RU{Пост-инкремент}\EN{Post-increment}\PTBR{Pós-incremento}\ES{Post-incremento}\FR{Post-incrémentation}}
\newcommand{\PostDecrement}{\RU{Пост-декремент}\EN{Post-decrement}\PTBR{Pós-decremento}\ES{Post-decremento}\FR{Post-décrémentation}}
\newcommand{\PreIncrement}{\RU{Пре-инкремент}\EN{Pre-increment}\PTBR{Pré-incremento}\ES{Pre-incremento}\FR{Pré-incrémentation}}
\newcommand{\PreDecrement}{\RU{Пре-декремент}\EN{Pre-decrement}\PTBR{Pré-decremento}\ES{Pre-decremento}\FR{Post-incrémentation}}

% MIPS
\newcommand{\GlobalPointer}{\RU{Глобальный указатель}\EN{Global Pointer}\PTBR{Ponteiro Global}\ES{Apuntador Global}\FR{Pointeur Global}}

% other
\ifdefined\RUSSIAN
\newcommand{\HERMIT}{Андрей \q{herm1t} Баранович}
\else
\newcommand{\HERMIT}{Andrey \q{herm1t} Baranovich}
\fi

\newcommand{\garbage}{\RU{мусор}\EN{garbage}\PTBR{Lixo}\ES{Basura}\FR{déchets}}
\newcommand{\IntelSyntax}{\RU{Синтаксис Intel}\EN{Intel syntax}\PTBR{Sintaxe Intel}\ES{Sintaxis Intel}\FR{Syntaxe Intel}}
\newcommand{\ATTSyntax}{\RU{Синтаксис AT\&T}\EN{AT\&T syntax}\PTBR{Sintaxe AT\&T}\ES{Sintaxis AT\&T}\FR{Syntaxe AT\&T}}
\newcommand{\randomNoise}{\RU{случайный шум}\EN{random noise}\PTBR{Ruído aleatório}\ES{Ruido aleatorio}\FR{bruit aléatoire}}
\newcommand{\Example}{\RU{Пример}\EN{Example}\PTBR{Exemplo}\ES{Ejemplo}\FR{Exemple}}
\newcommand{\argument}{\RU{аргумент}\EN{argument}\PTBR{argumento}\ES{argumento}\FR{argument}}
\newcommand{\MarkedInIDAAs}{\RU{маркируется в \IDA как}\EN{marked in \IDA as}\PTBR{Marcado no \IDA como}\ES{Marcado en \IDA como}\FR{marqué dans \IDA comme}}
\newcommand{\stepover}{\RU{сделать шаг, не входя в функцию}\EN{step over}\PTBR{passar por cima}\ES{pasar por encima}\FR{enjamber}}
\newcommand{\ShortHotKeyCheatsheet}{\RU{Краткий справочник горячих клавиш}\EN{Hot-keys cheatsheet}\PTBR{Cheatsheet de teclas de atalho}\ES{Cheatsheet de teclas de acceso rápido}\FR{Anti-sèche des touches de raccourci}}

\newcommand{\assemblyOutput}{\RU{вывод на ассемблере}\EN{assembly output}\PTBR{saída do assembly}\ES{salida del ensamblador}\FR{résultat en sortie de l'assembleur}}

% was in common_URLS.tex:
\newcommand{\URLWPDA}{%
	\RU{\href{http://go.yurichev.com/17012}{Wikipedia: Выравнивание данных}}%
	\EN{\href{http://go.yurichev.com/17013}{Wikipedia: Data structure alignment}}%
	\ES{\href{URL}{\ESph{}}}%
	\PL{\href{URL}{\PLph{}}}%
	\PTBR{\href{URL}{\PTBRph{}}}%
	\IT{\href{URL}{\ITAph{}}}%
	\DE{\href{URL}{\DEph{}}}%
	\FR{\href{URL}{\FRph{}}}% Link : https://fr.wikipedia.org/wiki/Alignement_en_m%C3%A9moire {Wikipedia: Alignement en mémoire}
}

\newcommand{\OracleTablesName}{oracle tables\xspace}
\newcommand{\oracletables}{\OracleTablesName\footnote{\href{http://go.yurichev.com/17014}{yurichev.com}}\xspace}

\newcommand{\WPMAO}{%
	\RU{\href{http://go.yurichev.com/17015}{Wikipedia: Умножение-сложение}}%
	\EN{\href{http://go.yurichev.com/17016}{Wikipedia: Multiply–accumulate operation}}%
	\ES{\href{URL}{\ESph{}}}%
	\DE{\href{URL}{\DEph{}}}%
	\PL{\href{URL}{\PLph{}}}%
	\PTBR{\href{URL}{\PTBRph{}}}%
	\ITA{\href{URL}{\ITAph{}}}%
	\FR{\href{URL}{\FRph{}}}% Link : https://fr.wikipedia.org/wiki/Multiply-accumulate {Wikipedia: Multiply-accumulate}
}

\newcommand{\BGREPURL}{\href{http://go.yurichev.com/17017}{GitHub}}
\newcommand{\FNMSDNROTxURL}{\footnote{\href{http://go.yurichev.com/17018}{MSDN}}}

\newcommand{\YurichevIDAIDCScripts}{http://go.yurichev.com/17019}

% ML prefix is for multi-lingual words and sentences:
\newcommand{\MLHeap}{\EN{Heap}\RU{Куча}\PTBR{Heap}\FR{Heap}}
\newcommand{\MLStack}{\EN{Stack}\RU{Стэк}\PTBR{Pilha}\FR{Pile}}
\newcommand{\MLStartOfHeap}{\RU{Начало кучи}\EN{Start of heap}\PTBR{começo da heap}\FR{Début du heap}}
\newcommand{\MLStartOfStack}{\RU{Вершина стека}\EN{Start of stack}\PTBR{começo da pilha}\FR{Début de la pile}}
\newcommand{\MLinputA}{\RU{вход А}\EN{input A}\FR{entrée A}}
\newcommand{\MLinputB}{\RU{вход Б}\EN{input B}\FR{entrée B}}
\newcommand{\MLoutput}{\RU{выход}\EN{output}\FR{sortie}}

% sources: books, etc
\newcommand{\TAOCPvolI}{Donald E. Knuth, \IT{The Art of Computer Programming}, Volume 1, 3rd ed., (1997)}
\newcommand{\TAOCPvolII}{Donald E. Knuth, \IT{The Art of Computer Programming}, Volume 2, 3rd ed., (1997)}
\newcommand{\Russinovich}{Mark Russinovich, \IT{Microsoft Windows Internals}}
\newcommand{\Schneier}{Bruce Schneier, \IT{Applied Cryptography}, (John Wiley \& Sons, 1994)}
\newcommand{\AgnerFog}{Agner Fog, \IT{The microarchitecture of Intel, AMD and VIA CPUs}, (2016)}
\newcommand{\AgnerFogCPP}{Agner Fog, \IT{Optimizing software in C++} (2015)}
\newcommand{\AgnerFogCC}{Agner Fog, \IT{Calling conventions} (2015)}
\newcommand{\JavaBook}{[Tim Lindholm, Frank Yellin, Gilad Bracha, Alex Buckley, \IT{The Java(R) Virtual Machine Specification / Java SE 7 Edition}]
\footnote{\AlsoAvailableAs \url{https://docs.oracle.com/javase/specs/jvms/se7/jvms7.pdf}; \url{http://docs.oracle.com/javase/specs/jvms/se7/html/}}}

\newcommand{\ARMPCS}{\InSqBrackets{\IT{Procedure Call Standard for the ARM 64-bit Architecture (AArch64)}, (2013)}\footnote{\AlsoAvailableAs \url{http://go.yurichev.com/17287}}}

\newcommand{\IgorSkochinsky}{[Igor Skochinsky, \IT{Compiler Internals: Exceptions and RTTI}, (2012)] \footnote{\AlsoAvailableAs \url{http://go.yurichev.com/17294}}}

\newcommand{\PietrekSEH}{[Matt Pietrek, \IT{A Crash Course on the Depths of Win32\texttrademark{} Structured Exception Handling}, (1997)]\footnote{\AlsoAvailableAs \url{http://go.yurichev.com/17293}}}

\newcommand{\PietrekPE}{Matt Pietrek, \IT{An In-Depth Look into the Win32 Portable Executable File Format}, (2002)]}

\newcommand{\PietrekPEURL}{\PietrekPE\footnote{\AlsoAvailableAs \url{http://go.yurichev.com/17318}}}

\newcommand{\RitchieDevC}{[Dennis M. Ritchie, \IT{The development of the C language}, (1993)]\footnote{\AlsoAvailableAs \url{http://go.yurichev.com/17264}}}

\newcommand{\RitchieThompsonUNIX}{[D. M. Ritchie and K. Thompson, \IT{The UNIX Time Sharing System}, (1974)]\footnote{\AlsoAvailableAs \url{http://go.yurichev.com/17270}}}

\newcommand{\DrepperTLS}{[Ulrich Drepper, \IT{ELF Handling For Thread-Local Storage}, (2013)]\footnote{\AlsoAvailableAs \url{http://go.yurichev.com/17272}}}

\newcommand{\DrepperMemory}{[Ulrich Drepper, \IT{What Every Programmer Should Know About Memory}, (2007)]\footnote{\AlsoAvailableAs \url{http://go.yurichev.com/17341}}}

\newcommand{\AlephOne}{[Aleph One, \IT{Smashing The Stack For Fun And Profit}, (1996)]\footnote{\AlsoAvailableAs \url{http://go.yurichev.com/17266}}}

\ifdefined\RUSSIAN
\newcommand{\KRBook}{[Брайан Керниган, Деннис Ритчи, \IT{Язык программирования Си}, (1988)]}
\newcommand{\CppOneOneStd}{Стандарт Си++11}
\newcommand{\CNotes}{Денис Юричев, \IT{Заметки о языке программирования Си/Си++}}
\else
\newcommand{\KRBook}{[Brian W. Kernighan, Dennis M. Ritchie, \IT{The C Programming Language}, (1988)]}
\newcommand{\CppOneOneStd}{C++11 standard}
\newcommand{\CNotes}{Dennis Yurichev, \IT{C/C++ programming language notes}}
\fi

\newcommand{\ARMSixFourRef}{\IT{ARM Architecture Reference Manual, ARMv8, for ARMv8-A architecture profile}, (2013)}
\newcommand{\ARMSixFourRefURL}{\InSqBrackets{\ARMSixFourRef}\footnote{\AlsoAvailableAs \url{http://yurichev.com/mirrors/ARMv8-A_Architecture_Reference_Manual_(Issue_A.a).pdf}}}

\newcommand{\SysVABI}{[Michael Matz, Jan Hubicka, Andreas Jaeger, Mark Mitchell, \IT{System V Application Binary Interface. AMD64 Architecture Processor Supplement}, (2013)]
\footnote{\AlsoAvailableAs \url{https://software.intel.com/sites/default/files/article/402129/mpx-linux64-abi.pdf}}}

\newcommand{\IOSABI}{\InSqBrackets{\IT{iOS ABI Function Call Guide}, (2010)}\footnote{\AlsoAvailableAs \url{http://go.yurichev.com/17276}}}

\newcommand{\CNineNineStd}{\IT{ISO/IEC 9899:TC3 (C C99 standard)}, (2007)}

\newcommand{\TCPPPL}{Bjarne Stroustrup, \IT{The C++ Programming Language, 4th Edition}, (2013)}

\newcommand{\AMDOptimization}{\InSqBrackets{\IT{Software Optimization Guide for AMD Family 16h Processors}, (2013)}}
% note = "\AlsoAvailableAs \url{http://go.yurichev.com/17285}",

\newcommand{\IntelOptimization}{\InSqBrackets{\IT{Intel® 64 and IA-32 Architectures Optimization Reference Manual}, (2014)}}
% note = "\AlsoAvailableAs \url{http://go.yurichev.com/17342}",

\newcommand{\TAOUP}{Eric S. Raymond, \IT{The Art of UNIX Programming}, (2003)}
% note = "\AlsoAvailableAs \url{http://go.yurichev.com/17277}",

\newcommand{\HenryWarren}{Henry S. Warren, \IT{Hacker's Delight}, (2002)}

\newcommand{\ParashiftCPPFAQ}{Marshall Cline, \IT{C++ FAQ}}
% note = "\AlsoAvailableAs \url{http://go.yurichev.com/17291}",

\newcommand{\ARMCookBook}{Advanced RISC Machines Ltd, \IT{The ARM Cookbook}, (1994)}
% note = "\AlsoAvailableAs \url{http://go.yurichev.com/17273}",

\newcommand{\ARMSevenRef}{\IT{ARM(R) Architecture Reference Manual, ARMv7-A and ARMv7-R edition}, (2012)}

\newcommand{\MAbrash}{Michael Abrash, \IT{Graphics Programming Black Book}, 1997}

\newcommand{\radare}{rada.re}

\newcommand{\RedditHiringThread}{\href{https://www.reddit.com/r/ReverseEngineering/comments/4sbd11/rreverseengineerings_2016_triannual_hiring_thread/}{2016}}
\newcommand{\NetsecHiringThread}{\href{https://www.reddit.com/r/netsec/comments/552rz1/rnetsecs_q4_2016_information_security_hiring/}{2016}}

