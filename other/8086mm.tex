\section{\IFRU{Модель памяти в 8086}{8086 memory model}}
\index{8086!\IFRU{Модель памяти}{Memory model}}
\label{8086_memory_model}

\IFRU{Разбирая 16-битные программы для MS-DOS или Win16}{Dealing with 16-bit programs for MS-DOS or Win16}
(\ref{dongle_16bit_dos} \OrENRU \ref{win16_near_far_pointers}),
\IFRU{мы можем увидеть что указатель состоит из двух 16-битных значений.
Что это означает? О да, еще один дивный артефакт MS-DOS и 8086}
{we can see that pointer consisting of two 16-bit values.
What it means? Oh yes, that is another MS-DOS and 8086 weird artefact}.

8086/8088 \IFRU{был 16-битным процессором, но мог адресовать 20-битное адресное
пространство (таким образом мог адресовать 1MB внешней памяти)}
{was a 16-bit CPU, but was able to address 20-bit address RAM 
(thus resulting 1MB external memory)}.
\IFRU{Внешняя адресное пространство было разделено между \ac{RAM} (максимум 640KB),
\ac{ROM}, окна для видеопамяти, EMS-карт, и т.д}
{External memory address space was divided between \ac{RAM} (640KB max),
\ac{ROM}, windows for video memory, EMS cards, etc}.

\IFRU{Припомним также что 8086/8088 был на самом деле наследником 8-битного процессора 8080}
{Let's also recall that 8086/8088 was in fact inheritor of 8-bit 8080 CPU}.
\IFRU{Процессор 8080 имел 16-битное адресное пространство, т.е., мог адресовать только}
{The 8080 has 16-bit memory spaces, i.e., it was able to address only} 64KB.
\IFRU{И возможно в расчете на портирование старого ПО\footnote{Хотя я и не уверен на 100\%},
8086 может поддерживать 64-килобайтные
окна, одновременно много таких, расположенных внутри одномегабайтного адресного пространства}
{And probably of old software porting reason\footnote{I'm not 100\% sure here},
8086 can support 64KB windows, many of them placed simultaneously
within 1MB address space}.
\IFRU{Это, в каком-то смысле, игрушечная виртуализация}
{This is some kind of toy-level virtualization}.
\IFRU{Все регистры 8086 16-битные, так что, чтобы адресовать больше, специальные сегментные
регистры (CS, DS, ES, SS) были введены}
{All 8086 registers are 16-bit, so to address more, a special segment registers (CS, DS, ES, SS) were
introduced}.
\IFRU{Каждый 20-битный указатель вычисляется используя значения из пары состоящей из сегментного регистра
и адресного регистра (например DS:BX) вот так}
{Each 20-bit pointer is calculated using values from a segment register and 
an address register pair (e.g. DS:BX) as follows:}

\begin{center}
$real\_address = (segment\_register \ll 4) + address\_register$
\end{center}

\IFRU{Например, окно памяти для графики (\ac{EGA}, \ac{VGA}) на старых IBM PC-совместимых компьютерах
имело размер 64KB.}
{For example, graphics (\ac{EGA}, \ac{VGA}) video \ac{RAM} window on old IBM PC-compatibles 
has size of 64KB.}
\IFRU{Для доступа к нему, значение 0xA000 должно быть записано в один из сегментных регистров,
например, в DS.}
{For accessing it, a 0xA000 value should be stored in one of segment registers, e.g. into DS.}
\IFRU{Тогда DS:0 будет адресовать самый первый байт видеопамяти, а DS:0xFFFF ~--- самый последний байт.}
{Then DS:0 will address the very first byte of video \ac{RAM} and DS:0xFFFF is the very last byte of RAM.}
\IFRU{А реальный адрес на 20-битной адресной шине, на самом деле будет от 0xA0000 до 0xAFFFF.}
{The real address on 20-bit address bus, however, will range from 0xA0000 to 0xAFFFF.}

\IFRU{Программа может содержать жесткопривязанные адреса вроде 0x1234, но \ac{OS} может иметь необходимость
загрузить программу по другим адресам, так что она пересчитает значения для сегментных регистров так,
что программа будет нормально работать не обращая внимания на то,
в каком месте памяти она была расположена.}
{The program may contain hardcoded addresses like 0x1234, but \ac{OS} may need to load program on arbitrary
addresses, so it recalculates segment register values in such a way, so the program will not care about
where in the RAM it is placed.}

\IFRU{Так что, любой указатель в окружении старой MS-DOS на самом деле состоял из адреса сегмента
и адреса внутри сегмента, т.е., из двух 16-битных значений. 20-битного значения было бы достаточно для
этого, хотя, тогда пришлось бы вычислять адреса слишком часто: так что передача большего количества
информации в стеке ~--- это более хороший баланс между экономией места и удобством}
{So, any pointer it old MS-DOS environment was in fact consisted of segment address and the address inside
segment, i.e., two 16-bit values. 20-bit was enough for that, though, but one will need to recalculate
the addresses very often: passing more information on stack is seems better space/convenience balance}.

\IFRU{Кстати, из-за всего этого, не было возможным выделить блок памяти больше чем}
{By the way, because of all this, it was not possible to allocate the memory block larger than} 64KB.

\index{80286}
\index{80386}
\IFRU{В 80286 сегментные регистры получили новую роль селекторов, имеющих немного другую ф-цию.}
{Segment registers were reused at 80286 as selectors, serving different function.}

\index{MS-DOS!DOS extenders}
\IFRU{Когда появился процессор 80386 и компьютеры с большей памятью,
MS-DOS была всё еще популярна, так что появились DOS-экстендеры: на самом деле это уже был шаг
к ``серьезным'' \ac{OS}, они переключали \ac{CPU} в защищенный режим и предлагали куда лучшее \ac{API} для
программ, которые всё еще предполагалось запускать в MS-DOS.}
{When 80386 CPU and computers with bigger \ac{RAM} were introduced,
MS-DOS was still popular, so the DOS extenders are emerged: these were in fact
a step toward ``serious'' \ac{OS},
switching CPU into protected mode and providing much better memory \ac{API}s for the programs 
which still needs to be runned from MS-DOS.}
\IFRU{Широко известные примеры это DOS/4GW (игра DOOM была скомпилирована под него), Phar Lap, PMODE}
{Widely popular examples include DOS/4GW (DOOM video game was compiled for it), Phar Lap, PMODE.} \\
\\
\index{Windows 3.x}
\index{Win32}
\IFRU{Кстати, точно такой же способ адресации памяти был и в 16-битной линейке Windows 3.x, перед Win32}
{By the way, the same was of addressing memory was in 16-bit line of Windows 3.x, before Win32}.

