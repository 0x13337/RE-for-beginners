\chapter{\RU{Модель памяти в 8086}\EN{8086 memory model}}
\index{Intel!8086!\RU{Модель памяти}\EN{Memory model}}
\index{MS-DOS}
\label{8086_memory_model}

\RU{Разбирая 16-битные программы для MS-DOS или Win16}\EN{When dealing with 16-bit programs for MS-DOS or Win16}
(\myref{dongle_16bit_dos} \OrENRU \myref{win16_near_far_pointers}),
\RU{мы можем увидеть, что указатель состоит из двух 16-битных значений.
Что это означает? О да, еще один дивный артефакт MS-DOS и 8086}
\EN{we can see that the pointers consist of two 16-bit values.
What do they mean? Oh yes, that is another weird MS-DOS and 8086 artefact}.

8086/8088 \RU{был 16-битным процессором, но мог адресовать 20-битное адресное
пространство (таким образом мог адресовать 1MB внешней памяти)}
\EN{was a 16-bit CPU, but was able to address 20-bit address in RAM 
(thus being able to access 1MB of external memory)}.
\RU{Внешняя адресное пространство было разделено между \ac{RAM} (максимум 640KB),
\ac{ROM}, окна для видеопамяти, EMS-карт, \etc{}.}
\EN{The external memory address space was divided between \ac{RAM} (640KB max),
\ac{ROM}, windows for video memory, EMS cards, \etc{}.}

\RU{Припомним также что 8086/8088 был на самом деле наследником 8-битного процессора 8080}
\EN{Let's also recall that 8086/8088 was in fact an inheritor of the 8-bit 8080 CPU}.
\RU{Процессор 8080 имел 16-битное адресное пространство, т.е. мог адресовать только}
\EN{The 8080 has a 16-bit memory space, i.e., it was able to address only} 64KB.
\RU{И возможно в расчете на портирование старого ПО\footnote{Хотя я и не уверен на 100\%},
8086 может поддерживать 64-килобайтные
окна, одновременно много таких, расположенных внутри одномегабайтного адресного пространства}
\EN{And probably because of old software porting reason\footnote{I'm not 100\% sure here},
8086 can support many 64KB windows simultaneously, placed
within the 1MB address space}.
\RU{Это, в каком-то смысле, игрушечная виртуализация}
\EN{This is some kind of a toy-level virtualization}.
\index{x86!\Registers!CS}
\index{x86!\Registers!DS}
\index{x86!\Registers!ES}
\index{x86!\Registers!SS}
\RU{Все регистры 8086 16-битные, так что, чтобы адресовать больше, специальные сегментные
регистры (CS, DS, ES, SS) были введены}
\EN{All 8086 registers are 16-bit, so to address more, special segment registers (CS, DS, ES, SS) were
introduced}.
\RU{Каждый 20-битный указатель вычисляется, используя значения из пары состоящей из сегментного регистра
и адресного регистра (например DS:BX) вот так}
\EN{Each 20-bit pointer is calculated using the values from a segment register and 
an address register pair (e.g. DS:BX) as follows:}

\begin{center}
$real\_address = (segment\_register \ll 4) + address\_register$
\end{center}

\RU{Например, окно памяти для графики (\ac{EGA}, \ac{VGA}) на старых IBM PC-совместимых компьютерах
имело размер 64KB.}
\EN{For example, the graphics (\ac{EGA}, \ac{VGA}) video \ac{RAM} window on old IBM PC-compatibles 
has a size of 64KB.}
\RU{Для доступа к нему, значение 0xA000 должно быть записано в один из сегментных регистров,
например, в DS.}
\EN{To access it, a value of 0xA000 has to be stored in one of the segment registers, e.g. into DS.}
\RU{Тогда DS:0 будет адресовать самый первый байт видеопамяти, а DS:0xFFFF ~--- самый последний байт.}
\EN{Then DS:0 will address the first byte of video \ac{RAM} and DS:0xFFFF ~--- the last byte of RAM.}
\RU{А реальный адрес на 20-битной адресной шине, на самом деле будет от 0xA0000 до 0xAFFFF.}
\EN{The real address on the 20-bit address bus, however, will range from 0xA0000 to 0xAFFFF.}

\RU{Программа может содержать жесткопривязанные адреса вроде 0x1234, но \ac{OS} может иметь необходимость
загрузить программу по другим адресам, так что она пересчитает значения для сегментных регистров так,
что программа будет нормально работать, не обращая внимания на то,
в каком месте памяти она была расположена.}
\EN{The program may contain hard-coded addresses like 0x1234, but the \ac{OS} may need to load the program at arbitrary
addresses, so it recalculates the segment register values in a way that the program does not have to care 
where it's placed in the RAM.}

\RU{Так что, любой указатель в окружении старой MS-DOS на самом деле состоял из адреса сегмента
и адреса внутри сегмента, т.е. из двух 16-битных значений. 20-битного значения было бы достаточно для
этого, хотя, тогда пришлось бы вычислять адреса слишком часто: так что передача большего количества
информации в стеке ~--- это более хороший баланс между экономией места и удобством}
\EN{So, any pointer in the old MS-DOS environment in fact consisted of the segment address and the address inside
segment, i.e., two 16-bit values. 20-bit was enough for that, though, but we needed to recalculate
the addresses very often: passing more information on the stack seemed a better space/convenience balance}.

\RU{Кстати, из-за всего этого, не было возможным выделить блок памяти больше чем}
\EN{By the way, because of all this it was not possible to allocate a memory block larger than} 64KB.

\index{Intel!80286}
\index{Intel!80386}
\RU{В 80286 сегментные регистры получили новую роль селекторов, имеющих немного другую функцию.}
\EN{The segment registers were reused at 80286 as selectors, serving a different function.}

\index{MS-DOS!DOS extenders}
\RU{Когда появился процессор 80386 и компьютеры с большей памятью,
MS-DOS была всё еще популярна, так что появились DOS-экстендеры: на самом деле это уже был шаг
к ``серьезным'' \ac{OS}, они переключали \ac{CPU} в защищенный режим и предлагали куда лучшее \ac{API} для
программ, которые всё еще предполагалось запускать в MS-DOS.}
\EN{When the 80386 CPU and computers with bigger \ac{RAM} were introduced,
MS-DOS was still popular, so the DOS extenders emerged: these were in fact
a step toward a ``serious'' \ac{OS},
switching the CPU in protected mode and providing much better memory \ac{API}s for the programs 
which still needed to run under MS-DOS.}
\RU{Широко известные примеры это DOS/4GW (игра DOOM была скомпилирована под него), Phar Lap, PMODE}
\EN{Widely popular examples include DOS/4GW (the DOOM video game was compiled for it), Phar Lap, PMODE.} \\
\\
\index{Windows!Windows 3.x}
\index{Windows!Win32}
\RU{Кстати, точно такой же способ адресации памяти был и в 16-битной линейке Windows 3.x, перед Win32}
\EN{By the way, the same way of addressing memory was used in the 16-bit line of Windows 3.x, before Win32}.

