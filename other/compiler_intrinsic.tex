\chapter{Compiler intrinsic}
\index{Compiler intrinsic}
\label{sec:compiler_intrinsic}

\index{x86!\Instructions!ROL}
\index{x86!\Instructions!ROR}
\RU{Специфичная для компилятора ф-ция не являющаяся обычной библиотечной ф-цией.
Компилятор вместо её вызова генерирует определенный машинный код.
Нередко, это псевдофункции для определенной инструкции \ac{CPU}. \\
\\
Например, в языках \CCpp нет операции циклического сдвига, а во многих \ac{CPU} она есть.
Чтобы программисту были доступны эти инструкции, в MSVC есть псевдофункции 
\IT{\_rotl()} \AndENRU \IT{\_rotr()}\FNMSDNROTxURL{},
которые компилятором напрямую транслируются в x86-инструкции \TT{ROL}/\TT{ROR}. \\
\\
Еще один пример это ф-ции позволяющие генерировать SSE-инструкции прямо в коде.}
\EN{A function specific to a compiler which is not an usual library function.
The compiler generates a specific machine code instead of a call to it.
It is often a pseudofunction for specific \ac{CPU} instruction. \\
\\
For example, there are no cyclic shift operations in \CCpp languages, but they are present in most \ac{CPU}s.
For programmer's convenience, at least MSVC has pseudofunctions
\IT{\_rotl()} \AndENRU \IT{\_rotr()}\FNMSDNROTxURL{}
which are translated by the compiler directly to the ROL/ROR x86 instructions. \\
\\
Another example are functions to generate SSE-instructions right in the code.}

\RU{Полный список intrinsics от MSVC}\EN{Full list of MSVC intrinsics}:
\href{http://go.yurichev.com/17254}{MSDN}.

