\section{Endianness\RU{ (порядок байт)}}
\label{sec:endianness}

\IFRU{Endianness (порядок байт) это способ представления чисел в памяти}
{Endianness is a way of representing values in memory}.

\subsection{Big-endian\RU{ (от старшего к младшему)}}

\IFRU{Число}{A} \TT{0x12345678} \IFRU{будет представлено в памяти так}{value will be represented in memory as}:

\begin{center}
\begin{tabular}{ | l | l | }
\hline
\cellcolor{blue!25} \IFRU{адрес в памяти}{address in memory} & \cellcolor{blue!25} \IFRU{значение байта}{byte value} \\
\hline
+0 & 0x12 \\
\hline
+1 & 0x34 \\
\hline
+2 & 0x56 \\
\hline
+3 & 0x78 \\
\hline
\end{tabular}
\end{center}

\IFRU{CPU с таким порядком включают в себя}{Big-endian CPUs are including} Motorola 68k, IBM POWER.

\subsection{Little-endian\RU{ (от младшего к старшему)}}

\IFRU{Число}{A} \TT{0x12345678} \IFRU{будет представлено в памяти так}{value will be represented in memory as}:

\begin{center}
\begin{tabular}{ | l | l | }
\hline
\cellcolor{blue!25} \IFRU{адрес в памяти}{address in memory} & \cellcolor{blue!25} \IFRU{значение байта}{byte value} \\
\hline
+0 & 0x78 \\
\hline
+1 & 0x56 \\
\hline
+2 & 0x34 \\
\hline
+3 & 0x12 \\
\hline
\end{tabular}
\end{center}

\IFRU{CPU с таким порядком байт включают в себя}{Little-endian CPUs are including} Intel x86.

\subsection{Bi-endian\RU{ (переключаемый порядок)}}

\IFRU{CPU поддерживающие оба порядка, и его можно переключать, включают в себя}
{CPUs which may switch between endianness are ARM, PowerPC, SPARC, MIPS, \ac{IA64}}, \IFRU{итд}{etc}.

\subsection{\IFRU{Конвертирование}{Converting data}}

\index{TCP/IP}
\IFRU{Сетевые пакеты TCP/IP используют соглашение big-endian, вот почему программа работающая на little-endian архитектуре
должна конвертировать значения используя ф-ции}
{TCP/IP network data packets are used big-endian conventions, so that is why a program working on little-endian architecture
should convert values using} \TT{htonl()} \AndENRU \TT{htons()}\EN{ functions}.

\IFRU{Порядок байт big-endian в среде TCP/IP также называется}
{Big-endian convention in the TCP/IP environment is also called} ``network byte order'',
\IFRU{а}{while} little-endian\EMDASH{}``host byte order''.

\index{x86!\Instructions!BSWAP}
\RU{Инструкция }\TT{BSWAP} \IFRU{также может использоваться для конвертирования}
{instruction is also can be used for the conversion}.

