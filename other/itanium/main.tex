\chapter{Itanium}
\label{itanium}
\myindex{Itanium}
\RU{Еще одна очень интересная архитектура (хотя и почти провальная) это Intel Itanium (\ac{IA64})}
\EN{Although almost failed, Intel Itanium (\ac{IA64}) is a very interesting architecture}.
\RU{Другие \ac{OOE}-процессоры сами решают, как переставлять инструкции и исполнять их параллельно,
\ac{EPIC} это была попытка сдвинуть эти решения на компилятор: дать ему возможность самому 
группировать инструкции во время компиляции.}
\EN{While \ac{OOE} CPUs decides how to rearrange their instructions and execute them in parallel,
\ac{EPIC} was an attempt to shift these decisions to the compiler:
to let it group the instructions at the compile stage.}

\RU{Это вылилось в очень сложные компиляторы}
\EN{This resulted in notoriously complex compilers.}

\RU{Вот один пример \ac{IA64}-кода: простой криптоалгоритм из ядра Linux}
\EN{Here is one sample of \ac{IA64} code: simple cryptographic algorithm from the Linux kernel}:

\lstinputlisting[caption=Linux kernel 3.2.0.4]{other/itanium/tea_from_linux.c}

\RU{И вот как он был скомпилирован}\EN{Here is how it was compiled}:

\lstinputlisting[caption=Linux Kernel 3.2.0.4 \RU{для}\EN{for} Itanium 2 (McKinley)]{other/itanium/ia64_linux_3.2.0.4_mckinley.lst}

\RU{Прежде всего, все инструкции \ac{IA64} сгруппированы в пачки (bundle) из трех инструкций}
\EN{First of all, all \ac{IA64} instructions are grouped into 3-instruction bundles}.
\RU{Каждая пачка имеет размер 16 байт (128 бит) и состоит из template-кода (5 бит) и трех инструкций (41 бит на каждую).}
\EN{Each bundle has a size of 16 bytes (128 bits) and consists of template code (5 bits) + 3 instructions (41 bits for each).}
\RU{\IDA показывает пачки как 6+6+4 байт --- вы можете легко заметить эту повторяющуюся структуру}
\EN{\IDA shows the bundles as 6+6+4 bytes~---you can easily spot the pattern}.

\RU{Все 3 инструкции каждой пачки обычно исполняются одновременно, если только у какой-то инструкции
нет \q{стоп-бита}}
\EN{All 3 instructions from each bundle usually executes simultaneously, unless one of instructions has a
\q{stop bit}}.

\RU{Вероятно, инженеры Intel и HP собрали статистику наиболее встречающихся шаблонных сочетаний
инструкций и решили ввести типы пачек (\ac{AKA} \q{templates}): код пачки определяет типы инструкций
в пачке}
\EN{Supposedly, Intel and HP engineers gathered statistics on most frequent instruction patterns and decided to bring
bundle types (\ac{AKA} \q{templates}): a bundle code defines the instruction types in the bundle}.
\RU{Их всего 12}\EN{There are 12 of them}.
\RU{Например, нулевой тип это \TT{MII}, что означает: первая инструкция это Memory (загрузка
или запись в память), вторая и третья это I (инструкция, работающая с целочисленными значениями).}
\EN{For example, the zeroth bundle type is \TT{MII}, which implies 
the first instruction is Memory (load or store), the second and third ones are I (integer instructions).}
\RU{Еще один пример, тип 0x1d: \TT{MFB}: первая инструкция это Memory (загрузка или запись
в память), вторая это Float (инструкция, работающая с \ac{FPU}), третья это Branch (инструкция
перехода).}
\EN{Another example is the bundle of type 0x1d: \TT{MFB}:
the first instruction is Memory (load or store), the second one is Float 
(\ac{FPU} instruction), and the third is Branch (branch instruction).}

\RU{Если компилятор не может подобрать подходящую инструкцию в соответствующее место пачки,
он может вставить \ac{NOP}:
вы можете здесь увидеть инструкции \TT{nop.i} (\ac{NOP} на том месте где должна была бы находиться
целочисленная инструкция) или \TT{nop.m} (инструкция обращения к памяти должна была находиться
здесь).}
\EN{If the compiler cannot pick a suitable instruction for the relevant bundle slot, it may insert a \ac{NOP}:
you can see here the
\TT{nop.i} instructions (\ac{NOP} at the place where the integer instruction might be) or \TT{nop.m} 
(a memory instruction might be at this slot).}
\RU{Если вручную писать на ассемблере, \ac{NOP}-ы могут вставляться автоматически}
\EN{\ac{NOP}s are inserted automatically when one uses assembly language manually}.

\RU{И это еще не все. Пачки тоже могут быть объединены в группы}
\EN{And that is not all. Bundles are also grouped}.
\RU{Каждая пачка может иметь \q{стоп-бит}, так что все следующие друг за другом пачки вплоть до той,
что имеет стоп-бит, могут быть исполнены одновременно}
\EN{Each bundle may have a \q{stop bit},
so all the consecutive bundles with a terminating bundle which has the \q{stop bit} 
can be executed simultaneously}.
\RU{На практике, Itanium 2 может исполнять 2 пачки одновременно, таким образом, исполнять
6 инструкций одновременно}
\EN{In practice, Itanium 2 can execute 2 bundles at once, resulting in the execution of 6 instructions at once}.

\RU{Так что все инструкции внутри пачки и группы не могут мешать друг другу (т.е. не должны
иметь data hazard-ов)}
\EN{So all instructions inside a bundle and a bundle group cannot interfere with each other 
(i.e., must not have data hazards)}.
\RU{А если это так, то результаты будут непредсказуемые.}
\EN{If they do, the results are to be undefined.}

\RU{На ассемблере, каждый стоп-бит маркируется как две точки с запятой (\TT{;;}) после инструкции.}
\EN{Each stop bit is marked in assembly language as two semicolons (\TT{;;}) after the instruction.}
\RU{Так, инструкции на [90-ac] могут быть исполнены одновременно: они не мешают друг другу}
\EN{So, the instructions at [90-ac] may be executed simultaneously:
they do not interfere}. \RU{Следующая группа:}\EN{The next group is} [b0-cc].

\RU{Мы также видим стоп-бит на 10c}\EN{We also see a stop bit at 10c}.
\RU{Следующая инструкция на 110 также имеет стоп-бит}\EN{The next instruction at 110 has a stop bit too}.
\RU{Это значит, что эти инструкции должны исполняться изолированно от всех остальных (как в \ac{CISC})}
\EN{This implies that these instructions must be executed isolated from all others (as in \ac{CISC})}.
\RU{Действительно: следующая инструкция на 110 использует результат, полученный от предыдущей (значение
в регистре r26), так что они не могут исполняться одновременно}
\EN{Indeed: the next instruction at 110 uses the result from the previous one (the value in register r26),
so they cannot be executed at the same time}.
\RU{Должно быть, компилятор не смог найти лучший способ распараллелить инструкции, или, иными
словами, загрузить \ac{CPU} насколько это возможно, отсюда так много стоп-битов и \ac{NOP}-ов}
\EN{Apparently, the compiler was not able to find a better way to parallelize the instructions,
in other words, to load \ac{CPU} as much as possible, hence too much stop bits and \ac{NOP}s}.
\RU{Писать на ассемблере вручную это также очень трудная задача: программист должен группировать
инструкции вручную}
\EN{Manual assembly programming is a tedious job as well: the programmer has to group the instructions manually}.

\RU{У программиста остается возможность добавлять стоп-биты к каждой инструкции, но это
сведет на нет всю мощность Itanium, ради которой он создавался.}
\EN{The programmer is still able to add stop bits to each instructions, but this will degrade
the performance that Itanium was made for.}

\RU{Интересные примеры написания \ac{IA64}-кода вручную можно найти в исходниках ядра Linux}
\EN{An interesting examples of manual \ac{IA64} assembly code can be found in the Linux kernel's sources}:

\url{http://go.yurichev.com/17322}.

\RU{Еще пара вводных статей об ассемблере Itanium}
\EN{Another introductory paper on Itanium assembly}:
[Mike Burrell, \IT{Writing Efficient Itanium 2 Assembly Code} (2010)]\footnote{\AlsoAvailableAs \url{http://yurichev.com/mirrors/RE/itanium.pdf}},
[papasutra of haquebright, \IT{WRITING SHELLCODE FOR IA-64} (2001)]\footnote{\AlsoAvailableAs \url{http://phrack.org/issues/57/5.html}}.

\RU{Еще одна интересная особенность Itanium это \IT{speculative execution} (исполнение инструкций
заранее, когда еще не известно, нужно ли это) и бит NaT (\q{not a thing}), отдаленно напоминающий
\gls{NaN}-числа}
\EN{Another very interesting Itanium feature is the \IT{speculative execution} and the NaT (\q{not a thing}) bit,
somewhat resembling \gls{NaN} numbers}: \\
\href{http://go.yurichev.com/17323}{MSDN}.

