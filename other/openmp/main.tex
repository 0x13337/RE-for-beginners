\chapter{OpenMP}
\label{openmp}
\index{OpenMP}

OpenMP \RU{это один из простейших способов распараллелить работу простого алгоритма}
\EN{is one of the simplest ways to parallelize simple algorithms}.

\index{Bitcoin}
\RU{В качестве примера, попытаемся написать программу для вычисления криптографического \IT{nonce}}
\EN{As an example, let's try to build a program to compute a cryptographic \IT{nonce}}.
\RU{В моем простейшем примере, \IT{nonce} это число, добавляемое к нешифрованному тексту, чтобы получить
хэш с какой-то особенностью}
\EN{In my simplistic example, 
the \IT{nonce} is a number added to the plain unencrypted text in order to produce a hash with some specific 
features}.
\RU{Например, на одной из стадии, протокол Bitcoin требует найти такую \IT{nonce}, чтобы в результате
хэширования подряд шли определенное количество нулей.}
\EN{For example, at some step, the Bitcoin protocol requires to find such \IT{nonce} so the resulting hash
contains a specific number of consecutive zeroes.}
\RU{Это еще называется}\EN{This is also called} ``proof of work''
\footnote{\RU{\href{http://go.yurichev.com/17100}{wikipedia}}\EN{\href{http://go.yurichev.com/17101}{wikipedia}}} 
(\RU{т.е. система доказывает, что она произвела какие-то очень ресурсоёмкие вычисления и затратила
время на это}
\EN{i.e., the system proves that it did some intensive calculations and spent some time for it}).

\index{SHA512}
\RU{Мой пример не связан с}\EN{My example is not related to} Bitcoin\EN{ in any way}, 
\RU{он будет пытаться добавлять числа к строке}\EN{it will try to add numbers to the} ``hello, world!\_''
\RU{чтобы найти такое число, при котором строка вида}\EN{string in order to find such number that when} 
``hello, world!\_<number>'' \RU{после хеширования алгоритмом SHA512 будет содержать как минимум 3 нулевых
байта в начале}\EN{is hashed with the SHA512 algorithm, it will contain at least 3 zero bytes}.

\RU{Ограничимся перебором всех чисел в интервале}\EN{Let's limit our brute-force to the interval in}
0..INT32\_MAX-1 (\RU{т.е.}\EN{i.e.}, \TT{0x7FFFFFFE} \OrENRU 2147483646).

\RU{Алгоритм очень простой}\EN{The algorithm is pretty straightforward}:

\lstinputlisting{other/openmp/openmp_example.c}

\EN{The }\TT{check\_nonce()} \RU{просто добавляет число к строке, хеширует алгоритмом SHA512 и проверяет 
3 нулевых байта в начале}\EN{function just adds a number to the string, 
hashes it with the SHA512 algorithm and checks for 3 zero bytes in the result}.

\RU{Очень важная часть кода --- это}\EN{A very important part of the code is}:

\begin{lstlisting}
	#pragma omp parallel for
	for (i=0; i<INT32_MAX; i++)
		check_nonce (i);
\end{lstlisting}

\RU{Да, вот настолько просто, без}\EN{Yes, that simple, without} \TT{\#pragma} \RU{мы просто вызываем}
\EN{we just call} \TT{check\_nonce()} \RU{для каждого числа от 0 до}\EN{for each number from 0 to} 
\TT{INT32\_MAX} (\TT{0x7fffffff} \OrENRU 2147483647).
\RU{С}\EN{With} \TT{\#pragma}, \RU{компилятор добавляет специальный код, который разрежет интервал цикла
на меньшие интервалы, чтобы запустить их на доступных ядрах \ac{CPU}}\EN{the compiler adds some special 
code which slices the loop interval into smaller ones,
to run them on all \ac{CPU} cores available}
\footnote{N.B.: \RU{Я намеренно использовал простейший пример, но на практике, 
применение OpenMP может быть труднее и сложнее}\EN{I intentionally demonstrate here the simplest possible
example, but in practice, the usage of OpenMP can be harder and more complex}}.

\RU{Пример может быть скомпилирован}\EN{The example can be compiled}
\footnote{\RU{файлы }sha512.(c|h) \AndENRU u64.h \RU{можно взять из библиотеки OpenSSL}
\EN{files can be taken from the OpenSSL library}:
\url{http://go.yurichev.com/17324}}
\InENRU MSVC 2012:
% FIXME1: \footnote{other source code files can be downloaded here: ...} 

\begin{lstlisting}
cl openmp_example.c sha512.obj /openmp /O1 /Zi /Faopenmp_example.asm
\end{lstlisting}

\RU{Или в}\EN{Or in} GCC:

\begin{lstlisting}
gcc -fopenmp 2.c sha512.c -S -masm=intel
\end{lstlisting}

\section{MSVC}

\RU{Вот как}\EN{Now this is how} MSVC 2012 \RU{генерирует главный цикл}\EN{generates the main loop}:

\lstinputlisting[caption=MSVC 2012]{other/openmp/MSVC_loop.asm}

\RU{Ф-ции с префиксом}\EN{All functions prefixed by} \TT{vcomp} 
\RU{связаны с}\EN{are} OpenMP\RU{ и находятся в файле}\EN{-related and are stored in the} 
vcomp*.dll\EN{ file}.
\RU{Так что тут запускается группа тредов}\EN{So here a group of threads is started}.

\RU{Посмотрим на}\EN{Let's take a look on} \TT{\_main\$omp\$1}:

\lstinputlisting[caption=MSVC 2012]{other/openmp/MSVC_1.asm}

\RU{Эта функция будет запущена}\EN{This function is to be started} $n$ \RU{раз параллельно, где}
\EN{times in parallel, where} $n$ \RU{это число ядер \ac{CPU}}\EN{is the number of \ac{CPU} cores}.
\TT{vcomp\_for\_static\_simple\_init()} \RU{вычисляет интервал для конструкта}
\EN{calculates the interval for the} for() \RU{для текущего треда, в зависимости от текущего номера треда}
\EN{construct for the current
thread, depending on the current thread's number}.
\RU{Значения начала и конца цикла записаны в локальных переменных}
\EN{The loop's start and end values are stored in the} \TT{\$T1} \AndENRU \TT{\$T2}\EN{ local variables}.
\RU{Вы также можете заметить}\EN{You may also notice} \TT{7ffffffeh} (\OrENRU 2147483646) 
\RU{как аргумент для функции}\EN{as an argument to the} 
\TT{vcomp\_for\_static\_simple\_init()}
\RU{это количество итераций всего цикла, оно будет поделено на равные части}
\EN{function---this is the number of iterations for the whole loop, to be divided evenly}.

\RU{Потом мы видим новый цикл с вызовом функции}\EN{Then we see a new loop with a call to the} 
\TT{check\_nonce()} \RU{делающей всю работу}\EN{function, which does all the work}.

\RU{Я также добавил немного кода в начале функции}
\EN{I also added some code in the beginning of the} \TT{check\_nonce()} \RU{для сбора статистики,
с какими аргументами эта функция вызывалась}\EN{function to gather statistics about
the arguments with which the function was called}.

\RU{Вот что мы видим если запустим}\EN{This is what we see when we run it}:

\begin{lstlisting}
threads=4
...
checked=2800000
checked=3000000
checked=3200000
checked=3300000
found (thread 3): [hello, world!_1611446522]. seconds spent=3
__min[0]=0x00000000 __max[0]=0x1fffffff
__min[1]=0x20000000 __max[1]=0x3fffffff
__min[2]=0x40000000 __max[2]=0x5fffffff
__min[3]=0x60000000 __max[3]=0x7ffffffe
\end{lstlisting}

\RU{Да, результат правильный, первые 3 байта это нули}\EN{Yes, the result is correct, the first 3 bytes are zeroes}:

\begin{lstlisting}
C:\...\sha512sum test
000000f4a8fac5a4ed38794da4c1e39f54279ad5d9bb3c5465cdf57adaf60403
df6e3fe6019f5764fc9975e505a7395fed780fee50eb38dd4c0279cb114672e2 *test
\end{lstlisting}

\RU{Оно требует}\EN{The running time is} $\approx2..3$ 
\RU{секунды на моем 4-х ядерном}\EN{seconds on my 4-core} 
Intel Xeon E3-1220 3.10 GHz.
\RU{В}\EN{In the} task manager \RU{я вижу 5 тредов}\EN{I see 5 threads}: \RU{один главный тред + 4 запущенных}
\EN{1 main thread + 4 more}.
\RU{Я не делал никаких оптимизаций, чтобы оставить пример в как можно более простом виде}
\EN{I did not do any further optimizations to keep my example as small and clear as possible}.
\RU{Но, наверное, этот алгоритм может работать быстрее}\EN{But probably it can be done much faster}.
\RU{У моего}\EN{My} \ac{CPU} \RU{4 ядра, вот почему}\EN{has 4 cores, that is why} OpenMP 
\RU{запустил именно 4 треда}\EN{started exactly 4 threads}.

\RU{Глядя на таблицу статистики, можно легко увидеть, что цикл был разделен очень точно на 4 равных 
части}\EN{By looking at the statistics table we can clearly see how the loop was sliced in 4 even 
parts}.
\RU{Ну хорошо, почти равных, если не учитывать последний бит}\EN{Oh well, almost even, 
if we don't consider the last bit}.

\RU{Имеются также прагмы и для}\EN{There are also pragmas for} 
\glslink{atomic operation}{\RU{атомарных операций}\EN{atomic operations}}.

\RU{Посмотрим, как вот этот код будет скомпилирован}\EN{Let's see how this code is compiled}:

\begin{lstlisting}
	#pragma omp atomic
	checked++;

	#pragma omp critical
	if ((checked % 100000)==0)
		printf ("checked=%d\n", checked);
\end{lstlisting}

\lstinputlisting[caption=MSVC 2012]{other/openmp/MSVC_2.asm}

\RU{Как выясняется, функция}\EN{As it turns out, the} \TT{vcomp\_atomic\_add\_i4()} 
\RU{в}\EN{function in the} vcomp*.dll \RU{это просто крохотная функция имеющая инструкцию}
\EN{is just a tiny function
with the} \TT{LOCK XADD}\EN{ instruction}\footnote{\RU{О префиксе LOCK читайте больше}
\EN{Read more about LOCK prefix}: \myref{x86_lock}}\EN{ in it}.

\TT{vcomp\_enter\_critsect()} \RU{в конце концов вызывает функцию win32 \ac{API}}
\EN{eventually calling win32 \ac{API} function} \TT{EnterCriticalSection()}
\footnote{\RU{О критических секциях читайте больше тут}\EN{You can read more about critical sections 
here}: \myref{critical_sections}}.

\section{GCC}

GCC 4.8.1 \RU{выдает программу показывающую точно такую же таблицу со статистикой}
\EN{produces a program which shows exactly the same statistics table}, 
\RU{так что, реализация GCC делит цикл на части точно так же}
\EN{so, GCC's implementation divides the loop in parts in the same fashion}.

\begin{lstlisting}[caption=GCC 4.8.1]
	mov	edi, OFFSET FLAT:main._omp_fn.0
	call	GOMP_parallel_start
	mov	edi, 0
	call	main._omp_fn.0
	call	GOMP_parallel_end
\end{lstlisting}

\RU{В отличие от реализации MSVC, то, что делает код GCC, это запускает 3 треда, но также запускает 
четвертый прямо в текущем треде}\EN{Unlike MSVC's implementation, what GCC code does is to start 3 threads,
and run the fourth in the current thread}. \RU{Так что здесь всего 4 треда а не 5 как в случае
с MSVC}\EN{So there are 4 threads instead of the 5 in MSVC}.

\RU{Вот функция}\EN{Here is the} \TT{main.\_omp\_fn.0}\EN{ function}:
 
\lstinputlisting[caption=GCC 4.8.1]{other/openmp/GCC_1.asm}

\RU{Здесь мы видим это деление явно: вызывая}\EN{Here we see the division clearly: by calling} 
\TT{omp\_get\_num\_threads()} \AndENRU \TT{omp\_get\_thread\_num()}
\RU{мы получаем количество запущенных тредов, а также номер текущего треда}
\EN{we get the number of threads running, and also the current thread's number}, \RU{и затем определяем
интервал цикла}\EN{and then determine the loop's interval}.
\RU{И затем запускаем}\EN{Then we run} \TT{check\_nonce()}.

GCC \RU{также вставляет инструкцию}\EN{also inserted the} \TT{LOCK ADD} 
\RU{прямо в том месте кода, где MSVC сгенерировал вызов отдельной функции в DLL}
\EN{instruction right in the code, unlike MSVC, which generated a call to a separate DLL function}:

\lstinputlisting[caption=GCC 4.8.1]{other/openmp/GCC_2.asm}

\RU{Ф-ции с префиксом}\EN{The functions prefixed with} GOMP \RU{это часть библиотеки}\EN{are from} 
GNU OpenMP\EN{ library}.
\RU{В отличие от}\EN{Unlike} vcomp*.dll, \RU{её исходный код свободно доступен}
\EN{its source code is freely available}: 
\href{http://go.yurichev.com/17102}{GitHub}.

