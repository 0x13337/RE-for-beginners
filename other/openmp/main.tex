\section{OpenMP}
\label{openmp}
\index{OpenMP}

OpenMP \IFRU{это один из простейших способов распараллелить работу простого алгоритма}
{is one of the simpliest ways to parallelize simple algorithm}.

\index{Bitcoin}
\IFRU{В качестве примера, попытаемся написать программу для вычисления криптографического \IT{nonce}}
{As an example, let's try to build a program to compute cryptographic \IT{nonce}}.
\IFRU{В моем простейшем примере, \IT{nonce} это число, добавляемое к нешифрованному тексту, чтобы получить
хэш с какой-то особенностью}
{In my simplistic example, 
\IT{nonce} is a number added to the plain unencrypted text in order to produce hash with some specific 
feauture}.
\IFRU{Например, на одной из стадии, протокол Bitcoin требует найти такую \IT{nonce}, чтобы в результате
хэширования подряд шли определенное количество нулей}
{For example, at some step, Bitcoin protocol require to find a such \IT{nonce} so resulting hash
will contain specific number of running zeroes}. 
\IFRU{Это еще называется}{This is also called} ``proof of work''
\footnote{\IFRU{\url{https://ru.wikipedia.org/wiki/Proof-of-work}}
{\url{https://en.wikipedia.org/wiki/Proof-of-work_system}}} 
(\IFRU{т.е., система доказывает, что она произвела какие-то очень ресурсоёмкие вычисления и затратила
время на это}
{i.e., system prove it did some intensive calculations and spent some time for it}).

\index{SHA512}
\IFRU{Мой пример не связан с}{My example is not related to} Bitcoin, 
\IFRU{он будет пытаться добавлять числа к строке}{it will try to add a numbers to the} ``hello, world!\_''
\IFRU{чтобы найти такое число, при котором строка вида}{string in order to find such number when} 
``hello, world!\_<number>'' \IFRU{после хеширования алгоритмом SHA512 будет содержать как минимум 3 нулевых
байта в начале}{will contain at least 3 zero bytes after hashing this string by SHA512 algorithm}.

\IFRU{Ограничимся перебором всех чисел в интервале}{Let's limit our brute-force to the interval in}
0..INT32\_MAX-1 (\IFRU{т.е.}{i.e.}, \TT{0x7ffffffe} \OrENRU 2147483646).

\IFRU{Алгоритм очень простой}{The algorithm is pretty straightforward}:

\lstinputlisting{other/openmp/openmp_example.c}

\TT{check\_nonce()} \IFRU{просто добавляет число к строке, хеширует алгоритмом SHA512 и проверяет 
3 нулевых байта в начале}{function is just add a number to the string, 
hashes it by SHA512 and checks for 3 zero bytes in the result}.

\IFRU{Очень важная часть кода это}{Very important part of the code is}:

\begin{lstlisting}
	#pragma omp parallel for
	for (i=0; i<INT32_MAX; i++)
		check_nonce (i);
\end{lstlisting}

\IFRU{Да, вот настолько просто, без}{Yes, that simple, without} \TT{\#pragma} \IFRU{мы просто вызываем}
{we just call} \TT{check\_nonce()} \IFRU{для каждого числа от 0 до}{for each number from 0 to} 
\TT{INT32\_MAX} (\TT{0x7fffffff} \OrENRU 2147483647).
\IFRU{С}{With} \TT{\#pragma}, \IFRU{компилятор добавляет специальный код, который разрежет интервал цикла
на меньшие интервалы, чтобы запустить их на доступных ядрах \ac{CPU}}{a compiler adds a special 
code which will splice the loop interval to smaller intervals,
to run them by all \ac{CPU} cores avaiable}
\footnote{N.B.: \IFRU{Я намеренно использовал простейший пример, но на практике, 
применение OpenMP может быть труднее и сложнее}{I intentionally demonstrate here simplest possible
example, but in practice, usage of OpenMP may be harder and more complex}}.

\IFRU{Пример может быть скомпилирован}{The example may be compiled}
\footnote{\IFRU{файлы }{}sha512.(c|h) \AndENRU u64.h \IFRU{можно взять из библиотеки OpenSSL}
{files can be taken from the OpenSSL library}:
\url{http://www.openssl.org/source/}}
\InENRU MSVC 2012:
% FIXME: \footnote{other source code files can be downloaded here: ...} 

\begin{lstlisting}
cl openmp_example.c sha512.obj /openmp /O1 /Zi /Faopenmp_example.asm
\end{lstlisting}

\IFRU{Или в}{Or in} GCC:

\begin{lstlisting}
gcc -fopenmp 2.c sha512.c -S -masm=intel
\end{lstlisting}

\subsection{MSVC}

\IFRU{Вот как}{Now that's how} MSVC 2012 \IFRU{генерирует главный цикл}{generates main loop}:

\lstinputlisting[caption=MSVC 2012]{other/openmp/MSVC_loop.asm}

\IFRU{Ф-ции с префиксом}{All functions prefixed by} \TT{vcomp} 
\IFRU{связаны с}{are} OpenMP\IFRU{ и находятся в файле}{-related and stored in the} 
vcomp*.dll\IFRU{}{ file}.
\IFRU{Так что тут запускается группа тредов}{So here is a group of threads are started}.

\IFRU{Посмотрим на}{Let's take a look on} \TT{\_main\$omp\$1}:

\lstinputlisting[caption=MSVC 2012]{other/openmp/MSVC_1.asm}

\IFRU{Эта ф-ция будет запущена}{This function will be started} $n$ \IFRU{раз параллельно, где}
{times in parallel, where} $n$ \IFRU{это число ядер \ac{CPU}}{is number of \ac{CPU} cores}.
\TT{vcomp\_for\_static\_simple\_init()} \IFRU{вычисляет интервал для конструкта}
{is calculating interval for the} for() \IFRU{для текущего треда, в зависимости от текущего номера треда}
{construct for the current
thread, depending on the current thread number}.
\IFRU{Значения начала и конца цикла записаны в локальных переменных}
{Loop begin and end values are stored in} \TT{\$T1} \AndENRU \TT{\$T2}\IFRU{}{ local variables}.
\IFRU{Вы также можете заметить}{You may also notice} \TT{7ffffffeh} (\OrENRU 2147483646) 
\IFRU{как аргумент для ф-ции}{as an argument to the} 
\TT{vcomp\_for\_static\_simple\_init()}
\IFRU{это количество итераций всего цикла, оно будет поделено на равные части}
{function---this is a number of iterations of the whole loop to by divided evenly}.

\IFRU{Потом мы видим новый цикл с вызовом ф-ции}{Then we see a new loop with a call to} 
\TT{check\_nonce()} \IFRU{делающей всю работу}{function which do all work}.

\IFRU{Я также добавил немного кода в начале ф-ции}
{I also added some code in the beginning of} \TT{check\_nonce()} \IFRU{для сбора статистики,
с какими аргументами эта ф-ция вызывалась}{function to gather statistics, with which
arguments the function was called}.

\IFRU{Вот что мы видим если запустим}{This is what we see while run it}:

\begin{lstlisting}
threads=4
...
checked=2800000
checked=3000000
checked=3200000
checked=3300000
found (thread 3): [hello, world!_1611446522]. seconds spent=3
__min[0]=0x00000000 __max[0]=0x1fffffff
__min[1]=0x20000000 __max[1]=0x3fffffff
__min[2]=0x40000000 __max[2]=0x5fffffff
__min[3]=0x60000000 __max[3]=0x7ffffffe
\end{lstlisting}

\IFRU{Да, результат правильный, первые 3 байта это нули}{Yes, result is correct, first 3 bytes are zeroes}:

\begin{lstlisting}
C:\...\sha512sum test
000000f4a8fac5a4ed38794da4c1e39f54279ad5d9bb3c5465cdf57adaf60403df6e3fe6019f5764fc9975e505a7395fed78
0fee50eb38dd4c0279cb114672e2 *test
\end{lstlisting}

\IFRU{Оно требует}{Running time is} $\approx2..3$ 
\IFRU{секунды на моем 4-х ядерном}{seconds on my 4-core} 
Intel Xeon E3-1220 3.10 GHz.
\IFRU{В}{In the} task manager \IFRU{я вижу 5 тредов}{I see 5 threads}: \IFRU{один главный тред + 4 запущенных}
{1 main thread + 4 more started}.
\IFRU{Я не делал никаких оптимизаций, чтобы оставить пример в как можно более простом виде}
{I did not any further optimizations to keep my example as small and clear as possible}.
\IFRU{Но, наверное, этот алгоритм может работать быстрее}{But probably it can be done much faster}.
\IFRU{У моего}{My} \ac{CPU} \IFRU{4 ядра, вот почему}{has 4 cores, that is why} OpenMP 
\IFRU{запустил именно 4 треда}{started exactly 4 threads}.

\IFRU{Глядя на таблицу статистики, можно легко увидеть, что цикл был разделен очень точно на 4 равных 
части}{By looking at the statistics table we can clearly see how loop was finely spliced by 4 even 
parts}.
\IFRU{Ну хорошо, почти равных, если не учитывать последний бит}{Oh well, almost even, 
if not to consider the last bit}.

\IFRU{Имеются также прагмы и для}{There are also pragmas for} 
\glslink{atomic operation}{\IFRU{атомарных операций}{atomic operations}}.

\IFRU{Посмотрим, как вот этот код будет скомпилирован}{Let's see how this code is compiled}:

\begin{lstlisting}
	#pragma omp atomic
	checked++;

	#pragma omp critical
	if ((checked % 100000)==0)
		printf ("checked=%d\n", checked);
\end{lstlisting}

\lstinputlisting[caption=MSVC 2012]{other/openmp/MSVC_2.asm}

\IFRU{Как выясняется, ф-ция}{As it turns out,} \TT{vcomp\_atomic\_add\_i4()} 
\IFRU{в}{function in the} vcomp*.dll \IFRU{это просто крохотная ф-ция имеющая инструкцию}
{is just a a tiny function
having} \TT{LOCK XADD}\IFRU{}{ instruction}\footnote{\IFRU{О префиксе LOCK читайте больше}
{Read more about LOCK prefix}: \ref{x86_lock}}.

\TT{vcomp\_enter\_critsect()} \IFRU{в конце концов вызывает ф-цию win32 \ac{API}}
{eventually callins win32 \ac{API} function} \TT{EnterCriticalSection()}
\footnote{\IFRU{О критических секциях читайте больше тут}{Read more about critical sections 
here}: \ref{critical_sections}}.

\subsection{GCC}

GCC 4.8.1 \IFRU{выдает программу показывающую точно такую же таблицу со статистикой}
{produces the program which shows exactly the same statistics table}, 
\IFRU{так что, реализация GCC делит цикл на части точно также}
{so, GCC implementation divides the loop by parts in the same fashion}.

\begin{lstlisting}[caption=GCC 4.8.1]
	mov	edi, OFFSET FLAT:main._omp_fn.0
	call	GOMP_parallel_start
	mov	edi, 0
	call	main._omp_fn.0
	call	GOMP_parallel_end
\end{lstlisting}

\IFRU{В отличие от реализации MSVC, то, что делает код GCC, это запускает 3 треда, но также запускает 
четвертый прямо в текущем треде}{Unlike MSVC implementation, what GCC code is doing is starting 3 threads,
but also runs fourth in the current thread}. \IFRU{Так что здесь всего 4 треда а не 5 как в случае
с MSVC}{So there will be 4 threads instead of 5 as in MSVC}.

\IFRU{Вот ф-ция}{Here is a} \TT{main.\_omp\_fn.0}\IFRU{}{ function}:
 
\lstinputlisting[caption=GCC 4.8.1]{other/openmp/GCC_1.asm}

\IFRU{Здесь мы видим это деление явно: вызывая}{Here we see that division clearly: by calling to} 
\TT{omp\_get\_num\_threads()} \AndENRU \TT{omp\_get\_thread\_num()}
\IFRU{мы получаем количество запущенных тредов, а также номер текущего треда}
{we got number of threads running, and also current thread number}, \IFRU{и затем определяем
интервал цикла}{and then determine loop interval}.
\IFRU{И затем запускаем}{Then run} \TT{check\_nonce()}.

GCC \IFRU{также вставляет инструкцию}{also inserted} \TT{LOCK ADD} 
\IFRU{прямо в том месте кода, где MSVC сгенерировал вызов отдельной ф-ции в DLL}
{instruction right in the code, where MSVC generated call to separate DLL function}:

\lstinputlisting[caption=GCC 4.8.1]{other/openmp/GCC_2.asm}

\IFRU{Ф-ции с префиксом}{Functions prefixed with} GOMP \IFRU{это часть библиотеки}{are from} 
GNU OpenMP\IFRU{}{ library}.
\IFRU{В отличие от}{Unlike} vcomp*.dll, \IFRU{её исходный код свободно доступен}
{its sources are freely available}: 
\url{https://github.com/mirrors/gcc/tree/master/libgomp}.

