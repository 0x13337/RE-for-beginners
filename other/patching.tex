\chapter{\RU{Модификация исполняемых файлов}\EN{Executable files patching}}

\section{\RU{Текстовые строки}\EN{Text strings}}

\RU{Сишные строки проще всего модифицировать (если они не зашифрованы) в любом шестнадцатеричном редакторе}
\EN{C strings are most easily patched (unless they are encrypted) in any hex editor}.
\RU{Эта техника доступна даже для тех, кто вовсе не разбирается в машинном коде и форматах исполняемых
файлов}\EN{This technique available even for those who are not aware of machine code and executabe file formats}.
\RU{Новая строка не должна быть длиннее старой, потому что имеется риск затереть какую-то другую переменную
или код}\EN{New string should not be bigger than old, because it's a risk to overwrite some other value or code
there}.
\index{MS-DOS}
\RU{Используя этот метод, очень много ПО было \IT{локализовано} во времена MS-DOS, как минимум,
в странах бывшего СССР, в 80-х и 90-х}
\EN{Using this method, a lot of software was \IT{localized} in MS-DOS era, at least in ex-USSR countries in 80's
and 90's}.
\RU{Отсюда наличие очень странных аббревиатур и сокращений в \IT{локализованном} ПО: 
там просто не было места для более
длинных строк}\EN{It was a reason why so weird abbreviations was present in \IT{localized} software: it was no room for
longer strings}.

\index{Borland Delphi}
\RU{В строках в Delphi, длина строки также должна быть поправлена, если нужно}
\EN{As of Delphi strings, a string size should also be corrected, if needed}.

\section{x86\RU{-код}\EN{ code}}

\RU{Часто необходимые задачи}\EN{Frequent patching tasks are}:

\begin{itemize}

\item \RU{Часто нужно просто запретить исполнение какой-либо инструкции}
\EN{One of the most frequently job is to disable some instruction}.
\RU{И чаще всего, это можно сделать заполняя её байтом}\EN{It is often done by filling it by byte} 
\TT{0x90} (\ac{NOP}).

\item \RU{Условные переходы, имеющие опкод вроде}\EN{Conditional jumps, which have opcode like} \TT{74 xx} (\JZ), 
\RU{так же могут быть заполнены двумя \ac{NOP}-ами}\EN{may also be filled by two \ac{NOP}s}.
\RU{Также возможно запретить исполнение условного перехода записав 0 во второй байт}
\EN{It is also possible to disable conditional jump by writing 0 at the second byte} (\IT{jump offset}).

\item \RU{Еще одна часто необходимая задача это сделать условный переход всегда срабатывающим}
\EN{Another frequent job is to make conditional jump to trigger always}: 
\RU{это возможно при помощи записи}\EN{this can be done by writing} \TT{0xEB} \RU{вместо опкода, это значит}
\EN{instead of opcode, it mean} \JMP.

\index{stdcall}
\item \RU{Исполнение ф-ции может быть запрещено, если записать}\EN{A function execution can be disabled by writing}
\RETN (0xC3) \RU{в её начале}\EN{at its beginning}.
\RU{Это справедливо для всех ф-ций кроме}\EN{This is true for all functions excluding} \TT{stdcall} 
(\ref{sec:stdcall}).
\RU{При модификации ф-ций}\EN{While patching} \TT{stdcall}\RU{, нужно в начале определить количество аргументов 
(например, отыскав \RETN в этой ф-ции),
и использовать}\EN{ functions, one should determine number of arguments (for example, 
by finding \RETN in this function), 
and use} \RETN \RU{с 16-битным аргументом}\EN{with 16-bit argument} (0xC2).

\item \RU{Иногда, запрещенная ф-ция должна возвращать}\EN{Sometimes, a disabled functions should return} 0 \OrENRU 1.
\RU{Это можно сделать при помощи}\EN{This can be done by} \TT{MOV EAX, 0} \OrENRU \TT{MOV EAX, 1}, 
\RU{но это слишком многословно}\EN{but it's slightly verbose}.
\RU{Способ получше это}\EN{Better way is} \TT{XOR EAX, EAX} (2 \RU{байта}\EN{bytes}) \OrENRU 
\TT{XOR EAX, EAX / INC EAX} (3 \RU{байта}\EN{bytes}).

\end{itemize}

\RU{ПО может быть защищено от модификаций}\EN{A software may be protected against modifications}.
\RU{Эта защита чаще всего реализуется путем чтения кода и вычисления контрольной суммы}
\EN{This protection is often done by reading executable code and doing some checksumming}.
\RU{Следовательно, код должен быть прочитан перед тем как защита сработает}\EN{Therefore, 
the code should be read before protection will be triggered}.
\RU{Это можно определить установив брякпойнт на чтение памяти}
\EN{This can be determined by setting breakpoint on reading memory}.

\index{tracer}
\RU{В \tracer имеется опция BPM для этого}\EN{\tracer has BPM option for this}.

\RU{Релоки в исполняемых PE-файлах}\EN{PE executable file relocs} (\ref{subsec:relocs}) 
\RU{не должны быть тронуты, потому что загрузчик Windows перезапишет ваш новый код}
\EN{should not be touched while patching, 
because Windows loader will overwrite a new code}.
\index{Hiew}
(\RU{Они выделяются серым в Hiew, например}\EN{They are grayed in Hiew, for example}:
\figname \ref{fig:scanf_ex3_hiew_1}).
\RU{В качестве последней меры, можно записать \JMP для обхода релока, либо же придется модифицировать таблицу
релоков}\EN{As a last resort, it is possible to write jumps circumventing relocs, 
or one will need to edit relocs table}.

