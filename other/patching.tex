\chapter{\RU{Модификация исполняемых файлов}\EN{Executable files patching}}

\section{\RU{Текстовые строки}\EN{Text strings}}

\RU{Сишные строки проще всего модифицировать (если они не зашифрованы) в любом шестнадцатеричном редакторе}
\EN{The C strings are the thing that is the easiest to patch (unless they are encrypted) in any hex editor}.
\RU{Эта техника доступна даже для тех, кто вовсе не разбирается в машинном коде и форматах исполняемых
файлов}\EN{This technique is available even for those who are not aware of machine code and executable file formats}.
\RU{Новая строка не должна быть длиннее старой, потому что имеется риск затереть какую-то другую переменную
или код}\EN{The new string has not to be bigger than the old one, because there's a risk of overwriting another value or code
there}.
\index{MS-DOS}
\RU{Используя этот метод, очень много ПО было \IT{локализовано} во времена MS-DOS, как минимум,
в странах бывшего СССР, в 80-х и 90-х}
\EN{Using this method, a lot of software was \IT{localized} in the MS-DOS era, at least in the ex-USSR countries in 80's
and 90's}.
\RU{Отсюда наличие очень странных аббревиатур и сокращений в \IT{локализованном} ПО: 
там просто не было места для более
длинных строк}\EN{It was the reason why some weird abbreviations were present in the \IT{localized} software: there was no room for
longer strings}.

\index{Borland Delphi}
\RU{В строках в Delphi, длина строки также должна быть поправлена, если нужно}
\EN{As for Delphi strings, the string's size must also be corrected, if needed}.

\section{x86\RU{-код}\EN{ code}}
\label{x86_patching}

\RU{Часто необходимые задачи}\EN{Frequent patching tasks are}:

\begin{itemize}

\index{x86!\Instructions!NOP}
\item \RU{Часто нужно просто запретить исполнение какой-либо инструкции}
\EN{One of the most frequent jobs is to disable some instruction}.
\RU{И чаще всего, это можно сделать, заполняя её байтом}\EN{It is often done by filling it using byte} 
\TT{0x90} (\ac{NOP}).

\item \RU{Условные переходы, имеющие опкод вроде}\EN{Conditional jumps, which have an opcode like} \TT{74 xx} (\JZ), 
\RU{так же могут быть заполнены двумя \ac{NOP}-ами}\EN{can be filled with two \ac{NOP}s}.
\RU{Также возможно запретить исполнение условного перехода записав 0 во второй байт}
\EN{It is also possible to disable a conditional jump by writing 0 at the second byte} (\IT{jump offset}).

\index{x86!\Instructions!JMP}
\item \RU{Еще одна часто необходимая задача это сделать условный переход всегда срабатывающим}
\EN{Another frequent job is to make a conditional jump to always trigger}: 
\RU{это возможно при помощи записи}\EN{this can be done by writing} \TT{0xEB} 
\RU{вместо опкода, это значит}
\EN{instead of the opcode, which stands for} \JMP.

\index{x86!\Instructions!RET}
\index{stdcall}
\item \RU{Исполнение функции может быть запрещено, если записать}\EN{A function's execution can be disabled by writing}
\RETN (0xC3) \RU{в её начале}\EN{at its beginning}.
\RU{Это справедливо для всех функций кроме}\EN{This is true for all functions excluding} \TT{stdcall} 
(\myref{sec:stdcall}).
\RU{При модификации функций}\EN{While patching} \TT{stdcall}\RU{, нужно в начале определить количество аргументов 
(например, отыскав \RETN в этой функции),
и использовать}\EN{ functions, one has to determine the number of arguments (for example, 
by finding \RETN in this function), 
and use} \RETN \RU{с 16-битным аргументом}\EN{with a 16-bit argument} (0xC2).

\index{x86!\Instructions!MOV}
\index{x86!\Instructions!XOR}
\index{x86!\Instructions!INC}
\item \RU{Иногда, запрещенная функция должна возвращать}\EN{Sometimes, a disabled functions has to return} 0 \OrENRU 1.
\RU{Это можно сделать при помощи}\EN{This can be done by} \TT{MOV EAX, 0} \OrENRU \TT{MOV EAX, 1}, 
\RU{но это слишком многословно}\EN{but it's slightly verbose}.
\RU{Способ получше это}\EN{A better way is} \TT{XOR EAX, EAX} (2 \RU{байта}\EN{bytes} \TT{0x31 0xC0}) 
\OrENRU 
\TT{XOR EAX, EAX / INC EAX} (3 \RU{байта}\EN{bytes} \TT{0x31 0xC0 0x40}).

\end{itemize}

\RU{ПО может быть защищено от модификаций}\EN{A software may be protected against modifications}.
\RU{Эта защита чаще всего реализуется путем чтения кода и вычисления контрольной суммы}
\EN{This protection is often done by reading the executable code and calculating a checksum}.
\RU{Следовательно, код должен быть прочитан перед тем как защита сработает}\EN{Therefore, 
the code must be read before protection is triggered}.
\RU{Это можно определить установив точку останова на чтение памяти}
\EN{This can be determined by setting a breakpoint on reading memory}.

\index{tracer}
\RU{В \tracer имеется опция BPM для этого}\EN{\tracer has the BPM option for this}.

\RU{Релоки в исполняемых PE-файлах}\EN{PE executable file relocs} (\myref{subsec:relocs}) 
\RU{не должны быть тронуты, потому что загрузчик Windows перезапишет ваш новый код.}
\EN{must not to be touched while patching, 
because the Windows loader may overwrite your new code.}
\index{Hiew}
(\RU{Они выделяются серым в Hiew, например}\EN{They are grayed in Hiew, for example}:
\figref{fig:scanf_ex3_hiew_1}).
\RU{В качестве последней меры, можно записать \JMP для обхода релока, либо же придется модифицировать таблицу
релоков.}
\EN{As a last resort, it is possible to write jumps that circumvent the relocs, 
or you will need to edit the relocs table.}

