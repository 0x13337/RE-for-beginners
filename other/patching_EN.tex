\section{Executable files patching}

\subsection{Text strings}

The C strings are the thing that is the easiest to patch (unless they are encrypted) in any hex editor.
This technique is available even for those who are not aware of machine code and executable file formats.
The new string has not to be bigger than the old one, because there's a risk of overwriting another value or code
there.
\myindex{MS-DOS}

Using this method, a lot of software was \IT{localized} in the MS-DOS era, at least in the ex-USSR countries in 80's
and 90's.
It was the reason why some weird abbreviations were present in the \IT{localized} software: there was no room for
longer strings.

\myindex{Borland Delphi}

As for Delphi strings, the string's size must also be corrected, if needed.

\subsection{x86 code}
\label{x86_patching}

Frequent patching tasks are:

\myindex{x86!\Instructions!NOP}
\begin{itemize}

\item 
One of the most frequent jobs is to disable some instruction.
It is often done by filling it using byte 
\TT{0x90} (\ac{NOP}).

\item Conditional jumps, which have an opcode like \TT{74 xx} (\JZ), 
can be filled with two \ac{NOP}s.

It is also possible to disable a conditional jump by writing 0 at the second byte (\IT{jump offset}).

\myindex{x86!\Instructions!JMP}
\item 
Another frequent job is to make a conditional jump to always trigger: 
this can be done by writing \TT{0xEB} 
instead of the opcode, which stands for \JMP.

\myindex{x86!\Instructions!RET}
\myindex{stdcall}
\item A function's execution can be disabled by writing \RETN (0xC3) at its beginning.
This is true for all functions excluding \TT{stdcall} (\myref{sec:stdcall}).
While patching \TT{stdcall} functions, one has to determine the number of arguments (for example, 
by finding \RETN in this function), 
and use \RETN with a 16-bit argument (0xC2).

\myindex{x86!\Instructions!MOV}
\myindex{x86!\Instructions!XOR}
\myindex{x86!\Instructions!INC}
\item Sometimes, a disabled functions has to return 0 or 1.
This can be done by \TT{MOV EAX, 0} or \TT{MOV EAX, 1}, 
but it's slightly verbose.\\
A better way is \TT{XOR EAX, EAX} (2 bytes \TT{0x31 0xC0}) or \TT{XOR EAX, EAX / INC EAX} (3 bytes \TT{0x31 0xC0 0x40}).

\end{itemize}

A software may be protected against modifications.

This protection is often done by reading the executable code and calculating a checksum.
Therefore, 
the code must be read before protection is triggered.

This can be determined by setting a breakpoint on reading memory.

\myindex{tracer}
\tracer has the BPM option for this.

PE executable file relocs (\myref{subsec:relocs}) 
must not to be touched while patching, 
because the Windows loader may overwrite your new code.
\myindex{Hiew}
(They are grayed in Hiew, for example:
\figref{fig:scanf_ex3_hiew_1}).

As a last resort, it is possible to write jumps that circumvent the relocs, 
or you will have to edit the relocs table.

