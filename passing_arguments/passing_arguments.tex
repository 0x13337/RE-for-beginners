\section{\IFRU{Передача параметров через стек}{Passing arguments via stack}}

\IFRU{Как мы уже успели заметить, вызывающая функция передает аргументы для вызываемой через стек. 
А как вызываемая функция имеет к ним доступ?}
{Now we figured out that caller function passing arguments to callee via stack. 
But how callee\footnote{function being called} access them?}

\lstinputlisting{passing_arguments/ex.c}

\IFRU{Имеем в итоге}{What we have after compilation} (MSVC 2010 Express):

\lstinputlisting{passing_arguments/msvc.asm}

\IFRU{Итак, здесь видно: в функции \main заталкиваются три числа в стек и вызывается 
функция \TT{f(int,int,int)}.}
{What we see is that 3 numbers are pushing to stack in function \main and \TT{f(int,int,int)} is called then.}
\IFRU{Внутри \TT{f()}, доступ к аргументам, также как и к локальным переменным, происходит через макросы: 
\TT{\_a\$ = 8}, но разница в том, что эти смещения со знаком \IT{плюс}, 
таким образом если прибавить макрос \TT{\_a\$} к указателю на \EBP, то адресуется \IT{внешняя} 
часть стека относительно \EBP.}
{Argument access inside \TT{f()} is organized with help of macros like: \TT{\_a\$ = 8}, 
in the same way as local variables accessed, but difference in that these offsets are positive 
(addressed with \IT{plus} sign).
So, adding \TT{\_a\$} macro to \EBP register value, \IT{outer} side of stack frame is addressed.}

\IFRU{Далее все более-менее просто: значение a помещается в \EAX. Далее \EAX умножается при помощи инструкции \IMUL на то что лежит в \TT{\_b}, так в \EAX остается произведение\footnote{результат умножения} этих двух значений.}
{Then \TT{a} value is stored into \EAX. After \IMUL instruction execution, \EAX value is 
product\footnote{result of multiplication} of \EAX and what is stored in \TT{\_b}.}
\IFRU{Далее к регистру \EAX прибавляется то что лежит в \TT{\_c}.}{After \IMUL execution, \ADD is 
summing \EAX and what is stored in \TT{\_c}.}
\IFRU{Значение из \EAX никуда не нужно перекладывать, оно уже лежит где надо. Возвращаем управление вызываемой 
функции ~--- она возьмет значение из \EAX и отправит его в \printf.}
{Value in \EAX is not needed to be moved: it is already in place it need. Now return to caller ~--- it 
will take value from \EAX and used it as \printf argument.}

\IFRU{Скомпилируем то же в GCC 4.4.1 и посмотрим результат в \IDA:}{Let's compile the same in GCC 4.4.1:}

\lstinputlisting{passing_arguments/gcc.asm}

\IFRU{Практически то же самое, если не считать мелких отличий описанных раннее.}{Almost the same result.}
