\chapter{%
\RU{Простейшая функция}%
\EN{The simplest Function}%
\ES{La función más simple}%
\PTBR{Brazilian portuguese text here}%
}

\RU{Наверное, простейшая из возможных функций это та что возвращает некоторую константу:}%
\EN{The simplest possible function is arguably one that simply returns a constant value:}%
\ES{La función más simple posible es sin duda uno que simplemente devuelve un valor constante:}

\RU{Вот, например}\EN{Here it is}\ES{Ejemplo}:

\lstinputlisting[caption=\EN{\CCpp Code}\RU{Код на \CCpp}]{patterns/00_ret/1.c}

\RU{Скомпилируем её!}%
\EN{Lets compile it!}%
\ES{Vamos a compilar!}

\section{x86}

\RU{И вот что делает оптимизирующий GCC}\EN{Here's what both the optimizing GCC and MSVC compilers produce on the x86 platform}%
\ES{Esto es lo que optimiza el GCC  y los compiladores de MSVC los cuales se producen en la plataforma x86}:

\lstinputlisting[caption=\Optimizing GCC/MSVC (\assemblyOutput)]{patterns/00_ret/1.s}

\index{x86!\Instructions!RET}
\RU{Здесь только две инструкции. Первая помещает значение 123 в регистр \EAX, который используется
для передачи возвращаемых значений. Вторая это \RET, которая возвращает управление в вызывающую функцию.}%
\EN{There are just two instructions: the first places the value 123 into the \EAX register, which is used by convention for storing the return
value and the second one is \RET, which returns execution to the \gls{caller}.}%
\ES{Hay solo dos instrucciones: los primeros lugares el valor 123 en el registro \EAX, que se utiliza por convención para almacenar el valor de retorno y el segundo es \RET, que devuelve la ejecución al llamado.}

\RU{Вызывающая функция возьмет результат из регистра \EAX.}%
\EN{The caller will take the result from the \EAX register.}%
\ES{La persona que llama tomar el resultado del registro \EAX.}

\ifdefined\IncludeARM
\section{ARM}

\RU{А что насчет ARM?}\EN{There are a few differences on the ARM platform:}%
\ES{Hay algunas diferencias en la plataforma ARM:}

\lstinputlisting[caption=\OptimizingKeilVI (\ARMMode) ASM Output]{patterns/00_ret/1_Keil_ARM_O3.s}

\RU{ARM использует регистр \Reg{0} для возврата значений, так что здесь 123 помещается в \Reg{0}.}%
\EN{ARM uses the register \Reg{0} for returning the results of functions, so 123 is copied into \Reg{0}.}%
\ES{ARM utiliza el registro \Reg{0} para el retorno de los resultados de las funciones, por lo que se copia 123 en \Reg{0}.}

\RU{Адрес возврата (\ac{RA}) в ARM не сохраняется в локальном стеке, а в регистре \ac{LR}.
Так что инструкция \TT{BX LR} делает переход по этому адресу, и это то же самое что и вернуть управление
в вызывающую ф-цию.}%
%Maybe explain what a link register is, or if it is just a normal register, say so?
\EN{The return address is not saved on the local stack in the ARM \ac{ISA}, but rather in the link register, 
so the \TT{BX LR} instruction causes execution to jump to that address\EMDASH{}effectively returning execution to the \gls{caller}.}%
\ES{La dirección que retorna no es guardada en la pila local de la ARM \ac{ISA}, sino más bien en el registro de enlace,
entonces la instrucción \TT{BX LR} provoca que la ejecución de un salto a la dirección eficazmente retornando la ejecución a donde fue llamada(caller ``nuestro llamador'').}
\fi

\index{ARM!\Instructions!MOV}
\index{x86!\Instructions!MOV}
\RU{Нужно отметить, что название инструкции \MOV в x86 и ARM сбивает с толку.}%
\EN{It is worth noting that \MOV is a misleading name for the instruction in both x86 and ARM \ac{ISA}s.}%
\ES{Vale la pena señalar que \MOV es un nombre engañoso para la instrucción tanto en x86 y ARM \ac{ISA}s.}

\RU{На самом деле, данные не \IT{перемещаются}, а скорее \IT{копируются}.}%
\EN{The data is not in fact \IT{moved}, but \IT{copied}.}%
\ES{Los datos de hecho, no se \IT{mueven}, sino que se \IT{copian}.}

\ifdefined\IncludeMIPS
\section{MIPS}

\label{MIPS_leaf_function_ex1}
\RU{Есть два способа называть регистры в мире MIPS. По номеру (от \$0 до \$31) или по псевдоимени (\$V0, \$A0, \etc{}.).}%
\EN{There are two naming conventions used in the world of MIPS when naming registers: by number (from \$0 to \$31) or by pseudoname (\$V0, \$A0, \etc{}).}%
\ES{Hay dos convenciones de nomenclatura utilizadas en el mundo de MIPS al nombrar registros: por número (de \$0 a \$31) o por seudónimo (\$V0, \$A0, etc).}

\RU{Вывод на ассемблере в GCC показывает регистры по номерам:}%
\EN{The GCC assembly output below lists registers by number:}%
\ES{El ensamblado de GCC produce una salida en listando los registros por numeros:}

\lstinputlisting[caption=\Optimizing GCC 4.4.5 (\assemblyOutput)]{patterns/00_ret/MIPS.s}

\RU{\dots а \IDA --- по псевдоименам:}%
\EN{\dots while \IDA does it---by their pseudonames:}%
\ES{\dots Mientras que \IDA lo hace por sus seudónimos:}

\lstinputlisting[caption=\Optimizing GCC 4.4.5 (IDA)]{patterns/00_ret/MIPS_IDA.lst}

\RU{Так что регистр \$2 (или \$V0) используется для возврата значений.}%
\EN{The \$2 (or \$V0) register is used to store the function's return value.}%
\ES{El \$2 (o \$V0) registro es usado para almacenar el valor devuelto por la función.}
\index{MIPS!\Pseudoinstructions!LI}
\RU{\INS{LI} это ``Load Immediate'', и это эквивалент \MOV в MIPS.}%
\EN{\INS{LI} stands for ``Load Immediate'' and is the MIPS equivalent to \MOV.}%
\ES{\INS{LI} significa ``Load Immediate'' (``carga inmediata'') y es el MIPS equivalente a \MOV.}

\index{MIPS!\Instructions!J}
\RU{Другая инструкция это инструкция перехода (J или JR), которая возвращает управление в \glslink{caller}{вызывающую ф-цию}, переходя по адресу в регистре \$31 (или \$RA).}%
\EN{The other instruction is the jump instruction (J or JR) which returns the execution flow to the \gls{caller}, jumping to the address in the \$31 (or \$RA) register.}%
\ES{La otra instrucción es el jump (``salto'') que es  (J o JR) el cual retorna la ejecución fluida para el llamado, da un salto a la dirección del registro \$31 (o \$RA).}
\RU{Это аналог регистра \ac{LR} в ARM.}%
\EN{This is the register analogous to \ac{LR} in ARM.}%
\ES{Este es el registro análoga a \ac{LR} en ARM.}

\index{MIPS!Branch delay slot}
\RU{Но почему инструкция загрузки (LI) и инструкция перехода (J или JR) поменены местами? Это артефакт \ac{RISC} и называется он ``branch delay slot''.}%
\EN{You might be wondering why positions of the the load instruction (LI) and the jump instruction (J or JR) are swapped. This is due to a \ac{RISC} feature called ``branch delay slot''.}%
\ES{Es posible que se pregunte por qué posiciones de la instrucción de la carga (LI) y el salto instrucciones (J y JR) se intercambian.Esto es debido a una característica llamada RISC (``ranura de retardo rama'').}

\RU{На самом деле, нам не нужно вникать в эти детали. Нужно просто запомнить: в MIPS инструкция после инструкции перехода исполняется \IT{перед} инструкцией перехода.}%
\EN{The reason this happens is a quirk in the architecture of some RISC \ac{ISA}s and isn't important for our purposes---we just need to remember that in MIPS, the instruction following a jump or branch instruction
is executed \IT{before} the jump/branch instruction itself.}%
\ES{La razón por la que esto sucede es una peculiaridad en la arquitectura RISC de algunas ISA y no es importante para nuestros propósitos --– 
sólo tenemos que recordar que en MIPS, la instrucción que sigue a un salto o instrucción de salto se ejecuta antes de la instrucción / rama salto en sí.}

\RU{Таким образом, инструкция перехода всегда поменена местами с той, которая должна быть исполнена перед ней.}%
\EN{As a consequence, branch instructions always swap places with the instruction which must be executed beforehand.}%
\ES{Como consecuencia, las instrucciones de ramificación siempre intercambiar lugares con la instrucción que debe ser ejecutado de antemano.}
% A footnote/link to http://en.wikipedia.org/wiki/Delay_slot#Branch_delay_slots or
% something similar might be useful for the people more interested in it.

\subsection{\RU{Еще кое-что об именах инструкций и регистров в MIPS}\EN{A note about MIPS instruction/register names}\ES{Una nota acerca de la instrucción MIPS nombres / Registro}}

\RU{Имена регистров и инструкций в мире MIPS традиционно пишутся в нижнем регистре.
Но мы будем использовать верхний регистр, потому что имена инструкций и регистров других \ac{ISA} в этой книге так же в верхнем регистре.}%
\EN{Register and instruction names in the world of MIPS are traditionally written in lowercase.
However, for the sake of consistency, we'll stick to using uppercase letters, as it is the convention followed by all other \ac{ISA}s featured this book.}%
\ES{Registros y nombre de instrucciones en el mundo de MIPS tradicionalmente se escriben en minúsculas.
Sin embargo, en aras de la  coherencia, que me quedo con el uso de letras mayúsculas, ya que es la convención seguida por el resto de las ISAs destacados aquí.}

\fi
