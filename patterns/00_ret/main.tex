\chapter{\RU{Простейшая функция}\EN{Simplest possible function}}

\RU{Наверное, простейшая из возможных функций это та что возвращает некоторую константу.}
\EN{Probably, the simplest possible function is that one which just returns some constant value.}

\RU{Вот, например}\EN{Here it is}:

\lstinputlisting{patterns/00_ret/1.c}

\section{x86}

\RU{И вот что делает оптимизирующий GCC}\EN{And that's what optimizing GCC compiler does}:

\lstinputlisting[caption=\Optimizing GCC]{patterns/00_ret/1.s}

\RU{Результат работы MSVC точно такой же}\EN{MSVC's result is very same}.

\index{x86!\Instructions!RET}
\RU{Здесь только две инструкции: первая помещает значение 123 в регистр \EAX, который используется
для передачи возвращаемых значений и вторая это \RET, которая возвращает управление в вызывающую ф-цию.}
\EN{There are just two instructions: the first is placing the value 123 into the \EAX register which is used for return
value passing and the second one is \RET, which returns execution to the \gls{caller}.}
\RU{Вызывающая ф-ция возьмет результат из регистра \EAX}\EN{The caller will take the result from the \EAX register}.

\section{ARM}

\RU{А что насчет}\EN{What about} ARM?

\lstinputlisting[caption=\OptimizingKeilVI (\ARMMode)]{patterns/00_ret/1_Keil_ARM_O3.s}

\RU{ARM использует регистр \Reg{0} для возврата значений, так что здесь 123 помещается в \Reg{0}.}
\EN{ARM uses \Reg{0} register for returning results, so 123 is placed into \Reg{0} here.}

\RU{Адрес возврата (\ac{RA}) в ARM не сохраняется в локальном стеке, а в регистре \ac{LR}.}
\EN{The return address (\ac{RA}) is not saved in the local stack in ARM, but rather in the \ac{LR} register.}
\RU{Так что инструкция \TT{BX LR} делает переход по этому адресу, и это то же самое что и вернуть управление
в вызывающую ф-цию.}
\EN{So the \TT{BX LR} instruction is jumping to that address, effectively, returning execution to the \gls{caller}.}

\index{ARM!\Instructions!MOV}
\index{x86!\Instructions!MOV}
\RU{Нужно отметить, что название инструкции \MOV в x86 и ARM сбивает с толку.}
\EN{It should be noted that \MOV is confusing name for instruction in both x86 and ARM \ac{ISA}s. }
\RU{На самом деле, данные не \IT{перемещаются}, а скорее \IT{копируются}.}
\EN{In fact, data is not \IT{moved}, it's rather \IT{copied}.}

\ifdefined\IncludeMIPS
\ifx\RUSSIAN\undefined
\section{MIPS}

\label{MIPS_leaf_function_ex1}
There are two ways of registers naming are used in the MIPS world. By number (from \$0 to \$31)
or by pseudoname (\$V0, \$A0, etc).
Assembly output done by GCC shows registers by number:

\lstinputlisting[caption=\Optimizing GCC 4.4.5 (\assemblyOutput)]{patterns/00_ret/MIPS.s}

\dots while IDA --- by pseudoname:

\lstinputlisting[caption=\Optimizing GCC 4.4.5 (IDA)]{patterns/00_ret/MIPS_IDA.lst}

So the \$2 (or \$V0) register is used for value returning.
\index{MIPS!\Pseudoinstructions!LI}
LI is ``Load Immediate''.

\index{MIPS!\Instructions!J}
The other instruction is jump instruction (J or JR) which returns execution flow to the \gls{caller},
jumping to the address in \$31 or \$RA.
This is the register analogous to \ac{LR} in ARM.

By why load instruction (LI) and jump instruction (J or JR) are swapped?
\index{MIPS!Branch delay slot}
This is merely \ac{RISC} artifact and called ``branch delay slot''.
Actually, we don't need to get into it. We should just memorize: the instruction after jump or branch instruction
is executed \IT{before} jump instruction.
Hence, jump instruction is always swapped with the one which should be executed before.

\subsection{Note about MIPS instruction/register names}

Register names and instruction names in MIPS world are traditionally written in lowercase.
But I decided to stick to uppercase, because instruction and register names of other \ac{ISA}s 
are all written in uppercase in this book.

\fi
\fi
