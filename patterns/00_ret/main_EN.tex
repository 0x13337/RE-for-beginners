\chapterold{The simplest Function}

The simplest possible function is arguably one that simply returns a constant value:

Here it is:

\lstinputlisting[caption=\EN{\CCpp Code}]{patterns/00_ret/1.c}

Lets compile it!

\sectionold{x86}

Here's what both the optimizing GCC and MSVC compilers produce on the x86 platform:

\lstinputlisting[caption=\Optimizing GCC/MSVC (\assemblyOutput)]{patterns/00_ret/1.s}

\myindex{x86!\Instructions!RET}
There are just two instructions: the first places the value 123 into the \EAX register, 
which is used by convention for storing the return
value and the second one is \RET, which returns execution to the \gls{caller}.

The caller will take the result from the \EAX register.

\sectionold{ARM}

There are a few differences on the ARM platform:

\lstinputlisting[caption=\OptimizingKeilVI (\ARMMode) ASM Output]{patterns/00_ret/1_Keil_ARM_O3.s}

ARM uses the register \Reg{0} for returning the results of functions, so 123 is copied into \Reg{0}.

%Maybe explain what a link register is, or if it is just a normal register, say so?
The return address is not saved on the local stack in the ARM \ac{ISA}, but rather in the link register, 
so the \INS{BX LR} instruction causes execution to jump to that address\EMDASH{}effectively returning execution to the \gls{caller}.

\myindex{ARM!\Instructions!MOV}
\myindex{x86!\Instructions!MOV}
It is worth noting that \MOV is a misleading name for the instruction in both x86 and ARM \ac{ISA}s.

The data is not in fact \IT{moved}, but \IT{copied}.

\sectionold{MIPS}

\label{MIPS_leaf_function_ex1}
There are two naming conventions used in the world of MIPS when naming registers: by number (from \$0 to \$31) or by pseudoname (\$V0, \$A0, \etc{}).

The GCC assembly output below lists registers by number:

\lstinputlisting[caption=\Optimizing GCC 4.4.5 (\assemblyOutput)]{patterns/00_ret/MIPS.s}

\dots while \IDA does it---by their pseudonames:

\lstinputlisting[caption=\Optimizing GCC 4.4.5 (IDA)]{patterns/00_ret/MIPS_IDA.lst}

The \$2 (or \$V0) register is used to store the function's return value.
\myindex{MIPS!\Pseudoinstructions!LI}
\INS{LI} stands for ``Load Immediate'' and is the MIPS equivalent to \MOV.

\myindex{MIPS!\Instructions!J}
The other instruction is the jump instruction (J or JR) which returns the execution flow to the \gls{caller}, jumping to the address in the \$31 (or \$RA) register.

This is the register analogous to \ac{LR} in ARM.

\myindex{MIPS!Branch delay slot}
You might be wondering why positions of the the load instruction (LI) and the jump instruction (J or JR) are swapped. This is due to a \ac{RISC} feature called ``branch delay slot''.

The reason this happens is a quirk in the architecture of some RISC \ac{ISA}s and isn't important for our purposes---we just need to remember that in MIPS, the instruction following a jump or branch instruction
is executed \IT{before} the jump/branch instruction itself.

As a consequence, branch instructions always swap places with the instruction which must be executed beforehand.
% A footnote/link to http://en.wikipedia.org/wiki/Delay_slot#Branch_delay_slots or
% something similar might be useful for the people more interested in it.

\subsectionold{A note about MIPS instruction/register names}

Register and instruction names in the world of MIPS are traditionally written in lowercase.
However, for the sake of consistency, we'll stick to using uppercase letters,
as it is the convention followed by all other \ac{ISA}s featured this book.

