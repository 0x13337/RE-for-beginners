\chapter{A mais simples função}

A função mais simples possível é indiscutivelmente aquela que simplesmente retorna um valor constante:

Aqui está:

\lstinputlisting[caption=\PTBRph{}]{patterns/00_ret/1.c}

Vamos compilar!

\section{x86}

Aqui está o que ambos compiladores, GCC com otimização e MSVC produzem na plataforma x86:

\lstinputlisting[caption=\Optimizing GCC/MSVC (\assemblyOutput)]{patterns/00_ret/1.s}

\index{x86!\Instructions!RET}
Há somente duas instruções: a primeira coloca o valor 123 no registrador \EAX, que é usado por convenção para guardar o valor de retorno e a segunda é a \RET, que retorna a execução para onde a função foi chamada.

O resultado será obtido no registrador \EAX.

% TO translate: about ARM, MIPS

