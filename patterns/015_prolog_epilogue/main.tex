\ifdefined\ENGLISH
\chapter{Function prologue and epilogue}
\label{sec:prologepilog}
\index{Function epilogue}
\index{Function prologue}

A function prologue is a sequence of instructions at the start of a function. It often looks something like the following code fragment:

\begin{lstlisting}
    push    ebp
    mov     ebp, esp
    sub     esp, X
\end{lstlisting}

What these instruction do: save the value in the \EBP register,
set the value of the \EBP register to the value of the \ESP and then allocate space on the stack 
for local variables.

The value in the \EBP stays the same over the period of the function execution and is to be used for local variables and 
arguments access. 
For the same purpose one can use \ESP, but since it changes over time this approach is not too convenient.

The function epilogue frees the allocated space in the stack, returns the value in the \EBP register back to its initial state 
and returns the control flow to the \gls{callee}:

\begin{lstlisting}
    mov    esp, ebp
    pop    ebp
    ret    0
\end{lstlisting}

% what about calling convention?
Function prologues and epilogues are usually detected in disassemblers for function delimitation.

\ifx\LITE\undefined
\section{\Recursion}

\index{\Recursion}
Epilogues and prologues can negatively affect the recursion performance.

More about recursion in this book: \myref{Recursion_and_tail_call}.
\fi % LITE
\fi % ENGLISH

\ifdefined\RUSSIAN
\chapter{Пролог и эпилог функций}
\label{sec:prologepilog}
\index{Function epilogue}
\index{Function prologue}

Пролог функции это инструкции в самом начале функции. Как правило это что-то вроде такого фрагмента кода:

\begin{lstlisting}
    push    ebp
    mov     ebp, esp
    sub     esp, X
\end{lstlisting}

Эти инструкции делают следующее: сохраняют значение регистра \EBP на будущее, выставляют \EBP равным \ESP,
затем подготавливают место в стеке для хранения локальных переменных.

\EBP сохраняет свое значение на протяжении всей функции, он будет использоваться здесь для доступа 
к локальным переменным и аргументам. Можно было бы использовать и \ESP, но он постоянно меняется и 
это не очень удобно.

Эпилог функции аннулирует выделенное место в стеке, восстанавливает значение \EBP на старое и возвращает 
управление в вызывающую функцию:

\begin{lstlisting}
    mov    esp, ebp
    pop    ebp
    ret    0
\end{lstlisting}

% what about calling convention?
Пролог и эпилог функции обычно находятся в дизассемблерах для отделения функций друг от друга.

\ifx\LITE\undefined
\section{\Recursion}

\index{\Recursion}
Наличие эпилога и пролога может несколько ухудшить эффективность рекурсии.

Больше о рекурсии в этой книге: \myref{Recursion_and_tail_call}.
\fi % LITE
\fi % RUSSIAN

\ifdefined\BRAZILIAN
\chapter{Cabeçalhos e rodapés de funções}
\label{sec:prologepilog}
\index{\PTBRph{}} % Function epilogue
\index{\PTBRph{}} % Function prologue

Um cabeçalho de uma função é uma sequência de instruções no começo da função. Ele geralmente se parece com algo como o código a seguir:

\begin{lstlisting}
    push    ebp
    mov     ebp, esp
    sub     esp, X
\end{lstlisting}

O que essas instruções fazem: salvam o valor no registrador \EBP, mudam o valor de \EBP para o valor em \ESP e então aloca espaço na pilha para variáveis locais.

O valor em \EBP continua o mesmo depois do período de execução da função e é para ser usado para variáveis locais e acessos de argumentos.
Para o mesmo propósito pode ser usado o \ESP, mas considerando que ele muda com o tempo, essa abordagem não é muito conveniente.

O rodapé da função libera o espaço alocado na pilha, retorna o valor do registrador \EBP de volta ao seu estado inicial e retorna o controle de volta para a chamada da função:

\begin{lstlisting}
    mov    esp, ebp
    pop    ebp
    ret    0
\end{lstlisting}

Os cabeçalhos e rodapés das funções geralmente são detectados na desassemblagem para a delimitação das funções.

\ifx\LITE\undefined
\section{\Recursion}

\ac{TBT}

\index{\Recursion}
Epilogues and prologues can negatively affect the recursion performance.

More about recursion in this book: \myref{Recursion_and_tail_call}.
\fi % LITE
\fi % BRAZILIAN

\ifdefined\SPANISH
\chapter{Prologo y epilogo de funciones}
\label{sec:prologepilog}
\index{\ESph{}} % Function epilogue
\index{\ESph{}} % Function prologue

El prologo de una funcion es una secuencia de instrucciones al inicio de esta.
Por lo general luce mas o menos como el siguiente fragmento de codigo:

\begin{lstlisting}
    push    ebp
    mov     ebp, esp
    sub     esp, X
\end{lstlisting}

Lo que estas instrucciones hacen es: guardan el valor en el registro \EBP, establece el valor del registro \EBP al valor del registro \ESP y luego asigna espacio en la pila para variables locales.

El valor de \EBP permanece igual durante el periodo de ejecucion de la funcion, y es usado para variables locales y acceso a los argumentos.
Para los mismos fines uno puede usar \ESP, pero como este cambia con el tiempo, este enfoque no es muy conveniente.

El epilogo de la funcion libera el espacio asignado en la pila, coloca el valor del registro \EBP vuelta su estado inicial y retorna el control de flujo a la la funcion llamada:

\begin{lstlisting}
    mov    esp, ebp
    pop    ebp
    ret    0
\end{lstlisting}

Los prologos y epilogos de funciones usualmente son detectados por desensambladores para la delimitacion de funciones.

\ifx\LITE\undefined
\section{\Recursion}
\index{\Recursion}

Epilogos y prologos pueden afectar negativamente el rendimiento de la recursion.
Mas acerca de la recursion en este libro: \myref{Recursion_and_tail_call}.

\fi % LITE
\fi % SPANISH

