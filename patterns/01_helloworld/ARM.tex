\section{ARM}
\label{sec:hw_ARM}

\index{\idevices}
\index{Raspberry Pi}
\index{Xcode}
\index{LLVM}
\index{Keil}
\RU{Для экспериментов с процессором ARM я выбрал два компилятора}
\EN{For my experiments with ARM processors I chose two compilers}: 
\RU{популярный в embedded-среде}\EN{popular in the embedded area} Keil Release 6/2013 
\RU{и среду разработки}\EN{and} Apple Xcode 4.6.3 \EN{IDE} (\RU{с компилятором}\EN{with} LLVM-GCC 4.2 \EN{compiler}
\footnote{\EN{It is indeed so: Apple Xcode 4.6.3 uses open-source GCC as front-end compiler and LLVM 
code generator}\RU{Это действительно так: Apple Xcode 4.6.3 использует опен-сорсный GCC как компилятор
переднего плана и коде-генератор LLVM}}), 
\RU{генерирующую код для ARM-совместимых процессоров и}\EN{which produces code for ARM-compatible processors and} \ac{SOC} \InENRU \idevices, 
\RU{планшетных компьютеров для Windows 8 и Windows RT}\EN{Windows 8 and Window RT tables}\footnote{\url{http://en.wikipedia.org/wiki/List_of_Windows_8_and_RT_tablet_devices}} 
\RU{и таких устройствах как}\EN{and also such devices as} Raspberry Pi.

\RU{Везде в этой книге, кроме как если указано иное, идет речь о 32-битном ARM}
\EN{32-bit ARM code is used in all cases in this book, if not mentioned otherwise}.

\subsection{\NonOptimizingKeil + \ARMMode}

\RU{Для начала скомпилируем наш пример в Keil}\EN{Let's start by compiling our example in Keil}:

\begin{lstlisting}
armcc.exe --arm --c90 -O0 1.c 
\end{lstlisting}

\index{\IntelSyntax}
\RU{Компилятор \IT{armcc} генерирует листинг на ассемблере в формате Intel,}
\EN{The \IT{armcc} compiler produces assembly listings in Intel-syntax} 
\RU{но он содержит некоторые высокоуровневые макросы, связанные с ARM}
\EN{but it has high-level ARM-processor related macros}\footnote{
\RU{например, он показывает инструкции \PUSH/\POP, отсутствующие в режиме ARM}
\EN{e.g. ARM mode lacks \PUSH/\POP instructions}}, 
\RU{а нам важнее увидеть инструкции ``как есть'', так что посмотрим скомпилированный результат в \IDA}
\EN{but it is more important for us to see the instructions ``as is'' so let's see the compiled result in \IDA}.

\begin{lstlisting}[caption=\NonOptimizingKeil + \ARMMode + \IDA]
.text:00000000             main
.text:00000000 10 40 2D E9    STMFD   SP!, {R4,LR}
.text:00000004 1E 0E 8F E2    ADR     R0, aHelloWorld ; "hello, world"
.text:00000008 15 19 00 EB    BL      __2printf
.text:0000000C 00 00 A0 E3    MOV     R0, #0
.text:00000010 10 80 BD E8    LDMFD   SP!, {R4,PC}

.text:000001EC 68 65 6C 6C+aHelloWorld  DCB "hello, world",0    ; DATA XREF: main+4
\end{lstlisting}

\index{ARM!\ARMMode}
\index{ARM!\ThumbMode}
\index{ARM!\ThumbTwoMode}
\RU{Вот чуть-чуть фактов о процессоре ARM, которые желательно знать}
\EN{Here are a couple of ARM-related facts that we should know in order to proceed}.
\RU{Процессор ARM имеет по крайней мере два основных режима: режим ARM и thumb}
\EN{An ARM processor has at least two major modes: ARM mode and thumb mode}. 
\RU{В первом (ARM) режиме доступны все инструкции и каждая имеет размер 32 бита (или 4 байта)}
\EN{In the first (ARM) mode, all instructions are enabled and each is 32 bits (4 bytes) in size}. 
\RU{Во втором режиме (thumb) каждая инструкция имеет размер 16 бит (или 2 байта)}
\EN{In the second (thumb) mode each instruction is 16 bits (2 bytes) in size}
\footnote{\RU{Кстати, инструкции фиксированного размера удобны тем, что всегда можно легко узнать адрес 
следующей (или предыдущей) инструкции. Эта особенность будет рассмотрена в секции о 
switch()~(\ref{sec:SwitchARMLot}).}
\EN{By the way, fixed-length instructions are handy in a way that one can calculate the next (or previous) 
instruction's address without effort. This feature will be discussed in switch()~(\ref{sec:SwitchARMLot}) section.}
}.
\RU{Режим thumb может выглядеть привлекательнее тем, что программа на нем может быть 1) компактнее; 2) эффективнее исполняться на микроконтроллере с 16-битной шиной данных}
\EN{Thumb mode may look attractive because programs that use it may 1) be compact and
2) execute faster on microcontrollers having a 16-bit memory datapath}. 
\RU{Но за всё нужно платить: в режиме thumb куда меньше возможностей процессора, например, возможен доступ только к 8-и регистрам процессора, и, чтобы совершить некоторые действия, выполнимые в режиме ARM одной инструкцией, нужны несколько thumb-инструкций}
\EN{Nothing comes for free. In thumb mode, there is a reduced instruction set, 
only 8 registers are accessible and one needs several thumb instructions for doing some operations when you only need one in ARM mode}.

\RU{Начиная с ARMv7, имеется также поддержка инструкций thumb-2. Это thumb, расширенный до поддержки куда большего числа инструкций}
\EN{Starting from ARMv7 the thumb-2 instruction set is also available. 
This is an extended thumb mode that supports a much larger instruction set}.
\RU{Распространено заблуждение, что thumb-2 ~--- это смесь ARM и thumb. Это не верно. Просто thumb-2 был дополнен до
более полной поддержки возможностей процессора, что теперь может легко конкурировать с режимом ARM.}
\EN{There is a common misconception that thumb-2 is a mix of ARM and thumb. This is not correct. 
Rather, thumb-2 was extended to fully support processor features so it could
compete with ARM mode.}
\RU{Программа для процессора ARM может представлять смесь процедур, скомпилированных для обоих режимов}
\EN{A program for the ARM processor may be a mix of procedures compiled for both modes}.
\RU{Основное количество приложений для \idevices скомпилировано для набора инструкций thumb-2, потому что Xcode
делает так по умолчанию}
\EN{The majority of \idevices applications are compiled for the thumb-2 instruction set because Xcode does this by default}.

\RU{В вышеприведённом примере можно легко увидеть, что каждая инструкция имеет размер 4 байта}
\EN{In the example we can easily see each instruction has a size of 4 bytes}.
\RU{Действительно, ведь мы же компилировали наш код для режима ARM а не thumb}
\EN{Indeed, we compiled our code for ARM mode, not for thumb}.

\index{ARM!\Instructions!STMFD}
\index{ARM!\Instructions!POP}
\RU{Самая первая инструкция}\EN{The very first instruction}, \TT{``STMFD SP!, \{R4,LR\}''}\footnote{\ac{STMFD}}, 
\RU{работает как инструкция}\EN{works as an x86} \PUSH \RU{в x86}\EN{instruction},
\RU{записывая значения двух регистров}\EN{writing the values of two registers}
(\Reg{4} \AndENRU \ac{LR}) \RU{в стек}\EN{into the stack}.
\RU{Действительно, в выдаваемом листинге на ассемблере компилятор \IT{armcc}, для упрощения, указывает здесь инструкцию}
\EN{Indeed, in the output listing from the \IT{armcc} compiler, for the sake of simplification, 
actually shows the} \TT{``PUSH \{r4,lr\}''}\EN{ instruction}.
\RU{Но это не совсем точно, инструкция \PUSH доступна только в режиме thumb, поэтому,
во избежание путаницы, я предложил работать в \IDA}
\EN{But it is not quite correct. The \PUSH instruction is only available in thumb mode.
So, to make things less messy, that is why I suggested working in \IDA}.

\RU{Итак, эта инструкция уменьшает \ac{SP}, чтобы он указывал на место в стеке, свободное для записи
новых значений, затем записывает значения регистров \Reg{4} и \ac{LR} 
по адресу в памяти, на который указывает измененный регистр \ac{SP}}
\EN{This instruction first \glspl{decrement} \ac{SP} so it will point to the place in the stack
that is free for new entries, then it writes the values of the \Reg{4} and \ac{LR} registers at the address
in changed \ac{SP}}.

\RU{Эта инструкция, как и инструкция \PUSH в режиме thumb, может сохранить в стеке одновременно несколько значений регистров, что может быть очень удобно}
\EN{This instruction (like the \PUSH instruction in thumb mode) is able to save several register values at once and this may be useful}. 
\RU{Кстати, такого в x86 нет}\EN{By the way, there is no such thing in x86}.
\RU{Также следует заметить, что \TT{STMFD} ~--- генерализация инструкции \PUSH (то есть, расширяет её возможности), потому что может работать с любым регистром, а не только с \ac{SP}, это тоже может быть очень удобно}
\EN{It can also be noted that the \TT{STMFD} instruction is a generalization 
of the \PUSH instruction (extending its features), since it can work with any register, not just with \ac{SP}, and this can be very useful}.

\index{\PICcode}
\index{ARM!\Instructions!ADR}
\RU{Инструкция}\EN{The} \TT{``ADR R0, aHelloWorld''}
\RU{прибавляет значение регистра \ac{PC} к смещению, где хранится строка}
\EN{instruction adds the value in the \ac{PC} register to the offset where the}
\IT{``hello, world''}\EN{ string is located}.
\RU{Причем здесь \ac{PC}, можно спросить}\EN{How is the \TT{PC} register used here, one might ask}?
\RU{Притом, что это так называемый ``\PICcode''}\EN{This is so-called ``\PICcode''.}
\footnote{
	\RU{Читайте больше об этом в соответствующем разделе}
	\EN{Read more about it in relevant section}~(\ref{sec:PIC})
	}
\RU{он предназначен для исполнения будучи не привязанным к каким-либо адресам в памяти}
\EN{It is intended to be executed at a non-fixed address in memory}.
\RU{В опкоде инструкции \TT{ADR} указывается разница между адресом этой инструкции и местом, где хранится строка}
\EN{In the opcode of the \TT{ADR} instruction, the difference between the address of this instruction and the place where the string is located is encoded}.
\RU{Эта разница всегда будет постоянной, вне зависимости от того, куда был загружен \ac{OS}
наш код}
\EN{The difference will always be the same,
independent of the address where the code is loaded by the \ac{OS}}.
\RU{Поэтому всё, что нужно ~--- это прибавить адрес текущей инструкции (из \ac{PC}), чтобы получить текущий абсолютный адрес нашей Си-строки}
\EN{That's why all we need is to add the address of the current instruction (from \ac{PC}) in order to get the absolute address of our C-string in memory}.

\index{ARM!\Registers!Link Register}
\index{ARM!\Instructions!BL}
\RU{Инструкция} \TT{``BL \_\_2printf''}\footnote{Branch with Link}
\RU{вызывает функцию \printf}\EN{instruction calls the \printf function}. 
\RU{Работа этой инструкции состоит из двух фаз}
\EN{Here's how this instruction works}: 
\begin{itemize}
\item
\RU{записать адрес после инструкции \TT{BL} (\TT{0xC}) в регистр \ac{LR}}
\EN{write the address following the \TT{BL} instruction (\TT{0xC}) into the \ac{LR}};
\item
\RU{затем собственно передать управление в \printf, записав адрес этой функции в регистр \ac{PC}}
\EN{then pass control flow into \printf by writing its address into the \ac{PC} register}.
\end{itemize}

\RU{Ведь когда функция \printf закончит работу, нужно знать, куда вернуть управление, поэтому закончив работу, всякая функция передает управление по адресу, записанному в регистре \ac{LR}}
\EN{When \printf finishes its work it must have information about where it must return control.
That's why each function passes control to the address stored in the \ac{LR} register}.

\RU{В этом разница между ``чистыми'' \ac{RISC}-процессорами вроде ARM и \ac{CISC}-процессорами как x86,
где адрес возврата записывается в стек}
\EN{That is the difference between ``pure'' \ac{RISC}-processors like ARM and \ac{CISC}-processors like x86,
where the return address is stored on the stack}\footnote{\RU{Подробнее об этом будет описано в следующей главе}\EN{Read more about this in next section}~(\ref{sec:stack})}.

\RU{Кстати, 32-битный абсолютный адрес, либо же смещение, невозможно закодировать в 32-битной инструкции \TT{BL}, в ней есть место только для 24-х бит}
\EN{By the way, an absolute 32-bit address or offset cannot be encoded in the 32-bit \TT{BL} instruction because
it only has space for 24 bits}.
\RU{Также следует отметить, что из-за того, что все инструкции в режиме ARM имеют длину 4 байта (32 бита), и инструкции могут находится только по адресам кратным 4, то последние 2 бита (всегда нулевых) можно не кодировать.}
\EN{It is also worth noting all ARM-mode instructions have a size of 4 bytes (32 bits).
Hence they can only be located on 4-byte boundary addresses.
This means the the last 2 bits of the instruction address (which are always zero bits) may be omitted.}
\RU{В итоге имеем 26 бит, при помощи которых, можно закодировать смещение}
\EN{In summary, we have 26 bits for offset encoding. This is enough to represent offset} $\pm{}\approx{}32M$.

\index{ARM!\Instructions!MOV}
\RU{Следующая инструкция}\EN{Next, the} \TT{``MOV R0, \#0''}\footnote{MOVe}
\RU{просто записывает $0$ в регистр \Reg{0}}\EN{instruction just writes $0$ into the \Reg{0} register}.
\RU{Ведь наша Си-функция возвращает $0$, а возвращаемое значение всякая функция оставляет в \Reg{0}}
\EN{That's because our C-function returns $0$ and the return value is to be placed in the \Reg{0} register}.

\index{ARM!\Registers!Link Register}
\index{ARM!\Instructions!LDMFD}
\index{ARM!\Instructions!POP}
\RU{Последняя инструкция  ~---}\EN{The last instruction} \TT{``LDMFD SP!, {R4,PC}''}\footnote{\ac{LDMFD}} \RU{это инструкция, обратная от}\EN{is an inverse instruction of} \TT{STMFD}. 
\RU{Она загружает из стека значения для сохранения их в \Reg{4} и \ac{PC}, увеличивая \glslink{stack pointer}{указатель стека} \ac{SP}}
\EN{It loads values from the stack in order to save them into \Reg{4} and \ac{PC}, and \glslink{increment}{increments} the \gls{stack pointer} \ac{SP}}.
\RU{Это, в каком-то смысле, аналог \POP}\EN{It can be said that it is similar to \POP}. 
N.B. \RU{Самая первая инструкция \TT{STMFD} сохранила в стеке \Reg{4} и \ac{LR}, а \IT{восстанавливаются} во время исполнения \TT{LDMFD} регистры \Reg{4} и \ac{PC}}
\EN{The very first instruction \TT{STMFD} saves the \Reg{4} and \ac{LR} registers pair on the stack, but \Reg{4} and \ac{PC} are \IT{restored} during execution of \TT{LDMFD}}.

\RU{Как я уже описывал, в регистре \ac{LR} обычно сохраняется адрес места, куда нужно всякой функции вернуть управление}
\EN{As I wrote before, the address of the place to where each function must return control is usually saved in the \ac{LR}
register}.
\RU{Самая первая инструкция сохраняет это значение в стеке, потому что наша функция \main позже будет сама пользоваться этим регистром, в момент вызова \printf}
\EN{The very first function saves its value in the stack because our
\main function will use the register in order to call \printf}.
\RU{А затем, в конце функции, это значение можно сразу записать в \ac{PC}, таким образом, передав управление туда, откуда была вызвана наша функция}
\EN{In the function end this value can be written to the \ac{PC} register, thus passing control to where our function was called}.
\RU{Так как функция \main обычно самая главная в \CCpp, управление будет возвращено в загрузчик \ac{OS}, либо куда-то в \ac{CRT} 
или что-то в этом роде}
\EN{Since our \main function is usually the primary function in \CCpp,
control will be returned to the \ac{OS} loader or to a point in \ac{CRT},
or something like that}.

\index{ARM!DCB}
\TT{DCB}\RU{ ~--- директива ассемблера, описывающая массивы байт или ASCII-строк, аналог директивы DB в 
x86-ассемблере}
\EN{~is an assembly language directive defining an array of bytes or ASCII strings, akin to the DB directive 
in x86-assembly language}.

\subsection{\NonOptimizingKeil: \ThumbMode}

\RU{Скомпилируем тот же пример в Keil для режима thumb}\EN{Let's compile the same example using Keil in thumb mode}:

\begin{lstlisting}
armcc.exe --thumb --c90 -O0 1.c 
\end{lstlisting}

\RU{Получим (в \IDA)}\EN{We will get (in \IDA)}:

\begin{lstlisting}[caption=\NonOptimizingKeil + \ThumbMode + \IDA]
.text:00000000             main
.text:00000000 10 B5          PUSH    {R4,LR}
.text:00000002 C0 A0          ADR     R0, aHelloWorld ; "hello, world"
.text:00000004 06 F0 2E F9    BL      __2printf
.text:00000008 00 20          MOVS    R0, #0
.text:0000000A 10 BD          POP     {R4,PC}

.text:00000304 68 65 6C 6C+aHelloWorld  DCB "hello, world",0    ; DATA XREF: main+2
\end{lstlisting}

\RU{Сразу бросаются в глаза двухбайтные (16-битные) опкоды ~--- это, как я уже упоминал, thumb}
\EN{We can easily spot the 2-byte (16-bit) opcodes. This is, as I mentioned, thumb}.
\index{ARM!\Instructions!BL}
\RU{Кроме инструкции \TT{BL}.}\EN{The \TT{BL} instruction however}
\RU{Но на самом деле она состоит из двух 16-битных инструкций}
\EN{consists of two 16-bit instructions}.
\RU{Это потому, что загрузить в \ac{PC} смещение, по которому находится функция \printf, используя так мало места в одном 16-битном опкоде, нельзя}
\EN{This is because it is impossible to load an offset for the \printf function into \ac{PC} while using the small space in one 16-bit opcode}.
\RU{Поэтому первая 16-битная инструкция загружает старшие 10 бит смещения, а вторая ~--- младшие 11 бит смещения}
\EN{That's why the first 16-bit instruction loads the higher 10 bits of the offset and the second instruction loads 
the lower 11 bits of the offset}.
\RU{Как я уже упоминал, все инструкции в thumb-режиме имеют длину 2 байта (или 16 бит)}
\EN{As I mentioned, all instructions in thumb mode have a size of 2 bytes (or 16 bits)}.
\RU{Поэтому невозможна такая ситуация, когда thumb-инструкция начинается по нечетному адресу}
\EN{This means it is impossible for a thumb-instruction to be at an odd address whatsoever}.
\RU{Учитывая сказанное, последний бит адреса можно не кодировать}
\EN{Given the above, the last address bit may be omitted while encoding instructions}.
\RU{Таким образом, в итоге, в thumb-инструкции \TT{BL} кодируется смещение}
\EN{In summary, in the \TT{BL} thumb-instruction} $\pm{}\approx{}2M$ 
\RU{от текущего адреса}\EN{can be encoded as the offset from the current address}.

\RU{Остальные инструкции в функции: \PUSH и \POP работают почти так же, как и описанные \TT{STMFD}/\TT{LDMFD}, только регистр \ac{SP} здесь не указывается явно}
\EN{As for the other instructions in the function: \PUSH and \POP work just like the described \TT{STMFD}/\TT{LDMFD} but the \ac{SP} register is not mentioned explicitly here}.
\TT{ADR} \RU{работает так же, как и в предыдущем примере}\EN{works just like in previous example}.
\TT{MOVS} \RU{записывает $0$ в регистр \Reg{0} для возврата нуля}
\EN{writes $0$ into the \Reg{0} register in order to return zero}.

\subsection{\OptimizingXcode + \ARMMode}

Xcode 4.6.3 \RU{без включенной оптимизации выдает слишком много лишнего кода, поэтому остановимся на той версии, где как можно меньше инструкций}
\EN{without optimization turned on produces a lot of redundant code so we'll study the version where the instruction count is as small as possible}: \Othree.

\begin{lstlisting}[caption=\OptimizingXcode + \ARMMode]
__text:000028C4             _hello_world
__text:000028C4 80 40 2D E9   STMFD           SP!, {R7,LR}
__text:000028C8 86 06 01 E3   MOV             R0, #0x1686
__text:000028CC 0D 70 A0 E1   MOV             R7, SP
__text:000028D0 00 00 40 E3   MOVT            R0, #0
__text:000028D4 00 00 8F E0   ADD             R0, PC, R0
__text:000028D8 C3 05 00 EB   BL              _puts
__text:000028DC 00 00 A0 E3   MOV             R0, #0
__text:000028E0 80 80 BD E8   LDMFD           SP!, {R7,PC}

__cstring:00003F62 48 65 6C 6C+aHelloWorld_0  DCB "Hello world!",0
\end{lstlisting}

\RU{Инструкции}\EN{The instructions} \TT{STMFD} \AndENRU \TT{LDMFD} \RU{нам уже знакомы}\EN{are familiar to us}.

\RU{Инструкция \MOV просто записывает число \TT{0x1686} в регистр \Reg{0} ~--- это смещение, указывающее на строку ``Hello world!''}
\EN{The \MOV instruction just writes the number \TT{0x1686} into the \Reg{0} register.
This is the offset pointing to the ``Hello world!'' string}.

\RU{Регистр \TT{R7}, по стандарту, принятому в}\EN{The \TT{R7} register as it is standardized in}\cite{IOSABI}
\RU{ ~--- это}\EN{is a} frame pointer\RU{ , о нем будет рассказано позже.}\EN{. More on it below.}

\index{ARM!\Instructions!MOVT}
\RU{Инструкция}\EN{The} \TT{MOVT R0, \#0} \RU{записывает $0$ в старшие 16 бит регистра}
\EN{instruction writes $0$ into higher 16 bits of the register}.
\RU{Дело в том, что обычная инструкция \MOV в режиме ARM может записывать какое-либо значение только в младшие 16 бит регистра, ведь в ней нельзя закодировать больше}
\EN{The issue here is that the generic \MOV instruction in ARM mode may write only the lower 16 bits of the register}.
\RU{Помните, что в режиме ARM опкоды всех инструкций ограничены длиной в 32 бита. Конечно, это ограничение не касается перемещений между регистрами.}
\EN{Remember, all instruction opcodes in ARM mode are limited in size to 32 bits. Of course, this limitation is not related to moving between registers.}
\RU{Поэтому для записи в старшие биты (от 16-го по 31-го включительно) существует дополнительная команда \TT{MOVT}}
\EN{That's why an additional instruction \TT{MOVT} exists for writing into the higher bits (from 16 to 31 inclusive)}.
\RU{Впрочем, здесь её использование избыточно, потому что инструкция \TT{``MOV R0, \#0x1686''} выше и так обнулила старшую часть регистра}
\EN{However, its usage here is redundant because the \TT{``MOV R0, \#0x1686''} instruction above cleared the higher part of the register}.
\RU{Возможно, это недочет компилятора}\EN{This is probably a shortcoming of the compiler}.

\index{ARM!\Instructions!ADD}
\RU{Инструкция}\EN{The} \TT{``ADD R0, PC, R0''} \RU{прибавляет \ac{PC} к \Reg{0} для вычисления действительного адреса строки ``Hello world!''. Как нам уже известно, это ``\PICcode'', поэтому такая корректива необходима}
\EN{instruction adds the value in the \ac{PC} to the value in the \Reg{0}, to calculate absolute address of the ``Hello world!'' string. 
As we already know, it is ``\PICcode'' so this correction is essential here}.

\RU{Инструкция \TT{BL} вызывает \puts вместо \printf}
\EN{The \TT{BL} instruction calls the \puts function instead of \printf}.

\label{puts}
\index{puts() \RU{вместо}\EN{instead of} printf()}
\RU{Компилятор заменил вызов \printf на \puts. 
Действительно, \printf с одним аргументом это почти аналог \puts.}
\EN{GCC replaced the first \printf call with \puts.
Indeed: \printf with a sole argument is almost analogous to \puts.} 

\RU{\IT{Почти}, если принять условие, что в строке не будет управляющих символов \printf, 
начинающихся со знака процента. Тогда эффект от работы этих двух функций будет разным}
\EN{\IT{Almost} because we need to be sure the string will not contain printf-control 
statements starting with \IT{\%}: then the effect of these two functions would be different}
\footnote{
\RU{Также нужно заметить, что \puts не требует символа перевода строки '\textbackslash{}n' в конце строки,
поэтому его здесь нет.}
\EN{It should also be noted the \puts does not require a '\textbackslash{}n' new line symbol 
at the end of a string, so we do not see it here.}}.

\RU{Зачем компилятор заменил один вызов на другой? Потому что \puts работает быстрее}
\EN{Why did the compiler replace the \printf with \puts? Because \puts is faster}
\footnote{\url{http://www.ciselant.de/projects/gcc_printf/gcc_printf.html}}. 

\RU{Видимо потому, что \puts проталкивает символы в stdout не сравнивая каждый со знаком процента.}
\EN{\puts works faster because it just passes characters to stdout without comparing each to the \IT{\%} symbol.}

\RU{Далее уже знакомая инструкция}\EN{Next, we see the familiar} 
\TT{``MOV R0, \#0''}\RU{, служащая для установки в $0$ возвращаемого значения функции}
\EN{instruction intended to set the \Reg{0} register to $0$}.

\subsection{\OptimizingXcode + \ThumbTwoMode}

\RU{По умолчанию,}\EN{By default} Xcode 4.6.3 
\RU{генерирует код для режима thumb-2 примерно в такой манере}
\EN{generates code for thumb-2 in this manner}:

\begin{lstlisting}[caption=\OptimizingXcode + \ThumbTwoMode]
__text:00002B6C                   _hello_world
__text:00002B6C 80 B5          PUSH            {R7,LR}
__text:00002B6E 41 F2 D8 30    MOVW            R0, #0x13D8
__text:00002B72 6F 46          MOV             R7, SP
__text:00002B74 C0 F2 00 00    MOVT.W          R0, #0
__text:00002B78 78 44          ADD             R0, PC
__text:00002B7A 01 F0 38 EA    BLX             _puts
__text:00002B7E 00 20          MOVS            R0, #0
__text:00002B80 80 BD          POP             {R7,PC}

...

__cstring:00003E70 48 65 6C 6C 6F 20+aHelloWorld  DCB "Hello world!",0xA,0
\end{lstlisting}

\index{ThumbTwoMode}
\RU{Инструкции \TT{BL} и \TT{BLX} в thumb, как мы помним, кодируются как пара 16-битных инструкций, 
а в thumb-2 эти \IT{суррогатные} опкоды расширены так, что новые инструкции кодируются здесь как 
32-битные инструкции}
\EN{The \TT{BL} and \TT{BLX} instructions in thumb mode, as we recall, are encoded as a pair
of 16-bit instructions.
In thumb-2 these \IT{surrogate} opcodes are extended in such a way so that new instructions
may be encoded here as 32-bit instructions}.
\RU{Это можно заметить по тому что опкоды thumb-2 инструкций всегда начинаются с \TT{0xFx} либо с \TT{0xEx}}
\EN{That's
easily observable~---opcodes of thumb-2 instructions also begin with \TT{0xFx} or \TT{0xEx}}.
\RU{Но в листинге \IDA байты опкода переставлены местами,
это из-за того, что в процессоре ARM инструкции кодируются так:
в начале последний байт, потом первый (для thumb и thumb-2 режима), либо, 
(для инструкций в режиме ARM) в начале четвертый байт, затем третий, второй и первый 
(т.е., другой \gls{endianness})}
\EN{But in the \IDA listings
two opcode bytes are swapped (for thumb and thumb-2 modes).  For instructions in ARM mode, the order is the fourth byte, then the third,
then the second and finally the first (due to different \gls{endianness})}.
\index{ARM!\Instructions!MOVW}
\index{ARM!\Instructions!MOVT.W}
\index{ARM!\Instructions!BLX}
\RU{Так что мы видим здесь что инструкции \TT{MOVW}, \TT{MOVT.W} и \TT{BLX} начинаются с}
\EN{So as we can see, the \TT{MOVW}, \TT{MOVT.W} and \TT{BLX} instructions begin with} \TT{0xFx}.

\RU{Одна из thumb-2 инструкций это}\EN{One of the thumb-2 instructions is}
\TT{``MOVW R0, \#0x13D8''}\RU{ ~--- она записывает 16-битное число в младшую часть регистра \Reg{0}}
\EN{~---it writes a 16-bit value into the lower part of the \Reg{0} register}.

\RU{Еще}\EN{Also,} \TT{``MOVT.W R0, \#0''}\RU{ ~--- эта инструкция работает так же, как и}
\EN{~works just like} 
\TT{MOVT} \RU{из предыдущего примера, но она работает в}\EN{from the previous example but it works in} thumb-2.

\index{ARM!\RU{переключение режимов}\EN{mode switching}}
\index{ARM!\Instructions!BLX}
\RU{Помимо прочих отличий, здесь используется инструкция}\EN{Among other differences, here the} \TT{BLX} \RU{вместо}\EN{instruction is used instead of} \TT{BL}.
\RU{Отличие в том, что помимо сохранения адреса возврата в регистре \ac{LR} и передаче управления в функцию \puts, происходит смена режима процессора с thumb на ARM, либо наоборот}
\EN{The difference is that, besides saving the \ac{RA} in the \ac{LR} register and passing control to the \puts function, the processor is also switching from thumb mode to ARM (or back)}.
\RU{Здесь это нужно потому, что инструкция, куда ведет переход, выглядит так (она закодирована в режиме ARM)}\EN{This instruction is placed here since the instruction to which control is passed looks like (it is encoded in ARM mode)}:

\begin{lstlisting}
__symbolstub1:00003FEC _puts           ; CODE XREF: _hello_world+E
__symbolstub1:00003FEC 44 F0 9F E5     LDR  PC, =__imp__puts
\end{lstlisting}

\RU{Итак, внимательный читатель может задать справедливый вопрос: почему бы не вызывать \puts сразу в 
том же месте кода, где он нужен?}
\EN{So, the observant reader may ask: why not call \puts right at the point in the code where it is needed?}

\RU{Но это не очень выгодно (в плане экономия места) и вот почему}
\EN{Because it is not very space-efficient}.

\index{\RU{Динамически подгружаемые библиотеки}\EN{Dynamically loaded libraries}}
\RU{Практически любая программа использует внешние динамические библиотеки (будь то DLL в Windows, .so в *NIX 
либо .dylib в \MacOSX)}\EN{Almost any program uses external dynamic libraries (like DLL in Windows, .so in *NIX or .dylib in \MacOSX)}.
\RU{В динамических библиотеках находятся часто используемые библиотечные функции, в том числе стандартная функция Си \puts}
\EN{Often-used library functions are stored in dynamic libraries, including the standard C-function \puts}.

\index{Relocation}
\RU{В исполняемом бинарном файле}\EN{In an executable binary file} 
(Windows PE .exe, ELF \RU{либо}\EN{or} Mach-O) \RU{имеется секция импортов, список символов (функций либо глобальных переменных) импортируемых из внешних модулей, а также названия самих модулей}
\EN{an import section is present.
This is a list of symbols (functions or global variables) being imported from external modules along with the names of these modules}.

\RU{Загрузчик \ac{OS} загружает необходимые модули и, перебирая импортируемые символы в основном модуле, проставляет правильные адреса каждого символа}
\EN{The \ac{OS} loader loads all modules it needs and, while enumerating import symbols in the primary module, determines the correct addresses of each symbol}.

\RU{В нашем случае,}\EN{In our case,} \IT{\_\_imp\_\_puts} 
\RU{это 32-битная переменная, куда загрузчик \ac{OS} запишет правильный адрес этой же функции во внешней библиотеке}
\EN{is a 32-bit variable where the \ac{OS} loader will write the correct address of the function in an external library}. 
\RU{Так что инструкция \TT{LDR} просто берет 32-битное значение из этой переменной и, записывая его в регистр \ac{PC}, просто передает туда управление}
\EN{Then the \TT{LDR} instruction just takes the 32-bit value from this variable and writes it into the \ac{PC} register, passing control to it}.

\RU{Чтобы уменьшить время работы загрузчика \ac{OS}, нужно чтобы ему пришлось записать адрес каждого символа только один раз, в соответствующее выделенное для них место}
\EN{So, in order to reduce the time that an \ac{OS} loader needs for doing this procedure, it is good idea for it to write the address of each symbol only once to a specially-allocated place just for it}.

\index{thunk-\RU{функции}\EN{functions}}
\RU{К тому же, как мы уже убедились, нельзя одной инструкцией загрузить в регистр 32-битное число без обращений к памяти}
\EN{Besides, as we have already figured out, it is impossible to load a 32-bit value into a register 
while using only one instruction without a memory access}.
\RU{Так что наиболее оптимально выделить отдельную функцию, работающую в режиме ARM, 
чья единственная цель ~--- передавать управление дальше, в динамическую библиотеку.}
\EN{So, it is optimal to allocate a separate function working in ARM mode with only one 
goal~---to pass control to the dynamic library}
\RU{И затем ссылаться на эту короткую функцию из одной инструкции (так называемую \glslink{thunk function}{thunk-функцию}) из thumb-кода}
\EN{and then to jump to this short one-instruction function (the so-called \gls{thunk function}) from thumb-code}.

\index{ARM!\Instructions!BL}
\RU{Кстати, в предыдущем примере (скомпилированном для режима ARM), переход при помощи инструкции \TT{BL} ведет 
на такую же \glslink{thunk function}{thunk-функцию}, однако режим процессора не переключается (отсюда, отсутствие ``X'' в мнемонике инструкции)}
\EN{By the way, in the previous example (compiled for ARM mode) control passed by the \TT{BL} instruction goes to the 
same \gls{thunk function}.
However the processor mode is not switched (hence the absence of an ``X'' in the instruction mnemonic)}.

