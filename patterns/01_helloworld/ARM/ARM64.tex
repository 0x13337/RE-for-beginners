\subsection{ARM64}

\subsubsection{GCC}

\RU{Компилируем пример в}\EN{Let's compile the example using} GCC 4.8.1 \InENRU ARM64:

\lstinputlisting[numbers=left,label=hw_ARM64_GCC,caption=\NonOptimizing GCC 4.8.1 + objdump]
{patterns/01_helloworld/ARM/hw.lst}

\RU{В ARM64 нет режима thumb и thumb-2, только ARM, так что тут только 32-битные инструкции.}
\EN{There are no thumb and thumb-2 modes in ARM64, only ARM, so there are 32-bit instructions only.}
\RU{Регистров тут в 2 раза больше}\EN{Registers count is doubled}: \ref{ARM64_GPRs}.
\RU{64-битные регистры теперь имеют префикс}\EN{64-bit registers has} 
\TT{X-}\EN{ prefixes, while its 32-bit parts}\RU{, а их 32-битные части}\EMDASH{}\TT{W-}.

\index{ARM!\Instructions!STP}
\RU{Инструкция }\TT{STP}\EN{ instruction} (\IT{Store Pair}) 
\RU{сохраняет в стеке сразу два регистра}\EN{saves two registers in stack simultaneously}: \RegX{29} \InENRU \RegX{30}.
\RU{Конечно, эта инструкция может сохранять эту пару где угодно в памяти, но здесь указан регистр \ac{SP}, так что,
пара сохраняется именно в стеке.}
\EN{Of course, this instruction is able to save this pair at random place of memory, 
but \ac{SP} register is specified here, so the pair is saved in stack.}
\RU{Регистры в ARM64 64-битные, каждый это 8 байт, так что для хранения двух регистров нужно именно 16 байт.}
\EN{ARM64 registers are 64-bit ones, each contain 8 bytes, so one need 16 bytes for saving two registers.}

\RU{Восклицательный знак после операнда означает, что в начале от \ac{SP} будет отнято 16, и только затем
значения из пары регистров будут записаны в стек.}
\EN{Exclamation mark after operand mean that 16 will be subtracted from \ac{SP} first, and only then
values from registers pair will be written into the stack.}
\RU{Это называется}\EN{This is also called} \IT{pre-index}.
\RU{Больше о разнице между}\EN{About difference between} \IT{post-index} \AndENRU \IT{pre-index}, 
\RU{описано здесь}\EN{read here}: \ref{ARM_postindex_vs_preindex}.

\RU{Таким образом, в терминах более знакомого всем процессора x86, первая инструкция это просто аналог 
пары инструкций}
\EN{Hence, in terms of more familiar x86, the first instruction is just analogous to pair of}
\TT{PUSH X29} \AndENRU \TT{PUSH X30}.
\RegX{29} \EN{is used as \ac{FP} in ARM64}\RU{в ARM64 используется как \ac{FP}}, \EN{and}\RU{а} \RegX{30} 
\EN{as}\RU{как} \ac{LR}, \RU{поэтому они сохраняются в прологе ф-ции и
восстанавливаются в эпилоге}\EN{so that's why they are saved in function prologue and restored in function
epilogue}.

\EN{The second instruction saves}\RU{Вторая инструкция записывает} \ac{SP} \InENRU \RegX{29} (\OrENRU \ac{FP}).
\RU{Это нужно для установки стекового фрейма ф-ции}\EN{This is needed for function stack frame setup}.

\label{pointers_ADRP_and_ADD}
\index{ARM!\Instructions!ADRP/ADD pair}
\RU{Инструкции }\TT{ADRP} \AndENRU \ADD \EN{instructions are needed for forming address of the 
string}\RU{нужны для формирования адреса строки} ``Hello!'' \EN{in the \RegX{0} register}\RU{в регистре \RegX{0}}, 
\RU{ведь первый аргумент ф-ции передается через этот регистр}\EN{because first function argument is passed
in this register}.
\RU{Но в ARM нет инструкций, при помощи которых можно записать в регистр длинное число}\EN{But there are
no instructions in ARM helping to write large number into register} 
(\RU{потому что сама длина инструкции ограничена 4-ю байтами, больше об этом здесь}\EN{because instruction
length is limited by 4 bytes, read more about it here}: \ref{ARM_big_constants_loading}).
\RU{Так что нужно использовать несколько инструкций}\EN{So several instructions should be used}.
\RU{Первая инструкция}\EN{The first instruction} (\TT{ADRP}) \EN{writes address of 4Kb page where string is
located into \RegX{0}}\RU{записывает в \RegX{0} адрес 4-килобайтной страницы где находится строка}, 
\EN{and the second one}\RU{а вторая} (\ADD) \RU{просто прибавляет к этому адресу остаток}\EN{just adds
reminder to the address}.
\EN{Read more about}\RU{Читайте больше об этом}: \ref{ARM64_relocs}.

\TT{0x400000 + 0x648 = 0x400648}, \EN{and we see our ``Hello!'' C-string in the \TT{.rodata} data segment at this
address}\RU{и мы видим что в секции данных \TT{.rodata} по этому адресу как раз находится наша
Си-строка ``Hello!''}.

\index{ARM!\Instructions!BL}
\RU{Затем, при помощи инструкции \TT{BL} вызывается \puts, это уже рассматривалось раннее: \ref{puts}.}
\EN{\puts is called then using \TT{BL} instruction, this was already discussed before: \ref{puts}.}

\RU{Инструкция }\MOV \EN{instruction writes $0$ into}\RU{записывает $0$ в} \RegW{0}. 
\RegW{0} \RU{это младшие 32 бита регистра}\EN{is low 32 bits of} \RegX{0}\EN{ register}:

\begin{center}
\begin{tabular}{ | l | l | }
\hline
\RU{Старшие 32 бита}\EN{High 32-bit part}\ES{Parte alta de 32 bits}\PTBRph{}\PLph{}\ITAph{}\DEph{}\THAph{} & \RU{младшие 32 бита}\EN{low 32-bit part}\ES{parte baja de 32 bits}\PTBRph{}\PLph{}\ITAph{}\DEph{}\THAph{} \\
\hline
\multicolumn{2}{ | c | }{X0} \\
\hline
\multicolumn{1}{ | c | }{} & \multicolumn{1}{ c | }{W0} \\
\hline
\end{tabular}
\end{center}


\RU{А результат ф-ции возвращается через \RegX{0}, и \main возвращает $0$, 
так что вот так готовится возвращаемый результат.}
\EN{Function result is returning via \RegX{0} and \main returning $0$, so that's how returning
result is prepared.}
\RU{Почему именно 32-битная часть}\EN{But why 32-bit part}?
\RU{Потому в ARM64, как и в x86-64, тип \Tint оставили 32-битным, для лучшей совместимости.}
\EN{Because \Tint in ARM64, just like in x86-64, is still 32-bit, for better compatibility.}
\RU{Следовательно, раз уж ф-ция возвращает 32-битный \Tint, то нужно заполнить только 32 младших бита 
регистра \RegX{0}.}
\EN{So if function returning 32-bit \Tint, only 32 lowest bits of \RegX{0} register should be filled.}

\RU{Для того, чтобы удостовериться в этом, я немного отредактировал свой пример и перекомпилировал его.}
\EN{In order to get sure about it, I changed by example slightly and recompiled it.}
\RU{Теперь}\EN{Now} \main \RU{возвращает 64-битное значение}\EN{returns 64-bit value}:

\begin{lstlisting}[caption=\main \RU{возвращающая значение типа}\EN{returning a value of} \TT{uint64\_t}\EN{ type}]
#include <stdio.h>
#include <stdint.h>

uint64_t main()
{
        printf ("Hello!\n");
        return 0;
};
\end{lstlisting}

\RU{Результат точно такой же, только \MOV в той строке теперь выглядит так:}
\EN{Result is very same, but that's how \MOV at that line is now looks like:}

\begin{lstlisting}[caption=\NonOptimizing GCC 4.8.1 + objdump]
  4005a4:       d2800000        mov     x0, #0x0                        // #0
\end{lstlisting}

\index{ARM!\Instructions!LDP}
\RU{Далее, при помощи инструкции \TT{LDP} (\IT{Load Pair}), восстанавливаются регистры \RegX{29} и \RegX{30}.}
\EN{\TT{LDP} (\IT{Load Pair}) then restores \RegX{29} and \RegX{30} registers.}
\RU{Восклицательного знака после инструкции нет: это означает, что в начале значения достаются из стека,
и только потом \ac{SP} увеличивается на 16.}
\EN{There are no exclamation mark after instruction: this mean, the value is first loaded from the stack,
only then \ac{SP} value is increased by 16.}
\RU{Это называется}\EN{This is called} \IT{post-index}.

\index{ARM!\Instructions!RET}
\RU{В ARM64 есть новая инструкция}\EN{New instruction appeared in ARM64}: \RET. 
\RU{Она работает так же как и}\EN{It works just as} \TT{BX LR}, \RU{но там добавлен специальный бит,
подсказывающий процессору, что это именно выход из ф-ции, а не просто переход, чтобы процессор
мог более оптимально исполнять эту инструкцию}\EN{but a special \IT{hint} bit is added, showing to the \ac{CPU}
that this is return from the function, not just another branch instruction, so it can execute it more optimally}.

\RU{Из-за простоты этой ф-ции, оптимизирующий GCC генерирует точно такой же код.}
\EN{Due to simplicity of the function, optimizing GCC generates the very same code.}
