\subsection{\NonOptimizingKeilVI (\ARMMode)}

\RU{Для начала скомпилируем наш пример в Keil}\EN{Let's start by compiling our example in Keil}:

\begin{lstlisting}
armcc.exe --arm --c90 -O0 1.c 
\end{lstlisting}

\index{\IntelSyntax}
\RU{Компилятор \IT{armcc} генерирует листинг на ассемблере в формате Intel,}
\EN{The \IT{armcc} compiler produces assembly listings in Intel-syntax} 
\RU{но он содержит некоторые высокоуровневые макросы, связанные с ARM}
\EN{but it has high-level ARM-processor related macros}\footnote{
\RU{например, он показывает инструкции \PUSH/\POP, отсутствующие в режиме ARM}
\EN{e.g. ARM mode lacks \PUSH/\POP instructions}}, 
\RU{а нам важнее увидеть инструкции ``как есть'', так что посмотрим скомпилированный результат в \IDA}
\EN{but it is more important for us to see the instructions ``as is'' so let's see the compiled result in \IDA}.

\begin{lstlisting}[caption=\NonOptimizingKeilVI (\ARMMode) \IDA]
.text:00000000             main
.text:00000000 10 40 2D E9    STMFD   SP!, {R4,LR}
.text:00000004 1E 0E 8F E2    ADR     R0, aHelloWorld ; "hello, world"
.text:00000008 15 19 00 EB    BL      __2printf
.text:0000000C 00 00 A0 E3    MOV     R0, #0
.text:00000010 10 80 BD E8    LDMFD   SP!, {R4,PC}

.text:000001EC 68 65 6C 6C+aHelloWorld  DCB "hello, world",0    ; DATA XREF: main+4
\end{lstlisting}

\RU{В вышеприведённом примере можно легко увидеть, что каждая инструкция имеет размер 4 байта.}
\EN{In the example, we can easily see each instruction has a size of 4 bytes.}
\RU{Действительно, ведь мы же компилировали наш код для режима ARM а не thumb.}
\EN{Indeed, we compiled our code for ARM mode, not for thumb.}

\index{ARM!\Instructions!STMFD}
\index{ARM!\Instructions!POP}
\RU{Самая первая инструкция}\EN{The very first instruction}, \TT{``STMFD SP!, \{R4,LR\}''}\footnote{\ac{STMFD}}, 
\RU{работает как инструкция}\EN{works as an x86} \PUSH \RU{в x86}\EN{instruction},
\RU{записывая значения двух регистров}\EN{writing the values of two registers}
(\Reg{4} \AndENRU \ac{LR}) \RU{в стек}\EN{into the stack}.
\RU{Действительно, в выдаваемом листинге на ассемблере компилятор \IT{armcc}, для упрощения, указывает здесь инструкцию}
\EN{Indeed, in the output listing from the \IT{armcc} compiler, for the sake of simplification, 
actually shows the} \TT{``PUSH \{r4,lr\}''}\EN{ instruction}.
\RU{Но это не совсем точно, инструкция \PUSH доступна только в режиме thumb, поэтому,
во избежание путаницы, я предложил работать в \IDA}
\EN{But it is not quite precise. The \PUSH instruction is only available in thumb mode.
So, to make things less messy, we're doing this in \IDA}.

\RU{Итак, эта инструкция уменьшает \ac{SP}, чтобы он указывал на место в стеке, свободное для записи
новых значений, затем записывает значения регистров \Reg{4} и \ac{LR} 
по адресу в памяти, на который указывает измененный регистр \ac{SP}}
\EN{This instruction first \glspl{decrement} \ac{SP} so it will point to the place in the stack
that is free for new entries, then it writes the values of the \Reg{4} and \ac{LR} registers at the address
stored in the modified \ac{SP}}.

\RU{Эта инструкция, как и инструкция \PUSH в режиме thumb, может сохранить в стеке одновременно несколько значений регистров, что может быть очень удобно}
\EN{This instruction (like the \PUSH instruction in thumb mode) is able to save several register values at once and this may be useful}. 
\RU{Кстати, такого в x86 нет}\EN{By the way, there is no such thing in x86}.
\RU{Также следует заметить, что \TT{STMFD} ~--- генерализация инструкции \PUSH (то есть, расширяет её возможности), потому что может работать с любым регистром, а не только с \ac{SP}.}
\EN{It can also be noted that the \TT{STMFD} instruction is a generalization 
of the \PUSH instruction (extending its features), since it can work with any register, not just with \ac{SP}.}
\RU{Другими словами, \TT{STMFD} можно использовать для записи набора регистров в указанном месте памяти.}
\EN{In other words, \TT{STMFD} may be used for storing pack of registers at the specified memory address.}

\index{\PICcode}
\index{ARM!\Instructions!ADR}
\RU{Инструкция}\EN{The} \TT{``ADR R0, aHelloWorld''}
\RU{прибавляет значение регистра \ac{PC} к смещению, где хранится строка}
\EN{instruction adds the value in the \ac{PC} register to the offset where the}
\IT{``hello, world''}\EN{ string is located}.
\RU{Причем здесь \ac{PC}, можно спросить}\EN{How the \TT{PC} register is used here, one might ask}?
\RU{Притом, что это так называемый ``\PICcode''}\EN{This is so-called ``\PICcode''.}
\footnote{
	\RU{Читайте больше об этом в соответствующем разделе}
	\EN{Read more about it in relevant section}~(\myref{sec:PIC})
	}
\RU{он предназначен для исполнения будучи не привязанным к каким-либо адресам в памяти}
\EN{It is intended to be executed at a non-fixed address in memory}.
\RU{В опкоде инструкции \TT{ADR} указывается разница между адресом этой инструкции и местом, где хранится строка}
\EN{In the opcode of the \TT{ADR} instruction, the difference between the address of this instruction and the place where the string is located is encoded}.
\RU{Эта разница всегда будет постоянной, вне зависимости от того, куда был загружен \ac{OS}
наш код}
\EN{The difference will always be the same,
independent of the address where the code is loaded by the \ac{OS}}.
\RU{Поэтому всё, что нужно ~--- это прибавить адрес текущей инструкции (из \ac{PC}), чтобы получить текущий абсолютный адрес нашей Си-строки}
\EN{That's why all we need is to add the address of the current instruction (from \ac{PC}) in order to get the absolute address of our C-string in memory}.

\index{ARM!\Registers!Link Register}
\index{ARM!\Instructions!BL}
\RU{Инструкция} \TT{``BL \_\_2printf''}\footnote{Branch with Link}
\RU{вызывает функцию \printf}\EN{instruction calls the \printf function}. 
\RU{Работа этой инструкции состоит из двух фаз}
\EN{Here's how this instruction works}: 
\begin{itemize}
\item
\RU{записать адрес после инструкции \TT{BL} (\TT{0xC}) в регистр \ac{LR}}
\EN{write the address following the \TT{BL} instruction (\TT{0xC}) into the \ac{LR}};
\item
\RU{затем собственно передать управление в \printf, записав адрес этой функции в регистр \ac{PC}}
\EN{then pass control flow into \printf by writing its address into the \ac{PC} register}.
\end{itemize}

\RU{Ведь когда функция \printf закончит работу, нужно знать, куда вернуть управление, поэтому закончив работу, всякая функция передает управление по адресу, записанному в регистре \ac{LR}}
\EN{When \printf finishes its work it must have information about where it must return control.
That's why each function passes control to the address stored in the \ac{LR} register}.

\RU{В этом разница между ``чистыми'' \ac{RISC}-процессорами вроде ARM и \ac{CISC}-процессорами как x86,
где адрес возврата обычно записывается в стек}
\EN{That is the difference between ``pure'' \ac{RISC}-processors like ARM and \ac{CISC}-processors like x86,
where the return address is usually stored on the stack}\footnote{\RU{Подробнее об этом будет описано в следующей главе}\EN{Read more about this in next section}~(\myref{sec:stack})}.

\RU{Кстати, 32-битный абсолютный адрес, либо же смещение, невозможно закодировать в 32-битной инструкции \TT{BL}, в ней есть место только для 24-х бит}
\EN{By the way, an absolute 32-bit address or offset cannot be encoded in the 32-bit \TT{BL} instruction because
it only has space for 24 bits}.
\RU{Как мы помним, все инструкции в режиме ARM имеют длину 4 байта (32 бита), и инструкции могут находится только по адресам кратным 4, то последние 2 бита (всегда нулевых) можно не кодировать.}
\EN{As we may remember, all ARM-mode instructions have a size of 4 bytes (32 bits).
Hence, they can only be located on 4-byte boundary addresses.
This means that the last 2 bits of the instruction address (which are always zero bits) may be omitted.}
\RU{В итоге имеем 26 бит, при помощи которых, можно закодировать}
\EN{In summary, we have 26 bits for offset encoding. This is enough to encode} $current\_PC \pm{} \approx{}32M$.

\index{ARM!\Instructions!MOV}
\RU{Следующая инструкция}\EN{Next, the} \TT{``MOV R0, \#0''}\footnote{MOVe}
\RU{просто записывает $0$ в регистр \Reg{0}}\EN{instruction just writes $0$ into the \Reg{0} register}.
\RU{Ведь наша Си-функция возвращает $0$, а возвращаемое значение всякая функция оставляет в \Reg{0}}
\EN{That's because our C-function returns $0$ and the return value is to be placed in the \Reg{0} register}.

\index{ARM!\Registers!Link Register}
\index{ARM!\Instructions!LDMFD}
\index{ARM!\Instructions!POP}
\RU{Последняя инструкция  ~---}\EN{The last instruction} \TT{``LDMFD SP!, {R4,PC}''}\footnote{\ac{LDMFD}} \RU{это инструкция, обратная от}\EN{is an inverse instruction of} \TT{STMFD}. 
\RU{Она загружает из стека (или любого другого места в памяти) значения для сохранения их в \Reg{4} и \ac{PC}, увеличивая \glslink{stack pointer}{указатель стека} \ac{SP}}
\EN{It loads values from the stack (or any other memory place) in order to save them into \Reg{4} and \ac{PC}, and \glslink{increment}{increments} the \gls{stack pointer} \ac{SP}}.
\RU{Здесь она работает как аналог \POP}\EN{It works like \POP here}.\\
N.B. \RU{Самая первая инструкция \TT{STMFD} сохранила в стеке \Reg{4} и \ac{LR}, а \IT{восстанавливаются} во время исполнения \TT{LDMFD} регистры \Reg{4} и \ac{PC}}
\EN{The very first instruction \TT{STMFD} saves the \Reg{4} and \ac{LR} registers pair on the stack, but \Reg{4} and \ac{PC} are \IT{restored} during execution of \TT{LDMFD}}.

\RU{Как я уже описывал, в регистре \ac{LR} обычно сохраняется адрес места, куда нужно всякой функции вернуть управление}
\EN{As I wrote before, the address of the place to where each function must return control is usually saved in the \ac{LR}
register}.
\RU{Самая первая инструкция сохраняет это значение в стеке, потому что наша функция \main позже будет сама пользоваться этим регистром, в момент вызова \printf}
\EN{The very first function saves its value in the stack because our
\main function will use the register in order to call \printf}.
\RU{А затем, в конце функции, это значение можно сразу записать прямо в \ac{PC}, таким образом, передав управление туда, откуда была вызвана наша функция}
\EN{In the function end, this value can be written directly to the \ac{PC} register, thus passing control to where our function was called}.
\RU{Так как функция \main обычно самая главная в \CCpp, управление будет возвращено в загрузчик \ac{OS}, либо куда-то в \ac{CRT} 
или что-то в этом роде}
\EN{Since our \main function is usually the primary function in \CCpp,
control will be returned to the \ac{OS} loader or to a point in \ac{CRT},
or something like that}.

\index{ARM!DCB}
\TT{DCB}\RU{ ~--- директива ассемблера, описывающая массивы байт или ASCII-строк, аналог директивы DB в 
x86-ассемблере}
\EN{~is an assembly language directive defining an array of bytes or ASCII strings, akin to the DB directive 
in x86-assembly language}.
