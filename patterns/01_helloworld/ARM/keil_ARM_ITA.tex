\subsectionold{\NonOptimizingKeilVI (\ARMMode)}

Iniziamo a compilare il nostro esempio in Keil:

\begin{lstlisting}
armcc.exe --arm --c90 -O0 1.c 
\end{lstlisting}

\myindex{\IntelSyntax}
Il compilatore \IT{armcc} produce un listato assembly con sintassi Intel,
e utilizza macro di alto livello legate al processore ARM
\footnote{ad esempio, l' ARM mode 'e privo delle istruzioni \PUSH/\POP},
tuttavia e' piu' importante per noi vedere le istruzioni \q{cosi' come sono}, quindi guardiamo il risultato compilato con \IDA.

\begin{lstlisting}[caption=\NonOptimizingKeilVI (\ARMMode) \IDA]
.text:00000000             main
.text:00000000 10 40 2D E9    STMFD   SP!, {R4,LR}
.text:00000004 1E 0E 8F E2    ADR     R0, aHelloWorld ; "hello, world"
.text:00000008 15 19 00 EB    BL      __2printf
.text:0000000C 00 00 A0 E3    MOV     R0, #0
.text:00000010 10 80 BD E8    LDMFD   SP!, {R4,PC}

.text:000001EC 68 65 6C 6C+aHelloWorld  DCB "hello, world",0    ; DATA XREF: main+4
\end{lstlisting}

Nell'esempio possiamo facilmente vedere che ogni istruzione ha lunghezza pari a 4 byte.
Difatti abbiamo compilato il codice per la modalita' ARM e non Thumb.

\myindex{ARM!\Instructions!STMFD}
\myindex{ARM!\Instructions!POP}
La prima istruzione, \TT{STMFD SP!, \{R4,LR\}}\footnote{\ac{STMFD}},
funzione come l' istruzione \PUSH in x86, scrivendo i valori di due registri (\Reg{4} \ITAph{} \ac{LR}) nello stack.
Infatti il listato di output prodotto dal compilatore \IT{armcc}, per semplificazione, mostra l'istruzione \INS{PUSH \{r4,lr\}}.
Ma cio' non e' del tutto esatto. L'istruzione\PUSH e' disponibile solo in modalita' Thumb. Utilizziamo quindi \IDA per non fare confusione.

Questa istruzione dapprima \glslink{decrement}{decrementa} il valore di \ac{SP} cosi' da farlo puntare alla porzione dello stack che' e' libera di ospitare nuovi dati, quindi salva il valore dei registri \Reg{4} e \ac{LR} all'indirizzo memorizzato nel registro \ac{SP} appena modificato.

Questa istruzione (esattamente come \PUSH in Thumb mode) e' in grado di salvare il valore di piu' registri contemporaneamente, cosa che e' puo' risultare molto utile. 
A proposito, non ha un equivalente in x86.
Si puo' notare anche che l'istruzione \TT{STMFD} e' una generalizzazione dell'istruzione \PUSH (che estende le sue funzionalita'), poiche' puo' funzionare con qualunque registro, e non solo \ac{SP}.
In altre parole, \TT{STMFD} puo' essere usata per memorizzare un insieme di registri all'indirizzo di memoria specificato.

\myindex{\PICcode}
\myindex{ARM!\Instructions!ADR}
L'istruzione \INS{ADR R0, aHelloWorld}
aggiunge o sottrae il valore nel registro \ac{PC} all'offset dove e' memorizzata la stringa \TT{hello, world}.
Ci si potrebbe chiedere, come e' utilizzato qui il registro \TT{PC}?
Cio' e' detto \q{\PICcode}
\footnote{Maggiori informazioni sono fornite nella sezione~(\myref{sec:PIC})}.
Questo tipo di codice puo' essere eseguito a indirizzi non fissi (variabili)in memoria.
In altre parole, e' un indirizzamento relativo a \ac{PC} (\ac{PC}-relative addressing).
L'istruzione \TT{ADR} tiene conto della differenza tra l'indirizzo di questa istruzione e l'indirizzo dove si trova la stringa.
Questa differenza (offset) sara' sempre la stessa, a prescindere dall'indirizzo in cui nostro codice sara' caricato dall'\ac{OS}.
Cio' spiega perche' bisogna soltanto aggiungere l'indirizzo dell'istruzione corrente (from \ac{PC}) per ottenere l'indirizzo assoluto in memoria della nostra stringa C.

\myindex{ARM!\Registers!Link Register}
\myindex{ARM!\Instructions!BL}
L'istruzione \INS{BL \_\_2printf}\footnote{Branch with Link} chiama la funzione \printf. 
Questa istruzione funziona cosi': 
\begin{itemize}
\item memorizza l'indirizzo successivo all'istruzione \INS{BL} (\TT{0xC}) nel registro \ac{LR};
\item quindi passa il controllo a \printf scrivendo il suo indirizzo nel registro \ac{PC}.
\end{itemize}

Quando la funzione \printf termina la sua esecuzione, deve sapere a chi restituire il controllo (dove ritornare). Per questo motivo ogni funzione passa il controllo all'indirizzo memorizzato nel registro \ac{LR}.

Questa e' una differenza tra processori \ac{RISC} \q{puri} come ARM e processori simili a \ac{CISC} come x86, nei quali il return address e' solitamente memorizzato nello stack
\footnote{Maggiori informazioni si trovano nella prossima sezione~(\myref{sec:stack})}.

A proposito, un indirizzo assoluto o un offset a 32-bit non puo' essere codificato nell'istruzione a 32-bit \TT{BL} poiche' ha solo spazio per 24 bit.
Come potremmo ricordare, tutte le istruzioni in ARM-mode hanno dimensione fissa di 4 byte (32 bit).
Dunque possono essere collocate solo su indirizzi allineati a 4-byte.
Cio' implica che gli ultimi 2 bits dell'indirizzo dell'istruzione (che sono sempre zero) possono essere omessi.
Abbiamo in definitiva 26 bit per la codifica dell'offset (offset encoding). E cio e' sufficiente per codificare $current\_PC \pm{} \approx{}32M$.

\myindex{ARM!\Instructions!MOV}
L'istruzione successiva, \INS{MOV R0, \#0}\footnote{\ITAph{} MOVe} scrive soltanto 0 nel registro \Reg{0}.
Questo succede perche' la nostra funzione C restituisce 0, ed il valore di ritorno deve essere memorizzato nel registro \Reg{0}.

\myindex{ARM!\Registers!Link Register}
\myindex{ARM!\Instructions!LDMFD}
\myindex{ARM!\Instructions!POP}
L'ultima istruzione \INS{LDMFD SP!, {R4,PC}}\footnote{\ac{LDMFD} e' l'istruzione inversa rispetto a  \ac{STMFD}}.
Carica i valori dallo stack (o qualunque altra zona di memoria) per salvarli nei registri \Reg{4} e \ac{PC}, e \glslink{increment}{incrementa} lo \gls{stack pointer} \ac{SP}.
In questo caso funziona come \POP.\\
N.B. La prima istruzione \TT{STMFD} aveva salvato la coppia di registri \Reg{4} e \ac{LR} sullo stack, ma \Reg{4} e \ac{PC} vengono \IT{ripristinati} durante l'esecuzione di \TT{LDMFD}.

Come gia' sappiamo, l'indirizzo del posto a cui ogni funzione devere restituire il controllo e' solitamente salvato nel registro \ac{LR}.
La prima istruzione salva il suo valore nello stack perche' lo stesso registro sara' usato dalla nostra funzione \main per la chiamata a \printf.
Al termine della funzione, questo valore puo' essere scritto direttamente nel registro \ac{PC}, passando di fatti il controllo al punto in cui la nostra funzione era stata chiamata.

Dal momento che \main e' solitamente la funzione principale in \CCpp,
il controllo sara' restituito al loader dell' \ac{OS} oppure ad un punto in una \ac{CRT},
o qualcosa del genere.

Tutto cio' consente di omettere l'istruzione \TT{BX LR} alla fine della funzione.

\myindex{ARM!DCB}
\TT{DCB} e' una direttiva assembly che definisce un array di byte o una stringa ASCII, analoga alla direttiva DB in linguaggio assembly x86.

