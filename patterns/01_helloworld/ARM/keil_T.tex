\subsection{\NonOptimizingKeilVI (\ThumbMode)}

\RU{Скомпилируем тот же пример в Keil для режима thumb}\EN{Let's compile the same example using Keil in thumb mode}:

\begin{lstlisting}
armcc.exe --thumb --c90 -O0 1.c 
\end{lstlisting}

\RU{Получим (в \IDA)}\EN{We will get (in \IDA)}:

\begin{lstlisting}[caption=\NonOptimizingKeilVI (\ThumbMode) + \IDA]
.text:00000000             main
.text:00000000 10 B5          PUSH    {R4,LR}
.text:00000002 C0 A0          ADR     R0, aHelloWorld ; "hello, world"
.text:00000004 06 F0 2E F9    BL      __2printf
.text:00000008 00 20          MOVS    R0, #0
.text:0000000A 10 BD          POP     {R4,PC}

.text:00000304 68 65 6C 6C+aHelloWorld  DCB "hello, world",0    ; DATA XREF: main+2
\end{lstlisting}

\RU{Сразу бросаются в глаза двухбайтные (16-битные) опкоды ~--- это, как я уже упоминал, thumb}
\EN{We can easily spot the 2-byte (16-bit) opcodes. This is, as I mentioned, thumb}.
\index{ARM!\Instructions!BL}
\RU{Кроме инструкции \TT{BL}.}\EN{The \TT{BL} instruction, however, }
\RU{Но на самом деле она состоит из двух 16-битных инструкций}
\EN{consists of two 16-bit instructions}.
\RU{Это потому, что загрузить в \ac{PC} смещение, по которому находится функция \printf, используя так мало места в одном 16-битном опкоде, нельзя}
\EN{This is because it is impossible to load an offset for the \printf function into \ac{PC} while using the small space in one 16-bit opcode}.
\RU{Так что первая 16-битная инструкция загружает старшие 10 бит смещения, а вторая ~--- младшие 11 бит смещения}
\EN{Therefore, the first 16-bit instruction loads the higher 10 bits of the offset and the second instruction loads 
the lower 11 bits of the offset}.
\RU{Как я уже упоминал, все инструкции в thumb-режиме имеют длину 2 байта (или 16 бит)}
\EN{As I mentioned, all instructions in thumb mode have a size of 2 bytes (or 16 bits)}.
\RU{Поэтому невозможна такая ситуация, когда thumb-инструкция начинается по нечетному адресу}
\EN{This means it is impossible for a thumb-instruction to be at an odd address whatsoever}.
\RU{Учитывая сказанное, последний бит адреса можно не кодировать}
\EN{Given the above, the last address bit may be omitted while encoding instructions}.
\RU{Таким образом, в thumb-инструкции \TT{BL} можно закодировать адрес}
\EN{In summary, \TT{BL} thumb-instruction can encode the address} $current\_PC \pm{}\approx{}2M$.

\index{ARM!\Instructions!PUSH}
\index{ARM!\Instructions!POP}
\RU{Остальные инструкции в функции: \PUSH и \POP здесь работают почти так же, как и описанные \TT{STMFD}/\TT{LDMFD}, только регистр \ac{SP} здесь не указывается явно}
\EN{As for the other instructions in the function: \PUSH and \POP work here just like the described \TT{STMFD}/\TT{LDMFD} only the \ac{SP} register is not mentioned explicitly here}.
\TT{ADR} \RU{работает так же, как и в предыдущем примере}\EN{works just like in the previous example}.
\TT{MOVS} \RU{записывает $0$ в регистр \Reg{0} для возврата нуля}
\EN{writes $0$ into the \Reg{0} register in order to return zero}.
