\section{ARM}
\label{sec:hw_ARM}

\index{\idevices}
\index{Raspberry Pi}
\index{Xcode}
\index{LLVM}
\index{Keil}
\RU{Для экспериментов с процессором ARM я использовал несколько компиляторов:}
\EN{For my experiments with ARM processors I used several compilers:} 

\begin{itemize}
\item \RU{Популярный в embedded-среде}\EN{Popular in the embedded area} Keil Release 6/2013.

\item Apple Xcode 4.6.3 \EN{IDE} (\RU{с компилятором}\EN{with} LLVM-GCC 4.2 \EN{compiler}
\footnote{\EN{It is indeed so: Apple Xcode 4.6.3 uses open-source GCC as front-end compiler and LLVM 
code generator}\RU{Это действительно так: Apple Xcode 4.6.3 использует опен-сорсный GCC как компилятор
переднего плана и коде-генератор LLVM}}.

\item GCC 4.8.1 (Linaro) (\RU{для}\EN{for} ARM64).

\item Apple Xcode 5.1.1 (\RU{для}\EN{for} ARM64).

\item GCC 4.9 (Linaro) (\RU{для}\EN{for} ARM64), 
\RU{доступный как исполняемые файлы для win32 на}\EN{available as win32-executables at} 
\url{http://www.linaro.org/projects/armv8/}.

\end{itemize}

\RU{Везде в этой книге, кроме как если указано иное, идет речь о 32-битном ARM.}
\EN{32-bit ARM code is used in all cases in this book, if not mentioned otherwise.}

\RU{Когда речь идет о 64-битном ARM, он называется здесь ARM64.}
\EN{If we talk about 64-bit ARM here, it will be called ARM64.}

% subsections
\subsection{\NonOptimizingKeilVI (\ARMMode)}

\RU{Для начала скомпилируем наш пример в Keil}\EN{Let's start by compiling our example in Keil}:

\begin{lstlisting}
armcc.exe --arm --c90 -O0 1.c 
\end{lstlisting}

\index{\IntelSyntax}
\RU{Компилятор \IT{armcc} генерирует листинг на ассемблере в формате Intel.}
\EN{The \IT{armcc} compiler produces assembly listings in Intel-syntax} 
\RU{Этот листинг содержит некоторые высокоуровневые макросы, связанные с ARM}%
\EN{but it has high-level ARM-processor related macros}\footnote{
\RU{например, он показывает инструкции \PUSH/\POP, отсутствующие в режиме ARM}
\EN{e.g. ARM mode lacks \PUSH/\POP instructions}}, 
\RU{а нам важнее увидеть инструкции \q{как есть}, так что посмотрим скомпилированный результат в \IDA.}
\EN{but it is more important for us to see the instructions \q{as is} so let's see the compiled result in \IDA.}

\begin{lstlisting}[caption=\NonOptimizingKeilVI (\ARMMode) \IDA]
.text:00000000             main
.text:00000000 10 40 2D E9    STMFD   SP!, {R4,LR}
.text:00000004 1E 0E 8F E2    ADR     R0, aHelloWorld ; "hello, world"
.text:00000008 15 19 00 EB    BL      __2printf
.text:0000000C 00 00 A0 E3    MOV     R0, #0
.text:00000010 10 80 BD E8    LDMFD   SP!, {R4,PC}

.text:000001EC 68 65 6C 6C+aHelloWorld  DCB "hello, world",0    ; DATA XREF: main+4
\end{lstlisting}

\RU{В вышеприведённом примере можно легко увидеть, что каждая инструкция имеет размер 4 байта.}
\EN{In the example, we can easily see each instruction has a size of 4 bytes.}
\RU{Действительно, ведь мы же компилировали наш код для режима ARM, а не Thumb.}
\EN{Indeed, we compiled our code for ARM mode, not for Thumb.}

\index{ARM!\Instructions!STMFD}
\index{ARM!\Instructions!POP}
\RU{Самая первая инструкция}\EN{The very first instruction}, \TT{STMFD SP!, \{R4,LR\}}\footnote{\ac{STMFD}}, 
\RU{работает как инструкция}\EN{works as an x86} \PUSH \RU{в x86}\EN{instruction},
\RU{записывая значения двух регистров}\EN{writing the values of two registers}
(\Reg{4} \AndENRU \ac{LR}) \RU{в стек}\EN{into the stack}.
\RU{Действительно, в выдаваемом листинге на ассемблере компилятор \IT{armcc} для упрощения указывает здесь инструкцию}
\EN{Indeed, in the output listing from the \IT{armcc} compiler, for the sake of simplification, 
actually shows the} \TT{PUSH \{r4,lr\}}\EN{ instruction}.
\RU{Но это не совсем точно, инструкция \PUSH доступна только в режиме Thumb, поэтому,
во избежание путаницы, я предложил работать в \IDA}%
\EN{But that is not quite precise. The \PUSH instruction is only available in Thumb mode.
So, to make things less confusing, we're doing this in \IDA}.

\RU{Итак, эта инструкция уменьшает \ac{SP}, чтобы он указывал на место в стеке, свободное для записи
новых значений, затем записывает значения регистров \Reg{4} и \ac{LR} 
по адресу в памяти, на который указывает измененный регистр \ac{SP}}%
\EN{This instruction \glspl{decrement} first the \ac{SP} so it points to the place in the stack
that is free for new entries, then it saves the values of the \Reg{4} and \ac{LR} registers at the address
stored in the modified \ac{SP}}.

\RU{Эта инструкция, как и инструкция \PUSH в режиме Thumb, может сохранить в стеке одновременно несколько значений регистров, что может быть очень удобно}%
\EN{This instruction (like the \PUSH instruction in Thumb mode) is able to save several register values at once which can be very useful}. 
\RU{Кстати, такого в x86 нет}\EN{By the way, this has no equivalent in x86}.
\RU{Также следует заметить, что \TT{STMFD}~--- генерализация инструкции \PUSH (то есть расширяет её возможности), потому что может работать с любым регистром, а не только с \ac{SP}.}
\EN{It can also be noted that the \TT{STMFD} instruction is a generalization 
of the \PUSH instruction (extending its features), since it can work with any register, not just with \ac{SP}.}
\RU{Другими словами, \TT{STMFD} можно использовать для записи набора регистров в указанном месте памяти.}
\EN{In other words, \TT{STMFD} may be used for storing pack of registers at the specified memory address.}

\index{\PICcode}
\index{ARM!\Instructions!ADR}
\RU{Инструкция}\EN{The} \TT{ADR R0, aHelloWorld}
\RU{прибавляет или отнимает значение регистра \ac{PC} к смещению, где хранится строка}
\EN{instruction adds or subtracts the value in the \ac{PC} register to the offset where the}
\TT{hello, world}\EN{ string is located}.
\RU{Причем здесь \ac{PC}, можно спросить}\EN{How is the \TT{PC} register used here, one might ask}?
\RU{Притом, что это так называемый \q{\PICcode}}\EN{This is so-called \q{\PICcode}.}
\footnote{
	\RU{Читайте больше об этом в соответствующем разделе}
	\EN{Read more about it in relevant section}~(\myref{sec:PIC})
	}
\RU{он предназначен для исполнения будучи не привязанным к каким-либо адресам в памяти}%
\EN{Such code can be be executed at a non-fixed address in memory}.
\EN{In other words, this is \ac{PC}-relative addressing.}
\RU{Другими словами, это относительная от \ac{PC} адресация.}
\RU{В опкоде инструкции \TT{ADR} указывается разница между адресом этой инструкции и местом, где хранится строка}%
\EN{The \TT{ADR} instruction takes into account the difference between the address of this instruction and the address where the string is located}.
\RU{Эта разница всегда будет постоянной, вне зависимости от того, куда был загружен \ac{OS} наш код}%
\EN{This difference (offset) is always to be the same, no matter at what address our code is loaded by the \ac{OS}}.
\RU{Поэтому всё, что нужно~--- это прибавить адрес текущей инструкции (из \ac{PC}), чтобы получить текущий абсолютный адрес нашей Си-строки}%
\EN{That's why all we need is to add the address of the current instruction (from \ac{PC}) in order to get the absolute memory address of our C-string}.

\index{ARM!\Registers!Link Register}
\index{ARM!\Instructions!BL}
\RU{Инструкция} \TT{BL \_\_2printf}\footnote{Branch with Link}
\RU{вызывает функцию \printf}\EN{instruction calls the \printf function}. 
\RU{Работа этой инструкции состоит из двух фаз}%
\EN{Here's how this instruction works}: 
\begin{itemize}
\item
\RU{записать адрес после инструкции \TT{BL} (\TT{0xC}) в регистр \ac{LR}}%
\EN{store the address following the \TT{BL} instruction (\TT{0xC}) into the \ac{LR}};
\item
\RU{передать управление в \printf, записав адрес этой функции в регистр \ac{PC}}%
\EN{then pass the control to the \printf by writing its address into the \ac{PC} register}.
\end{itemize}

\RU{Ведь когда функция \printf закончит работу, нужно знать, куда вернуть управление, поэтому закончив работу, всякая функция передает управление по адресу, записанному в регистре \ac{LR}}%
\EN{When \printf finishes its execution it must have information about where it needs to return the control to.
That's why each function passes control to the address stored in the \ac{LR} register}.

\RU{В этом разница между \q{чистыми} \ac{RISC}-процессорами вроде ARM и \ac{CISC}-процессорами как x86,
где адрес возврата обычно записывается в стек}%
\EN{That is a difference between \q{pure} \ac{RISC}-processors like ARM and \ac{CISC}-processors like x86,
where the return address is usually stored on the stack}\footnote{\RU{Подробнее об этом будет описано в следующей главе}\EN{Read more about this in next section}~(\myref{sec:stack})}.

\RU{Кстати, 32-битный абсолютный адрес (либо смещение) невозможно закодировать в 32-битной инструкции \TT{BL}, в ней есть место только для 24-х бит}%
\EN{By the way, an absolute 32-bit address or offset cannot be encoded in the 32-bit \TT{BL} instruction because
it only has space for 24 bits}.
\RU{Поскольку все инструкции в режиме ARM имеют длину 4 байта (32 бита) и инструкции могут находится только по адресам кратным 4, то последние 2 бита (всегда нулевых) можно не кодировать.}
\EN{As we may remember, all ARM-mode instructions have a size of 4 bytes (32 bits).
Hence, they can only be located on 4-byte boundary addresses.
This implies that the last 2 bits of the instruction address (which are always zero bits) may be omitted.}
\RU{В итоге имеем 26 бит, при помощи которых можно закодировать}
\EN{In summary, we have 26 bits for offset encoding. This is enough to encode} $current\_PC \pm{} \approx{}32M$.

\index{ARM!\Instructions!MOV}
\RU{Следующая инструкция}\EN{Next, the} \TT{MOV R0, \#0}\footnote{MOVe}
\RU{просто записывает 0 в регистр \Reg{0}}\EN{instruction just writes $0$ into the \Reg{0} register}.
\RU{Ведь наша Си-функция возвращает 0, а возвращаемое значение всякая функция оставляет в \Reg{0}}%
\EN{That's because our C-function returns 0 and the return value is to be placed in the \Reg{0} register}.

\index{ARM!\Registers!Link Register}
\index{ARM!\Instructions!LDMFD}
\index{ARM!\Instructions!POP}
\RU{Последняя инструкция}\EN{The last instruction} \TT{LDMFD SP!, {R4,PC}}\footnote{\ac{LDMFD}}\RU{~--- это инструкция, обратная}\EN{ is an inverse instruction of} \TT{STMFD}. 
\RU{Она загружает из стека (или любого другого места в памяти) значения для сохранения их в \Reg{4} и \ac{PC}, увеличивая \glslink{stack pointer}{указатель стека} \ac{SP}.}
\EN{It loads values from the stack (or any other memory place) in order to save them into \Reg{4} and \ac{PC}, and \glslink{increment}{increments} the \gls{stack pointer} \ac{SP}.}
\RU{Здесь она работает как аналог \POP}\EN{It works like \POP here}.\\
N.B. \RU{Самая первая инструкция \TT{STMFD} сохранила в стеке \Reg{4} и \ac{LR}, а \IT{восстанавливаются} во время исполнения \TT{LDMFD} регистры \Reg{4} и \ac{PC}}%
\EN{The very first instruction \TT{STMFD} saved the \Reg{4} and \ac{LR} registers pair on the stack, but \Reg{4} and \ac{PC} are \IT{restored} during the \TT{LDMFD} execution}.

\RU{Как я уже описывал, в регистре \ac{LR} обычно сохраняется адрес места, куда нужно всякой функции вернуть управление}%
\EN{As I mentioned before, the address of the place where each function must return control to is usually saved in the \ac{LR} register}.
\RU{Самая первая инструкция сохраняет это значение в стеке, потому что наша функция \main позже будет сама пользоваться этим регистром в момент вызова \printf}%
\EN{The very first instruction saves its value in the stack because the same register will be used by our
\main function when calling \printf}.
\RU{А затем, в конце функции, это значение можно сразу записать прямо в \ac{PC}, таким образом, передав управление туда, откуда была вызвана наша функция}%
\EN{In the function's end, this value can be written directly to the \ac{PC} register, thus passing control to where our function was called}.
\RU{Так как функция \main обычно самая главная в \CCpp, управление будет возвращено в загрузчик \ac{OS}, либо куда-то в \ac{CRT} 
или что-то в этом роде.}
\EN{Since \main is usually the primary function in \CCpp,
the control will be returned to the \ac{OS} loader or to a point in a \ac{CRT},
or something like that.}

\index{ARM!DCB}
\TT{DCB}\RU{~--- директива ассемблера, описывающая массивы байт или ASCII-строк, аналог директивы DB в 
x86-ассемблере}%
\EN{~is an assembly language directive defining an array of bytes or ASCII strings, akin to the DB directive 
in x86-assembly language}.

\subsection{\NonOptimizingKeilVI (\ThumbMode)}

\RU{Скомпилируем тот же пример в Keil для режима thumb}\EN{Let's compile the same example using Keil in thumb mode}:

\begin{lstlisting}
armcc.exe --thumb --c90 -O0 1.c 
\end{lstlisting}

\RU{Получим (в \IDA)}\EN{We are getting (in \IDA)}:

\begin{lstlisting}[caption=\NonOptimizingKeilVI (\ThumbMode) + \IDA]
.text:00000000             main
.text:00000000 10 B5          PUSH    {R4,LR}
.text:00000002 C0 A0          ADR     R0, aHelloWorld ; "hello, world"
.text:00000004 06 F0 2E F9    BL      __2printf
.text:00000008 00 20          MOVS    R0, #0
.text:0000000A 10 BD          POP     {R4,PC}

.text:00000304 68 65 6C 6C+aHelloWorld  DCB "hello, world",0    ; DATA XREF: main+2
\end{lstlisting}

\RU{Сразу бросаются в глаза двухбайтные (16-битные) опкоды ~--- это, как я уже упоминал, thumb}
\EN{We can easily spot the 2-byte (16-bit) opcodes. This is, as I mentioned, thumb}.
\index{ARM!\Instructions!BL}
\RU{Кроме инструкции \TT{BL}.}\EN{The \TT{BL} instruction, however, }
\RU{Но на самом деле она состоит из двух 16-битных инструкций}
\EN{consists of two 16-bit instructions}.
\RU{Это потому, что загрузить в \ac{PC} смещение, по которому находится функция \printf, используя так мало места в одном 16-битном опкоде, нельзя}
\EN{This is because it is impossible to load an offset for the \printf function into \ac{PC} while using the small space in one 16-bit opcode}.
\RU{Так что первая 16-битная инструкция загружает старшие 10 бит смещения, а вторая ~--- младшие 11 бит смещения}
\EN{Therefore, the first 16-bit instruction loads the higher 10 bits of the offset and the second instruction loads 
the lower 11 bits of the offset}.
\RU{Как я уже упоминал, все инструкции в thumb-режиме имеют длину 2 байта (или 16 бит)}
\EN{As I mentioned, all instructions in thumb mode have a size of 2 bytes (or 16 bits)}.
\RU{Поэтому невозможна такая ситуация, когда thumb-инструкция начинается по нечетному адресу.}
\EN{This implies it is impossible for a thumb-instruction to be at an odd address whatsoever.}
\RU{Учитывая сказанное, последний бит адреса можно не кодировать}
\EN{Given the above, the last address bit may be omitted while encoding instructions}.
\RU{Таким образом, в thumb-инструкции \TT{BL} можно закодировать адрес}
\EN{In summary, \TT{BL} thumb-instruction can encode the address} $current\_PC \pm{}\approx{}2M$.

\index{ARM!\Instructions!PUSH}
\index{ARM!\Instructions!POP}
\RU{Остальные инструкции в функции: \PUSH и \POP здесь работают почти так же, как и описанные \TT{STMFD}/\TT{LDMFD}, только регистр \ac{SP} здесь не указывается явно}
\EN{As for the other instructions in the function: \PUSH and \POP work here just like the described \TT{STMFD}/\TT{LDMFD} only the \ac{SP} register is not mentioned explicitly here}.
\TT{ADR} \RU{работает так же, как и в предыдущем примере}\EN{works just like in the previous example}.
\TT{MOVS} \RU{записывает $0$ в регистр \Reg{0} для возврата нуля}
\EN{writes $0$ into the \Reg{0} register in order to return zero}.

\subsection{\OptimizingXcodeIV (\ARMMode)}

Xcode 4.6.3 \RU{без включенной оптимизации выдает слишком много лишнего кода, поэтому включим оптимизацию компилятора (ключ \Othree), потому что там меньше инструкций.}
\EN{without optimization turned on produces a lot of redundant code so we'll study optimized output, where the instruction count is as small as possible, setting the compiler switch \Othree.}

\begin{lstlisting}[caption=\OptimizingXcodeIV (\ARMMode)]
__text:000028C4             _hello_world
__text:000028C4 80 40 2D E9   STMFD           SP!, {R7,LR}
__text:000028C8 86 06 01 E3   MOV             R0, #0x1686
__text:000028CC 0D 70 A0 E1   MOV             R7, SP
__text:000028D0 00 00 40 E3   MOVT            R0, #0
__text:000028D4 00 00 8F E0   ADD             R0, PC, R0
__text:000028D8 C3 05 00 EB   BL              _puts
__text:000028DC 00 00 A0 E3   MOV             R0, #0
__text:000028E0 80 80 BD E8   LDMFD           SP!, {R7,PC}

__cstring:00003F62 48 65 6C 6C+aHelloWorld_0  DCB "Hello world!",0
\end{lstlisting}

\RU{Инструкции}\EN{The instructions} \TT{STMFD} \AndENRU \TT{LDMFD} \RU{нам уже знакомы}\EN{are already familiar to us}.

\index{ARM!\Instructions!MOV}
\RU{Инструкция \MOV просто записывает число \TT{0x1686} в регистр \Reg{0} ~--- это смещение, указывающее на строку ``Hello world!''}
\EN{The \MOV instruction just writes the number \TT{0x1686} into the \Reg{0} register.
This is the offset pointing to the ``Hello world!'' string}.

\RU{Регистр \TT{R7} (по стандарту, принятому в \cite{IOSABI}) это frame pointer, о нем будет рассказано позже.}
\EN{The \TT{R7} register (as it is standardized in \cite{IOSABI}) is a frame pointer. More on that below.}

\index{ARM!\Instructions!MOVT}
\RU{Инструкция}\EN{The} \TT{MOVT R0, \#0} (MOVe Top) \RU{записывает $0$ в старшие 16 бит регистра}
\EN{instruction writes $0$ into higher 16 bits of the register}.
\RU{Дело в том, что обычная инструкция \MOV в режиме ARM может записывать какое-либо значение только в младшие 16 бит регистра, ведь в ней нельзя закодировать больше}
\EN{The issue here is that the generic \MOV instruction in ARM mode may write only the lower 16 bits of the register}.
\RU{Помните, что в режиме ARM опкоды всех инструкций ограничены длиной в 32 бита. Конечно, это ограничение не касается перемещений данных между регистрами.}
\EN{Remember, all instruction opcodes in ARM mode are limited in size to 32 bits. Of course, this limitation is not related to moving data between registers.}
\RU{Поэтому для записи в старшие биты (от 16-го по 31-го включительно) существует дополнительная команда \TT{MOVT}}
\EN{That's why an additional instruction \TT{MOVT} exists for writing into the higher bits (from 16 to 31 inclusive)}.
\RU{Впрочем, здесь её использование избыточно, потому что инструкция \TT{``MOV R0, \#0x1686''} выше и так обнулила старшую часть регистра}
\EN{Its usage here, however, is redundant because the \TT{``MOV R0, \#0x1686''} instruction above cleared the higher part of the register}.
\RU{Возможно, это недочет компилятора}\EN{This is probably a shortcoming of the compiler}.

\index{ARM!\Instructions!ADD}
\RU{Инструкция}\EN{The} \TT{``ADD R0, PC, R0''} \RU{прибавляет \ac{PC} к \Reg{0} для вычисления действительного адреса строки ``Hello world!''. Как нам уже известно, это ``\PICcode'', поэтому такая корректива необходима}
\EN{instruction adds the value in the \ac{PC} to the value in the \Reg{0}, to calculate absolute address of the ``Hello world!'' string. 
As we already know, it is ``\PICcode'' so this correction is essential here}.

\RU{Инструкция \TT{BL} вызывает \puts вместо \printf}
\EN{The \TT{BL} instruction calls the \puts function instead of \printf}.

\label{puts}
\index{\CStandardLibrary!puts()}
\index{puts() \RU{вместо}\EN{instead of} printf()}
\RU{Компилятор заменил вызов \printf на \puts. 
Действительно, \printf с одним аргументом это почти аналог \puts.}
\EN{GCC replaced the first \printf call with \puts.
Indeed: \printf with a sole argument is almost analogous to \puts.} 

\RU{\IT{Почти}, если принять условие, что в строке не будет управляющих символов \printf, 
начинающихся со знака процента. Тогда эффект от работы этих двух функций будет разным}
\EN{\IT{Almost}, because the two functions are producing the same result only in case the 
string does not contain printf format identifiers starting with \IT{\%}. 
In case it does, the effect of these two functions would be different}
\footnote{
\RU{Также нужно заметить, что \puts не требует символа перевода строки '\textbackslash{}n' в конце строки,
поэтому его здесь нет.}
\EN{It has also to be noted the \puts does not require a '\textbackslash{}n' new line symbol 
at the end of a string, so we do not see it here.}}.

\RU{Зачем компилятор заменил один вызов на другой? Наверное потому что \puts работает быстрее}
\EN{Why did the compiler replace the \printf with \puts? Probably because \puts is faster}
\footnote{\url{http://go.yurichev.com/17063}}. 

\RU{Видимо потому, что \puts проталкивает символы в \gls{stdout} не сравнивая каждый со знаком процента.}
\EN{\puts works faster because it just passes characters to \gls{stdout} without comparing every one of them with the \IT{\%} symbol.}

\RU{Далее уже знакомая инструкция}\EN{Next, we see the familiar} 
\TT{``MOV R0, \#0''}\RU{, служащая для установки в $0$ возвращаемого значения функции}
\EN{instruction intended to set the \Reg{0} register to $0$}.

\subsection{\OptimizingXcodeIV (\ThumbTwoMode)}

\RU{По умолчанию}\EN{By default} Xcode 4.6.3 
\RU{генерирует код для режима Thumb-2 примерно в такой манере}%
\EN{generates code for Thumb-2 in this manner}:

\begin{lstlisting}[caption=\OptimizingXcodeIV (\ThumbTwoMode)]
__text:00002B6C                   _hello_world
__text:00002B6C 80 B5          PUSH            {R7,LR}
__text:00002B6E 41 F2 D8 30    MOVW            R0, #0x13D8
__text:00002B72 6F 46          MOV             R7, SP
__text:00002B74 C0 F2 00 00    MOVT.W          R0, #0
__text:00002B78 78 44          ADD             R0, PC
__text:00002B7A 01 F0 38 EA    BLX             _puts
__text:00002B7E 00 20          MOVS            R0, #0
__text:00002B80 80 BD          POP             {R7,PC}

...

__cstring:00003E70 48 65 6C 6C 6F 20+aHelloWorld  DCB "Hello world!",0xA,0
\end{lstlisting}

\index{\ThumbTwoMode}
\index{ARM!\Instructions!BL}
\index{ARM!\Instructions!BLX}
\RU{Инструкции \TT{BL} и \TT{BLX} в Thumb, как мы помним, кодируются как пара 16-битных инструкций, 
а в Thumb-2 эти \IT{суррогатные} опкоды расширены так, что новые инструкции кодируются здесь как 
32-битные инструкции}%
\EN{The \TT{BL} and \TT{BLX} instructions in Thumb mode, as we recall, are encoded as a pair
of 16-bit instructions.
In Thumb-2 these \IT{surrogate} opcodes are extended in such a way so that new instructions
may be encoded here as 32-bit instructions}.
\RU{Это можно заметить по тому что опкоды Thumb-2 инструкций всегда начинаются с \TT{0xFx} либо с \TT{0xEx}}%
\EN{That is obvious considering that the opcodes of the Thumb-2 instructions always begin with \TT{0xFx} or \TT{0xEx}}.
\RU{Но в листинге \IDA байты опкода переставлены местами.
Это из-за того, что в процессоре ARM инструкции кодируются так:
в начале последний байт, потом первый (для Thumb и Thumb-2 режима), либо, 
(для инструкций в режиме ARM) в начале четвертый байт, затем третий, второй и первый 
(т.е. другой \gls{endianness})}%
\EN{But in the \IDA listing
the opcode bytes are swapped because for ARM processor the instructions are encoded as follows: 
last byte comes first and after that comes the first one (for Thumb and Thumb-2 modes) 
or for instructions in ARM mode the fourth byte comes first, then the third,
then the second and finally the first (due to different \gls{endianness})}.

\RU{Вот так байты следуют в листингах IDA:}
\EN{So that is how bytes are located in IDA listings:}
\begin{itemize}
\item \RU{для режимов ARM и ARM64}\EN{for ARM and ARM64 modes}: 4-3-2-1;
\item \RU{для режима Thumb}\EN{for Thumb mode}: 2-1;
\item \RU{для пары 16-битных инструкций в режиме Thumb-2}\EN{for 16-bit instructions pair in Thumb-2 mode}: 2-1-4-3.
\end{itemize}

\index{ARM!\Instructions!MOVW}
\index{ARM!\Instructions!MOVT.W}
\index{ARM!\Instructions!BLX}
\RU{Так что мы видим здесь что инструкции \TT{MOVW}, \TT{MOVT.W} и \TT{BLX} начинаются с}
\EN{So as we can see, the \TT{MOVW}, \TT{MOVT.W} and \TT{BLX} instructions begin with} \TT{0xFx}.

\RU{Одна из Thumb-2 инструкций это}\EN{One of the Thumb-2 instructions is}
\TT{MOVW R0, \#0x13D8}\RU{~--- она записывает 16-битное число в младшую часть регистра \Reg{0}, очищая старшие биты.}
\EN{~---it stores a 16-bit value into the lower part of the \Reg{0} register, clearing the higher bits.}

\RU{Ещё}\EN{Also,} \TT{MOVT.W R0, \#0}\RU{~--- эта инструкция работает так же, как и}
\EN{~works just like} 
\TT{MOVT} \RU{из предыдущего примера, но она работает в}\EN{from the previous example only it works in} Thumb-2.

\index{ARM!\RU{переключение режимов}\EN{mode switching}}
\index{ARM!\Instructions!BLX}
\RU{Помимо прочих отличий, здесь используется инструкция}
\EN{Among the other differences, the} \TT{BLX} 
\RU{вместо}\EN{instruction is used in this case instead of the} \TT{BL}.
\RU{Отличие в том, что помимо сохранения адреса возврата в регистре \ac{LR} и передаче управления 
в функцию \puts, происходит смена режима процессора с Thumb/Thumb-2 на режим ARM (либо назад).}
\EN{The difference is that, besides saving the \ac{RA} in the \ac{LR} register and passing control 
to the \puts function, the processor is also switching from Thumb/Thumb-2 mode to ARM mode (or back).}
\RU{Здесь это нужно потому, что инструкция, куда ведет переход, выглядит так (она закодирована в режиме ARM)}%
\EN{This instruction is placed here since the instruction to which control is passed looks like (it is encoded in ARM mode)}:

\begin{lstlisting}
__symbolstub1:00003FEC _puts           ; CODE XREF: _hello_world+E
__symbolstub1:00003FEC 44 F0 9F E5     LDR  PC, =__imp__puts
\end{lstlisting}

\EN{This is essentially jump to place where \puts address is written in imports' section.}
\RU{Это просто переход на место, где записан адрес \puts в секции импортов.}

\RU{Итак, внимательный читатель может задать справедливый вопрос: почему бы не вызывать \puts сразу в 
том же месте кода, где он нужен?}
\EN{So, the observant reader may ask: why not call \puts right at the point in the code where it is needed?}

\RU{Но это не очень выгодно из-за экономии места и вот почему}%
\EN{Because it is not very space-efficient}.

\index{\RU{Динамически подгружаемые библиотеки}\EN{Dynamically loaded libraries}}
\RU{Практически любая программа использует внешние динамические библиотеки (будь то DLL в Windows, .so в *NIX 
либо .dylib в \MacOSX)}\EN{Almost any program uses external dynamic libraries (like DLL in Windows, .so in *NIX or .dylib in \MacOSX)}.
\RU{В динамических библиотеках находятся часто используемые библиотечные функции, в том числе стандартная функция Си \puts}%
\EN{The dynamic libraries contain frequently used library functions, including the standard C-function \puts}.

\index{Relocation}
\RU{В исполняемом бинарном файле}\EN{In an executable binary file} 
(Windows PE .exe, ELF \RU{либо}\EN{or} Mach-O) \RU{имеется секция импортов, список символов (функций либо глобальных переменных) импортируемых из внешних модулей, а также названия самих модулей}%
\EN{an import section is present.
This is a list of symbols (functions or global variables) imported from external modules along with the names of the modules themselves}.

\RU{Загрузчик \ac{OS} загружает необходимые модули и, перебирая импортируемые символы в основном модуле, проставляет правильные адреса каждого символа}%
\EN{The \ac{OS} loader loads all modules it needs and, while enumerating import symbols in the primary module, determines the correct addresses of each symbol}.

\RU{В нашем случае,}\EN{In our case,} \IT{\_\_imp\_\_puts} 
\RU{это 32-битная переменная, куда загрузчик \ac{OS} запишет правильный адрес этой же функции во внешней библиотеке}%
\EN{is a 32-bit variable used by the \ac{OS} loader to store the correct address of the function in an external library}. 
\RU{Так что инструкция \TT{LDR} просто берет 32-битное значение из этой переменной, и, записывая его в регистр \ac{PC}, просто передает туда управление}%
\EN{Then the \TT{LDR} instruction just reads the 32-bit value from this variable and writes it into the \ac{PC} register, passing control to it}.

\RU{Чтобы уменьшить время работы загрузчика \ac{OS}, 
нужно чтобы ему пришлось записать адрес каждого символа только один раз, 
в соответствующее, выделенное для них, место.}
\EN{So, in order to reduce the time the \ac{OS} loader needs for completing this procedure, 
it is good idea if it writes the address of each symbol only once, to a dedicated place.}

\index{thunk-\RU{функции}\EN{functions}}
\RU{К тому же, как мы уже убедились, нельзя одной инструкцией загрузить в регистр 32-битное число без обращений к памяти}%
\EN{Besides, as we have already figured out, it is impossible to load a 32-bit value into a register 
while using only one instruction without a memory access}.
\RU{Так что наиболее оптимально выделить отдельную функцию, работающую в режиме ARM, 
чья единственная цель~--- передавать управление дальше, в динамическую библиотеку.}
\EN{Therefore, the optimal solution is to allocate a separate function working in ARM mode with sole 
goal to pass control to the dynamic library}
\RU{И затем ссылаться на эту короткую функцию из одной инструкции (так называемую \glslink{thunk function}{thunk-функцию}) из Thumb-кода}%
\EN{and then to jump to this short one-instruction function (the so-called \gls{thunk function}) from the Thumb-code}.

\index{ARM!\Instructions!BL}
\RU{Кстати, в предыдущем примере (скомпилированном для режима ARM), переход при помощи инструкции \TT{BL} ведет 
на такую же \glslink{thunk function}{thunk-функцию}, однако режим процессора не переключается (отсюда отсутствие \q{X} в мнемонике инструкции)}%
\EN{By the way, in the previous example (compiled for ARM mode) the control is passed by the \TT{BL} to the 
same \gls{thunk function}.
The processor mode, however, is not being switched (hence the absence of an \q{X} in the instruction mnemonic)}.

\subsubsection{\EN{More about thunk-functions}\RU{Еще о thunk-функциях}}
\index{thunk-\RU{функции}\EN{functions}}

\RU{Thunk-функции трудновато понять, вероятно из-за путаницы в терминах.}
\EN{Thunk-functions are hard to understand, apparently because of misnomer.}
\RU{Вот ещё несколько описаний этих функций:}
\EN{Here are couple more descriptions of these functions:}

\begin{framed}
\begin{quotation}
“A piece of coding which provides an address:”, according to P. Z. Ingerman, 
who invented thunks in 1961 as a way of binding actual parameters to their formal 
definitions in Algol-60 procedure calls. If a procedure is called with an expression 
in the place of a formal parameter, the compiler generates a thunk which computes 
the expression and leaves the address of the result in some standard location.

\dots

Microsoft and IBM have both defined, in their Intel-based systems, a “16-bit environment” 
(with bletcherous segment registers and 64K address limits) and a “32-bit environment” 
(with flat addressing and semi-real memory management). The two environments can both be 
running on the same computer and OS (thanks to what is called, in the Microsoft world, 
WOW which stands for Windows On Windows). MS and IBM have both decided that the process 
of getting from 16- to 32-bit and vice versa is called a “thunk”; for Windows 95, 
there is even a tool THUNK.EXE called a “thunk compiler”.
\end{quotation}
\end{framed}
% TODO FIXME move to bibliography and quote properly above the quote
( \url{http://www.catb.org/jargon/html/T/thunk.html} )

\subsection{ARM64}

\subsubsection{GCC}

\RU{Компилируем пример в}\EN{Let's compile the example using} GCC 4.8.1 \InENRU ARM64:

\lstinputlisting[numbers=left,label=hw_ARM64_GCC,caption=\NonOptimizing GCC 4.8.1 + objdump]
{patterns/01_helloworld/ARM/hw.lst}

\RU{В ARM64 нет режима thumb и thumb-2, только ARM, так что тут только 32-битные инструкции.}
\EN{There are no thumb and thumb-2 modes in ARM64, only ARM, so there are 32-bit instructions only.}
\RU{Регистров тут в 2 раза больше}\EN{Registers count is doubled}: \myref{ARM64_GPRs}.
\RU{64-битные регистры теперь имеют префикс}\EN{64-bit registers has} 
\TT{X-}\EN{ prefixes, while its 32-bit parts}\RU{, а их 32-битные части}\EMDASH{}\TT{W-}.

\index{ARM!\Instructions!STP}
\EN{The }\RU{Инструкция }\TT{STP}\EN{ instruction} (\IT{Store Pair}) 
\RU{сохраняет в стеке сразу два регистра}\EN{saves two registers in the stack simultaneously}: \RegX{29} \InENRU \RegX{30}.
\RU{Конечно, эта инструкция может сохранять эту пару где угодно в памяти, но здесь указан регистр \ac{SP}, так что
пара сохраняется именно в стеке.}
\EN{Of course, this instruction is able to save this pair at a random place of memory, 
but the \ac{SP} register is specified here, so the pair is saved in the stack.}
\RU{Регистры в ARM64 64-битные, каждый имеет длину в 8 байт, так что для хранения двух регистров нужно именно 16 байт.}
\EN{ARM64 registers are 64-bit ones, each has a size of 8 bytes, so one needs 16 bytes for saving two registers.}

\RU{Восклицательный знак после операнда означает, что сначала от \ac{SP} будет отнято 16 и только затем
значения из пары регистров будут записаны в стек.}
\EN{Exclamation mark after operand mean that 16 is to be subtracted from \ac{SP} first, and only then
values from registers pair are to be written into the stack.}
\RU{Это называется}\EN{This is also called} \IT{pre-index}.
\RU{Больше о разнице между}\EN{About the difference between} \IT{post-index} \AndENRU \IT{pre-index} 
\RU{описано здесь}\EN{read here}: \myref{ARM_postindex_vs_preindex}.

\RU{Таким образом, в терминах более знакомого всем процессора x86, первая инструкция~--- это просто аналог 
пары инструкций}
\EN{Hence, in the terms of more familiar x86, the first instruction is just an analogue to pair of}
\TT{PUSH X29} \AndENRU \TT{PUSH X30}.
\RegX{29} \EN{is used as \ac{FP} in ARM64}\RU{в ARM64 используется как \ac{FP}}, \EN{and}\RU{а} \RegX{30} 
\EN{as}\RU{как} \ac{LR}, \RU{поэтому они сохраняются в прологе функции и
восстанавливаются в эпилоге}\EN{so that's why they are saved in the function prologue and restored in the function epilogue}.

\EN{The second instruction copies}\RU{Вторая инструкция копирует} \ac{SP} \InENRU \RegX{29} (\OrENRU \ac{FP}).
\RU{Это нужно для установки стекового фрейма функции}\EN{This is done to set up the function stack frame}.

\label{pointers_ADRP_and_ADD}
\index{ARM!\Instructions!ADRP/ADD pair}
\RU{Инструкции }\TT{ADRP} \AndENRU \ADD \EN{instructions are used to fill the 
string}\RU{нужны для формирования адреса строки} \q{Hello!} \EN{address into the \RegX{0} register}\RU{в регистре \RegX{0}}, 
\RU{ведь первый аргумент функции передается через этот регистр}\EN{because the first function argument is passed
in this register}.
\RU{Но в ARM нет инструкций, при помощи которых можно записать в регистр длинное число}\EN{There are
no instructions, whatsoever, in ARM that can store a large number into a register} 
(\RU{потому что сама длина инструкции ограничена 4-я байтами. Больше об этом здесь}\EN{because the instruction
length is limited to 4 bytes, read more about it here}: \myref{ARM_big_constants_loading}).
\RU{Так что нужно использовать несколько инструкций}\EN{So several instructions must be utilised}.
\RU{Первая инструкция}\EN{The first instruction} (\TT{ADRP}) \EN{writes address of 4Kb page where the string is
located into \RegX{0}}\RU{записывает в \RegX{0} адрес 4-килобайтной страницы где находится строка}, 
\EN{and the the second one}\RU{а вторая} (\ADD) \RU{просто прибавляет к этому адресу остаток}\EN{just adds
reminder to the address}.
\EN{More about that in}\RU{Читайте больше об этом}: \myref{ARM64_relocs}.

\TT{0x400000 + 0x648 = 0x400648}, \EN{and we see our \q{Hello!} C-string in the \TT{.rodata} data segment at this
address}\RU{и мы видим, что в секции данных \TT{.rodata} по этому адресу как раз находится наша
Си-строка \q{Hello!}}.

\index{ARM!\Instructions!BL}
\RU{Затем при помощи инструкции \TT{BL} вызывается \puts. Это уже рассматривалось ранее: \myref{puts}.}
\EN{\puts is called afterwards using \TT{BL} instruction. This was already discussed: \myref{puts}.}

\RU{Инструкция }\MOV \EN{instruction writes $0$ into}\RU{записывает $0$ в} \RegW{0}. 
\RegW{0} \RU{это младшие 32 бита 64-битного регистра}\EN{is low 32 bits of 64-bit} \RegX{0}\EN{ register}:

\begin{center}
\begin{tabular}{ | l | l | }
\hline
\RU{Старшие 32 бита}\EN{High 32-bit part}\ES{Parte alta de 32 bits}\PTBRph{}\PLph{}\ITAph{}\DEph{}\THAph{} & \RU{младшие 32 бита}\EN{low 32-bit part}\ES{parte baja de 32 bits}\PTBRph{}\PLph{}\ITAph{}\DEph{}\THAph{} \\
\hline
\multicolumn{2}{ | c | }{X0} \\
\hline
\multicolumn{1}{ | c | }{} & \multicolumn{1}{ c | }{W0} \\
\hline
\end{tabular}
\end{center}


\RU{А результат функции возвращается через \RegX{0}, и \main возвращает $0$, 
так что вот так готовится возвращаемый результат.}
\EN{The function result is returned via \RegX{0} and \main returns $0$, so that's how the return
result is prepared.}
\RU{Почему именно 32-битная часть}\EN{But why using the 32-bit part}?
\RU{Потому в ARM64, как и в x86-64, тип \Tint оставили 32-битным, для лучшей совместимости.}
\EN{Because \Tint data type in ARM64, just like in x86-64, is still 32-bit, for better compatibility.}
\RU{Следовательно, раз уж функция возвращает 32-битный \Tint, то нужно заполнить только 32 младших бита 
регистра \RegX{0}.}
\EN{So if a function returns 32-bit \Tint, only the low 32 bits of \RegX{0} register has to be filled.}

\RU{Для того, чтобы удостовериться в этом, я немного отредактировал свой пример и перекомпилировал его.}
\EN{In order to verify this, I changed my example slightly and recompiled it.}
\RU{Теперь}\EN{Now} \main \RU{возвращает 64-битное значение}\EN{returns 64-bit value}:

\begin{lstlisting}[caption=\main \RU{возвращающая значение типа}\EN{returning a value of} \TT{uint64\_t}\EN{ type}]
#include <stdio.h>
#include <stdint.h>

uint64_t main()
{
        printf ("Hello!\n");
        return 0;
}
\end{lstlisting}

\RU{Результат точно такой же, только \MOV в той строке теперь выглядит так:}
\EN{The result is the same, but that's how \MOV at that line looks like now:}

\begin{lstlisting}[caption=\NonOptimizing GCC 4.8.1 + objdump]
  4005a4:       d2800000        mov     x0, #0x0                        // #0
\end{lstlisting}

\index{ARM!\Instructions!LDP}
\RU{Далее при помощи инструкции \TT{LDP} (\IT{Load Pair}) восстанавливаются регистры \RegX{29} и \RegX{30}.}
\EN{\TT{LDP} (\IT{Load Pair}) then restores \RegX{29} and \RegX{30} registers.}
\RU{Восклицательного знака после инструкции нет. Это означает, что сначала значения достаются из стека,
и только потом \ac{SP} увеличивается на 16.}
\EN{There is no exclamation mark after the instruction: this implies that the value is first loaded from the stack,
only then \ac{SP} it is increased by 16.}
\RU{Это называется}\EN{This is called} \IT{post-index}.

\index{ARM!\Instructions!RET}
\RU{В ARM64 есть новая инструкция}\EN{New instruction appeared in ARM64}: \RET. 
\RU{Она работает так же как и}\EN{It works just as} \TT{BX LR}, \RU{но там добавлен специальный бит,
подсказывающий процессору, что это именно выход из функции, а не просто переход, чтобы процессор
мог более оптимально исполнять эту инструкцию}\EN{only a special \IT{hint} bit is added, informing the \ac{CPU}
that this is a return from a function, not just another jump instruction, so it can execute it more optimally}.

\RU{Из-за простоты этой функции оптимизирующий GCC генерирует точно такой же код.}
\EN{Due to the simplicity of the function, optimizing GCC generates the very same code.}

