\subsection{ARM}
\label{sec:hw_ARM}

\myindex{\idevices}
\myindex{Raspberry Pi}
\myindex{Xcode}
\myindex{LLVM}
\myindex{Keil}
Für die Experimente mit ARM-Prozessoren wurden verschiedene Compiler genutzt:

\begin{itemize}
\item Verbreitet im Embedded-Bereich: Keil Release 6/2013.

\item Apple Xcode 4.6.3 IDE mit dem LLVM-GCC 4.2-Compiler
\footnote{Tatsächlich nutzt Apple Xcode 4.6.3 GCC als Front-End-Compiler und LLVM 
Code Generator}.

\item GCC 4.9 (Linaro) (für ARM64), verfügbar als Win32-Executable unter \url{http://go.yurichev.com/17325}.

\end{itemize}
Wenn nicht anders angegeben wird immer der 32-Bit ARM-Code (inklusive Thumb und Thumb-2-Mode) genutzt.
Wenn von 64-Bit ARM die Rede ist, dann wird ARM64 geschrieben.

% subsections
%\subsubsection{\NonOptimizingKeilVI (\ARMMode)}

Let's start by compiling our example in Keil:

\begin{lstlisting}
armcc.exe --arm --c90 -O0 1.c 
\end{lstlisting}

\myindex{\IntelSyntax}
The \IT{armcc} compiler produces assembly listings in Intel-syntax, but it has high-level ARM-processor related macros
\footnote{e.g. ARM mode lacks \PUSH/\POP instructions}, 
but it is more important for us to see the instructions \q{as is} so let's see the compiled result in \IDA.

\begin{lstlisting}[caption=\NonOptimizingKeilVI (\ARMMode) \IDA]
.text:00000000             main
.text:00000000 10 40 2D E9    STMFD   SP!, {R4,LR}
.text:00000004 1E 0E 8F E2    ADR     R0, aHelloWorld ; "hello, world"
.text:00000008 15 19 00 EB    BL      __2printf
.text:0000000C 00 00 A0 E3    MOV     R0, #0
.text:00000010 10 80 BD E8    LDMFD   SP!, {R4,PC}

.text:000001EC 68 65 6C 6C+aHelloWorld  DCB "hello, world",0    ; DATA XREF: main+4
\end{lstlisting}

In the example, we can easily see each instruction has a size of 4 bytes.
Indeed, we compiled our code for ARM mode, not for Thumb.

\myindex{ARM!\Instructions!STMFD}
\myindex{ARM!\Instructions!POP}
The very first instruction, \INS{STMFD SP!, \{R4,LR\}}\footnote{\ac{STMFD}}, 
works as an x86 \PUSH instruction, writing the values of two registers (\Reg{4} and \ac{LR}) into the stack.

Indeed, in the output listing from the \IT{armcc} compiler, for the sake of simplification, 
actually shows the \INS{PUSH \{r4,lr\}} instruction.
But that is not quite precise. The \PUSH instruction is only available in Thumb mode.
So, to make things less confusing, we're doing this in \IDA.

This instruction first \glspl{decrement} the \ac{SP} so it points to the place in the stack
that is free for new entries, then it saves the values of the \Reg{4} and \ac{LR} registers at the address
stored in the modified \ac{SP}.

This instruction (like the \PUSH instruction in Thumb mode) is able to save several register values at once which can be very useful.
By the way, this has no equivalent in x86.
It can also be noted that the \TT{STMFD} instruction is a generalization 
of the \PUSH instruction (extending its features), since it can work with any register, not just with \ac{SP}.
In other words, \TT{STMFD} may be used for storing a set of registers at the specified memory address.

\myindex{\PICcode}
\myindex{ARM!\Instructions!ADR}
The \INS{ADR R0, aHelloWorld}
instruction adds or subtracts the value in the \ac{PC} register to the offset where the \TT{hello, world} string is located.
How is the \TT{PC} register used here, one might ask?
This is called \q{\PICcode}\footnote{Read more about it in relevant section~(\myref{sec:PIC})}.

Such code can be be executed at a non-fixed address in memory.
In other words, this is \ac{PC}-relative addressing.
The \INS{ADR} instruction takes into account the difference between the address of this instruction and the address where the string is located.
This difference (offset) is always to be the same, no matter at what address our code is loaded by the \ac{OS}.
That's why all we need is to add the address of the current instruction (from \ac{PC}) in order to get the absolute memory address of our C-string.

\myindex{ARM!\Registers!Link Register}
\myindex{ARM!\Instructions!BL}
\INS{BL \_\_2printf}\footnote{Branch with Link} instruction calls the \printf function. 
Here's how this instruction works: 

\begin{itemize}
\item store the address following the \INS{BL} instruction (\TT{0xC}) into the \ac{LR};
\item then pass the control to \printf by writing its address into the \ac{PC} register.
\end{itemize}

When \printf finishes its execution it must have information about where it needs to return the control to.
That's why each function passes control to the address stored in the \ac{LR} register.

That is a difference between \q{pure} \ac{RISC}-processors like ARM and \ac{CISC}-processors like x86,
where the return address is usually stored on the stack.
Read more about this in next section~(\myref{sec:stack}).

By the way, an absolute 32-bit address or offset cannot be encoded in the 32-bit \TT{BL} instruction because
it only has space for 24 bits.
As we may remember, all ARM-mode instructions have a size of 4 bytes (32 bits).
Hence, they can only be located on 4-byte boundary addresses.
This implies that the last 2 bits of the instruction address (which are always zero bits) may be omitted.
In summary, we have 26 bits for offset encoding. This is enough to encode $current\_PC \pm{} \approx{}32M$.

\myindex{ARM!\Instructions!MOV}
Next, the \INS{MOV R0, \#0}\footnote{Meaning MOVe} instruction just writes 0 into the \Reg{0} register.
That's because our C-function returns 0 and the return value is to be placed in the \Reg{0} register.

\myindex{ARM!\Registers!Link Register}
\myindex{ARM!\Instructions!LDMFD}
\myindex{ARM!\Instructions!POP}
The last instruction \INS{LDMFD SP!, {R4,PC}}\footnote{\ac{LDMFD} is an inverse instruction of \ac{STMFD}}.
It loads values from the stack (or any other memory place) in order to save them into \Reg{4} and \ac{PC}, and \glslink{increment}{increments} the \gls{stack pointer} \ac{SP}.
It works like \POP here.\\
N.B. The very first instruction \TT{STMFD} saved the \Reg{4} and \ac{LR} registers pair on the stack, but \Reg{4} and \ac{PC} are \IT{restored} during the \TT{LDMFD} execution.

As we already know, the address of the place where each function must return control to is usually saved in the \ac{LR} register.
The very first instruction saves its value in the stack because the same register will be used by our
\main function when calling \printf.
In the function's end, this value can be written directly to the \ac{PC} register, thus passing control to where our function was called.

Since \main is usually the primary function in \CCpp,
the control will be returned to the \ac{OS} loader or to a point in a \ac{CRT},
or something like that.

All that allows omitting the \INS{BX LR} instruction at the end of the function.

\myindex{ARM!DCB}
\TT{DCB} is an assembly language directive defining an array of bytes or ASCII strings, akin to the DB directive 
in the x86-assembly language.


%\subsectionold{\NonOptimizingKeilVI (\ThumbMode)}

Let's compile the same example using Keil in Thumb mode:

\begin{lstlisting}
armcc.exe --thumb --c90 -O0 1.c 
\end{lstlisting}

We are getting (in \IDA):

\begin{lstlisting}[caption=\NonOptimizingKeilVI (\ThumbMode) + \IDA]
.text:00000000             main
.text:00000000 10 B5          PUSH    {R4,LR}
.text:00000002 C0 A0          ADR     R0, aHelloWorld ; "hello, world"
.text:00000004 06 F0 2E F9    BL      __2printf
.text:00000008 00 20          MOVS    R0, #0
.text:0000000A 10 BD          POP     {R4,PC}

.text:00000304 68 65 6C 6C+aHelloWorld  DCB "hello, world",0    ; DATA XREF: main+2
\end{lstlisting}

We can easily spot the 2-byte (16-bit) opcodes. This is, as was already noted, Thumb.
\myindex{ARM!\Instructions!BL}
The \TT{BL} instruction, however, consists of two 16-bit instructions.
This is because it is impossible to load an offset for the \printf function while using the small space in one 16-bit opcode.
Therefore, the first 16-bit instruction loads the higher 10 bits of the offset and the second instruction loads 
the lower 11 bits of the offset.

% TODO:
% BL has space for 11 bits, so if we don't encode the lowest bit,
% then we should get 11 bits for the upper half, and 12 bits for the lower half.
% And the highest bit encodes the sign, so the destination has to be within
% \pm 4M of current_PC.
% This may be less if adding the lower half does not carry over,
% but I'm not sure --all my programs have 0 for the upper half,
% and don't carry over for the lower half.
% It would be interesting to check where __2printf is located relative to 0x8
% (I think the program counter is the next instruction on a multiple of 4
% for THUMB).
% The lower 11 bytes of the BL instructions and the even bit are
% 000 0000 0110 | 001 0010 1110 0 = 000 0000 0110 0010 0101 1100 = 0x00625c,
% so __2printf should be at 0x006264.
% But if we only have 10 and 11 bits, then the offset would be:
% 00 0000 0110 | 01 0010 1110 0 = 0 0000 0011 0010 0101 1100 = 0x00325c,
% so __2printf should be at 0x003264.
% In this case, though, the new program counter can only be 1M away,
% because of the highest bit is used for the sign.

As was noted, all instructions in Thumb mode have a size of 2 bytes (or 16 bits).
This implies it is impossible for a Thumb-instruction to be at an odd address whatsoever.
Given the above, the last address bit may be omitted while encoding instructions.

In summary, the \TT{BL} Thumb-instruction can encode an address in $current\_PC \pm{}\approx{}2M$.

\myindex{ARM!\Instructions!PUSH}
\myindex{ARM!\Instructions!POP}
As for the other instructions in the function: \PUSH and \POP work here just like the described \TT{STMFD}/\TT{LDMFD} only the \ac{SP} register is not mentioned explicitly here.
\TT{ADR} works just like in the previous example.
\TT{MOVS} writes 0 into the \Reg{0} register in order to return zero.


%\subsectionold{\OptimizingXcodeIV (\ARMMode)}

Xcode 4.6.3 without optimization turned on produces a lot of redundant code so we'll study optimized output, where the instruction count is as small as possible, setting the compiler switch \Othree.

\begin{lstlisting}[caption=\OptimizingXcodeIV (\ARMMode)]
__text:000028C4             _hello_world
__text:000028C4 80 40 2D E9   STMFD           SP!, {R7,LR}
__text:000028C8 86 06 01 E3   MOV             R0, #0x1686
__text:000028CC 0D 70 A0 E1   MOV             R7, SP
__text:000028D0 00 00 40 E3   MOVT            R0, #0
__text:000028D4 00 00 8F E0   ADD             R0, PC, R0
__text:000028D8 C3 05 00 EB   BL              _puts
__text:000028DC 00 00 A0 E3   MOV             R0, #0
__text:000028E0 80 80 BD E8   LDMFD           SP!, {R7,PC}

__cstring:00003F62 48 65 6C 6C+aHelloWorld_0  DCB "Hello world!",0
\end{lstlisting}

The instructions \TT{STMFD} and \TT{LDMFD} are already familiar to us.

\myindex{ARM!\Instructions!MOV}

The \MOV instruction just writes the number \TT{0x1686} into the \Reg{0} register.
This is the offset pointing to the \q{Hello world!} string.

The \TT{R7} register (as it is standardized in \IOSABI) is a frame pointer. More on that below.

\myindex{ARM!\Instructions!MOVT}
The \TT{MOVT R0, \#0} (MOVe Top) instruction writes 0 into higher 16 bits of the register.
The issue here is that the generic \MOV instruction in ARM mode may write only the lower 16 bits of the register.

Remember, all instruction opcodes in ARM mode are limited in size to 32 bits. Of course, this limitation is not related to moving data between registers.
That's why an additional instruction \TT{MOVT} exists for writing into the higher bits (from 16 to 31 inclusive).
Its usage here, however, is redundant because the \TT{MOV R0, \#0x1686} instruction above cleared the higher part of the register.
This is probably a shortcoming of the compiler.
% TODO:
% I think, more specifically, the string is not put in the text section,
% ie. the compiler is actually not using position-independent code,
% as mentioned in the next paragraph.
% MOVT is used because the assembly code is generated before the relocation,
% so the location of the string is not yet known,
% and the high bits may still be needed.

\myindex{ARM!\Instructions!ADD}
The \TT{ADD R0, PC, R0} instruction adds the value in the \ac{PC} to the value in the \Reg{0}, to calculate the absolute address of the \q{Hello world!} string. 
As we already know, it is \q{\PICcode} so this correction is essential here.

The \INS{BL} instruction calls the \puts function instead of \printf.

\label{puts}
\myindex{\CStandardLibrary!puts()}
\myindex{puts() instead of printf()}

GCC replaced the first \printf call with \puts.
Indeed: \printf with a sole argument is almost analogous to \puts. 

\IT{Almost}, because the two functions are producing the same result only in case the 
string does not contain printf format identifiers starting with \IT{\%}. 
In case it does, the effect of these two functions would be different
\footnote{It has also to be noted the \puts does not require a `\textbackslash{}n' new line symbol 
at the end of a string, so we do not see it here.}.

Why did the compiler replace the \printf with \puts? Probably because \puts is faster
\footnote{\href{http://go.yurichev.com/17063}{ciselant.de/projects/gcc\_printf/gcc\_printf.html}}. 

Because it just passes characters to \gls{stdout} without comparing every one of them with the \IT{\%} symbol.

Next, we see the familiar \TT{MOV R0, \#0} instruction intended to set the \Reg{0} register to 0.

%\subsectionold{\OptimizingXcodeIV (\ThumbTwoMode)}

By default Xcode 4.6.3 generates code for Thumb-2 in this manner:

\begin{lstlisting}[caption=\OptimizingXcodeIV (\ThumbTwoMode)]
__text:00002B6C                   _hello_world
__text:00002B6C 80 B5          PUSH            {R7,LR}
__text:00002B6E 41 F2 D8 30    MOVW            R0, #0x13D8
__text:00002B72 6F 46          MOV             R7, SP
__text:00002B74 C0 F2 00 00    MOVT.W          R0, #0
__text:00002B78 78 44          ADD             R0, PC
__text:00002B7A 01 F0 38 EA    BLX             _puts
__text:00002B7E 00 20          MOVS            R0, #0
__text:00002B80 80 BD          POP             {R7,PC}

...

__cstring:00003E70 48 65 6C 6C 6F 20+aHelloWorld  DCB "Hello world!",0xA,0
\end{lstlisting}

% Q: If you subtract 0x13D8 from 0x3E70,
% you actually get a location that is not in this function, or in _puts.
% How is PC-relative addressing done in THUMB2?
% A: it's not Thumb-related. there are just mess with two different segments. TODO: rework this listing.

\myindex{\ThumbTwoMode}
\myindex{ARM!\Instructions!BL}
\myindex{ARM!\Instructions!BLX}

The \TT{BL} and \TT{BLX} instructions in Thumb mode, as we recall, are encoded as a pair of 16-bit instructions.
In Thumb-2 these \IT{surrogate} opcodes are extended in such a way so that new instructions may be encoded here as 32-bit instructions.

That is obvious considering that the opcodes of the Thumb-2 instructions always begin with \TT{0xFx} or \TT{0xEx}.

But in the \IDA listing
the opcode bytes are swapped because for ARM processor the instructions are encoded as follows: 
last byte comes first and after that comes the first one (for Thumb and Thumb-2 modes) 
or for instructions in ARM mode the fourth byte comes first, then the third,
then the second and finally the first (due to different \gls{endianness}).

So that is how bytes are located in IDA listings:
\begin{itemize}
\item for ARM and ARM64 modes: 4-3-2-1;
\item for Thumb mode: 2-1;
\item for 16-bit instructions pair in Thumb-2 mode: 2-1-4-3.
\end{itemize}

\myindex{ARM!\Instructions!MOVW}
\myindex{ARM!\Instructions!MOVT.W}
\myindex{ARM!\Instructions!BLX}

So as we can see, the \TT{MOVW}, \TT{MOVT.W} and \TT{BLX} instructions begin with \TT{0xFx}.

One of the Thumb-2 instructions is \TT{MOVW R0, \#0x13D8} ~---it stores a 16-bit value into the lower part of the \Reg{0} register, clearing the higher bits.

Also, \TT{MOVT.W R0, \#0} ~works just like \TT{MOVT} from the previous example only it works in Thumb-2.

\myindex{ARM!mode switching}
\myindex{ARM!\Instructions!BLX}

Among the other differences, the \TT{BLX} instruction is used in this case instead of the \TT{BL}.

The difference is that, besides saving the \ac{RA} in the \ac{LR} register and passing control 
to the \puts function, the processor is also switching from Thumb/Thumb-2 mode to ARM mode (or back).

This instruction is placed here since the instruction to which control is passed looks like (it is encoded in ARM mode):

\begin{lstlisting}
__symbolstub1:00003FEC _puts           ; CODE XREF: _hello_world+E
__symbolstub1:00003FEC 44 F0 9F E5     LDR  PC, =__imp__puts
\end{lstlisting}

This is essentially a jump to the place where the address of \puts is written in the imports' section.

So, the observant reader may ask: why not call \puts right at the point in the code where it is needed?

Because it is not very space-efficient.

\myindex{Dynamically loaded libraries}
Almost any program uses external dynamic libraries (like DLL in Windows, .so in *NIX or .dylib in \MacOSX).
The dynamic libraries contain frequently used library functions, including the standard C-function \puts.

\myindex{Relocation}
In an executable binary file (Windows PE .exe, ELF or Mach-O) an import section is present.
This is a list of symbols (functions or global variables) imported from external modules along with the names of the modules themselves.

The \ac{OS} loader loads all modules it needs and, while enumerating import symbols in the primary module, determines the correct addresses of each symbol.

In our case, \IT{\_\_imp\_\_puts} is a 32-bit variable used by the \ac{OS} loader to store the correct address of the function in an external library. 
Then the \TT{LDR} instruction just reads the 32-bit value from this variable and writes it into the \ac{PC} register, passing control to it.

So, in order to reduce the time the \ac{OS} loader needs for completing this procedure, 
it is good idea to write the address of each symbol only once, to a dedicated place.

\myindex{thunk-functions}
Besides, as we have already figured out, it is impossible to load a 32-bit value into a register 
while using only one instruction without a memory access.

Therefore, the optimal solution is to allocate a separate function working in ARM mode with the sole 
goal of passing control to the dynamic library and then to jump to this short one-instruction function (the so-called \gls{thunk function}) from the Thumb-code.

\myindex{ARM!\Instructions!BL}
By the way, in the previous example (compiled for ARM mode) the control is passed by the \TT{BL} to the 
same \gls{thunk function}.
The processor mode, however, is not being switched (hence the absence of an \q{X} in the instruction mnemonic).

\subsubsectionold{More about thunk-functions}
\myindex{thunk-functions}

Thunk-functions are hard to understand, apparently, because of a misnomer.
The simplest way to understand it as adaptors or convertors of one type of jack to another.
For example, an adaptor allowing the insertion of a British power plug into an American wall socket, or vice-versa. 
Thunk functions are also sometimes called \IT{wrappers}.

Here are a couple more descriptions of these functions:

\begin{framed}
\begin{quotation}
“A piece of coding which provides an address:”, according to P. Z. Ingerman, 
who invented thunks in 1961 as a way of binding actual parameters to their formal 
definitions in Algol-60 procedure calls. If a procedure is called with an expression 
in the place of a formal parameter, the compiler generates a thunk which computes 
the expression and leaves the address of the result in some standard location.

\dots

Microsoft and IBM have both defined, in their Intel-based systems, a “16-bit environment” 
(with bletcherous segment registers and 64K address limits) and a “32-bit environment” 
(with flat addressing and semi-real memory management). The two environments can both be 
running on the same computer and OS (thanks to what is called, in the Microsoft world, 
WOW which stands for Windows On Windows). MS and IBM have both decided that the process 
of getting from 16- to 32-bit and vice versa is called a “thunk”; for Windows 95, 
there is even a tool, THUNK.EXE, called a “thunk compiler”.
\end{quotation}
\end{framed}
% TODO FIXME move to bibliography and quote properly above the quote
( \href{http://go.yurichev.com/17362}{The Jargon File} )


%\subsubsection{ARM64}

\myparagraph{GCC}

Let's compile the example using GCC 4.8.1 in ARM64:

\lstinputlisting[numbers=left,label=hw_ARM64_GCC,caption=\NonOptimizing GCC 4.8.1 + objdump]{patterns/01_helloworld/ARM/hw.lst}

There are no Thumb and Thumb-2 modes in ARM64, only ARM, so there are 32-bit instructions only.
The Register count is doubled: \myref{ARM64_GPRs}.
64-bit registers have \TT{X-} prefixes, while its 32-bit parts---\TT{W-}.

\myindex{ARM!\Instructions!STP}
The \TT{STP} instruction (\IT{Store Pair}) 
saves two registers in the stack simultaneously: \RegX{29} in \RegX{30}.

Of course, this instruction is able to save this pair at an arbitrary place in memory, 
but the \ac{SP} register is specified here, so the pair is saved in the stack.

ARM64 registers are 64-bit ones, each has a size of 8 bytes, so one needs 16 bytes for saving two registers.

The exclamation mark (``!'') after the operand means that 16 is to be subtracted from \ac{SP} first, and only then
are values from register pair to be written into the stack.
This is also called \IT{pre-index}.
About the difference between \IT{post-index} and \IT{pre-index} 
read here: \myref{ARM_postindex_vs_preindex}.

Hence, in terms of the more familiar x86, the first instruction is just an analogue to a pair of
\TT{PUSH X29} and \TT{PUSH X30}.
\RegX{29} is used as \ac{FP} in ARM64, and \RegX{30} 
as \ac{LR}, so that's why they are saved in the function prologue and restored in the function epilogue.

The second instruction copies \ac{SP} in \RegX{29} (or \ac{FP}).
This is made so to set up the function stack frame.

\label{pointers_ADRP_and_ADD}
\myindex{ARM!\Instructions!ADRP/ADD pair}
\TT{ADRP} and \ADD instructions are used to fill the 
address of the string \q{Hello!} into the \RegX{0} register, 
because the first function argument is passed
in this register.
There are no instructions, whatsoever, in ARM that can store a large number into a register (because the instruction
length is limited to 4 bytes, read more about it here: \myref{ARM_big_constants_loading}).
So several instructions must be utilized. The first instruction (\TT{ADRP}) writes the address of the 4KiB page, where the string is
located, into \RegX{0}, 
and the second one (\ADD) just adds the remainder to the address.
More about that in: \myref{ARM64_relocs}.

\TT{0x400000 + 0x648 = 0x400648}, and we see our \q{Hello!} C-string in the \TT{.rodata} data segment at this address.

\myindex{ARM!\Instructions!BL}

\puts is called afterwards using the \TT{BL} instruction. This was already discussed: \myref{puts}.

\MOV writes 0 into \RegW{0}. 
\RegW{0} is the lower 32 bits of the 64-bit \RegX{0} register:

\begin{center}
\begin{tabular}{ | l | l | }
\hline
\RU{Старшие 32 бита}\EN{High 32-bit part}\ES{Parte alta de 32 bits}\PTBRph{}\PLph{}\ITAph{}\DEph{}\THAph{} & \RU{младшие 32 бита}\EN{low 32-bit part}\ES{parte baja de 32 bits}\PTBRph{}\PLph{}\ITAph{}\DEph{}\THAph{} \\
\hline
\multicolumn{2}{ | c | }{X0} \\
\hline
\multicolumn{1}{ | c | }{} & \multicolumn{1}{ c | }{W0} \\
\hline
\end{tabular}
\end{center}


The function result is returned via \RegX{0} and \main returns 0, so that's how the return result is prepared.
But why use the 32-bit part?

Because the \Tint data type in ARM64, just like in x86-64, is still 32-bit, for better compatibility.

So if a function returns a 32-bit \Tint, only the lower 32 bits of \RegX{0} register have to be filled.

In order to verify this, let's change this example slightly and recompile it.
Now \main returns a 64-bit value:

\begin{lstlisting}[caption=\main returning a value of \TT{uint64\_t} type]
#include <stdio.h>
#include <stdint.h>

uint64_t main()
{
        printf ("Hello!\n");
        return 0;
}
\end{lstlisting}

The result is the same, but that's how \MOV at that line looks like now:

\begin{lstlisting}[caption=\NonOptimizing GCC 4.8.1 + objdump]
  4005a4:       d2800000        mov     x0, #0x0      // #0
\end{lstlisting}

\myindex{ARM!\Instructions!LDP}

\INS{LDP} (\IT{Load Pair}) then restores the \RegX{29} and \RegX{30} registers.

There is no exclamation mark after the instruction: this implies that the value is first loaded from the stack,
and only then is \ac{SP} increased by 16.
This is called \IT{post-index}.

\myindex{ARM!\Instructions!RET}
A new instruction appeared in ARM64: \RET. 
It works just as \TT{BX LR}, only a special \IT{hint} bit is added, informing the \ac{CPU}
that this is a return from a function, not just another jump instruction, so it can execute it more optimally.

Due to the simplicity of the function, optimizing GCC generates the very same code.

