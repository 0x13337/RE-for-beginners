\subsection{\OptimizingXcodeIV (\ARMMode)}

Xcode 4.6.3 \RU{без включенной оптимизации выдает слишком много лишнего кода, поэтому включим оптимизацию компилятора (ключ \Othree), потому что там меньше инструкций.}
\EN{without optimization turned on produces a lot of redundant code so we'll study optimized output, where the instruction count is as small as possible, setting the compiler switch \Othree.}

\begin{lstlisting}[caption=\OptimizingXcodeIV (\ARMMode)]
__text:000028C4             _hello_world
__text:000028C4 80 40 2D E9   STMFD           SP!, {R7,LR}
__text:000028C8 86 06 01 E3   MOV             R0, #0x1686
__text:000028CC 0D 70 A0 E1   MOV             R7, SP
__text:000028D0 00 00 40 E3   MOVT            R0, #0
__text:000028D4 00 00 8F E0   ADD             R0, PC, R0
__text:000028D8 C3 05 00 EB   BL              _puts
__text:000028DC 00 00 A0 E3   MOV             R0, #0
__text:000028E0 80 80 BD E8   LDMFD           SP!, {R7,PC}

__cstring:00003F62 48 65 6C 6C+aHelloWorld_0  DCB "Hello world!",0
\end{lstlisting}

\RU{Инструкции}\EN{The instructions} \TT{STMFD} \AndENRU \TT{LDMFD} \RU{нам уже знакомы}\EN{are already familiar to us}.

\index{ARM!\Instructions!MOV}
\RU{Инструкция \MOV просто записывает число \TT{0x1686} в регистр \Reg{0}~--- это смещение, указывающее на строку \q{Hello world!}}%
\EN{The \MOV instruction just writes the number \TT{0x1686} into the \Reg{0} register.
This is the offset pointing to the \q{Hello world!} string}.

\RU{Регистр \TT{R7} (по стандарту, принятому в \cite{IOSABI}) это frame pointer, о нем будет рассказано позже.}
\EN{The \TT{R7} register (as it is standardized in \cite{IOSABI}) is a frame pointer. More on that below.}

\index{ARM!\Instructions!MOVT}
\RU{Инструкция}\EN{The} \TT{MOVT R0, \#0} (MOVe Top) \RU{записывает 0 в старшие 16 бит регистра}%
\EN{instruction writes 0 into higher 16 bits of the register}.
\RU{Дело в том, что обычная инструкция \MOV в режиме ARM может записывать какое-либо значение только в младшие 16 бит регистра, ведь в ней нельзя закодировать больше}%
\EN{The issue here is that the generic \MOV instruction in ARM mode may write only the lower 16 bits of the register}.
\RU{Помните, что в режиме ARM опкоды всех инструкций ограничены длиной в 32 бита. Конечно, это ограничение не касается перемещений данных между регистрами.}
\EN{Remember, all instruction opcodes in ARM mode are limited in size to 32 bits. Of course, this limitation is not related to moving data between registers.}
\RU{Поэтому для записи в старшие биты (с 16-го по 31-й включительно) существует дополнительная команда \TT{MOVT}}%
\EN{That's why an additional instruction \TT{MOVT} exists for writing into the higher bits (from 16 to 31 inclusive)}.
\RU{Впрочем, здесь её использование избыточно, потому что инструкция \TT{MOV R0, \#0x1686} выше и так обнулила старшую часть регистра}%
\EN{Its usage here, however, is redundant because the \TT{MOV R0, \#0x1686} instruction above cleared the higher part of the register}.
\RU{Возможно, это недочет компилятора}\EN{This is probably a shortcoming of the compiler}.

\index{ARM!\Instructions!ADD}
\RU{Инструкция}\EN{The} \TT{ADD R0, PC, R0} \RU{прибавляет \ac{PC} к \Reg{0} для вычисления действительного адреса строки \q{Hello world!}. Как нам уже известно, это \q{\PICcode}, поэтому такая корректива необходима}%
\EN{instruction adds the value in the \ac{PC} to the value in the \Reg{0}, to calculate absolute address of the \q{Hello world!} string. 
As we already know, it is \q{\PICcode} so this correction is essential here}.

\RU{Инструкция \TT{BL} вызывает \puts вместо \printf}%
\EN{The \TT{BL} instruction calls the \puts function instead of \printf}.

\label{puts}
\index{\CStandardLibrary!puts()}
\index{puts() \RU{вместо}\EN{instead of} printf()}
\RU{Компилятор заменил вызов \printf на \puts. 
Действительно, \printf с одним аргументом это почти аналог \puts.}
\EN{GCC replaced the first \printf call with \puts.
Indeed: \printf with a sole argument is almost analogous to \puts.} 
\RU{\IT{Почти}, если принять условие, что в строке не будет управляющих символов \printf, 
начинающихся со знака процента. Тогда эффект от работы этих двух функций будет разным}%
\EN{\IT{Almost}, because the two functions are producing the same result only in case the 
string does not contain printf format identifiers starting with \IT{\%}. 
In case it does, the effect of these two functions would be different}%
\footnote{
\RU{Также нужно заметить, что \puts не требует символа перевода строки `\textbackslash{}n' в конце строки,
поэтому его здесь нет.}
\EN{It has also to be noted the \puts does not require a `\textbackslash{}n' new line symbol 
at the end of a string, so we do not see it here.}}.

\RU{Зачем компилятор заменил один вызов на другой? Наверное потому что \puts работает быстрее}%
\EN{Why did the compiler replace the \printf with \puts? Probably because \puts is faster}%
\footnote{\url{http://go.yurichev.com/17063}}. 
\RU{\puts проталкивает символы в \gls{stdout} не сравнивая каждый со знаком процента.}
\EN{\puts works faster because it just passes characters to \gls{stdout} without comparing every one of them with the \IT{\%} symbol.}

\RU{Далее уже знакомая инструкция}\EN{Next, we see the familiar} 
\TT{MOV R0, \#0}\RU{, служащая для установки в 0 возвращаемого значения функции}%
\EN{instruction intended to set the \Reg{0} register to 0}.
