\subsubsection{\OptimizingXcodeIV (\ARMMode)}

Xcode 4.6.3 without optimization turned on produces a lot of redundant code so we'll study optimized output, where the instruction count is as small as possible, setting the compiler switch \Othree.

\begin{lstlisting}[caption=\OptimizingXcodeIV (\ARMMode)]
__text:000028C4             _hello_world
__text:000028C4 80 40 2D E9   STMFD           SP!, {R7,LR}
__text:000028C8 86 06 01 E3   MOV             R0, #0x1686
__text:000028CC 0D 70 A0 E1   MOV             R7, SP
__text:000028D0 00 00 40 E3   MOVT            R0, #0
__text:000028D4 00 00 8F E0   ADD             R0, PC, R0
__text:000028D8 C3 05 00 EB   BL              _puts
__text:000028DC 00 00 A0 E3   MOV             R0, #0
__text:000028E0 80 80 BD E8   LDMFD           SP!, {R7,PC}

__cstring:00003F62 48 65 6C 6C+aHelloWorld_0  DCB "Hello world!",0
\end{lstlisting}

The instructions \TT{STMFD} and \TT{LDMFD} are already familiar to us.

\myindex{ARM!\Instructions!MOV}

The \MOV instruction just writes the number \TT{0x1686} into the \Reg{0} register.
This is the offset pointing to the \q{Hello world!} string.

The \TT{R7} register (as it is standardized in \IOSABI) is a frame pointer. More on that below.

\myindex{ARM!\Instructions!MOVT}
The \TT{MOVT R0, \#0} (MOVe Top) instruction writes 0 into higher 16 bits of the register.
The issue here is that the generic \MOV instruction in ARM mode may write only the lower 16 bits of the register.

Remember, all instruction opcodes in ARM mode are limited in size to 32 bits. Of course, this limitation is not related to moving data between registers.
That's why an additional instruction \TT{MOVT} exists for writing into the higher bits (from 16 to 31 inclusive).
Its usage here, however, is redundant because the \TT{MOV R0, \#0x1686} instruction above cleared the higher part of the register.
This is supposedly a shortcoming of the compiler.
% TODO:
% I think, more specifically, the string is not put in the text section,
% ie. the compiler is actually not using position-independent code,
% as mentioned in the next paragraph.
% MOVT is used because the assembly code is generated before the relocation,
% so the location of the string is not yet known,
% and the high bits may still be needed.

\myindex{ARM!\Instructions!ADD}
The \TT{ADD R0, PC, R0} instruction adds the value in the \ac{PC} to the value in the \Reg{0}, to calculate the absolute address of the \q{Hello world!} string. 
As we already know, it is \q{\PICcode} so this correction is essential here.

The \INS{BL} instruction calls the \puts function instead of \printf.

\label{puts}
\myindex{\CStandardLibrary!puts()}
\myindex{puts() instead of printf()}

GCC replaced the first \printf call with \puts.
Indeed: \printf with a sole argument is almost analogous to \puts. 

\IT{Almost}, because the two functions are producing the same result only in case the 
string does not contain printf format identifiers starting with \IT{\%}. 
In case it does, the effect of these two functions would be different
\footnote{It has also to be noted the \puts does not require a `\textbackslash{}n' new line symbol 
at the end of a string, so we do not see it here.}.

Why did the compiler replace the \printf with \puts? Presumably because \puts is faster
\footnote{\href{http://go.yurichev.com/17063}{ciselant.de/projects/gcc\_printf/gcc\_printf.html}}. 

Because it just passes characters to \gls{stdout} without comparing every one of them with the \IT{\%} symbol.

Next, we see the familiar \TT{MOV R0, \#0} instruction intended to set the \Reg{0} register to 0.
