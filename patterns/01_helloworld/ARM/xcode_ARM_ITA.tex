\subsection{\OptimizingXcodeIV (\ARMMode)}

Xcode 4.6.3 senza ottimizzazioni produce un sacco di codice ridondante, percio' studieremo l'output ottimizzato in cui le 
le istruzioni sono il meno possibile, settando lo switch del compilatore \Othree.

\begin{lstlisting}[caption=\OptimizingXcodeIV (\ARMMode)]
__text:000028C4             _hello_world
__text:000028C4 80 40 2D E9   STMFD           SP!, {R7,LR}
__text:000028C8 86 06 01 E3   MOV             R0, #0x1686
__text:000028CC 0D 70 A0 E1   MOV             R7, SP
__text:000028D0 00 00 40 E3   MOVT            R0, #0
__text:000028D4 00 00 8F E0   ADD             R0, PC, R0
__text:000028D8 C3 05 00 EB   BL              _puts
__text:000028DC 00 00 A0 E3   MOV             R0, #0
__text:000028E0 80 80 BD E8   LDMFD           SP!, {R7,PC}

__cstring:00003F62 48 65 6C 6C+aHelloWorld_0  DCB "Hello world!",0
\end{lstlisting}

Le istruzioni \TT{STMFD} e \TT{LDMFD} ci sono gia' familiari.

\myindex{ARM!\Instructions!MOV}

L'istruzione \MOV scrive il numero \TT{0x1686} nel registro \Reg{0} .
Questo e' l'offset che punta alla stringa \q{Hello world!} .

Il registro \TT{R7} (per come standardizzato in \cite{IOSABI}) e' un frame pointer. Maggiori informazioni in basso.

\myindex{ARM!\Instructions!MOVT}
L'istruzione \TT{MOVT R0, \#0} (MOVe Top) scrive 0 nei 16 bit alti (higher 16 bits) del registro.
Il problema qui e' che l'istruzione generico \MOV in ARM mode potrebbe scrivere solo i 16 bit bassi del registro.

Ricorda che tutti gli opcode delle istruzioni in ARM mode sono limitati ad una lunghezza di 32 bit. Ovviamente questa limitazione non riguarda lo spostamento dei dati tra registri.
E questo spiega perche' esiste l'istruzione aggiuntiva \TT{MOVT} per scrivere nelle parti alte (da 16 a 31, inclusi).
Il suo uso qui e' comunque ridondante, perche' l'istruzione \TT{MOV R0, \#0x1686} di sopra ha azzerato la parte alta del registro.
E' probabilmente un difetto/svista del compilatore.
% TODO:
% I think, more specifically, the string is not put in the text section,
% ie. the compiler is actually not using position-independent code,
% as mentioned in the next paragraph.
% MOVT is used because the assembly code is generated before the relocation,
% so the location of the string is not yet known,
% and the high bits may still be needed.

\myindex{ARM!\Instructions!ADD}
L'istruzione \TT{ADD R0, PC, R0} aggiunge il valore in \ac{PC} al valore in \Reg{0}, per calcolare l'indirizzo assoluto della stringa \q{Hello world!}. 
Come sappiamo, si tratta di \q{\PICcode} e quindi questa correzione risulta essenziale in questo caso.

L'istruzione \INS{BL} chiama la funzione \puts ivece di \printf.

\label{puts}
\myindex{\CStandardLibrary!puts()}
\myindex{puts() instead of printf()}

GCC ha sostituito la prima chiamata a \printf con \puts.
Infatti: \printf con un solo argomento e' quasi analoga a \puts. 

\IT{Quasi}, perche' le due funzioni producono lo stesso risultato solo nel caso in cui la stringa non contiene 
identificatori di formato (format identifiers) che iniziano con \IT{\%}. 
In caso contrario l'effetto di queste due funzioni sarebbe diverso
\footnote{Bisogna anche notare che \puts non richiede un simbolo new line `\textbackslash{}n' alla fine della stringa,
per questo non lo vediamo qui.}.

Perche' il compilatore ha sostituito \printf con \puts? Probabilment perche' \puts e' piu' veloce
\footnote{\href{http://go.yurichev.com/17063}{ciselant.de/projects/gcc\_printf/gcc\_printf.html}}. 

Poiche' passa direttamente i caratteri a \gls{stdout} senza confrontare ciascuno di essi con il simbolo \IT{\%}.

Andando avanti, vediamo la familiare istruzione \TT{MOV R0, \#0} che imposta il registro \Reg{0} a to 0.
