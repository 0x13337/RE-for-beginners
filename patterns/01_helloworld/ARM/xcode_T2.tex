\subsection{\OptimizingXcodeIV (\ThumbTwoMode)}

\RU{По умолчанию}\EN{By default} Xcode 4.6.3 
\RU{генерирует код для режима Thumb-2 примерно в такой манере}%
\EN{generates code for Thumb-2 in this manner}:

\begin{lstlisting}[caption=\OptimizingXcodeIV (\ThumbTwoMode)]
__text:00002B6C                   _hello_world
__text:00002B6C 80 B5          PUSH            {R7,LR}
__text:00002B6E 41 F2 D8 30    MOVW            R0, #0x13D8
__text:00002B72 6F 46          MOV             R7, SP
__text:00002B74 C0 F2 00 00    MOVT.W          R0, #0
__text:00002B78 78 44          ADD             R0, PC
__text:00002B7A 01 F0 38 EA    BLX             _puts
__text:00002B7E 00 20          MOVS            R0, #0
__text:00002B80 80 BD          POP             {R7,PC}

...

__cstring:00003E70 48 65 6C 6C 6F 20+aHelloWorld  DCB "Hello world!",0xA,0
\end{lstlisting}

\index{\ThumbTwoMode}
\index{ARM!\Instructions!BL}
\index{ARM!\Instructions!BLX}
\RU{Инструкции \TT{BL} и \TT{BLX} в Thumb, как мы помним, кодируются как пара 16-битных инструкций, 
а в Thumb-2 эти \IT{суррогатные} опкоды расширены так, что новые инструкции кодируются здесь как 
32-битные инструкции}%
\EN{The \TT{BL} and \TT{BLX} instructions in Thumb mode, as we recall, are encoded as a pair
of 16-bit instructions.
In Thumb-2 these \IT{surrogate} opcodes are extended in such a way so that new instructions
may be encoded here as 32-bit instructions}.
\RU{Это можно заметить по тому что опкоды Thumb-2 инструкций всегда начинаются с \TT{0xFx} либо с \TT{0xEx}}%
\EN{That is obvious considering that the opcodes of the Thumb-2 instructions always begin with \TT{0xFx} or \TT{0xEx}}.
\RU{Но в листинге \IDA байты опкода переставлены местами.
Это из-за того, что в процессоре ARM инструкции кодируются так:
в начале последний байт, потом первый (для Thumb и Thumb-2 режима), либо, 
(для инструкций в режиме ARM) в начале четвертый байт, затем третий, второй и первый 
(т.е. другой \gls{endianness})}%
\EN{But in the \IDA listing
the opcode bytes are swapped because for ARM processor the instructions are encoded as follows: 
last byte comes first and after that comes the first one (for Thumb and Thumb-2 modes) 
or for instructions in ARM mode the fourth byte comes first, then the third,
then the second and finally the first (due to different \gls{endianness})}.

\RU{Вот так байты следуют в листингах IDA:}
\EN{So that is how bytes are located in IDA listings:}
\begin{itemize}
\item \RU{для режимов ARM и ARM64}\EN{for ARM and ARM64 modes}: 4-3-2-1;
\item \RU{для режима Thumb}\EN{for Thumb mode}: 2-1;
\item \RU{для пары 16-битных инструкций в режиме Thumb-2}\EN{for 16-bit instructions pair in Thumb-2 mode}: 2-1-4-3.
\end{itemize}

\index{ARM!\Instructions!MOVW}
\index{ARM!\Instructions!MOVT.W}
\index{ARM!\Instructions!BLX}
\RU{Так что мы видим здесь что инструкции \TT{MOVW}, \TT{MOVT.W} и \TT{BLX} начинаются с}
\EN{So as we can see, the \TT{MOVW}, \TT{MOVT.W} and \TT{BLX} instructions begin with} \TT{0xFx}.

\RU{Одна из Thumb-2 инструкций это}\EN{One of the Thumb-2 instructions is}
\TT{MOVW R0, \#0x13D8}\RU{~--- она записывает 16-битное число в младшую часть регистра \Reg{0}, очищая старшие биты.}
\EN{~---it stores a 16-bit value into the lower part of the \Reg{0} register, clearing the higher bits.}

\RU{Ещё}\EN{Also,} \TT{MOVT.W R0, \#0}\RU{~--- эта инструкция работает так же, как и}
\EN{~works just like} 
\TT{MOVT} \RU{из предыдущего примера, но она работает в}\EN{from the previous example only it works in} Thumb-2.

\index{ARM!\RU{переключение режимов}\EN{mode switching}}
\index{ARM!\Instructions!BLX}
\RU{Помимо прочих отличий, здесь используется инструкция}
\EN{Among the other differences, the} \TT{BLX} 
\RU{вместо}\EN{instruction is used in this case instead of the} \TT{BL}.
\RU{Отличие в том, что помимо сохранения адреса возврата в регистре \ac{LR} и передаче управления 
в функцию \puts, происходит смена режима процессора с Thumb/Thumb-2 на режим ARM (либо назад).}
\EN{The difference is that, besides saving the \ac{RA} in the \ac{LR} register and passing control 
to the \puts function, the processor is also switching from Thumb/Thumb-2 mode to ARM mode (or back).}
\RU{Здесь это нужно потому, что инструкция, куда ведет переход, выглядит так (она закодирована в режиме ARM)}%
\EN{This instruction is placed here since the instruction to which control is passed looks like (it is encoded in ARM mode)}:

\begin{lstlisting}
__symbolstub1:00003FEC _puts           ; CODE XREF: _hello_world+E
__symbolstub1:00003FEC 44 F0 9F E5     LDR  PC, =__imp__puts
\end{lstlisting}

\EN{This is essentially jump to place where \puts address is written in imports' section.}
\RU{Это просто переход на место, где записан адрес \puts в секции импортов.}

\RU{Итак, внимательный читатель может задать справедливый вопрос: почему бы не вызывать \puts сразу в 
том же месте кода, где он нужен?}
\EN{So, the observant reader may ask: why not call \puts right at the point in the code where it is needed?}

\RU{Но это не очень выгодно из-за экономии места и вот почему}%
\EN{Because it is not very space-efficient}.

\index{\RU{Динамически подгружаемые библиотеки}\EN{Dynamically loaded libraries}}
\RU{Практически любая программа использует внешние динамические библиотеки (будь то DLL в Windows, .so в *NIX 
либо .dylib в \MacOSX)}\EN{Almost any program uses external dynamic libraries (like DLL in Windows, .so in *NIX or .dylib in \MacOSX)}.
\RU{В динамических библиотеках находятся часто используемые библиотечные функции, в том числе стандартная функция Си \puts}%
\EN{The dynamic libraries contain frequently used library functions, including the standard C-function \puts}.

\index{Relocation}
\RU{В исполняемом бинарном файле}\EN{In an executable binary file} 
(Windows PE .exe, ELF \RU{либо}\EN{or} Mach-O) \RU{имеется секция импортов, список символов (функций либо глобальных переменных) импортируемых из внешних модулей, а также названия самих модулей}%
\EN{an import section is present.
This is a list of symbols (functions or global variables) imported from external modules along with the names of the modules themselves}.

\RU{Загрузчик \ac{OS} загружает необходимые модули и, перебирая импортируемые символы в основном модуле, проставляет правильные адреса каждого символа}%
\EN{The \ac{OS} loader loads all modules it needs and, while enumerating import symbols in the primary module, determines the correct addresses of each symbol}.

\RU{В нашем случае,}\EN{In our case,} \IT{\_\_imp\_\_puts} 
\RU{это 32-битная переменная, куда загрузчик \ac{OS} запишет правильный адрес этой же функции во внешней библиотеке}%
\EN{is a 32-bit variable used by the \ac{OS} loader to store the correct address of the function in an external library}. 
\RU{Так что инструкция \TT{LDR} просто берет 32-битное значение из этой переменной, и, записывая его в регистр \ac{PC}, просто передает туда управление}%
\EN{Then the \TT{LDR} instruction just reads the 32-bit value from this variable and writes it into the \ac{PC} register, passing control to it}.

\RU{Чтобы уменьшить время работы загрузчика \ac{OS}, 
нужно чтобы ему пришлось записать адрес каждого символа только один раз, 
в соответствующее, выделенное для них, место.}
\EN{So, in order to reduce the time the \ac{OS} loader needs for completing this procedure, 
it is good idea if it writes the address of each symbol only once, to a dedicated place.}

\index{thunk-\RU{функции}\EN{functions}}
\RU{К тому же, как мы уже убедились, нельзя одной инструкцией загрузить в регистр 32-битное число без обращений к памяти}%
\EN{Besides, as we have already figured out, it is impossible to load a 32-bit value into a register 
while using only one instruction without a memory access}.
\RU{Так что наиболее оптимально выделить отдельную функцию, работающую в режиме ARM, 
чья единственная цель~--- передавать управление дальше, в динамическую библиотеку.}
\EN{Therefore, the optimal solution is to allocate a separate function working in ARM mode with sole 
goal to pass control to the dynamic library}
\RU{И затем ссылаться на эту короткую функцию из одной инструкции (так называемую \glslink{thunk function}{thunk-функцию}) из Thumb-кода}%
\EN{and then to jump to this short one-instruction function (the so-called \gls{thunk function}) from the Thumb-code}.

\index{ARM!\Instructions!BL}
\RU{Кстати, в предыдущем примере (скомпилированном для режима ARM), переход при помощи инструкции \TT{BL} ведет 
на такую же \glslink{thunk function}{thunk-функцию}, однако режим процессора не переключается (отсюда отсутствие \q{X} в мнемонике инструкции)}%
\EN{By the way, in the previous example (compiled for ARM mode) the control is passed by the \TT{BL} to the 
same \gls{thunk function}.
The processor mode, however, is not being switched (hence the absence of an \q{X} in the instruction mnemonic)}.

\subsubsection{\EN{More about thunk-functions}\RU{Еще о thunk-функциях}}
\index{thunk-\RU{функции}\EN{functions}}

\RU{Thunk-функции трудновато понять, вероятно, из-за путаницы в терминах.}
\EN{Thunk-functions are hard to understand, apparently, because of misnomer.}

\RU{Проще всего представлять их как адаптеры-переходники из одного типа разъемов в другой.}
\EN{Most simple way to understand it as adaptors or convertors of one type of jack to another.}
\RU{Например, адаптер позволяющее вставить в американскую розетку британскую вилку, или наоборот.}
\EN{For example, an adaptor allowing to insert Britain power plug into USA wall socket, or otherwise.} 

\EN{Thunk functions are also sometimes called \IT{wrappers}.}
\RU{Thunk-функции также иногда называются \IT{wrapper-ами}. \IT{Wrap} в английском языке это \IT{обертывать}, \IT{завертывать}.}

\RU{Вот еще несколько описаний этих функций:}
\EN{Here are couple more descriptions of these functions:}

\begin{framed}
\begin{quotation}
“A piece of coding which provides an address:”, according to P. Z. Ingerman, 
who invented thunks in 1961 as a way of binding actual parameters to their formal 
definitions in Algol-60 procedure calls. If a procedure is called with an expression 
in the place of a formal parameter, the compiler generates a thunk which computes 
the expression and leaves the address of the result in some standard location.

\dots

Microsoft and IBM have both defined, in their Intel-based systems, a “16-bit environment” 
(with bletcherous segment registers and 64K address limits) and a “32-bit environment” 
(with flat addressing and semi-real memory management). The two environments can both be 
running on the same computer and OS (thanks to what is called, in the Microsoft world, 
WOW which stands for Windows On Windows). MS and IBM have both decided that the process 
of getting from 16- to 32-bit and vice versa is called a “thunk”; for Windows 95, 
there is even a tool THUNK.EXE called a “thunk compiler”.
\end{quotation}
\end{framed}
% TODO FIXME move to bibliography and quote properly above the quote
( \href{http://go.yurichev.com/17362}{The Jargon File} )
