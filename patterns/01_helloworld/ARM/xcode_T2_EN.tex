\subsection{\OptimizingXcodeIV (\ThumbTwoMode)}

By default Xcode 4.6.3 generates code for Thumb-2 in this manner:

\begin{lstlisting}[caption=\OptimizingXcodeIV (\ThumbTwoMode)]
__text:00002B6C                   _hello_world
__text:00002B6C 80 B5          PUSH            {R7,LR}
__text:00002B6E 41 F2 D8 30    MOVW            R0, #0x13D8
__text:00002B72 6F 46          MOV             R7, SP
__text:00002B74 C0 F2 00 00    MOVT.W          R0, #0
__text:00002B78 78 44          ADD             R0, PC
__text:00002B7A 01 F0 38 EA    BLX             _puts
__text:00002B7E 00 20          MOVS            R0, #0
__text:00002B80 80 BD          POP             {R7,PC}

...

__cstring:00003E70 48 65 6C 6C 6F 20+aHelloWorld  DCB "Hello world!",0xA,0
\end{lstlisting}

% Q: If you subtract 0x13D8 from 0x3E70,
% you actually get a location that is not in this function, or in _puts.
% How is PC-relative addressing done in THUMB2?
% A: it's not Thumb-related. there are just mess with two different segments. TODO: rework this listing.

\myindex{\ThumbTwoMode}
\myindex{ARM!\Instructions!BL}
\myindex{ARM!\Instructions!BLX}

The \TT{BL} and \TT{BLX} instructions in Thumb mode, as we recall, are encoded as a pair of 16-bit instructions.
In Thumb-2 these \IT{surrogate} opcodes are extended in such a way so that new instructions may be encoded here as 32-bit instructions.

That is obvious considering that the opcodes of the Thumb-2 instructions always begin with \TT{0xFx} or \TT{0xEx}.

But in the \IDA listing
the opcode bytes are swapped because for ARM processor the instructions are encoded as follows: 
last byte comes first and after that comes the first one (for Thumb and Thumb-2 modes) 
or for instructions in ARM mode the fourth byte comes first, then the third,
then the second and finally the first (due to different \gls{endianness}).

So that is how bytes are located in IDA listings:
\begin{itemize}
\item for ARM and ARM64 modes: 4-3-2-1;
\item for Thumb mode: 2-1;
\item for 16-bit instructions pair in Thumb-2 mode: 2-1-4-3.
\end{itemize}

\myindex{ARM!\Instructions!MOVW}
\myindex{ARM!\Instructions!MOVT.W}
\myindex{ARM!\Instructions!BLX}

So as we can see, the \TT{MOVW}, \TT{MOVT.W} and \TT{BLX} instructions begin with \TT{0xFx}.

One of the Thumb-2 instructions is \TT{MOVW R0, \#0x13D8} ~---it stores a 16-bit value into the lower part of the \Reg{0} register, clearing the higher bits.

Also, \TT{MOVT.W R0, \#0} ~works just like \TT{MOVT} from the previous example only it works in Thumb-2.

\myindex{ARM!mode switching}
\myindex{ARM!\Instructions!BLX}

Among the other differences, the \TT{BLX} instruction is used in this case instead of the \TT{BL}.

The difference is that, besides saving the \ac{RA} in the \ac{LR} register and passing control 
to the \puts function, the processor is also switching from Thumb/Thumb-2 mode to ARM mode (or back).

This instruction is placed here since the instruction to which control is passed looks like (it is encoded in ARM mode):

\begin{lstlisting}
__symbolstub1:00003FEC _puts           ; CODE XREF: _hello_world+E
__symbolstub1:00003FEC 44 F0 9F E5     LDR  PC, =__imp__puts
\end{lstlisting}

This is essentially a jump to the place where the address of \puts is written in the imports' section.

So, the observant reader may ask: why not call \puts right at the point in the code where it is needed?

Because it is not very space-efficient.

\myindex{Dynamically loaded libraries}
Almost any program uses external dynamic libraries (like DLL in Windows, .so in *NIX or .dylib in \MacOSX).
The dynamic libraries contain frequently used library functions, including the standard C-function \puts.

\myindex{Relocation}
In an executable binary file (Windows PE .exe, ELF or Mach-O) an import section is present.
This is a list of symbols (functions or global variables) imported from external modules along with the names of the modules themselves.

The \ac{OS} loader loads all modules it needs and, while enumerating import symbols in the primary module, determines the correct addresses of each symbol.

In our case, \IT{\_\_imp\_\_puts} is a 32-bit variable used by the \ac{OS} loader to store the correct address of the function in an external library. 
Then the \TT{LDR} instruction just reads the 32-bit value from this variable and writes it into the \ac{PC} register, passing control to it.

So, in order to reduce the time the \ac{OS} loader needs for completing this procedure, 
it is good idea to write the address of each symbol only once, to a dedicated place.

\myindex{thunk-functions}
Besides, as we have already figured out, it is impossible to load a 32-bit value into a register 
while using only one instruction without a memory access.

Therefore, the optimal solution is to allocate a separate function working in ARM mode with the sole 
goal of passing control to the dynamic library and then to jump to this short one-instruction function (the so-called \gls{thunk function}) from the Thumb-code.

\myindex{ARM!\Instructions!BL}
By the way, in the previous example (compiled for ARM mode) the control is passed by the \TT{BL} to the 
same \gls{thunk function}.
The processor mode, however, is not being switched (hence the absence of an \q{X} in the instruction mnemonic).

\subsubsection{More about thunk-functions}
\myindex{thunk-functions}

Thunk-functions are hard to understand, apparently, because of a misnomer.
The simplest way to understand it as adaptors or convertors of one type of jack to another.
For example, an adaptor allowing the insertion of a British power plug into an American wall socket, or vice-versa. 
Thunk functions are also sometimes called \IT{wrappers}.

Here are a couple more descriptions of these functions:

\begin{framed}
\begin{quotation}
“A piece of coding which provides an address:”, according to P. Z. Ingerman, 
who invented thunks in 1961 as a way of binding actual parameters to their formal 
definitions in Algol-60 procedure calls. If a procedure is called with an expression 
in the place of a formal parameter, the compiler generates a thunk which computes 
the expression and leaves the address of the result in some standard location.

\dots

Microsoft and IBM have both defined, in their Intel-based systems, a “16-bit environment” 
(with bletcherous segment registers and 64K address limits) and a “32-bit environment” 
(with flat addressing and semi-real memory management). The two environments can both be 
running on the same computer and OS (thanks to what is called, in the Microsoft world, 
WOW which stands for Windows On Windows). MS and IBM have both decided that the process 
of getting from 16- to 32-bit and vice versa is called a “thunk”; for Windows 95, 
there is even a tool, THUNK.EXE, called a “thunk compiler”.
\end{quotation}
\end{framed}
% TODO FIXME move to bibliography and quote properly above the quote
( \href{http://go.yurichev.com/17362}{The Jargon File} )

