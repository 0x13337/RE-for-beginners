\section{GCC\EMDASH{}\EN{one more thing}\ITA{Ancora un'altra cosa}\RU{ещё кое-что}\NL{Nog een ding}}
\label{use_parts_of_C_strings}

\NL{Het feit dat een \IT{anonieme} C-string het type \IT{const} heeft (\myref{string_is_const_char}), 
en dat C-string gealloceerd in het segment met de constanten gegarandeerd immuteerbaar zijn, heeft een interessant gevolg:
De compiler kan een specifiek gedeelte van de string gebruiken.}%
\RU{Тот факт, что \IT{анонимная} Си-строка имеет тип \IT{const} (\myref{string_is_const_char}), 
и тот факт, что выделенные в сегменте констант Си-строки гаратировано неизменяемые (immutable), 
ведет к интересному следствию: компилятор может использовать определенную часть строки.}%
\EN{The fact that an \IT{anonymous} C-string has \IT{const} type (\myref{string_is_const_char}), 
and that C-strings allocated in constants segment are guaranteed to be immutable, has an interesting consequence:
the compiler may use a specific part of the string.}
\ITA{Il fatto che una stringa C \IT{anonima} ha \IT{const} tipo (\myref{string_is_const_char}), 
e che le stringhe C allocate nel constants segment siano garantite essere immutabili, hanno una conseguenza interessante:
il compilatore potrebbe utilizzare una parte specifica della stringa.}


\RU{Вот простой пример}\EN{Let's try this example}\ITA{Un esempio}\NL{Laten we dit voorbeeld proberen}:

\begin{lstlisting}
#include <stdio.h>

int f1()
{
	printf ("world\n");
}

int f2()
{
	printf ("hello world\n");
}

int main()
{
	f1();
	f2();
}
\end{lstlisting}

\RU{Среднестатистический компилятор с \CCpp (включая MSVC) выделит место для двух строк, но вот что делает GCC 4.8.1}%
\EN{Common \CCpp{}-compilers (including MSVC) allocate two strings, but let's see what GCC 4.8.1 does}%
\ITA{I comuni compilatori \CCpp{}- (incluso MSVC) allocano due stringhe, ma vediamo cosa fa GCC 4.8.1}%
\NL{Veelgebruikte \CCpp{}-compilers (inclusief MSVC) alloceren twee strings, maar laat ons eens kijken wat GCC 4.8.1 doet}:

\begin{lstlisting}[caption=GCC 4.8.1 + \RU{листинг в }IDA\EN{ listing}]
f1              proc near

s               = dword ptr -1Ch

                sub     esp, 1Ch
                mov     [esp+1Ch+s], offset s ; "world\n"
                call    _puts
                add     esp, 1Ch
                retn
f1              endp

f2              proc near

s               = dword ptr -1Ch

                sub     esp, 1Ch
                mov     [esp+1Ch+s], offset aHello ; "hello "
                call    _puts
                add     esp, 1Ch
                retn
f2              endp

aHello          db 'hello '
s               db 'world',0xa,0
\end{lstlisting}

\NL{Inderdaad: wanneer we de \q{hello world} string printen, 
Worden deze twee woorden aangrenzend in het geheugen geplaatsd, en \puts, die geroepen wordt van de f2()
functie, is er zicht niet van bewust dat deze string opgedeeld is. 
In feite is hij ook niet opgedeeld, hij is slechts \q{virtueel} opgedeeld in deze listing.}%
\EN{Indeed: when we print the \q{hello world} string
these two words are positioned in memory adjacently and \puts called from f2() 
function is not aware that this string is divided. 
In fact, it's not divided; it's divided only \q{virtually}, in this listing.}%
\ITA{Quando stampiamo la stringa \q{hello world} le due parole sono posizionate adiacenti in memoria e \puts chiamata dalla funzione f2() non è al corrente che la stringa è in divisa. Infatti non lo è; è divisa soltanto \q{virtualmente}, in questo listato.}%
\RU{Действительно, когда мы выводим строку \q{hello world}, 
эти два слова расположены в памяти впритык друг к другу и \puts, вызываясь из функции f2(), вообще не знает,
что эти строки разделены. Они и не разделены на самом деле, они разделены
только \q{виртуально}, в нашем листинге.}

\RU{Когда \puts вызывается из f1(), он использует строку \q{world} плюс нулевой байт. \puts не знает, что там ещё есть какая-то строка перед этой!}%
\EN{When \puts is called from f1(), it uses the \q{world} string plus a zero byte. \puts is not aware that there is something before this string!}%
\ITA{Quando \puts è chiamata dalla funzione f1(), usa la stringa \q{world} più un byte zero. \puts non sa che c'è qualcos'altro prima di questa stringa!}%
\NL{Wanneer \puts aangeroepen wordt vanuit f1(), gebruikt het de \q{world} string plus een nulbyte. \puts is er zich niet van bewust dat er nog iets voor deze string staat!}

\RU{Этот трюк часто используется (по крайней мере в GCC) и может сэкономить немного памяти.}%
\EN{This clever trick is often used by at least GCC and can save some memory.}%
\ITA{Questo trucco intelligente è usato spesso, almeno da GCC, e può far risparmiare un po' di memoria.}%
\NL{Dit slimme truukje wordt vaak gebruikt door op zijn minst GCC, en kan geheugen besparen.}

