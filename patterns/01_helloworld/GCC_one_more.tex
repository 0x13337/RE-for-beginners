\section{GCC\EMDASH{}\EN{one more thing}\RU{еще кое-что}}
\label{use_parts_of_C_strings}

\RU{Тот факт, что \IT{анонимная} Си-строка имеет тип}\EN{The fact that an \IT{anonymous} C-string has} 
\IT{const}\EN{ type} (\myref{string_is_const_char}), 
\RU{и тот факт, что выделенные в сегменте констант Си-строки гаратировано неизменяемые (immutable), 
ведет к интересному следствию}\EN{and the 
fact that C-strings allocated in constants segment are guaranteed to be immutable, has an interesting consequence}:
\RU{компилятор может использовать определенную часть строки}\EN{the compiler may use a specific part of string}.

\RU{Вот простой пример}\EN{Let's try this example}:

\begin{lstlisting}
#include <stdio.h>

int f1()
{
	printf ("world\n");
};

int f2()
{
	printf ("hello world\n");
};

int main()
{
	f1();
	f2();
};
\end{lstlisting}

\RU{Среднестатистический компилятор с \CCpp (включая MSVC) выделит место для двух строк, но вот что делает 
GCC 4.8.1}\EN{Common \CCpp{}-compilers (including MSVC) will allocate two strings, but let's see what 
GCC 4.8.1 does}:

\begin{lstlisting}[caption=GCC 4.8.1 + \RU{листинг в }IDA\EN{ listing}]
f1              proc near

s               = dword ptr -1Ch

                sub     esp, 1Ch
                mov     [esp+1Ch+s], offset s ; "world\n"
                call    _puts
                add     esp, 1Ch
                retn
f1              endp

f2              proc near

s               = dword ptr -1Ch

                sub     esp, 1Ch
                mov     [esp+1Ch+s], offset aHello ; "hello "
                call    _puts
                add     esp, 1Ch
                retn
f2              endp

aHello          db 'hello '
s               db 'world',0xa,0
\end{lstlisting}

\RU{Действительно: когда мы выводим строку}\EN{Indeed: when we print the ``hello world'' string}, 
\RU{эти два слова расположены в памяти в притык друг к другу и \puts, вызываясь из ф-ции f2(), вообще не знает
что эти строки разделены}\EN{these two words are positioned in memory adjacently and \puts called from f2() 
function is not aware this string is divided}. \RU{Они и не разделены на самом деле, они разделены
только ``виртуально'', в нашем листинге}\EN{It's not divided in fact, it's divided only ``virtually'', in this
listing}.

\RU{Когда}\EN{When} \puts \RU{вызывается из f1(), он использует строку}\EN{is called from f1(), it uses} 
``world'' \RU{плюс нулевой байт}\EN{string plus zero byte}. \puts \RU{не знает что там еще есть какая-то строка
перед этой}\EN{is not aware there is something before this string}!

\RU{Этот трюк часто используется по крайней мере в GCC и может сэкономить немного памяти.}
\EN{This clever trick is often used by at least GCC and can save some memory.}
