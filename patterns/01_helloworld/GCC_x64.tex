\subsection{GCC\EMDASH{}x86-64}

\index{x86-64}
\IFRU{Попробуем GCC в 64-битном Linux}{Let's also try GCC in 64-bit Linux}:

\lstinputlisting[caption=GCC 4.4.6 x64]{patterns/01_helloworld/GCC_x64_\LANG.s}

\IFRU{В Linux, *BSD и MacOSX для x86-64 также принят способ передачи аргументов ф-ции через регистры}
{A method to pass function arguments in registers are also used in Linux, *BSD and MacOSX}\cite{SysVABI}.
\IFRU{6 первых аргументов передаются через регистры}{6 first arguments are passed in}
\RDI, \RSI, \RDX, \RCX, \Reg{8}, \Reg{9}\IFRU{, а остальные}{ registers, and others}\EMDASH{}\IFRU{через стек}{via
stack}.

\IFRU{Так что указатель на строку передается через \EDI (32-битную часть регистра)}{So the pointer to the
string is passed in \EDI (32-bit part of register)}.
\IFRU{Но почему не через 64-битную часть}{But why not to use 64-bit part}, \RDI?

\IFRU{Важно запомнить что в 64-битном режиме, все инструкции \MOV записывающие что-либо в 
младшую 32-битную часть регистра,
обнуляют старшие 32-бита}{It is important to keep in mind that all \MOV instructions in 64-bit mode
writing something into lower 32-bit register part, clearing higher 32-bits}\cite{Intel}.
\IFRU{То есть, инструкция}{I.e., the} \TT{MOV EAX, 011223344h} \IFRU{корректно запишет это значение в \RAX, 
старшие биты сбросятся в ноль}{will write a value correctly into \RAX, higher bits will be cleared}.

\IFRU{Если посмотреть в \IDA скомпилированный объектный файл (.o), увидим также опкоды всех инструкций}
{If to open compiled object file (.o), we will also see all instruction's opcodes}
\footnote{\IFRU{Это нужно задать в}{This should be enabled in} 
Options $\rightarrow$ Disassembly $\rightarrow$ Number of opcode bytes}:

\lstinputlisting[caption=GCC 4.4.6 x64]{patterns/01_helloworld/GCC_x64.lst}

\label{hw_EDI_instead_of_RDI}
\IFRU{Как видно, инструкция, записывающая в \EDI по адресу \TT{0x4004D4}, занимает 5 байт}
{As we can see, the instruction writing into \EDI at \TT{0x4004D4} occupies 5 bytes}.
\IFRU{Та же инструкция, записывающая 32-битное значение в \RDI, будет занимать 7 байт}
{The same instruction, writing 32-bit value into \RDI will occupy 7 bytes}.
\IFRU{Возможно, GCC решил немного сэкономить}
{Apparently, GCC tries to save some space}. 
\IFRU{К тому же, вероятно, он уверен что сегмент данных где хранится строка,
никогда не будет расположен в адресах выше 4\gls{GiB}}{Besides, it can be sure that the data segment containing
the string will not be allocated at the addresses higher than 4\gls{GiB}}.

\label{SysVABI_input_EAX}
\IFRU{Здесь мы также видим обнуление регистра \EAX перед вызовом \printf}{We also see \EAX register
clearance before \printf function call}.
\IFRU{Это делается потому, что по стандарту передачи аргументов в *NIX для x86-64 
в \EAX передается количество задействованных векторных регистров}{This is done because a number of
used vector registers is passed in \EAX by standard}:
``with variable arguments passes information about the number of vector registers used''\cite{SysVABI}.
