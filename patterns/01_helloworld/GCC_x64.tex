\subsection{GCC\EMDASH{}x86-64}

\index{x86-64}
\RU{Попробуем GCC в 64-битном Linux}\EN{Let's also try GCC in 64-bit Linux}:

\lstinputlisting[caption=GCC 4.4.6 x64]{patterns/01_helloworld/GCC_x64.s.\LANG}

\RU{В Linux, *BSD и \MacOSX для x86-64 также принят способ передачи аргументов ф-ции через регистры}
\EN{A method to pass function arguments in registers is also used in Linux, *BSD and \MacOSX}\cite{SysVABI}.
\RU{6 первых аргументов передаются через регистры}\EN{The first 6 arguments are passed in the}
\RDI, \RSI, \RDX, \RCX, \Reg{8}, \Reg{9}\RU{, а остальные}\EN{ registers, and the rest}\EMDASH{}\RU{через стек}\EN{via
the stack}.

\RU{Так что указатель на строку передается через \EDI (32-битную часть регистра)}\EN{So the pointer to the
string is passed in \EDI (32-bit part of register)}.
\RU{Но почему не через 64-битную часть}\EN{But why not use the 64-bit part}, \RDI?

\RU{Важно запомнить что в 64-битном режиме, все инструкции \MOV записывающие что-либо в 
младшую 32-битную часть регистра,
обнуляют старшие 32-бита}\EN{It is important to keep in mind that all \MOV instructions in 64-bit mode
that write something into the lower 32-bit register part, also clear the higher 32-bits}\cite{Intel}.
\RU{То есть, инструкция}\EN{I.e., the} \TT{MOV EAX, 011223344h} \RU{корректно запишет это значение в \RAX, 
старшие биты сбросятся в ноль}\EN{writes a value into \RAX correctly, since the higher bits will be cleared}.

\RU{Если посмотреть в \IDA скомпилированный объектный файл (.o), увидим также опкоды всех инструкций}
\EN{If we open the compiled object file (.o), we can also see all instruction's opcodes}
\footnote{\RU{Это нужно задать в}\EN{This must be enabled in} 
Options $\rightarrow$ Disassembly $\rightarrow$ Number of opcode bytes}:

\lstinputlisting[caption=GCC 4.4.6 x64]{patterns/01_helloworld/GCC_x64.lst}

\label{hw_EDI_instead_of_RDI}
\RU{Как видно, инструкция, записывающая в \EDI по адресу \TT{0x4004D4}, занимает 5 байт}
\EN{As we can see, the instruction that writes into \EDI at \TT{0x4004D4} occupies 5 bytes}.
\RU{Та же инструкция, записывающая 64-битное значение в \RDI, занимает 7 байт.}
\EN{The same instruction writing a 64-bit value into \RDI occupies 7 bytes.}
\RU{Возможно, GCC решил немного сэкономить}
\EN{Apparently, GCC is trying to save some space}. 
\RU{К тому же, вероятно, он уверен, что сегмент данных где хранится строка,
никогда не будет расположен в адресах выше 4\gls{GiB}.}
\EN{Besides, it can be sure that the data segment containing
the string will not be allocated at the addresses higher than 4\gls{GiB}.}

\label{SysVABI_input_EAX}
\RU{Здесь мы также видим обнуление регистра \EAX перед вызовом \printf}\EN{We also see that the \EAX register
was cleared before the \printf function call}.
\RU{Это делается потому, что по стандарту передачи аргументов в *NIX для x86-64 
в \EAX передается количество задействованных векторных регистров}\EN{This is done because the number of
used vector registers is passed in \EAX by standard}:
``with variable arguments passes information about the number of vector registers used''\cite{SysVABI}.
