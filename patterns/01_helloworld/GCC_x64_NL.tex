\subsection{GCC\EMDASH{}x86-64}

\myindex{x86-64}
Laat ons ook eens kijken naar GCC in 64-bit Linux:

\lstinputlisting[caption=GCC 4.4.6 x64]{patterns/01_helloworld/GCC_x64.s.\LANG}

Een methode om functieargumenten door te geven in registers wordt ook gebruikt in Linux, *BSD en \MacOSX \SysVABI.

De eerste 6 argumenten worden doorgegeven in de \RDI, \RSI, \RDX, \RCX, \Reg{8}, \Reg{9} registers, en de rest --- via de stack.

De pointer naar de string wordt dus doorgegeven via \EDI (het 32-bit gedeelte van het register).
Maar waarom gebruikt men niet het 64-bit gedeelte, \RDI?

Het is belangrijk te onthouden dat alle \MOV instructies in 64-bit modus, die iets schrijven in het onderste 32-bit gedeelte van het register, ook het bovenste 32-bit gedeelte leegmaken.
\INS{MOV EAX, 011223344h} schrijft een waarde correct weg in \RAX, aangezien de bovenste bits zullen worden leeggemaakt.

Als we het gecompileerde object-bestand (.o) openen, kunnen we ook de opcodes zien van alle instructies
\footnote{Dit moet ook geactiveerd worden in \textbf{Options $\rightarrow$ Disassembly $\rightarrow$ Number of opcode bytes}}:

\lstinputlisting[caption=GCC 4.4.6 x64]{patterns/01_helloworld/GCC_x64.lst}

\label{hw_EDI_instead_of_RDI}
Zoals je kan zien, bezet de instructie die in \EDI schrijft op \TT{0x4004D4} 5 bytes.
Dezelfde instructie die een 64-bit waarde in \RDI schrijft, bezet 7 bytes.
Blijkbaar probeert GCC wat plaats te besparen.
Daarnaast kunnen we met zekerheid zeggen dat het data segment dat de string bevat, niet zal gealloceerd worden op de adressen hoger dan 4\gls{GiB}.

\label{SysVABI_input_EAX}
We zien ook dat het \EAX register leeggemaakt is voor de \printf functie call.
Dit wordt gedaan omdat het aantal gebruikte vector registers wordt doorgegeven in \EAX in *NIX systemen op x86-64.

