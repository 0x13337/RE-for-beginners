\subsubsection{GCC\EMDASH{}x86}

\IFRU{Теперь скомпилируем то же самое компилятором GCC 4.4.1 в Linux}
{Now let's try to compile the same \CCpp code in the GCC 4.4.1 compiler in Linux}: \TT{gcc 1.c -o 1}

\IFRU{Затем, при помощи \IDA{}, посмотрим, как создалась функция \main.}
{After, with the assistance of the \IDA disassembler, let's see how the \main function was created.} 

(\IDA, \IFRU{как и MSVC, показывает код в Intel-синтаксисе}{like MSVC, shows code in Intel-syntax}).

N.B. \IFRU{Мы также можем заставить GCC генерировать листинги в этом формате при помощи ключей}
{We could also have GCC produce assembly listings in Intel-syntax by applying the options} 
\TT{-S -masm=intel}

\begin{lstlisting}[caption=GCC]
main            proc near

var_10          = dword ptr -10h

                push    ebp
                mov     ebp, esp
                and     esp, 0FFFFFFF0h
                sub     esp, 10h
                mov     eax, offset aHelloWorld ; "hello, world"
                mov     [esp+10h+var_10], eax
                call    _printf
                mov     eax, 0
                leave
                retn
main            endp
\end{lstlisting}

\index{Function prologue}
\index{x86!\Instructions!AND}
\IFRU{Почти то же самое. 
Адрес строки ``hello, world'', лежащей в сегменте данных, вначале сохраняется в \EAX, затем записывается в стек.
А еще в прологе функции мы видим \TT{AND ESP, 0FFFFFFF0h} ~--- 
эта инструкция выравнивает значение в \ESP по 16-байтной границе, делая все значения 
в стеке также выровненными по этой границе (процессор более эффективно работает с переменными, расположенными
в памяти по адресам кратным 4 или 16)\footnote{\URLWPDA}.}
{The result is almost the same.
The address of the ``hello, world'' string (stored in the data segment) is saved in the \EAX register first and then it is stored on the stack.
Also in the function prologue we see \TT{AND ESP, 0FFFFFFF0h}~---this 
instruction aligns the value in the \ESP register on a 16-byte boundary.
This results in all values in the stack being aligned.
(The CPU performs better if the values it is dealing with are located in memory at addresses aligned 
on a 4- or 16-byte boundary)\footnote{\URLWPDA}.}

\index{x86!\Instructions!SUB}
\TT{SUB ESP, 10h} \IFRU{выделяет в стеке 16 байт. Хотя, как будет видно далее, здесь достаточно только 4.}
{allocates 16 bytes on the stack. Although, as we can see hereafter, only 4 are necessary here.} 

\IFRU{Это происходит потому, что количество выделяемого места в локальном стеке тоже выровнено по 
16-байтной границе.}
{This is because the size of the allocated stack is also aligned on a 16-byte boundary.}

% TODO: rewrite.
\index{x86!\Instructions!PUSH}
\IFRU{Адрес строки (или указатель на строку) затем записывается прямо в стек без помощи инструкции \PUSH.
\IT{var\_10} по совместительству ~--- и локальная переменная и одновременно аргумент для \printf{}. Подробнее об этом будет ниже.}
{The string address (or a pointer to the string) is then written directly onto the stack space 
without using the \PUSH instruction.
\IT{var\_10}~---is a local variable and is also an argument for \printf{}.
Read about it below.}

\IFRU{Затем вызывается \printf.}{Then the \printf function is called.}

\IFRU{В отличие от MSVC, GCC в компиляции без включенной оптимизации генерирует \TT{MOV EAX, 0} вместо 
более короткого опкода.}{Unlike MSVC, when GCC is compiling without optimization turned on,
it emits \TT{MOV EAX, 0} instead of a shorter opcode.}

\index{x86!\Instructions!LEAVE}
\IFRU{Последняя инструкция \LEAVE ~--- это аналог команд \TT{MOV ESP, EBP} и \TT{POP EBP} ~--- 
то есть возврат \glslink{stack pointer}{указателя стека} и регистра \EBP в первоначальное состояние.} 
{The last instruction, \LEAVE~---is the equivalent of the \TT{MOV ESP, EBP} and \TT{POP EBP} instruction 
pair~---in other words, this instruction sets the \gls{stack pointer} (\ESP) back and restores 
the \EBP register to its initial state.}

\IFRU{Это необходимо, т.к., в начале функции мы модифицировали регистры \ESP и \EBP (при помощи}
{This is necessary since we modified these register values (\ESP and \EBP) at the 
beginning of the function (executing}
\TT{\MOV EBP, ESP} / \TT{AND ESP, ...}).

\subsubsection{GCC: \ATTSyntax}

\IFRU{Попробуем посмотреть, как выглядит то же самое в AT\&T-синтаксисе языка ассемблера.}
{Let's see how this can be represented in the AT\&T syntax of assembly language.}
\IFRU{Этот синтаксис больше распространен в UNIX-мире.}
{This syntax is much more popular in the UNIX-world.}

\begin{lstlisting}[caption=\IFRU{компилируем в}{let's compile in} GCC 4.7.3]
gcc -S 1_1.c
\end{lstlisting}

\IFRU{Получим такой файл:}{We get this:}

\lstinputlisting[caption=GCC 4.7.3]{patterns/01_helloworld/GCC.s}

\IFRU{Здесь много макросов (начинающихся с точки). Они нам пока не интересны.}
{There are a lot of macros (beginning with dot). These are not very interesting to us so far.}
\IFRU{Пока что, ради упрощения, мы можем
их игнорировать и впредь (кроме макроса \IT{.string}, при помощи которого кодируется последовательность символов, 
оканчивающихся нулем, такие же строки как в Си). И тогда получится следующее}
{For now, for the sake of simplification, we can ignore them (except the \IT{.string} macro which
encodes a null-terminated character sequence just like a C-string). Then we'll see this}
\footnote{\IFRU{Кстати, для уменьшения генерации ``лишних'' макросов, можно использовать такой ключ GCC}
{This GCC option can be used to eliminate ``unnecessary'' macros}: 
\IT{-fno-asynchronous-unwind-tables}}:

\lstinputlisting[caption=GCC 4.7.3]{patterns/01_helloworld/GCC_refined.s}

\index{\ATTSyntax}
\index{\IntelSyntax}
\IFRU{Основные отличия синтаксиса Intel и AT\&T следующие:}
{Some of the major differences between Intel and AT\&T syntax are:}

\begin{itemize}

\item
\IFRU{Операнды записываются наоборот.}{Operands are written backwards.}

\IFRU{В Intel-синтаксисе: <инструкция> <операнд назначения> <операнд-источник>.}
{In Intel-syntax: <instruction> <destination operand> <source operand>.}

\IFRU{В AT\&T-синтаксисе: <инструкция> <операнд-источник> <операнд назначения>.}
{In AT\&T syntax: <instruction> <source operand> <destination operand>.}

\IFRU{Чтобы легче понимать разницу, можно запомнить следующее}
{Here is a way to think about them}: \IFRU{когда вы работаете с Intel-синтаксисом ~--- можете в уме ставить знак равенства ($=$) между операндами,}
{when you deal with Intel-syntax, you can put in equality sign ($=$) in your mind between operands}
\IFRU{а когда с AT\&T-синтаксисом ~--- мысленно ставьте стрелку направо}
{and when you deal with AT\&T-syntax put in a right arrow} 
($\rightarrow$)
\footnote{
\index{\CStandardLibrary!memcpy()}
\index{\CStandardLibrary!strcpy()}
\IFRU{Кстати, в некоторых стандартных функциях библиотеки Си (например, memcpy(), strcpy()) также применяется 
расстановка аргументов как в Intel-синтаксисе: вначале указатель в памяти на блок назначения, 
затем указатель на блок-источник.}{By the way, in some C standard functions (e.g., memcpy(), strcpy()) arguments
are listed in the same way as in Intel-syntax: pointer to destination memory block at the beginning and then
pointer to source memory block.}}.

\item
AT\&T: \IFRU{Перед именами регистров ставится знак процента (\%), а перед числами знак доллара (\$).}
{Before register names a percent sign must be written (\%) and before numbers a dollar sign (\$).}
\IFRU{Вместо квадратных скобок применяются круглые.}{Parentheses are used instead of brackets.}

\item
AT\&T: \IFRU{К каждой инструкции добавляется специальный символ, определяющий тип данных:}
{A special symbol is to be added to each instruction defining the type of data:}

\begin{itemize}
\item l --- long (32 \IFRU{бита}{bits})
\item w --- word (16 \IFRU{бит}{bits})
\item b --- byte (8 \IFRU{бит}{bits})
\end{itemize}

\end{itemize}

\IFRU{Возвращаясь к результату компиляции: он идентичен тому, который мы посмотрели в \IDA.}
{Let's go back to the compiled result: it is identical to what we saw in \IDA.}
\IFRU{Одна мелочь}{With one subtle difference}: \TT{0FFFFFFF0h} \IFRU{записывается как}{is written as} \TT{\$-16}.
\IFRU{Это то же самое}{It is the same}: \TT{16} \IFRU{в десятичной системе это}{in the decimal system is} \TT{0x10} 
\IFRU{в шестнадцатеричной}{in hexadecimal}. 
\TT{-0x10} \IFRU{будет как раз}{is equal to} \TT{0xFFFFFFF0} 
(\IFRU{в рамках 32-битных чисел}{for a 32-bit data type}).

\index{x86!\Instructions!MOV}
% to be proofreaded
\IFRU{Ещё: возвращаемый результат устанавливается в 0 обычной инструкцией \MOV а не \XOR}
{One more thing: the return value is to be set to 0 by using usual \MOV, not \XOR}.
\MOV \IFRU{просто загружает значение в регистр}{just loads value to a register}. 
\IFRU{Её название не очень удачное (данные не перемещаются), 
в других архитектурах подобная инструкция обычно носит название 
``load'' или что-то в этом роде}{Its name is not felicitous (data are not moved),
this instruction in other architectures has name ``load'' or something like that}.

