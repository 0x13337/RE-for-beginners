\section{MIPS}

\subsection{\RU{Кое-что насчет ``глобального указателя'' (``global pointer'')}\EN{Word about ``global pointer''}}
\label{MIPS_GP}

\index{MIPS!\GlobalPointer}
\RU{Важная концепция в MIPS это ``глобальный указатель'' (``global pointer'').}
\EN{One important MIPS concept is ``global pointer''.}
\RU{Как мы уже возможно знаем, каждая инструкция в MIPS имеет размер в 32 бита, так что невозможно
закодировать 32-битный адрес внутри одной инструкции: вместо этого нужно использовать пару инструкций
(как это сделал GCC для загрузки адреса текстовой строки в нашем примере).}
\EN{As we may already know, each MIPS instruction has size of 32 bits, so it's impossible to embed 32-bit
address into one instruction: a pair should be used for this 
(like GCC did in our example for the text string address loading).}

\RU{С другой стороны, используя только одну инструкцию, 
возможно загружать данные по адресам в пределах $register-32768...register+32767$ (потому что 16 бит
знакового смещения можно закодировать водной инструкции).}
\EN{It's possible, however, to load data from the address in range of $register-32768...register+32767$ using one
single instruction (because 16 bits of signed offset could be encoded in single instruction).}
\RU{Так мы можем выделить какой-то регистр для этих целей и еще выделить буфер в 64KiB для самых 
чаще всего используемых данных.}
\EN{So we can allocate some register for this purpose and also allocate 64KiB area of most used data.}
\RU{Выделенный регистр называется ``глобальный указатель'' (``global pointer'') и он указывает на середину
области 64KiB.}
\EN{This allocated register is called ``global pointer'' and it points to the middle of the 64KiB area.}
\RU{Эта область обычно содержит глобальные переменные и адреса импортированных ф-ций вроде \printf,
потому что разработчики GCC решили что получение адреса ф-ции должно быть как можно более быстрой операцией,
чтобы она исполнялась со скоростью исполнения одной инструкции вместо двух.}
\EN{This area usually contains global variables and addresses of imported functions like \printf, 
because GCC developers decided that getting address of some function must be as fast as single instruction
execution instead of two.}
\RU{В ELF-файле, эта 64KiB-область находится частично в секции .sbss (``small \ac{BSS}'', для неинициализированных
данных) и в секции .sdata (``small data'', для инициализированных данных).}
\EN{In an ELF file this 64KiB area is located partly in .sbss (``small \ac{BSS}'', for not initialized data) and 
.sdata (``small data'', for initialized data) sections.}

\RU{Это значит, что программист может выбирать, к чему нужен как можно более быстрый доступ, и затем расположить
это в секциях .sdata/.sbss.}
\EN{This means that the programmer may choose what data he/she wants to be accessed fast and place it into 
.sdata/.sbss.}

\RU{Некоторые олд-скульные программисты могут вспомнить модель памяти в MS-DOS \myref{8086_memory_model} 
или в менеджерах памяти вроде XMS/EMS, где вся память делилась на блоки по 64KiB.}
\EN{Some old-school programmers may recall the MS-DOS memory model \myref{8086_memory_model} 
or the MS-DOS memory managers like XMS/EMS where all memory was divided in 64KiB blocks.}

\index{PowerPC}
\RU{Эта концепция применяется не только в MIPS. По крайней мере PowerPC также использует эту технику.}
\EN{This concept is not unique to MIPS. At least PowerPC uses this technique as well.}

\subsection{\Optimizing GCC}

\lstinputlisting[caption=\Optimizing GCC 4.4.5 (\assemblyOutput),numbers=left]{patterns/01_helloworld/MIPS/hw_O3.s.\LANG}

\RU{Так что регистр \$GP в прологе ф-ции выставляется в середину этой области.}
\EN{The \$GP register is set in the function prologue to point the middle of this area.}
\RU{Регистр \ac{RA} также сохраняется в локальном стеке.}
\EN{The \ac{RA} register is also saved in the local stack.}
\RU{Здесь также используется \puts вместо \printf.}
\EN{\puts is also used here instead of \printf.}
\index{MIPS!\Instructions!LW}
\RU{Адрес ф-ции \puts загружается в \$25 используя инструкцию LW (``Load Word'').}
\EN{The address of the \puts function is loaded into \$25 using LW the instruction (``Load Word'').}
\index{MIPS!\Instructions!LUI}
\index{MIPS!\Instructions!ADDIU}
\RU{Затем адрес текстовой строки загружается в \$4 используя пару инструкций: LUI (``Load Upper Immediate'') и
ADDIU (``Add Immediate Unsigned Word'').}
\EN{Then the address of the text string is loaded to \$4 using LUI (``Load Upper Immediate'') and 
ADDIU (``Add Immediate Unsigned Word'') instruction pair.}
\RU{LUI устанавливает старшие 16 бит регистра (поэтому в имени инструкции присутствует ``upper'') и ADDIU
прибавляет младшие 16 бит к адресу.}
\EN{LUI sets the high 16 bits of the register (hence ``upper'' word in instruction name) and ADDIU adds
the lower 16 bits of the address.}
\RU{ADDIU следует за JALR (помните о \IT{branch delay slots}?).}
\EN{ADDIU follows JALR (remember \IT{branch delay slots}?).}
\RU{Регистр \$4 также называется \$A0, который используется для передачи первого аргумента ф-ции}
\EN{The register \$4 is also called \$A0, which is used for passing the first function argument}
\footnote{\RU{Таблица регистров в MIPS доступна в приложении}\EN{The MIPS registers table 
is available in appendix} \myref{MIPS_registers_ref}}.

\index{MIPS!\Instructions!JALR}
\RU{JALR (``Jump and Link Register'') делает переход по адресу в регистре \$25 (там адрес \puts) 
при этом сохраняя адрес следующей инструкции (LW) в \ac{RA}.}
\EN{JALR (``Jump and Link Register'') jumps to the address stored in the \$25 register (address of \puts) 
while saving the address of the next instruction (LW) in \ac{RA}.}
\RU{Это так же как и в ARM}\EN{This is very similar to ARM}.
\RU{И еще одна важная вещь это то что адрес сохраняемый в \ac{RA} это адрес не следующей инструкции (потому что
это \IT{delay slot} и исполняется перед инструкцией перехода),
но адрес инструкции после нее (после \IT{delay slot}).}
\EN{Oh, and one important thing is that the address saved in \ac{RA} is not the address of the next instruction (because
it's in a \IT{delay slot} and is executed before the jump instruction),
but the address of the instruction after the next one (after the \IT{delay slot}).}
\RU{Таким образом, во время исполнения \TT{JALR}, в \ac{RA} записывается $PC + 8$, и в нашем случае, это адрес
инструкции LW следующей после ADDIU.}
\EN{Hence, $PC + 8$ is written to \ac{RA} during the execution of \TT{JALR}, in our case, this is the address of the LW
instruction next to ADDIU.}

\RU{LW (``Load Word'') в строке 19 восстанавливает \ac{RA} из локального стека 
(эта инструкция скорее часть эпилога ф-ции).}
\EN{LW (``Load Word'') at line 19 restores \ac{RA} from the local stack 
(this instruction is rather part of function epilogue).}

\index{MIPS!\Pseudoinstructions!MOVE}
\RU{MOVE на строке 22 копирует значение из регистра \$0 (\$ZERO) в \$2 (\$V0).}
\EN{MOVE at line 22 copies the value from \$0 (\$ZERO) register to \$2 (\$V0).}
\RU{В MIPS есть \IT{константный} регистр, всегда содержащий ноль.}
\EN{MIPS has a \IT{constant} register, which always holds zero.}
\RU{Должно быть, разработчики MIPS решили что 0 это самая востребованная константа в программировании,
так что пусть будет использоваться регистр \$0, всякий раз, когда будет нужен 0.}
\EN{Apparently, the MIPS developers came with the idea that zero is in fact the busiest constant in the computer programming,
so let's just use \$0 register every time zero is needed.}
\RU{Другой интересный факт: в MIPS нет инструкции копирующей значения из регистра в регистр.}
\EN{Another interesting fact is that MIPS lacks instruction which transfers data between registers.}
\RU{На самом деле}\EN{In fact}, \TT{MOVE DST, SRC} \RU{это}\EN{is} \TT{ADD DST, SRC, \$ZERO} ($DST=SRC+0$), 
\RU{которая делает тоже самое}\EN{which does the same}.
\RU{Очевидно, разработчики MIPS хотели сделать как можно более компактную таблицу опкодов.}
\EN{Apparently, MIPS developers wanted to have compact opcode table.}
\RU{Это не значит что сложение происходит во время каждой инструкции MOVE.}
\EN{This does not mean an actual addition happens at each MOVE instruction.}
\RU{Скорее всего, эти псевдоинструкции оптимизируются в \ac{CPU} и \ac{ALU} никогда не используется.}
\EN{Most likely, the \ac{CPU} optimizes these pseudoinstructions and \ac{ALU} is never used.}

\index{MIPS!\Instructions!J}
\RU{J в строке 24 делает переход по адресу в \ac{RA}, и это работает как выход из ф-ции.}
\EN{J at line 24 jumps to the address in \ac{RA}, which is effectively performing return from the function.}
\RU{ADDIU после J на самом деле исполняется перед J (помните о \IT{branch delay slots}?) 
и это часть эпилога ф-ции.}
\EN{ADDIU after J is in fact executed before J (remember \IT{branch delay slots}?) 
and is part of function epilogue.}

\RU{Вот также листинг сгенерированный \IDA. Каждый регистр имеет свое псевдоимя:}
\EN{Here is also a listing generated by \IDA. Each register here has its own pseudoname:}

\lstinputlisting[caption=\Optimizing GCC 4.4.5 (\IDA),numbers=left]{patterns/01_helloworld/MIPS/hw_O3_IDA.lst.\LANG}

\RU{Инструкция на строке 15 сохраняет GP в локальном стеке, и эта инструкция мистическим образом отсутствует
в листинге от GCC, может быть из-за ошибки в самом GCC\footnote{Очевидно, ф-ция вывода листингов не так критична
для пользователей GCC, поэтому там вполне могут быть неисправленные ошибки.}.}
\EN{The instruction at line 15 saves GP value into the local stack, and this instruction is missing mysteriously from the GCC output listing, maybe by a GCC error\footnote{Apparently, functions generating listings 
are not so critical to GCC users, so some unfixed errors may still exist.}.}
\RU{Значение GP должно быть сохранено, потому что всякая ф-ция может работать со своим собственным окном данных
размером 64KiB.}
\EN{The GP value should be saved indeed, because each function can use its own 64KiB data window.}

\RU{Регистр, содержащий адрес ф-ции \puts называется \$T9, потому что регистры с префиксом T- называются
``temporaries'' и их содержимое можно не сохранять.}
\EN{The register containing the \puts address is called \$T9, because registers prefixed with T- are called
``temporaries'' and their contents may not be preserved.}

\subsection{\NonOptimizing GCC}

\lstinputlisting[caption=\NonOptimizing GCC 4.4.5 (\assemblyOutput),numbers=left]{patterns/01_helloworld/MIPS/hw_O0.s.\LANG}

\NonOptimizing GCC \RU{более многословный}\EN{is more verbose}.
\RU{Мы видим, что регистр FP используется как указатель на фрейм стека.}
\EN{We see here that register FP is used as a pointer to the stack frame.}
\RU{Мы также видим 3 \ac{NOP}-а.}\EN{We also see 3 \ac{NOP}s.}
\RU{Второй и третий следуют за инструкциями перехода.}
\EN{The second and third of which follow the branch instructions.}

\RU{Я полагаю (хотя и не уверен), компилятор GCC всегда добавляет \ac{NOP}-ы (из-за \IT{branch delay slots})
после инструкций переходов и затем, если включена оптимизация, от них может избавляться.}
\EN{I guess (not sure, though) that the GCC compiler always adds \ac{NOP}s (because of \IT{branch delay slots}) after branch
instructions and then, if optimization is turned on, maybe eliminates them.}
\RU{Так что они остались здесь}\EN{So in this case they are left here}.

\RU{Вот также листинг от \IDA:}
\EN{Here is also \IDA listing:}

\lstinputlisting[caption=\NonOptimizing GCC 4.4.5 (\IDA),numbers=left]{patterns/01_helloworld/MIPS/hw_O0_IDA.lst.\LANG}

\index{MIPS!\Pseudoinstructions!LA}
\RU{Интересно, \IDA распознала пару инструкций LUI/ADDIU и собрала их в одну псевдоинструкцию 
LA (``Load Address'') в строке 15.}
\EN{Interestingly, \IDA recognized LUI/ADDIU instructions pair and coalesces them into one 
LA (``Load Address'') pseudoinstruction at line 15.}
\RU{Мы также видим, что размер этой псевдоинструкции 8 байт!}
\EN{We may also see that this pseudoinstruction has size of 8 bytes!}
\RU{Это псевдоинструкция (или \IT{макрос}) потому что это не настоящая инструкция MIPS, но скорее
просто удобное имя для пары инструкций.}
\EN{This is a pseudoinstruction (or \IT{macro}) because it's not a real MIPS instruction, but rather
handy name for an instruction pair.}

\index{MIPS!\Pseudoinstructions!NOP}
\index{MIPS!\Instructions!OR}
\RU{Еще кое что: \IDA не распознала \ac{NOP}-инструкции на строках 22, 26 и 41.}
\EN{Another thing is that \IDA doesn't recognize \ac{NOP} instructions, so here they are at lines 22, 26 and 41.}
\RU{Это}\EN{It is} \TT{OR \$AT, \$ZERO}.
\RU{По своей сути, это инструкция выполняющая операцию ИЛИ к содержимому регистра \$AT с нулем, что,
конечно же, холостая операция.}
\EN{Essentially, this instruction applies OR operation to contents of \$AT register
with zero, which is, of course, idle instruction.}
\RU{MIPS, как и многие другие \ac{ISA}, не имеет отдельной \ac{NOP}-инструкции.}
\EN{MIPS, like many other \ac{ISA}s, doesn't have a separate \ac{NOP} instruction.}

\subsection{\RU{Роль стекового фрейма в этом примере}\EN{Role of the stack frame in this example}}

\RU{Адрес текстовой строки передается в регистре.}
\EN{The address of the text string is passed in register.}
\RU{Так зачем устанавливать локальный стек?}\EN{Why setup local stack anyway?}
\RU{Причина в том, что значения регистров \ac{RA} и GP должны быть сохранены где-то
(потому что вызывается \printf) и для этого используется локальный стек.}
\EN{The reason for this lies in the fact that registers \ac{RA} and GP's values should be saved somewhere 
(because \printf is called), and the local stack is used for this purpose.}
\RU{Если бы это была \gls{leaf function}, тогда можно было бы избавиться от пролога и эпилога ф-ции,
например:}
\EN{If this was a \gls{leaf function}, it would have been possible to get rid of the function prologue and epilogue,
for example:} \myref{MIPS_leaf_function_ex1}.

\subsection{\Optimizing GCC: \RU{загрузим в}\EN{load it into} GDB}

\index{GDB}
\lstinputlisting[caption=\RU{пример сессии в GDB}\EN{sample GDB session}]{patterns/01_helloworld/MIPS/O3_GDB.txt}
