\subsection{MSVC\EMDASH{}x86-64}

\index{x86-64}
\IFRU{Попробуем также 64-битный MSVC}{Let's also try 64-bit MSVC}:

\lstinputlisting[caption=MSVC 2012 x64]{patterns/01_helloworld/MSVC_x64.asm}

\IFRU{В x86-64 все регистры были расширены до 64-х бит и теперь имеют префикс \TT{R-}}
{As of x86-64, all registers were extended to 64-bit and now have \TT{R-} prefix}.
\index{fastcall}
\IFRU{Чтобы поменьше задействовать стек (иными словами, поменьше обращаться к внешней памяти), уже давно имелся
довольно популярный метод передачи аргументов ф-ции через регистры}
{In order to use stack not that often (in other words, to access external memory not that often), there exist
a popular way to pass function arguments via registers} (fastcall: \ref{fastcall}).
\IFRU{Т.е., часть аргументов ф-ции передается через регистры и часть}{I.e., one part
of function arguments are passed in registers, other part}\EMDASH{}\IFRU{через стек}{via stack}.
\IFRU{В Win64 первые 4 аргумента ф-ции передаются через регистры}{In Win64, 4 function arguments
are passed in} \RCX, \RDX, \Reg{8}, \Reg{9}\EN{ registers}.
\IFRU{Это мы здесь и видим: указатель на строку в \printf теперь передается не через стек, а через регистр \RCX}
{That is what we see here: a pointer to the string for \printf is now passed not in stack, but in \RCX register}.

\IFRU{Указатели теперь 64-битные, так что они передаются через 64-битные части регистров (имеющие префикс \TT{R-})}
{Pointers are 64-bit now, so they are passed in 64-bit part of registers (which have \TT{R-} prefix)}.
\IFRU{Но для обратной совместимости, к регистрам можно обращаться и к 32-битным частям, используя префикс \TT{E-}}
{But for the backward compatibility, it is still possible to access 32-bit parts, using \TT{E-} prefix}.

\IFRU{Вот, например, как выглядит регистр}{That is how} \RAX/\EAX/\AX/\AL 
\IFRU{в 64-битных x86-совместимых \ac{CPU}}{looks like in 64-bit x86-compatible \ac{CPU}s}:

\RegTableOne{RAX}{EAX}{AX}{AH}{AL}

\IFRU{Ф-ция \main возвращает значение типа \Tint, который, в Си, вероятно для лучшей совместимости и портабельности,
оставили 32-битным, вот почему в конце ф-ции \main обнуляется не \RAX а \EAX, т.е., 32-битная часть регистра}
{\main function returns \Tint{}-typed value, which is, in C \ac{PL}, for the better backward compatibility
and portability, is still 32-bit, so that is why \EAX register is cleared at the function end (i.e., 32-bit
part of register) instead of \RAX{}}.

