\subsection{MSVC\EMDASH{}x86-64}

\index{x86-64}
\RU{Попробуем также 64-битный MSVC}\EN{Let's also try 64-bit MSVC}\ITA{Proviamo anche con MSVC a 64-bit}\NL{Laat ons ook eens kijken naar 64-bit MSVC}\PTBR{Vamos tentar também o MSVC 64-bits}:

\lstinputlisting[caption=MSVC 2012 x64]{patterns/01_helloworld/MSVC_x64.asm}

\index{fastcall}

\ifdefined\RUSSIAN
В x86-64 все регистры были расширены до 64-х бит и теперь имеют префикс \TT{R-}.
Чтобы поменьше задействовать стек (иными словами, поменьше обращаться кэшу и внешней памяти), уже давно имелся
довольно популярный метод передачи аргументов функции через регистры (\IT{fastcall})
\ifx\LITE\undefined \myref{fastcall} \fi
.
Т.е. часть аргументов функции передается через регистры и часть ---через стек.
В Win64 первые 4 аргумента функции передаются через регистры \RCX, \RDX, \Reg{8}, \Reg{9}.
Это мы здесь и видим: указатель на строку в \printf теперь передается не через стек, а через регистр \RCX.
Указатели теперь 64-битные, так что они передаются через 64-битные части регистров (имеющие префикс \TT{R-}).
Но для обратной совместимости можно обращаться и к нижним 32 битам регистров используя префикс \TT{E-}.
Вот как выглядит регистр \RAX/\EAX/\AX/\AL в x86-64:

\RegTableOne{RAX}{EAX}{AX}{AH}{AL}

Функция \main возвращает значение типа \Tint, который в \CCpp, вероятно для лучшей совместимости и переносимости,
оставили 32-битным. Вот почему в конце функции \main обнуляется не \RAX, а \EAX, т.е. 32-битная часть регистра.
Также видно, что 40 байт выделяются в локальном стеке.
Это \q{shadow space} которое мы будем рассматривать позже: \myref{shadow_space}.
\fi % RUSSIAN

\ifdefined\ENGLISH
In x86-64, all registers were extended to 64-bit and now their names have an \TT{R-} prefix.
In order to use the stack less often (in other words, to access external memory/cache less often), there exists
a popular way to pass function arguments via registers (\IT{fastcall})
\ifx\LITE\undefined \myref{fastcall} \fi
.
I.e., a part of the function arguments is passed in registers, the rest---via the stack.
In Win64, 4 function arguments are passed in the \RCX, \RDX, \Reg{8}, \Reg{9} registers.
That is what we see here: a pointer to the string for \printf is now passed not in the stack, but in the \RCX register.
The pointers are 64-bit now, so they are passed in the 64-bit registers (which have the \TT{R-} prefix).
However, for backward compatibility, it is still possible to access the 32-bit parts, using the \TT{E-} prefix.
This is how the \RAX/\EAX/\AX/\AL register looks like in x86-64:

\RegTableOne{RAX}{EAX}{AX}{AH}{AL}

The \main function returns an \Tint{}-typed value, which is, in \CCpp, for better backward compatibility
and portability, still 32-bit, so that is why the \EAX register is cleared at the function end (i.e., the 32-bit
part of the register) instead of \RAX{}.
There are also 40 bytes allocated in the local stack.
This is called the \q{shadow space}, about which we are going to talk later: \myref{shadow_space}.
\fi % ENGLISH

\ifdefined\ITALIAN
In x86-64, tutti i registri sono stati estesi a 64-bit ed il loro nome ha il prefisso \TT{R-}.
Per usare lo stack meno spesso (in altre parole, per accedere meno spesso alla memoria esterna/cache), esiste un metodo molto diffuso per passare gli argomenti delle funzioni tramite i registri (\IT{fastcall})
\ifx\LITE\undefined \myref{fastcall} \fi
.
Ovvero, una parte degli argomenti è passata attraverso i registri, il resto ---attraverso lo stack.
In Win64, 4 argpmenti di funzione sono passati nei registri \RCX, \RDX, \Reg{8}, \Reg{9}.
Questo è ciò che vediamo qui: un puntatore alla stringa per \printf è adesso passato nel registro \RCX anziché tramite lo stack.
I puntatori adesso sono a 64-bit , quindi sono passati nei registri a 64-bit (aventi il prefisso \TT{R-}).
E' comunque possibile, per retrocompatibilità, accedere alle parti a 32-bit parts, usando il prefisso \TT{E-}.
I registri \RAX/\EAX/\AX/\AL in x86-64 appaiono così:

\RegTableOne{RAX}{EAX}{AX}{AH}{AL}

La funzione \main restituisce un valore di tipo \Tint{}, che in \CCpp, per migliore retrocompatibilità e portabilità, resta ancora a 32-bit, motivo per cui il registro \EAX viene svuotato invece di \RAX{} alla fine della funzione (i.e., la parte a 32-bit
del registro).
Ci sono anche 40 byte allocati nello stack locale.
Questo spazio è detto \q{shadow space}, di cui parleremo più avanti: \myref{shadow_space}.
\fi % ITALIAN

\ifdefined\DUTCH
In x86-64 zijn alle registers uitgebreid tot 64-bit, en hebben hun namen een \TT{R-} prefix gekregen.
Om de stack minder te gebruiken (met andere woorden, om het externe geheugen/cache minder vaak te benaderen), bestaat
er een populaire manier om functies parameters door te geven via registers (\IT{fastcall})
\ifx\LITE\undefined \myref{fastcall} \fi
.
Bv., een deel van de parameters wordt doorgegeven via het register, de rest --- via de stack.
In Win64, worden 4 functie parameters doorgegeven via de \RCX, \RDX, \Reg{8}, \Reg{9} registers.
Dat is wat we hier zien: een pointer naar de string voor \printf wordt doorgegeven, niet via de stack, maar via het \RCX register.
De pointers zijn 64-bit nu, dus worden ze doorgegeven in de 64-bit registers (dewelke de \TT{R-} prefix hebben).
Voor backward compatibility is het echter nog steeds mogelijk om de 32-bit gedeelten aan te spreken, door gebruik te maken van de \TT{E-} prefix.
Dit is hoe de \RAX/\EAX/\AX/\AL registers eruit zien in x86-64:

\RegTableOne{RAX}{EAX}{AX}{AH}{AL}

De \main functie geeft een \Tint{}-typed waarde terug, hetwelk, in \CCpp, voor betere backward compatibiliteit
en portabiliteit, nog steeds 32-bit is. Daarom wordt het \EAX register ook leeggemaakt bij het einde van de functie
(het 32-bit gedeelte van het register) in plaats van \RAX{}.
Er zijn ook 40 bytes gealloceerd op de lokale stack.
Dit wordt de \q{shadow space} genoemd, waarover we het later nog gaan hebben: \myref{shadow_space}.
\fi % DUTCH

\ifdefined\BRAZILIAN
No x86-64, todos os registradores foram extendidos para 64-bits e agora seus nomes contém um \TT{R-} no prefixo.
A fim de diminuir a frequência com que a stack (pilha) é usada (em outras palavras, para acessar memória externa/cache menos vezes),
existe uma maneira popular de passar argumentos para funções através dos registradores (\IT{fastcall})
\ifx\LITE\undefined \myref{fastcall} \fi
.
Por exemplo, uma parte dos argumentos da função é passada nos registradores, o resto pela stack.
No Win64, 4 argumentos de funções são passados através dos registradores \RCX, \RDX, \Reg{8}, \Reg{9}.
Que é o que nós vemos, um ponteiro para a string para o printf() não é passado pela stack, mas no registrador \RCX.
Os ponteiros são 64-bits agora, então, eles são passados através dos registradores de 64-bits (que tem prefixo \TT{R-}).
Entretanto, para compatibilidade, ainda é possível acessar partes de 32-bits, usando o prefixo \TT{E-}.
É assim que os registradores \RAX/\EAX/\AX/\AL se parecem no x86-64:

\RegTableOne{RAX}{EAX}{AX}{AH}{AL}

A função \main retorna um valor do tipo inteiro, que em \CCpp é melhor para compatibilidade com versões anteriores e portabilidade,
de 32-bits, por isso o registrador \EAX é limpo no final da função (a parte de 32-bits do registrador) ao invés de \RAX.
Há também 40 bytes alocados na pilha local.
Que é chamado de ``shadow space'', o qual falaremos mais tarde: \myref{shadow_space}.
\fi % BRAZILIAN

