\subsection{MSVC}

\RU{Компилируем в}\EN{Let's compile it in} MSVC 2010:

\begin{lstlisting}
cl 1.cpp /Fa1.asm
\end{lstlisting}

\RU{(Ключ /Fa означает сгенерировать листинг на ассемблере)}
\EN{(/Fa option means generate assembly listing file)}

\begin{lstlisting}[caption=MSVC 2010]
CONST	SEGMENT
$SG3830	DB	'hello, world', 00H
CONST	ENDS
PUBLIC	_main
EXTRN	_printf:PROC
; Function compile flags: /Odtp
_TEXT	SEGMENT
_main	PROC
	push	ebp
	mov	ebp, esp
	push	OFFSET $SG3830
	call	_printf
	add	esp, 4
	xor	eax, eax
	pop	ebp
	ret	0
_main	ENDP
_TEXT	ENDS
\end{lstlisting}

\RU{MSVC выдает листинги в Intel-овском синтаксисе.}\EN{MSVC produces assembly listings in Intel-syntax.} 
\RU{Разница между Intel-синтаксисом и AT\&T будет рассмотрена немного позже:}
\EN{The difference between Intel-syntax and AT\&T-syntax will be discussed in} 
\ref{ATT_syntax}.

\RU{Компилятор сгенерировал файл \TT{1.obj}, который впоследствии будет слинкован линкером в \TT{1.exe}.} 
\EN{The compiler generated \TT{1.obj} file will be linked into \TT{1.exe}.}

\RU{В нашем случае этот файл состоит из двух сегментов: \TT{CONST} (для данных-констант) и \TT{\_TEXT} (для кода).}
\EN{In our case, the file contains two segments: \TT{CONST} (for data constants) and \TT{\_TEXT} (for code).} 

\index{\CLanguageElements!const}
\label{string_is_const_char}
\RU{Строка \TT{``hello, world''} в \CCpp имеет тип \TT{const char[]} \cite[p176, 7.3.2]{TCPPPL}, 
однако не имеет имени.}
\EN{The string \TT{``hello, world''} in \CCpp has type \TT{const char[]} \cite[p176, 7.3.2]{TCPPPL},
however it does not have its own name.}

\RU{Но компилятору нужно как-то с ней работать, так что он дает ей внутреннее имя \TT{\$SG3830}.}
\EN{The compiler needs to deal with the string somehow so it defines the internal name \TT{\$SG3830} for it.}

\RU{Так что пример можно было бы переписать вот так}\EN{So the example may be rewritten as}:

\lstinputlisting{patterns/01_helloworld/hw_2.c}

\RU{Вернемся к листингу на ассемблере. Как видно, строка заканчивается нулевым байтом ~--- 
это требования стандарта \CCpp для строк}
\EN{Let's go back to the assembly listing. 
As we can see, the string is terminated by a zero byte, which is standard for \CCpp strings}.
\RU{Больше о строках в Си}\EN{More about C strings}: \ref{C_strings}.

\RU{В сегменте кода \TT{\_TEXT} находится пока только одна функция}
\EN{In the code segment, \TT{\_TEXT}, there is only one function so far}: \main.

\RU{Функция \main, как и практически все функции, начинается с пролога и заканчивается эпилогом}
\EN{The function \main starts with prologue code and ends with epilogue code (like almost any function)}
\footnote{\RU{Об этом смотрите подробнее в разделе о прологе и эпилоге функции}
\EN{You can read more about it in section about function prolog and epilog}
~(\ref{sec:prologepilog}).}.

\index{x86!\Instructions!CALL}
\RU{Далее следует вызов функции \printf}
\EN{After the function prologue we see the call to the \printf function}: \TT{CALL \_printf}. 

\index{x86!\Instructions!PUSH}
\RU{Перед этим вызовом адрес строки (или указатель на неё) с нашим приветствием при помощи инструкции \PUSH помещается в стек.}
\EN{Before the call the string address (or a pointer to it) containing our greeting is placed on the stack with the help of the \PUSH instruction.}

\RU{После того, как функция \printf возвращает управление в функцию \main, адрес строки (или указатель на неё) всё еще лежит в стеке.}
\EN{When the \printf function returns flow control to the \main function, the string address (or a pointer to it) is still on stack.}

\RU{Так как он больше не нужен, то \glslink{stack pointer}{указатель стека} (регистр \ESP) корректируется.} 
\EN{Since we do not need it anymore, the \gls{stack pointer} (the \ESP register) needs to be corrected.}

\index{x86!\Instructions!ADD}
\TT{ADD ESP, 4} \RU{означает прибавить 4 к значению в регистре \ESP.}
\EN{means add 4 to the value in the \ESP register.}

\RU{Почему 4? Так как это 32-битный код, для передачи адреса нужно аккурат 4 байта. В x64-коде это 8 байт.}
\EN{Why 4? Since it is 32-bit code, we need exactly 4 bytes for address passing through the stack. 
It is 8 bytes in x64-code.}

\TT{``ADD ESP, 4''} \RU{эквивалентно \TT{``POP регистр''}, но без использования какого-либо регистра\footnote{Флаги
процессора, впрочем, модифицируются}.}
\EN{is effectively equivalent to \TT{``POP register''} but without using any register\footnote{CPU flags, however, are modified}.}

\index{Intel C++}
\index{\oracle}
\index{x86!\Instructions!POP}

\RU{Некоторые компиляторы, например, Intel C++ Compiler, в этой же ситуации могут вместо 
\ADD сгенерировать \TT{POP ECX} (подобное можно встретить, например, в коде \oracle{}, им скомпилированном),
что почти то же самое, только портится значение в регистре \ECX.}
\EN{Some compilers (like the Intel C++ Compiler) in the same situation may emit \TT{POP ECX} 
instead of \ADD (e.g., such a pattern can be observed in the \oracle{} code as it is compiled by the Intel C++ compiler).
This instruction has almost the same effect but the \ECX register contents will be rewritten.}
\RU{Возможно, компилятор применяет \TT{POP ECX}, потому что эта инструкция короче (1 байт против 3).}
\EN{The Intel C++ compiler probably uses \TT{POP ECX} since this instruction's opcode is shorter then 
\TT{ADD ESP, x} (1 byte against 3).}

\RU{Вот пример оттуда}\EN{Here is an example from it}:

\begin{lstlisting}[caption=\oracle 10.2 Linux (\RU{файл }app.o\EN{ file})]
.text:0800029A                 push    ebx
.text:0800029B                 call    qksfroChild
.text:080002A0                 pop     ecx
\end{lstlisting}

\RU{О стеке можно прочитать в соответствующем разделе}
\EN{Read more about the stack in section}~(\ref{sec:stack}).

\index{\CLanguageElements!return}
\RU{После вызова \printf в оригинальном коде на \CCpp указано \TT{return 0} ~--- вернуть $0$ 
в качестве результата функции \main.} 
\EN{After the call to \printf, in the original \CCpp code was 
\TT{return 0}~---return $0$ as the result of the \main function.}

\index{x86!\Instructions!XOR}
\RU{В сгенерированном коде это обеспечивается инструкцией}
\EN{In the generated code this is implemented by the instruction} \TT{XOR EAX, EAX} 

\index{x86!\Instructions!MOV}
\RU{\XOR, на самом деле, как легко догадаться, ``исключающее ИЛИ''}
\EN{\XOR is in fact, just ``eXclusive OR''}
\footnote{\href{http://go.yurichev.com/17118}{wikipedia}}
\RU{, но компиляторы часто используют его вместо простого}
\EN{but the compilers often use it instead of}
\TT{MOV EAX, 0}\RU{ ~--- снова потому, что опкод короче (2 байта против 5).}
\EN{~---again because it is a slightly shorter opcode (2 bytes against 5).}

\index{x86!\Instructions!SUB}
\RU{Бывает так, что некоторые компиляторы генерируют}\EN{Some compilers emit}
\TT{SUB EAX, EAX}, 
\RU{что значит \IT{отнять значение в \EAX от значения в \EAX}, что в любом случае даст 0 в результате.}
\EN{which means \IT{SUBtract the value in the \EAX from the value in \EAX}, which in any case will result in zero.}

\index{x86!\Instructions!RET}
\RU{Самая последняя инструкция \RET возвращает управление в вызывающую функцию.
Обычно это код \CCpp \ac{CRT}, который, в свою очередь, 
вернёт управление операционной системе.}
\EN{The last instruction \RET returns control flow to the \gls{caller}.
Usually, it is \CCpp \ac{CRT} code, which in turn returns control to the \ac{OS}.}
