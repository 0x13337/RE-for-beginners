\subsubsection{MSVC\EMDASH{}x86}

\IFRU{Компилируем в}{Let's compile it in} MSVC 2010:

\begin{lstlisting}
cl 1.cpp /Fa1.asm
\end{lstlisting}

\IFRU
{(Ключ /Fa означает сгенерировать листинг на ассемблере)}
{(/Fa option means generate assembly listing file)}

\begin{lstlisting}[caption=MSVC 2010]
CONST	SEGMENT
$SG3830	DB	'hello, world', 00H
CONST	ENDS
PUBLIC	_main
EXTRN	_printf:PROC
; Function compile flags: /Odtp
_TEXT	SEGMENT
_main	PROC
	push	ebp
	mov	ebp, esp
	push	OFFSET $SG3830
	call	_printf
	add	esp, 4
	xor	eax, eax
	pop	ebp
	ret	0
_main	ENDP
_TEXT	ENDS
\end{lstlisting}

\IFRU{MSVC выдает листинги в Intel-овском синтаксисе.}{MSVC produces assembly listings in Intel-syntax.} 
\IFRU{Разница между Intel-синтаксисом и AT\&T будет рассмотрена немного позже.}
{The difference between Intel-syntax and AT\&T-syntax will be discussed hereafter.}

\IFRU{Компилятор сгенерировал файл \TT{1.obj}, который впоследствии будет слинкован линкером в \TT{1.exe}.} 
{The compiler generated \TT{1.obj} file will be linked into \TT{1.exe}.}

\IFRU{В нашем случае этот файл состоит из двух сегментов: \TT{CONST} (для данных-констант) и \TT{\_TEXT} (для кода).}
{In our case, the file contain two segments: \TT{CONST} (for data constants) and \TT{\_TEXT} (for code).} 

\index{\CLanguageElements!const}
\IFRU{Строка \TT{``hello, world''} в \CCpp имеет тип \TT{const char*}, однако не имеет имени.}
{The string \TT{``hello, world''} in \CCpp has type \TT{const char*}, however it does not have
its own name.}

\IFRU{Но компилятору нужно как-то с ней работать, так что он дает ей внутреннее имя \TT{\$SG3830}.}
{The compiler needs to deal with the string somehow so it defines the internal name \TT{\$SG3830} for it.}

\IFRU{Так что пример можно было бы переписать вот так}{So the example may be rewritten as}:

\lstinputlisting{patterns/01_helloworld/hw_2.c}

\IFRU{Вернемся к листингу на ассемблере. Как видно, строка заканчивается нулевым байтом ~--- 
это требования стандарта \CCpp для строк}
{Let's back to the assembly listing. 
As we can see, the string is terminated by a zero byte which is standard for \CCpp strings}.
\IFRU{Больше о строках в Си}{More about C strings}: \ref{C_strings}.

\IFRU{В сегменте кода \TT{\_TEXT} находится пока только одна функция}
{In the code segment, \TT{\_TEXT}, there is only one function so far}: \main.

\IFRU{Функция \main, как и практически все функции, начинается с пролога и заканчивается эпилогом}
{The function \main starts with prologue code and ends with epilogue code (like almost any function)}
\footnote{\IFRU{Об этом смотрите подробнее в разделе о прологе и эпилоге функции}
{Read more about it in section about function prolog and epilog}
~(\ref{sec:prologepilog}).}.

\index{x86!\Instructions!CALL}
\IFRU{Далее следует вызов функции \printf}
{After the function prologue we see the call to the \printf function}: \TT{CALL \_printf}. 

\index{x86!\Instructions!PUSH}
\IFRU
{Перед этим вызовом адрес строки (или указатель на неё) с нашим приветствием при помощи инструкции \PUSH помещается в стек.}
{Before the call the string address (or a pointer to it) containing our greeting is placed on the stack with the help of the \PUSH instruction.}

\IFRU{После того, как функция \printf возвращает управление в функцию \main, адрес строки (или указатель на неё) всё еще лежит в стеке.}
{When the \printf function returns flow control to the \main function, string address (or pointer to it) is still in stack.}

\IFRU{Так как он больше не нужен, то \glslink{stack pointer}{указатель стека} (регистр \ESP) корректируется.} 
{Since we do not need it anymore the \gls{stack pointer} (the \ESP register) needs to be corrected.}

\index{x86!\Instructions!ADD}
\TT{ADD ESP, 4} \IFRU{означает прибавить 4 к значению в регистре \ESP.}
{means add 4 to the value in the \ESP register.}

\IFRU
{Почему 4? Так как это 32-битный код, для передачи адреса нужно аккурат 4 байта. В x64-коде это 8 байт.}
{Why 4? Since it is 32-bit code we need exactly 4 bytes for address passing through the stack. 
It is 8 bytes in x64-code.}

\TT{``ADD ESP, 4''} \IFRU{эквивалентно \TT{``POP регистр''}, но без использования какого-либо регистра\footnote{Флаги
процессора, впрочем, модифицируются}.}
{is effectively equivalent to \TT{``POP register''} but without using any register\footnote{CPU flags, however, are modified}.}

\index{Intel C++}
\index{Oracle RDBMS}
\index{x86!\Instructions!POP}
\IFRU{Некоторые компиляторы, например, Intel C++ Compiler, в этой же ситуации могут вместо 
\ADD сгенерировать \TT{POP ECX} (подобное можно встретить, например, в коде \oracle{}, им скомпилированном),
что почти то же самое, только портится значение в регистре \ECX.}
{Some compilers (like Intel C++ Compiler) in the same situation may emit \TT{POP ECX} 
instead of \ADD (e.g. such a pattern can be observed in the \oracle{} code as it is compiled by Intel C++ compiler).
This instruction has almost the same effect but the \ECX register contents will be rewritten.}

\IFRU
{Возможно, компилятор применяет \TT{POP ECX}, потому что эта инструкция короче (1 байт против 3).}
{The Intel C++ compiler probably uses \TT{POP ECX} since this instruction's opcode is shorter then 
\TT{ADD ESP, x} (1 byte against 3).}

\IFRU{О стеке можно прочитать в соответствующем разделе}
{Read more about the stack in section}~(\ref{sec:stack}).

\index{\CLanguageElements!return}
\IFRU{После вызова \printf в оригинальном коде на \CCpp указано \TT{return 0} ~--- вернуть $0$ 
в качестве результата функции \main.} 
{After the call to \printf, in the original \CCpp code was 
\TT{return 0}~---return $0$ as the result of the \main function.}

\index{x86!\Instructions!XOR}
\IFRU{В сгенерированном коде это обеспечивается инструкцией}
{In the generated code this is implemented by instruction} \TT{XOR EAX, EAX} 

\index{x86!\Instructions!MOV}
\IFRU{\XOR, на самом деле, как легко догадаться, ``исключающее ИЛИ''}
{\XOR is in fact, just ``eXclusive OR''}
\footnote{\url{http://en.wikipedia.org/wiki/Exclusive_or}}
\IFRU{, но компиляторы часто используют его вместо простого}
{but compilers often use it instead of}
\TT{MOV EAX, 0}\IFRU{ ~--- снова потому, что опкод короче (2 байта против 5).}
{~---again because it is a slightly shorter opcode (2 bytes against 5).}

\index{x86!\Instructions!SUB}
\IFRU{Бывает так, что некоторые компиляторы генерируют}{Some compilers emit}
\TT{SUB EAX, EAX}, 
\IFRU
{что значит \IT{отнять значение в \EAX от значения в \EAX}, что в любом случае даст 0 в результате.}
{which means \IT{SUBtract the value in the \EAX from the value in \EAX}, which in any case will result zero.}

\index{x86!\Instructions!RET}
\IFRU{Самая последняя инструкция \RET возвращает управление в вызывающую функцию.
Обычно это код \CCpp \ac{CRT}, который, в свою очередь, 
вернёт управление операционной системе.}
{The last instruction \RET returns control flow to the \gls{caller}.
Usually, it is \CCpp \ac{CRT} code which in turn returns control to the \ac{OS}.}
