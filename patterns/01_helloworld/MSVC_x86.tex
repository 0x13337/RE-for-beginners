\subsection{MSVC}

\RU{Компилируем в}\EN{Let's compile it in}\NL{We compileren het in}\PTBR{Vamos compilar esse código no} MSVC 2010:

\begin{lstlisting}
cl 1.cpp /Fa1.asm
\end{lstlisting}

\RU{(Ключ /Fa означает сгенерировать листинг на ассемблере)}%
\EN{(/Fa option instructs the compiler to generate assembly listing file)}%
\NL{(/Fa optie zorgt ervoor dat de compiler het bestand met assembly listing genereert)}%
\PTBR{(A opção /Fa instrui o compilador para gerar o arquivo de listagem em assembly)}%

\begin{lstlisting}[caption=MSVC 2010]
CONST	SEGMENT
$SG3830	DB	'hello, world', 0AH, 00H
CONST	ENDS
PUBLIC	_main
EXTRN	_printf:PROC
; Function compile flags: /Odtp
_TEXT	SEGMENT
_main	PROC
	push	ebp
	mov	ebp, esp
	push	OFFSET $SG3830
	call	_printf
	add	esp, 4
	xor	eax, eax
	pop	ebp
	ret	0
_main	ENDP
_TEXT	ENDS
\end{lstlisting}

\ifx\LITE\undefined
\RU{MSVC выдает листинги в синтаксисе Intel.}\EN{MSVC produces assembly listings in Intel-syntax.}\NLph{}
\RU{Разница между синтаксисом Intel и AT\&T будет рассмотрена немного позже:}
\EN{The difference between Intel-syntax and AT\&T-syntax will be discussed in} 
\NL{Het verschil tussen Intel-syntax en AT\&T-syntax zal besproken worden in:}\myref{ATT_syntax}.
\PTBRph{}
\fi

\RU{Компилятор сгенерировал файл \TT{1.obj}, который впоследствии будет слинкован линкером в \TT{1.exe}.
В нашем случае этот файл состоит из двух сегментов: \TT{CONST} (для данных-констант) и \TT{\_TEXT} (для кода).}%
\EN{The compiler generated the file, \TT{1.obj}, which is to be linked into \TT{1.exe}.
In our case, the file contains two segments: \TT{CONST} (for data constants) and \TT{\_TEXT} (for code).}%
\NL{De compiler heeft het bestand, \TT{1.obj} gegenereerd, hetwelk gelinkt wordt tot \TT{1.exe}.
In ons geval bevat het bestand twee segmenten: \TT{CONST} (voor data constanten) en \TT{\_TEXT}(voor code).}%
\PTBR{O compilador gerou o arquivo \TT{1.obj}, que está ligado a \TT{1.exe}.
No nosso caso, o arquivo contém dois segmentos: \TT{CONST} (para informações que são constantes) e \TT{\_TEXT} (para o código).}

\index{\CLanguageElements!const}
\label{string_is_const_char}
\RU{Строка \TT{hello, world} в \CCpp имеет тип \TT{const char[]}\cite[p176, 7.3.2]{TCPPPL}, однако не имеет имени.
Но компилятору нужно как-то с ней работать, поэтому он дает ей внутреннее имя \TT{\$SG3830}.}%
\EN{The string \TT{hello, world} in \CCpp has type \TT{const char[]}\cite[p176, 7.3.2]{TCPPPL}, but it does not have its own name.
The compiler needs to deal with the string somehow so it defines the internal name \TT{\$SG3830} for it.}%
\NL{De string \TT{hello, world} in \CCpp is van het type \TT{const char[]}\cite[p176, 7.3.2]{TCPPPL}, maar heeft geen eigen naam.
De compiler moet een manier hebben om met de string om te kunnen, en definieert er daarom de interne naam \TT{\$SG3830} voor.}%
\PTBR{A string \TT{hello, word} em \CCpp tem seu tipo const \TT{const char[]}\cite[p176, 7.3.2]{TCPPPL}, mas não tem um nome.
O compilador precisa lidar com essa string de alguma maneira, definindo então o nome de \TT{\$SG3830} para ela.}%

\RU{Поэтому пример можно было бы переписать вот так:}%
\EN{That is why the example may be rewritten as follows:}%
\NL{Daarom kan het voorbeeld herschreven worden als volgt:}%
\PTBR{Assim, o código pode ser reescrito da seguinte maneira:}%

\lstinputlisting{patterns/01_helloworld/hw_2.c}

\RU{Вернемся к листингу на ассемблере. Как видно, строка заканчивается нулевым байтом~--- это требования стандарта \CCpp для строк.}%
\EN{Let's go back to the assembly listing. As we can see, the string is terminated by a zero byte, which is standard for \CCpp strings.}%
\NL{Laten we terug gaan naar de assembly listing. Zoals je kan zien, wordt de string beeindigd door een nul-byte. Dit is standaard voor \CCpp strings.}%
\PTBR{Vamos voltar para a listagem em assembly. Como podemos ver, a string é delimitada por um byte de valor zero, o que é padrão para strings em \CCpp.}%
\RU{Больше о строках в \CCpp}\EN{More about \CCpp strings}\NL{Meer over \CCpp strings}\PTBR{Mais sobre strings em \CCpp}: \myref{C_strings}.

\RU{В сегменте кода \TT{\_TEXT} находится пока только одна функция: \main{}.}%
\EN{In the code segment, \TT{\_TEXT}, there is only one function so far: \main{}.}%
\NL{In het code segment, \TT{\_TEXT}, is er slechts een functie tot nu toe: \main{}.}%
\PTBR{No segmento de código \TT{\_TEXT}, só há uma função por enquanto: \main{}.}%
\RU{Функция \main, как и практически все функции, начинается с пролога и заканчивается эпилогом}%
\EN{The function \main starts with prologue code and ends with epilogue code (like almost any function)}%
\NL{De functie \main begint met een proloog code en eindigt met een epiloog code (zoals bijna elke functie)}%
\PTBR{A função \main{} começa com um código como cabeçalho e termina com outro como rodapé (quase como qualquer outra função)}%
\footnote{\RU{Об этом смотрите подробнее в разделе о прологе и эпилоге функции}%
\EN{You can read more about it in the section about function prologues and epilogues}%
\NL{Je kan hier meer over lezen in de sectie over functieprologen en epilogen}%
\PTBRph{}
~(\myref{sec:prologepilog}).}.

\index{x86!\Instructions!CALL}
\RU{Далее следует вызов функции \printf{}: \TT{CALL \_printf}.}%
\EN{After the function prologue we see the call to the \printf{} function: \TT{CALL \_printf}.}%
\NL{Na de functie proloog zien we de call naar de \printf{} functie: \TT{CALL \_printf}.}%
\PTBR{Depois do cabeçalho da função, podemos ver a chamada para a função \printf{}: \TT{CALL \_printf}.}%
\index{x86!\Instructions!PUSH}
\RU{Перед этим вызовом адрес строки (или указатель на неё) с нашим приветствием при помощи инструкции \PUSH помещается в стек.}
\EN{Before the call the string address (or a pointer to it) containing our greeting is placed on the stack with the help of the \PUSH instruction.}
\NL{Voor de call wordt het adres van de string (of een pointer ernaar) die onze begroeting bevat, op de stack geplaatsd met de hulp van de \PUSH instructie.}
\PTBR{Antes da chamada, o endereço da string (ou um ponteiro para o mesmo) contendo nossa saudação (``Hello, world!'') é colocado na stack com a ajuda a instrução \PUSH.}

\RU{После того, как функция \printf возвращает управление в функцию \main, адрес строки (или указатель на неё) всё ещё лежит в стеке.
Так как он больше не нужен, то \glslink{stack pointer}{указатель стека} (регистр \ESP) корректируется.}%
\EN{When the \printf function returns the control to the \main function, the string address (or a pointer to it) is still on the stack.
Since we do not need it anymore, the \gls{stack pointer} (the \ESP register) needs to be corrected.}%
\NL{Wanneer de \printf functie de controle teruggeeft aan de \main functie, staat het string adres (of de pointer ernaar) nog steeds op de stack.
Aangezien we dit niet meer nodig hebben, moet de \gls{stack pointer} (het \ESP register) gecorrigeerd worden.}%
\PTBR{Quando o a função printf() retorna o controle para a função main(), o endereço da string (ou o ponteiro para a mesma) ainda está na stack.
Como não precisamos mais dela, o apontador da stack (registrador \ESP) precisa ser corrigido.}

\index{x86!\Instructions!ADD}
\RU{\TT{ADD ESP, 4} означает прибавить 4 к значению в регистре \ESP.}%
\EN{\TT{ADD ESP, 4} means add 4 to the \ESP register value.}%
\NL{\TT{ADD ESP, 4} betekent dat er 4 wordt opgeteld bij de \ESP registerwaarde.}%
\PTBR{\TT{ADD ESP, 4} significa adicionar 4 para o valor do registrador \ESP.}%

\RU{Почему 4? Так как это 32-битный код, для передачи адреса нужно 4 байта. В x64-коде это 8 байт.
\TT{ADD ESP, 4} эквивалентно \TT{POP регистр}, но без использования какого-либо регистра\footnote{Флаги процессора, впрочем, модифицируются}.}%
\EN{Why 4? Since this is a 32-bit program, we need exactly 4 bytes for address passing through the stack. If it was x64 code we would need 8 bytes.
\TT{ADD ESP, 4} is effectively equivalent to \TT{POP register} but without using any register\footnote{CPU flags, however, are modified}.}%
\NL{Waarom 4? Aangezien dit een 32-bit programma is, hebben we exact 4 bytes nodig om een adres door te geven via de stack. als het x64 code was, zouden we 8 bytes nodig gehad hebben.
\TT{ADD ESP, 4} is een effectief equivalent voor \TT{POP register} maar zonder gebruik van een register\footnote{CPU flags worden echter wel aangepast}.}%
\PTBR{Mas por que 4? Como esse é um programa de 32-bits, nós precisamos exatamente 4 bytes para endereço passando pela stack.
\TT{ADD ESP, 4} é equivalente a um POP mas sem precisar de nenhum registrador\footnote{\ac{TBT}: CPU flags worden echter wel aangepast}.}%

\index{Intel C++}
\index{\oracle}
\index{x86!\Instructions!POP}

\RU{Некоторые компиляторы, например, Intel C++ Compiler, в этой же ситуации могут вместо 
\ADD сгенерировать \TT{POP ECX} (подобное можно встретить, например, в коде \oracle{}, им скомпилированном),
что почти то же самое, только портится значение в регистре \ECX.
Возможно, компилятор применяет \TT{POP ECX}, потому что эта инструкция короче (1 байт у \TT{POP} против 3 у \TT{ADD}).}%
\EN{For the same purpose, some compilers (like the Intel C++ Compiler) may emit \TT{POP ECX} 
instead of \ADD (e.g., such a pattern can be observed in the \oracle{} code as it is compiled with the Intel C++ compiler).
This instruction has almost the same effect but the \ECX register contents will be overwritten.
The Intel C++ compiler probably uses \TT{POP ECX} since this instruction's opcode is shorter than \TT{ADD ESP, x} (1 byte for \TT{POP} against 3 for \TT{ADD}).}%
\NL{Met dezelfde reden zullen sommige compilers (zoals de Intel C++ Compiler) gebruik maken van \TT{POP ECX}
in plaats van \ADD (een dergelijk patroon kan waargenomen worden in de \oracle{} code aangezien deze gecompileerd is met de Intel C++ compiler).
Deze instructie heeft bijna hetzelfde effect, maar de inhoud van het \ECX register zal overschreven worden.
De Intel C++ Compiler gebruikt waarschijnlijk \TT{POP ECX} aangezien de opcode van deze instructie korter is dan \TT{ADD ESP, x} (1 byte voor \TT{POP} tegen 3 voor \TT{ADD}).}%
\PTBR{Pelos mesmos motivos, alguns compiladores (como o Intel C++ Compiler) podem emitir \TT{POP ECX} ao invés de \ADD (esse padrão pode ser observado no código do \oracle{} pois ele é compilado com o Intel C++ Compiler).
Essa instrução tem quase o mesmo efeito mas o conteúdo de ECX seria apagado.
O Intel C++ provavelmente usa \TT{POP ECX} pois o opcode é menor do que \TT{ADD ESP, x} (1 byte para \POP ao invés de 3 para \ADD).}%

\RU{Вот пример использования \POP вместо \ADD из \oracle{}:}%
\EN{Here is an example of using \POP instead of \ADD from \oracle{}:}%
\NL{Hier is een voorbeeld van het gebruik van \POP in plaats van \ADD van \oracle{}:}%
\PTBR{Aqui está um exemplo do uso de \POP ao invés de \ADD do \oracle{}:}%

\begin{lstlisting}[caption=\oracle 10.2 Linux (\RU{файл }app.o\EN{ file}\NL{ bestand}\PTBR{ file})]
.text:0800029A                 push    ebx
.text:0800029B                 call    qksfroChild
.text:080002A0                 pop     ecx
\end{lstlisting}

%\RU{О стеке можно прочитать в соответствующем разделе}
%\EN{Read more about the stack in section}
%\NL{Lees meer over de stack in de sectie} ~(\myref{sec:stack}).
\index{\CLanguageElements!return}
\RU{После вызова \printf в оригинальном коде на \CCpp указано \TT{return 0}~--- вернуть 0 в качестве результата функции \main.}%
\EN{After calling \printf, the original \CCpp code contains the statement \TT{return 0}~---return 0 as the result of the \main function.}%
\NL{Na \printf aan te roepen, bevat de originele \CCpp code het statement \TT{return 0}~---return 0 als resultaat van de \main functie.}%
\PTBR{Depois de chamar \printf{}, o código original em \CCpp contém a declaração \TT{return 0} --- return 0 como o resultado da função \main{}.}%

\index{x86!\Instructions!XOR}
\RU{В сгенерированном коде это обеспечивается инструкцией \INS{XOR EAX, EAX}.}%
\EN{In the generated code this is implemented by the instruction \INS{XOR EAX, EAX}.}%
\NL{In de gegenereerde code wordt dit geimplementeerd door de instructie \INS{XOR EAX, EAX}.}%
\PTBR{No código gerado, isso é implementado pela instrução \INS{XOR EAX, EAX}.}%

\index{x86!\Instructions!MOV}

\RU{\XOR, как легко догадаться~--- \q{исключающее ИЛИ}\footnote{\href{http://go.yurichev.com/17118}{wikipedia}}, но компиляторы часто используют его вместо простого
\INS{MOV EAX, 0} --- снова потому, что опкод короче (2 байта у \XOR против 5 у \MOV).}%
\EN{\XOR is in fact just \q{eXclusive OR}\footnote{\href{http://go.yurichev.com/17118}{wikipedia}} but the compilers often use it instead of
\INS{MOV EAX, 0}---again because it is a slightly shorter opcode (2 bytes for \XOR against 5 for \MOV).}%
\NL{\XOR is feitelijk simpelweg \q{eXclusive OR}\footnote{\href{http://go.yurichev.com/17118}{wikipedia}} maar de compilers gebruiken het vaak in plaats van
\INS{MOV EAX, 0} --- wederom omdat de opcode hiervoor iets korter is (2 bytes voor \XOR tegenover 5 voor \MOV).}%
\PTBR{\XOR é a condição lógica ``ou exclusivo''\footnote{\href{http://go.yurichev.com/17118}{wikipedia}} que os compiladores geralmente usam ao invés de 
\INS{MOV EAX, 0} --- de novo por causa de um pequeno decréscimo no número de bytes necessários (2 bytes para \XOR contra 5 para a instrução \MOV).}%

\index{x86!\Instructions!SUB}
\RU{Некоторые компиляторы генерируют \INS{SUB EAX, EAX}, что значит \IT{отнять значение в} \EAX \IT{от значения в }\EAX, что в любом случае даст 0 в результате.}%
\EN{Some compilers emit \INS{SUB EAX, EAX}, which means \IT{SUBtract the value in the} \EAX \IT{from the value in} \EAX, which, in any case, results in zero.}%
\NL{Sommige compilers gebruiken \INS{SUB EAX, EAX}, wat staat voor \IT{verminder de waarde in} \EAX \IT{met de waarde in} \EAX, wat in elke situatie resulteert in nul.}%
\PTBR{Alguns compiladores também usam \INS{SUB EAX, EAX}, que significa, SUBtrair o valor contido em \EAX do valor em \EAX, que, em qualquer caso, resultará em zero.}%

\index{x86!\Instructions!RET}
\RU{Самая последняя инструкция \RET возвращает управление в вызывающую функцию. Обычно это код \CCpp \ac{CRT}, который, в свою очередь, вернёт управление в \ac{OS}.}
\EN{The last instruction \RET returns the control to the \gls{caller}. Usually, this is \CCpp \ac{CRT} code, which, in turn, returns control to the \ac{OS}.}
\NL{De laatste instructie \RET geeft de controle terug aan de \gls{caller}. Gewoonlijk is dit \CCpp \ac{CRT} code, die op zijn beurt de controle teruggeeft aan het \ac{OS}.}
\PTBR{A última instrução \RET retorna o controle para onde a função foi chamada. Geralmente, isso é código \CCpp \ac{CRT}, que retorna o controle para o sistema operacional.}

