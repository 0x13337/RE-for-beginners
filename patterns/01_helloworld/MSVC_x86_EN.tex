\subsection{MSVC}

Let's compile it in MSVC 2010:

\begin{lstlisting}
cl 1.cpp /Fa1.asm
\end{lstlisting}

(\TT{/Fa} option instructs the compiler to generate assembly listing file)

\begin{lstlisting}[caption=MSVC 2010]
CONST	SEGMENT
$SG3830	DB	'hello, world', 0AH, 00H
CONST	ENDS
PUBLIC	_main
EXTRN	_printf:PROC
; Function compile flags: /Odtp
_TEXT	SEGMENT
_main	PROC
	push	ebp
	mov	ebp, esp
	push	OFFSET $SG3830
	call	_printf
	add	esp, 4
	xor	eax, eax
	pop	ebp
	ret	0
_main	ENDP
_TEXT	ENDS
\end{lstlisting}

MSVC produces assembly listings in Intel-syntax.
The difference between Intel-syntax and AT\&T-syntax will be discussed in \myref{ATT_syntax}.

The compiler generated the file, \TT{1.obj}, which is to be linked into \TT{1.exe}.
In our case, the file contains two segments: \TT{CONST} (for data constants) and \TT{\_TEXT} (for code).

\myindex{\CLanguageElements!const}
\label{string_is_const_char}
The string \TT{hello, world} in \CCpp has type \TT{const char[]}\cite[p176, 7.3.2]{TCPPPL}, but it does not have its own name.
The compiler needs to deal with the string somehow so it defines the internal name \TT{\$SG3830} for it.

That is why the example may be rewritten as follows:

\lstinputlisting{patterns/01_helloworld/hw_2.c}

Let's go back to the assembly listing. As we can see, the string is terminated by a zero byte, which is standard for \CCpp strings.
More about \CCpp strings: \myref{C_strings}.

In the code segment, \TT{\_TEXT}, there is only one function so far: \main{}.
The function \main starts with prologue code and ends with epilogue code (like almost any function)
\footnote{You can read more about it in the section about function prologues and epilogues ~(\myref{sec:prologepilog}).}.

\myindex{x86!\Instructions!CALL}
After the function prologue we see the call to the \printf{} function: \INS{CALL \_printf}.
\myindex{x86!\Instructions!PUSH}
Before the call the string address (or a pointer to it) containing our greeting is placed on the stack with the help of the \PUSH instruction.

When the \printf function returns the control to the \main function, the string address (or a pointer to it) is still on the stack.
Since we do not need it anymore, the \gls{stack pointer} (the \ESP register) needs to be corrected.

\myindex{x86!\Instructions!ADD}
\INS{ADD ESP, 4} means add 4 to the \ESP register value.

Why 4? Since this is a 32-bit program, we need exactly 4 bytes for address passing through the stack. If it was x64 code we would need 8 bytes.
\INS{ADD ESP, 4} is effectively equivalent to \INS{POP register} but without using any register\footnote{CPU flags, however, are modified}.

\myindex{Intel C++}
\myindex{\oracle}
\myindex{x86!\Instructions!POP}

For the same purpose, some compilers (like the Intel C++ Compiler) may emit \TT{POP ECX} 
instead of \ADD (e.g., such a pattern can be observed in the \oracle{} code as it is compiled with the Intel C++ compiler).
This instruction has almost the same effect but the \ECX register contents will be overwritten.
The Intel C++ compiler probably uses \TT{POP ECX} since this instruction's opcode is shorter than \TT{ADD ESP, x} (1 byte for \TT{POP} against 3 for \TT{ADD}).

Here is an example of using \POP instead of \ADD from \oracle{}:

\begin{lstlisting}[caption=\oracle 10.2 Linux (app.o file)]
.text:0800029A                 push    ebx
.text:0800029B                 call    qksfroChild
.text:080002A0                 pop     ecx
\end{lstlisting}

%Read more about the stack in section
% ~(\myref{sec:stack}).
\myindex{\CLanguageElements!return}
After calling \printf, the original \CCpp code contains the statement \TT{return 0}~---return 0 as the result of the \main function.

\myindex{x86!\Instructions!XOR}
In the generated code this is implemented by the instruction \INS{XOR EAX, EAX}.

\myindex{x86!\Instructions!MOV}

\XOR is in fact just \q{eXclusive OR}\footnote{\href{http://go.yurichev.com/17118}{wikipedia}} but the compilers often use it instead of
\INS{MOV EAX, 0}---again because it is a slightly shorter opcode (2 bytes for \XOR against 5 for \MOV).

\myindex{x86!\Instructions!SUB}
Some compilers emit \INS{SUB EAX, EAX}, which means \IT{SUBtract the value in the} \EAX \IT{from the value in} \EAX, which, in any case, results in zero.

\myindex{x86!\Instructions!RET}
The last instruction \RET returns the control to the \gls{caller}. Usually, this is \CCpp \ac{CRT} code, which, in turn, returns control to the \ac{OS}.

