\chapter{\HelloWorldSectionName}
\label{sec:helloworld}

\RU{Продолжим, используя знаменитый пример из книги}
<<<<<<< HEAD
\EN{Let's use the famous example from the book}
``The C programming Language''\cite{Kernighan:1988:CPL:576122}:
=======
\EN{Let's continue with the famous example from the}
\q{The C programming Language} \cite{Kernighan:1988:CPL:576122}\EN{ book}:
>>>>>>> importchanges

\lstinputlisting{patterns/01_helloworld/hw.c}

\section{x86}

\subsubsection{MSVC\EMDASH{}x86}

\IFRU{Компилируем в}{Let's compile it in} MSVC 2010:

\begin{lstlisting}
cl 1.cpp /Fa1.asm
\end{lstlisting}

\IFRU
{(Ключ /Fa означает сгенерировать листинг на ассемблере)}
{(/Fa option means generate assembly listing file)}

\begin{lstlisting}[caption=MSVC 2010]
CONST	SEGMENT
$SG3830	DB	'hello, world', 00H
CONST	ENDS
PUBLIC	_main
EXTRN	_printf:PROC
; Function compile flags: /Odtp
_TEXT	SEGMENT
_main	PROC
	push	ebp
	mov	ebp, esp
	push	OFFSET $SG3830
	call	_printf
	add	esp, 4
	xor	eax, eax
	pop	ebp
	ret	0
_main	ENDP
_TEXT	ENDS
\end{lstlisting}

\IFRU{MSVC выдает листинги в Intel-овском синтаксисе.}{MSVC produces assembly listings in Intel-syntax.} 
\IFRU{Разница между Intel-синтаксисом и AT\&T будет рассмотрена немного позже.}
{The difference between Intel-syntax and AT\&T-syntax will be discussed hereafter.}

\IFRU{Компилятор сгенерировал файл \TT{1.obj}, который впоследствии будет слинкован линкером в \TT{1.exe}.} 
{The compiler generated \TT{1.obj} file will be linked into \TT{1.exe}.}

\IFRU{В нашем случае этот файл состоит из двух сегментов: \TT{CONST} (для данных-констант) и \TT{\_TEXT} (для кода).}
{In our case, the file contain two segments: \TT{CONST} (for data constants) and \TT{\_TEXT} (for code).} 

\index{\CLanguageElements!const}
\IFRU{Строка \TT{``hello, world''} в \CCpp имеет тип \TT{const char*}, однако не имеет имени.}
{The string \TT{``hello, world''} in \CCpp has type \TT{const char*}, however it does not have
its own name.}

\IFRU{Но компилятору нужно как-то с ней работать, так что он дает ей внутреннее имя \TT{\$SG3830}.}
{The compiler needs to deal with the string somehow so it defines the internal name \TT{\$SG3830} for it.}

\IFRU{Так что пример можно было бы переписать вот так}{So the example may be rewritten as}:

\lstinputlisting{patterns/01_helloworld/hw_2.c}

\IFRU{Вернемся к листингу на ассемблере. Как видно, строка заканчивается нулевым байтом ~--- 
это требования стандарта \CCpp для строк}
{Let's back to the assembly listing. 
As we can see, the string is terminated by a zero byte which is standard for \CCpp strings}.
\IFRU{Больше о строках в Си}{More about C strings}: \ref{C_strings}.

\IFRU{В сегменте кода \TT{\_TEXT} находится пока только одна функция}
{In the code segment, \TT{\_TEXT}, there is only one function so far}: \main.

\IFRU{Функция \main, как и практически все функции, начинается с пролога и заканчивается эпилогом}
{The function \main starts with prologue code and ends with epilogue code (like almost any function)}
\footnote{\IFRU{Об этом смотрите подробнее в разделе о прологе и эпилоге функции}
{Read more about it in section about function prolog and epilog}
~(\ref{sec:prologepilog}).}.

\index{x86!\Instructions!CALL}
\IFRU{Далее следует вызов функции \printf}
{After the function prologue we see the call to the \printf function}: \TT{CALL \_printf}. 

\index{x86!\Instructions!PUSH}
\IFRU
{Перед этим вызовом адрес строки (или указатель на неё) с нашим приветствием при помощи инструкции \PUSH помещается в стек.}
{Before the call the string address (or a pointer to it) containing our greeting is placed on the stack with the help of the \PUSH instruction.}

\IFRU{После того, как функция \printf возвращает управление в функцию \main, адрес строки (или указатель на неё) всё еще лежит в стеке.}
{When the \printf function returns flow control to the \main function, string address (or pointer to it) is still in stack.}

\IFRU{Так как он больше не нужен, то \glslink{stack pointer}{указатель стека} (регистр \ESP) корректируется.} 
{Since we do not need it anymore the \gls{stack pointer} (the \ESP register) needs to be corrected.}

\index{x86!\Instructions!ADD}
\TT{ADD ESP, 4} \IFRU{означает прибавить 4 к значению в регистре \ESP.}
{means add 4 to the value in the \ESP register.}

\IFRU
{Почему 4? Так как это 32-битный код, для передачи адреса нужно аккурат 4 байта. В x64-коде это 8 байт.}
{Why 4? Since it is 32-bit code we need exactly 4 bytes for address passing through the stack. 
It is 8 bytes in x64-code.}

\TT{``ADD ESP, 4''} \IFRU{эквивалентно \TT{``POP регистр''}, но без использования какого-либо регистра\footnote{Флаги
процессора, впрочем, модифицируются}.}
{is effectively equivalent to \TT{``POP register''} but without using any register\footnote{CPU flags, however, are modified}.}

\index{Intel C++}
\index{Oracle RDBMS}
\index{x86!\Instructions!POP}
\IFRU{Некоторые компиляторы, например, Intel C++ Compiler, в этой же ситуации могут вместо 
\ADD сгенерировать \TT{POP ECX} (подобное можно встретить, например, в коде \oracle{}, им скомпилированном),
что почти то же самое, только портится значение в регистре \ECX.}
{Some compilers (like Intel C++ Compiler) in the same situation may emit \TT{POP ECX} 
instead of \ADD (e.g. such a pattern can be observed in the \oracle{} code as it is compiled by Intel C++ compiler).
This instruction has almost the same effect but the \ECX register contents will be rewritten.}

\IFRU
{Возможно, компилятор применяет \TT{POP ECX}, потому что эта инструкция короче (1 байт против 3).}
{The Intel C++ compiler probably uses \TT{POP ECX} since this instruction's opcode is shorter then 
\TT{ADD ESP, x} (1 byte against 3).}

\IFRU{О стеке можно прочитать в соответствующем разделе}
{Read more about the stack in section}~(\ref{sec:stack}).

\index{\CLanguageElements!return}
\IFRU{После вызова \printf в оригинальном коде на \CCpp указано \TT{return 0} ~--- вернуть $0$ 
в качестве результата функции \main.} 
{After the call to \printf, in the original \CCpp code was 
\TT{return 0}~---return $0$ as the result of the \main function.}

\index{x86!\Instructions!XOR}
\IFRU{В сгенерированном коде это обеспечивается инструкцией}
{In the generated code this is implemented by instruction} \TT{XOR EAX, EAX} 

\index{x86!\Instructions!MOV}
\IFRU{\XOR, на самом деле, как легко догадаться, ``исключающее ИЛИ''}
{\XOR is in fact, just ``eXclusive OR''}
\footnote{\url{http://en.wikipedia.org/wiki/Exclusive_or}}
\IFRU{, но компиляторы часто используют его вместо простого}
{but compilers often use it instead of}
\TT{MOV EAX, 0}\IFRU{ ~--- снова потому, что опкод короче (2 байта против 5).}
{~---again because it is a slightly shorter opcode (2 bytes against 5).}

\index{x86!\Instructions!SUB}
\IFRU{Бывает так, что некоторые компиляторы генерируют}{Some compilers emit}
\TT{SUB EAX, EAX}, 
\IFRU
{что значит \IT{отнять значение в \EAX от значения в \EAX}, что в любом случае даст 0 в результате.}
{which means \IT{SUBtract the value in the \EAX from the value in \EAX}, which in any case will result zero.}

\index{x86!\Instructions!RET}
\IFRU{Самая последняя инструкция \RET возвращает управление в вызывающую функцию.
Обычно это код \CCpp \ac{CRT}, который, в свою очередь, 
вернёт управление операционной системе.}
{The last instruction \RET returns control flow to the \gls{caller}.
Usually, it is \CCpp \ac{CRT} code which in turn returns control to the \ac{OS}.}

\ifdefined\IncludeGCC
\subsection{GCC}

\RU{Теперь скомпилируем то же самое компилятором GCC 4.4.1 в Linux}
\EN{Now let's try to compile the same \CCpp code in the GCC 4.4.1 compiler in Linux}: \TT{gcc 1.c -o 1}

\RU{Затем, при помощи \IDA{}, посмотрим, как скомпилировалась функция \main.}
\EN{Next, with the assistance of the \IDA disassembler, let's see how the \main function was created.} 

(\IDA, \RU{как и MSVC, показывает код в синтаксисе Intel}\EN{like MSVC, uses Intel-syntax})
\footnote{N.B. \RU{Мы также можем заставить GCC генерировать листинги в этом формате при помощи ключей}
\EN{We could also have GCC produce assembly listings in Intel-syntax by applying the options} 
\TT{-S -masm=intel}.}.

\begin{lstlisting}[caption=\RU{код в}\EN{code in} \IDA]
main            proc near

var_10          = dword ptr -10h

                push    ebp
                mov     ebp, esp
                and     esp, 0FFFFFFF0h
                sub     esp, 10h
                mov     eax, offset aHelloWorld ; "hello, world\n"
                mov     [esp+10h+var_10], eax
                call    _printf
                mov     eax, 0
                leave
                retn
main            endp
\end{lstlisting}

\index{Function prologue}
\index{x86!\Instructions!AND}
\RU{Почти то же самое. 
Адрес строки ``hello, world'', лежащей в сегменте данных, вначале сохраняется в \EAX, затем записывается в стек.
А ещё в прологе функции мы видим \TT{AND ESP, 0FFFFFFF0h}~--- 
эта инструкция выравнивает значение в \ESP по 16-байтной границе, делая все значения 
в стеке также выровненными по этой границе (процессор более эффективно работает с переменными, расположенными
в памяти по адресам кратным 4 или 16)\footnote{\URLWPDA}.}
\EN{The result is almost the same.
The address of the ``hello, world'' string (stored in the data segment) is loaded in the \EAX register first and then it is saved onto the stack.
In addition, the function prologue contains \TT{AND ESP, 0FFFFFFF0h}~---this 
instruction aligns the \ESP register value on a 16-byte boundary.
This results in all values in the stack being aligned the same way (The CPU performs better if the values it is dealing with are located in memory at addresses aligned 
on a 4- or 16-byte boundary)\footnote{\URLWPDA}.}

\index{x86!\Instructions!SUB}
\TT{SUB ESP, 10h} \RU{выделяет в стеке 16 байт. Хотя, как будет видно далее, здесь достаточно только 4.}
\EN{allocates 16 bytes on the stack. Although, as we can see hereafter, only 4 are necessary here.} 

\RU{Это происходит потому, что количество выделяемого места в локальном стеке тоже выровнено по 
16-байтной границе.}
\EN{This is because the size of the allocated stack is also aligned on a 16-byte boundary.}

% TODO1: rewrite.
\index{x86!\Instructions!PUSH}
\RU{Адрес строки (или указатель на строку) затем записывается прямо в стек без помощи инструкции \PUSH.
\IT{var\_10} одновременно и локальная переменная и аргумент для \printf{}. Подробнее об этом будет ниже.}
\EN{The string address (or a pointer to the string) is then stored directly onto the stack without using the \PUSH instruction.
\IT{var\_10}~---is a local variable and is also an argument for \printf{}.
Read about it below.}

\RU{Затем вызывается \printf.}\EN{Then the \printf function is called.}

\RU{В отличие от MSVC, GCC в компиляции без включенной оптимизации генерирует \TT{MOV EAX, 0} вместо 
более короткого опкода.}\EN{Unlike MSVC, when GCC is compiling without optimization turned on,
it emits \TT{MOV EAX, 0} instead of a shorter opcode.}

\index{x86!\Instructions!LEAVE}
\RU{Последняя инструкция \LEAVE~--- это аналог команд \TT{MOV ESP, EBP} и \TT{POP EBP}~--- 
то есть возврат \glslink{stack pointer}{указателя стека} и регистра \EBP в первоначальное состояние.} 
\EN{The last instruction, \LEAVE~---is the equivalent of the \TT{MOV ESP, EBP} and \TT{POP EBP} instruction 
pair~---in other words, this instruction sets the \gls{stack pointer} (\ESP) back and restores 
the \EBP register to its initial state.}

\RU{Это необходимо, т.к., в начале функции мы модифицировали регистры \ESP и \EBP (при помощи}
\EN{This is necessary since we modified these register values (\ESP and \EBP) at the 
beginning of the function (executing}
\TT{\MOV EBP, ESP} / \TT{AND ESP, \ldots}).

\subsection{GCC: \ATTSyntax}
\label{ATT_syntax}

\RU{Попробуем посмотреть, как выглядит то же самое в AT\&T-синтаксисе языка ассемблера.}
\EN{Let's see how this can be represented in assembly language AT\&T syntax.}
\RU{Этот синтаксис больше распространен в UNIX-мире.}
\EN{This syntax is much more popular in the UNIX-world.}

\begin{lstlisting}[caption=\RU{компилируем в}\EN{let's compile in} GCC 4.7.3]
gcc -S 1_1.c
\end{lstlisting}

\RU{Получим такой файл:}\EN{We get this:}

\lstinputlisting[caption=GCC 4.7.3]{patterns/01_helloworld/GCC.s}

\RU{Здесь много макросов (начинающихся с точки). Они нам пока не интересны.}
\EN{The listing contains many macros (beginning with dot). These are not interesting for us at the moment.}
\RU{Пока что, ради упрощения, мы можем
их игнорировать (кроме макроса \IT{.string}, при помощи которого кодируется последовательность символов, 
оканчивающихся нулем~--- такие же строки как в Си). И тогда получится следующее}%
\EN{For now, for the sake of simplification, we can ignore them (except the \IT{.string} macro which
encodes a null-terminated character sequence just like a C-string). Then we'll see this}%
%
% TODO: I would suggest moving this particular footnote to the main text. IMHO this will improve the readability.
\footnote{\RU{Кстати, для уменьшения генерации ``лишних'' макросов, можно использовать такой ключ GCC}%
\EN{This GCC option can be used to eliminate ``unnecessary'' macros}: 
\IT{-fno-asynchronous-unwind-tables}}:

\lstinputlisting[caption=GCC 4.7.3]{patterns/01_helloworld/GCC_refined.s}

\index{\ATTSyntax}
\index{\IntelSyntax}
\RU{Основные отличия синтаксиса Intel и AT\&T следующие:}
\EN{Some of the major differences between Intel and AT\&T syntax are:}

\begin{itemize}

\item
\RU{Операнды записываются наоборот.}\EN{Source and destination operands are written in opposite order.}

\RU{В Intel-синтаксисе: <инструкция> <операнд назначения> <операнд-источник>.}
\EN{In Intel-syntax: <instruction> <destination operand> <source operand>.}

\RU{В AT\&T-синтаксисе: <инструкция> <операнд-источник> <операнд назначения>.}
\EN{In AT\&T syntax: <instruction> <source operand> <destination operand>.}

\RU{Чтобы легче понимать разницу, можно запомнить следующее}%
\EN{Here is a way to easy memorise the difference}: \RU{когда вы работаете с Intel-синтаксисом~--- можете в уме ставить знак равенства ($=$) между операндами,}
\EN{when you deal with Intel-syntax, you can imagine that there is an equality sign ($=$) between operands}
\RU{а когда с AT\&T-синтаксисом~--- мысленно ставьте стрелку направо}
\EN{and when you deal with AT\&T-syntax imagine there is a right arrow} 
($\rightarrow$)
\footnote{
\index{\CStandardLibrary!memcpy()}
\index{\CStandardLibrary!strcpy()}
\RU{Кстати, в некоторых стандартных функциях библиотеки Си (например, memcpy(), strcpy()) также применяется 
расстановка аргументов как в Intel-синтаксисе: вначале указатель в памяти на блок назначения, 
затем указатель на блок-источник.}\EN{By the way, in some C standard functions (e.g., memcpy(), strcpy()) the arguments
are listed in the same way as in Intel-syntax: pointer to the destination memory block at the beginning and then
pointer to the source memory block.}}.

\item
AT\&T: \RU{Перед именами регистров ставится знак процента (\%), а перед числами знак доллара (\$).}
\EN{Before register names, a percent sign must be written (\%) and before numbers a dollar sign (\$).}
\RU{Вместо квадратных скобок применяются круглые.}\EN{Parentheses are used instead of brackets.}

\item
AT\&T: \RU{К каждой инструкции добавляется специальный символ, определяющий тип данных:}
\EN{Suffix is added to instructions to define the operand size:}

\begin{itemize}
\item q --- quad (64 \RU{бита}\EN{bits})
\item l --- long (32 \RU{бита}\EN{bits})
\item w --- word (16 \RU{бит}\EN{bits})
\item b --- byte (8 \RU{бит}\EN{bits})
\end{itemize}

% TODO1 simple example may be? \RU{Например mov\textbf{l}, movb, movw представляют различые версии инсструкция mov} \EN {For example: movl, movb, movw are variations of the mov instruciton}

\end{itemize}

\RU{Возвращаясь к результату компиляции: он идентичен тому, который мы посмотрели в \IDA.}
\EN{Let's go back to the compiled result: it is identical to what we saw in \IDA.}
\RU{Одна мелочь}\EN{With one subtle difference}: \TT{0FFFFFFF0h} \RU{записывается как}\EN{is presented as} \TT{\$-16}.
\RU{Это то же самое}\EN{It is the same thing}: \TT{16} \RU{в десятичной системе это}\EN{in the decimal system is} \TT{0x10} 
\RU{в шестнадцатеричной}\EN{in hexadecimal}. 
\TT{-0x10} \RU{будет как раз}\EN{is equal to} \TT{0xFFFFFFF0} 
(\RU{в рамках 32-битных чисел}\EN{for a 32-bit data type}).

\index{x86!\Instructions!MOV}
\RU{Ещё: возвращаемый результат устанавливается в 0 обычной инструкцией \MOV, а не \XOR}%
\EN{One more thing: the return value is to be set to 0 by using usual \MOV, not \XOR}.
\MOV \RU{просто загружает значение в регистр}\EN{just loads value to a register}. 
\RU{Её название не очень удачное (данные не перемещаются, а копируются). 
В других архитектурах подобная инструкция обычно носит название 
``LOAD'' или ``STORE'' или что-то в этом роде.}
\EN{Its name is a misnomer (data is not moved but rather copied).
In other architectures, this instruction is named ``LOAD'' or ``STORE'' or something similar.}

\fi

\section{x86-64}
\subsection{MSVC\EMDASH{}x86-64}

\index{x86-64}
\RU{Попробуем также 64-битный MSVC}\EN{Let's also try 64-bit MSVC}:

\lstinputlisting[caption=MSVC 2012 x64]{patterns/01_helloworld/MSVC_x64.asm}

\RU{В x86-64 все регистры были расширены до 64-х бит и теперь имеют префикс \TT{R-}}%
\EN{In x86-64, all registers were extended to 64-bit and now their names have an \TT{R-} prefix}.
\index{fastcall}
\RU{Чтобы поменьше задействовать стек (иными словами, поменьше обращаться кэшу и внешней памяти), уже давно имелся
довольно популярный метод передачи аргументов функции через регистры}
\EN{In order to use the stack less often (in other words, to access external memory/cache less often), there exists
a popular way to pass function arguments via registers} 
(fastcall%
\ifx\LITE\undefined%
: \myref{fastcall}%
\fi
).
\RU{Т.е. часть аргументов функции передается через регистры и часть}\EN{I.e., a part
of the function arguments are passed in registers, the rest}\EMDASH{}\RU{через стек}\EN{via the stack}.
\RU{В Win64 первые 4 аргумента функции передаются через регистры}\EN{In Win64, 4 function arguments
are passed in} \RCX, \RDX, \Reg{8}, \Reg{9}\EN{ registers}.
\RU{Это мы здесь и видим: указатель на строку в \printf теперь передается не через стек, а через регистр \RCX}%
\EN{That is what we see here: a pointer to the string for \printf is now passed not in stack, but in the \RCX register}.

\RU{Указатели теперь 64-битные, так что они передаются через 64-битные части регистров (имеющие префикс \TT{R-})}%
\EN{The pointers are 64-bit now, so they are passed in the 64-bit registers (which have the \TT{R-} prefix)}.
\RU{Но для обратной совместимости можно обращаться и к нижним 32 битам регистров используя префикс \TT{E-}}%
\EN{However, for backward compatibility, it is still possible to access the 32-bit parts, using the \TT{E-} prefix}.

\RU{Вот как выглядит регистр}\EN{This is how} \RAX/\EAX/\AX/\AL 
\RU{в 64-битных x86-совместимых \ac{CPU}}\EN{looks like in 64-bit x86-compatible \ac{CPU}s}:

\RegTableOne{RAX}{EAX}{AX}{AH}{AL}

\RU{Функция \main возвращает значение типа \Tint, который в \CCpp, вероятно для лучшей совместимости и переносимости,
оставили 32-битным. Вот почему в конце функции \main обнуляется не \RAX, а \EAX, т.е. 32-битная часть регистра.}
\EN{The \main function returns an \Tint{}-typed value, which is, in the \CCpp, for better backward compatibility
and portability, still 32-bit, so that is why the \EAX register is cleared at the function end (i.e., 32-bit
part of register) instead of \RAX{}.}

\RU{Также видно, что 40 байт выделяются в локальном стеке}\EN{There are also 40 bytes allocated in the local stack}.
\RU{Это}\EN{This is called} ``shadow space'', 
\RU{которое мы будем рассматривать позже}%
\EN{about which we are going to talk later}: \myref{shadow_space}.

\ifdefined\IncludeGCC
\subsection{GCC\EMDASH{}x86-64}

\index{x86-64}
\RU{Попробуем GCC в 64-битном Linux}\EN{Let's also try GCC in 64-bit Linux}:

\lstinputlisting[caption=GCC 4.4.6 x64]{patterns/01_helloworld/GCC_x64.s.\LANG}

\RU{В Linux, *BSD и \MacOSX для x86-64 также принят способ передачи аргументов функции через регистры}
\EN{A method to pass function arguments in registers is also used in Linux, *BSD and \MacOSX}\cite{SysVABI}.
\RU{6 первых аргументов передаются через регистры}\EN{The first 6 arguments are passed in the}
\RDI, \RSI, \RDX, \RCX, \Reg{8}, \Reg{9}\RU{, а остальные}\EN{ registers, and the rest}\EMDASH{}\RU{через стек}\EN{via
the stack}.

\RU{Так что указатель на строку передается через \EDI (32-битную часть регистра)}\EN{So the pointer to the
string is passed in \EDI (32-bit part of register)}.
\RU{Но почему не через 64-битную часть}\EN{But why not use the 64-bit part}, \RDI?

\RU{Важно запомнить что в 64-битном режиме все инструкции \MOV, записывающие что-либо в 
младшую 32-битную часть регистра,
обнуляют старшие 32-бита}\EN{It is important to keep in mind that all \MOV instructions in 64-bit mode
that write something into the lower 32-bit register part, also clear the higher 32-bits}\cite{Intel}.
\RU{То есть, инструкция}\EN{I.e., the} \TT{MOV EAX, 011223344h} \RU{корректно запишет это значение в \RAX, 
старшие биты сбросятся в ноль}\EN{writes a value into \RAX correctly, since the higher bits will be cleared}.

\RU{Если посмотреть в \IDA скомпилированный объектный файл (.o), увидим также опкоды всех инструкций}%
\EN{If we open the compiled object file (.o), we can also see all instruction's opcodes}%
\footnote{\RU{Это нужно задать в}\EN{This must be enabled in} 
Options $\rightarrow$ Disassembly $\rightarrow$ Number of opcode bytes}:

\lstinputlisting[caption=GCC 4.4.6 x64]{patterns/01_helloworld/GCC_x64.lst}

\label{hw_EDI_instead_of_RDI}
\RU{Как видно, инструкция, записывающая в \EDI по адресу \TT{0x4004D4}, занимает 5 байт}%
\EN{As we can see, the instruction that writes into \EDI at \TT{0x4004D4} occupies 5 bytes}.
\RU{Та же инструкция, записывающая 64-битное значение в \RDI, занимает 7 байт.}
\EN{The same instruction writing a 64-bit value into \RDI occupies 7 bytes.}
\RU{Возможно, GCC решил немного сэкономить}%
\EN{Apparently, GCC is trying to save some space}. 
\RU{К тому же, вероятно, он уверен, что сегмент данных, где хранится строка,
никогда не будет расположен в адресах выше 4\gls{GiB}.}
\EN{Besides, it can be sure that the data segment containing
the string will not be allocated at the addresses higher than 4\gls{GiB}.}

\label{SysVABI_input_EAX}
\RU{Здесь мы также видим обнуление регистра \EAX перед вызовом \printf}\EN{We also see that the \EAX register
was cleared before the \printf function call}.
\RU{Это делается потому что по стандарту передачи аргументов в *NIX для x86-64 
в \EAX передается количество задействованных векторных регистров}\EN{This is done because the number of
used vector registers is passed in \EAX by standard}:
``with variable arguments passes information about the number of vector registers used'' \cite{SysVABI}.

\fi

\ifdefined\IncludeGCC
\section{GCC\EMDASH{}\EN{one more thing}\RU{еще кое-что}}
\label{use_parts_of_C_strings}

\RU{Тот факт, что \IT{анонимная} Си-строка имеет тип}\EN{The fact that an \IT{anonymous} C-string has} 
\IT{const}\EN{ type} (\ref{string_is_const_char}), 
\RU{и тот факт, что выделенные в сегменте констант Си-строки гаратировано неизменяемые (immutable), 
ведет к интересному следствию}\EN{and the 
fact C-strings allocated in constants segment are guaranteed to be immutable, has an interesting consequence}:
\RU{компилятор может использовать определенную часть строки}\EN{the compiler may use a specific part of string}.

\RU{Вот простой пример}\EN{Let's try this example}:

\begin{lstlisting}
#include <stdio.h>

int f1()
{
	printf ("world\n");
};

int f2()
{
	printf ("hello world\n");
};

int main()
{
	f1();
	f2();
};
\end{lstlisting}

\RU{Среднестатистический компилятор с \CCpp (включая MSVC) выделит место для двух строк, но вот что делает 
GCC 4.8.1}\EN{Common \CCpp{}-compilers (including MSVC) will allocate two strings, but let's see what 
GCC 4.8.1 does}:

\begin{lstlisting}[caption=GCC 4.8.1 + \RU{листинг в }IDA\EN{ listing}]
f1              proc near

s               = dword ptr -1Ch

                sub     esp, 1Ch
                mov     [esp+1Ch+s], offset s ; "world"
                call    _puts
                add     esp, 1Ch
                retn
f1              endp

f2              proc near

s               = dword ptr -1Ch

                sub     esp, 1Ch
                mov     [esp+1Ch+s], offset aHello ; "hello "
                call    _puts
                add     esp, 1Ch
                retn
f2              endp

aHello          db 'hello '
s               db 'world',0
\end{lstlisting}

\RU{Действительно: когда мы выводим строку}\EN{Indeed: when we print the ``hello world'' string}, 
\RU{эти два слова расположены в памяти в притык друг к другу и \puts, вызываясь из ф-ции f2(), вообще не знает
что эти строки разделены}\EN{these two words are positioned in memory adjacently and \puts called from f2() 
function is not aware this string is divided}. \RU{Они и не разделены на самом деле, они разделены
только ``виртуально'', в нашем листинге}\EN{It's not divided in fact, it's divided only ``virtually'', in this
listing}.

\RU{Когда}\EN{When} \puts \RU{вызывается из f1(), он использует строку}\EN{is called from f1(), it uses} 
``world'' \RU{плюс нулевой байт}\EN{string plus zero byte}. \puts \RU{не знает что там еще есть какая-то строка
перед этой}\EN{is not aware there is something before this string}!

\RU{Этот трюк часто используется по крайней мере в GCC и может сэкономить немного памяти.}
\EN{This clever trick is often used by at least GCC and can save some memory.}

\fi
\ifdefined\IncludeARM
\section{ARM}
\label{sec:hw_ARM}

\index{\idevices}
\index{Raspberry Pi}
\index{Xcode}
\index{LLVM}
\index{Keil}
\RU{Для экспериментов с процессором ARM я использовал несколько компиляторов:}
\EN{For my experiments with ARM processors I used several compilers:} 

\begin{itemize}
\item \RU{Популярный в embedded-среде}\EN{Popular in the embedded area} Keil Release 6/2013.

\item Apple Xcode 4.6.3 \EN{IDE} (\RU{с компилятором}\EN{with} LLVM-GCC 4.2 \EN{compiler}
\footnote{\EN{It is indeed so: Apple Xcode 4.6.3 uses open-source GCC as front-end compiler and LLVM 
code generator}\RU{Это действительно так: Apple Xcode 4.6.3 использует опен-сорсный GCC как компилятор
переднего плана и коде-генератор LLVM}}.

\item GCC 4.8.1 (Linaro) (\RU{для}\EN{for} ARM64).

\item GCC 4.9 (Linaro) (\RU{для}\EN{for} ARM64), 
\RU{доступный как исполняемые файлы для win32 на}\EN{available as win32-executables at} 
\url{http://www.linaro.org/projects/armv8/}.

\end{itemize}

\RU{Везде в этой книге, кроме как если указано иное, идет речь о 32-битном ARM.}
\EN{32-bit ARM code is used in all cases in this book, if not mentioned otherwise.}

\RU{Когда речь идет о 64-битном ARM, он называется здесь ARM64.}
\EN{If we talk about 64-bit ARM here, it will be called ARM64.}

% subsections
\subsection{\NonOptimizingKeilVI (\ARMMode)}

\RU{Для начала скомпилируем наш пример в Keil}\EN{Let's start by compiling our example in Keil}:

\begin{lstlisting}
armcc.exe --arm --c90 -O0 1.c 
\end{lstlisting}

\index{\IntelSyntax}
\RU{Компилятор \IT{armcc} генерирует листинг на ассемблере в формате Intel.}
\EN{The \IT{armcc} compiler produces assembly listings in Intel-syntax} 
\RU{Этот листинг содержит некоторые высокоуровневые макросы, связанные с ARM}%
\EN{but it has high-level ARM-processor related macros}\footnote{
\RU{например, он показывает инструкции \PUSH/\POP, отсутствующие в режиме ARM}
\EN{e.g. ARM mode lacks \PUSH/\POP instructions}}, 
\RU{а нам важнее увидеть инструкции \q{как есть}, так что посмотрим скомпилированный результат в \IDA.}
\EN{but it is more important for us to see the instructions \q{as is} so let's see the compiled result in \IDA.}

\begin{lstlisting}[caption=\NonOptimizingKeilVI (\ARMMode) \IDA]
.text:00000000             main
.text:00000000 10 40 2D E9    STMFD   SP!, {R4,LR}
.text:00000004 1E 0E 8F E2    ADR     R0, aHelloWorld ; "hello, world"
.text:00000008 15 19 00 EB    BL      __2printf
.text:0000000C 00 00 A0 E3    MOV     R0, #0
.text:00000010 10 80 BD E8    LDMFD   SP!, {R4,PC}

.text:000001EC 68 65 6C 6C+aHelloWorld  DCB "hello, world",0    ; DATA XREF: main+4
\end{lstlisting}

\RU{В вышеприведённом примере можно легко увидеть, что каждая инструкция имеет размер 4 байта.}
\EN{In the example, we can easily see each instruction has a size of 4 bytes.}
\RU{Действительно, ведь мы же компилировали наш код для режима ARM, а не Thumb.}
\EN{Indeed, we compiled our code for ARM mode, not for Thumb.}

\index{ARM!\Instructions!STMFD}
\index{ARM!\Instructions!POP}
\RU{Самая первая инструкция}\EN{The very first instruction}, \TT{STMFD SP!, \{R4,LR\}}\footnote{\ac{STMFD}}, 
\RU{работает как инструкция}\EN{works as an x86} \PUSH \RU{в x86}\EN{instruction},
\RU{записывая значения двух регистров}\EN{writing the values of two registers}
(\Reg{4} \AndENRU \ac{LR}) \RU{в стек}\EN{into the stack}.
\RU{Действительно, в выдаваемом листинге на ассемблере компилятор \IT{armcc} для упрощения указывает здесь инструкцию}
\EN{Indeed, in the output listing from the \IT{armcc} compiler, for the sake of simplification, 
actually shows the} \TT{PUSH \{r4,lr\}}\EN{ instruction}.
\RU{Но это не совсем точно, инструкция \PUSH доступна только в режиме Thumb, поэтому,
во избежание путаницы, я предложил работать в \IDA}%
\EN{But that is not quite precise. The \PUSH instruction is only available in Thumb mode.
So, to make things less confusing, we're doing this in \IDA}.

\RU{Итак, эта инструкция уменьшает \ac{SP}, чтобы он указывал на место в стеке, свободное для записи
новых значений, затем записывает значения регистров \Reg{4} и \ac{LR} 
по адресу в памяти, на который указывает измененный регистр \ac{SP}}%
\EN{This instruction \glspl{decrement} first the \ac{SP} so it points to the place in the stack
that is free for new entries, then it saves the values of the \Reg{4} and \ac{LR} registers at the address
stored in the modified \ac{SP}}.

\RU{Эта инструкция, как и инструкция \PUSH в режиме Thumb, может сохранить в стеке одновременно несколько значений регистров, что может быть очень удобно}%
\EN{This instruction (like the \PUSH instruction in Thumb mode) is able to save several register values at once which can be very useful}. 
\RU{Кстати, такого в x86 нет}\EN{By the way, this has no equivalent in x86}.
\RU{Также следует заметить, что \TT{STMFD}~--- генерализация инструкции \PUSH (то есть расширяет её возможности), потому что может работать с любым регистром, а не только с \ac{SP}.}
\EN{It can also be noted that the \TT{STMFD} instruction is a generalization 
of the \PUSH instruction (extending its features), since it can work with any register, not just with \ac{SP}.}
\RU{Другими словами, \TT{STMFD} можно использовать для записи набора регистров в указанном месте памяти.}
\EN{In other words, \TT{STMFD} may be used for storing pack of registers at the specified memory address.}

\index{\PICcode}
\index{ARM!\Instructions!ADR}
\RU{Инструкция}\EN{The} \TT{ADR R0, aHelloWorld}
\RU{прибавляет или отнимает значение регистра \ac{PC} к смещению, где хранится строка}
\EN{instruction adds or subtracts the value in the \ac{PC} register to the offset where the}
\TT{hello, world}\EN{ string is located}.
\RU{Причем здесь \ac{PC}, можно спросить}\EN{How is the \TT{PC} register used here, one might ask}?
\RU{Притом, что это так называемый \q{\PICcode}}\EN{This is so-called \q{\PICcode}.}
\footnote{
	\RU{Читайте больше об этом в соответствующем разделе}
	\EN{Read more about it in relevant section}~(\myref{sec:PIC})
	}
\RU{он предназначен для исполнения будучи не привязанным к каким-либо адресам в памяти}%
\EN{Such code can be be executed at a non-fixed address in memory}.
\EN{In other words, this is \ac{PC}-relative addressing.}
\RU{Другими словами, это относительная от \ac{PC} адресация.}
\RU{В опкоде инструкции \TT{ADR} указывается разница между адресом этой инструкции и местом, где хранится строка}%
\EN{The \TT{ADR} instruction takes into account the difference between the address of this instruction and the address where the string is located}.
\RU{Эта разница всегда будет постоянной, вне зависимости от того, куда был загружен \ac{OS} наш код}%
\EN{This difference (offset) is always to be the same, no matter at what address our code is loaded by the \ac{OS}}.
\RU{Поэтому всё, что нужно~--- это прибавить адрес текущей инструкции (из \ac{PC}), чтобы получить текущий абсолютный адрес нашей Си-строки}%
\EN{That's why all we need is to add the address of the current instruction (from \ac{PC}) in order to get the absolute memory address of our C-string}.

\index{ARM!\Registers!Link Register}
\index{ARM!\Instructions!BL}
\RU{Инструкция} \TT{BL \_\_2printf}\footnote{Branch with Link}
\RU{вызывает функцию \printf}\EN{instruction calls the \printf function}. 
\RU{Работа этой инструкции состоит из двух фаз}%
\EN{Here's how this instruction works}: 
\begin{itemize}
\item
\RU{записать адрес после инструкции \TT{BL} (\TT{0xC}) в регистр \ac{LR}}%
\EN{store the address following the \TT{BL} instruction (\TT{0xC}) into the \ac{LR}};
\item
\RU{передать управление в \printf, записав адрес этой функции в регистр \ac{PC}}%
\EN{then pass the control to the \printf by writing its address into the \ac{PC} register}.
\end{itemize}

\RU{Ведь когда функция \printf закончит работу, нужно знать, куда вернуть управление, поэтому закончив работу, всякая функция передает управление по адресу, записанному в регистре \ac{LR}}%
\EN{When \printf finishes its execution it must have information about where it needs to return the control to.
That's why each function passes control to the address stored in the \ac{LR} register}.

\RU{В этом разница между \q{чистыми} \ac{RISC}-процессорами вроде ARM и \ac{CISC}-процессорами как x86,
где адрес возврата обычно записывается в стек}%
\EN{That is a difference between \q{pure} \ac{RISC}-processors like ARM and \ac{CISC}-processors like x86,
where the return address is usually stored on the stack}\footnote{\RU{Подробнее об этом будет описано в следующей главе}\EN{Read more about this in next section}~(\myref{sec:stack})}.

\RU{Кстати, 32-битный абсолютный адрес (либо смещение) невозможно закодировать в 32-битной инструкции \TT{BL}, в ней есть место только для 24-х бит}%
\EN{By the way, an absolute 32-bit address or offset cannot be encoded in the 32-bit \TT{BL} instruction because
it only has space for 24 bits}.
\RU{Поскольку все инструкции в режиме ARM имеют длину 4 байта (32 бита) и инструкции могут находится только по адресам кратным 4, то последние 2 бита (всегда нулевых) можно не кодировать.}
\EN{As we may remember, all ARM-mode instructions have a size of 4 bytes (32 bits).
Hence, they can only be located on 4-byte boundary addresses.
This implies that the last 2 bits of the instruction address (which are always zero bits) may be omitted.}
\RU{В итоге имеем 26 бит, при помощи которых можно закодировать}
\EN{In summary, we have 26 bits for offset encoding. This is enough to encode} $current\_PC \pm{} \approx{}32M$.

\index{ARM!\Instructions!MOV}
\RU{Следующая инструкция}\EN{Next, the} \TT{MOV R0, \#0}\footnote{MOVe}
\RU{просто записывает 0 в регистр \Reg{0}}\EN{instruction just writes $0$ into the \Reg{0} register}.
\RU{Ведь наша Си-функция возвращает 0, а возвращаемое значение всякая функция оставляет в \Reg{0}}%
\EN{That's because our C-function returns 0 and the return value is to be placed in the \Reg{0} register}.

\index{ARM!\Registers!Link Register}
\index{ARM!\Instructions!LDMFD}
\index{ARM!\Instructions!POP}
\RU{Последняя инструкция}\EN{The last instruction} \TT{LDMFD SP!, {R4,PC}}\footnote{\ac{LDMFD}}\RU{~--- это инструкция, обратная}\EN{ is an inverse instruction of} \TT{STMFD}. 
\RU{Она загружает из стека (или любого другого места в памяти) значения для сохранения их в \Reg{4} и \ac{PC}, увеличивая \glslink{stack pointer}{указатель стека} \ac{SP}.}
\EN{It loads values from the stack (or any other memory place) in order to save them into \Reg{4} and \ac{PC}, and \glslink{increment}{increments} the \gls{stack pointer} \ac{SP}.}
\RU{Здесь она работает как аналог \POP}\EN{It works like \POP here}.\\
N.B. \RU{Самая первая инструкция \TT{STMFD} сохранила в стеке \Reg{4} и \ac{LR}, а \IT{восстанавливаются} во время исполнения \TT{LDMFD} регистры \Reg{4} и \ac{PC}}%
\EN{The very first instruction \TT{STMFD} saved the \Reg{4} and \ac{LR} registers pair on the stack, but \Reg{4} and \ac{PC} are \IT{restored} during the \TT{LDMFD} execution}.

\RU{Как я уже описывал, в регистре \ac{LR} обычно сохраняется адрес места, куда нужно всякой функции вернуть управление}%
\EN{As I mentioned before, the address of the place where each function must return control to is usually saved in the \ac{LR} register}.
\RU{Самая первая инструкция сохраняет это значение в стеке, потому что наша функция \main позже будет сама пользоваться этим регистром в момент вызова \printf}%
\EN{The very first instruction saves its value in the stack because the same register will be used by our
\main function when calling \printf}.
\RU{А затем, в конце функции, это значение можно сразу записать прямо в \ac{PC}, таким образом, передав управление туда, откуда была вызвана наша функция}%
\EN{In the function's end, this value can be written directly to the \ac{PC} register, thus passing control to where our function was called}.
\RU{Так как функция \main обычно самая главная в \CCpp, управление будет возвращено в загрузчик \ac{OS}, либо куда-то в \ac{CRT} 
или что-то в этом роде.}
\EN{Since \main is usually the primary function in \CCpp,
the control will be returned to the \ac{OS} loader or to a point in a \ac{CRT},
or something like that.}

\index{ARM!DCB}
\TT{DCB}\RU{~--- директива ассемблера, описывающая массивы байт или ASCII-строк, аналог директивы DB в 
x86-ассемблере}%
\EN{~is an assembly language directive defining an array of bytes or ASCII strings, akin to the DB directive 
in x86-assembly language}.

\subsection{\NonOptimizingKeilVI (\ThumbMode)}

\RU{Скомпилируем тот же пример в Keil для режима thumb}\EN{Let's compile the same example using Keil in thumb mode}:

\begin{lstlisting}
armcc.exe --thumb --c90 -O0 1.c 
\end{lstlisting}

\RU{Получим (в \IDA)}\EN{We are getting (in \IDA)}:

\begin{lstlisting}[caption=\NonOptimizingKeilVI (\ThumbMode) + \IDA]
.text:00000000             main
.text:00000000 10 B5          PUSH    {R4,LR}
.text:00000002 C0 A0          ADR     R0, aHelloWorld ; "hello, world"
.text:00000004 06 F0 2E F9    BL      __2printf
.text:00000008 00 20          MOVS    R0, #0
.text:0000000A 10 BD          POP     {R4,PC}

.text:00000304 68 65 6C 6C+aHelloWorld  DCB "hello, world",0    ; DATA XREF: main+2
\end{lstlisting}

\RU{Сразу бросаются в глаза двухбайтные (16-битные) опкоды ~--- это, как я уже упоминал, thumb}
\EN{We can easily spot the 2-byte (16-bit) opcodes. This is, as I mentioned, thumb}.
\index{ARM!\Instructions!BL}
\RU{Кроме инструкции \TT{BL}.}\EN{The \TT{BL} instruction, however, }
\RU{Но на самом деле она состоит из двух 16-битных инструкций}
\EN{consists of two 16-bit instructions}.
\RU{Это потому, что загрузить в \ac{PC} смещение, по которому находится функция \printf, используя так мало места в одном 16-битном опкоде, нельзя}
\EN{This is because it is impossible to load an offset for the \printf function into \ac{PC} while using the small space in one 16-bit opcode}.
\RU{Так что первая 16-битная инструкция загружает старшие 10 бит смещения, а вторая ~--- младшие 11 бит смещения}
\EN{Therefore, the first 16-bit instruction loads the higher 10 bits of the offset and the second instruction loads 
the lower 11 bits of the offset}.
\RU{Как я уже упоминал, все инструкции в thumb-режиме имеют длину 2 байта (или 16 бит)}
\EN{As I mentioned, all instructions in thumb mode have a size of 2 bytes (or 16 bits)}.
\RU{Поэтому невозможна такая ситуация, когда thumb-инструкция начинается по нечетному адресу.}
\EN{This implies it is impossible for a thumb-instruction to be at an odd address whatsoever.}
\RU{Учитывая сказанное, последний бит адреса можно не кодировать}
\EN{Given the above, the last address bit may be omitted while encoding instructions}.
\RU{Таким образом, в thumb-инструкции \TT{BL} можно закодировать адрес}
\EN{In summary, \TT{BL} thumb-instruction can encode the address} $current\_PC \pm{}\approx{}2M$.

\index{ARM!\Instructions!PUSH}
\index{ARM!\Instructions!POP}
\RU{Остальные инструкции в функции: \PUSH и \POP здесь работают почти так же, как и описанные \TT{STMFD}/\TT{LDMFD}, только регистр \ac{SP} здесь не указывается явно}
\EN{As for the other instructions in the function: \PUSH and \POP work here just like the described \TT{STMFD}/\TT{LDMFD} only the \ac{SP} register is not mentioned explicitly here}.
\TT{ADR} \RU{работает так же, как и в предыдущем примере}\EN{works just like in the previous example}.
\TT{MOVS} \RU{записывает $0$ в регистр \Reg{0} для возврата нуля}
\EN{writes $0$ into the \Reg{0} register in order to return zero}.

\subsection{\OptimizingXcodeIV (\ARMMode)}

Xcode 4.6.3 \RU{без включенной оптимизации выдает слишком много лишнего кода, поэтому включим оптимизацию компилятора (ключ \Othree), потому что там меньше инструкций.}
\EN{without optimization turned on produces a lot of redundant code so we'll study optimized output, where the instruction count is as small as possible, setting the compiler switch \Othree.}

\begin{lstlisting}[caption=\OptimizingXcodeIV (\ARMMode)]
__text:000028C4             _hello_world
__text:000028C4 80 40 2D E9   STMFD           SP!, {R7,LR}
__text:000028C8 86 06 01 E3   MOV             R0, #0x1686
__text:000028CC 0D 70 A0 E1   MOV             R7, SP
__text:000028D0 00 00 40 E3   MOVT            R0, #0
__text:000028D4 00 00 8F E0   ADD             R0, PC, R0
__text:000028D8 C3 05 00 EB   BL              _puts
__text:000028DC 00 00 A0 E3   MOV             R0, #0
__text:000028E0 80 80 BD E8   LDMFD           SP!, {R7,PC}

__cstring:00003F62 48 65 6C 6C+aHelloWorld_0  DCB "Hello world!",0
\end{lstlisting}

\RU{Инструкции}\EN{The instructions} \TT{STMFD} \AndENRU \TT{LDMFD} \RU{нам уже знакомы}\EN{are already familiar to us}.

\index{ARM!\Instructions!MOV}
\RU{Инструкция \MOV просто записывает число \TT{0x1686} в регистр \Reg{0} ~--- это смещение, указывающее на строку ``Hello world!''}
\EN{The \MOV instruction just writes the number \TT{0x1686} into the \Reg{0} register.
This is the offset pointing to the ``Hello world!'' string}.

\RU{Регистр \TT{R7} (по стандарту, принятому в \cite{IOSABI}) это frame pointer, о нем будет рассказано позже.}
\EN{The \TT{R7} register (as it is standardized in \cite{IOSABI}) is a frame pointer. More on that below.}

\index{ARM!\Instructions!MOVT}
\RU{Инструкция}\EN{The} \TT{MOVT R0, \#0} (MOVe Top) \RU{записывает $0$ в старшие 16 бит регистра}
\EN{instruction writes $0$ into higher 16 bits of the register}.
\RU{Дело в том, что обычная инструкция \MOV в режиме ARM может записывать какое-либо значение только в младшие 16 бит регистра, ведь в ней нельзя закодировать больше}
\EN{The issue here is that the generic \MOV instruction in ARM mode may write only the lower 16 bits of the register}.
\RU{Помните, что в режиме ARM опкоды всех инструкций ограничены длиной в 32 бита. Конечно, это ограничение не касается перемещений данных между регистрами.}
\EN{Remember, all instruction opcodes in ARM mode are limited in size to 32 bits. Of course, this limitation is not related to moving data between registers.}
\RU{Поэтому для записи в старшие биты (от 16-го по 31-го включительно) существует дополнительная команда \TT{MOVT}}
\EN{That's why an additional instruction \TT{MOVT} exists for writing into the higher bits (from 16 to 31 inclusive)}.
\RU{Впрочем, здесь её использование избыточно, потому что инструкция \TT{``MOV R0, \#0x1686''} выше и так обнулила старшую часть регистра}
\EN{Its usage here, however, is redundant because the \TT{``MOV R0, \#0x1686''} instruction above cleared the higher part of the register}.
\RU{Возможно, это недочет компилятора}\EN{This is probably a shortcoming of the compiler}.

\index{ARM!\Instructions!ADD}
\RU{Инструкция}\EN{The} \TT{``ADD R0, PC, R0''} \RU{прибавляет \ac{PC} к \Reg{0} для вычисления действительного адреса строки ``Hello world!''. Как нам уже известно, это ``\PICcode'', поэтому такая корректива необходима}
\EN{instruction adds the value in the \ac{PC} to the value in the \Reg{0}, to calculate absolute address of the ``Hello world!'' string. 
As we already know, it is ``\PICcode'' so this correction is essential here}.

\RU{Инструкция \TT{BL} вызывает \puts вместо \printf}
\EN{The \TT{BL} instruction calls the \puts function instead of \printf}.

\label{puts}
\index{\CStandardLibrary!puts()}
\index{puts() \RU{вместо}\EN{instead of} printf()}
\RU{Компилятор заменил вызов \printf на \puts. 
Действительно, \printf с одним аргументом это почти аналог \puts.}
\EN{GCC replaced the first \printf call with \puts.
Indeed: \printf with a sole argument is almost analogous to \puts.} 

\RU{\IT{Почти}, если принять условие, что в строке не будет управляющих символов \printf, 
начинающихся со знака процента. Тогда эффект от работы этих двух функций будет разным}
\EN{\IT{Almost}, because the two functions are producing the same result only in case the 
string does not contain printf format identifiers starting with \IT{\%}. 
In case it does, the effect of these two functions would be different}
\footnote{
\RU{Также нужно заметить, что \puts не требует символа перевода строки '\textbackslash{}n' в конце строки,
поэтому его здесь нет.}
\EN{It has also to be noted the \puts does not require a '\textbackslash{}n' new line symbol 
at the end of a string, so we do not see it here.}}.

\RU{Зачем компилятор заменил один вызов на другой? Наверное потому что \puts работает быстрее}
\EN{Why did the compiler replace the \printf with \puts? Probably because \puts is faster}
\footnote{\url{http://go.yurichev.com/17063}}. 

\RU{Видимо потому, что \puts проталкивает символы в \gls{stdout} не сравнивая каждый со знаком процента.}
\EN{\puts works faster because it just passes characters to \gls{stdout} without comparing every one of them with the \IT{\%} symbol.}

\RU{Далее уже знакомая инструкция}\EN{Next, we see the familiar} 
\TT{``MOV R0, \#0''}\RU{, служащая для установки в $0$ возвращаемого значения функции}
\EN{instruction intended to set the \Reg{0} register to $0$}.

\subsection{\OptimizingXcodeIV (\ThumbTwoMode)}

\RU{По умолчанию}\EN{By default} Xcode 4.6.3 
\RU{генерирует код для режима Thumb-2 примерно в такой манере}%
\EN{generates code for Thumb-2 in this manner}:

\begin{lstlisting}[caption=\OptimizingXcodeIV (\ThumbTwoMode)]
__text:00002B6C                   _hello_world
__text:00002B6C 80 B5          PUSH            {R7,LR}
__text:00002B6E 41 F2 D8 30    MOVW            R0, #0x13D8
__text:00002B72 6F 46          MOV             R7, SP
__text:00002B74 C0 F2 00 00    MOVT.W          R0, #0
__text:00002B78 78 44          ADD             R0, PC
__text:00002B7A 01 F0 38 EA    BLX             _puts
__text:00002B7E 00 20          MOVS            R0, #0
__text:00002B80 80 BD          POP             {R7,PC}

...

__cstring:00003E70 48 65 6C 6C 6F 20+aHelloWorld  DCB "Hello world!",0xA,0
\end{lstlisting}

\index{\ThumbTwoMode}
\index{ARM!\Instructions!BL}
\index{ARM!\Instructions!BLX}
\RU{Инструкции \TT{BL} и \TT{BLX} в Thumb, как мы помним, кодируются как пара 16-битных инструкций, 
а в Thumb-2 эти \IT{суррогатные} опкоды расширены так, что новые инструкции кодируются здесь как 
32-битные инструкции}%
\EN{The \TT{BL} and \TT{BLX} instructions in Thumb mode, as we recall, are encoded as a pair
of 16-bit instructions.
In Thumb-2 these \IT{surrogate} opcodes are extended in such a way so that new instructions
may be encoded here as 32-bit instructions}.
\RU{Это можно заметить по тому что опкоды Thumb-2 инструкций всегда начинаются с \TT{0xFx} либо с \TT{0xEx}}%
\EN{That is obvious considering that the opcodes of the Thumb-2 instructions always begin with \TT{0xFx} or \TT{0xEx}}.
\RU{Но в листинге \IDA байты опкода переставлены местами.
Это из-за того, что в процессоре ARM инструкции кодируются так:
в начале последний байт, потом первый (для Thumb и Thumb-2 режима), либо, 
(для инструкций в режиме ARM) в начале четвертый байт, затем третий, второй и первый 
(т.е. другой \gls{endianness})}%
\EN{But in the \IDA listing
the opcode bytes are swapped because for ARM processor the instructions are encoded as follows: 
last byte comes first and after that comes the first one (for Thumb and Thumb-2 modes) 
or for instructions in ARM mode the fourth byte comes first, then the third,
then the second and finally the first (due to different \gls{endianness})}.

\RU{Вот так байты следуют в листингах IDA:}
\EN{So that is how bytes are located in IDA listings:}
\begin{itemize}
\item \RU{для режимов ARM и ARM64}\EN{for ARM and ARM64 modes}: 4-3-2-1;
\item \RU{для режима Thumb}\EN{for Thumb mode}: 2-1;
\item \RU{для пары 16-битных инструкций в режиме Thumb-2}\EN{for 16-bit instructions pair in Thumb-2 mode}: 2-1-4-3.
\end{itemize}

\index{ARM!\Instructions!MOVW}
\index{ARM!\Instructions!MOVT.W}
\index{ARM!\Instructions!BLX}
\RU{Так что мы видим здесь что инструкции \TT{MOVW}, \TT{MOVT.W} и \TT{BLX} начинаются с}
\EN{So as we can see, the \TT{MOVW}, \TT{MOVT.W} and \TT{BLX} instructions begin with} \TT{0xFx}.

\RU{Одна из Thumb-2 инструкций это}\EN{One of the Thumb-2 instructions is}
\TT{MOVW R0, \#0x13D8}\RU{~--- она записывает 16-битное число в младшую часть регистра \Reg{0}, очищая старшие биты.}
\EN{~---it stores a 16-bit value into the lower part of the \Reg{0} register, clearing the higher bits.}

\RU{Ещё}\EN{Also,} \TT{MOVT.W R0, \#0}\RU{~--- эта инструкция работает так же, как и}
\EN{~works just like} 
\TT{MOVT} \RU{из предыдущего примера, но она работает в}\EN{from the previous example only it works in} Thumb-2.

\index{ARM!\RU{переключение режимов}\EN{mode switching}}
\index{ARM!\Instructions!BLX}
\RU{Помимо прочих отличий, здесь используется инструкция}
\EN{Among the other differences, the} \TT{BLX} 
\RU{вместо}\EN{instruction is used in this case instead of the} \TT{BL}.
\RU{Отличие в том, что помимо сохранения адреса возврата в регистре \ac{LR} и передаче управления 
в функцию \puts, происходит смена режима процессора с Thumb/Thumb-2 на режим ARM (либо назад).}
\EN{The difference is that, besides saving the \ac{RA} in the \ac{LR} register and passing control 
to the \puts function, the processor is also switching from Thumb/Thumb-2 mode to ARM mode (or back).}
\RU{Здесь это нужно потому, что инструкция, куда ведет переход, выглядит так (она закодирована в режиме ARM)}%
\EN{This instruction is placed here since the instruction to which control is passed looks like (it is encoded in ARM mode)}:

\begin{lstlisting}
__symbolstub1:00003FEC _puts           ; CODE XREF: _hello_world+E
__symbolstub1:00003FEC 44 F0 9F E5     LDR  PC, =__imp__puts
\end{lstlisting}

\EN{This is essentially jump to place where \puts address is written in imports' section.}
\RU{Это просто переход на место, где записан адрес \puts в секции импортов.}

\RU{Итак, внимательный читатель может задать справедливый вопрос: почему бы не вызывать \puts сразу в 
том же месте кода, где он нужен?}
\EN{So, the observant reader may ask: why not call \puts right at the point in the code where it is needed?}

\RU{Но это не очень выгодно из-за экономии места и вот почему}%
\EN{Because it is not very space-efficient}.

\index{\RU{Динамически подгружаемые библиотеки}\EN{Dynamically loaded libraries}}
\RU{Практически любая программа использует внешние динамические библиотеки (будь то DLL в Windows, .so в *NIX 
либо .dylib в \MacOSX)}\EN{Almost any program uses external dynamic libraries (like DLL in Windows, .so in *NIX or .dylib in \MacOSX)}.
\RU{В динамических библиотеках находятся часто используемые библиотечные функции, в том числе стандартная функция Си \puts}%
\EN{The dynamic libraries contain frequently used library functions, including the standard C-function \puts}.

\index{Relocation}
\RU{В исполняемом бинарном файле}\EN{In an executable binary file} 
(Windows PE .exe, ELF \RU{либо}\EN{or} Mach-O) \RU{имеется секция импортов, список символов (функций либо глобальных переменных) импортируемых из внешних модулей, а также названия самих модулей}%
\EN{an import section is present.
This is a list of symbols (functions or global variables) imported from external modules along with the names of the modules themselves}.

\RU{Загрузчик \ac{OS} загружает необходимые модули и, перебирая импортируемые символы в основном модуле, проставляет правильные адреса каждого символа}%
\EN{The \ac{OS} loader loads all modules it needs and, while enumerating import symbols in the primary module, determines the correct addresses of each symbol}.

\RU{В нашем случае,}\EN{In our case,} \IT{\_\_imp\_\_puts} 
\RU{это 32-битная переменная, куда загрузчик \ac{OS} запишет правильный адрес этой же функции во внешней библиотеке}%
\EN{is a 32-bit variable used by the \ac{OS} loader to store the correct address of the function in an external library}. 
\RU{Так что инструкция \TT{LDR} просто берет 32-битное значение из этой переменной, и, записывая его в регистр \ac{PC}, просто передает туда управление}%
\EN{Then the \TT{LDR} instruction just reads the 32-bit value from this variable and writes it into the \ac{PC} register, passing control to it}.

\RU{Чтобы уменьшить время работы загрузчика \ac{OS}, 
нужно чтобы ему пришлось записать адрес каждого символа только один раз, 
в соответствующее, выделенное для них, место.}
\EN{So, in order to reduce the time the \ac{OS} loader needs for completing this procedure, 
it is good idea if it writes the address of each symbol only once, to a dedicated place.}

\index{thunk-\RU{функции}\EN{functions}}
\RU{К тому же, как мы уже убедились, нельзя одной инструкцией загрузить в регистр 32-битное число без обращений к памяти}%
\EN{Besides, as we have already figured out, it is impossible to load a 32-bit value into a register 
while using only one instruction without a memory access}.
\RU{Так что наиболее оптимально выделить отдельную функцию, работающую в режиме ARM, 
чья единственная цель~--- передавать управление дальше, в динамическую библиотеку.}
\EN{Therefore, the optimal solution is to allocate a separate function working in ARM mode with sole 
goal to pass control to the dynamic library}
\RU{И затем ссылаться на эту короткую функцию из одной инструкции (так называемую \glslink{thunk function}{thunk-функцию}) из Thumb-кода}%
\EN{and then to jump to this short one-instruction function (the so-called \gls{thunk function}) from the Thumb-code}.

\index{ARM!\Instructions!BL}
\RU{Кстати, в предыдущем примере (скомпилированном для режима ARM), переход при помощи инструкции \TT{BL} ведет 
на такую же \glslink{thunk function}{thunk-функцию}, однако режим процессора не переключается (отсюда отсутствие \q{X} в мнемонике инструкции)}%
\EN{By the way, in the previous example (compiled for ARM mode) the control is passed by the \TT{BL} to the 
same \gls{thunk function}.
The processor mode, however, is not being switched (hence the absence of an \q{X} in the instruction mnemonic)}.

\subsubsection{\EN{More about thunk-functions}\RU{Еще о thunk-функциях}}
\index{thunk-\RU{функции}\EN{functions}}

\RU{Thunk-функции трудновато понять, вероятно из-за путаницы в терминах.}
\EN{Thunk-functions are hard to understand, apparently because of misnomer.}
\RU{Вот ещё несколько описаний этих функций:}
\EN{Here are couple more descriptions of these functions:}

\begin{framed}
\begin{quotation}
“A piece of coding which provides an address:”, according to P. Z. Ingerman, 
who invented thunks in 1961 as a way of binding actual parameters to their formal 
definitions in Algol-60 procedure calls. If a procedure is called with an expression 
in the place of a formal parameter, the compiler generates a thunk which computes 
the expression and leaves the address of the result in some standard location.

\dots

Microsoft and IBM have both defined, in their Intel-based systems, a “16-bit environment” 
(with bletcherous segment registers and 64K address limits) and a “32-bit environment” 
(with flat addressing and semi-real memory management). The two environments can both be 
running on the same computer and OS (thanks to what is called, in the Microsoft world, 
WOW which stands for Windows On Windows). MS and IBM have both decided that the process 
of getting from 16- to 32-bit and vice versa is called a “thunk”; for Windows 95, 
there is even a tool THUNK.EXE called a “thunk compiler”.
\end{quotation}
\end{framed}
% TODO FIXME move to bibliography and quote properly above the quote
( \url{http://www.catb.org/jargon/html/T/thunk.html} )

\subsection{ARM64}

\subsubsection{GCC}

\RU{Компилируем пример в}\EN{Let's compile the example using} GCC 4.8.1 \InENRU ARM64:

\lstinputlisting[numbers=left,label=hw_ARM64_GCC,caption=\NonOptimizing GCC 4.8.1 + objdump]
{patterns/01_helloworld/ARM/hw.lst}

\RU{В ARM64 нет режима thumb и thumb-2, только ARM, так что тут только 32-битные инструкции.}
\EN{There are no thumb and thumb-2 modes in ARM64, only ARM, so there are 32-bit instructions only.}
\RU{Регистров тут в 2 раза больше}\EN{Registers count is doubled}: \myref{ARM64_GPRs}.
\RU{64-битные регистры теперь имеют префикс}\EN{64-bit registers has} 
\TT{X-}\EN{ prefixes, while its 32-bit parts}\RU{, а их 32-битные части}\EMDASH{}\TT{W-}.

\index{ARM!\Instructions!STP}
\EN{The }\RU{Инструкция }\TT{STP}\EN{ instruction} (\IT{Store Pair}) 
\RU{сохраняет в стеке сразу два регистра}\EN{saves two registers in the stack simultaneously}: \RegX{29} \InENRU \RegX{30}.
\RU{Конечно, эта инструкция может сохранять эту пару где угодно в памяти, но здесь указан регистр \ac{SP}, так что
пара сохраняется именно в стеке.}
\EN{Of course, this instruction is able to save this pair at a random place of memory, 
but the \ac{SP} register is specified here, so the pair is saved in the stack.}
\RU{Регистры в ARM64 64-битные, каждый имеет длину в 8 байт, так что для хранения двух регистров нужно именно 16 байт.}
\EN{ARM64 registers are 64-bit ones, each has a size of 8 bytes, so one needs 16 bytes for saving two registers.}

\RU{Восклицательный знак после операнда означает, что сначала от \ac{SP} будет отнято 16 и только затем
значения из пары регистров будут записаны в стек.}
\EN{Exclamation mark after operand mean that 16 is to be subtracted from \ac{SP} first, and only then
values from registers pair are to be written into the stack.}
\RU{Это называется}\EN{This is also called} \IT{pre-index}.
\RU{Больше о разнице между}\EN{About the difference between} \IT{post-index} \AndENRU \IT{pre-index} 
\RU{описано здесь}\EN{read here}: \myref{ARM_postindex_vs_preindex}.

\RU{Таким образом, в терминах более знакомого всем процессора x86, первая инструкция~--- это просто аналог 
пары инструкций}
\EN{Hence, in the terms of more familiar x86, the first instruction is just an analogue to pair of}
\TT{PUSH X29} \AndENRU \TT{PUSH X30}.
\RegX{29} \EN{is used as \ac{FP} in ARM64}\RU{в ARM64 используется как \ac{FP}}, \EN{and}\RU{а} \RegX{30} 
\EN{as}\RU{как} \ac{LR}, \RU{поэтому они сохраняются в прологе функции и
восстанавливаются в эпилоге}\EN{so that's why they are saved in the function prologue and restored in the function epilogue}.

\EN{The second instruction copies}\RU{Вторая инструкция копирует} \ac{SP} \InENRU \RegX{29} (\OrENRU \ac{FP}).
\RU{Это нужно для установки стекового фрейма функции}\EN{This is done to set up the function stack frame}.

\label{pointers_ADRP_and_ADD}
\index{ARM!\Instructions!ADRP/ADD pair}
\RU{Инструкции }\TT{ADRP} \AndENRU \ADD \EN{instructions are used to fill the 
string}\RU{нужны для формирования адреса строки} \q{Hello!} \EN{address into the \RegX{0} register}\RU{в регистре \RegX{0}}, 
\RU{ведь первый аргумент функции передается через этот регистр}\EN{because the first function argument is passed
in this register}.
\RU{Но в ARM нет инструкций, при помощи которых можно записать в регистр длинное число}\EN{There are
no instructions, whatsoever, in ARM that can store a large number into a register} 
(\RU{потому что сама длина инструкции ограничена 4-я байтами. Больше об этом здесь}\EN{because the instruction
length is limited to 4 bytes, read more about it here}: \myref{ARM_big_constants_loading}).
\RU{Так что нужно использовать несколько инструкций}\EN{So several instructions must be utilised}.
\RU{Первая инструкция}\EN{The first instruction} (\TT{ADRP}) \EN{writes address of 4Kb page where the string is
located into \RegX{0}}\RU{записывает в \RegX{0} адрес 4-килобайтной страницы где находится строка}, 
\EN{and the the second one}\RU{а вторая} (\ADD) \RU{просто прибавляет к этому адресу остаток}\EN{just adds
reminder to the address}.
\EN{More about that in}\RU{Читайте больше об этом}: \myref{ARM64_relocs}.

\TT{0x400000 + 0x648 = 0x400648}, \EN{and we see our \q{Hello!} C-string in the \TT{.rodata} data segment at this
address}\RU{и мы видим, что в секции данных \TT{.rodata} по этому адресу как раз находится наша
Си-строка \q{Hello!}}.

\index{ARM!\Instructions!BL}
\RU{Затем при помощи инструкции \TT{BL} вызывается \puts. Это уже рассматривалось ранее: \myref{puts}.}
\EN{\puts is called afterwards using \TT{BL} instruction. This was already discussed: \myref{puts}.}

\RU{Инструкция }\MOV \EN{instruction writes $0$ into}\RU{записывает $0$ в} \RegW{0}. 
\RegW{0} \RU{это младшие 32 бита 64-битного регистра}\EN{is low 32 bits of 64-bit} \RegX{0}\EN{ register}:

\begin{center}
\begin{tabular}{ | l | l | }
\hline
\RU{Старшие 32 бита}\EN{High 32-bit part}\ES{Parte alta de 32 bits}\PTBRph{}\PLph{}\ITAph{}\DEph{}\THAph{} & \RU{младшие 32 бита}\EN{low 32-bit part}\ES{parte baja de 32 bits}\PTBRph{}\PLph{}\ITAph{}\DEph{}\THAph{} \\
\hline
\multicolumn{2}{ | c | }{X0} \\
\hline
\multicolumn{1}{ | c | }{} & \multicolumn{1}{ c | }{W0} \\
\hline
\end{tabular}
\end{center}


\RU{А результат функции возвращается через \RegX{0}, и \main возвращает $0$, 
так что вот так готовится возвращаемый результат.}
\EN{The function result is returned via \RegX{0} and \main returns $0$, so that's how the return
result is prepared.}
\RU{Почему именно 32-битная часть}\EN{But why using the 32-bit part}?
\RU{Потому в ARM64, как и в x86-64, тип \Tint оставили 32-битным, для лучшей совместимости.}
\EN{Because \Tint data type in ARM64, just like in x86-64, is still 32-bit, for better compatibility.}
\RU{Следовательно, раз уж функция возвращает 32-битный \Tint, то нужно заполнить только 32 младших бита 
регистра \RegX{0}.}
\EN{So if a function returns 32-bit \Tint, only the low 32 bits of \RegX{0} register has to be filled.}

\RU{Для того, чтобы удостовериться в этом, я немного отредактировал свой пример и перекомпилировал его.}
\EN{In order to verify this, I changed my example slightly and recompiled it.}
\RU{Теперь}\EN{Now} \main \RU{возвращает 64-битное значение}\EN{returns 64-bit value}:

\begin{lstlisting}[caption=\main \RU{возвращающая значение типа}\EN{returning a value of} \TT{uint64\_t}\EN{ type}]
#include <stdio.h>
#include <stdint.h>

uint64_t main()
{
        printf ("Hello!\n");
        return 0;
}
\end{lstlisting}

\RU{Результат точно такой же, только \MOV в той строке теперь выглядит так:}
\EN{The result is the same, but that's how \MOV at that line looks like now:}

\begin{lstlisting}[caption=\NonOptimizing GCC 4.8.1 + objdump]
  4005a4:       d2800000        mov     x0, #0x0                        // #0
\end{lstlisting}

\index{ARM!\Instructions!LDP}
\RU{Далее при помощи инструкции \TT{LDP} (\IT{Load Pair}) восстанавливаются регистры \RegX{29} и \RegX{30}.}
\EN{\TT{LDP} (\IT{Load Pair}) then restores \RegX{29} and \RegX{30} registers.}
\RU{Восклицательного знака после инструкции нет. Это означает, что сначала значения достаются из стека,
и только потом \ac{SP} увеличивается на 16.}
\EN{There is no exclamation mark after the instruction: this implies that the value is first loaded from the stack,
only then \ac{SP} it is increased by 16.}
\RU{Это называется}\EN{This is called} \IT{post-index}.

\index{ARM!\Instructions!RET}
\RU{В ARM64 есть новая инструкция}\EN{New instruction appeared in ARM64}: \RET. 
\RU{Она работает так же как и}\EN{It works just as} \TT{BX LR}, \RU{но там добавлен специальный бит,
подсказывающий процессору, что это именно выход из функции, а не просто переход, чтобы процессор
мог более оптимально исполнять эту инструкцию}\EN{only a special \IT{hint} bit is added, informing the \ac{CPU}
that this is a return from a function, not just another jump instruction, so it can execute it more optimally}.

\RU{Из-за простоты этой функции оптимизирующий GCC генерирует точно такой же код.}
\EN{Due to the simplicity of the function, optimizing GCC generates the very same code.}


\fi
\ifdefined\IncludeMIPS
\section{MIPS}

\subsection{\RU{О \q{глобальном указателе} (\q{global pointer})}\EN{A word about \q{global pointer}}}
\label{MIPS_GP}

\index{MIPS!\GlobalPointer}
\RU{\q{Глобальный указатель} (\q{global pointer})~--- это важная концепция в MIPS.}
\EN{One important MIPS concept is \q{global pointer}.}
\RU{Как мы уже возможно знаем, каждая инструкция в MIPS имеет размер 32 бита, поэтому невозможно
закодировать 32-битный адрес внутри одной инструкции. Вместо этого нужно использовать пару инструкций
(как это сделал GCC для загрузки адреса текстовой строки в нашем примере).}
\EN{As we may already know, each MIPS instruction has size of 32 bits, so it's impossible to embed 32-bit
address into one instruction: a pair has to be used for this 
(like GCC did in our example for the text string address loading).}

\RU{С другой стороны, используя только одну инструкцию, 
возможно загружать данные по адресам в пределах $register-32768...register+32767$, потому что 16 бит
знакового смещения можно закодировать в одной инструкции).}
\EN{It's possible, however, to load data from the address in range of $register-32768...register+32767$ using one
single instruction (because 16 bits of signed offset could be encoded in single instruction).}
\RU{Так мы можем выделить какой-то регистр для этих целей и ещё выделить буфер в 64KiB для самых 
частоиспользуемых данных.}
\EN{So we can allocate some register for this purpose and also allocate 64KiB area of most used data.}
\RU{Выделенный регистр называется \q{глобальный указатель} (\q{global pointer}) и он указывает на середину
области 64KiB.}
\EN{This allocated register is called \q{global pointer} and it points to the middle of the 64KiB area.}
\RU{Эта область обычно содержит глобальные переменные и адреса импортированных функций вроде \printf,
потому что разработчики GCC решили, что получение адреса функции должно быть как можно более быстрой операцией,
исполняющейся за одну инструкцию вместо двух.}
\EN{This area usually contains global variables and addresses of imported functions like \printf, 
because GCC developers decided that getting address of some function must be as fast as single instruction
execution instead of two.}
\RU{В ELF-файле эта 64KiB-область находится частично в секции .sbss (\q{small \ac{BSS}}) для неинициализированных
данных и в секции .sdata (\q{small data}) для инициализированных данных.}
\EN{In an ELF file this 64KiB area is located partly in sections .sbss (\q{small \ac{BSS}}) for not initialized data and 
.sdata (\q{small data}) for initialized data.}

\RU{Это значит что программист может выбирать, к чему нужен как можно более быстрый доступ, и затем расположить
это в секциях .sdata/.sbss.}
\EN{This implies that the programmer may choose what data he/she wants to be accessed fast and place it into 
.sdata/.sbss.}

\RU{Некоторые программисты \q{старой школы} могут вспомнить модель памяти в MS-DOS \myref{8086_memory_model} 
или в менеджерах памяти вроде XMS/EMS, где вся память делилась на блоки по 64KiB.}
\EN{Some old-school programmers may recall the MS-DOS memory model \myref{8086_memory_model} 
or the MS-DOS memory managers like XMS/EMS where all memory was divided in 64KiB blocks.}

\index{PowerPC}
\RU{Эта концепция применяется не только в MIPS. По крайней мере PowerPC также использует эту технику.}
\EN{This concept is not unique to MIPS. At least PowerPC uses this technique as well.}

\subsection{\Optimizing GCC}

\EN{Lets consider the following example which illustrates \q{global pointer} concept.}
\RU{Рассмотрим следующий пример, иллюстрирующий концепцию \q{глобального указателя}.}

\lstinputlisting[caption=\Optimizing GCC 4.4.5 (\assemblyOutput),numbers=left]{patterns/01_helloworld/MIPS/hw_O3.s.\LANG}

\RU{Как видно, регистр \$GP в прологе функции выставляется в середину этой области.}
\EN{As we see, the \$GP register is set in the function prologue to point the middle of this area.}
\RU{Регистр \ac{RA} сохраняется в локальном стеке.}
\EN{The \ac{RA} register is also saved in the local stack.}
\RU{Здесь также используется \puts вместо \printf.}
\EN{\puts is also used here instead of \printf.}
\index{MIPS!\Instructions!LW}
\RU{Адрес функции \puts загружается в \$25 инструкцией LW (\q{Load Word}).}
\EN{The address of the \puts function is loaded into \$25 using LW the instruction (\q{Load Word}).}
\index{MIPS!\Instructions!LUI}
\index{MIPS!\Instructions!ADDIU}
\RU{Затем адрес текстовой строки загружается в \$4 парой инструкций LUI (\q{Load Upper Immediate}) и
ADDIU (\q{Add Immediate Unsigned Word}).}
\EN{Then the address of the text string is loaded to \$4 using LUI (\q{Load Upper Immediate}) and 
ADDIU (\q{Add Immediate Unsigned Word}) instruction pair.}
\RU{LUI устанавливает старшие 16 бит регистра (поэтому в имени инструкции присутствует \q{upper}) и ADDIU
прибавляет младшие 16 бит к адресу.}
\EN{LUI sets the high 16 bits of the register (hence \q{upper} word in instruction name) and ADDIU adds
the lower 16 bits of the address.}
\RU{ADDIU следует за JALR (помните о \IT{branch delay slots}?).}
\EN{ADDIU follows JALR (remember \IT{branch delay slots}?).}
\RU{Регистр \$4 также называется \$A0, который используется для передачи первого аргумента функции}%
\EN{The register \$4 is also called \$A0, which is used for passing the first function argument}%
\footnote{\RU{Таблица регистров в MIPS доступна в приложении}\EN{The MIPS registers table %
is available in appendix} \myref{MIPS_registers_ref}}.

\index{MIPS!\Instructions!JALR}
\RU{JALR (\q{Jump and Link Register}) делает переход по адресу в регистре \$25 (там адрес \puts) 
при этом сохраняя адрес следующей инструкции (LW) в \ac{RA}.}
\EN{JALR (\q{Jump and Link Register}) jumps to the address stored in the \$25 register (address of \puts) 
while saving the address of the next instruction (LW) in \ac{RA}.}
\RU{Это так же как и в ARM}\EN{This is very similar to ARM}.
\RU{И ещё одна важная вещь: адрес сохраняемый в \ac{RA} это адрес не следующей инструкции (потому что
это \IT{delay slot} и исполняется перед инструкцией перехода),
а инструкции после неё (после \IT{delay slot}).}
\EN{Oh, and one important thing is that the address saved in \ac{RA} is not the address of the next instruction (because
it's in a \IT{delay slot} and is executed before the jump instruction),
but the address of the instruction after the next one (after the \IT{delay slot}).}
\RU{Таким образом во время исполнения \TT{JALR} в \ac{RA} записывается $PC + 8$. В нашем случае это адрес
инструкции LW следующей после ADDIU.}
\EN{Hence, $PC + 8$ is written to \ac{RA} during the execution of \TT{JALR}, in our case, this is the address of the LW
instruction next to ADDIU.}

\RU{LW (\q{Load Word}) в строке 19 восстанавливает \ac{RA} из локального стека 
(эта инструкция скорее часть эпилога функции).}
\EN{LW (\q{Load Word}) at line 19 restores \ac{RA} from the local stack 
(this instruction is rather part of function epilogue).}

\index{MIPS!\Pseudoinstructions!MOVE}
\RU{MOVE в строке 22 копирует значение из регистра \$0 (\$ZERO) в \$2 (\$V0).}
\EN{MOVE at line 22 copies the value from \$0 (\$ZERO) register to \$2 (\$V0).}
\label{MIPS_zero_register}
\RU{В MIPS есть \IT{константный} регистр, всегда содержащий ноль.}
\EN{MIPS has a \IT{constant} register, which always holds zero.}
\RU{Должно быть, разработчики MIPS решили что 0 это самая востребованная константа в программировании,
так что пусть будет использоваться регистр \$0, всякий раз, когда будет нужен 0.}
\EN{Apparently, the MIPS developers came with the idea that zero is in fact the busiest constant in the computer programming,
so let's just use \$0 register every time zero is needed.}
\RU{Другой интересный факт: в MIPS нет инструкции, копирующей значения из регистра в регистр.}
\EN{Another interesting fact is that MIPS lacks instruction which transfers data between registers.}
\RU{На самом деле}\EN{In fact}, \TT{MOVE DST, SRC} \RU{это}\EN{is} \TT{ADD DST, SRC, \$ZERO} ($DST=SRC+0$), 
\RU{которая делает тоже самое}\EN{which does the same}.
\RU{Очевидно, разработчики MIPS хотели сделать как можно более компактную таблицу опкодов.}
\EN{Apparently, MIPS developers wanted to have compact opcode table.}
\RU{Это не значит, что сложение происходит во время каждой инструкции MOVE.}
\EN{This does not mean an actual addition happens at each MOVE instruction.}
\RU{Скорее всего, эти псевдоинструкции оптимизируются в \ac{CPU} и \ac{ALU} никогда не используется.}
\EN{Most likely, the \ac{CPU} optimizes these pseudoinstructions and \ac{ALU} is never used.}

\index{MIPS!\Instructions!J}
\RU{J в строке 24 делает переход по адресу в \ac{RA}, и это работает как выход из функции.}
\EN{J at line 24 jumps to the address in \ac{RA}, which is effectively performing return from the function.}
\RU{ADDIU после J на самом деле исполняется перед J (помните о \IT{branch delay slots}?) 
и это часть эпилога функции.}
\EN{ADDIU after J is in fact executed before J (remember \IT{branch delay slots}?) 
and is part of function epilogue.}

\RU{Вот листинг сгенерированный \IDA. Каждый регистр имеет свой псевдоним:}
\EN{Here is also a listing generated by \IDA. Each register here has its own pseudoname:}

\lstinputlisting[caption=\Optimizing GCC 4.4.5 (\IDA),numbers=left]{patterns/01_helloworld/MIPS/hw_O3_IDA.lst.\LANG}

\RU{Инструкция в строке 15 сохраняет GP в локальном стеке. Эта инструкция мистическим образом отсутствует
в листинге от GCC, может быть из-за ошибки в самом GCC\footnote{Очевидно, функция вывода листингов не так критична
для пользователей GCC, поэтому там вполне могут быть неисправленные ошибки.}.}
\EN{The instruction at line 15 saves GP value into the local stack, and this instruction is missing mysteriously from the GCC output listing, maybe by a GCC error\footnote{Apparently, functions generating listings 
are not so critical to GCC users, so some unfixed errors may still exist.}.}
\RU{Значение GP должно быть сохранено, потому что всякая функция может работать со своим собственным окном данных
размером 64KiB.}
\EN{The GP value has to be saved indeed, because each function can use its own 64KiB data window.}

\RU{Регистр, содержащий адрес функции \puts называется \$T9, потому что регистры с префиксом T- называются
\q{temporaries} и их содержимое можно не сохранять.}
\EN{The register containing the \puts address is called \$T9, because registers prefixed with T- are called
\q{temporaries} and their contents may not be preserved.}

\subsection{\NonOptimizing GCC}

\NonOptimizing GCC \RU{более многословный}\EN{is more verbose}.

\lstinputlisting[caption=\NonOptimizing GCC 4.4.5 (\assemblyOutput),numbers=left]{patterns/01_helloworld/MIPS/hw_O0.s.\LANG}

\RU{Мы видим, что регистр FP используется как указатель на фрейм стека.}
\EN{We see here that register FP is used as a pointer to the stack frame.}
\RU{Мы также видим 3 \ac{NOP}-а.}\EN{We also see 3 \ac{NOP}s.}
\RU{Второй и третий следуют за инструкциями перехода.}
\EN{The second and third of which follow the branch instructions.}

\RU{Я полагаю (хотя и не уверен), что компилятор GCC всегда добавляет \ac{NOP}-ы (из-за \IT{branch delay slots})
после инструкций переходов и затем, если включена оптимизация, от них может избавляться.}
\EN{I guess (not sure, though) that the GCC compiler always adds \ac{NOP}s (because of \IT{branch delay slots}) after branch
instructions and then, if optimization is turned on, maybe eliminates them.}
\RU{Так что они остались здесь}\EN{So in this case they are left here}.

\RU{Вот также листинг от \IDA:}
\EN{Here is also \IDA listing:}

\lstinputlisting[caption=\NonOptimizing GCC 4.4.5 (\IDA),numbers=left]{patterns/01_helloworld/MIPS/hw_O0_IDA.lst.\LANG}

\index{MIPS!\Pseudoinstructions!LA}
\RU{Интересно что \IDA распознала пару инструкций LUI/ADDIU и собрала их в одну псевдоинструкцию 
LA (\q{Load Address}) в строке 15.}
\EN{Interestingly, \IDA recognized LUI/ADDIU instructions pair and coalesces them into one 
LA (\q{Load Address}) pseudoinstruction at line 15.}
\RU{Мы также видим, что размер этой псевдоинструкции 8 байт!}
\EN{We may also see that this pseudoinstruction has size of 8 bytes!}
\RU{Это псевдоинструкция (или \IT{макрос}), потому что это не настоящая инструкция MIPS, а скорее
просто удобное имя для пары инструкций.}
\EN{This is a pseudoinstruction (or \IT{macro}) because it's not a real MIPS instruction, but rather
handy name for an instruction pair.}

\index{MIPS!\Pseudoinstructions!NOP}
\index{MIPS!\Instructions!OR}
\RU{Ещё кое что: \IDA не распознала \ac{NOP}-инструкции в строках 22, 26 и 41.}
\EN{Another thing is that \IDA doesn't recognize \ac{NOP} instructions, so here they are at lines 22, 26 and 41.}
\RU{Это}\EN{It is} \TT{OR \$AT, \$ZERO}.
\RU{По своей сути это инструкция, применяющая операцию ИЛИ к содержимому регистра \$AT с нулем, что,
конечно же, холостая операция.}
\EN{Essentially, this instruction applies OR operation to contents of \$AT register
with zero, which is, of course, idle instruction.}
\RU{MIPS, как и многие другие \ac{ISA}, не имеет отдельной \ac{NOP}-инструкции.}
\EN{MIPS, like many other \ac{ISA}s, doesn't have a separate \ac{NOP} instruction.}

\subsection{\RU{Роль стекового фрейма в этом примере}\EN{Role of the stack frame in this example}}

\RU{Адрес текстовой строки передается в регистре.}
\EN{The address of the text string is passed in register.}
\RU{Так зачем устанавливать локальный стек?}\EN{Why setup local stack anyway?}
\RU{Причина в том, что значения регистров \ac{RA} и GP должны быть сохранены где-то
(потому что вызывается \printf) и для этого используется локальный стек.}
\EN{The reason for this lies in the fact that registers \ac{RA} and GP's values has to be saved somewhere 
(because \printf is called), and the local stack is used for this purpose.}
\RU{Если бы это была \gls{leaf function}, тогда можно было бы избавиться от пролога и эпилога функции. Например:}
\EN{If this was a \gls{leaf function}, it would have been possible to get rid of the function prologue and epilogue,
for example:} \myref{MIPS_leaf_function_ex1}.

\subsection{\Optimizing GCC: \RU{загрузим в}\EN{load it into} GDB}

\index{GDB}
\lstinputlisting[caption=\RU{пример сессии в GDB}\EN{sample GDB session}]{patterns/01_helloworld/MIPS/O3_GDB.txt}

\fi

\section{\Conclusion{}}

\RU{Основная разница между кодом x86/ARM и x64/ARM64 в том, что указатель на строку теперь 64-битный.}
\EN{The main difference between x86/ARM and x64/ARM64 code is that pointer to the string is now 64-bits in length.}
\RU{Действительно, ведь для того современные \ac{CPU} и стали 64-битными, потому что подешевела память,
<<<<<<< HEAD
её теперь можно поставить в компьютер намного больше, и чтобы её адресовать, 32-х бит уже
недостаточно.}
\EN{Indeed, modern \ac{CPU}s are now 64-bit due to both the reduced cost of memory and the greater demand for it by modern applications. We can add much more memory to our computers than 32-bit pointers are able to address.}
=======
её теперь можно поставить в компьютер намного больше, но чтобы её адресовать, 32-х бит уже недостаточно.}
\EN{Indeed, modern \ac{CPU}s are 64-bit now because memory is cheaper nowadays. We can
add much more memory to our computers but 32-bit pointers are not enough to address it.}
>>>>>>> importchanges
\RU{Поэтому все указатели теперь 64-битные.}
\EN{As such, all pointers are now 64-bit.}

% sections
\ifdefined\IncludeExercises
\sectionold{\Exercises}

\begin{itemize}
	\item \url{http://challenges.re/48}
	\item \url{http://challenges.re/49}
\end{itemize}


\fi
