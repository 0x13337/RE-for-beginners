\subsection{\RU{Сохранение адреса куда должно вернуться управление после вызова функции}
\EN{Save the return address where a function must return control after execution}}

\subsubsection{x86}

\index{x86!\Instructions!CALL}
\RU{При вызове другой функции через \CALL сначала в стек записывается адрес, указывающий на место аккурат после 
инструкции \CALL, затем делается безусловный переход (почти как \TT{JMP}) на адрес, указанный в операнде.} 
\EN{While calling another function with a \CALL instruction the address of the point exactly after the \CALL instruction is saved 
to the stack and then an unconditional jump to the address in the CALL operand is executed.} 

\index{x86!\Instructions!PUSH}
\index{x86!\Instructions!JMP}
\RU{\CALL ~--- это аналог пары инструкций \TT{PUSH address\_after\_call / JMP}}
\EN{The \CALL instruction is equivalent to a \TT{PUSH address\_after\_call / JMP operand} instruction pair}.

\index{x86!\Instructions!RET}
\index{x86!\Instructions!POP}
\RU{\RET вытаскивает из стека значение и передает управление по этому адресу ~--- 
это аналог пары инструкций \TT{POP tmp / JMP tmp}.}
\EN{\RET fetches a value from the stack and jumps to it~---it is equivalent to a \TT{POP tmp / JMP tmp} instruction pair.}

\index{\Stack!\RU{Переполнение стека}\EN{Stack overflow}}
\index{\Recursion}
\RU{Крайне легко устроить переполнение стека, запустив бесконечную рекурсию:}
\EN{Overflowing the stack is straightforward. Just run eternal recursion:}

\begin{lstlisting}
void f()
{
	f();
};
\end{lstlisting}

\RU{MSVC 2008 предупреждает о проблеме:}\EN{MSVC 2008 reports the problem:}

\begin{lstlisting}
c:\tmp6>cl ss.cpp /Fass.asm
Microsoft (R) 32-bit C/C++ Optimizing Compiler Version 15.00.21022.08 for 80x86
Copyright (C) Microsoft Corporation.  All rights reserved.

ss.cpp
c:\tmp6\ss.cpp(4) : warning C4717: 'f' : recursive on all control paths, function will cause runtime stack overflow
\end{lstlisting}

\dots \RU{но, тем не менее, создает нужный код}\EN{but generates the right code anyway}:

\begin{lstlisting}
?f@@YAXXZ PROC						; f
; File c:\tmp6\ss.cpp
; Line 2
	push	ebp
	mov	ebp, esp
; Line 3
	call	?f@@YAXXZ				; f
; Line 4
	pop	ebp
	ret	0
?f@@YAXXZ ENDP						; f
\end{lstlisting}

\dots \RU{причем, если включить оптимизацию (\Ox), то будет даже интереснее, без переполнения стека, 
но работать будет \IT{корректно}\footnote{здесь ирония}:}
\EN{Also if we turn on optimization (\Ox option) the optimized code will not overflow the stack 
but instead will work \IT{correctly}\footnote{irony here}:}

\begin{lstlisting}
?f@@YAXXZ PROC						; f
; File c:\tmp6\ss.cpp
; Line 2
$LL3@f:
; Line 3
	jmp	SHORT $LL3@f
?f@@YAXXZ ENDP						; f
\end{lstlisting}

\RU{GCC 4.4.1 генерирует точно такой же код в обоих случаях, хотя и не предупреждает о проблеме.}
\EN{GCC 4.4.1 generates similar code in both cases, although without issuing any warning about the problem.}

\subsubsection{ARM}

\index{ARM!\Registers!Link Register}
\RU{Программы для ARM также используют стек для сохранения \ac{RA}, куда нужно вернуться, но несколько иначе}\EN{ARM
programs also use the stack for saving return addresses, but differently}.
\RU{Как уже упоминалось в секции}\EN{As mentioned in} ``\HelloWorldSectionName''~(\ref{sec:hw_ARM}),
\RU{\ac{RA} записывается в регистр}\EN{the \ac{RA} is saved to the} \ac{LR} (\gls{link register}).
\RU{Но если есть необходимость вызывать какую-то другую функцию и использовать регистр \ac{LR} еще
раз, его значение желательно сохранить}
\EN{However, if one needs to call another function and use the \ac{LR} register
one more time its value should be saved}.
\index{Function prologue}
\RU{Обычно это происходит в прологе функции, часто мы видим там инструкцию вроде}
\EN{Usually it is saved in the function prologue. Often, we see instructions like}
\index{ARM!\Instructions!PUSH}
\index{ARM!\Instructions!POP}
\TT{``PUSH {R4-R7,LR}''} \RU{, а в эпилоге}\EN{along with this instruction in epilogue}
\TT{``POP {R4-R7,PC}''}\RU{ ~--- так сохраняются регистры, которые будут использоваться в текущей функции, в том числе}
\EN{~---thus register values
to be used in the function are saved in the stack, including} \ac{LR}.

\index{ARM!Leaf function}
\RU{Тем не менее, если некая функция не вызывает никаких более функций, в терминологии ARM она называется}
\EN{Nevertheless, if a function never calls any other function, in ARM terminology it is called a}
\IT{\gls{leaf function}}\footnote{\url{http://infocenter.arm.com/help/index.jsp?topic=/com.arm.doc.faqs/ka13785.html}}. 
\RU{Как следствие, ``leaf''-функция не сохраняет регистр \ac{LR} (потому что не изменяет его).}
\EN{As a consequence, leaf functions do not save the \ac{LR} register (because doesn't modify it).}
\RU{А если эта функция небольшая, использует мало регистров, она может не использовать стек вообще}
\EN{If this function is small and uses a small number of registers, it may not use the stack at all}.
\RU{Таким образом, в ARM возможен вызов небольших leaf-функций не используя стек}
\EN{Thus, it is possible to call leaf functions without using the stack}.
\RU{Это может быть быстрее чем в старых x86, ведь внешняя память для стека не используется}
\EN{This can be faster than on older x86 because external RAM is not used for the stack}
\footnote{\RU{Когда-то, очень давно, на PDP-11 и VAX на инструкцию CALL (вызов других функций) могло тратиться
вплоть до 50\% времени (возможно из-за работы с памятью),
поэтому считалось, что много небольших функций это \glslink{anti-pattern}{анти-паттерн}}
\EN{Some time ago, on PDP-11 and VAX, the CALL instruction (calling other functions) was expensive; up to 50\%
of execution time might be spent on it, so it was common sense that big number of small function is \gls{anti-pattern}}\cite[Chapter 4, Part II]{Raymond:2003:AUP:829549}.}.
\RU{Либо это может быть полезным для тех ситуаций, когда память для стека еще не выделена либо недоступна}
\EN{It can be useful for such situations when memory for the stack is not yet allocated or not available}.

\EN{Some examples of leaf functions here are}\RU{Некоторые примеры таких ф-ций здесь}: \lstname 
\ref{ARM_leaf_example1}, \ref{ARM_leaf_example2}, 
\ref{ARM_leaf_example3}, \ref{ARM_leaf_example4}, \ref{ARM_leaf_example5},
\ref{ARM_leaf_example6}, \ref{ARM_leaf_example7}, \ref{ARM_leaf_example8},
\ref{ARM_leaf_example9}, \ref{ARM_leaf_example10}.
