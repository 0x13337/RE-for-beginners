\subsection{Salvare l'indirizzo di ritorno della funzione}

\subsubsection{x86}

\myindex{x86!\Instructions!CALL}
Quando si chiama una funzione con l'istruzione \CALL, l'indirizzo del punto esattamente dopo la \CALL viene salvato nello stack, e successivamente
viene eseguito un jump non condizionale all'indirizzo dell'operando di CALL.

\myindex{x86!\Instructions!PUSH}
\myindex{x86!\Instructions!JMP}
L'istruzione \CALL e' equivalente alla coppia di istruzioni \INS{PUSH indirizzo\_dopo\_call / JMP operando}.

\myindex{x86!\Instructions!RET}
\myindex{x86!\Instructions!POP}
\RET preleva un valore dallo stack e effettua un jump ad esso~--- cio' equivale alla coppia di istruzioni \TT{POP tmp / JMP tmp}.

\myindex{\Stack!\RU{Переполнение стека}\EN{Stack overflow}\PTBRph{}}
\myindex{\Recursion}

Riempire lo stack fino allo straripamento e' semplicissimo. Basta ricorrere alla ricorsione eterna:

\begin{lstlisting}
void f()
{
	f();
};
\end{lstlisting}

MSVC 2008 riporta il problema:

\begin{lstlisting}
c:\tmp6>cl ss.cpp /Fass.asm
Microsoft (R) 32-bit C/C++ Optimizing Compiler Version 15.00.21022.08 for 80x86
Copyright (C) Microsoft Corporation.  All rights reserved.

ss.cpp
c:\tmp6\ss.cpp(4) : warning C4717: 'f' : recursive on all control paths, function will cause runtime stack overflow
\end{lstlisting}

\dots ma genera in ogni caso il codice correttamente:

\begin{lstlisting}
?f@@YAXXZ PROC						; f
; File c:\tmp6\ss.cpp
; Line 2
	push	ebp
	mov	ebp, esp
; Line 3
	call	?f@@YAXXZ				; f
; Line 4
	pop	ebp
	ret	0
?f@@YAXXZ ENDP						; f
\end{lstlisting}

\dots Se attiviamo le ottimizzazioni del compilatore (\TT{\Ox} option) il codice ottimizzato non causera' overflow dello stack 
e funzionera' invece \IT{correttamente}\footnote{sarcasmo, si fa per dire}:

\begin{lstlisting}
?f@@YAXXZ PROC						; f
; File c:\tmp6\ss.cpp
; Line 2
$LL3@f:
; Line 3
	jmp	SHORT $LL3@f
?f@@YAXXZ ENDP						; f
\end{lstlisting}

GCC 4.4.1 genera codice simile in antrambi i casi, senza avvertire del problema.

\subsubsection{ARM}

\myindex{ARM!\Registers!Link Register}
Anche i programmi ARM usano lo stack per salvare gli indirizzi di ritorno, ma lo fanno in maniera diversa.
Come detto in \q{\HelloWorldSectionName}~(\myref{sec:hw_ARM}),
As mentioned in 
il \ac{RA} viene salvato nel \ac{LR} (\gls{link register}).
Se si presenta comunque la necessita' di chiamare un'altra funzione ed usare il registro \ac{LR} ancora una volta, 
il suo valore deve essere salvato.
\myindex{Function prologue}
Solitamente questo valore e' slvato nel preambolo della funzione.

\myindex{ARM!\Instructions!PUSH}
\myindex{ARM!\Instructions!POP}
Spesso vediamo istruzioni come \INS{PUSH {R4-R7,LR}} insieme ad isrtuzioni nell'epilogo come 
\INS{POP {R4-R7,PC}}---percio' i valori dei registri che saranno usati nella funzione vengono salvati nello stack, incluso \ac{LR}.

\myindex{ARM!Leaf function}
Ciononostante, se una funzione non chiama al suo interno nessun'altra funzione, in terminologia \ac{RISC} e' detta 
\IT{\gls{leaf function}}, o funzione foglia.\footnote{\href{http://go.yurichev.com/17064}{infocenter.arm.com/help/index.jsp?topic=/com.arm.doc.faqs/ka13785.html}}. 
Di conseguenza, le leaf functions non salvano il registro \ac{LR} register (perche' difatti non lo modificano).
Se una simile funzione e' molto breve e usa un piccolo numero di registri, potrebbe non usare del tutto lo stack. 
E' quindi possible chiamare le leaf functions senza usare lo stack, cosa che puo' essere piu' veloce che sulle macchine x86 perche' ;a RA< esterna non viene usata per lo stack
\footnote{Tempo fa, su PDP-11 and VAX, l'istruzione CALL instruction (chiamare altre funzioni) era costosa; poteva richiedere fino al 50\%
del tempo di esecuzione, ed era quindi consuetudine pensare che avere un grande numero di piccole funzioni fosee un \gls{anti-pattern} [\TAOUP Chapter 4, Part II].}.
Lo stesso principio puo' tornare utile quando la memoria per lo stack non e' stata ancora allocata o non e' disponibile.

Alcuni esempi di funzioni foglia:
\myref{ARM_leaf_example1}, \myref{ARM_leaf_example2}, 
\myref{ARM_leaf_example3}, \myref{ARM_leaf_example4}, \myref{ARM_leaf_example5},
\myref{ARM_leaf_example6}, \myref{ARM_leaf_example7}, \myref{ARM_leaf_example10}.
