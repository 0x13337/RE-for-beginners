\subsubsection{Passing function arguments}

The most popular way to pass parameters in x86 is called \q{cdecl}:

\begin{lstlisting}
push arg3
push arg2
push arg1
call f
add esp, 12 ; 4*3=12
\end{lstlisting}

\Gls{callee} functions get their arguments via the stack pointer.

Therefore, this is how the argument values are located in the stack before the execution of the \ttf{} function's very first instruction:

\begin{center}
\begin{tabular}{ | l | l | }
\hline
ESP & return address \\
\hline
ESP+4 & \argument \#1, \MarkedInIDAAs{} \TT{arg\_0} \\
\hline
ESP+8 & \argument \#2, \MarkedInIDAAs{} \TT{arg\_4} \\
\hline
ESP+0xC & \argument \#3, \MarkedInIDAAs{} \TT{arg\_8} \\
\hline
\dots & \dots \\
\hline
\end{tabular}
\end{center}

For more information on other calling conventions see also section~(\myref{sec:callingconventions}).

\par
By the way, the \gls{callee} function does not have any information about how many arguments were passed.
C functions with a variable number of arguments (like \printf) determine their number using format string specifiers (which begin with the \% symbol).

If we write something like:

\begin{lstlisting}
printf("%d %d %d", 1234);
\end{lstlisting}

\printf will print 1234, and then two random numbers, which were lying next to it in the stack.

\par
That's why it is not very important how we declare the \main function: as \main, \TT{main(int argc, char *argv[])} or \TT{main(int argc, char *argv[], char *envp[])}.

In fact, the \ac{CRT}-code is calling \main roughly as:
	
\begin{lstlisting}
push envp
push argv
push argc
call main
...
\end{lstlisting}

If you declare \main as \main without arguments, they are, nevertheless, still present in the stack, but are not used.
If you declare \main as  \TT{main(int argc, char *argv[])},
you will be able to use first two arguments, and the third will remain \q{invisible} for your function.
Even more, it is possible to declare \TT{main(int argc)}, and it will work.

\myparagraph{Alternative ways of passing arguments}

It is worth noting that nothing obliges programmers to pass arguments through the stack. It is not a requirement.
One could implement any other method without using the stack at all.

A somewhat popular way among assembly language newbies is to pass arguments via global variables, like:

\lstinputlisting[caption=Assembly code]{patterns/02_stack/global_args.asm}

But this method has obvious drawback: \IT{do\_something()} function cannot call itself recursively (or via another function),
because it has to zap its own arguments.
The same story with local variables: if to hold them in global variables, the function couldn't call itself.
And this is also not thread-safe
\footnote{Correctly implemented, each thread would have its own stack with its own arguments/variables.}.
A method to store such information in stack makes this easier---it can hold as many function arguments and/or values,
as much space it has.

[\TAOCPvolI{}, 189] mentions even weirder schemes particularly convenient on IBM System/360.

\myindex{MS-DOS}
\myindex{x86!\Instructions!INT}

MS-DOS had a way of passing all function arguments via registers, for example, this is piece of
code for ancient 16-bit MS-DOS prints ``Hello, world!'':

\begin{lstlisting}
mov  dx, msg      ; address of message
mov  ah, 9        ; 9 mean "print string" function
int  21h          ; DOS "syscall"

mov  ah, 4ch      ; "terminate program" function
int  21h          ; DOS "syscall"

msg  db 'Hello, World!\$'
\end{lstlisting}

\myindex{fastcall}
This is quite similar to \myref{fastcall} method.
And also it's very similar to calling syscalls in Linux (\myref{linux_syscall}) and Windows.

\myindex{x86!\Flags!CF}
If a MS-DOS function is going to return a boolean value (i.e., single bit, usually indicating error state),
\TT{CF} flag was often used.

For example:

\begin{lstlisting}
mov ah, 3ch       ; create file
lea dx, filename
mov cl, 1
int 21h
jc  error
mov file_handle, ax
...
error:
...
\end{lstlisting}

In case of error, \TT{CF} flag is raised. Otherwise, handle of newly created file is returned via \TT{AX}.

This method is still used by assembly language programmers.
In Windows Research Kernel source code (which is quite similar to Windows 2003) we can find something like this
(file \IT{base/ntos/ke/i386/cpu.asm}):

\begin{lstlisting}
        public  Get386Stepping
Get386Stepping  proc

        call    MultiplyTest            ; Perform multiplication test
        jnc     short G3s00             ; if nc, muttest is ok
        mov     ax, 0
        ret
G3s00:
        call    Check386B0              ; Check for B0 stepping
        jnc     short G3s05             ; if nc, it's B1/later
        mov     ax, 100h                ; It is B0/earlier stepping
        ret

G3s05:
        call    Check386D1              ; Check for D1 stepping
        jc      short G3s10             ; if c, it is NOT D1
        mov     ax, 301h                ; It is D1/later stepping
        ret

G3s10:
        mov     ax, 101h                ; assume it is B1 stepping
        ret

	...

MultiplyTest    proc

        xor     cx,cx                   ; 64K times is a nice round number
mlt00:  push    cx
        call    Multiply                ; does this chip's multiply work?
        pop     cx
        jc      short mltx              ; if c, No, exit
        loop    mlt00                   ; if nc, YEs, loop to try again
        clc
mltx:
        ret

MultiplyTest    endp
\end{lstlisting}

