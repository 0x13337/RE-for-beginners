\subsection{Передача параметров функции}

Самый распространенный способ передачи параметров в x86 называется \q{cdecl}:

\begin{lstlisting}
push arg3
push arg2
push arg1
call f
add esp, 12 ; 4*3=12
\end{lstlisting}

Вызываемая функция получает свои параметры также через указатель стека.

Следовательно, так расположены значения в стеке перед исполнением самой первой инструкции функции \ttf{}:

\begin{center}
\begin{tabular}{ | l | l | }
\hline
ESP & адрес возврата \\
\hline
ESP+4 & \argument \#1, \MarkedInIDAAs{} \TT{arg\_0} \\
\hline
ESP+8 & \argument \#2, \MarkedInIDAAs{} \TT{arg\_4} \\
\hline
ESP+0xC & \argument \#3, \MarkedInIDAAs{} \TT{arg\_8} \\
\hline
\dots & \dots \\
\hline
\end{tabular}
\end{center}

\ifx\LITE\undefined
См. также в соответствующем разделе о других способах передачи аргументов через стек~(\myref{sec:callingconventions}).
\fi % LITE

Важно отметить, что, в общем, никто не заставляет программистов передавать параметры именно через стек, это не является требованием к исполняемому коду.
Вы можете делать это совершенно иначе, не используя стек вообще.

К примеру, можно выделять в \glslink{heap}{куче} место для аргументов, заполнять их и передавать в функцию указатель на это место через \EAX. И это вполне будет работать
\footnote{Например, в книге Дональда Кнута \q{Искусство программирования}, в разделе 1.4.1 
посвященном подпрограммам \cite[раздел 1.4.1]{Knuth:1998:ACP:521463}, 
мы можем прочитать о возможности располагать параметры для вызываемой подпрограммы после инструкции \JMP,
передающей управление подпрограмме. Кнут описывает, что это было особенно удобно для компьютеров IBM System/360.}.
% FIXME:
% I am unsure about what this comment means.
% My guess is that the arguments are put in the memory position after
% the jump instruction, so you could say:
% "one way to supply arguments to a subroutine is simply to list them in memory
% after the \JMP instruction that passes control to the subroutine."
% Right now, "after" also sounds like it refers to the time after
% the jump happens, which I think is too late.
Однако традиционно сложилось, что в x86 и ARM передача аргументов происходит именно через стек.

\par
Кстати, вызываемая функция не имеет информации о количестве переданных ей аргументов.
Функции Си с переменным количеством аргументов (как \printf) определяют их количество по спецификаторам строки формата (начинающиеся со знака \%).

Если написать что-то вроде:

\begin{lstlisting}
printf("%d %d %d", 1234);
\end{lstlisting}

\printf выведет 1234, затем ещё два случайных числа, которые волею случая оказались в стеке рядом.

\par
Вот почему не так уж и важно, как объявлять функцию \main{}: как \main{}, \TT{main(int argc, char *argv[])} либо \TT{main(int argc, char *argv[], char *envp[])}.

В реальности, \ac{CRT}-код вызывает \main примерно так:
	
\begin{lstlisting}
push envp
push argv
push argc
call main
...
\end{lstlisting}

Если вы объявляете \main без аргументов, они, тем не менее, присутствуют в стеке, но не используются.
Если вы объявите \main как \TT{main(int argc, char *argv[])}, 
вы можете использовать два первых аргумента, а третий останется для вашей функции \q{невидимым}.
Более того, можно даже объявить \TT{main(int argc)}, и это будет работать.

