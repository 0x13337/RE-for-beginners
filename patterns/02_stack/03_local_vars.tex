\subsection{\RU{Хранение локальных переменных}\EN{Local variable storage}\PTBR{Armazenamento de variáveis locais}}

\RU{Функция может выделить для себя некоторое место в стеке для локальных переменных, просто отодвинув 
\glslink{stack pointer}{указатель стека} глубже к концу стека.}%
\EN{A function could allocate space in the stack for its local variables just by decreasing 
the \gls{stack pointer} towards the stack bottom.}%
\PTBR{Uma função poderia alocar espaço na pilha para suas variáveis locais simplesmente decrementando o ponteiro da pilha.}%
% I think here, "stack bottom" means the lowest address in the stack space,
% but the reader might also think it means towards the top of the stack space,
% like in a pop, so you might change "towards the stack bottom" to
% "towards the lowest address of the stack", or just take it out,
% since "decreasing" also suggests that.
\RU{Это очень быстро вне зависимости от количества локальных переменных.}%
\EN{Hence, it's very fast, no matter how many local variables are defined.}%
\PTBR{Consequentemente, é muito rápido, não importando quantas variáveis locais serão definidas.}%

\RU{Хранить локальные переменные в стеке не является необходимым требованием. 
Вы можете хранить локальные переменные где угодно. 
Но по традиции всё сложилось так.}%
\EN{It is also not a requirement to store local variables in the stack.
You could store local variables wherever you like, 
but traditionally this is how it's done.}%
\PTBR{Também não é um requisito armazenar variáveis locais na pilha.
Você pode armazenar variáveis locais onde você quiser, mas, tradicionalmente, é assim que é feito.}%

