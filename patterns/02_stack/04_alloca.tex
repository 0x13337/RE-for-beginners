\subsection{x86: \IFRU{Функция alloca()}{alloca() function}}
\label{alloca}
\index{\CStandardLibrary!alloca()}
\IFRU{Интересен случай с функцией \TT{alloca()}}
{It is worth noting the \TT{alloca()} function.}\footnote{
\IFRU
{В MSVC, реализацию функции можно посмотреть в файлах}
{In MSVC, the function implementation can be found in} 
  \TT{alloca16.asm} 
  \AndENRU 
  \TT{chkstk.asm} 
  \InENRU 
  \TT{C:\textbackslash{}Program Files (x86)\textbackslash{}Microsoft Visual Studio 10.0\textbackslash{}VC\textbackslash{}crt\textbackslash{}src\textbackslash{}intel}}. 

\IFRU{Эта функция работает как \TT{malloc()}, но выделяет память прямо в стеке.} 
{This function works like \TT{malloc()} but allocates memory just on the stack.}

\IFRU{Память освобождать через \TT{free()} не нужно, так как эпилог функции~(\ref{sec:prologepilog})
вернет \ESP назад в изначальное состояние и выделенная память просто аннулируется.}
{The allocated memory chunk does not need to be freed via a \TT{free()} function call since the 
function epilogue~(\ref{sec:prologepilog}) will return \ESP back to its initial state and 
the allocated memory will be just annulled.} 

\IFRU{Интересна реализация функции \TT{alloca()}.}
{It is worth noting how \TT{alloca()} is implemented.}

\IFRU{Эта функция, если упрощенно, просто сдвигает \ESP вглубь стека 
на столько байт, сколько вам нужно и возвращает \ESP в качестве указателя на выделенный блок.}
{In simple terms, this function just shifts \ESP downwards toward the stack bottom by the number of bytes you 
need and sets \ESP as a pointer to the \IT{allocated} block.}
\IFRU{Попробуем:}{Let's try:}

\lstinputlisting{patterns/02_stack/2_1.c}

\IFRU{(Функция \TT{\_snprintf()} работает так же, как и \printf, только вместо выдачи результата в 
stdout (т.е., на терминал или в консоль),
записывает его в буфер \TT{buf}. \puts выдает содержимое буфера \TT{buf} в stdout. Конечно, можно было бы
заменить оба этих вызова на один \printf, но мне нужно проиллюстрировать использование небольшого буфера.)}
{(\TT{\_snprintf()} function works just like \printf, but instead of dumping the result into stdout (e.g., to terminal or 
console), it writes to the \TT{buf} buffer. \puts copies \TT{buf} contents to stdout. Of course, these two
function calls might be replaced by one \printf call, but I would like to illustrate small buffer usage.)}

\subsubsection{MSVC}

\IFRU{Компилируем}{Let's compile} (MSVC 2010):

\lstinputlisting[caption=MSVC 2010]{patterns/02_stack/2_2_msvc.asm}

\index{Compiler intrinsic}
\IFRU {Единственный параметр в \TT{alloca()} передается через \EAX, а не как обычно через стек}
{The sole \TT{alloca()} argument is passed via \EAX (instead of pushing into stack)}
\footnote{
\IFRU{Это потому, что alloca() ~--- это не сколько функция, сколько т.н. \IT{compiler intrinsic} (\ref{sec:compiler_intrinsic})}
{It is because alloca() is rather compiler intrinsic (\ref{sec:compiler_intrinsic}) than usual function}.

\IFRU{Одна из причин, почему здесь нужна именно функция, а не несколько инструкций прямо в коде в том, что в реализации 
функции alloca() от \ac{MSVC}
есть также код, читающий из только что выделенной памяти, чтобы \ac{OS} подключила физическую память к этому региону \ac{VM}.}
{One of the reason there is a separate function instead of couple instructions just in the code,
because \ac{MSVC} implementation
of the alloca() function also has a code which reads from the memory just allocated, in order to let \ac{OS} to map
physical memory to this \ac{VM} region.}
}.
\IFRU{После вызова \TT{alloca()}, \ESP теперь указывает на блок в 600 байт, который 
мы можем использовать под \TT{buf}.}
{After the \TT{alloca()} call, \ESP points to the block of 600 bytes and we can 
use it as memory for the \TT{buf} array.}

\subsubsection{GCC + \IntelSyntax}

\IFRU{А GCC 4.4.1 обходится без вызова других функций:}
{GCC 4.4.1 can do the same without calling external functions:}

\lstinputlisting[caption=GCC 4.7.3]{patterns/02_stack/2_1_gcc_intel_O3_\LANG.asm}

\subsubsection{GCC + \ATTSyntax}

\IFRU{Посмотрим на тот же код, только в синтаксисе AT\&T}{Let's see the same code, but in AT\&T syntax}:

\lstinputlisting[caption=GCC 4.7.3]{patterns/02_stack/2_1_gcc_ATT_O3.s}

\index{\ATTSyntax}
\IFRU{Всё то же самое, что и в прошлом листинге.}{The code is the same as in the previous listing.}

N.B. \IFRU{Например,}{E.g.} \TT{movl \$3, 20(\%esp)} 
\IFRU{ ~--- это аналог}{is analogous to} \TT{mov DWORD PTR [esp+20], 3} \IFRU{в Intel-синтаксисе:}
{ in Intel-syntax}\IFRU{ при адресации памяти в виде}{~---when addressing memory in form} \IT{\IFRU{регистр+смещение}{register+offset}}, 
\IFRU{это записывается в AT\&T синтаксисе как}{it is written in AT\&T syntax as} 
\TT{\IFRU{смещение}{offset}(\%\IFRU{регистр}{register})}.

