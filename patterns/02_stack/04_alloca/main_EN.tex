\subsubsection{x86: alloca() function}
\label{alloca}
\myindex{\CStandardLibrary!alloca()}

\newcommand{\AllocaSrcPath}{C:\textbackslash{}Program Files (x86)\textbackslash{}Microsoft Visual Studio 10.0\textbackslash{}VC\textbackslash{}crt\textbackslash{}src\textbackslash{}intel}

It is worth noting the \TT{alloca()} function
\footnote{In MSVC, the function implementation can be found in \TT{alloca16.asm} and \TT{chkstk.asm} in \\
\TT{\AllocaSrcPath{}}}.
This function works like \TT{malloc()}, but allocates memory directly on the stack.
% page break added to prevent "\vref on page boundary" error. it may be dropped in future.
The allocated memory chunk does not need to be freed via a \TT{free()} function call, \\
since the function epilogue (\myref{sec:prologepilog}) returns \ESP back to its initial state and 
the allocated memory is just \IT{dropped}.
It is worth noting how \TT{alloca()} is implemented.
In simple terms, this function just shifts \ESP downwards toward the stack bottom by the number of bytes you need and sets \ESP as a pointer to the \IT{allocated} block.

Let's try:

\lstinputlisting{patterns/02_stack/04_alloca/2_1.c}

\TT{\_snprintf()} function works just like \printf, but instead of dumping the result into \gls{stdout} (e.g., to terminal or 
console), it writes it to the \TT{buf} buffer. Function \puts copies the contents of \TT{buf} to \gls{stdout}. Of course, these two
function calls might be replaced by one \printf call, but we have to illustrate small buffer usage.

\myparagraph{MSVC}

Let's compile (MSVC 2010):

\lstinputlisting[caption=MSVC 2010]{patterns/02_stack/04_alloca/2_2_msvc.asm}

\myindex{Compiler intrinsic}
The sole \TT{alloca()} argument is passed via \EAX (instead of pushing it into the stack)
\footnote{It is because alloca() is rather a compiler intrinsic (\myref{sec:compiler_intrinsic}) than a normal function.
One of the reasons we need a separate function instead of just a couple of instructions in the code,
is because the \ac{MSVC} alloca() implementation also has code which reads from the memory just allocated, in order to let the \ac{OS} map
physical memory to this \ac{VM} region.
After the \TT{alloca()} call, \ESP points to the block of 600 bytes and we can use it as memory for the \TT{buf} array.}.

\myparagraph{GCC + \IntelSyntax}

GCC 4.4.1 does the same without calling external functions:

\lstinputlisting[caption=GCC 4.7.3]{patterns/02_stack/04_alloca/2_1_gcc_intel_O3_EN.asm}

\myparagraph{GCC + \ATTSyntax}

Let's see the same code, but in AT\&T syntax:

\lstinputlisting[caption=GCC 4.7.3]{patterns/02_stack/04_alloca/2_1_gcc_ATT_O3.s}

\myindex{\ATTSyntax}
The code is the same as in the previous listing.

By the way, \INS{movl \$3, 20(\%esp)} corresponds to
\INS{mov DWORD PTR [esp+20], 3} in Intel-syntax.
In the AT\&T syntax, the register+offset format of addressing memory looks like
\TT{offset(\%{register})}.

