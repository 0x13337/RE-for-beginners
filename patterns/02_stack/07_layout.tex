\section{\RU{Разметка типичного стека}\EN{Typical stack layout}}

\RU{Разметка типичного стека в 32-битной среде на момент начала ф-ции выглядит так}
\EN{A very typical stack layout in a 32-bit environment at the start of a function}:

\begin{center}
\begin{tabular}{ | l | l | }
\hline
\dots & \dots \\
\hline
ESP-0xC & \RU{локальная переменная}\EN{local variable} \#2, \MarkedInIDAAs{} \TT{var\_8} \\
\hline
ESP-8 & \RU{локальная переменная}\EN{local variable} \#1, \MarkedInIDAAs{} \TT{var\_4} \\
\hline
ESP-4 & \RU{сохраненное значение}\EN{saved value of} \EBP \\
\hline
ESP & \RU{адрес возврата}\EN{return address} \\
\hline
ESP+4 & \argument \#1, \MarkedInIDAAs{} \TT{arg\_0} \\
\hline
ESP+8 & \argument \#2, \MarkedInIDAAs{} \TT{arg\_4} \\
\hline
ESP+0xC & \argument \#3, \MarkedInIDAAs{} \TT{arg\_8} \\
\hline
\dots & \dots \\
\hline
\end{tabular}
\end{center}
