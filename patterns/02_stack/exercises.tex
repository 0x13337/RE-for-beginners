\section{\Exercises}

\subsection{\Exercise \#1}
\label{exercise_stack_1}

\RU{Если это скомпилировать в MSVC и запустить, появится три числа. Откуда они берутся? 
Откуда они берутся если скомпилировать в MSVC с оптимизациями (\Ox)?}
\EN{If to compile this piece of code in MSVC and run, a three number will be printed. 
Where they are came from?
Where they are came from if to compile it in MSVC with optimization (\Ox)?}
\RU{Почему в GCC ситуация совсем иная}\EN{Why the situation is completely different in GCC}?

\begin{lstlisting}
#include <stdio.h>

int main()
{
	printf ("%d, %d, %d\n");

	return 0;
};
\end{lstlisting}

\Answer{}: \ref{exercise_solutions_stack_1}.

\subsection{\Exercise \#2}
\label{exercise_stack_2}

\WhatThisCodeDoes\

\begin{lstlisting}[caption=MSVC 2010 /Ox]
$SG3103	DB	'%d', 0aH, 00H

_main	PROC
	push	0
	call	DWORD PTR __imp___time64
	push	edx
	push	eax
	push	OFFSET $SG3103 ; '%d'
	call	DWORD PTR __imp__printf
	add	esp, 16
	xor	eax, eax
	ret	0
_main	ENDP
\end{lstlisting}

\begin{lstlisting}[caption=\Optimizing Keil 5.03 (\ARMMode)]
main PROC
        PUSH     {r4,lr}
        MOV      r0,#0
        BL       time
        MOV      r1,r0
        ADR      r0,|L0.32|
        BL       __2printf
        MOV      r0,#0
        POP      {r4,pc}
        ENDP

|L0.32|
        DCB      "%d\n",0
\end{lstlisting}

\begin{lstlisting}[caption=\Optimizing Keil 5.03 (\ThumbMode)]
main PROC
        PUSH     {r4,lr}
        MOVS     r0,#0
        BL       time
        MOVS     r1,r0
        ADR      r0,|L0.20|
        BL       __2printf
        MOVS     r0,#0
        POP      {r4,pc}
        ENDP

|L0.20|
        DCB      "%d\n",0
\end{lstlisting}

\Answer{}: \ref{exercise_solutions_stack_2}.
