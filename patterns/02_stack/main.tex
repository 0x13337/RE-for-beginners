\section{\Stack}
\label{sec:stack}
\index{\Stack}

\IFRU{Стек в компьютерных науках ~--- это одна из наиболее фундаментальных вещей}
{A stack is one of the most fundamental data structures in computer science}
\footnote{\url{http://en.wikipedia.org/wiki/Call_stack}}.

\IFRU{Технически, это просто блок памяти в памяти процесса + регистр \ESP или \RSP в x86 или x64, либо \SP в ARM, который указывает где-то в пределах этого блока.}
{Technically, it is just a block of memory in process memory along with the \ESP or \RSP register in x86 or x64, or the \SP register in ARM, as a pointer within the block.}

\index{ARM!\Instructions!PUSH}
\index{ARM!\Instructions!POP}
\index{x86!\Instructions!PUSH}
\index{x86!\Instructions!POP}
\IFRU{Часто используемые инструкции для работы со стеком ~--- это \PUSH и \POP (в x86 и thumb-режиме ARM). 
\PUSH уменьшает \ESP/\RSP/\SP на $4$ в 32-битном режиме (или на $8$ в 64-битном),
затем записывает по адресу, на который указывает \ESP/\RSP/\SP, содержимое своего единственного операнда.}
{The most frequently used stack access instructions are \PUSH and \POP (in both x86 and ARM thumb-mode). 
\PUSH subtracts $4$ in 32-bit mode (or $8$ in 64-bit mode) from \ESP/\RSP/\SP and then writes the contents of its sole operand to the memory address pointed to by \ESP/\RSP/\SP.} 

\IFRU{\POP это обратная операция ~--- сначала достает из \glslink{stack pointer}{указателя стека} значение и помещает его в операнд 
(который очень часто является регистром) и затем увеличивает указатель стека на $4$ (или $8$).}
{\POP is the reverse operation: get the data from memory pointed to by \SP, 
put it in the operand (often a register) and then add $4$ (or $8$) to the \gls{stack pointer}.}

\IFRU{В самом начале \glslink{stack pointer}{регистр-указатель} указывает на конец стека.}
{After stack allocation the \gls{stack pointer} points to the end of stack.}
\IFRU{\PUSH уменьшает \glslink{stack pointer}{регистр-указатель}, а \POP ~--- увеличивает.}
{\PUSH increases the \gls{stack pointer} and \POP decreases it.}
\IFRU{Конец стека находится в начале блока памяти, выделенного под стек. Это странно, но это так.}
{The end of the stack is actually at the beginning of the memory allocated for the stack block. 
It seems strange, but it is so.}

\IFRU{В процессоре ARM, тем не менее, есть поддержка стеков, растущих как в сторону уменьшения, так и в
сторону увеличения}
{Nevertheless ARM has not only instructions supporting ascending stacks but also descending stacks}. \\
\index{ARM!\Instructions!STMFD}
\index{ARM!\Instructions!LDMFD}
\index{ARM!\Instructions!STMED}
\index{ARM!\Instructions!LDMED}
\index{ARM!\Instructions!STMFA}
\index{ARM!\Instructions!LDMFA}
\index{ARM!\Instructions!STMEA}
\index{ARM!\Instructions!LDMEA}
\IFRU{Например, инструкции}{For example the} 
STMFD\footnote{\STMFDdesc}/LDMFD\footnote{\LDMFDDESC}, 
STMED\footnote{\STMEDdesc}/LDMED\footnote{\LDMEDdesc} 
\IFRU{предназначены для descending-стека, т.е. уменьшающегося}{instructions are intended to deal with 
a descending stack}.
\IFRU{Инструкции}{The}
STMFA\footnote{\STMFAdesc}/LMDFA\footnote{\LDMFAdesc}, 
STMEA\footnote{\STMEAdesc}/LDMEA\footnote{\LDMEAdesc} 
\IFRU{предназначены для ascending-стека, т.е. увеличивающегося}{instructions are intended to deal with 
an ascending stack}.

\subsection{\IFRU{Почему стек растет в обратную сторону?}{Why stack grows backward?}}

\IFRU{Интуитивно мы можем подумать, что как и любая другая структура данных, стек мог бы расти вперед, 
т.е. в сторону увеличения адресов}{Intuitively, we might think that, like any other data structure, 
the stack may grow upward, i.e., towards higher addresses}.

\IFRU{Причина, почему стек растет назад, вероятно, историческая}
{The reason the stack grows backward is probably historical}.
\IFRU{Когда компьютеры были большие и занимали целую комнату, было очень легко разделить сегмент на две части,
для \glslink{heap}{кучи} и стека}{When computers were big and occupied a whole room, 
it was easy to divide memory into two parts, one for the \gls{heap} and one for the stack}.
\IFRU{Конечно, ведь заранее было неизвестно, насколько большой может быть \glslink{heap}{куча} или стек, 
так что это решение было самым простым}{Of course, 
it was unknown how big the \gls{heap} and the stack would be during program execution, 
so this solution was simplest possible}.

\begin{center}
	\begin{tikzpicture}
	\tikzstyle{every path}=[thick]

	\node [rectangle,draw,minimum width=6cm, minimum height=2cm] (memory) {};
	\node [] [right=0.2cm of memory.west] (heap) {Heap};
	\node [] [left=0.2cm of memory.east] (stack) {Stack};

	\node [] (center1) [right=2cm of memory.west] {};
	\node [] (center2) [left=2cm of memory.east] {};

	\draw [->] (heap) -- (center1);
	\draw [->] (stack) -- (center2);

	\node [] [above left=1.1cm and 0.2cm of heap] (t1) {\IFRU{Начало кучи}{Start of heap}};
	\node [] [above right=1.1cm and 0.2cm of stack] (t2) {\IFRU{Вершина стека}{Start of stack}};

	\draw [->] (t1) -- (memory.west);
	\draw [->] (t2) -- (memory.east);

	\end{tikzpicture}
\end{center}

\IFRU{В}{In} \cite{Ritchie74} \IFRU{можно прочитать}{we can read}:

\begin{framed}
\begin{quotation}
The user-core part of an image is divided into three logical segments. The program text segment begins at location 0 in the virtual address space. During execution, this segment is write-protected and a single copy of it is shared among all processes executing the same program. At the first 8K byte boundary above the program text segment in the virtual address space begins a nonshared, writable data segment, the size of which may be extended by a system call. Starting at the highest address in the virtual address space is a stack segment, which automatically grows downward as the hardware's stack pointer fluctuates.
\end{quotation}
\end{framed}

\subsection{\IFRU{Для чего используется стек?}{What is the stack used for?}}

% subsubsections here
\subsection{\RU{Сохранение адреса куда должно вернуться управление после вызова функции}
\EN{Save the return address where a function must return control after execution}}

\subsubsection{x86}

\index{x86!\Instructions!CALL}
\RU{При вызове другой функции через \CALL сначала в стек записывается адрес, указывающий на место аккурат после 
инструкции \CALL, затем делается безусловный переход (почти как \TT{JMP}) на адрес, указанный в операнде.} 
\EN{While calling another function with a \CALL instruction the address of the point exactly after the \CALL instruction is saved 
to the stack and then an unconditional jump to the address in the CALL operand is executed.} 

\index{x86!\Instructions!PUSH}
\index{x86!\Instructions!JMP}
\RU{\CALL ~--- это аналог пары инструкций \TT{PUSH address\_after\_call / JMP}}
\EN{The \CALL instruction is equivalent to a \TT{PUSH address\_after\_call / JMP operand} instruction pair}.

\index{x86!\Instructions!RET}
\index{x86!\Instructions!POP}
\RU{\RET вытаскивает из стека значение и передает управление по этому адресу ~--- 
это аналог пары инструкций \TT{POP tmp / JMP tmp}.}
\EN{\RET fetches a value from the stack and jumps to it~---it is equivalent to a \TT{POP tmp / JMP tmp} instruction pair.}

\index{\Stack!\RU{Переполнение стека}\EN{Stack overflow}}
\index{\Recursion}
\RU{Крайне легко устроить переполнение стека, запустив бесконечную рекурсию:}
\EN{Overflowing the stack is straightforward. Just run eternal recursion:}

\begin{lstlisting}
void f()
{
	f();
};
\end{lstlisting}

\RU{MSVC 2008 предупреждает о проблеме:}\EN{MSVC 2008 reports the problem:}

\begin{lstlisting}
c:\tmp6>cl ss.cpp /Fass.asm
Microsoft (R) 32-bit C/C++ Optimizing Compiler Version 15.00.21022.08 for 80x86
Copyright (C) Microsoft Corporation.  All rights reserved.

ss.cpp
c:\tmp6\ss.cpp(4) : warning C4717: 'f' : recursive on all control paths, function will cause runtime stack overflow
\end{lstlisting}

\dots \RU{но, тем не менее, создает нужный код}\EN{but generates the right code anyway}:

\begin{lstlisting}
?f@@YAXXZ PROC						; f
; File c:\tmp6\ss.cpp
; Line 2
	push	ebp
	mov	ebp, esp
; Line 3
	call	?f@@YAXXZ				; f
; Line 4
	pop	ebp
	ret	0
?f@@YAXXZ ENDP						; f
\end{lstlisting}

\dots \RU{причем, если включить оптимизацию (\Ox), то будет даже интереснее, без переполнения стека, 
но работать будет \IT{корректно}\footnote{здесь ирония}:}
\EN{Also if we turn on optimization (\Ox option) the optimized code will not overflow the stack 
but instead will work \IT{correctly}\footnote{irony here}:}

\begin{lstlisting}
?f@@YAXXZ PROC						; f
; File c:\tmp6\ss.cpp
; Line 2
$LL3@f:
; Line 3
	jmp	SHORT $LL3@f
?f@@YAXXZ ENDP						; f
\end{lstlisting}

\RU{GCC 4.4.1 генерирует точно такой же код в обоих случаях, хотя и не предупреждает о проблеме.}
\EN{GCC 4.4.1 generates similar code in both cases, although without issuing any warning about the problem.}

\subsubsection{ARM}

\index{ARM!\Registers!Link Register}
\RU{Программы для ARM также используют стек для сохранения \ac{RA}, куда нужно вернуться, но несколько иначе}\EN{ARM
programs also use the stack for saving return addresses, but differently}.
\RU{Как уже упоминалось в секции}\EN{As mentioned in} ``\HelloWorldSectionName''~(\ref{sec:hw_ARM}),
\RU{\ac{RA} записывается в регистр}\EN{the \ac{RA} is saved to the} \ac{LR} (\gls{link register}).
\RU{Но если есть необходимость вызывать какую-то другую функцию и использовать регистр \ac{LR} еще
раз, его значение желательно сохранить}
\EN{However, if one needs to call another function and use the \ac{LR} register
one more time its value should be saved}.
\index{Function prologue}
\RU{Обычно это происходит в прологе функции, часто мы видим там инструкцию вроде}
\EN{Usually it is saved in the function prologue. Often, we see instructions like}
\index{ARM!\Instructions!PUSH}
\index{ARM!\Instructions!POP}
\TT{``PUSH {R4-R7,LR}''} \RU{, а в эпилоге}\EN{along with this instruction in epilogue}
\TT{``POP {R4-R7,PC}''}\RU{ ~--- так сохраняются регистры, которые будут использоваться в текущей функции, в том числе}
\EN{~---thus register values
to be used in the function are saved in the stack, including} \ac{LR}.

\index{ARM!Leaf function}
\RU{Тем не менее, если некая функция не вызывает никаких более функций, в терминологии ARM она называется}
\EN{Nevertheless, if a function never calls any other function, in ARM terminology it is called a}
\IT{\gls{leaf function}}\footnote{\url{http://infocenter.arm.com/help/index.jsp?topic=/com.arm.doc.faqs/ka13785.html}}. 
\RU{Как следствие, ``leaf''-функция не сохраняет регистр \ac{LR} (потому что не изменяет его).}
\EN{As a consequence, leaf functions do not save the \ac{LR} register (because doesn't modify it).}
\RU{А если эта функция небольшая, использует мало регистров, она может не использовать стек вообще}
\EN{If this function is small and uses a small number of registers, it may not use the stack at all}.
\RU{Таким образом, в ARM возможен вызов небольших leaf-функций не используя стек}
\EN{Thus, it is possible to call leaf functions without using the stack}.
\RU{Это может быть быстрее чем в старых x86, ведь внешняя память для стека не используется}
\EN{This can be faster than on older x86 because external RAM is not used for the stack}
\footnote{\RU{Когда-то, очень давно, на PDP-11 и VAX на инструкцию CALL (вызов других функций) могло тратиться
вплоть до 50\% времени (возможно из-за работы с памятью),
поэтому считалось, что много небольших функций это \glslink{anti-pattern}{анти-паттерн}}
\EN{Some time ago, on PDP-11 and VAX, the CALL instruction (calling other functions) was expensive; up to 50\%
of execution time might be spent on it, so it was common sense that big number of small function is \gls{anti-pattern}}\cite[Chapter 4, Part II]{Raymond:2003:AUP:829549}.}.
\RU{Либо это может быть полезным для тех ситуаций, когда память для стека еще не выделена либо недоступна}
\EN{It can be useful for such situations when memory for the stack is not yet allocated or not available}.

\EN{Some examples of leaf functions here are}\RU{Некоторые примеры таких ф-ций здесь}: \listingname 
\ref{ARM_leaf_example1}, \ref{ARM_leaf_example2}, 
\ref{ARM_leaf_example3}, \ref{ARM_leaf_example4}, \ref{ARM_leaf_example5},
\ref{ARM_leaf_example6}, \ref{ARM_leaf_example7}, \ref{ARM_leaf_example10}.

\subsection{\RU{Передача параметров для функции}\EN{Passing function arguments}}

\RU{Самый распространенный способ передачи параметров в x86 называется}
\EN{The most popular way to pass parameters in x86 is called} ``cdecl'':

\begin{lstlisting}
push arg3
push arg2
push arg1
call f
add esp, 4*3
\end{lstlisting}

\RU{Вызываемая функция получает свои параметры также через указатель стека.}
\EN{\Gls{callee} functions get their arguments via the stack pointer.}

\RU{Следовательно, так будут расположены значения в стеке перед исполнением самой первой инструкции
ф-ции \ttf{}:}
\EN{Therefore, this is how the argument values will be located in the stack before the execution
of the \ttf{} function's very first instruction:}

\begin{center}
\begin{tabular}{ | l | l | }
\hline
ESP & \RU{адрес возврата}\EN{return address} \\
\hline
ESP+4 & \argument \#1, \MarkedInIDAAs{} \TT{arg\_0} \\
\hline
ESP+8 & \argument \#2, \MarkedInIDAAs{} \TT{arg\_4} \\
\hline
ESP+0xC & \argument \#3, \MarkedInIDAAs{} \TT{arg\_8} \\
\hline
\dots & \dots \\
\hline
\end{tabular}
\end{center}

\RU{См. также в соответствующем разделе о других способах передачи аргументов через стек}
\EN{For more information on other calling conventions see also section}~(\myref{sec:callingconventions}).
\RU{Важно отметить, что, в общем, никто не заставляет программистов передавать параметры именно через стек,
это не является требованием к исполняемому коду.}
\EN{It is worth noting that nothing obliges programmers to pass arguments through the stack. It is not a requirement.}
\RU{Вы можете делать это совершенно иначе, не используя стек вообще.}
\EN{One could implement any other method without using the stack at all.}

\RU{К примеру, можно выделять в \glslink{heap}{куче} место для аргументов, 
заполнять их и передавать в функцию указатель на это место через \EAX. И это вполне будет работать}
\EN{For example, it is possible to allocate a space for arguments in the \gls{heap}, fill it and pass it to a function 
via a pointer to this block in the \EAX register. This will work}
\footnote{\RU{Например, в книге Дональда Кнута ``Искусство программирования'', в разделе 1.4.1 
посвященном подпрограммам\cite[раздел 1.4.1]{Knuth:1998:ACP:521463}, 
мы можем прочитать о возможности располагать параметры для вызываемой подпрограммы после инструкции \JMP,
передающей управление подпрограмме. Кнут описывает что это было особенно удобно для компьютеров IBM System/360.}
\EN{For example, in the ``The Art of Computer Programming'' book by Donald Knuth, 
in section 1.4.1 dedicated to subroutines\cite[section 1.4.1]{Knuth:1998:ACP:521463},
we could read that one way to supply arguments to a subroutine is simply to list them after the \JMP instruction
passing control to subroutine. Knuth explains that this method was particularly convenient on IBM System/360.}}.
\RU{Однако, так традиционно сложилось, что в x86 и ARM передача аргументов происходит именно через стек.}
\EN{However, it is a convenient custom in x86 and ARM to use the stack for this purpose.} \\
\\
\RU{Кстати, вызываемая ф-ция не имеет информации, сколько аргументов было ей было передано.}
\EN{By the way, the \gls{callee} function does not have any information about how many arguments were passed.}
\RU{Функции Си с переменным количеством аргументов (как \printf) определяют их количество по 
спецификаторам строки формата (начинающиеся со знака \%).}
\EN{C functions with a variable number of arguments (like \printf) determine their number using format string  specifiers (which begin with the \% symbol).}
\RU{Если написать что-то вроде}\EN{If we write something like} 

\begin{lstlisting}
printf("%d %d %d", 1234);
\end{lstlisting}

\printf \RU{выведет 1234, затем еще два случайных числа, которые волею случая оказались в стеке рядом.}
\EN{will print 1234, and then two random numbers, which were laying next to it in the stack.}\\
\\
\RU{Вот почему не так уж и важно, как объявлять ф-цию \main}
\EN{That's why it is not very important how we declare the \main function}: \RU{как}\EN{as} \main, 
\TT{main(int argc, char *argv[])} 
\RU{либо}\EN{or} \TT{main(int argc, char *argv[], char *envp[])}.

\RU{В реальности, \ac{CRT}-код вызывает \main примерно так:}
\EN{In fact, the \ac{CRT}-code is calling \main roughly as:}

\begin{lstlisting}
push envp
push argv
push argc
call main
...
\end{lstlisting}

\RU{Если вы объявляете \main как \main без аргументов, они, тем не менее, присутствуют в стеке, но не используются.}
\EN{If you declare \main as \main without arguments, they are, nevertheless, still present in the stack, but
are not used.}
\RU{Если вы объявите \main как}\EN{If you declare \main as} \TT{main(int argc, char *argv[])}, 
\RU{вы будете использовать два аргумента, а третий останется для вашей ф-ции ``невидимым''.}
\EN{you will use two arguments, and the third will remain ``invisible'' for your function.}
\RU{Более того, можно даже объявить}\EN{Even further, it is possible to declare} \TT{main(int argc)}, 
\RU{и это будет работать}\EN{and it will work}.


\subsection{\RU{Хранение локальных переменных}\EN{Local variable storage}}

\RU{Функция может выделить для себя некоторое место в стеке для локальных переменных, просто отодвинув 
\glslink{stack pointer}{указатель стека} глубже к концу стека.}
\EN{A function could allocate space in the stack for its local variables just by decreasing 
the \gls{stack pointer} towards the stack bottom.}
% I think here, "stack bottom" means the lowest address in the stack space,
% but the reader might also think it means towards the top of the stack space,
% like in a pop, so you might change "towards the stack bottom" to
% "towards the lowest address of the stack", or just take it out,
% since "decreasing" also suggests that.
\RU{Это очень быстро вне зависимости от количества локальных переменных.}
\EN{Hence, it's very fast, no matter how many local variables are defined.}

\RU{Хранить локальные переменные в стеке не является необходимым требованием. 
Вы можете хранить локальные переменные где угодно. 
Но по традиции всё сложилось так.}
\EN{It is also not a requirement to store local variables in the stack.
You could store local variables wherever you like, 
but traditionally this is how it's done.}

\subsection{x86: \IFRU{Функция alloca()}{alloca() function}}
\label{alloca}
\index{\CStandardLibrary!alloca()}
\IFRU{Интересен случай с функцией \TT{alloca()}}
{It is worth noting the \TT{alloca()} function.}\footnote{
\IFRU
{В MSVC, реализацию функции можно посмотреть в файлах}
{In MSVC, the function implementation can be found in} 
  \TT{alloca16.asm} 
  \AndENRU 
  \TT{chkstk.asm} 
  \InENRU 
  \TT{C:\textbackslash{}Program Files (x86)\textbackslash{}Microsoft Visual Studio 10.0\textbackslash{}VC\textbackslash{}crt\textbackslash{}src\textbackslash{}intel}}. 

\IFRU{Эта функция работает как \TT{malloc()}, но выделяет память прямо в стеке.} 
{This function works like \TT{malloc()} but allocates memory just on the stack.}

\IFRU{Память освобождать через \TT{free()} не нужно, так как эпилог функции~(\ref{sec:prologepilog})
вернет \ESP назад в изначальное состояние и выделенная память просто аннулируется.}
{The allocated memory chunk does not need to be freed via a \TT{free()} function call since the 
function epilogue~(\ref{sec:prologepilog}) will return \ESP back to its initial state and 
the allocated memory will be just annulled.} 

\IFRU{Интересна реализация функции \TT{alloca()}.}
{It is worth noting how \TT{alloca()} is implemented.}

\IFRU{Эта функция, если упрощенно, просто сдвигает \ESP вглубь стека 
на столько байт, сколько вам нужно и возвращает \ESP в качестве указателя на выделенный блок.}
{This function, if to simplify, just shifts \ESP downwards toward the stack bottom by the number of bytes you 
need and sets \ESP as a pointer to the \IT{allocated} block.}
\IFRU{Попробуем:}{Let's try:}

\lstinputlisting{patterns/02_stack/2_1.c}

\IFRU{(Функция \TT{\_snprintf()} работает так же, как и \printf, только вместо выдачи результата в 
stdout (т.е., на терминал или в консоль),
записывает его в буфер \TT{buf}. \puts выдает содержимое буфера \TT{buf} в stdout. Конечно, можно было бы
заменить оба этих вызова на один \printf, но мне нужно проиллюстрировать использование небольшого буфера.)}
{(\TT{\_snprintf()} function works just like \printf, but instead of dumping the result into stdout (e.g., to terminal or 
console), it writes to the \TT{buf} buffer. \puts copies \TT{buf} contents to stdout. Of course, these two
function calls might be replaced by one \printf call, but I would like to illustrate small buffer usage.)}

\subsubsection{MSVC}

\IFRU{Компилируем}{Let's compile} (MSVC 2010):

\lstinputlisting[caption=MSVC 2010]{patterns/02_stack/2_2_msvc.asm}

\index{Compiler intrinsic}
\IFRU {Единственный параметр в \TT{alloca()} передается через \EAX, а не как обычно через стек}
{The sole \TT{alloca()} argument passed via \EAX (instead of pushing into stack)}
\footnote{
\IFRU{Это потому, что alloca() ~--- это не сколько функция, сколько т.н. \IT{compiler intrinsic} (\ref{sec:compiler_intrinsic})}
{It is because alloca() is rather compiler intrinsic (\ref{compiler_intrinsic}) than usual function}.

\IFRU{Одна из причин, почему здесь нужна именно функция, а не несколько инструкций прямо в коде в том, что в реализации 
функции alloca() от \ac{MSVC}
есть также код, читающий из только что выделенной памяти, чтобы \ac{OS} подключила физическую память к этому региону \ac{VM}.}
{One of the reason there is a separate function instead of couple instructions just in the code,
because \ac{MSVC} implementation
of the alloca() function also has a code which reads from the memory just allocated, in order to let \ac{OS} to map
physical memory to this \ac{VM} region.}
}.
\IFRU{После вызова \TT{alloca()}, \ESP теперь указывает на блок в 600 байт, который 
мы можем использовать под \TT{buf}.}
{After the \TT{alloca()} call, \ESP points to the block of 600 bytes and we can 
use it as memory for the \TT{buf} array.}

\subsubsection{GCC + \IntelSyntax}

\IFRU{А GCC 4.4.1 обходится без вызова других функций:}
{GCC 4.4.1 can do the same without calling external functions:}

\lstinputlisting[caption=GCC 4.7.3]{patterns/02_stack/2_1_gcc_intel_O3_\LANG.asm}

\subsubsection{GCC + \ATTSyntax}

\IFRU{Посмотрим на тот же код, только в синтаксисе AT\&T}{Let's see the same code, but in AT\&T syntax}:

\lstinputlisting[caption=GCC 4.7.3]{patterns/02_stack/2_1_gcc_ATT_O3.s}

\index{\ATTSyntax}
\IFRU{Всё то же самое, что и в прошлом листинге.}{The code is the same as in the previous listing.}

N.B. \IFRU{Например,}{E.g.} \TT{movl \$3, 20(\%esp)} 
\IFRU{ ~--- это аналог}{is analogous to} \TT{mov DWORD PTR [esp+20], 3} \IFRU{в Intel-синтаксисе:}
{in Intel-syntax}\IFRU{ при адресации памяти в виде}{~---when addressing memory in form} \IT{\IFRU{регистр+смещение}{register+offset}}, 
\IFRU{это записывается в AT\&T синтаксисе как}{it is written in AT\&T syntax as} 
\TT{\IFRU{смещение}{offset}(\%\IFRU{регистр}{register})}.


\subsubsection{(Windows) SEH}
\myindex{Windows!Structured Exception Handling}

\ifdefined\RUSSIAN
В стеке хранятся записи \ac{SEH} для функции (если они присутствуют).
Читайте больше о нем здесь: (\myref{sec:SEH}).
\fi % RUSSIAN

\ifdefined\ENGLISH
\ac{SEH} records are also stored on the stack (if they are present).
Read more about it: (\myref{sec:SEH}).
\fi % ENGLISH

\ifdefined\BRAZILIAN
\ac{SEH} também são guardados na pilha (se estiverem presentes).
\PTBRph{}: (\myref{sec:SEH}).
\fi % BRAZILIAN

\ifdefined\ITALIAN
I record \ac{SEH}, se presenti, sono anch'essi memorizzati nello stack.
Maggiori informazioni qui: (\myref{sec:SEH}).
\fi % ITALIAN

\subsection{\RU{Защита от переполнений буфера}\EN{Buffer overflow protection}\PTBR{Proteção contra estouro de buffer}}

\RU{Здесь больше об этом}\EN{More about it here}\PTBR{Mais sobre aqui}~(\myref{subsec:bufferoverflow}).



\subsection{\IFRU{Разметка типичного стека}{Typical stack layout}}

\IFRU{Разметка типичного стека в 32-битной среде на момент начала ф-ции выглядит так}
{A very typical stack layout in a 32-bit environment at the start of a function}:

\begin{center}
\begin{tabular}{ | l | l | }
\hline
\dots & \dots \\
\hline
ESP-0xC & \IFRU{локальная переменная}{local variable} \#2, \MarkedInIDAAs{} \TT{var\_8} \\
\hline
ESP-8 & \IFRU{локальная переменная}{local variable} \#1, \MarkedInIDAAs{} \TT{var\_4} \\
\hline
ESP-4 & \IFRU{сохраненное значение}{saved value of} \EBP \\
\hline
ESP & \IFRU{адрес возврата}{return address} \\
\hline
ESP+4 & \argument \#1, \MarkedInIDAAs{} \TT{arg\_0} \\
\hline
ESP+8 & \argument \#2, \MarkedInIDAAs{} \TT{arg\_4} \\
\hline
ESP+0xC & \argument \#3, \MarkedInIDAAs{} \TT{arg\_8} \\
\hline
\dots & \dots \\
\hline
\end{tabular}
\end{center}

