\subsection{ARM: \RU{3 аргумента}\EN{3 arguments}}

\RU{В ARM традиционно принята такая схема передачи аргументов в функцию: 
4 первых аргумента через регистры \Reg{0}-\Reg{3}; а остальные~--- через стек}
\EN{ARM's traditional scheme for passing arguments (calling convention) behaves as follows:
the first 4 arguments are passed through the \Reg{0}-\Reg{3} registers; the remaining arguments via the stack}.
\RU{Это немного похоже на то, как аргументы передаются в}\EN{This resembles the arguments passing scheme in} 
fastcall~(\myref{fastcall}) \OrENRU win64~(\myref{sec:callingconventions_win64}).

\subsubsection{32-\RU{битный}\EN{bit} ARM}

\myparagraph{\NonOptimizingKeilVI (\ARMMode)}

\begin{lstlisting}[caption=\NonOptimizingKeilVI (\ARMMode)]
.text:00000000 main
.text:00000000 10 40 2D E9   STMFD   SP!, {R4,LR}
.text:00000004 03 30 A0 E3   MOV     R3, #3
.text:00000008 02 20 A0 E3   MOV     R2, #2
.text:0000000C 01 10 A0 E3   MOV     R1, #1
.text:00000010 08 00 8F E2   ADR     R0, aADBDCD     ; "a=%d; b=%d; c=%d"
.text:00000014 06 00 00 EB   BL      __2printf
.text:00000018 00 00 A0 E3   MOV     R0, #0          ; return 0
.text:0000001C 10 80 BD E8   LDMFD   SP!, {R4,PC}
\end{lstlisting}

\RU{Итак, первые 4 аргумента передаются через регистры \Reg{0}-\Reg{3}, по порядку: 
указатель на формат-строку для \printf
в \Reg{0}, затем 1 в \Reg{1}, 2 в \Reg{2} и 3 в \Reg{3}.}
\EN{So, the first 4 arguments are passed via the \Reg{0}-\Reg{3} registers in this order:
a pointer to the \printf format string in 
\Reg{0}, then 1 in \Reg{1}, 2 in \Reg{2} and 3 in \Reg{3}.}

\RU{Инструкция на}\EN{The instruction at} \TT{0x18} \RU{записывает}\EN{writes} 0 \RU{в}\EN{to} \Reg{0}%
\EMDASH{}\RU{это выражение в Си}\EN{this is} \IT{return 0}\EN{ C-statement}.

\RU{Пока что здесь нет ничего необычного}\EN{There is nothing unusual so far}.

\OptimizingKeilVI \RU{генерирует точно такой же код}\EN{generates the same code}.

\myparagraph{\OptimizingKeilVI (\ThumbMode)}

\begin{lstlisting}[caption=\OptimizingKeilVI (\ThumbMode)]
.text:00000000 main
.text:00000000 10 B5        PUSH    {R4,LR}
.text:00000002 03 23        MOVS    R3, #3
.text:00000004 02 22        MOVS    R2, #2
.text:00000006 01 21        MOVS    R1, #1
.text:00000008 02 A0        ADR     R0, aADBDCD     ; "a=%d; b=%d; c=%d"
.text:0000000A 00 F0 0D F8  BL      __2printf
.text:0000000E 00 20        MOVS    R0, #0
.text:00000010 10 BD        POP     {R4,PC}
\end{lstlisting}

\RU{Здесь нет особых отличий от неоптимизированного варианта для режима ARM.}
\EN{There is no significant difference from the non-optimized code for ARM mode.}

\myparagraph{\OptimizingKeilVI (\ARMMode) + \RU{убираем}\EN{let's remove} return}
\label{ARM_B_to_printf}

\RU{Немного переделаем пример, убрав}\EN{Let's rework example slightly by removing} \IT{return 0}:

\begin{lstlisting}
#include <stdio.h>

void main()
{
	printf("a=%d; b=%d; c=%d", 1, 2, 3);
};
\end{lstlisting}

\RU{Результат получится необычным:}
\EN{The result is somewhat unusual:}

\begin{lstlisting}[caption=\OptimizingKeilVI (\ARMMode)]
.text:00000014 main
.text:00000014 03 30 A0 E3   MOV     R3, #3
.text:00000018 02 20 A0 E3   MOV     R2, #2
.text:0000001C 01 10 A0 E3   MOV     R1, #1
.text:00000020 1E 0E 8F E2   ADR     R0, aADBDCD     ; "a=%d; b=%d; c=%d\n"
.text:00000024 CB 18 00 EA   B       __2printf
\end{lstlisting}

\index{ARM!\Registers!Link Register}
\index{ARM!\Instructions!B}
\index{Function epilogue}
\RU{Это оптимизированная версия (\Othree) для режима ARM, и здесь мы видим последнюю инструкцию 
\TT{B} вместо привычной нам \TT{BL}}\EN{This is the optimized (\Othree) version for ARM mode and this time we see \TT{B} as the last instruction instead of the familiar \TT{BL}}.
\RU{Отличия между этой оптимизированной версией и предыдущей, скомпилированной без оптимизации, 
ещё и в том, 
что здесь нет пролога и эпилога функции (инструкций, сохраняющих состояние регистров \TT{\Reg{0}} и \ac{LR})}%
\EN{Another difference between this optimized version and the previous one (compiled without optimization)
is the lack of function prologue and epilogue (instructions preserving the \TT{\Reg{0}} and \ac{LR} registers values)}.
\index{x86!\Instructions!JMP}
\RU{Инструкция \TT{B} просто переходит на другой адрес, без манипуляций с регистром \ac{LR}, то есть
это аналог \JMP в x86}%
\EN{The \TT{B} instruction just jumps to another address, without any manipulation of the \ac{LR} register,
similar to \JMP in x86}.
\RU{Почему это работает нормально? Потому что этот код эквивалентен предыдущему.}
\EN{Why does it work? Because this code is, in fact, effectively equivalent to the previous.}
\RU{Основных причин две: 1) стек не модифицируется, как и \glslink{stack pointer}{указатель стека} \ac{SP}; 2) вызов функции \printf последний, 
после него ничего не происходит}\EN{There are two main reasons: 1) neither the stack nor \ac{SP} (the \gls{stack pointer}) is modified;
2) the call to \printf is the last instruction, so there is nothing going on afterwards}.
\RU{Функция \printf, отработав, просто возвращает управление по адресу, записанному в \ac{LR}.}
\EN{On completion, the \printf function simply returns the control to the address 
stored in \ac{LR}.}
\RU{Но в \ac{LR} находится адрес места, откуда была вызвана наша функция!
А следовательно, управление из \printf вернется сразу туда.}
\EN{Since the \ac{LR} currently stores the address of the point from where our function
was called then the control from \printf will be returned to that point.}
\RU{Значит нет нужды сохранять \ac{LR}, потому что нет нужны модифицировать \ac{LR}}%
\EN{Therefore we do not need to save \ac{LR} because we do not need to modify \ac{LR}}.
\RU{А нет нужды модифицировать \ac{LR}, потому что нет иных вызовов функций, кроме \printf, к тому же, после этого вызова не нужно ничего здесь больше делать}%
\EN{And we do not need to modify \ac{LR} because there are no other function calls except \printf. Furthermore,
after this call we do not to do anything else}!
\RU{Поэтому такая оптимизация возможна}\EN{That is the reason such optimization is possible}.

\RU{Эта оптимизация часто используется в функциях, где последнее выражение~--- это вызов другой функции.}
\EN{This optimization is often used in functions where the last statement is a call to another function.}

\RU{Ещё один похожий пример описан здесь}\EN{A similar example is presented here}:
\myref{jump_to_last_printf}.

\subsubsection{ARM64}

\myparagraph{\NonOptimizing GCC (Linaro) 4.9}

\lstinputlisting[caption=\NonOptimizing GCC (Linaro) 4.9]{patterns/03_printf/ARM/ARM3_O0.lst.\LANG}

\index{ARM!\Instructions!STP}
\RU{Итак, первая инструкция STP (Store Pair) сохраняет \ac{FP} (X29) и \ac{LR} (X30) в стеке.}
\EN{The first instruction STP (Store Pair) saves \ac{FP} (X29) and \ac{LR} (X30) in the stack.}
\RU{Вторая инструкция \TT{ADD X29, SP, 0} формирует стековый фрейм.}
\EN{The second \TT{ADD X29, SP, 0} instruction forms the stack frame.}
\RU{Это просто запись значения \ac{SP} в X29.}
\EN{It is just writing the value of \ac{SP} into X29.}

\index{ARM!\Instructions!ADRP/ADD pair}
\RU{Далее уже знакомая пара инструкций \TT{ADRP}/\ADD формирует указатель на строку.}%
\EN{Next, we see the familiar \TT{ADRP}/\ADD instruction pair, which forms a pointer to the string.}
\index{ARM64!lo12}
\RU{\IT{lo12} означает младшие 12 бит, т.е., линкер запишет младшие 12 бит адреса метки LC1 в опкод инструкции \ADD.}%
\EN{\IT{lo12} meaning low 12 bits, i.e., linker will write low 12 bits of LC1 address into the opcode of \ADD instruction.}

\RU{\TT{\%d} в формате \printf это 32-битный \Tint, так что 1, 2 и 3 заносятся в 32-битные части регистров.}
\EN{\TT{\%d} in \printf string format is a 32-bit \Tint, so the 1, 2 and 3 are loaded into 32-bit register parts.}

\Optimizing GCC (Linaro) 4.9 \RU{генерирует почти такой же код}\EN{generates the same code}.
