\subsection{ARM: 3 arguments}

ARM's traditional scheme for passing arguments (calling convention) behaves as follows:
the first 4 arguments are passed through the \Reg{0}-\Reg{3} registers; the remaining arguments via the stack.
This resembles the arguments passing scheme in 
fastcall~(\myref{fastcall}) or win64~(\myref{sec:callingconventions_win64}).

\subsubsection{32-bit ARM}

\myparagraph{\NonOptimizingKeilVI (\ARMMode)}

\begin{lstlisting}[caption=\NonOptimizingKeilVI (\ARMMode)]
.text:00000000 main
.text:00000000 10 40 2D E9   STMFD   SP!, {R4,LR}
.text:00000004 03 30 A0 E3   MOV     R3, #3
.text:00000008 02 20 A0 E3   MOV     R2, #2
.text:0000000C 01 10 A0 E3   MOV     R1, #1
.text:00000010 08 00 8F E2   ADR     R0, aADBDCD     ; "a=%d; b=%d; c=%d"
.text:00000014 06 00 00 EB   BL      __2printf
.text:00000018 00 00 A0 E3   MOV     R0, #0          ; return 0
.text:0000001C 10 80 BD E8   LDMFD   SP!, {R4,PC}
\end{lstlisting}

So, the first 4 arguments are passed via the \Reg{0}-\Reg{3} registers in this order:
a pointer to the \printf format string in 
\Reg{0}, then 1 in \Reg{1}, 2 in \Reg{2} and 3 in \Reg{3}.
The instruction at \GTT{0x18} writes 0 to \Reg{0}---this is \IT{return 0} C-statement.
There is nothing unusual so far.

\OptimizingKeilVI generates the same code.

\myparagraph{\OptimizingKeilVI (\ThumbMode)}

\begin{lstlisting}[caption=\OptimizingKeilVI (\ThumbMode)]
.text:00000000 main
.text:00000000 10 B5        PUSH    {R4,LR}
.text:00000002 03 23        MOVS    R3, #3
.text:00000004 02 22        MOVS    R2, #2
.text:00000006 01 21        MOVS    R1, #1
.text:00000008 02 A0        ADR     R0, aADBDCD     ; "a=%d; b=%d; c=%d"
.text:0000000A 00 F0 0D F8  BL      __2printf
.text:0000000E 00 20        MOVS    R0, #0
.text:00000010 10 BD        POP     {R4,PC}
\end{lstlisting}

There is no significant difference from the non-optimized code for ARM mode.

\myparagraph{\OptimizingKeilVI (\ARMMode) + let's remove return}
\label{ARM_B_to_printf}

Let's rework example slightly by removing \IT{return 0}:

\begin{lstlisting}
#include <stdio.h>

void main()
{
	printf("a=%d; b=%d; c=%d", 1, 2, 3);
};
\end{lstlisting}

The result is somewhat unusual:

\begin{lstlisting}[caption=\OptimizingKeilVI (\ARMMode)]
.text:00000014 main
.text:00000014 03 30 A0 E3   MOV     R3, #3
.text:00000018 02 20 A0 E3   MOV     R2, #2
.text:0000001C 01 10 A0 E3   MOV     R1, #1
.text:00000020 1E 0E 8F E2   ADR     R0, aADBDCD     ; "a=%d; b=%d; c=%d\n"
.text:00000024 CB 18 00 EA   B       __2printf
\end{lstlisting}

\myindex{ARM!\Registers!Link Register}
\myindex{ARM!\Instructions!B}
\myindex{Function epilogue}
This is the optimized (\Othree) version for ARM mode and this time we see \INS{B} as the last instruction instead of the familiar \INS{BL}.
Another difference between this optimized version and the previous one (compiled without optimization)
is the lack of function prologue and epilogue (instructions preserving the \Reg{0} and \ac{LR} registers values).
\myindex{x86!\Instructions!JMP}
The \INS{B} instruction just jumps to another address, without any manipulation of the \ac{LR} register,
similar to \JMP in x86.
Why does it work? Because this code is, in fact, effectively equivalent to the previous.
There are two main reasons: 1) neither the stack nor \ac{SP} (the \gls{stack pointer}) is modified;
2) the call to \printf is the last instruction, so there is nothing going on afterwards.
On completion, the \printf function simply returns the control to the address 
stored in \ac{LR}.
Since the \ac{LR} currently stores the address of the point from where our function
was called then the control from \printf will be returned to that point.
Therefore we do not need to save \ac{LR} because we do not need to modify \ac{LR}.
And we do not need to modify \ac{LR} because there are no other function calls except \printf. Furthermore,
after this call we do not to do anything else!
That is the reason such optimization is possible.

This optimization is often used in functions where the last statement is a call to another function.
A similar example is presented here:
\myref{jump_to_last_printf}.

\subsubsection{ARM64}

\myparagraph{\NonOptimizing GCC (Linaro) 4.9}

\lstinputlisting[caption=\NonOptimizing GCC (Linaro) 4.9]{patterns/03_printf/ARM/ARM3_O0_EN.lst}

\myindex{ARM!\Instructions!STP}

The first instruction \INS{STP} (\IT{Store Pair}) saves \ac{FP} (X29) and \ac{LR} (X30) in the stack.
The second \INS{ADD X29, SP, 0} instruction forms the stack frame.
It is just writing the value of \ac{SP} into X29.

\myindex{ARM!\Instructions!ADRP/ADD pair}
Next, we see the familiar \INS{ADRP}/\ADD instruction pair, which forms a pointer to the string.
\myindex{ARM64!lo12}
\IT{lo12} meaning low 12 bits, i.e., linker will write low 12 bits of LC1 address into the opcode of \ADD instruction.
\GTT{\%d} in \printf string format is a 32-bit \Tint, so the 1, 2 and 3 are loaded into 32-bit register parts.

\Optimizing GCC (Linaro) 4.9 generates the same code.

