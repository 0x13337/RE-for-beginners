\subsection{ARM: 8 arguments}

Usiamo nuovamente l'esempio con 9 argomenti della sezione precedente: \myref{example_printf8_x64}.

\lstinputlisting{patterns/03_printf/2.c}

\subsubsection{\OptimizingKeilVI: \ARMMode}

\begin{lstlisting}
.text:00000028             main
.text:00000028
.text:00000028             var_18 = -0x18
.text:00000028             var_14 = -0x14
.text:00000028             var_4  = -4
.text:00000028
.text:00000028 04 E0 2D E5  STR    LR, [SP,#var_4]!
.text:0000002C 14 D0 4D E2  SUB    SP, SP, #0x14
.text:00000030 08 30 A0 E3  MOV    R3, #8
.text:00000034 07 20 A0 E3  MOV    R2, #7
.text:00000038 06 10 A0 E3  MOV    R1, #6
.text:0000003C 05 00 A0 E3  MOV    R0, #5
.text:00000040 04 C0 8D E2  ADD    R12, SP, #0x18+var_14
.text:00000044 0F 00 8C E8  STMIA  R12, {R0-R3}
.text:00000048 04 00 A0 E3  MOV    R0, #4
.text:0000004C 00 00 8D E5  STR    R0, [SP,#0x18+var_18]
.text:00000050 03 30 A0 E3  MOV    R3, #3
.text:00000054 02 20 A0 E3  MOV    R2, #2
.text:00000058 01 10 A0 E3  MOV    R1, #1
.text:0000005C 6E 0F 8F E2  ADR    R0, aADBDCDDDEDFDGD ; "a=%d; b=%d; c=%d; d=%d; e=%d; f=%d; g=%"...
.text:00000060 BC 18 00 EB  BL     __2printf
.text:00000064 14 D0 8D E2  ADD    SP, SP, #0x14
.text:00000068 04 F0 9D E4  LDR    PC, [SP+4+var_4],#4
\end{lstlisting}

Il codice puo' essre diviso in piu' parti

\myindex{Function prologue}
\begin{itemize}
\item Preambolo della funzione:

\myindex{ARM!\Instructions!STR}
La prima istruzione \INS{STR LR, [SP,\#var\_4]!} salva \ac{LR} sullo stack, poiche' questo registro sara' usato per la chiamata a \printf.
Il punto esclamativo all fine indica il \IT{pre-index}.

Questo implica che \ac{SP} deve essere prima decrementato di 4, e successivamente \ac{LR} sara' salvato all'indirizzo memorizzato in \ac{SP}.
Tutto cio' e' simile a \PUSH in x86.
Maggiori informazioni qui: \myref{ARM_postindex_vs_preindex}.

\myindex{ARM!\Instructions!SUB}
La seconda istruzione \INS{SUB SP, SP, \#0x14} decrementa \ac{SP} (lo \gls{stack pointer}) per allocare \GTT{0x14} (20) byte sullo stack.
Infatti dobbiamo passare 5 valori a 32-bit tramite lo stack per la funzione \printf, e ciascuno di essi occupa 4 byte, che e' esattamente $5*4=20$.
Gli altri 4 valori a 32-bit saranno passati tramite registri.

\item Passaggio di 5, 6, 7 e 8 tramite lo stack: sono memorizzati nei registri \Reg{0}, \Reg{1}, \Reg{2} e \Reg{3}, rispettivamente.
Successivamente l'istruzione \INS{ADD R12, SP, \#0x18+var\_14} scrive l'indirizzo dello stack, dove queste 4 variabili saranno memorizzate,
nel registri \Reg{12}.
\myindex{IDA!var\_?}
\IT{var\_14} e' una macro assembly, uguale a -0x14, creata da \IDA per visualizzare in maniera conveniente il codice che accede allo stack.
Le macro \IT{var\_?} generate da \IDA riflettono le variabili locali nello stack.

Quindi, \GTT{SP+4} sara' memorizzato nel registro \Reg{12}.
\myindex{ARM!\Instructions!STMIA}
L'istruzione successiva \INS{STMIA R12, {R0-R3}} scrive il contenuto dei registri \Reg{0}-\Reg{3} alla memoria puntata da \Reg{12}.
\INS{STMIA} e' abbreviazione per \IT{Store Multiple Increment After}. 
\IT{\q{Increment After}} (incrementa dopo) implica che \Reg{12} deve essere incrementato di 4 dopo ciascuna scrittura di un valore nei registri.

\item Passaggio di 4 tramite lo stack: 4 e' memorizzato in \Reg{0} e questo valore, con l'aiuto dell'istruzione \INS{STR R0, [SP,\#0x18+var\_18]}, viene salvato sullo stack.
\IT{var\_18} e' -0x18, quindi l'offset deve essere 0, da cui il valore dal registro \Reg{0} (4) sara' scritto all'indirizzo memorizzato in \ac{SP}.

\item Passaggio di 1, 2 e 3 tramite registri:
I valori dei primi 3 numeri (a, b, c) (1, 2, 3 rispettivamente) sono passatti attraverso i registri
\Reg{1}, \Reg{2} e \Reg{3}
poco prima della chiamata a \printf call, e gli altri 5 valori sono passati tramite lo stack:

\item chiamata a \printf.

\myindex{Function epilogue}
\item epilogo della funzione:


L'istruzione \INS{ADD SP, SP, \#0x14} ripristina il puntatore \ac{SP} al suo valore precedente, pulendo quindi lo stack.
Ovviamente quello che era stato memorizzato nello stack rimarra li', e sara' probabilmente riscritto interamente durante l'e
secuzione delle funzioni seguenti.

\myindex{ARM!\Instructions!LDR}
L'istruzione \INS{LDR PC, [SP+4+var\_4],\#4} carica il valore di \ac{LR} salvato dallo stack nel nel registro \ac{PC}, causando quindi
l'uscita dalla funzione.
Non c'e' il punto esclamativo ---infatti \ac{PC} e' caricato prima dall'indirizzo memorizzato in \ac{SP} ($4+var\_4=4+(-4)=0$, questa istruzione e' quindi analoga a \INS{LDR PC, [SP],\#4}), e successivamente \ac{SP} e' incrementato di 4.
Questo e' detto \IT{post-index}\footnote{Maggiori dettagli: \myref{ARM_postindex_vs_preindex}.}.
Perche' \IDA mostra l'istruzione in quel modo?
Perche' vuole illustrare il layout dello stack ed il fatto che \GTT{var\_4} e' allocata per salvare il valore di \ac{LR} nello stack locale.
Questa istruzione e' piu' o meno simile a \INS{POP PC} in x86\footnote{E' impossibile settare il valore di \GTT{IP/EIP/RIP} usando \POP in x86, ma in ogni caso hai capito l'analogia.}.

\end{itemize}

\subsubsection{\OptimizingKeilVI: \ThumbMode}

\begin{lstlisting}
.text:0000001C             printf_main2
.text:0000001C
.text:0000001C             var_18 = -0x18
.text:0000001C             var_14 = -0x14
.text:0000001C             var_8  = -8
.text:0000001C
.text:0000001C 00 B5        PUSH    {LR}
.text:0000001E 08 23        MOVS    R3, #8
.text:00000020 85 B0        SUB     SP, SP, #0x14
.text:00000022 04 93        STR     R3, [SP,#0x18+var_8]
.text:00000024 07 22        MOVS    R2, #7
.text:00000026 06 21        MOVS    R1, #6
.text:00000028 05 20        MOVS    R0, #5
.text:0000002A 01 AB        ADD     R3, SP, #0x18+var_14
.text:0000002C 07 C3        STMIA   R3!, {R0-R2}
.text:0000002E 04 20        MOVS    R0, #4
.text:00000030 00 90        STR     R0, [SP,#0x18+var_18]
.text:00000032 03 23        MOVS    R3, #3
.text:00000034 02 22        MOVS    R2, #2
.text:00000036 01 21        MOVS    R1, #1
.text:00000038 A0 A0        ADR     R0, aADBDCDDDEDFDGD ; "a=%d; b=%d; c=%d; d=%d; e=%d; f=%d; g=%"...
.text:0000003A 06 F0 D9 F8  BL      __2printf
.text:0000003E
.text:0000003E             loc_3E   ; CODE XREF: example13_f+16
.text:0000003E 05 B0        ADD     SP, SP, #0x14
.text:00000040 00 BD        POP     {PC}
\end{lstlisting}

L'output e' quasi identico al precedente esempio. Tuttavia questo e' codice Thumb e i valori sono disposti nello stack in modo differente:
8 per primo, quindi 5, 6, 7, e infine 4.

\subsubsection{\OptimizingXcodeIV: \ARMMode}

\begin{lstlisting}
__text:0000290C             _printf_main2
__text:0000290C
__text:0000290C             var_1C = -0x1C
__text:0000290C             var_C  = -0xC
__text:0000290C
__text:0000290C 80 40 2D E9   STMFD  SP!, {R7,LR}
__text:00002910 0D 70 A0 E1   MOV    R7, SP
__text:00002914 14 D0 4D E2   SUB    SP, SP, #0x14
__text:00002918 70 05 01 E3   MOV    R0, #0x1570
__text:0000291C 07 C0 A0 E3   MOV    R12, #7
__text:00002920 00 00 40 E3   MOVT   R0, #0
__text:00002924 04 20 A0 E3   MOV    R2, #4
__text:00002928 00 00 8F E0   ADD    R0, PC, R0
__text:0000292C 06 30 A0 E3   MOV    R3, #6
__text:00002930 05 10 A0 E3   MOV    R1, #5
__text:00002934 00 20 8D E5   STR    R2, [SP,#0x1C+var_1C]
__text:00002938 0A 10 8D E9   STMFA  SP, {R1,R3,R12}
__text:0000293C 08 90 A0 E3   MOV    R9, #8
__text:00002940 01 10 A0 E3   MOV    R1, #1
__text:00002944 02 20 A0 E3   MOV    R2, #2
__text:00002948 03 30 A0 E3   MOV    R3, #3
__text:0000294C 10 90 8D E5   STR    R9, [SP,#0x1C+var_C]
__text:00002950 A4 05 00 EB   BL     _printf
__text:00002954 07 D0 A0 E1   MOV    SP, R7
__text:00002958 80 80 BD E8   LDMFD  SP!, {R7,PC}
\end{lstlisting}

\myindex{ARM!\Instructions!STMFA}
\myindex{ARM!\Instructions!STMIB}
Quasi lo stesso codice visto prima, ad eccezione dell'istruzione \INS{STMFA} (Store Multiple Full Ascending),
che e' sinonimo di \INS{STMIB} (Store Multiple Increment Before). 
Questa istruzione incrementa il valore nel registro \ac{SP} e solo successivamente scrive il prossimo valore del registro in memoria, invece che operare le due azioni in ordine inverso.

Un'altra cosa che salta all'occhio e' che le istruzioni sono disponste in maniera apparentemente casuale.
Ad esempio, il valore nel registro \Reg{0} e' manipolato in tre posti diversi
agli indirizzi \GTT{0x2918}, \GTT{0x2920} e \GTT{0x2928}, quando invece sarebbe stato possibile farlo in un punto solo.

Ad ogni modo, il compilatore ottimizzante avra' avuto le sue ragioni per ordinare le istruzioni in questa maniera ed ottenere una maggiore efficacia durante l'esecuzione del codice.

Solitamente il processero prova ad eseguire simultaneamentele istruzioni vicine.
Ad esempio, istruzioni come \INS{MOVT R0, \#0} e
\INS{ADD R0, PC, R0} non possono essere eseguite simultaneamente poiche' entrambe modificano il registro \Reg{0}. 
D'altra parte, \INS{MOVT R0, \#0} e \INS{MOV R2, \#4} 
possono invece essere eseguite simultaneamente poiche' l'effetto della loro esecuzione non genera conflitti tra loro.
Presumibilmente, il compilatore prova a generare codice in questo modo (quando possibile).
 
\subsubsection{\OptimizingXcodeIV: \ThumbTwoMode}

\begin{lstlisting}
__text:00002BA0               _printf_main2
__text:00002BA0
__text:00002BA0               var_1C = -0x1C
__text:00002BA0               var_18 = -0x18
__text:00002BA0               var_C  = -0xC
__text:00002BA0
__text:00002BA0 80 B5          PUSH     {R7,LR}
__text:00002BA2 6F 46          MOV      R7, SP
__text:00002BA4 85 B0          SUB      SP, SP, #0x14
__text:00002BA6 41 F2 D8 20    MOVW     R0, #0x12D8
__text:00002BAA 4F F0 07 0C    MOV.W    R12, #7
__text:00002BAE C0 F2 00 00    MOVT.W   R0, #0
__text:00002BB2 04 22          MOVS     R2, #4
__text:00002BB4 78 44          ADD      R0, PC  ; char *
__text:00002BB6 06 23          MOVS     R3, #6
__text:00002BB8 05 21          MOVS     R1, #5
__text:00002BBA 0D F1 04 0E    ADD.W    LR, SP, #0x1C+var_18
__text:00002BBE 00 92          STR      R2, [SP,#0x1C+var_1C]
__text:00002BC0 4F F0 08 09    MOV.W    R9, #8
__text:00002BC4 8E E8 0A 10    STMIA.W  LR, {R1,R3,R12}
__text:00002BC8 01 21          MOVS     R1, #1
__text:00002BCA 02 22          MOVS     R2, #2
__text:00002BCC 03 23          MOVS     R3, #3
__text:00002BCE CD F8 10 90    STR.W    R9, [SP,#0x1C+var_C]
__text:00002BD2 01 F0 0A EA    BLX      _printf
__text:00002BD6 05 B0          ADD      SP, SP, #0x14
__text:00002BD8 80 BD          POP      {R7,PC}
\end{lstlisting}

L'output e' quasi lo stesso dell'esempio precedente, ad eccezione dell'uso di istruzioni Thumb. 
% FIXME: also STMIA is used instead of STMIB,
% which is why it uses LR, which is 4 bytes ahead of SP

\subsubsection{ARM64}

\myparagraph{\NonOptimizing GCC (Linaro) 4.9}

% TODO translate to Italian:
\lstinputlisting[caption=\NonOptimizing GCC (Linaro) 4.9]{patterns/03_printf/ARM/ARM8_O0_EN.lst}

I primi 8 argomenti sono passati nei registri X- o W-: \ARMPCS.
Un puntatore ad una string richiede un registro a 64-bit, quindi e' passato in \RegX{0}.
Tutti gli altri valori hanno timpo \Tint a 32-bit, quindi sono memorizzati nella parte a 32-bit dei registri (W-).
Il nono argomento (8) e' passato tramite lo stack.
Infatti non e' possibile passare un grande numero di argomenti tramite registri, in quanto il loro numero e' limitato.

\Optimizing GCC (Linaro) 4.9 genera lo stesso codice.
