\section{ARM: \IFRU{3 аргумента}{3 arguments}}

\IFRU{В ARM традиционно принята такая схема передачи аргументов в функцию: 
4 первых аргумента через регистры \Reg{0}-\Reg{3}; а остальные ~--- через стек}
{Traditionally, ARM's scheme for passing arguments (calling convention) is as follows:
the first 4 arguments are passed in the \Reg{0}-\Reg{3} registers; the remaining arguments, via the stack}.
\IFRU{Это немного похоже на то, как аргументы передаются в}{This resembles the arguments passing scheme in} 
fastcall~(\ref{fastcall}) \OrENRU win64~(\ref{sec:callingconventions_win64}).

\subsection{\NonOptimizingKeil + \ARMMode}

\begin{lstlisting}[caption=\NonOptimizingKeil + \ARMMode]
.text:00000014             printf_main1
.text:00000014 10 40 2D E9                 STMFD   SP!, {R4,LR}
.text:00000018 03 30 A0 E3                 MOV     R3, #3
.text:0000001C 02 20 A0 E3                 MOV     R2, #2
.text:00000020 01 10 A0 E3                 MOV     R1, #1
.text:00000024 1D 0E 8F E2                 ADR     R0, aADBDCD     ; "a=%d; b=%d; c=%d\n"
.text:00000028 0D 19 00 EB                 BL      __2printf
.text:0000002C 10 80 BD E8                 LDMFD   SP!, {R4,PC}
\end{lstlisting}

\IFRU{Итак, первые 4 аргумента передаются через регистры \Reg{0}-\Reg{3}, по порядку: 
указатель на формат-строку для \printf
в \Reg{0}, затем $1$ в \Reg{1}, $2$ в \Reg{2} и $3$ в \Reg{3}}
{So, the first 4 arguments are passed via the \Reg{0}-\Reg{3} registers in this order:
a pointer to the \printf format string in 
\Reg{0}, then $1$ in \Reg{1}, $2$ in \Reg{2} and $3$ in \Reg{3}}.

\IFRU{Пока что здесь нет ничего необычного}{There is nothing unusual so far}.

\subsection{\OptimizingKeil + \ARMMode}
\label{ARM_B_to_printf}

\begin{lstlisting}[caption=\OptimizingKeil + \ARMMode]
.text:00000014                             EXPORT printf_main1
.text:00000014             printf_main1
.text:00000014 03 30 A0 E3                 MOV     R3, #3
.text:00000018 02 20 A0 E3                 MOV     R2, #2
.text:0000001C 01 10 A0 E3                 MOV     R1, #1
.text:00000020 1E 0E 8F E2                 ADR     R0, aADBDCD     ; "a=%d; b=%d; c=%d\n"
.text:00000024 CB 18 00 EA                 B       __2printf
\end{lstlisting}

\index{ARM!\Registers!Link Register}
\index{ARM!\Instructions!B}
\index{Function epilogue}
\IFRU{Это соптимизированная версия (\Othree) для режима ARM, и здесь мы видим последнюю инструкцию: 
\TT{B} вместо привычной нам \TT{BL}}{This is optimized (\Othree) version for ARM mode and here we see \TT{B} as
the last instruction instead of the familiar \TT{BL}}.
\IFRU{Отличия между этой соптимизированной версией и предыдущей, скомпилированной без оптимизации, 
еще и в том, 
что здесь нет пролога и эпилога функции (инструкций, сохраняющих состояние регистров \TT{\Reg{0}} и \LR)}
{Another difference between this optimized version and the previous one (compiled without optimization)
is also in the
fact that there is no function prologue and epilogue (instructions that save \TT{\Reg{0}} and \LR registers values)}.
\index{x86!\Instructions!JMP}
\IFRU{Инструкция \TT{B} просто переходит на другой адрес, без манипуляций с регистром \LR, то есть
это аналог \JMP в x86}
{The \TT{B} instruction just jumps to another address, without any manipulation of the \LR register,
that is, it is analogous to \JMP in x86}.
\IFRU{Почему это работает нормально? Потому что этот код эквивалентен предыдущему.}
{Why does it work? Because this code is, in fact, effectively equivalent to the previous.}
\IFRU{Основных причин две: 1) стек не модифицируется, как и \glslink{stack pointer}{указатель стека} \SP; 2) вызов функции \printf последний, 
после него ничего не происходит}{There are two main reasons: 1) neither the stack nor \SP, the \gls{stack pointer}, is modified;
2) the call to \printf is the last instruction, so there is nothing going on after it}.
\IFRU{Функция \printf, отработав, просто вернет управление по адресу, записанному в \LR}{After finishing, the \printf
function will just return control to the address stored in \LR}.
\IFRU{Но в \LR находится адрес места, откуда была вызвана наша функция}
{But the address of the point from where our function
was called is now in \LR}!
\IFRU{А следовательно, управление из \printf вернется сразу туда}
{Consequently, control from \printf will be returned to that point}.
\IFRU{Следовательно, нет нужды сохранять \LR, потому что нет нужны модифицировать \LR}
{As a consequence, we do not need to save \LR since we do not need to modify \LR}.
\IFRU{А нет нужды модифицировать \LR, потому что нет иных вызовов функций, кроме \printf, к тому же, после этого вызова не нужно ничего здесь больше делать}
{We do not need to modify \LR since there are no other function calls except \printf. Furthermore,
after this call we do not to do anything}!
\IFRU{Поэтому такая оптимизация возможна}
{That's why this optimization is possible}.

\IFRU{Еще один похожий пример описан в секции}{Another similar example was described in} 
``\SwitchCaseDefaultSectionName'' 
\IFRU{, здесь}{section, here}~(\ref{jump_to_last_printf}).

\subsection{\OptimizingKeil + \ThumbMode}

\begin{lstlisting}[caption=\OptimizingKeil + \ThumbMode]
.text:0000000C             printf_main1
.text:0000000C 10 B5                       PUSH    {R4,LR}
.text:0000000E 03 23                       MOVS    R3, #3
.text:00000010 02 22                       MOVS    R2, #2
.text:00000012 01 21                       MOVS    R1, #1
.text:00000014 A4 A0                       ADR     R0, aADBDCD     ; "a=%d; b=%d; c=%d\n"
.text:00000016 06 F0 EB F8                 BL      __2printf
.text:0000001A 10 BD                       POP     {R4,PC}
\end{lstlisting}

\IFRU{Здесь нет особых отличий от неоптимизированного варианта для режима ARM}
{There is no significant difference from the non-optimized code for ARM mode}.


