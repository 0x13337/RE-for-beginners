\section{MIPS}

\subsection{3 \RU{аргумента}\EN{arguments}}

\subsubsection{\Optimizing GCC 4.4.5}

\RU{Главное отличие от примера ``\HelloWorldSectionName'' в том, что здесь на самом деле
вызывается \printf вместо \puts и еще три аргумента передаются в регистрах}
\EN{The main difference with the ``\HelloWorldSectionName'' example is that in this case \printf is called
instead of \puts and 3 more arguments are passed through the registers} \$5 \dots \$7 (\OrENRU \$A0 \dots \$A2).

\RU{Вот почему эти регистры имеют префикс A-, это значит, что они используются для передачи аргументов.}
\EN{That is why these registers are prefixed with A-, which implies they are used for function arguments passing.}

\lstinputlisting[caption=\Optimizing GCC 4.4.5 (\assemblyOutput)]{patterns/03_printf/MIPS/printf3.O3.s.\LANG}

\lstinputlisting[caption=\Optimizing GCC 4.4.5 (IDA)]{patterns/03_printf/MIPS/printf3.O3.IDA.lst.\LANG}

\subsubsection{\NonOptimizing GCC 4.4.5}

\NonOptimizing GCC \RU{более многословен}\EN{is more verbose}:

\lstinputlisting[caption=\NonOptimizing GCC 4.4.5 (\assemblyOutput)]{patterns/03_printf/MIPS/printf3.O0.s.\LANG}

\lstinputlisting[caption=\NonOptimizing GCC 4.4.5 (IDA)]{patterns/03_printf/MIPS/printf3.O0.IDA.lst.\LANG}

\subsection{8 \RU{аргументов}\EN{arguments}}

\RU{Снова воспользуемся примером с 9-ю аргументами из предыдущей секции}\EN{Let's use again the example
with 9 arguments from the previous section}: \myref{example_printf8_x64}.

\lstinputlisting{patterns/03_printf/2.c}

\subsubsection{\Optimizing GCC 4.4.5}

\RU{Только 4 первых аргумента передаются в регистрах \$A0 \dots \$A3, так что остальные передаются 
через стек.}
\EN{Only the first 4 arguments are passed in the \$A0 \dots \$A3 registers, the rest are passed via the stack.}
\index{MIPS!O32}
\RU{Это соглашение о вызовах O32 (которое самое популярное в мире MIPS).
Другие соглашения о вызовах (как N32) могут наделять регистры другими ф-циями.}
\EN{This is the O32 calling convention (which is the most common one in the MIPS world).
Other calling conventions (like N32) may use the registers for different purposes.}

\index{MIPS!\Instructions!SW}
\RU{SW означает ``Store Word'' (записать слово, из регистра в память).}
\EN{SW abbreviates ``Store Word'' (from register to memory).}
\RU{В MIPS нет инструкции для записи значения в память, так что для этого используется пара инструкций (LI/SW).}
\EN{MIPS lacks instructions for storing a value into memory, so an instruction pair has to be used instead (LI/SW).}

\lstinputlisting[caption=\Optimizing GCC 4.4.5 (\assemblyOutput)]{patterns/03_printf/MIPS/printf8.O3.s.\LANG}

\lstinputlisting[caption=\Optimizing GCC 4.4.5 (IDA)]{patterns/03_printf/MIPS/printf8.O3.IDA.lst.\LANG}

\subsubsection{\NonOptimizing GCC 4.4.5}

\NonOptimizing GCC \RU{более многословен}\EN{is more verbose}:

\lstinputlisting[caption=\NonOptimizing GCC 4.4.5 (\assemblyOutput)]{patterns/03_printf/MIPS/printf8.O0.s.\LANG}

\lstinputlisting[caption=\NonOptimizing GCC 4.4.5 (IDA)]{patterns/03_printf/MIPS/printf8.O0.IDA.lst.\LANG}
