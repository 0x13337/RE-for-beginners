\subsection{MIPS}

\subsubsection{3 arguments}

\myparagraph{\Optimizing GCC 4.4.5}

The main difference with the \q{\HelloWorldSectionName} example is that in this case \printf is called
instead of \puts and 3 more arguments are passed through the registers \$5\dots \$7 (or \$A0\dots \$A2).
That is why these registers are prefixed with A-, which implies they are used for function arguments passing.

\lstinputlisting[caption=\Optimizing GCC 4.4.5 (\assemblyOutput)]{patterns/03_printf/MIPS/printf3.O3_EN.s}

\lstinputlisting[caption=\Optimizing GCC 4.4.5 (IDA)]{patterns/03_printf/MIPS/printf3.O3.IDA_EN.lst}

\IDA has coalesced pair of \INS{LUI} and \INS{ADDIU} instructions into one \INS{LA} pseudo instruction.
That's why there are no instruction at address 0x1C: because \INS{LA} \IT{occupies} 8 bytes.

\myparagraph{\NonOptimizing GCC 4.4.5}

\NonOptimizing GCC is more verbose:

\lstinputlisting[caption=\NonOptimizing GCC 4.4.5 (\assemblyOutput)]{patterns/03_printf/MIPS/printf3.O0_EN.s}

\lstinputlisting[caption=\NonOptimizing GCC 4.4.5 (IDA)]{patterns/03_printf/MIPS/printf3.O0.IDA_EN.lst}

\subsubsection{8 arguments}

Let's use again the example with 9 arguments from the previous section: \myref{example_printf8_x64}.

\lstinputlisting{patterns/03_printf/2.c}

\myparagraph{\Optimizing GCC 4.4.5}

Only the first 4 arguments are passed in the \$A0 \dots \$A3 registers, the rest are passed via the stack.
\myindex{MIPS!O32}

This is the O32 calling convention (which is the most common one in the MIPS world).
Other calling conventions (like N32) may use the registers for different purposes.

\myindex{MIPS!\Instructions!SW}

\INS{SW} abbreviates \q{Store Word} (from register to memory).
MIPS lacks instructions for storing a value into memory, so an instruction pair has to be used instead (\INS{LI}/\INS{SW}).

\lstinputlisting[caption=\Optimizing GCC 4.4.5 (\assemblyOutput)]{patterns/03_printf/MIPS/printf8.O3_EN.s}

\lstinputlisting[caption=\Optimizing GCC 4.4.5 (IDA)]{patterns/03_printf/MIPS/printf8.O3.IDA_EN.lst}

\myparagraph{\NonOptimizing GCC 4.4.5}

\NonOptimizing GCC is more verbose:

\lstinputlisting[caption=\NonOptimizing GCC 4.4.5 (\assemblyOutput)]{patterns/03_printf/MIPS/printf8.O0_EN.s}

\lstinputlisting[caption=\NonOptimizing GCC 4.4.5 (IDA)]{patterns/03_printf/MIPS/printf8.O0.IDA_EN.lst}

