\section{MIPS}

\subsection{3 argomenti}

\subsubsection{\Optimizing GCC 4.4.5}

La differenza principale con l'esempio \q{\HelloWorldSectionName} e' che in questo caso \printf e' chiamata 
al posto di \puts, e 3 argomenti aggiuntivi sono passati attraverso i registri \$5\dots \$7 (o \$A0\dots \$A2).
Questo e' il motivo per cui questi registri hanno il prefisso A-, che implica il loro uso per il passaggio di argomenti di funzioni.

\lstinputlisting[caption=\Optimizing GCC 4.4.5 (\assemblyOutput)]{patterns/03_printf/MIPS/printf3.O3.s.\LANG}

\lstinputlisting[caption=\Optimizing GCC 4.4.5 (IDA)]{patterns/03_printf/MIPS/printf3.O3.IDA.lst.\LANG}

\IDA ha fuso le coppie di istruzioni \INS{LUI} e \INS{ADDIU} in una unica pseudoistruzione \INS{LA}.
Questo e' il motivo per cui non c'e' nessuna istruzione all'indirizzo 0x1C: perche' \INS{LA} \IT{occupa} 8 byte.%

\subsubsection{\NonOptimizing GCC 4.4.5}

\NonOptimizing GCC e' piu' verboso:

\lstinputlisting[caption=\NonOptimizing GCC 4.4.5 (\assemblyOutput)]{patterns/03_printf/MIPS/printf3.O0.s.\LANG}

\lstinputlisting[caption=\NonOptimizing GCC 4.4.5 (IDA)]{patterns/03_printf/MIPS/printf3.O0.IDA.lst.\LANG}

\subsection{8 argomenti}

Usiamo nuovamente l'esempio con 9 argomenti dalla sezione prcedente: \myref{example_printf8_x64}.

\lstinputlisting{patterns/03_printf/2.c}

\subsubsection{\Optimizing GCC 4.4.5}

Solo i primi 4 argomenti sono passati nei registri \$A0 \dots \$A3, gli altri sono passati tramite lo stack.
\myindex{MIPS!O32}

Questa e' la calling convention O32 (che e' la piu' comune nel mondo MIPS).
Altre calling conventions (come N32) possono usare i registri per scopi diversi.

\myindex{MIPS!\Instructions!SW}

\INS{SW} e' l'abbreviazione di \q{Store Word} (da un registro alla memoria).
MIPS manca di istruzioni per memorizzare un valore in memoria, e' quindi necessario usare una coppia di istruzioni (LI/SW).

\lstinputlisting[caption=\Optimizing GCC 4.4.5 (\assemblyOutput)]{patterns/03_printf/MIPS/printf8.O3.s.\LANG}

\lstinputlisting[caption=\Optimizing GCC 4.4.5 (IDA)]{patterns/03_printf/MIPS/printf8.O3.IDA.lst.\LANG}

\subsubsection{\NonOptimizing GCC 4.4.5}

\NonOptimizing GCC e' piu' verboso:

\lstinputlisting[caption=\NonOptimizing GCC 4.4.5 (\assemblyOutput)]{patterns/03_printf/MIPS/printf8.O0.s.\LANG}

\lstinputlisting[caption=\NonOptimizing GCC 4.4.5 (IDA)]{patterns/03_printf/MIPS/printf8.O0.IDA.lst.\LANG}
