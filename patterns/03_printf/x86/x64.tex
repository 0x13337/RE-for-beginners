\subsection{x64: \RU{8 аргументов}\EN{8 arguments}}

\index{x86-64}
\label{example_printf8_x64}
\RU{Для того, чтобы посмотреть, как остальные аргументы будут передаваться через стек, 
изменим пример еще раз, 
увеличив количество передаваемых аргументов до 9 (строка формата \printf и 8 переменных типа \Tint)}
\EN{To see how other arguments will be passed via the stack, let's change our example again by increasing the number of arguments
to be passed to 9 (\printf format string + 8 \Tint variables)}:

\lstinputlisting{patterns/03_printf/2.c}

\subsubsection{MSVC}

\RU{Как уже было сказано раннее, первые 4 аргумента в Win64 передаются в регистрах}
\EN{As we saw before, the first 4 arguments are passed in the} \RCX, \RDX, \Reg{8}, \Reg{9}
\RU{, а остальные --- через стек}\EN{ registers in Win64, while all the rest---via the stack}.
\RU{Здесь мы это и видим}\EN{That is what we see here}.
\RU{Впрочем, инструкция \PUSH не используется, вместо нее, при помощи \MOV, значения сразу записываются в стек}
\EN{However, the \MOV instruction, instead of \PUSH, is used for preparing the stack, so the values are written
to the stack in a straightforward manner}.

\lstinputlisting[caption=MSVC 2012 x64]{patterns/03_printf/x86/2_MSVC_x64.asm.\LANG}

\RU{Наблюдательный читатель может спросить, почему для значений типа \Tint отводится 8 байт,
ведь нужно только 4?}
\EN{The observant reader may ask why are 8 bytes allocated for \Tint values, when 4 is enough?}
\RU{Да, это нужно запомнить: для значений всех типов более коротких чем 64-бита, отводится 8 байт.}
\EN{Yes, this should be memorized: 8 bytes are allocated for any data type shorter than 64 bits.}
\RU{Это сделано для удобства: так всегда легко рассчитать адрес того или иного аргумента.}
\EN{It's done for convenience: it makes it easy to calculate the address of this or that argument.}
\RU{К тому же, все они расположены по выровненным адресам в памяти.}
\EN{Besides, they are all located at aligned memory addresses.}
% also for local variables?
\RU{В 32-битных средах точно также: для всех типов резервируется 4 байта в стеке.}
\EN{It's the same in 32-bit environments: 4 bytes are reserved for all data types.}

\subsubsection{GCC}

\RU{В *NIX-системах для x86-64 ситуация похожая, вот только первые 6 аргументов передаются через}
\EN{In *NIX OS-es, it's the same for x86-64, except that the first 6 arguments are passed in the} \RDI, \RSI,
\RDX, \RCX, \Reg{8}, \Reg{9}\EN{ registers}.
\RU{Остальные --- через стек}\EN{All the rest---via the stack}.
\RU{GCC генерирует код записывающий указатель на строку в \EDI вместо \RDI --- 
это мы уже рассмотрели чуть раньше}\EN{GCC generates the code that writes the string pointer into \EDI instead of \RDI{}---we
saw this thing before}: \ref{hw_EDI_instead_of_RDI}.

\RU{Почему перед вызовом \printf очищается регистр \EAX, мы уже рассмотрели раннее}
\EN{We also saw before the \EAX register being cleared before a \printf call}: \ref{SysVABI_input_EAX}.

\lstinputlisting[caption=\Optimizing GCC 4.4.6 x64]{patterns/03_printf/x86/2_GCC_x64.s.\LANG}

\ifdefined\IncludeGDB
\subsubsection{GCC + GDB}
\index{GDB}

\RU{Попробуем этот пример в}\EN{Let's try this example in} \ac{GDB}.

\begin{lstlisting}
$ gcc -g 2.c -o 2
\end{lstlisting}

\begin{lstlisting}
$ gdb 2
GNU gdb (GDB) 7.6.1-ubuntu
Copyright (C) 2013 Free Software Foundation, Inc.
License GPLv3+: GNU GPL version 3 or later <http://gnu.org/licenses/gpl.html>
This is free software: you are free to change and redistribute it.
There is NO WARRANTY, to the extent permitted by law.  Type "show copying"
and "show warranty" for details.
This GDB was configured as "x86_64-linux-gnu".
For bug reporting instructions, please see:
<http://www.gnu.org/software/gdb/bugs/>...
Reading symbols from /home/dennis/polygon/2...done.
\end{lstlisting}

\begin{lstlisting}[caption=\RU{ставим брякпойнт на \printf{,} запускаем}\EN{let's set the breakpoint to \printf{,} and run}]
(gdb) b printf
Breakpoint 1 at 0x400410
(gdb) run
Starting program: /home/dennis/polygon/2 

Breakpoint 1, __printf (format=0x400628 "a=%d; b=%d; c=%d; d=%d; e=%d; f=%d; g=%d; h=%d\n") at printf.c:29
29	printf.c: No such file or directory.
\end{lstlisting}

\RU{В регистрах}\EN{Registers} \RSI/\RDX/\RCX/\Reg{8}/\Reg{9} 
\RU{всё предсказуемо}\EN{have the values which should be there}.
\RU{А }\RIP \RU{содержит адрес самой первой инструкции ф-ции}\EN{has the address of the very first instruction
of the} \printf\EN{ function}.

\begin{lstlisting}
(gdb) info registers
rax            0x0	0
rbx            0x0	0
rcx            0x3	3
rdx            0x2	2
rsi            0x1	1
rdi            0x400628	4195880
rbp            0x7fffffffdf60	0x7fffffffdf60
rsp            0x7fffffffdf38	0x7fffffffdf38
r8             0x4	4
r9             0x5	5
r10            0x7fffffffdce0	140737488346336
r11            0x7ffff7a65f60	140737348263776
r12            0x400440	4195392
r13            0x7fffffffe040	140737488347200
r14            0x0	0
r15            0x0	0
rip            0x7ffff7a65f60	0x7ffff7a65f60 <__printf>
...
\end{lstlisting}

\begin{lstlisting}[caption=\RU{смотрим на строку формата}\EN{let's inspect the format string}]
(gdb) x/s $rdi
0x400628:	"a=%d; b=%d; c=%d; d=%d; e=%d; f=%d; g=%d; h=%d\n"
\end{lstlisting}

\RU{Дампим стек на этот раз с командой x/g}\EN{Let's dump the stack with the x/g command this time}\EMDASH{}g 
\RU{означает}\EN{means} \IT{giant words}, \RU{т.е., 64-битные слова}\EN{i.e., 64-bit words}.

\begin{lstlisting}
(gdb) x/10g $rsp
0x7fffffffdf38:	0x0000000000400576	0x0000000000000006
0x7fffffffdf48:	0x0000000000000007	0x00007fff00000008
0x7fffffffdf58:	0x0000000000000000	0x0000000000000000
0x7fffffffdf68:	0x00007ffff7a33de5	0x0000000000000000
0x7fffffffdf78:	0x00007fffffffe048	0x0000000100000000
\end{lstlisting}

\RU{Самый первый элемент стека, как и в прошлый раз, это}\EN{The very first stack element, 
just like in the previous case, is the} \ac{RA}.
\RU{Через стек также передаются 3 значения}\EN{3 values are also passed in stack}: 6, 7, 8.
\RU{Видно, что 8 передается с не очищенной старшей 32-битной частью}\EN{We also see that 8 is passed
with the high 32-bits not cleared}: \TT{0x00007fff00000008}.
\RU{Это нормально, ведь передаются числа типа \Tint, а они 32-битные}\EN{That's OK, because the values have
\Tint type, which is 32-bit}.
\RU{Так что в старшей части регистра или памяти стека остался ``случайный мусор''}\EN{So, the high register
or stack element part may contain ``random garbage''}.

\RU{\ac{GDB} показывает всю ф-цию \main, если попытаться посмотреть, куда возвратится управление после
исполнения \printf}\EN{If you take a look at where control flow will return after the execution of \printf,
\ac{GDB} will show the whole \main function}:

\begin{lstlisting}
(gdb) set disassembly-flavor intel
(gdb) disas 0x0000000000400576
Dump of assembler code for function main:
   0x000000000040052d <+0>:	push   rbp
   0x000000000040052e <+1>:	mov    rbp,rsp
   0x0000000000400531 <+4>:	sub    rsp,0x20
   0x0000000000400535 <+8>:	mov    DWORD PTR [rsp+0x10],0x8
   0x000000000040053d <+16>:	mov    DWORD PTR [rsp+0x8],0x7
   0x0000000000400545 <+24>:	mov    DWORD PTR [rsp],0x6
   0x000000000040054c <+31>:	mov    r9d,0x5
   0x0000000000400552 <+37>:	mov    r8d,0x4
   0x0000000000400558 <+43>:	mov    ecx,0x3
   0x000000000040055d <+48>:	mov    edx,0x2
   0x0000000000400562 <+53>:	mov    esi,0x1
   0x0000000000400567 <+58>:	mov    edi,0x400628
   0x000000000040056c <+63>:	mov    eax,0x0
   0x0000000000400571 <+68>:	call   0x400410 <printf@plt>
   0x0000000000400576 <+73>:	mov    eax,0x0
   0x000000000040057b <+78>:	leave  
   0x000000000040057c <+79>:	ret    
End of assembler dump.
\end{lstlisting}

\RU{Заканчиваем исполнение \printf, исполняем инструкцию обнуляющую \EAX, 
удостоверяемся что в регистре \EAX именно ноль}\EN{Let's finish executing \printf, execute the instruction
zeroing \EAX, and note that the \EAX register has a value of exactly zero}.
\RIP \RU{указывает сейчас на инструкцию}\EN{now points to the} \TT{LEAVE}\RU{, т.е., предпоследнюю в ф-ции \main}
\EN{ instruction, i.e., the penultimate one in the \main function}.

\begin{lstlisting}
(gdb) finish
Run till exit from #0  __printf (format=0x400628 "a=%d; b=%d; c=%d; d=%d; e=%d; f=%d; g=%d; h=%d\n") at printf.c:29
a=1; b=2; c=3; d=4; e=5; f=6; g=7; h=8
main () at 2.c:6
6		return 0;
Value returned is $1 = 39
(gdb) next
7	};
(gdb) info registers
rax            0x0	0
rbx            0x0	0
rcx            0x26	38
rdx            0x7ffff7dd59f0	140737351866864
rsi            0x7fffffd9	2147483609
rdi            0x0	0
rbp            0x7fffffffdf60	0x7fffffffdf60
rsp            0x7fffffffdf40	0x7fffffffdf40
r8             0x7ffff7dd26a0	140737351853728
r9             0x7ffff7a60134	140737348239668
r10            0x7fffffffd5b0	140737488344496
r11            0x7ffff7a95900	140737348458752
r12            0x400440	4195392
r13            0x7fffffffe040	140737488347200
r14            0x0	0
r15            0x0	0
rip            0x40057b	0x40057b <main+78>
...
\end{lstlisting}
\fi
