\subsection{ARM}

\subsubsection{\OptimizingKeilVI (\ThumbMode)}

\begin{lstlisting}
.text:00000042             scanf_main
.text:00000042
.text:00000042             var_8           = -8
.text:00000042
.text:00000042 08 B5                       PUSH    {R3,LR}
.text:00000044 A9 A0                       ADR     R0, aEnterX     ; "Enter X:\n"
.text:00000046 06 F0 D3 F8                 BL      __2printf
.text:0000004A 69 46                       MOV     R1, SP
.text:0000004C AA A0                       ADR     R0, aD          ; "%d"
.text:0000004E 06 F0 CD F8                 BL      __0scanf
.text:00000052 00 99                       LDR     R1, [SP,#8+var_8]
.text:00000054 A9 A0                       ADR     R0, aYouEnteredD___ ; "You entered %d...\n"
.text:00000056 06 F0 CB F8                 BL      __2printf
.text:0000005A 00 20                       MOVS    R0, #0
.text:0000005C 08 BD                       POP     {R3,PC}
\end{lstlisting}

\myindex{\CLanguageElements!\Pointers}
\RU{Чтобы \scanf мог вернуть значение, ему нужно передать указатель на переменную типа \Tint.}
\EN{In order for \scanf to be able to read item it needs a parameter---pointer to an \Tint.}
\Tint\RU{~--- 32-битное значение, для его хранения нужно только 4 байта, и оно помещается в 
32-битный регистр.}
\EN{is 32-bit, so we need 4 bytes to store it somewhere in memory, and it fits exactly 
in a 32-bit register.}
\myindex{IDA!var\_?}
\RU{Место для локальной переменной \GTT{x} выделяется в стеке, \IDA наименовала её \IT{var\_8}. 
Впрочем, место для неё выделять не обязательно, т.к. \glslink{stack pointer}{указатель стека} \ac{SP} уже указывает на место, 
свободное для использования.}\EN{A place for the local variable \GTT{x} is allocated in the stack and \IDA
has named it \IT{var\_8}. It is not necessary, however, to allocate a such since \ac{SP} (\gls{stack pointer}) is already pointing to that space and it can be used directly.}
\RU{Так что значение указателя \ac{SP} копируется в регистр \Reg{1}, и вместе с format-строкой, 
передается в \scanf.}
\EN{So, \ac{SP}'s value is copied to the \Reg{1} register and, together with the format-string, passed
to \scanf.}
\myindex{ARM!\Instructions!LDR}
\RU{Позже, при помощи инструкции \INS{LDR}, это значение перемещается из стека в регистр \Reg{1}, 
чтобы быть переданным в \printf.}\EN{Later, with the help of the \INS{LDR} instruction, this value is moved
from the stack to the \Reg{1} register in order to be passed to \printf.}

\subsubsection{ARM64}

\lstinputlisting[caption=\NonOptimizing GCC 4.9.1 ARM64,numbers=left]{patterns/04_scanf/1_simple/ARM64_GCC491_O0.s.\LANG}

\RU{Под стековый фрейм выделяется 32 байта, что больше чем нужно. Вероятно, это связано с выравниваем по границе памяти?}%
\EN{There is 32 bytes are allocated for stack frame, which is bigger than it needed. Perhaps some memory aligning issue?}
\RU{Самая интересная часть~--- это поиск места под переменную $x$ в стековом фрейме (строка 22).}
\EN{The most interesting part is finding space for the $x$ variable in the stack frame (line 22).}
\RU{Почему 28? Почему-то, компилятор решил расположить эту переменную в конце стекового фрейма, а не в начале.}%
\EN{Why 28? Somehow, compiler decided to place this variable at the end of stack frame instead of beginning.}
\RU{Адрес потом передается в \scanf, которая просто сохраняет значение, введенное пользователем, в памяти
по этому адресу.}
\EN{The address is passed to \scanf, which just stores the user input value in the memory at that address.}
\RU{Это 32-битное значение типа \Tint}\EN{This is 32-bit value of type \Tint}.
\RU{Значение загружается в строке 27 и затем передается в \printf.}
\EN{The value is fetched at line 27 and then passed to \printf.}

