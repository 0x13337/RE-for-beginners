\subsectionold{ARM}

\subsubsectionold{\OptimizingKeilVI (\ThumbMode)}

\begin{lstlisting}
.text:00000042             scanf_main
.text:00000042
.text:00000042             var_8           = -8
.text:00000042
.text:00000042 08 B5                       PUSH    {R3,LR}
.text:00000044 A9 A0                       ADR     R0, aEnterX     ; "Enter X:\n"
.text:00000046 06 F0 D3 F8                 BL      __2printf
.text:0000004A 69 46                       MOV     R1, SP
.text:0000004C AA A0                       ADR     R0, aD          ; "%d"
.text:0000004E 06 F0 CD F8                 BL      __0scanf
.text:00000052 00 99                       LDR     R1, [SP,#8+var_8]
.text:00000054 A9 A0                       ADR     R0, aYouEnteredD___ ; "You entered %d...\n"
.text:00000056 06 F0 CB F8                 BL      __2printf
.text:0000005A 00 20                       MOVS    R0, #0
.text:0000005C 08 BD                       POP     {R3,PC}
\end{lstlisting}

\myindex{\CLanguageElements!\Pointers}

In order for \scanf to be able to read item it needs a parameter---pointer to an \Tint.
\Tint is 32-bit, so we need 4 bytes to store it somewhere in memory, and it fits exactly in a 32-bit register.
\myindex{IDA!var\_?}
A place for the local variable \GTT{x} is allocated in the stack and \IDA
has named it \IT{var\_8}. It is not necessary, however, to allocate a such since \ac{SP} (\gls{stack pointer}) is already pointing to that space and it can be used directly.

So, \ac{SP}'s value is copied to the \Reg{1} register and, together with the format-string, passed to \scanf.
\myindex{ARM!\Instructions!LDR}
Later, with the help of the \INS{LDR} instruction, this value is moved from the stack to the \Reg{1} register in order to be passed to \printf.

\subsubsectionold{ARM64}

\lstinputlisting[caption=\NonOptimizing GCC 4.9.1 ARM64,numbers=left]{patterns/04_scanf/1_simple/ARM64_GCC491_O0_EN.s}

There is 32 bytes are allocated for stack frame, which is bigger than it needed. Perhaps some memory aligning issue?
The most interesting part is finding space for the $x$ variable in the stack frame (line 22).
Why 28? Somehow, compiler decided to place this variable at the end of stack frame instead of beginning.
The address is passed to \scanf, which just stores the user input value in the memory at that address.
This is 32-bit value of type \Tint.
The value is fetched at line 27 and then passed to \printf.

