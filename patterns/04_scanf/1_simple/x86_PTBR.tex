\subsection{x86}

\subsubsection{MSVC}

Aqui está o que o resultado depois de se compilar com o MSVC 2010:

% TODO to translate
\lstinputlisting{patterns/04_scanf/1_simple/ex1_MSVC_EN.asm}

\TT{x} é uma variável local.

De acordo com os padrões de \CCpp ela só deve ser visível nessa função e não além dela.
Tradicionalmente, variáveis locais são guardadas na pilha.
Provavelmente há outras maneiras de alocá-las, mas no x86 é assim que é feito.

\myindex{x86!\Instructions!PUSH}
O objetivo da intrução que se segue após o cabeçalho da função, \INS{PUSH ECX},
não é para salvar o valor de \ECX
(perceba que não há a instrução \INS{POP ECX} no fim da função).

Na verdade, esse PUSH aloca 4 bytes na pilha para guardar a variável \TT{x}.

\label{stack_frame}
\myindex{\Stack!\PTBRph{}}
\myindex{x86!\Registers!EBP}
\TT{x} é para ser acessada com a ajuda do macro \TT{\_x\$} (que é igual a -4) e o registrador \EBP apontando para a posição atual.

Conforme a execução da função avança, \EBP está apontando para a posição atual da pilha,
sendo possível acessar variáveis locais e argumentos da função via \TT{EBP+offset}.

\myindex{x86!\Registers!ESP}
Também é possível usar \ESP para o mesmo objetivo, mas não é muito conveniente pois ele se altera com frequência.
O valor de \EBP pode ser visto como uma cópia do valor de \ESP no começo da execução da função.

Aqui está a aparência típica de uma pilha em um ambiente de 32-bits:

\begin{center}
\begin{tabular}{ | l | l | }
\hline
\dots & \dots \\
\hline
EBP-8 & variável local \#2, \MarkedInIDAAs{} \TT{var\_8} \\
\hline
EBP-4 & variável local \#1, \MarkedInIDAAs{} \TT{var\_4} \\
\hline
EBP & valor salvo de \EBP \\
\hline
EBP+4 & Endereço de retorno \\
\hline
EBP+8 & \argument \#1, \MarkedInIDAAs{} \TT{arg\_0} \\
\hline
EBP+0xC & \argument \#2, \MarkedInIDAAs{} \TT{arg\_4} \\
\hline
EBP+0x10 & \argument \#3, \MarkedInIDAAs{} \TT{arg\_8} \\
\hline
\dots & \dots \\
\hline
\end{tabular}
\end{center}

A função \scanf no nosso exemplo tem dois argumentos.

O primeiro é um ponteiro para a string contendo \TT{\%d} e a segunda é o endereço da variável \TT{x}.

\myindex{x86!\Instructions!LEA}
Primeiro, o endereço da variável \TT{x} é carregado no registrador \EAX pela instrução \TT{lea eax, DWORD PTR \_x\$[ebp].}.

\PTBRph{}

Nós podemos dizer que nesse caso, \LEA simplesmente armazena a soma do valor em \EBP e o macro \TT{\_x\$} no registrador \EAX.

É a mesma coisa que \INS{lea eax, [ebp-4]}.

Então, 4 está sendo subtraido do registrador \EBP e o resultado é carregado no registrador \EAX.
Depois, o valor em \EAX é empurrado para dentro da pilha e o \scanf é chamado.

\printf é chamado depois disso com seu primeiro argumento --- um ponteiro para a string:
\TT{You entered \%d...\textbackslash{}n}.

O segundo argumento é preparado com: \TT{mov ecx, [ebp-4]}.
A instrução armazena o valor de x e não o seu endereço, no registrador \ECX.

Depois, o valor em \ECX é armazenado na pilha e o último \printf é chamado.

% TODO: olly, 

\subsubsection{\PTBRph{}}

A propósito, esse simples exemplo é a demonstração do fato de que o compilador traduz a lista de expressões em \CCpp em uma lista sequêncial de instruções.
Não há nada entre as expressões em \CCpp, como também no código de máquina resultante, não há nada, o controle de fluxo passa de uma expressão para a outra diretamente.

