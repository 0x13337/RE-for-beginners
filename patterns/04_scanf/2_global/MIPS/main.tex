\ifx\RUSSIAN\undefined

\subsection{MIPS}

\subsubsection{Uninitialized global variable}

So now $x$ variable is global.
I made executable file rather than object one and load it into IDA.
IDA shows presence of $x$ variable in .sbss ELF section (remember about Global Pointer? \ref{MIPS_GP}),
since it's value is not initialized at start.

\lstinputlisting[caption=\Optimizing GCC 4.4.5 (IDA)]{patterns/04_scanf/2_global/MIPS/O3_IDA.lst}

IDA reduces amount of information, so I also made a listing using objdump and added my comments to it:

\lstinputlisting[caption=\Optimizing GCC 4.4.5 (objdump),numbers=left]{patterns/04_scanf/2_global/MIPS/O3_objdump.txt}

Now we see how address of $x$ variable is taken from a 64KiB data buffer using GP and adding
negative offset to it (line 18).
More than that: three external functions are used in our example (puts, scanf, printf), and their
addresses also taken from 64KiB data buffer using GP (lines 9, 16 and 26).
GP points to the middle of buffer, so such offset may give us a glue that all three function's addresses
plus address of $x$ variable is stored somewhere at the beginning of data buffer.
Indeed, our example is tiny.

\index{MIPS!\Pseudoinstructions!MOVE}
\index{MIPS!\Pseudoinstructions!NOP}
Another thing to mention is that the function is finished by two NOP's (\TT{move at,at} --- this is idle
instruction), in order to align next function's start on 16-byte boundary.

\subsubsection{Initialized global variable}

Let's alter our example to make $x$ variable some default value:

\begin{lstlisting}
int x=10; // default value
\end{lstlisting}

Now IDA shows that $x$ variable is residing in .data section:

\lstinputlisting[caption=\Optimizing GCC 4.4.5 (IDA)]{patterns/04_scanf/2_global/MIPS/O3_IDA_init.lst}

Why not .sdata? I don't know, I probably should specify some GCC option?
Nevertheless, now $x$ is in .data, that's general memory area, and we can now take a look
how to work with variables there.

\index{MIPS!\Instructions!LUI}
\index{MIPS!\Instructions!ADDIU}
Address of variable should be constructed using pair of instructions.
In our case that's LUI (``Load Upper Immediate'') and ADDIU (``Add Immediate Unsigned Word'').

Here is also objdump listing for close inspection:

\lstinputlisting[caption=\Optimizing GCC 4.4.5 (objdump)]{patterns/04_scanf/2_global/MIPS/O3_objdump_init.txt}

\index{MIPS!\Instructions!LUI}
\index{MIPS!\Instructions!ADDIU}
\index{MIPS!\Instructions!LW}
We see that for the first, address is constructed using LUI and ADDIU, but high part of address is still in
\$S0 register, and it's possible to add offset in LW (``Load Word'') instruction, so one single LW is enough 
for loading value from variable and pass it to \printf.

Registers holding temporary data are prefixed with T-, but there are also those prefixed with S-, 
contents of which will be preserved in other functions (i.e., ``saved''). 
That's why \$S0 was set at address 0x4006cc and was used again
at address 0x4006e8, after scanf() call. 
scanf() function doesn't change its value.

% TODO non-optimizing example?

\fi
