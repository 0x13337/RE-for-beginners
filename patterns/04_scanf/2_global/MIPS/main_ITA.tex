\subsection{MIPS}

\subsubsection{Variabili globali non inizializzate}

La variabile $x$ è ora globale.
Compiliamo in un file eseguibile anziché oggetto e carichiamolo in \IDA.
IDA mostra la variabile $x$ nella sezione ELF .sbss (ricordate il \q{Global Pointer}? \myref{MIPS_GP}),
poiché la variabile non è inizialmente inizializzata.

\lstinputlisting[caption=\Optimizing GCC 4.4.5 (IDA)]{patterns/04_scanf/2_global/MIPS/O3_IDA_EN.lst}

IDA riduce la quantità di informazioni, quindi facciamo anche un listing usando objdump e lo commentiamo:

\lstinputlisting[caption=\Optimizing GCC 4.4.5 (objdump),numbers=left]{patterns/04_scanf/2_global/MIPS/O3_objdump_EN.txt}

Vediamo che l'indirizzo della variabile $x$ viene letto da un buffer dati di 64KiB usando il GP a cui viene sommato un offset negativo (riga 18).
Inoltre gli indirizzi delle tre funzioni esterne usate nel programma (\puts, \scanf, \printf) sono anch'essi letti dal buffer dati globale di 64KiB usando il GP (righe 9, 16 e 26).
GP punta al metà del buffer, e gli offset suggeriscono che gli indirizzi delle tre funzioni e della variabile sono tutti memorizzati da qualche parte vicino all'inizio di quel buffer.
Ciò ha senso in quanto il nostro esempio è davvero molto piccolo.

\myindex{MIPS!\Pseudoinstructions!MOVE}
\myindex{MIPS!\Pseudoinstructions!NOP}

Un'altra cosa che vale la pena notare e' che la funzione finisce con due \ac{NOP} (\TT{MOVE \$AT,\$AT} --- una "idle instruction"), per allineare l'inizio della prossima funzione ad un confine di 16-byte.

\subsubsection{Initialized global variable}

Modifichiamo il nostro esempio assegnando un valore predefinito alla variabile $x$:

\lstinputlisting{patterns/04_scanf/2_global/default_value_EN.c}

Adesso IDA mostra che la variabile $x$ risiede nella sezione .data:

\lstinputlisting[caption=\Optimizing GCC 4.4.5 (IDA)]{patterns/04_scanf/2_global/MIPS/O3_IDA_init_EN.lst}

Perche' non in .sdata? Puo' dipendere da qualche opzione di GCC?

Ciononostante $x$ si trova in .data, e possiamo vedere come si lavora con variabili localizzate in questa area di memoria generica.

\myindex{MIPS!\Instructions!LUI}
\myindex{MIPS!\Instructions!ADDIU}

L'indirizzo della variabile deve essere formato utilizzando un paio di istruzioni.

Nel nostro caso sono \INS{LUI} (\q{Load Upper Immediate}) e \INS{ADDIU} (\q{Add Immediate Unsigned Word}).

Vediamo anche il listato di objdump per maggiore approfondimento:

\lstinputlisting[caption=\Optimizing GCC 4.4.5 (objdump)]{patterns/04_scanf/2_global/MIPS/O3_objdump_init_EN.txt}

\myindex{MIPS!\Instructions!LUI}
\myindex{MIPS!\Instructions!ADDIU}
\myindex{MIPS!\Instructions!LW}

Notiamo che l'indirizzo e' formato usando \INS{LUI} e \INS{ADDIU}, ma la parte alta dell'indirizzo e' ancora nel registro
\$S0 , ed e' possibile codificare l'offset in un'istruzione \INS{LW} (\q{Load Word}), percio' una singola \INS{LW} e' sufficiente 
per caricare un valore dalla variabile e passarlo a \printf.

I registri che memorizzano dati temporanei hanno il prefisso T-, ma qui vediamo anche alcuni con prefisso S-, il cui contenuto deve essere preservato prima del loro
utilizzo in altre funzioni (i.e., \q{saved}).
% FIXME:
% This needs to be clarified a bit, e.g. "the registers need to be preserved if a function is called and it wants to use them

Questo e' il motivo per cui il valore di \$S0 era stato settato all'indirizzo 0x4006cc e usato nuovamente all'indirizzo 0x4006e8, dopo la chiamata a \scanf. 
La funzione \scanf non ne cambia il valore.

% TODO non-optimized example?
