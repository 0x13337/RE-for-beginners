\subsectionold{ARM}

\subsubsectionold{ARM: \OptimizingKeilVI (\ThumbMode)}

\lstinputlisting[caption=\OptimizingKeilVI (\ThumbMode)]{patterns/04_scanf/3_checking_retval/ex3_ARM_Keil_thumb_O3.asm}

\myindex{ARM!\Instructions!CMP}
\myindex{ARM!\Instructions!BEQ}

The new instructions here are \CMP and \ac{BEQ}.

\CMP is analogous to the x86 instruction with the same name, it subtracts one of the arguments from the other and updates the conditional flags if needed.
% TODO: в мануале ARM $op1 + NOT(op2) + 1$ вместо вычитания

\myindex{ARM!\Registers!Z}
\myindex{x86!\Instructions!JZ}
\ac{BEQ} jumps to another address if the operands were equal to each other, or,
if the result of the last computation was 0, or if the Z flag is 1.
It behaves as \JZ in x86.

Everything else is simple: the execution flow forks in two branches, then the branches
converge at the point where 0 is written into the \Reg{0} as a function return value, and then the function ends.

\subsubsectionold{ARM64}

\lstinputlisting[caption=\NonOptimizing GCC 4.9.1 ARM64,numbers=left]{patterns/04_scanf/3_checking_retval/ARM64_GCC491_O0_EN.s}

\myindex{ARM!\Instructions!CMP}
\myindex{ARM!\Instructions!Bcc}
Code flow in this case forks with the use of \INS{CMP}/\INS{BNE} (Branch if Not Equal) instructions pair.

