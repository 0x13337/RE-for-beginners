\subsubsection{MIPS}

\lstinputlisting[caption=\Optimizing GCC 4.4.5 (IDA)]{patterns/04_scanf/3_checking_retval/MIPS_O3_IDA.lst}

\myindex{MIPS!\Instructions!BEQ}
\ifdefined\ENGLISH
\scanf returns the result of its work in register \$V0. It is checked at address 0x004006E4
by comparing the values in \$V0 with \$V1 (1 has been stored in \$V1 earlier, at 0x004006DC).
\INS{BEQ} stands for \q{Branch Equal}.
If the two values are equal (i.e., success), the execution jumps to address 0x0040070C.
\fi

\ifdefined\RUSSIAN
\scanf возвращает результат своей работы в регистре \$V0 и он проверяется по адресу 0x004006E4
сравнивая значения в \$V0 и \$V1 (1 записан в \$V1 ранее, на 0x004006DC).
\INS{BEQ} означает \q{Branch Equal} (переход если равно).
Если значения равны (т.е. в случае успеха), произойдет переход по адресу 0x0040070C.
\fi

\ifdefined\ITALIAN
\scanf restituisce il risultato del suo lavoro nel registro \$V0. Cio' viene controllato all'indirizzo 0x004006E4
confrontando il valore in \$V0 con quello in \$V1 (1 era stato memorizzato in \$V1 precedentemente, a 0x004006DC).
\INS{BEQ} sta per \q{Branch Equal}.
Se i due valori sono uguali (cioe' \scanf e' terminata con successo), l'esecuzione salta all'indirizzo 0x0040070C.
\fi
