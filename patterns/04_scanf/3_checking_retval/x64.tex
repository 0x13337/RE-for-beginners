\subsection{MSVC: x64}

\myindex{x86-64}

\ifdefined\RUSSIAN
Так как здесь мы работаем с переменными типа \Tint, а они в x86-64 остались 32-битными, то мы здесь видим, как продолжают использоваться регистры с префиксом \TT{E-}.
Но для работы с указателями, конечно, используются 64-битные части регистров с префиксом \TT{R-}.

\lstinputlisting[caption=MSVC 2012 x64]{patterns/04_scanf/3_checking_retval/ex3_MSVC_x64_RU.asm}
\fi

\ifdefined\ENGLISH
Since we work here with \Tint{}-typed variables, which are still 32-bit in x86-64, we see how the 32-bit part of the registers (prefixed with \TT{E-}) are used here as well.
While working with pointers, however, 64-bit register parts are used, prefixed with \TT{R-}.

\lstinputlisting[caption=MSVC 2012 x64]{patterns/04_scanf/3_checking_retval/ex3_MSVC_x64_EN.asm}
\fi

\ifdefined\BRAZILIAN
Como trabalhamos aqui com variáveis do tipo \Tint, que ainda são 32-bits no x86-64, nós vemos como a parte de 32-bits dos registradores (com o prefixo \TT{E-}) sao usadas aqui da mesma maneira.
No entanto, quando trabalhamos com ponteiros, as partes dos registradores de 64-bits são usadas, prefixadas com \TT{R-}.

% TODO translate
\lstinputlisting[caption=MSVC 2012 x64]{patterns/04_scanf/3_checking_retval/ex3_MSVC_x64_EN.asm}
\fi

