\subsection{ARM: \OptimizingKeilVI + \ThumbMode}

\begin{lstlisting}
.text:00000000 ; Segment type: Pure code
.text:00000000                 AREA .text, CODE
...
.text:00000000 main
.text:00000000                 PUSH    {R4,LR}
.text:00000002                 ADR     R0, aEnterX     ; "Enter X:\n"
.text:00000004                 BL      __2printf
.text:00000008                 LDR     R1, =x
.text:0000000A                 ADR     R0, aD          ; "%d"
.text:0000000C                 BL      __0scanf
.text:00000010                 LDR     R0, =x
.text:00000012                 LDR     R1, [R0]
.text:00000014                 ADR     R0, aYouEnteredD___ ; "You entered %d...\n"
.text:00000016                 BL      __2printf
.text:0000001A                 MOVS    R0, #0
.text:0000001C                 POP     {R4,PC}
...
.text:00000020 aEnterX         DCB "Enter X:",0xA,0    ; DATA XREF: main+2
.text:0000002A                 DCB    0
.text:0000002B                 DCB    0
.text:0000002C off_2C          DCD x                   ; DATA XREF: main+8
.text:0000002C                                         ; main+10
.text:00000030 aD              DCB "%d",0              ; DATA XREF: main+A
.text:00000033                 DCB    0
.text:00000034 aYouEnteredD___ DCB "You entered %d...",0xA,0 ; DATA XREF: main+14
.text:00000047                 DCB 0
.text:00000047 ; .text         ends
.text:00000047
...
.data:00000048 ; Segment type: Pure data
.data:00000048                 AREA .data, DATA
.data:00000048                 ; ORG 0x48
.data:00000048                 EXPORT x
.data:00000048 x               DCD 0xA                 ; DATA XREF: main+8
.data:00000048                                         ; main+10
.data:00000048 ; .data         ends
\end{lstlisting}

\RU{Итак, переменная \TT{x} теперь глобальная, и она расположена, почему-то, в другом сегменте, 
а именно сегменте данных}\EN{So, \TT{x} variable is now global and somehow,
it is now located in another segment, namely data segment} (\IT{.data}).
\RU{Можно спросить, почему текстовые строки расположены в сегменте кода (\IT{.text}) 
а \TT{x} нельзя было разместить тут же?}\EN{One could ask, why text strings are located in code segment
(\IT{.text}) and \TT{x} can be located right here?}
\RU{Потому что эта переменная, и как следует из определения, она может меняться. 
И может даже быть, меняться часто.}
\EN{Since this is variable, and by its definition, it can be changed. And probably, can be changed 
very often.}
\index{\RAM}
\index{\ROM}
\RU{Сегмент кода нередко может быть расположен в ПЗУ микроконтроллера (не забывайте, 
мы сейчас имеем дело с embedded-микроэлектроникой, где дефицит памяти ~--- это обычное дело),
а изменяемые переменные ~--- в ОЗУ.}
\EN{Segment of code not infrequently can be located in microcontroller ROM (remember, we now deal
with embedded microelectronics, and memory scarcity is common here), and changeable 
variables~---in \ac{RAM}.}
\RU{Хранить в ОЗУ неизменяемые данные, когда в наличии есть ПЗУ, не экономно.}
\EN{It is not very economically to store constant variables in RAM when one have ROM.}
\RU{К тому же, сегмент данных в ОЗУ с константами нужно было бы инициализировать перед работой,
ведь, после включения ОЗУ, очевидно, она содержит в себе случайную информацию.}
\EN{Furthermore, data segment with constants in RAM must be initialized before, 
since after RAM turning on, obviously, it contain random information.}

\index{\RU{Компоновщик}\EN{Linker}}
\RU{Далее, мы видим, в сегменте кода, хранится указатель на переменную \TT{x} (\TT{off\_2C}) и вообще, 
все операции с переменной, происходят через этот указатель.}
\EN{Onwards, we see, in code segment, a pointer to the \TT{x} (\TT{off\_2C}) variable, and all
operations with variable occurred via this pointer.}
\RU{Это связано с тем что переменная \TT{x} может быть расположена где-то довольно далеко от 
данного участка кода, так что её адрес нужно сохранить в непосредственной близости к этому коду.}
\EN{This is because \TT{x} variable can be located somewhere far from this code fragment, so its address
must be saved somewhere in close proximity to the code.}
\index{ARM!\Instructions!LDR}
\RU{Инструкция \TT{LDR} в thumb-режиме может адресовать только переменные в пределах вплоть 
до 1020 байт от места где она находится.}
\EN{\TT{LDR} instruction in thumb mode can address only variable in range of 1020 bytes from the point
it is located.}
\RU{Эта же инструкция в ARM-режиме ~--- переменные в пределах $\pm{}4095$ байт, таким образом,
адрес глобальной переменной \TT{x} нужно расположить в непосредственной близости, ведь нет никакой гарантии, 
что компоновщик\footnote{linker в англоязычной литературе} сможет разместить саму переменную где-то рядом, 
она может быть даже в другом чипе памяти!}
\EN{Same instruction in ARM-mode~---variables in range $\pm{}4095$ bytes, this, address of the \TT{x} variable
must be located somewhere in close proximity, because, there is no guarantee the linker will able to place
this variable near the code, it could be even in external memory chip!}

\index{\CLanguageElements!const}
\index{\ROM}
\RU{Еще одна вещь: если переменную объявить как \IT{const}, то компилятор Keil разместит её в 
сегменте \TT{.constdata}.}
\EN{One more thing: if variable will be declared as \IT{const}, Keil compiler shall allocate it in 
the \TT{.constdata} segment.}
\RU{Должно быть, впоследствии, компоновщик и этот сегмент сможет разместить в ПЗУ, вместе
с сегментом кода.}
\EN{Perhaps, thereafter, linker will able to place this segment in ROM too, along with code segment.}

