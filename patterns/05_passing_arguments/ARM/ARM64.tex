\subsection{ARM64}

\subsubsection{\Optimizing GCC (Linaro) 4.9}

\index{Fused multiply–add}
\index{ARM!\Instructions!MADD}
\RU{Тут всё просто}\EN{Everything here is simple}.
\EN{\TT{MADD} is just an instruction doing fused multiply/add (similar to the \TT{MLA} we already saw).}
\RU{\TT{MADD} это просто инструкция, производящая умножение и сложение одновременно (как \TT{MLA}, 
которую мы уже видели).}
\EN{All 3 arguments are passed in the 32-bit parts of X-registers.}
\RU{Все 3 аргумента передаются в 32-битных частях X-регистров.}
\EN{Indeed, the argument types are 32-bit \IT{int}'s.}
\RU{Действительно, типы аргументов это 32-битные \IT{int}'ы.}
\EN{The result is returned in \TT{W0}.}
\RU{Результат возвращается в \TT{W0}.}

\lstinputlisting[caption=\Optimizing GCC (Linaro) 4.9]{patterns/05_passing_arguments/ARM/ARM64_O3.s.\LANG}

\EN{I also extended all data types to 64-bit \TT{uint64\_t} and tested:}
\RU{Я также расширил все типы данных до 64-битных \TT{uint64\_t} и попробовал:}

\lstinputlisting{patterns/05_passing_arguments/ex64.c}

\begin{lstlisting}
f:
	madd	x0, x0, x1, x2
	ret
main:
	mov	x1, 13396
	adrp	x0, .LC8
	stp	x29, x30, [sp, -16]!
	movk	x1, 0x27d0, lsl 16
	add	x0, x0, :lo12:.LC8
	movk	x1, 0x122, lsl 32
	add	x29, sp, 0
	movk	x1, 0x58be, lsl 48
	bl	printf
	mov	w0, 0
	ldp	x29, x30, [sp], 16
	ret

.LC8:
	.string	"%lld\n"
\end{lstlisting}

\EN{The \ttf{} function is the same, only the whole 64-bit X-registers are now used.}
\RU{Функция \ttf{} точно такая же, только теперь используются полные части 64-битных X-регистров.}
\RU{Длинные 64-битные значения загружаются в регистры по частям, я описал это здесь}%
\EN{Long 64-bit values are loaded into the registers by parts, I have described it also here}: \myref{ARM_big_constants_loading}.

\subsubsection{\NonOptimizing GCC (Linaro) 4.9}

\EN{The non-optimizing compiler is more redundant:}
\RU{Неоптимизирующий компилятор выдает немного лишнего кода:}

\begin{lstlisting}
f:
	sub	sp, sp, #16
	str	w0, [sp,12]
	str	w1, [sp,8]
	str	w2, [sp,4]
	ldr	w1, [sp,12]
	ldr	w0, [sp,8]
	mul	w1, w1, w0
	ldr	w0, [sp,4]
	add	w0, w1, w0
	add	sp, sp, 16
	ret
\end{lstlisting}

\EN{The code saves its input arguments in the local stack, 
in case someone (or something) in this function needs using the \TT{W0...W2} 
registers. This prevents overwriting the original
function arguments, which may be needed again in the future.}
\RU{Код сохраняет входные аргументы в локальном стеке на случай если кому-то (или чему-то) в этой функции
понадобится использовать регистры \TT{W0...W2}, перезаписывая оригинальные аргументы функции, которые
могут понадобится в будущем.}
\RU{Это называется}\EN{This is called} \IT{Register Save Area.} \cite{ARM64_PCS}
\RU{Вызываемая функция не обязана сохранять их.}\EN{ The callee, however, is not obliged to save them.}
\RU{Это то же что и}\EN{This is somewhat similar to} ``Shadow Space'': \myref{shadow_space}.

\RU{Почему оптимизирующий GCC 4.9 убрал этот, сохраняющий аргументы, код?}
\EN{Why did the optimizing GCC 4.9 drop this argument saving code?}
\EN{Because it did some additional optimizing work and concluded
that the function arguments will not be needed in the future 
and also that the registers \TT{W0...W2} will not be used.}
\RU{Потому что он провел дополнительную работу по оптимизации и сделал вывод, 
что аргументы функции не понадобятся в будущем и регистры \TT{W0...W2} также не будут использоваться.}

\index{ARM!\Instructions!MUL}
\index{ARM!\Instructions!ADD}
\RU{Также мы видим пару инструкций \TT{MUL}/\TT{ADD} вместо одной \TT{MADD}.}
\EN{We also see a \TT{MUL}/\TT{ADD} instruction pair instead of single a \TT{MADD}.}
