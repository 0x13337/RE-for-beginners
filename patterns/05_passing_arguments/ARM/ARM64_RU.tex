\subsectionold{ARM64}

\subsubsectionold{\Optimizing GCC (Linaro) 4.9}

\myindex{Fused multiply–add}
\myindex{ARM!\Instructions!MADD}
Тут всё просто.
\TT{MADD} это просто инструкция, производящая умножение и сложение одновременно (как \TT{MLA}, 
которую мы уже видели).
Все 3 аргумента передаются в 32-битных частях X-регистров.
Действительно, типы аргументов это 32-битные \IT{int}'ы.
Результат возвращается в \TT{W0}.

\lstinputlisting[caption=\Optimizing GCC (Linaro) 4.9]{patterns/05_passing_arguments/ARM/ARM64_O3_RU.s}

Также расширим все типы данных до 64-битных \TT{uint64\_t} и попробуем:

\lstinputlisting{patterns/05_passing_arguments/ex64.c}

\begin{lstlisting}
f:
	madd	x0, x0, x1, x2
	ret
main:
	mov	x1, 13396
	adrp	x0, .LC8
	stp	x29, x30, [sp, -16]!
	movk	x1, 0x27d0, lsl 16
	add	x0, x0, :lo12:.LC8
	movk	x1, 0x122, lsl 32
	add	x29, sp, 0
	movk	x1, 0x58be, lsl 48
	bl	printf
	mov	w0, 0
	ldp	x29, x30, [sp], 16
	ret

.LC8:
	.string	"%lld\n"
\end{lstlisting}

Функция \ttf{} точно такая же, только теперь используются полные части 64-битных X-регистров.
Длинные 64-битные значения загружаются в регистры по частям, это описано здесь: \myref{ARM_big_constants_loading}.

\subsubsectionold{\NonOptimizing GCC (Linaro) 4.9}

Неоптимизирующий компилятор выдает немного лишнего кода:

\begin{lstlisting}
f:
	sub	sp, sp, #16
	str	w0, [sp,12]
	str	w1, [sp,8]
	str	w2, [sp,4]
	ldr	w1, [sp,12]
	ldr	w0, [sp,8]
	mul	w1, w1, w0
	ldr	w0, [sp,4]
	add	w0, w1, w0
	add	sp, sp, 16
	ret
\end{lstlisting}

Код сохраняет входные аргументы в локальном стеке на случай если кому-то (или чему-то) в этой функции
понадобится использовать регистры \TT{W0...W2}, перезаписывая оригинальные аргументы функции, которые
могут понадобится в будущем.
Это называется \IT{Register Save Area.} (\ARMPCS)
Вызываемая функция не обязана сохранять их.
Это то же что и \q{Shadow Space}: \myref{shadow_space}.

Почему оптимизирующий GCC 4.9 убрал этот, сохраняющий аргументы, код?

Потому что он провел дополнительную работу по оптимизации и сделал вывод, 
что аргументы функции не понадобятся в будущем и регистры \TT{W0...W2} также не будут использоваться.

\myindex{ARM!\Instructions!MUL}
\myindex{ARM!\Instructions!ADD}
Также мы видим пару инструкций \TT{MUL}/\TT{ADD} вместо одной \TT{MADD}.

