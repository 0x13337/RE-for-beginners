\subsection{\NonOptimizingKeilVI (\ARMMode)}

\begin{lstlisting}
.text:000000A4 00 30 A0 E1                 MOV     R3, R0
.text:000000A8 93 21 20 E0                 MLA     R0, R3, R1, R2
.text:000000AC 1E FF 2F E1                 BX      LR
...
.text:000000B0             main
.text:000000B0 10 40 2D E9                 STMFD   SP!, {R4,LR}
.text:000000B4 03 20 A0 E3                 MOV     R2, #3
.text:000000B8 02 10 A0 E3                 MOV     R1, #2
.text:000000BC 01 00 A0 E3                 MOV     R0, #1
.text:000000C0 F7 FF FF EB                 BL      f
.text:000000C4 00 40 A0 E1                 MOV     R4, R0
.text:000000C8 04 10 A0 E1                 MOV     R1, R4
.text:000000CC 5A 0F 8F E2                 ADR     R0, aD_0        ; "%d\n"
.text:000000D0 E3 18 00 EB                 BL      __2printf
.text:000000D4 00 00 A0 E3                 MOV     R0, #0
.text:000000D8 10 80 BD E8                 LDMFD   SP!, {R4,PC}
\end{lstlisting}

\RU{В функции \main просто вызываются две функции, в первую (\ttf) передается три значения.}
\EN{The \main function simply calls two other functions, with three values passed to the 
first one~---(\ttf).}

\RU{Как я уже упоминал, первые 4 значения в ARM обычно передаются в 
первых 4-х регистрах (\Reg{0}-\Reg{3}).}
\EN{As I mentioned before, in ARM the first 4 values are usually passed in 
the first 4 registers (\Reg{0}-\Reg{3}).}

\EN{The }\RU{Функция }\ttf\RU{, как видно, использует три первых регистра (\Reg{0}-\Reg{2}) как аргументы.}
\EN{function, as it seems, uses the first 3 registers (\Reg{0}-\Reg{2}) as arguments.}

\index{ARM!\Instructions!MLA}
\EN{The }\RU{Инструкция }\TT{MLA} (\IT{Multiply Accumulate}) \RU{перемножает два первых операнда (\Reg{3} и \Reg{1}), 
прибавляет к произведению
третий операнд (\Reg{2}) и помещает результат в нулевой регистр (\Reg{0}), через который, по стандарту, 
возвращаются значения функций.}
\EN{instruction multiplies its first two operands (\Reg{3} and \Reg{1}), adds the third operand (\Reg{2}) to the product and stores
the result into the zeroth register (\Reg{0}), via which, by standard, functions return values.}

\index{Fused multiply–add}
\RU{Умножение и сложение одновременно}\EN{Multiplication and addition at once}\footnote{\WPMAO} 
(\IT{Fused multiply–add}) \RU{это часто применяемая операция. Кстати, аналогичной
инструкции в x86 не было до появления FMA-инструкций в SIMD}%
\EN{is a very useful operation. By the way, there was no such instruction in x86 
before FMA-instructions appeared in SIMD}%
\footnote{\href{http://go.yurichev.com/17103}{wikipedia}}.

\RU{Самая первая инструкция}\EN{The very first} \TT{MOV R3, R0}, \RU{по-видимому, избыточна (можно было бы обойтись только одной инструкцией \TT{MLA}).}
\EN{instruction is, apparently, redundant (a single \TT{MLA} instruction could be used here instead).} 
\RU{Компилятор не оптимизировал её, ведь, это компиляция без оптимизации.}
\EN{The compiler has not optimized it, since this is non-optimizing compilation.}

\index{ARM!\RU{Переключение режимов}\EN{Mode switching}}
\index{ARM!\Instructions!BX}
\RU{Инструкция \TT{BX} возвращает управление по адресу, записанному в \ac{LR} и, если нужно, 
переключает режимы процессора с Thumb на ARM или наоборот.}
\EN{The \TT{BX} instruction returns the control to the address stored in the \ac{LR} register and, if necessary, 
switches the processor mode from Thumb to ARM or vice versa.}
\RU{Это может быть необходимым потому, что, как мы видим, 
функции \ttf неизвестно, из какого кода она будет вызываться, из ARM или Thumb.}
\EN{This can be necessary since, as we can see, function \ttf is not aware from what kind of code it may be
called, ARM or Thumb.}
\RU{Поэтому, если она будет вызываться из кода Thumb, \TT{BX} не только возвращает
управление в вызывающую функцию, но также переключает процессор в режим Thumb.}
\EN{Thus, if it gets called from Thumb code, 
\TT{BX} is not only returns control to the calling function,
but also switches the processor mode to Thumb.}
\RU{Либо не переключит, если функция вызывалась из кода для режима ARM: \cite[A2.3.2]{ARMref}.}
\EN{Or not switch, if the function was called from ARM code \cite[A2.3.2]{ARMref}.}
% look for "BXWritePC()" in manual
