\subsection{\NonOptimizingKeilVI (\ARMMode)}

\begin{lstlisting}
.text:000000A4 00 30 A0 E1                 MOV     R3, R0
.text:000000A8 93 21 20 E0                 MLA     R0, R3, R1, R2
.text:000000AC 1E FF 2F E1                 BX      LR
...
.text:000000B0             main
.text:000000B0 10 40 2D E9                 STMFD   SP!, {R4,LR}
.text:000000B4 03 20 A0 E3                 MOV     R2, #3
.text:000000B8 02 10 A0 E3                 MOV     R1, #2
.text:000000BC 01 00 A0 E3                 MOV     R0, #1
.text:000000C0 F7 FF FF EB                 BL      f
.text:000000C4 00 40 A0 E1                 MOV     R4, R0
.text:000000C8 04 10 A0 E1                 MOV     R1, R4
.text:000000CC 5A 0F 8F E2                 ADR     R0, aD_0        ; "%d\n"
.text:000000D0 E3 18 00 EB                 BL      __2printf
.text:000000D4 00 00 A0 E3                 MOV     R0, #0
.text:000000D8 10 80 BD E8                 LDMFD   SP!, {R4,PC}
\end{lstlisting}

La funzione \main chiama altre due funzioni, con tre valori passati alla prima ~---(\ttf).

Come detto in precedenza, in ARM i primi 4 valori sono solitamente passati nei primi 4 registri (\Reg{0}-\Reg{3}).

La funzione \ttf function, come si deve, usa i primi 3 registri (\Reg{0}-\Reg{2}) come argomenti.

\myindex{ARM!\Instructions!MLA}
L'istruzione \TT{MLA} (\IT{Multiply Accumulate}) 
moltiplica i suoi primi due operandi (\Reg{3} e \Reg{1}), aggiunge al prodotto il terzo operando (\Reg{2}) e salva il risultato
nel zeresimo registro (\Reg{0}), attraverso il quale, da standard, le funzioni restituiscono i valori.

\myindex{Fused multiply–add}
La moltiplicazione e addizione fatte in una volta sola \footnote{\WPMAO} (\IT{Fused multiply–add}) e' un'operazione molto utile. 
Non vi era alcune funzione analoga in x86 prima dell'avvento delle istruzioni FMA in SIMD.
\footnote{\href{http://go.yurichev.com/17103}{wikipedia}}.

La prima istruzione \TT{MOV R3, R0}, 
e' apparentemente ridondante (al suo posto sarebbe potuta essere usata una singola istruzione \TT{MLA}). 
Il compilatore, come previsto, non ha quindi ottimizzato il codice.

\myindex{ARM!Mode switching}
\myindex{ARM!\Instructions!BX}

L'istruzione \TT{BX} restituisce il controllo all'indirizzo memorizzato nel registro \ac{LR} e, se necessario, 
effettua lo switch della modalita' del processore da Thumb a ARM o viceversa.
Cio' puo' essere necessario in quanto, come possiamo vedere ,la funzione \ttf non e' al corrente di che tipo di codice potrebbe esere chiamato in seguito (ARM o Thumb).
Dunque, se viene chiamata da codice Thumb \TT{BX} non resituisce soltanto il controllo alla funzione chiamante ma switcha anche la modalita' del processore a Thumb. 
Se la funzione viene chiamata da codice ARM, non effettua lo scambio di modalita' [\ARMSevenRef A2.3.2].
% look for "BXWritePC()" in manual
