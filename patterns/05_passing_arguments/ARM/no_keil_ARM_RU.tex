\subsection{\NonOptimizingKeilVI (\ARMMode)}

\begin{lstlisting}
.text:000000A4 00 30 A0 E1                 MOV     R3, R0
.text:000000A8 93 21 20 E0                 MLA     R0, R3, R1, R2
.text:000000AC 1E FF 2F E1                 BX      LR
...
.text:000000B0             main
.text:000000B0 10 40 2D E9                 STMFD   SP!, {R4,LR}
.text:000000B4 03 20 A0 E3                 MOV     R2, #3
.text:000000B8 02 10 A0 E3                 MOV     R1, #2
.text:000000BC 01 00 A0 E3                 MOV     R0, #1
.text:000000C0 F7 FF FF EB                 BL      f
.text:000000C4 00 40 A0 E1                 MOV     R4, R0
.text:000000C8 04 10 A0 E1                 MOV     R1, R4
.text:000000CC 5A 0F 8F E2                 ADR     R0, aD_0        ; "%d\n"
.text:000000D0 E3 18 00 EB                 BL      __2printf
.text:000000D4 00 00 A0 E3                 MOV     R0, #0
.text:000000D8 10 80 BD E8                 LDMFD   SP!, {R4,PC}
\end{lstlisting}

В функции \main просто вызываются две функции, в первую (\ttf) передается три значения.
Как уже было упомянуто, первые 4 значения в ARM обычно передаются в первых 4-х регистрах (\Reg{0}-\Reg{3}).
Функция \ttf, как видно, использует три первых регистра (\Reg{0}-\Reg{2}) как аргументы.

\myindex{ARM!\Instructions!MLA}
Инструкция \TT{MLA} (\IT{Multiply Accumulate}) перемножает два первых операнда (\Reg{3} и \Reg{1}), 
прибавляет к произведению
третий операнд (\Reg{2}) и помещает результат в нулевой регистр (\Reg{0}), через который, по стандарту, 
возвращаются значения функций.

\myindex{Fused multiply–add}
Умножение и сложение одновременно\footnote{\WPMAO} 
(\IT{Fused multiply–add}) это часто применяемая операция. Кстати, аналогичной
инструкции в x86 не было до появления FMA-инструкций в SIMD
\footnote{\href{http://go.yurichev.com/17103}{wikipedia}}.

Самая первая инструкция \TT{MOV R3, R0}, по-видимому, избыточна (можно было бы обойтись только одной инструкцией \TT{MLA}).
Компилятор не оптимизировал её, ведь, это компиляция без оптимизации.

\myindex{ARM!Переключение режимов}
\myindex{ARM!\Instructions!BX}
Инструкция \TT{BX} возвращает управление по адресу, записанному в \ac{LR} и, если нужно, 
переключает режимы процессора с Thumb на ARM или наоборот.
Это может быть необходимым потому, что, как мы видим, 
функции \ttf неизвестно, из какого кода она будет вызываться, из ARM или Thumb.
Поэтому, если она будет вызываться из кода Thumb, \TT{BX} не только возвращает
управление в вызывающую функцию, но также переключает процессор в режим Thumb.
Либо не переключит, если функция вызывалась из кода для режима ARM: [\ARMSevenRef A2.3.2].

% look for "BXWritePC()" in manual
