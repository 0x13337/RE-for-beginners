\section{MIPS}

\lstinputlisting[caption=\Optimizing GCC 4.4.5]{patterns/05_passing_arguments/MIPS_O3_IDA.lst.\LANG}

\RU{Первые 4 аргумента ф-ции передаются в четырех регистрах с префиксами A-.}
\EN{The first four function arguments are passed in four registers prefixed by A-.}

\index{MIPS!\Instructions!MULT}
\RU{В MIPS есть два специальных регистра: HI и LO, которые выставляются в 64-битный результат умножения
во время исполнения инструкции \TT{MULT}.}
\EN{There are two special registers in MIPS: HI and LO which are filled with the 64-bit result of the multiplication during the execution of the \TT{MULT} instruction.}
\index{MIPS!\Instructions!MFLO}
\index{MIPS!\Instructions!MFHI}
\RU{К регистрам можно обращаться только используя инструкции \TT{MFLO} и \TT{MFHI}.}
\EN{These registers are accessible only by using the \TT{MFLO} and \TT{MFHI} instructions.}
\RU{Здесь, \TT{MFLO} берет младшую часть результата умножения и записывает в \$V0.}
\EN{\TT{MFLO} here takes the low-part of the multiplication result and stores it into \$V0.}

\RU{Так что старшая 32-битная часть результата игнорируется (содержимое регистра HI не используется).}
\EN{So the high 32-bit part of the multiplication result is dropped (the HI register content is not used).}
\RU{Действительно: мы ведь работаем с 32-битным типом \Tint.}
\EN{Indeed: we work with 32-bit \Tint data types here.}

\index{MIPS!\Instructions!ADDU}
\RU{И наконец, \TT{ADDU} (``Add Unsigned'') прибавляет значение третьего аргумента к результату.}
\EN{Finally, \TT{ADDU} (``Add Unsigned'') adds the value of the third argument to the result.}

\index{MIPS!\Instructions!ADD}
\index{MIPS!\Instructions!ADDU}
\index{Ada}
\index{Integer overflow}
\RU{В MIPS есть две разных инструкции сложения:}
\EN{There are two different addition instructions in MIPS:} \TT{ADD} \AndENRU \TT{ADDU}.
\RU{На самом деле, дело не в знаковых числах, но в исключениях: \TT{ADD} может вызвать исключение
во время переполнения, а это иногда полезно\footnote{\url{http://go.yurichev.com/17326}} и поддерживается,
например, в \ac{PL} Ada.}
\EN{The difference between them is not related to signedness, but to exceptions. \TT{ADD} can raise an exception on overflow, which is sometimes useful\footnote{\url{http://go.yurichev.com/17326}} and supported in Ada \ac{PL}, for instance.}
\TT{ADDU} \RU{не вызывает исключения во время переполнения}\EN{does not raise exceptions on overflow}.
\RU{А так как \CCpp не поддерживает всё это, мы видим здесь \TT{ADDU} вместо \TT{ADD}.}
\EN{Since \CCpp does not support this, in our example we see \TT{ADDU} instead of \TT{ADD}.}

\RU{32-битный результат оставляется в}\EN{The 32-bit result is left in} \$V0.

\index{MIPS!\Instructions!JAL}
\index{MIPS!\Instructions!JALR}
\RU{В \main есть новая для нас инструкция:}
\EN{There is a new instruction for us in \main:} \TT{JAL} (``Jump and Link''). 
\RU{Разница между JAL и JALR в том, что относительное смещение кодируется в первой инструкции,
а JALR переходит по абсолютному адресу, записанному в регистр (``Jump and Link Register'').}
\EN{The difference between JAL and JALR is that a relative offset is encoded in the first instruction, 
while JALR jumps to the absolute address stored in a register (``Jump and Link Register'').}
\RU{Обе ф-ции \ttf и \main расположены в одном объектном файле, так что относительный адрес
\ttf известен и фиксирован.}
\EN{Both \ttf and \main functions are located in the same object file, so the relative address of \ttf 
is known and fixed.}
