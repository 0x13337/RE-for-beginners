\section{x64}

\index{x86-64}
\RU{В x86-64 всё немного иначе, здесь аргументы функции (4 или 6) передаются через регистры, 
а \gls{callee} из читает их из регистров, а не из стека.}
\EN{The story is a bit different in x86-64. Function arguments (first 4 or first 6 of them) 
are passed in registers i.e. the \gls{callee} reads them from registers instead of reading them from the stack.}

\subsection{MSVC}

\Optimizing MSVC:

\lstinputlisting[caption=\Optimizing MSVC 2012 x64]{patterns/05_passing_arguments/x64_MSVC_Ox.asm.\LANG}

\RU{Как видно, очень компактная функция \ttf берет аргументы прямо из регистров.}
\EN{As we can see, the compact function \ttf takes all its arguments from the registers.}
\RU{Инструкция \LEA используется здесь для сложения чисел. 
Должно быть компилятор посчитал, что это будет эффективнее использования \TT{ADD}.}
\EN{The \LEA instruction here is used for addition,
apparently the compiler considered it faster than \TT{ADD}.}
\index{x86!\Instructions!LEA}
\RU{В самой \main{} \LEA{} также используется для подготовки первого и третьего аргумента: должно быть,
компилятор решил, что \LEA{} будет работать здесь быстрее, чем загрузка значения в регистр при помощи \MOV.}
\EN{\LEA is also used in the \main function to prepare the first and third \ttf arguments. The compiler
must have decided that this would work faster than the usual way of loading values into a register using \MOV instruction.}

\RU{Попробуем посмотреть вывод неоптимизирующего MSVC}\EN{Let's take a look at the non-optimizing MSVC output}:

\lstinputlisting[caption=MSVC 2012 x64]{patterns/05_passing_arguments/x64_MSVC_IDA.asm.\LANG}

\RU{Немного путаннее: все 3 аргумента из регистров зачем-то сохраняются в стеке.}
\EN{It looks somewhat puzzling because all 3 arguments from the registers are saved to the stack for some reason.}
\index{Shadow space}
\label{shadow_space}
\RU{Это называется}\EN{This is called} \q{shadow space}
\footnote{\href{http://go.yurichev.com/17256}{MSDN}}: 
\RU{каждая функция в Win64 может (хотя и не обязана) сохранять значения 4-х регистров там.}
\EN{every Win64 may (but is not required to) save all 4 register values there.}
\RU{Это делается по крайней мере из-за двух причин}\EN{This is done for two reasons}: 
1) \RU{в большой функции отвести целый регистр (а тем более 4 регистра) для входного аргумента 
слишком расточительно, так что к нему будет обращение через стек;}
\EN{it is too lavish to allocate a whole register (or even 4 registers) for an input argument,
so it will be accessed via stack;}
2) \RU{отладчик всегда знает, где найти аргументы функции в момент останова}\EN{the debugger is always
aware where to find the function arguments at a break}%
\footnote{\href{http://go.yurichev.com/17257}{MSDN}}.

\RU{Так что, какие-то большие функции могут сохранять входные аргументы в \q{shadows space} 
для использования в будущем, а небольшие функции, как наша, могут этого и не делать.}
\EN{So, some large functions can save their input arguments in the \q{shadows space} if they need to use them
during execution, but some small functions (like ours) may not do this.}

\RU{Место в стеке для \q{shadow space} выделяет именно \gls{caller}.}
\EN{It is a \gls{caller} responsibility to allocate \q{shadow space} in the stack.}

\ifdefined\IncludeGCC
\subsection{GCC}

\Optimizing GCC \RU{также делает понятный код}\EN{generates more or less understandable code}:

\lstinputlisting[caption=\Optimizing GCC 4.4.6 x64]{patterns/05_passing_arguments/x64_GCC_O3.s.\LANG}

\NonOptimizing GCC:

\lstinputlisting[caption=GCC 4.4.6 x64]{patterns/05_passing_arguments/x64_GCC.s.\LANG}

\index{Shadow space}
\RU{В соглашении о вызовах System V *NIX\cite{SysVABI} нет \q{shadow space}, но \gls{callee} тоже иногда
должен сохранять где-то аргументы, потому что, опять же, регистров может и не хватить на все действия.
Что мы здесь и видим.}
\EN{There are no \q{shadow space} requirements in System V *NIX\cite{SysVABI}, but the \gls{callee} may need to save
its arguments somewhere in case of registers shortage.}

\subsection{GCC: uint64\_t \RU{вместо}\EN{instead of} int}

\RU{Наш пример работал с 32-битным \Tint, поэтому использовались 32-битные части регистров с префиксом \TT{E-}.}
\EN{Our example works with 32-bit \Tint, that is why 32-bit register parts are used (prefixed by \TT{E-}).}

\RU{Его можно немного переделать, чтобы он заработал с 64-битными значениями}\EN{It can be altered slightly
in order to use 64-bit values}:

\lstinputlisting{patterns/05_passing_arguments/ex64.c}

\lstinputlisting[caption=\Optimizing GCC 4.4.6 x64]{patterns/05_passing_arguments/ex64_GCC_O3_IDA.asm.\LANG}

\RU{Собствено, всё то же самое, только используются регистры \IT{целиком}, с префиксом \TT{R-}.}
\EN{The code is the same, but this time the \IT{full size} registers (prefixed by \TT{R-}) are used.}
\fi
