\section{x64}

\myindex{x86-64}

The story is a bit different in x86-64. Function arguments (first 4 or first 6 of them) 
are passed in registers i.e. the \gls{callee} reads them from registers instead of reading them from the stack.

\subsection{MSVC}

\Optimizing MSVC:

\lstinputlisting[caption=\Optimizing MSVC 2012 x64]{patterns/05_passing_arguments/x64_MSVC_Ox_EN.asm}

As we can see, the compact function \ttf takes all its arguments from the registers.

The \LEA instruction here is used for addition,
apparently the compiler considered it faster than \TT{ADD}.
\myindex{x86!\Instructions!LEA}

\LEA is also used in the \main function to prepare the first and third \ttf arguments. The compiler
must have decided that this would work faster than the usual way of loading values into a register using \MOV instruction.

Let's take a look at the non-optimizing MSVC output:

\lstinputlisting[caption=MSVC 2012 x64]{patterns/05_passing_arguments/x64_MSVC_IDA_EN.asm}

It looks somewhat puzzling because all 3 arguments from the registers are saved to the stack for some reason.
\myindex{Shadow space}
\label{shadow_space}
This is called \q{shadow space}
\footnote{\href{http://go.yurichev.com/17256}{MSDN}}: 
every Win64 may (but is not required to) save all 4 register values there.
This is done for two reasons: 
1) it is too lavish to allocate a whole register (or even 4 registers) for an input argument,
so it will be accessed via stack;
2) the debugger is always aware where to find the function arguments at a break
\footnote{\href{http://go.yurichev.com/17257}{MSDN}}.

So, some large functions can save their input arguments in the \q{shadows space} if they need to use them
during execution, but some small functions (like ours) may not do this.

It is a \gls{caller} responsibility to allocate \q{shadow space} in the stack.

\subsection{GCC}

\Optimizing GCC generates more or less understandable code:

\lstinputlisting[caption=\Optimizing GCC 4.4.6 x64]{patterns/05_passing_arguments/x64_GCC_O3_EN.s}

\NonOptimizing GCC:

\lstinputlisting[caption=GCC 4.4.6 x64]{patterns/05_passing_arguments/x64_GCC_EN.s}

\myindex{Shadow space}

There are no \q{shadow space} requirements in System V *NIX\cite{SysVABI}, but the \gls{callee} may need to save
its arguments somewhere in case of registers shortage.

\subsection{GCC: uint64\_t instead of int}


Our example works with 32-bit \Tint, that is why 32-bit register parts are used (prefixed by \TT{E-}).

It can be altered slightly in order to use 64-bit values:

\lstinputlisting{patterns/05_passing_arguments/ex64.c}

\lstinputlisting[caption=\Optimizing GCC 4.4.6 x64]{patterns/05_passing_arguments/ex64_GCC_O3_IDA_EN.asm}

The code is the same, but this time the \IT{full size} registers (prefixed by \TT{R-}) are used.

