\subsection{x86}

\subsubsection{MSVC}

Рассмотрим пример, скомпилированный в (MSVC 2010 Express):

\lstinputlisting[label=src:passing_arguments_ex_MSVC_cdecl,caption=MSVC 2010 Express]{patterns/05_passing_arguments/msvc_RU.asm}

\myindex{x86!\Registers!EBP}
Итак, здесь видно: в функции \main заталкиваются три числа в стек и вызывается функция \TT{f(int,int,int)}.
 
Внутри \ttf доступ к аргументам, также как и к локальным переменным, происходит через макросы: 
\TT{\_a\$ = 8}, но разница в том, что эти смещения со знаком \IT{плюс}, 
таким образом если прибавить макрос \TT{\_a\$} к указателю на \EBP, то адресуется \IT{внешняя} 
часть \glslink{stack frame}{фрейма} стека относительно \EBP.

\myindex{x86!\Instructions!IMUL}
\myindex{x86!\Instructions!ADD}
Далее всё более-менее просто: значение $a$ помещается в \EAX. 
Далее \EAX умножается при помощи инструкции \IMUL на то, что лежит в \TT{\_b}, 
и в \EAX остается \glslink{product}{произведение} этих двух значений.

Далее к регистру \EAX прибавляется то, что лежит в \TT{\_c}.

Значение из \EAX никуда не нужно перекладывать, оно уже лежит где надо. 
Возвращаем управление вызываемой функции~--- она возьмет значение из \EAX и отправит его в \printf.

\subsubsection{MSVC + \olly}
\myindex{\olly}
Проиллюстрируем всё это в \olly.
Когда мы протрассируем до первой инструкции в \ttf, которая использует какой-то из аргументов
(первый), мы увидим, что \EBP указывает на \glslink{stack frame}{фрейм стека}. Он выделен красным прямоугольником.

Самый первый элемент \glslink{stack frame}{фрейма стека}~--- это сохраненное значение \EBP, 
затем \ac{RA}. Третий элемент это первый аргумент функции, затем второй аргумент и третий.

Для доступа к первому аргументу функции нужно прибавить к \EBP 8 (2 32-битных слова).

\olly в курсе этого, так что он добавил комментарии к элементам стека вроде
\q{RETURN from} и \q{Arg1 = \dots}, итд.

N.B.: аргументы функции являются членами фрейма стека вызывающей функции, а не текущей.
Поэтому \olly отметил элементы \q{Arg} как члены другого фрейма стека.

\begin{figure}[H]
\centering
\myincludegraphics{patterns/05_passing_arguments/olly.png}
\caption{\olly: внутри функции \ttf{}}
\label{fig:passing_arguments_olly}
\end{figure}


\subsubsection{GCC}

Скомпилируем то же в GCC 4.4.1 и посмотрим результат в \IDA:

\lstinputlisting[caption=GCC 4.4.1]{patterns/05_passing_arguments/gcc_RU.asm}

Практически то же самое, если не считать мелких отличий описанных ранее.

После вызова обоих функций \glslink{stack pointer}{указатель стека} не возвращается назад, 
потому что предпоследняя инструкция \TT{LEAVE} (\myref{x86_ins:LEAVE}) делает это за один раз, в конце исполнения.

