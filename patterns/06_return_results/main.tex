\chapter{\RU{Еще о возвращаемых результатах}\EN{More about results returning}}

\index{x86!\Registers!EAX}
\index{ARM!\Registers!R0}
\RU{Результат выполнения функции в x86 обычно возвращается}
\EN{As of x86, function execution result is usually returned}
\footnote{\Seealso: 
MSDN: Return Values (C++): \url{http://go.yurichev.com/17258}}
\RU{через регистр \EAX, 
а если результат имеет тип байт или символ (\Tchar), 
то в самой младшей части \EAX ~--- \AL. Если функция возвращает число с плавающей запятой, 
то будет использован регистр FPU \ST{0}.
В ARM обычно результат возвращается в регистре \Reg{0}.}
\EN{in the \EAX register. 
If it is byte type or character (\Tchar)~---then in the lowest register \EAX part~---\AL. 
If a function returns \Tfloat number, the FPU register 
\ST{0} is to be used instead.
In ARM, result is usually returned in the \Reg{0} register.}

\section{\RU{Попытка использовать результат ф-ции возвращающей \Tvoid}
\EN{Attempt to use result of function returning \Tvoid}}

\RU{Кстати, что будет если возвращаемое значение в ф-ции \main объявлять не как \Tint а как \Tvoid?}
\EN{By the way, what if returning value of the \main function will be declared not as \Tint but as \Tvoid?}

\RU{Т.н. startup-код вызывает \main примерно так:}
\EN{so-called startup-code is calling \main roughly as:}

\begin{lstlisting}
push envp
push argv
push argc
call main
push eax
call exit
\end{lstlisting}

\RU{Т.е., иными словами:}\EN{In other words:}

\begin{lstlisting}
exit(main(argc,argv,envp));
\end{lstlisting}

\RU{Если вы объявите \main как \Tvoid, и ничего не будете возвращать явно (при помощи выражения \IT{return}), 
то в единственный аргумент exit() попадет
то, что лежало в регистре \EAX на момент выхода из \main.}
\EN{If you declare \main as \Tvoid and nothing will be returned explicitly (by \IT{return} statement),
then something random, that was stored in the \EAX register at the moment of the \main finish, will come into
the sole exit() function argument.}
\RU{Там, скорее всего, будет какие-то случайное число, оставшееся от работы вашей ф-ции.}
\EN{Most likely, there will be a random value, left from your function execution.}
\RU{Так что, код завершения программы будет псевдослучайным.}
\EN{So, exit code of program will be pseudorandom.} \\

\RU{Я могу это проиллюстрировать}\EN{I can illustrate this fact}. 
\RU{Заметьте что у ф-ции}\EN{Please notice, the} \main \RU{тип возвращаемого значения именно}\EN{function 
has} \Tvoid\EN{ type}:

\begin{lstlisting}
#include <stdio.h>

void main()
{
	printf ("Hello, world!\n");
};
\end{lstlisting}

\RU{Скомпилируем в}\EN{Let's compile it in} Linux.

\index{puts() \RU{вместо}\EN{instead of} printf()}
GCC 4.8.1 \RU{заменила}\EN{replaced} \printf \RU{на}\EN{to} \puts 
(\RU{мы видели это прежде}\EN{we saw this before}: \ref{puts}), 
\RU{но это нормально, потому что}\EN{but that's OK, since} \puts \RU{возвращает количество
выведенных символов, так же как и}\EN{returns number of characters printed, just like} \printf.
\RU{Обратите внимание на то что}\EN{Please notice that} \EAX \RU{не обнуляется перед выходом их}\EN{is not 
zeroed before} \main\EN{ finish}.
\RU{Это значит}\EN{This means}, \EAX \RU{перед выходом из}\EN{value at the} \main 
\RU{будет содержать то, что}\EN{finish will contain what} \puts \RU{оставит там}\EN{left there}.

\begin{lstlisting}[caption=GCC 4.8.1]
.LC0:
	.string	"Hello, world!"
main:
	push	ebp
	mov	ebp, esp
	and	esp, -16
	sub	esp, 16
	mov	DWORD PTR [esp], OFFSET FLAT:.LC0
	call	puts
	leave
	ret
\end{lstlisting}

\index{bash}
\RU{Напишем небольшой скрипт на bash, показывающий статус возврата (``exit status'' или ``exit code'')}
\EN{Let' s write bash script, showing exit status}:

\begin{lstlisting}[caption=tst.sh]
#!/bin/sh
./hello_world
echo $?
\end{lstlisting}

\RU{И запустим}\EN{And run it}:

\begin{lstlisting}
$ tst.sh 
Hello, world!
14
\end{lstlisting}

$14$ \RU{это как раз количество выведенных символов}\EN{is a number of characters printed}.

\section{\RU{Что если не использовать результат ф-ции?}\EN{What if not to use function result?}}

\RU{\printf возвращает количество успешно выведенных символов, но результат работы этой ф-ции 
редко используется на практике.}
\EN{\printf returns count of characters successfully sent to output, but result of this function 
is rarely used in practice.}
\RU{Можно даже явно вызывать ф-ции, чей смысл именно в возвращаемых значениях, но явно не использовать их:}
\EN{It's possible to call functions which essence in returning values, but not to use them explicitely:}

\begin{lstlisting}
int f()
{
    // skip first 3 random values
    rand();
    rand();
    rand();
    // and use 4th
    return rand();
};
\end{lstlisting}

\EN{Result of rand() function will always be leaved in \EAX, in all four cases.}
\RU{Результат работы rand() будет оставаться в \EAX во всех четырех случаях.}
\EN{But in first 3 cases, a value in \EAX will be just thrown away.}
\RU{Но в первых трех случаях, значение лежащее в \EAX, будет просто выброшено.}

\section{\RU{Возврат структуры}\EN{Returning a structure}}

\index{\CLanguageElements!return}
\RU{Вернемся к тому факту, что возвращаемое значение остается в регистре \EAX}
\EN{Let's back to the fact the returning value is left in the \EAX register}.
\RU{Вот почему старые компиляторы Си не способны создавать функции, возвращающие нечто большее нежели 
помещается 
в один регистр (обычно, тип \Tint), а когда нужно, приходится возвращать через указатели, указываемые 
в аргументах.}
\EN{That is why old C compilers cannot create functions capable of returning something not fitting in one 
register (usually type \Tint) but if one needs it, one should return information via pointers passed 
in function arguments.}
\RU{Так что, как правило, если ф-ция должна вернуть несколько значений, она возвращает только одно, 
а остальные --- через указатели.}
\EN{So, usually, if a function should return several values, it returns only one, and 
all the rest---via pointers.}
\RU{Хотя, позже и стало возможным, вернуть, скажем, целую структуру, но этот метод до сих пор не 
очень популярен. 
Если функция должна вернуть структуру, вызывающая функция должна сама, скрыто и прозрачно для программиста, 
выделить место и передать указатель на него в качестве первого аргумента. Это почти то же самое 
что и сделать это вручную, но компилятор прячет это.}
\EN{Now it is possible, to return, let's say, whole structure, but still it is not very popular. 
If function must return a large structure, \gls{caller} must allocate it and pass pointer to it via first argument, 
transparently for programmer. 
That is almost the same as to pass pointer in first argument manually, but compiler hide this.}

\RU{Небольшой пример:}\EN{Small example:}

\lstinputlisting{patterns/06_return_results/6_1.c}

\dots \RU{получим}\EN{what we got} (MSVC 2010 \Ox):

\lstinputlisting{patterns/06_return_results/6_1.asm}

\RU{Имя внутреннего макроса для передачи указателя на структуру здесь это \TT{\$T3853}.}
\EN{Macro name for internal variable passing pointer to structure is \TT{\$T3853} here.}

\index{\CLanguageElements!C99}
\RU{Этот пример можно даже переписать используя расширения C99}\EN{This example can be rewritten using
C99 language extensions}:

\lstinputlisting{patterns/06_return_results/6_1_C99.c}

\lstinputlisting[caption=GCC 4.8.1]{patterns/06_return_results/6_1_C99.asm}

\RU{Как видно, ф-ция просто заполняет поля в структуре, выделенной вызывающей ф-цией. 
Как если бы передавался просто указатель на структуру.
Так что никаких проблем с эффективностью нет.}
\EN{As we may see, the function is just filling fields in the structure, allocated by
caller function. 
As if a pointer to the structure was passed.
So there are no performance drawbacks.}
