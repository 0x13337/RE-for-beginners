\section{\IFRU{И еще немного о возвращаемых результатах}{One more word about results returning.}}

\newcommand{\MSDNURL}{\href{http://msdn.microsoft.com/en-us/library/7572ztz4.aspx}{MSDN: Return Values (C++)}}

\index{x86!\Registers!EAX}
\index{ARM!\Registers!R0}
\IFRU{Результат выполнения функции в x86 обычно возвращается\footnote{См. также: \MSDNURL} через регистр \EAX, 
а если результат имеет тип байт или символ (\IT{char}), 
то в самой младшей части \EAX ~--- \AL. Если функция возвращает число с плавающей запятой, 
то регистр FPU \STZERO будет использован.
В ARM обычно результат возвращается в регистре R0.}
{As of x86, function execution result is usually returned\footnote{See also: \MSDNURL} in 
the \EAX register. 
If it is byte type or character (\IT{char})~---then in the lowest register \EAX part~---\AL. 
If function returns \Tfloat number, the FPU register 
\STZERO is to be used instead.
In ARM, result is usually returned in the \Rzero register.} \\
\\

\IFRU{Кстати, что будет если возвращаемое значение в ф-ции \main объявлять не как \IT{int} а как \IT{void}?}
{By the way, what if returning value of the \main function will be declared not as \IT{int} but as \IT{void}?}

\IFRU{Т.н. startup-код вызывает \main примерно так:}
{So called startup-code is calling \main roughly as:}

\begin{lstlisting}
push envp
push argv
push argc
call main
push eax
call exit
\end{lstlisting}

\IFRU{Т.е., иными словами:}{In other words:}

\begin{lstlisting}
exit(main(argc,argv,envp));
\end{lstlisting}

\IFRU{Если вы объявите \main как \IT{void}, и ничего не будете возвращать явно (при помощи выражения \IT{return}), 
то в единственный аргумент exit() попадет
то, что лежало в регистре \EAX на момент выхода из \main.}
{If you declare \main as \IT{void} and nothing will be returned explicitly (by \IT{return} statement),
then something random, that was stored in the \EAX register at the moment of the \main finish, will come into
the sole exit() function argument.}
\IFRU{Там, скорее всего, будет какие-то случайное число, оставшееся от работы вашей ф-ции.}
{Most likely, there will be a random value, leaved from your function execution.}
\IFRU{Так что, код завершения программы будет псевдослучайным.}
{So, exit code of program will be pseudorandom.} \\
\\

\index{\CLanguageElements!return}
\IFRU{Вернемся к тому факту, что возвращаемое значение остается в регистре \EAX}
{Let's back to the fact the returning value is leaved in the \EAX register}.
\IFRU{Вот почему старые компиляторы Си не способны создавать функции, возвращающие нечто большее нежели 
помещается 
в один регистр (обычно, тип \Tint), а когда нужно, приходится возвращать через указатели, указываемые 
в аргументах.}
{That is why old C compilers cannot create functions capable of returning something not fitting in one 
register (usually type \Tint) but if one needs it, one should return information via pointers passed 
in function arguments.}
\IFRU{Хотя, позже и стало возможным, вернуть, скажем, целую структуру, но этот метод до сих пор не 
очень популярен. 
Если функция должна вернуть структуру, вызывающая функция должна сама, скрыто и прозрачно для программиста, 
выделить место и передать указатель на него в качестве первого аргумента. Это почти то же самое 
что и сделать это вручную, но компилятор прячет это.}
{Now it is possible, to return, let's say, whole structure, but still it is not very popular. 
If function must return a large structure, \gls{caller} must allocate it and pass pointer to it via first argument, 
transparently for programmer. 
That is almost the same as to pass pointer in first argument manually, but compiler hide this.}

\IFRU{Небольшой пример:}{Small example:}

\lstinputlisting{patterns/06_return_results/6_1.c}

\dots \IFRU{получим}{what we got} (MSVC 2010 \Ox):

\lstinputlisting{patterns/06_return_results/6_1.asm}

\IFRU{Имя внутреннего макроса для передачи указателя на структуру здесь это \TT{\$T3853}.}
{Macro name for internal variable passing pointer to structure is \TT{\$T3853} here.}

