\chapter{More about results returning}

\myindex{x86!\Registers!EAX}

In x86, the result of function execution is usually returned
\footnote{\Seealso: MSDN: Return Values (C++): \href{http://go.yurichev.com/17258}{MSDN}}
in the \EAX register. 
If it is byte type or a character (\Tchar), then the lowest part of register \EAX (\AL) is used. 
If a function returns a \Tfloat number, the FPU register \ST{0} is used instead.
\myindex{ARM!\Registers!R0}
In ARM, the result is usually returned in the \Reg{0} register.

\section{Attempt to use the result of a function returning \Tvoid}

So, what if the \main function return value was declared of type \Tvoid and not \Tint?
The so-called startup-code is calling \main roughly as follows:

\begin{lstlisting}
push envp
push argv
push argc
call main
push eax
call exit
\end{lstlisting}

In other words:

\begin{lstlisting}
exit(main(argc,argv,envp));
\end{lstlisting}

If you declare \main as \Tvoid, nothing is to be returned explicitly (using the \IT{return} statement),
then something random, that was stored in the \EAX register at the end of \main becomes 
the sole argument of the exit() function.
Most likely, there will be a random value, left from your function execution, so the exit code of program is pseudorandom.
\par
We can illustrate this fact. 
Please note that here the \main function has a \Tvoid return type:

\begin{lstlisting}
#include <stdio.h>

void main()
{
	printf ("Hello, world!\n");
};
\end{lstlisting}

Let's compile it in Linux.

\myindex{puts() instead of printf()}
GCC 4.8.1 replaced \printf with \puts 
(we have seen this before: \myref{puts}), but that's OK,
since \puts returns the number of characters printed out, just like \printf.
Please notice that \EAX is not zeroed before \main's end.

This implies that the value of \EAX at the end of \main contains what \puts has left there.

\begin{lstlisting}[caption=GCC 4.8.1]
.LC0:
	.string	"Hello, world!"
main:
	push	ebp
	mov	ebp, esp
	and	esp, -16
	sub	esp, 16
	mov	DWORD PTR [esp], OFFSET FLAT:.LC0
	call	puts
	leave
	ret
\end{lstlisting}

\myindex{bash}

Let' s write a bash script that shows the exit status:

\begin{lstlisting}[caption=tst.sh]
#!/bin/sh
./hello_world
echo $?
\end{lstlisting}

And run it:

\begin{lstlisting}
$ tst.sh 
Hello, world!
14
\end{lstlisting}

14 is the number of characters printed.

\section{What if we do not use the function result?}

\printf returns the count of characters successfully output, but the result of this function 
is rarely used in practice.

It is also possible to call a function whose essence is in returning a value, and not use it:

\begin{lstlisting}
int f()
{
    // skip first 3 random values:
    rand();
    rand();
    rand();
    // and use 4th:
    return rand();
};
\end{lstlisting}

The result of the rand() function is left in \EAX, in all four cases.

But in the first 3 cases, the value in \EAX is just not used.

\section{Returning a structure}

\myindex{\CLanguageElements!return}

Let's go back to the fact that the return value is left in the \EAX register.

That is why old C compilers cannot create functions capable of returning something that does not fit in one 
register (usually \Tint), but if one needs it, one have to return information via pointers passed 
as function's arguments.

So, usually, if a function needs to return several values, it returns only one, and 
all the rest---via pointers.

Now it has become possible to return, let's say, an entire structure, but that is still not very popular. 
If a function has to return a large structure, the \gls{caller} must allocate it and pass a pointer to it via the first argument, transparently for the programmer. 
That is almost the same as to pass a pointer in the first argument manually, but the compiler hides it.

Small example:

\lstinputlisting{patterns/06_return_results/6_1.c}

\dots what we got (MSVC 2010 \Ox):

\lstinputlisting{patterns/06_return_results/6_1.asm}

The macro name for internal passing of pointer to a structure here is \GTT{\$T3853}.

\myindex{\CLanguageElements!C99}
This example can be rewritten using
the C99 language extensions:

\lstinputlisting{patterns/06_return_results/6_1_C99.c}

\lstinputlisting[caption=GCC 4.8.1]{patterns/06_return_results/6_1_C99.asm}

As we see, the function is just filling the structure's fields allocated by
the caller function,
as if a pointer to the structure was passed.
So there are no performance drawbacks.
