\section{\RU{Вычисление абсолютной величины}\EN{Calculating absolute value}}
\label{sec:abs}

\RU{Это простая функция}\EN{A simple function}:

\lstinputlisting{abs.c}

\subsection{\Optimizing MSVC}

\RU{Обычный способ генерации кода:}
\EN{This is how the code is usually generated:}

\lstinputlisting[caption=\Optimizing MSVC 2012 x64]{patterns/07_jcc/abs/abs_MSVC2012_Ox_x64.asm.\LANG}

\ifdefined\IncludeGCC
\RU{GCC 4.9 делает почти то же самое.}
\EN{GCC 4.9 does mostly the same.}
\fi

\ifdefined\IncludeARM
\subsection{\OptimizingKeilVI: \ThumbMode}

\lstinputlisting[caption=\OptimizingKeilVI: \ThumbMode]{patterns/07_jcc/abs/abs_Keil_thumb_O3.s.\LANG}

\index{ARM!\Instructions!RSB}
\RU{В ARM нет инструкции для изменения знака, так что компилятор Keil использует инструкцию \q{Reverse Subtract},
которая просто вычитает, но с операндами, переставленными наоборот.}
\EN{ARM lacks a negate instruction, so the Keil compiler uses the \q{Reverse Subtract} instruction, which just subtracts with reversed operands.}

\subsection{\OptimizingKeilVI: \ARMMode}

\RU{В режиме ARM можно добавлять коды условий к некоторым инструкций, что компилятор Keil и сделал:}
\EN{It is possible to add condition codes to some instructions in ARM mode, so that is what the Keil compiler does:}

\lstinputlisting[caption=\OptimizingKeilVI: \ARMMode]{patterns/07_jcc/abs/abs_Keil_ARM_O3.s.\LANG}

\RU{Теперь здесь нет условных переходов и это хорошо:}
\EN{Now there are no conditional jumps and this is good:} 
\myref{branch_predictors}.

\subsection{\NonOptimizing GCC 4.9 (ARM64)}

\index{ARM!\Instructions!XOR}
\RU{В ARM64 есть инструкция NEG для смены знака:}
\EN{ARM64 has instruction NEG for negating:}

\lstinputlisting[caption=\Optimizing GCC 4.9 (ARM64)]{patterns/07_jcc/abs/abs_GCC49_ARM64_O0.s.\LANG}
\fi

\ifdefined\IncludeMIPS
\subsection{MIPS}

\lstinputlisting[caption=\Optimizing GCC 4.4.5 (IDA)]{patterns/07_jcc/abs/MIPS_O3_IDA.lst.\LANG}

\index{MIPS!\Instructions!BLTZ}
\RU{Видим здесь новую инструкцию}\EN{Here we see a new instruction}: BLTZ (\q{Branch if Less Than Zero}).
\index{MIPS!\Instructions!SUBU}
\index{MIPS!\Pseudoinstructions!NEGU}
\RU{Тут есть также псевдоинструкция NEGU, которая на самом деле вычитает из нуля.}
\EN{There is also the NEGU pseudoinstruction, which just does subtraction from zero.}
\RU{Суффикс \q{U} в обоих инструкциях SUBU и NEGU означает, что при целочисленном переполнении исключение не
сработает.}
\EN{The \q{U} suffix in both SUBU and NEGU implies that no exception to be raised in case of integer overflow.}

\fi

\ifx\LITE\undefined
\subsection{\RU{Версия без переходов}\EN{Branchless version}?}

\RU{Возможна также версия и без переходов, мы рассмотрим её позже:}
\EN{You could have also a branchless version of this code. This we will review later:} \myref{chap:branchless_abs}.
\fi
