\section{\RU{Условный оператор}\EN{Conditional operator}}
\label{chap:cond}

\RU{Условный оператор (conditional operator) в}\EN{The conditional operator in} \CCpp \RU{это}\EN{is}:

\begin{lstlisting}
expression ? expression : expression
\end{lstlisting}

\RU{И вот пример}\EN{Here is an example}:

\lstinputlisting{patterns/07_jcc/cond_operator/cond.c}

\subsection{x86}

\RU{Старые и неоптимизирующие компиляторы генерируют код так, как если бы выражение \TT{if/else} было использовано
вместо:}
\EN{Old and non-optimizing compilers generate the code just as if an \TT{if/else} statement was used:}

\lstinputlisting[caption=\NonOptimizing MSVC 2008]{patterns/07_jcc/cond_operator/MSVC2008.asm.\LANG}

\lstinputlisting[caption=\Optimizing MSVC 2008]{patterns/07_jcc/cond_operator/MSVC2008_Ox.asm.\LANG}

\RU{Более новые компиляторы могут быть более краткими}\EN{Newer compilers are more concise}:

\lstinputlisting[caption=\Optimizing MSVC 2012 x64]{patterns/07_jcc/cond_operator/MSVC2012_Ox_x64.asm.\LANG}

\index{x86!\Instructions!CMOVcc}
\Optimizing GCC 4.8 \ForENRU x86 \RU{также использует инструкцию}\EN{also uses the} \TT{CMOVcc}\RU{, тогда как
неоптимизирующий GCC 4.8 использует условные переходы}\EN{ instruction, while the non-optimizing GCC 4.8 uses 
a conditional jumps}.

\ifdefined\IncludeARM
\subsection{ARM}

\index{x86!\Instructions!ADRcc}
\Optimizing Keil \ForENRU \RU{режима ARM также использует инструкцию \TT{ADRcc}, срабатывающую при некотором
условии}\EN{ARM mode also uses the conditional instructions \TT{ADRcc}}:

\lstinputlisting[label=cond_Keil_ARM_O3,caption=\OptimizingKeilVI (\ARMMode)]{patterns/07_jcc/cond_operator/Keil_ARM_O3.s.\LANG}

\RU{Без внешнего вмешательства, инструкции \TT{ADREQ} и \TT{ADRNE} никогда не исполнятся одновременно.}
\EN{Without manual intervention, the two instructions \TT{ADREQ} and \TT{ADRNE} cannot be executed in the same run.}

\Optimizing Keil \ForENRU \RU{режима Thumb вынужден использовать инструкции условного перехода, потому
что тут нет инструкции загрузки значения поддерживающую флаги условия}\EN{Thumb mode needs to use 
conditional jump instructions, since there are no load instructions
that support conditional flags}:

\lstinputlisting[caption=\OptimizingKeilVI (\ThumbMode)]{patterns/07_jcc/cond_operator/Keil_thumb_O3.s.\LANG}

\subsection{ARM64}

\Optimizing GCC (Linaro) 4.9 \ForENRU ARM64 \RU{также использует условные переходы}\EN{also uses conditional jumps}:

\lstinputlisting[label=cond_ARM64,caption=\Optimizing GCC (Linaro) 4.9]{patterns/07_jcc/cond_operator/ARM64_GCC_O3.s.\LANG}

\RU{Это потому что в ARM64 нет простой инструкции загрузки с флагами условия, как \TT{ADRcc} в 32-битном 
режиме ARM или \TT{CMOVcc} в x86}
\EN{That's because ARM64 doesn't have a simple load instruction with conditional flags, like \TT{ADRcc} in 32-bit 
ARM mode or CMOVcc in x86}
\cite[p390, C5.5]{ARM64ref}.
\index{ARM!\Instructions!CSEL}
\RU{Но с другой стороны, там есть инструкция \TT{CSEL} (``Conditional SELect''), но GCC 4.9 наверное, пока не так
хорош, чтобы генерировать её в таком фрагменте кода.}
\EN{It has, however, ``Conditional SELect'' instruction (\TT{CSEL}), but GCC 4.9 doesn't seem to be smart enough to 
use it in such piece of code.}
\fi

\ifdefined\IncludeMIPS
\subsection{MIPS}

\RU{А GCC 4.4.5 для MIPS не так хорош, к сожалению}\EN{Unfortunately, GCC 4.4.5 for MIPS is not very smart, either}:

\lstinputlisting[caption=\Optimizing GCC 4.4.5 (\assemblyOutput)]{patterns/07_jcc/cond_operator/MIPS_O3.s.\LANG}

\fi

\subsection{\RU{Перепишем, используя обычный \TT{if/else}}\EN{Let's rewrite it in an \TT{if/else} way}}

\lstinputlisting{patterns/07_jcc/cond_operator/cond2.c}

\index{x86!\Instructions!CMOVcc}
\RU{Интересно, оптимизирующий GCC 4.8 для x86 также может генерировать \TT{CMOVcc} в этом случае:}
\EN{Interestingly, optimizing GCC 4.8 for x86 was also able to use \TT{CMOVcc} in this case:}

\lstinputlisting[caption=\Optimizing GCC 4.8]{patterns/07_jcc/cond_operator/cond2_GCC_O3.s.\LANG}

\ifdefined\IncludeARM
\Optimizing Keil \InENRU \RU{режиме ARM генерирует код идентичный этому:}\EN{ARM mode generates 
code identical to} \lstref{cond_Keil_ARM_O3}.
\fi

\RU{Но оптимизирующий}\EN{But the optimizing} MSVC 2012 \RU{пока не так хорош}\EN{is not that good (yet)}.

\subsection{\Conclusion{}}

\RU{Почему оптимизирующие компиляторы стараются избавиться от условных переходов? Читайте больше об этом здесь:}
\EN{Why optimizing compilers try to get rid of conditional jumps? Read here about it:} \ref{branch_predictors}.

\ifdefined\IncludeExercises
\subsection{\Exercise}

\RU{Попробуйте переписать код в}\EN{Try rewriting the code in} \lstref{cond_ARM64} 
\RU{убрав все инструкции условного перехода, и используйте инструкцию \TT{CSEL}}\EN{by removing all 
conditional jump instructions and using the \TT{CSEL} instruction}.
\fi
