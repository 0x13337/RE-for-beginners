\section{\RU{Поиск минимального и максимального значения}\EN{Getting minimal and maximal values}}

\subsection{32-bit}

\lstinputlisting{patterns/07_jcc/minmax/minmax.c}

\lstinputlisting[caption=\NonOptimizing MSVC 2013]{patterns/07_jcc/minmax/minmax_MSVC_2013.asm.\LANG}

\index{x86!\Instructions!Jcc}
\RU{Эти две функции отличаются друг от друга только инструкцией условного перехода:
JGE (\q{Jump if Greater or Equal}~--- переход если больше или равно) используется в первой
и JLE (\q{Jump if Less or Equal}~--- переход если меньше или равно) во второй.}
\EN{These two functions differ only in the conditional jump instruction: 
JGE (\q{Jump if Greater or Equal}) is used in the first one
and JLE (\q{Jump if Less or Equal}) in the second.}

\index{\CompilerAnomaly}
\label{MSVC_double_JMP_anomaly}
\RU{Здесь есть ненужная инструкция \JMP в каждой функции, которую MSVC, наверное, оставил по ошибке.}
\EN{There is one unneeded \JMP instruction in each function, which MSVC probably left by mistake.}

\ifx\LITE\undefined
\subsubsection{\RU{Без переходов}\EN{Branchless}}

\RU{ARM в режиме Thumb напоминает нам x86-код:}
\EN{ARM for Thumb mode reminds us of x86 code:}

\lstinputlisting[caption=\OptimizingKeilVI (\ThumbMode)]{patterns/07_jcc/minmax/minmax_Keil_Thumb_O3.s.\LANG}

\index{ARM!\Instructions!Bcc}
\RU{Функции отличаются только инструкцией перехода: BGT и BLT.}
\EN{The functions differ in the branching instruction: BGT and BLT.}

\RU{А в режиме ARM можно использовать условные суффиксы, так что код более плотный.}
\EN{It's possible to use conditional suffixes in ARM mode, so the code is shorter.}
\RU{MOVcc будет исполнена только если условие верно:}
\EN{MOVcc is to be executed only if the condition is met:}
\index{ARM!\Instructions!MOVcc}

\lstinputlisting[caption=\OptimizingKeilVI (\ARMMode)]{patterns/07_jcc/minmax/minmax_Keil_ARM_O3.s.\LANG}

\index{x86!\Instructions!CMOVcc}
\Optimizing GCC 4.8.1 \RU{и оптимизирующий}\EN{and optimizing} MSVC 2013 
\RU{могут использовать инструкцию CMOVcc, которая аналогична MOVcc в ARM:}
\EN{can use CMOVcc instruction, which is analogous to MOVcc in ARM:}

\lstinputlisting[caption=\Optimizing MSVC 2013]{patterns/07_jcc/minmax/minmax_GCC481_O3.s.\LANG}

\subsection{64-bit}

\lstinputlisting{patterns/07_jcc/minmax/minmax64.c}

\RU{Тут есть ненужные перетасовки значений, но код в целом понятен:}
\EN{There is some unneeded value shuffling, but the code is comprehensible:}

\lstinputlisting[caption=\NonOptimizing GCC 4.9.1 ARM64]{patterns/07_jcc/minmax/minmax64_GCC_49_ARM64_O0.s}

\subsubsection{\RU{Без переходов}\EN{Branchless}}

\RU{Нет нужды загружать аргументы функции из стека, они уже в регистрах:}
\EN{No need to load function arguments from the stack, as they are already in the registers:}

\lstinputlisting[caption=\Optimizing GCC 4.9.1 x64]{patterns/07_jcc/minmax/minmax64_GCC_49_x64_O3.s.\LANG}

MSVC 2013 \RU{делает то же самое}\EN{does almost the same}.

\index{ARM!\Instructions!CSEL}
\RU{В ARM64 есть инструкция CSEL, которая работает точно также, как и MOVcc в ARM и CMOVcc в x86, но название
другое:}
\EN{ARM64 has the CSEL instruction, which works just as MOVcc in ARM or CMOVcc in x86, just the name is different:}
\q{Conditional SELect}.

\lstinputlisting[caption=\Optimizing GCC 4.9.1 ARM64]{patterns/07_jcc/minmax/minmax64_GCC_49_ARM64_O3.s.\LANG}

\ifdefined\IncludeMIPS
\subsection{MIPS}

\RU{А GCC 4.4.5 для MIPS не так хорош, к сожалению}\EN{Unfortunately, GCC 4.4.5 for MIPS is not that good}:

\lstinputlisting[caption=\Optimizing GCC 4.4.5 (IDA)]{patterns/07_jcc/minmax/minmax_MIPS_O3_IDA.lst.\LANG}

\RU{Не забывайте о \IT{branch delay slots}: первая MOVE исполняется \IT{перед} BEQZ,
вторая MOVE исполняется только если переход не произошел.}
\EN{Do not forget about the \IT{branch delay slots}: the first MOVE is executed \IT{before} BEQZ, 
the second MOVE is executed only if the branch wasn't taken.}

\fi
\fi
