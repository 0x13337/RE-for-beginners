\subsection{MIPS}

\RU{Одна отличительная особенность MIPS это отсутствие регистра флагов.}
\EN{One distinctive MIPS feature is the absence of flags.}
\RU{Очевидно, так было сделано для упрощения анализа зависимости данных (data dependency).}
\EN{Apparently, it was done to simplify the analysis of data dependencies.}

\index{x86!\Instructions!SETcc}
\index{MIPS!\Instructions!SLT}
\index{MIPS!\Instructions!SLTU}
\RU{Так что здесь есть инструкция, похожая на SETcc в x86: SLT (\q{Set on Less Than}~--- установить если
меньше чем, знаковая версия) и SLTU (беззнаковая версия).}
\EN{There are instructions similar to SETcc in x86: SLT (\q{Set on Less Than}: signed version) and 
SLTU (unsigned version).}
\RU{Эта инструкция устанавливает регистр-получатель в 1 если условие верно или в 0 в противном случае.}
\EN{These instructions sets destination register value to 1 if the condition is true or to 0 if otherwise.}

\index{MIPS!\Instructions!BEQ}
\index{MIPS!\Instructions!BNE}
\RU{Затем регистр-получатель проверяется, используя инструкцию 
BEQ (\q{Branch on Equal} --- переход если равно) или BNE (\q{Branch on Not Equal} --- переход если не равно) 
и может произойти переход.}
\EN{The destination register is then checked using BEQ (\q{Branch on Equal}) or BNE (\q{Branch on Not Equal}) 
and a jump may occur.}

\RU{Так что эта пара инструкций должна использоваться в MIPS для сравнения и перехода.}
\EN{So, this instruction pair has to be used in MIPS for comparison and branch.}

\RU{Начнем с знаковой версии нашей функции:}
\EN{Let's first start with the signed version of our function:}

\lstinputlisting[caption=\NonOptimizing GCC 4.4.5 (IDA)]{patterns/07_jcc/simple/O0_MIPS_signed_IDA.lst.\LANG}

\q{SLT REG0, REG0, REG1} \RU{сокращается в IDA до более короткой формы}\EN{is reduced by IDA to its 
shorter form} \q{SLT REG0, REG1}.
\index{MIPS!\Pseudoinstructions!BEQZ}
\RU{Мы также видим здесь псевдоинструкцию BEQZ (\q{Branch if Equal to Zero}~--- переход если равно нулю), 
которая, на самом деле, \q{BEQ REG, \$ZERO, LABEL}.}
\EN{We also see there BEQZ pseudoinstruction (\q{Branch if Equal to Zero}), 
which are in fact \q{BEQ REG, \$ZERO, LABEL}.}

\index{MIPS!\Instructions!SLTU}
\RU{Беззнаковая версия точно такая же, только здесь используется SLTU (беззнаковая версия, 
отсюда \q{U} в названии) вместо SLT:}
\EN{The unsigned version is just the same, but SLTU (unsigned version, hence \q{U} in name) is used instead of SLT:}

\lstinputlisting[caption=\NonOptimizing GCC 4.4.5 (IDA)]{patterns/07_jcc/simple/O0_MIPS_unsigned_IDA.lst}
