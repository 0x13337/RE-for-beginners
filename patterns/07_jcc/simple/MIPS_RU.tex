\subsection{MIPS}

Одна отличительная особенность MIPS это отсутствие регистра флагов.
Очевидно, так было сделано для упрощения анализа зависимости данных (data dependency).

\myindex{x86!\Instructions!SETcc}
\myindex{MIPS!\Instructions!SLT}
\myindex{MIPS!\Instructions!SLTU}
Так что здесь есть инструкция, похожая на \INS{SETcc} в x86: \INS{SLT} (\q{Set on Less Than}~--- установить если
меньше чем, знаковая версия) и \INS{SLTU} (беззнаковая версия).
Эта инструкция устанавливает регистр-получатель в 1 если условие верно или в 0 в противном случае.

\myindex{MIPS!\Instructions!BEQ}
\myindex{MIPS!\Instructions!BNE}
Затем регистр-получатель проверяется, используя инструкцию 
\INS{BEQ} (\q{Branch on Equal} --- переход если равно) или \INS{BNE} (\q{Branch on Not Equal} --- переход если не равно) 
и может произойти переход.
Так что эта пара инструкций должна использоваться в MIPS для сравнения и перехода.
Начнем с знаковой версии нашей функции:

\lstinputlisting[caption=\NonOptimizing GCC 4.4.5 (IDA)]{patterns/07_jcc/simple/O0_MIPS_signed_IDA_RU.lst}

\INS{SLT REG0, REG0, REG1} сокращается в IDA до более короткой формы \INS{SLT REG0, REG1}.
\myindex{MIPS!\Pseudoinstructions!BEQZ}
Мы также видим здесь псевдоинструкцию \INS{BEQZ} (\q{Branch if Equal to Zero}~--- переход если равно нулю), 
которая, на самом деле, \INS{BEQ REG, \$ZERO, LABEL}.

\myindex{MIPS!\Instructions!SLTU}
Беззнаковая версия точно такая же, только здесь используется \INS{SLTU} (беззнаковая версия, 
отсюда \q{U} в названии) вместо \INS{SLT}:

\lstinputlisting[caption=\NonOptimizing GCC 4.4.5 (IDA)]{patterns/07_jcc/simple/O0_MIPS_unsigned_IDA.lst}

