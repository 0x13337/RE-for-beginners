\subsection{ARM: \OptimizingKeilVI (\ARMMode)}
\index{\CLanguageElements!switch}

\lstinputlisting{patterns/08_switch/1_few/few_ARM_ARM_O3.asm}

\RU{Мы снова не сможем сказать, глядя на этот код, был ли в оригинальном исходном коде switch() 
либо же несколько операторов if().}
\EN{Again, by investigating this code we cannot say if it was a switch() in the original source code, 
or just a pack of if() statements.}

\index{ARM!\Instructions!ADRcc}
\RU{Так или иначе, мы снова видим здесь инструкции с предикатами, например, \ADREQ (\IT{(Equal)}), 
которая будет исполняться только
если $R0=0$, и тогда в \Reg{0} будет загружен адрес строки \IT{<<zero\textbackslash{}n>>}.}
\EN{Anyway, we see here predicated instructions again (like \ADREQ (\IT{Equal}))
which is triggered only in case $R0=0$, and then loads the address of the string \IT{<<zero\textbackslash{}n>>}
into \Reg{0}.}
\index{ARM!\Instructions!BEQ}
\RU{Следующая инструкция}\EN{The next instruction} \ac{BEQ}
\RU{перенаправит исполнение на}
\EN{redirects control flow to} \TT{loc\_170}, \RU{если}\EN{if} $R0=0$.

\RU{Кстати, наблюдательный читатель может спросить, сработает ли \ac{BEQ} нормально,
ведь \ADREQ перед ним уже заполнила регистр \Reg{0} чем-то другим?}
\EN{An astute reader may ask, will \ac{BEQ} trigger correctly since \ADREQ before it
has already filled the \Reg{0} register with another value?}
\RU{Сработает, потому что \ac{BEQ} проверяет флаги, установленные инструкцией \CMP, 
а \ADREQ флаги никак не модифицирует.}
\EN{Yes, it will since \ac{BEQ} checks the flags set by the \CMP instruction, 
and \ADREQ does not modify any flags at all.}

\RU{Далее всё просто и знакомо.}\EN{The rest of the instructions are already familiar to us.} 
\RU{Вызов}\EN{There is only one call to} \printf \RU{один, и в самом конце, 
мы уже рассматривали подобный трюк}\EN{, at the end, and we have already examined this trick here}%
~(\myref{ARM_B_to_printf}).
\RU{К вызову функции}\EN{In the end, there are three paths to} \printf{}\RU{ в конце ведут три пути}.

\index{ARM!\Instructions!ADRcc}
\index{ARM!\Instructions!CMP}
\RU{Последняя инструкция}\EN{The last instruction,} \TT{CMP R0, \#2} 
\RU{здесь нужна, чтобы узнать $a=2$ или нет.}
\EN{, is needed to check if $a=2$.}
\RU{Если это не так, то при помощи \ADRNE (\IT{Not Equal}) в \Reg{0} будет загружен указатель на 
строку \IT{<<something unknown \textbackslash{}n>>}, ведь $a$ уже было проверено на 0 и 1 до этого, 
и здесь $a$ точно не попадает под эти константы.}
\EN{If it is not true, then \ADRNE loads a pointer to the string \IT{<<something unknown \textbackslash{}n>>} 
into \Reg{0}, since $a$ was already checked to be equal to 0 or 1,
and we can sure that the $a$ variable is not equal to these numbers at this point.}
\RU{Ну а если}\EN{And if} $R0=2$, \RU{в \Reg{0} будет загружен указатель на строку}
\EN{a pointer to the string} \IT{<<two\textbackslash{}n>>} 
\RU{при помощи инструкции \ADREQ}\EN{will be loaded by \ADREQ into \Reg{0}}.

\subsection{ARM: \OptimizingKeilVI (\ThumbMode)}

\lstinputlisting{patterns/08_switch/1_few/few_ARM_thumb_O3.asm}

% FIXME а каким можно? к каким нельзя? \myref{} ->
\RU{Как я уже писал, в Thumb-режиме нет возможности добавлять условные предикаты к большинству инструкций,
так что Thumb-код вышел похожим на код x86 в стиле \ac{CISC}, вполне понятный.}
\EN{As I already mentioned, it is not possible to add conditional predicates to most instructions in Thumb
mode, so the Thumb-code here is somewhat similar to the easily understandable x86 \ac{CISC}-style code.}

\subsection{ARM64: \NonOptimizing GCC (Linaro) 4.9}

\lstinputlisting{patterns/08_switch/1_few/ARM64_GCC_O0.lst.\LANG}

\RU{Входное значение имеет тип \Tint, поэтому для него используется регистр \RegW{0},
а не целая часть регистра \RegX{0}.}
\EN{The type of the input value is \Tint, hence register \RegW{0} is used to hold it instead of the whole
\RegX{0} register.}
\RU{Указатели на строки передаются в \puts при помощи пары инструкций ADRP/ADD, 
как я и показывал в примере}%
\EN{The string pointers are passed to \puts using an ADRP/ADD instructions pair just like I demonstrated in the} 
\q{\HelloWorldSectionName}\EN{ example}:~\myref{pointers_ADRP_and_ADD}.

\subsection{ARM64: \Optimizing GCC (Linaro) 4.9}

\lstinputlisting{patterns/08_switch/1_few/ARM64_GCC_O3.lst.\LANG}

\RU{Фрагмент кода более оптимизированный}\EN{Better optimized piece of code}.
\RU{Инструкция }\TT{CBZ} (\IT{Compare and Branch on Zero}\RU{~--- сравнить и перейти если ноль}) 
\RU{совершает переход если}\EN{instruction does jump if} \RegW{0} \RU{ноль}\EN{is zero}.
\RU{Здесь также прямой переход на}\EN{There is also a direct jump to} \puts 
\RU{вместо вызова, как я уже описывал:~}%
\EN{instead of calling it, like I explained before:~}%
\myref{JMP_instead_of_RET}.
