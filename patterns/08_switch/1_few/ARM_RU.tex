\subsection{ARM: \OptimizingKeilVI (\ARMMode)}
\myindex{\CLanguageElements!switch}

\lstinputlisting{patterns/08_switch/1_few/few_ARM_ARM_O3.asm}

Мы снова не сможем сказать, глядя на этот код, был ли в оригинальном исходном коде switch() 
либо же несколько операторов if().

\myindex{ARM!\Instructions!ADRcc}
Так или иначе, мы снова видим здесь инструкции с предикатами, например, \ADREQ (\IT{(Equal)}), 
которая будет исполняться только
если $R0=0$, и тогда в \Reg{0} будет загружен адрес строки \IT{<<zero\textbackslash{}n>>}.

\myindex{ARM!\Instructions!BEQ}
Следующая инструкция \ac{BEQ} перенаправит исполнение на \TT{loc\_170}, если $R0=0$.

Кстати, наблюдательный читатель может спросить, сработает ли \ac{BEQ} нормально,
ведь \ADREQ перед ним уже заполнила регистр \Reg{0} чем-то другим?

Сработает, потому что \ac{BEQ} проверяет флаги, установленные инструкцией \CMP, 
а \ADREQ флаги никак не модифицирует.

Далее всё просто и знакомо. 
Вызов \printf один, и в самом конце, мы уже рассматривали подобный трюк~(\myref{ARM_B_to_printf}).
К вызову функции \printf{} в конце ведут три пути.

\myindex{ARM!\Instructions!ADRcc}
\myindex{ARM!\Instructions!CMP}
Последняя инструкция \TT{CMP R0, \#2} здесь нужна, чтобы узнать $a=2$ или нет.

Если это не так, то при помощи \ADRNE (\IT{Not Equal}) в \Reg{0} будет загружен указатель на 
строку \IT{<<something unknown \textbackslash{}n>>}, ведь $a$ уже было проверено на 0 и 1 до этого, 
и здесь $a$ точно не попадает под эти константы.

Ну а если $R0=2$, в \Reg{0} будет загружен указатель на строку \IT{<<two\textbackslash{}n>>} при помощи инструкции \ADREQ.

\subsection{ARM: \OptimizingKeilVI (\ThumbMode)}

\lstinputlisting{patterns/08_switch/1_few/few_ARM_thumb_O3.asm}

% FIXME а каким можно? к каким нельзя? \myref{} ->
Как уже было отмечено, в Thumb-режиме нет возможности добавлять условные предикаты к большинству инструкций,
так что Thumb-код вышел похожим на код x86 в стиле \ac{CISC}, вполне понятный.

\subsection{ARM64: \NonOptimizing GCC (Linaro) 4.9}

\lstinputlisting{patterns/08_switch/1_few/ARM64_GCC_O0_RU.lst}

Входное значение имеет тип \Tint, поэтому для него используется регистр \RegW{0},
а не целая часть регистра \RegX{0}.

Указатели на строки передаются в \puts при помощи пары инструкций ADRP/ADD, как было показано в примере
\q{\HelloWorldSectionName}:~\myref{pointers_ADRP_and_ADD}.

\subsection{ARM64: \Optimizing GCC (Linaro) 4.9}

\lstinputlisting{patterns/08_switch/1_few/ARM64_GCC_O3_RU.lst}

Фрагмент кода более оптимизированный.
Инструкция \TT{CBZ} (\IT{Compare and Branch on Zero}~--- сравнить и перейти если ноль) совершает переход если \RegW{0} ноль.
Здесь также прямой переход на \puts вместо вызова, как уже было описано:~\myref{JMP_instead_of_RET}.

