\subsection{MIPS}

\lstinputlisting[caption=\Optimizing GCC 4.4.5 (IDA)]{patterns/08_switch/1_few/MIPS_O3_IDA.lst.\LANG}

\index{MIPS!\Instructions!JR}

\RU{Функция всегда заканчивается вызовом \puts, так что здесь мы видим переход на \puts (JR: \q{Jump Register})
вместо перехода с сохранением \ac{RA} (\q{jump and link}).}
\EN{The function always ends with calling \puts,
so here we see a jump to \puts (JR: \q{Jump Register}) instead of \q{jump and link}.}
\RU{Мы говорили об этом ранее}\EN{We talked about this earlier}: \myref{JMP_instead_of_RET}.

\index{MIPS!Load delay slot}
\RU{Мы также часто видим NOP-инструкции после LW.}\EN{We also often see NOP instructions after LW ones.}
\RU{Это}\EN{This is} \q{load delay slot}: \RU{ещё один \IT{delay slot} в MIPS}\EN{another \IT{delay slot} in MIPS}.
\index{MIPS!\Instructions!LW}
\RU{Инструкция после LW может исполняться в тот момент, когда LW загружает значение из памяти.}
\EN{An instruction next to LW may execute at the moment while LW loads value from memory. }
\RU{Впрочем, следующая инструкция не должна использовать результат LW.}
\EN{However, the next instruction must not use the result of LW.}
\RU{Современные MIPS-процессоры ждут, если следующая инструкция использует результат LW, так что всё это уже
устарело, но GCC всё еще добавляет NOP-ы для более старых процессоров.}
\EN{Modern MIPS CPUs have a feature to wait if the next instruction uses result of LW, so this is somewhat outdated,
but GCC still adds NOPs for older MIPS CPUs.}
\RU{Вообще, это можно игнорировать}\EN{In general, it can be ignored}.
