\subsection{ARM: \OptimizingKeilVI (\ARMMode)}
\label{sec:SwitchARMLot}

\lstinputlisting[caption=\OptimizingKeilVI (\ARMMode)]{patterns/08_switch/2_lot/lot_ARM_ARM_O3.asm}

\RU{В этом коде используется та особенность режима ARM, 
что все инструкции в этом режиме имеют фиксированную длину 4 байта.}
\EN{This code makes use of the ARM mode feature in which all instructions have a fixed size of 4 bytes.}

\RU{Итак, не будем забывать, что максимальное значение для $a$ это $4$: всё что выше, должно вызвать
вывод строки \IT{<<something unknown\textbackslash{}n>>}.}
\EN{Let's keep in mind that the maximum value for $a$ is $4$ and any greater value will cause
\IT{<<something unknown\textbackslash{}n>>} string to be printed.}

\index{ARM!\Instructions!CMP}
\index{ARM!\Instructions!ADDCC}
\RU{Самая первая инструкция}\EN{The first} \TT{CMP R0, \#5} 
\RU{сравнивает входное значение в $a$ c $5$.}
\EN{instruction compares the input value of $a$ with $5$.}

\RU{Следующая инструкция}\EN{The next} \TT{ADDCC PC, PC, R0,LSL\#2}
\footnote{ADD\EMDASH\RU{складывание чисел}\EN{addition}}
\RU{сработает только в случае если}\EN{instruction is being executed only if} $R0 < 5$ (\IT{CC=Carry clear / Less than}). 
\RU{Следовательно, если}\EN{Consequently, if} \TT{ADDCC} \RU{не сработает}\EN{does not trigger} 
(\RU{это случай с}\EN{it is a} $R0 \geq 5$\EN{ case}), 
\RU{выполнится переход на метку}\EN{a jump to} 
\IT{default\_case}\EN{ label will occur}.

\RU{Но если}\EN{But if} $R0 < 5$ \AndENRU \TT{ADDCC} \RU{сработает, то произойдет следующее.}
\EN{triggers, the following is to be happen:}

\RU{Значение в \Reg{0} умножается на $4$}\EN{The value in \Reg{0} is multiplied by $4$}.
\RU{Фактически}\EN{In fact}, \TT{LSL\#2} \RU{в суффиксе инструкции означает \q{сдвиг влево на 2 бита}.}
\EN{at the instruction's suffix stands for \q{shift left by 2 bits}.}
\RU{Но как будет видно позже}\EN{But as we will see later}~(\myref{division_by_shifting}) \RU{в секции}\EN{in section} 
\q{\ShiftsSectionName}, 
\RU{сдвиг влево на 2 бита эквивалентeн его умножению на $4$.}
\EN{shift left by 2 bits is equivalent to multiplying by $4$.}

\RU{Затем полученное}\EN{Then we add} $R0*4$ \RU{прибавляется к текущему значению \ac{PC}}\EN{to
the current value in \ac{PC}}, 
\RU{совершая, таким образом, переход на одну из расположенных ниже инструкций \TT{B} (\IT{Branch}).}
\EN{thus jumping to one of the \TT{B} (\IT{Branch}) instructions located below.}

\RU{На момент исполнения}\EN{At the moment of the execution of} \TT{ADDCC},
\RU{содержимое \ac{PC} на 8 байт больше}\EN{the value in \ac{PC} is 8 bytes ahead} (\TT{0x180})%
\RU{, чем адрес по которому расположена сама инструкция} 
\EN{than the address at which the} \TT{ADDCC}\EN{ instruction is located} (\TT{0x178}), 
\RU{либо, говоря иным языком, на 2 инструкции больше.}
\EN{or, in other words, 2 instructions ahead.}

\index{ARM!\RU{Конвейер}\EN{Pipeline}}
\RU{Это связано с работой конвейера процессора ARM:
пока исполняется инструкция \TT{ADDCC}, процессор уже начинает обрабатывать инструкцию после следующей, 
поэтому \ac{PC} указывает туда. Этот факт нужно запомнить.}
\EN{This is how the pipeline in ARM processors works: when \TT{ADDCC} is executed,
the processor at the moment
is beginning to process the instruction after the next one,
so that is why \ac{PC} points there. This has to be memorized.}

\RU{Если $a=0$, тогда к \ac{PC} ничего не будет прибавлено и 
в \ac{PC} запишется актуальный на тот момент \ac{PC} (который больше на 8) 
и произойдет переход на метку \IT{loc\_180}. 
Это на 8 байт дальше места, где находится инструкция \TT{ADDCC}.}
\EN{If $a=0$, then is to be added to the value in \ac{PC},
and the actual value of the \ac{PC} will be written into \ac{PC} (which is 8 bytes ahead)
and a jump to the label \IT{loc\_180} will happen,
which is 8 bytes ahead of the point where the \TT{ADDCC} instruction is.}

\RU{Если}\EN{If} $a=1$, \RU{тогда в \ac{PC} запишется}\EN{then} 
$PC+8+a*4 = PC+8+1*4 = PC+12 = 0x184$\RU{. Это адрес метки \IT{loc\_184}}\EN{ will be written to \ac{PC},
which is the address of the \IT{loc\_184} label}.

\RU{При каждой добавленной к $a$ единице итоговый \ac{PC} увеличивается на $4$.}
\EN{With every $1$ added to $a$, the resulting \ac{PC} is increased by $4$.}
\RU{$4$ это длина инструкции в режиме ARM и одновременно с этим, 
длина каждой инструкции \TT{B}, их здесь следует 5 в ряд.}
\EN{$4$ is the instruction length in ARM mode and also, the length of each \TT{B} instruction,
of which there are 5 in row.}

\RU{Каждая из этих пяти инструкций \TT{B} передает управление дальше, где собственно и происходит то, 
что запрограммировано в операторе}
\EN{Each of these five \TT{B} instructions passes control further, to 
what was programmed in the}
\IT{switch()}.
\RU{Там происходит загрузка указателя на свою строку,}
\EN{Pointer loading of the corresponding string occurs there,}\etc{}.

\subsection{ARM: \OptimizingKeilVI (\ThumbMode)}

\lstinputlisting[caption=\OptimizingKeilVI (\ThumbMode)]{patterns/08_switch/2_lot/lot_ARM_thumb_O3.asm}

\index{ARM!\ThumbMode}
\index{ARM!\ThumbTwoMode}
\RU{В режимах Thumb и Thumb-2 уже нельзя надеяться на то, что все инструкции имеют одну длину.}
\EN{One cannot be sure that all instructions in Thumb and Thumb-2 modes has the same size.}
\RU{Можно даже сказать, что в этих режимах инструкции переменной длины, как в x86.}
\EN{It can even be said that in these modes the instructions have variable lengths, just like in x86.}

\index{jumptable}
\RU{Так что здесь добавляется специальная таблица, содержащая информацию о том, как много вариантов здесь,
не включая варианта по умолчанию, и смещения, для каждого варианта. Каждое смещение кодирует метку, куда нужно передать
управление в соответствующем случае.}
\EN{So there is a special table added that contains information about how much cases are there (not including 
default-case), and an offset for each with a label to which control must be passed in 
the corresponding case.}

\index{ARM!\RU{Переключение режимов}\EN{Mode switching}}
\index{ARM!\Instructions!BX}
\RU{Для того чтобы работать с таблицей и совершить переход, вызывается служебная функция}
\EN{A special function is present here in order to deal with the table and pass control, named}
\IT{\_\_ARM\_common\_switch8\_thumb}. 
\RU{Она начинается с инструкции}\EN{It starts with} \TT{BX PC}
\RU{, чья функция~--- переключить процессор в ARM-режим.}
\EN{, whose function is to switch the processor to ARM-mode.}
\RU{Далее функция, работающая с таблицей.}\EN{Then you see the function for table processing.} 
\RU{Она слишком сложная для рассмотрения в данном месте, так что я пропущу все объяснения.}
\EN{It is too complex to describe it here now, so I'm omitting it.}
% TODO explain it...

\index{ARM!\Registers!Link Register}
\RU{Но можно отметить, что эта функция использует регистр \ac{LR} как указатель на таблицу.}
\EN{It is interesting to note that the function uses the \ac{LR} register as a pointer to the table.}
\RU{Действительно, после вызова этой функции, 
в \ac{LR} был записан адрес после инструкции}
\EN{Indeed, after calling of this function, \ac{LR} contains the address after}
\TT{BL \_\_ARM\_common\_switch8\_thumb}
\RU{, а там как раз и начинается таблица.}
\EN{ instruction, where the table starts.}

\RU{Ещё можно отметить, что код для этого выделен в отдельную функцию для того, 
чтобы не нужно было каждый раз генерировать 
точно такой же фрагмент кода для каждого выражения switch().}
\EN{It is also worth noting that the code is generated as a separate function in order to reuse it, 
so the compiler not generates the same code for every switch() statement.}

\IDA 
\RU{распознала эту служебную функцию и таблицу автоматически дописала комментарии к меткам вроде}
\EN{successfully perceived it as a service function and a table, and added comments to the labels
like}\\
\TT{jumptable 000000FA case 0}.

