\section{\RU{Когда много \IT{case} в одном блоке}
\EN{When there are several \IT{case} statements in one block}}

\RU{Вот очень часто используемая конструкция: несколько \IT{case} может быть использовано в одном блоке:}
\EN{Here is a very widespread construction: several \IT{case} statements for a single block:}

\lstinputlisting{patterns/08_switch/3_several_cases/several_cases.c}

\RU{Слишком расточительно генерировать каждый блок для каждого случая, поэтому обычно
генерируется каждый блок плюс некий диспетчер.}
\EN{It's too wasteful to generate a block for each possible case,
so what is usually done is to generate each block plus some kind of dispatcher.}

\subsection{MSVC}

\lstinputlisting[caption=\Optimizing MSVC 2010,numbers=left]{patterns/08_switch/3_several_cases/several_cases_MSVC_2010_Ox.asm.\LANG}

\RU{Здесь видим две таблицы}\EN{We see two tables here}: 
\RU{первая таблица}\EN{the first table} (\TT{\$LN10@f}) \RU{это таблица индексов}\EN{is an index table},
\RU{а вторая таблица}\EN{and the second one} (\TT{\$LN11@f}) \RU{это массив указателей на блоки}\EN{is 
an array of pointers to blocks}.

\RU{В начале, входное значение используется как индекс в таблице индексов}\EN{First, the input value 
is used as an index in the index table} (\LineENRU 13). 

\RU{Вот краткое описание значений в таблице}\EN{Here is a short legend for the values in the table}: 
0 \RU{это первый блок \IT{case}}\EN{is the first \IT{case} block} (\RU{для значений}\EN{for values} 1, 2, 7, 10),
1 \RU{это второй}\EN{is the second one} (\RU{для значений}\EN{for values} 3, 4, 5),
2 \RU{это третий}\EN{is the third one} (\RU{для значений}\EN{for values} 8, 9, 21),
3 \RU{это четвертый}\EN{is the fourth one} (\RU{для значений}\EN{for value} 22),
4 \RU{это для блока по умолчанию}\EN{is for the default block}.

\RU{Мы получаем индекс для второй таблицы указателей на блоки и переходим туда}\EN{There we get an index for 
the second table of code pointers and we jump to it} (\LineENRU 14).

\EN{What is also worth noting is that there is no case for input value 0.}
\RU{Ещё нужно отметить то, что здесь нет случая для нулевого входного значения.}
\EN{That's why we see the \DEC instruction at line 10, and the table starts at $a=1$, 
because there is no need to allocate a table element for $a=0$.}
\RU{Поэтому мы видим инструкцию \DEC на строке 10 и таблица начинается с $a=1$.
Потому что незачем выделять в таблице элемент для $a=0$.}

\RU{Это очень часто используемый шаблон}\EN{This is a very widespread pattern}.

\RU{В чем же экономия}\EN{So why is this economical}?
\RU{Почему нельзя сделать так, как уже обсуждалось}\EN{Why isn't it possible to make it as before}
(\myref{switch_lot_GCC}), \RU{используя только одну таблицу, содержащую указатели на 
блоки}\EN{just with one table consisting of block pointers}?
\RU{Причина в том, что элементы в таблице индексов занимают только по 8-битному байту, поэтому всё это более 
компактно}\EN{The reason is that the elements in index table are 8-bit, hence it's all more compact}.

\ifdefined\IncludeGCC
\subsection{GCC}

GCC \RU{делает так, как уже обсуждалось}\EN{does the job in the way we already discussed} 
(\myref{switch_lot_GCC}), \RU{используя просто таблицу указателей}\EN{using just one table of pointers}.

\subsection{ARM64: \Optimizing GCC 4.9.1}

\RU{Во-первых, здесь нет кода, срабатывающего в случае если входное значение~--- 0, так что GCC пытается
сделать таблицу переходов более компактной и начинает с случая, когда входное значение~--- 1.}
\EN{There is no code to be triggered if the input value is 0, so GCC tries to make the jump table more compact
and so it starts at 1 as an input value.}

GCC 4.9.1 \ForENRU ARM64 \RU{использует даже более интересный трюк}\EN{uses an even cleverer trick}.
\RU{Он может закодировать все смещения как 8-битные байты}\EN{It's able to encode all offsets as 8-bit bytes}.
\RU{Вспомним, что все инструкции в ARM64 имеют размер в 4 байта.}
\EN{Let's recall that all ARM64 instructions have a size of 4 bytes.}
\RU{GCC также использует тот факт, что все смещения в моем крохотном примере находятся достаточно близко 
друг от друга.}
\EN{GCC is uses the fact that all offsets in my tiny example are in close proximity to each other.}
\RU{Так что таблица переходов состоит из байт.}\EN{So the jump table consisting of single bytes.}

\lstinputlisting[caption=\Optimizing GCC 4.9.1 ARM64]{patterns/08_switch/3_several_cases/ARM64_GCC491_O3.s.\LANG}

\RU{Я скомпилировал мой пример как объектный файл и открыл его в IDA. Вот таблица переходов:}
\EN{I just compiled my example to object file and opened it in IDA. Here is the jump table:}

\lstinputlisting[caption=jumptable in IDA]{patterns/08_switch/3_several_cases/ARM64_GCC491_O3_IDA.lst}

\RU{В случае 1, 9 будет умножено на 9 и прибавлено к адресу метки Lrtx4.}
\EN{So in case of 1, 9 is to be multiplied by 4 and added to the address of Lrtx4 label.}
\RU{В случае 22, 0 будет умножено на 4, в результате это 0.}
\EN{In case of 22, 0 is to be multiplied by 4, resulting in 0. }
\RU{Место сразу за меткой Lrtx4 это метка L7, где находится код, выводящий \q{22}.}
\EN{Right after the Lrtx4 label is the L7 label, where you can find the code that prints \q{22}.}
\RU{В сегменте кода нет таблицы переходов, место для нее выделено в отдельной секции .rodata
(нет особой нужды располагать её в сегменте кода).}
\EN{There is no jump table in the code segment, it's allocated in a separate .rodata section 
(there is no special need to place it in the code section).}

\RU{Там есть также отрицательные байты (0xF7). Они используются для перехода назад, на код, выводящий
строку \q{default} (на .L2).}
\EN{There are also negative bytes (0xF7), they are used for jumping back to the code that prints the \q{default} string 
(at .L2).}
\fi
