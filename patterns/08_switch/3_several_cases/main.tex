\section{\RU{Когда много \IT{case} в одном блоке}
\EN{When there are several \IT{case} in one block}}

\RU{Вот очень часто используемая конструкция: несколько \IT{case} может быть использовано в одном блоке:}
\EN{Here is also a very often used construction: several \IT{case} statements may be used in single block:}

\lstinputlisting{patterns/08_switch/3_several_cases/several_cases.c}

\RU{Слишком расточительно генерировать каждый блок для каждого случая, поэтому обычно
каждый блок генерируется плюс диспетчер.}
\EN{It's too wasteful to generate each block for each possible case,
so what is usually done, is each block generated plus some kind of dispatcher.}

\subsection{MSVC}

\lstinputlisting[caption=\Optimizing MSVC 2010,numbers=left]{patterns/08_switch/3_several_cases/several_cases_MSVC_2010_Ox.asm.\LANG}

\RU{Здесь видим две таблицы}\EN{We see two tables here}: 
\RU{первая таблица}\EN{the first table} (\TT{\$LN10@f}) \RU{это таблица индексов}\EN{is index table},
\RU{и вторая таблица}\EN{and the second table} (\TT{\$LN11@f}) \RU{это массив указателей на блоки}\EN{is 
an array of pointers to blocks}.

\RU{В начале, входное значение используется как индекс в таблице индексов}\EN{First, input value 
is used as index in index table} (\LineENRU 13). 

\RU{Вот краткое описание значений в таблице}\EN{Here is short legend for values in the table}: 
0 \RU{это первый блок \IT{case}}\EN{is first \IT{case} block} (\RU{для значений}\EN{for values} 1, 2, 7, 10),
1 \RU{это второй}\EN{is second} (\RU{для значений}\EN{for values} 3, 4, 5),
2 \RU{это третий}\EN{is third} (\RU{для значений}\EN{for values} 8, 9, 21),
3 \RU{это четвертый}\EN{is fourth} (\RU{для значений}\EN{for value} 22),
4 \RU{это для default-блока}\EN{is for default block}.

\RU{Мы получаем индекс для второй таблицы указателей на блоки и переходим туда}\EN{We get there index for 
the second table of block pointers and we we jump there} (\LineENRU 14).

\EN{What is also worth to note that there are no case for input value $0$.}
\RU{Что еще нужно отметить, так это то что здесь нет случая для нулевого входного значения.}
\EN{Hence, we see \DEC instruction at line 10, and the table is beginning at $a=1$. 
Because there are no need to allocate table element for $a=0$.}
\RU{Поэтому мы видим инструкцию \DEC на строке 10 и таблица начинается с $a=1$.
Потому что незачем выделять в таблице элемент для $a=0$.}

\RU{Это очень часто используемый шаблон}\EN{This is very often used pattern}.

\RU{В чем же экономия}\EN{So where economy is}?
\RU{Почему нельзя сделать так, как уже обсуждалось}\EN{Why it's not possible to make it as it was 
already discussed} (\ref{switch_lot_GCC}), \RU{используя только одну таблицу, содержащую указатели на 
блоки}\EN{just with one table, consisting of block pointers}?
\RU{Причина в том, что элементы в таблице индексов занимают только по 8-битному байту, поэтому всё это более 
компактно}\EN{The reason is because elements in index table has 8-bit byte type, hence it's all more compact}.

\subsection{GCC}

GCC \RU{делает так, как уже обсуждалось}\EN{do the job like it was already discussed} 
(\ref{switch_lot_GCC}), \RU{используя просто таблицу указателей}\EN{using just one table of pointers}.

\subsection{ARM64: \Optimizing GCC 4.9.1}

\RU{Во-первых, здесь нет кода, срабатывающего в случае если входное значение --- 0, так что GCC пытается
сделать таблицу переходов более компактной и начинает с случая, когда входное значение --- 1.}
\EN{First of all, there are no code triggering if input value is 0, so GCC tries to make jumptable more compact
and so it's started at the case of 1 as input value.}

GCC 4.9.1 \ForENRU ARM64 \RU{использует даже более интересный трюк}\EN{uses even more clever trick}.
\RU{Он может закодировать все смещения как 8-битные байты}\EN{It's able to encode all offsets as 8-bit bytes}.
\RU{Вспомним что все инструкции в ARM64 имеют размер в 4 байта.}
\EN{Let's remind ourselves, all ARM64 instructions has size of 4 bytes.}
\RU{GCC также использует тот факт, что все смещения в моем крохотном примере находятся достаточно близко 
друг от друга.}
\EN{GCC is also use the fact that all offsets in my tiny example are in close proximity to each other.}
\RU{Так что таблица переходов состоит из байт.}\EN{So the jumptable consisting of bytes.}

\lstinputlisting[caption=\Optimizing GCC 4.9.1 ARM64]{patterns/08_switch/3_several_cases/ARM64_GCC491_O3.s.\LANG}

\RU{Я скомпилировал мой пример как объектный файл и открыл его в IDA. Вот таблица переходов:}
\EN{I just compiled my example to object file and opened it in IDA. Here is a jump table:}

\lstinputlisting[caption=jumptable in IDA]{patterns/08_switch/3_several_cases/ARM64_GCC491_O3_IDA.lst}

\RU{В случае 1, 9 будет умножено на 9 и прибавлено к адресу метки Lrtx4.}
\EN{So in case of 1, 9 will be multiplied by 4 and added to the address of Lrtx4 label.}
\RU{В случае 22, 0 будет умножено на 4, в результате это 0.}
\EN{In case of 22, 0 is to be multiplied by 4 resulting 0. }
\RU{Место сразу за меткой Lrtx4, это метка L7, где находится код выводящий ``22''.}
\EN{The place right after Lrtx4 label is L7 label, where the code printing ``22'' is.}
\RU{В сегменте кода нет таблицы переходов, место для нее выделено в отдельной секции .rodata
(нет особой нужды располагать её в сегменте кода).}
\EN{There are no jumptable in the code segment, a place for it is allocated in separate .rodata section 
(there are no special need to place it in code section).}

\RU{Там есть также отрицательные байты (0xF7), они используются для перехода назад, на код выводящий
строку ``default'' (на .L2).}
\EN{There are also negative bytes (0xF7), they are used for jumping back to the code printing ``default'' string 
(at .L2).}
