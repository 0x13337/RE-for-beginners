\sectionold{When there are several \IT{case} statements in one block}

Here is a very widespread construction: several \IT{case} statements for a single block:

\lstinputlisting{patterns/08_switch/3_several_cases/several_cases.c}

It's too wasteful to generate a block for each possible case,
so what is usually done is to generate each block plus some kind of dispatcher.

\subsectionold{MSVC}

\lstinputlisting[caption=\Optimizing MSVC 2010,numbers=left]{patterns/08_switch/3_several_cases/several_cases_MSVC_2010_Ox_EN.asm}

We see two tables here: the first table (\TT{\$LN10@f}) is an index table, and the second one (\TT{\$LN11@f}) is an array of pointers to blocks.

First, the input value is used as an index in the index table (line 13). 

Here is a short legend for the values in the table: 
0 is the first \IT{case} block (for values 1, 2, 7, 10),
1 is the second one (for values 3, 4, 5),
2 is the third one (for values 8, 9, 21),
3 is the fourth one (for value 22),
4 is for the default block.

There we get an index for the second table of code pointers and we jump to it (line 14).

What is also worth noting is that there is no case for input value 0.

That's why we see the \DEC instruction at line 10, and the table starts at $a=1$, 
because there is no need to allocate a table element for $a=0$.

This is a very widespread pattern.

So why is this economical?
Why isn't it possible to make it as before
(\myref{switch_lot_GCC}), just with one table consisting of block pointers?
The reason is that the elements in index table are 8-bit, hence it's all more compact.

\subsectionold{GCC}

GCC does the job in the way we already discussed (\myref{switch_lot_GCC}), using just one table of pointers.

\subsectionold{ARM64: \Optimizing GCC 4.9.1}

There is no code to be triggered if the input value is 0, so GCC tries to make the jump table more compact
and so it starts at 1 as an input value.

GCC 4.9.1 for ARM64 uses an even cleverer trick.
It's able to encode all offsets as 8-bit bytes.

Let's recall that all ARM64 instructions have a size of 4 bytes.

GCC is uses the fact that all offsets in my tiny example are in close proximity to each other.
So the jump table consisting of single bytes.

\lstinputlisting[caption=\Optimizing GCC 4.9.1 ARM64]{patterns/08_switch/3_several_cases/ARM64_GCC491_O3_EN.s}

Let's compile this example to object file and open it in \IDA. Here is the jump table:

\lstinputlisting[caption=jumptable in IDA]{patterns/08_switch/3_several_cases/ARM64_GCC491_O3_IDA.lst}

So in case of 1, 9 is to be multiplied by 4 and added to the address of \TT{Lrtx4} label.

In case of 22, 0 is to be multiplied by 4, resulting in 0. 

Right after the \TT{Lrtx4} label is the \TT{L7} label, where you can find the code that prints \q{22}.

There is no jump table in the code segment, it's allocated in a separate .rodata section 
(there is no special need to place it in the code section).

There are also negative bytes (0xF7), they are used for jumping back to the code that prints the \q{default} string (at \TT{.L2}).

