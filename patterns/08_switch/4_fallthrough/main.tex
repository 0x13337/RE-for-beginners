\section{Fall-through}

\RU{Еще одно популярное использование}\EN{Another very popular usage of} \TT{switch()} 
\EN{is the fall-through}\RU{это т.н. ``fallthrough'' (``проваливаясь'')}.
\RU{Вот простой пример}\EN{Here is a small example}:

\lstinputlisting[numbers=left]{patterns/08_switch/4_fallthrough/fallthrough.c}

\RU{Если}\EN{If} $type=1$ (R), $read$ \RU{будет выставлен в}\EN{is to be set to} $1$, \RU{если}\EN{if} 
$type=2$ (W), $write$ \RU{будет выставлен в}\EN{is to be set to} $2$.
\RU{В случае}\EN{In case of} $type=3$ (RW), \RU{обе}\EN{both} $read$ \AndENRU $write$ \RU{будут 
выставлены в}\EN{is to be set to} $1$.

\RU{Фрагмент кода на строке 14 будет исполнен в двух случаях: если}\EN{The code at 
line 14 is executed in two cases: if} $type=RW$ \RU{или если}\EN{or if} $type=W$.
\RU{Там нет ``break'' для ``case RW'', и это нормально}\EN{There is no ``break'' 
for ``case RW'' and that's OK}.

\subsection{MSVC x86}

\lstinputlisting[caption=MSVC 2012]{patterns/08_switch/4_fallthrough/fallthrough_MSVC.asm}

\RU{Код почти полностью повторяет то что в исходнике.}
\EN{The code mostly resembles what is in the source.}
\RU{Там нет переходов между метками}\EN{There are no jumps between labels} \TT{\$LN4@f} \AndENRU 
\TT{\$LN3@f}: \RU{так что когда управление (code flow) находится на}\EN{so when code flow is at} 
\TT{\$LN4@f}, $read$ \RU{в начале выставляется в $1$, затем}\EN{is first set to $1$, then} $write$.
\EN{This is why it's called fall-through: code flow falls through one piece of code
(setting $read$) to another (setting $write$).}
\RU{Наверное, поэтому всё это и называется ``проваливаться'': управление проваливается через
один фрагмент кода (выставляющий $read$) в другой (выставляющий $write$).}
\RU{Если}\EN{If} $type=W$, \RU{мы оказываемся на}\EN{we land at} \TT{\$LN3@f}, 
\RU{так что код выставляющий $read$ в $1$ не исполнится}\EN{so no code setting $read$ to $1$ 
is executed}.

\ifdefined\IncludeARM
\subsection{ARM64}

\lstinputlisting[caption=GCC (Linaro) 4.9]{patterns/08_switch/4_fallthrough/fallthrough_ARM64.s.\LANG}

\RU{Почти то же самое}\EN{Merely the same thing}.
\RU{Здесь нет переходов между метками}\EN{There are no jumps between labels} \TT{.L4} 
\AndENRU \TT{.L3}.
\fi
