\subsection{Fall-through}

Ещё одно популярное использование оператора \TT{switch()} это т.н. \q{fallthrough} (\q{провал}).
Вот простой пример:

\lstinputlisting[numbers=left]{patterns/08_switch/4_fallthrough/fallthrough.c}

Если $type=1$ (R), $read$ будет выставлен в 1, если $type=2$ (W), $write$ будет выставлен в 1.
В случае $type=3$ (RW), обе $read$ и $write$ будут выставлены в 1.

Фрагмент кода на строке 14 будет исполнен в двух случаях: если $type=RW$ или если $type=W$.
Там нет \q{break} для \q{case RW}, и это нормально.

\subsubsection{MSVC x86}

\lstinputlisting[caption=MSVC 2012]{patterns/08_switch/4_fallthrough/fallthrough_MSVC.asm}

Код почти полностью повторяет то, что в исходнике.

Там нет переходов между метками \TT{\$LN4@f} и 
\TT{\$LN3@f}: так что когда управление (code flow) находится на 
\TT{\$LN4@f}, $read$ в начале выставляется в 1, затем $write$.

Наверное, поэтому всё это и называется \q{проваливаться}: управление проваливается через
один фрагмент кода (выставляющий $read$) в другой (выставляющий $write$).
Если $type=W$, мы оказываемся на \TT{\$LN3@f}, 
так что код выставляющий $read$ в 1 не исполнится.

\subsubsection{ARM64}

\lstinputlisting[caption=GCC (Linaro) 4.9]{patterns/08_switch/4_fallthrough/fallthrough_ARM64_RU.s}

Почти то же самое.
Здесь нет переходов между метками \TT{.L4} и \TT{.L3}.

