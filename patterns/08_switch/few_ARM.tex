\subsubsection{ARM: \OptimizingKeil + \ARMMode}
\index{\CLanguageElements!switch}

\lstinputlisting{patterns/08_switch/few_ARM_ARM_O3.asm}

\IFRU{Мы снова не сможем сказать, глядя на этот код, был ли в оригинальном исходном коде switch() 
либо же несколько if()-в.}
{Again, by investigating this code, we cannot say, was it switch() in the original source code, 
or pack of if() statements.}

\index{ARM!\Instructions!ADREQ}
\IFRU{Так или иначе, мы снова видим здесь инструкции с предикатами, например \ADREQ (\IT{(Equal)}), 
которая будет исполняться только
если $R0=0$, и тогда, в \Rzero будет загружен адрес строки \IT{<<zero\textbackslash{}n>>}.}
{Anyway, we see here predicated instructions again (like \ADREQ (\IT{Equal}))
which will be triggered only in $R0=0$ case, and the, address of the \IT{<<zero\textbackslash{}n>>}
string will be loaded into the \Rzero.}
\index{ARM!\Instructions!BEQ}
\IFRU{Следующая инструкция}{The next instruction} \ac{BEQ}
\IFRU{перенаправит исполнение на}{will redirect control flow to} \TT{loc\_170}, \IFRU{если}{if} $R0=0$.
\IFRU{Кстати, наблюдательный читатель может спросить, сработает ли \ac{BEQ} нормально,
ведь \ADREQ перед ним уже заполнила регистр \Rzero чем-то другим.}
{By the way, astute reader may ask, will \ac{BEQ} triggered right since \ADREQ before it
is already filled the \Rzero register with another value.}
\IFRU{Сработает, потому что \ac{BEQ} проверяет флаги установленные инструкцией \CMP, 
а \ADREQ флаги никак не модифицирует.}
{Yes, it will since \ac{BEQ} checking flags set by \CMP instruction, and \ADREQ not modifying flags
at all.}

\IFRU{Кстати, в ARM имеется также для некоторых инструкций суффикс \IT{-S}, указывающий, 
что эта инструкция будет модифицировать флаги, а при отсутствии суффикса ~--- не будет.}
{By the way, there is \IT{-S} suffix for some instructions in ARM,
indicating the instruction will set the flags according to the result, and without 
it~---the flags will not be touched.}
\index{ARM!\Instructions!ADD}
\index{ARM!\Instructions!ADDS}
\index{ARM!\Instructions!CMP}
\IFRU{Например, инструкция}{For example} \TT{ADD} \IFRU{в отличие от}{unlike} \TT{ADDS}
\IFRU{сложит два числа, но флаги не изменит}
{will add two numbers, but flags will not be touched}.
\IFRU{Такие инструкции удобно использовать
между \CMP где выставляются флаги и, например, инструкциями перехода, где флаги используются.}
{Such instructions are convenient to use between \CMP where flags are set and, 
e.g. conditional jumps, where flags are used.}

\IFRU{Далее всё просто и знакомо.}{Other instructions are already familiar to us.} 
\IFRU{Вызов}{There is only one call to} \printf \IFRU{один, и в самом конце, 
мы уже рассматривали подобный трюк здесь}{, at the end, and we already examined this trick here}
~(\ref{ARM_B_to_printf}).
\IFRU{К}{There are three paths to} \printf{}\IFRU{-у в конце ведут три пути}{at the end}.

\IFRU{Обратите внимание на то что происходит если $a=2$ и если $a$ не попадает под сравниваемые константы.}
{Also pay attention to what is going on if $a=2$ and if $a$ is not in range of constants it is comparing against.}
\index{ARM!\Instructions!ADRNE}
\index{ARM!\Instructions!CMP}
\IFRU{Инструкция }{}\TT{``CMP R0, \#2''} \IFRU{нужна чтобы узнать $a=2$ или нет}{instruction is needed here
to know, if $a=2$ or not}.
\IFRU{Если это не так, то при помощи \ADRNE (\IT{Not Equal}) в \Rzero будет загружен указатель на 
строку \IT{<<something unknown \textbackslash{}n>>}, ведь $a$ уже было проверено на $0$ и $1$ до этого, 
и здесь $a$ точно не попадает под эти константы.}
{If it is not true, then \ADRNE will load pointer to the string \IT{<<something unknown \textbackslash{}n>>} 
into \Rzero since $a$ was already
checked before to be equal to $0$ or $1$,
so we can be assured the $a$ variable is not equal to these numbers
at this point.}
\IFRU{Ну а если}{And if} $R0=2$, \IFRU{в \Rzero будет загружен указатель на строку}{a pointer to string} 
\IT{<<two\textbackslash{}n>>} 
\IFRU{при помощи инструкции \ADREQ}{will be loaded by \ADREQ into \Rzero}.

\subsubsection{ARM: \OptimizingKeil + \ThumbMode}

\lstinputlisting{patterns/08_switch/few_ARM_thumb_O3.asm}

\IFRU{Как я уже писал, в thumb-режиме нет возможности \IT{присоединять} предикаты к большинству инструкций,
так что thumb-код вышел похожим на код x86, вполне понятный.}
{As I already mentioned, there is no feature of \IT{connecting} predicates to majority of instructions in thumb
mode, so the thumb-code here is somewhat like the easily understandable x86 \ac{CISC}-code}

