\subsection{ARM: \OptimizingKeil + \ARMMode}
\label{sec:SwitchARMLot}

\lstinputlisting{patterns/08_switch/lot_ARM_ARM_O3.asm}

\RU{В этом коде используется та особенность режима ARM, что все инструкции в этом режиме имеют длину 4 байта.}
\EN{This code makes use of the ARM feature in which all instructions in the ARM mode has size of 4 bytes.}

\RU{Итак, не будем забывать, что максимальное значение для $a$ это $4$, всё что выше, должно вызвать
вывод строки}
\EN{Let's keep in mind the maximum value for $a$ is $4$ and any greater value must cause}
\IT{<<something unknown\textbackslash{}n>>}
\RU{.}\EN{string printing.}

\index{ARM!\Instructions!CMP}
\index{ARM!\Instructions!ADDCC}
\RU{Самая первая инструкция}\EN{The very first} \TT{``CMP R0, \#5''} 
\RU{сравнивает входное значение в $a$ c $5$.}
\EN{instruction compares $a$ input value with $5$.}

\RU{Следующая инструкция}\EN{The next} \TT{``ADDCC PC, PC, R0,LSL\#2''}
\footnote{ADD\EMDASH\RU{складывание чисел}\EN{addition}}
\RU{сработает только в случае если}\EN{instruction will execute only if} $R0 < 5$ (\IT{CC=Carry clear / Less than}). 
\RU{Следовательно, если}\EN{Consequently, if} \TT{ADDCC} \RU{не сработает}\EN{will not trigger} 
(\RU{это случай с}\EN{it is a} $R0 \geq 5$\EN{ case}), 
\RU{выполнится переход на метку}\EN{a jump to} 
\IT{default\_case}\RU{.}\EN{label will be occurred.}

\RU{Но если}\EN{But if} $R0 < 5$ \AndENRU \TT{ADDCC} \RU{сработает, то произойдет следующее:}
\EN{will trigger, following events will happen:}

\RU{Значение в \Reg{0} умножается на $4$}\EN{Value in the \Reg{0} is multiplied by $4$}.
\RU{Фактически}\EN{In fact}, \TT{LSL\#2} \RU{в конце инструкции означает ``сдвиг влево на 2 бита''.}
\EN{at the instruction's ending means ``shift left by 2 bits''.}
\RU{Но как будет видно позже}\EN{But as we will see later}~(\ref{division_by_shifting}) \RU{в секции}\EN{in} 
``\ShiftsSectionName'' \EN{section}, 
\RU{сдвиг влево на 2 бита это как раз эквивалентно его умножению на $4$.}
\EN{shift left by 2 bits is just equivalently to multiplying by $4$.}

\RU{Затем полученное}\EN{Then,} $R0*4$ \RU{прибавляется к текущему значению \ac{PC}}\EN{value we got, is added to
current value in the \ac{PC}}, 
\RU{совершая, таким образом, переход на одну из расположенных ниже инструкций \TT{B} (\IT{Branch}).}
\EN{thus jumping to one of \TT{B} (\IT{Branch}) instructions located below.}

\RU{На момент исполнения}\EN{At the moment of} \TT{ADDCC}\RU{,}\EN{ execution,}
\RU{содержимое \ac{PC} на 8 байт больше}\EN{value in the \ac{PC} is 8 bytes ahead} (\TT{0x180}) 
\RU{чем адрес по которому расположена сама инструкция} 
\EN{than address at which} \TT{ADDCC}\EN{ instruction is located} (\TT{0x178}), 
\RU{либо, говоря иным языком, на 2 инструкции больше.}
\EN{or, in other words, 2 instructions ahead.}

\index{ARM!\RU{Конвейер}\EN{Pipeline}}
\RU{Это связано с работой конвейера процессора ARM:
пока исполняется инструкция \TT{ADDCC}, процессор уже начинает обрабатывать инструкцию после следующей, 
поэтому \ac{PC} указывает туда.}
\EN{This is how ARM processor pipeline works: when \TT{ADDCC} instruction is executed,
the processor at the moment
is beginning to process instruction after the next one,
so that is why \ac{PC} pointing there.}

\RU{В случае, если $a=0$, тогда к \ac{PC} ничего не будет прибавлено, 
в \ac{PC} запишется актуальный на тот момент \ac{PC} (который больше на 8) 
и произойдет переход на метку \IT{loc\_180}, 
это на 8 байт дальше от места где находится инструкция \TT{ADDCC}.}
\EN{If $a=0$, then nothing will be added to the value in the \ac{PC},
and actual value in the \ac{PC} is to be written into the \ac{PC} (which is 8 bytes ahead)
and jump to the label \IT{loc\_180} will happen,
this is 8 bytes ahead of the point where \TT{ADDCC} instruction is.}

\RU{В случае, если}\EN{In case of} $a=1$, \RU{тогда в \ac{PC} запишется}\EN{then} 
$PC+8+a*4 = PC+8+1*4 = PC+16 = 0x184$\RU{, это адрес метки \IT{loc\_184}}\EN{will be written to the \ac{PC},
this is the address of the \IT{loc\_184} label}.

\RU{При каждой добавленной к $a$ единице, итоговый \ac{PC} увеличивается на $4$.}
\EN{With every $1$ added to $a$, resulting \ac{PC} increasing by $4$.}
\RU{$4$ это как раз длина инструкции  в режиме ARM и одновременно с этим, 
длина каждой инструкции \TT{B}, их здесь следует 5 в ряд.}
\EN{$4$ is also instruction length in ARM mode and also, length of each \TT{B} instruction length,
there are 5 of them in row.}

\RU{Каждая из этих пяти инструкций \TT{B}, передает управление дальше, где собственно и происходит то, 
что запрограммировано в}
\EN{Each of these five \TT{B} instructions passing control further, where something is going on, 
what was programmed in}
\IT{switch()}.
\RU{Там происходит загрузка указателя на свою строку, и т.д.}
\EN{Pointer loading to corresponding string occurring there, etc.}

\subsection{ARM: \OptimizingKeil + \ThumbMode}

\lstinputlisting{patterns/08_switch/lot_ARM_thumb_O3.asm}

\index{ARM!\ThumbMode}
\index{ARM!\ThumbTwoMode}
\RU{В режимах thumb и thumb-2, уже нельзя надеяться на то что все инструкции будут иметь одну длину.}
\EN{One cannot be sure all instructions in thumb and thumb-2 modes will have same size.}
\RU{Можно даже сказать, что в этих режимах инструкции переменной длины, как в x86.}
\EN{It is even can be said that in these modes instructions has variable length, just like in x86.}

\index{jumptable}
\RU{Так что здесь добавляется специальная таблица, содержащая информацию о том, как много вариантов здесь,
не включая default-варианта, и смещения, для каждого варианта, каждое кодирует метку, куда нужно передать
управление в соответствующем случае.}
\EN{So there is a special table added, containing information about how much cases are there, not including 
default-case, and offset, for each, each encoding a label, to which control must be passed in 
corresponding case.}

\index{ARM!\RU{Переключение режимов}\EN{Mode switching}}
\index{ARM!\Instructions!BX}
\RU{Для того чтобы работать с таблицей и совершить переход, вызывается служебная функция}
\EN{A special function here present in order to deal with the table and pass control, named} \\
\IT{\_\_ARM\_common\_switch8\_thumb}. 
\RU{Она начинается с инструкции}\EN{It is beginning with} \TT{``BX PC''}
\RU{, чья функция ~--- переключить процессор в ARM-режим.}
\EN{instruction, which function is to switch processor to ARM-mode.}
\RU{Далее функция, работающая с таблицей.}\EN{Then you may see the function for table processing.} 
\RU{Она слишком сложная для 
рассмотрения в данном месте, так что я пропущу объяснения.}
\EN{It is too complex for describing it here now, so I will omit elaborations.}
%TODO дописать когда-то?

\index{ARM!\Registers!Link Register}
\RU{Но можно отметить, что эта функция использует регистр \ac{LR} как указатель на таблицу.}
\EN{But it is interesting to note the function uses \ac{LR} register as a pointer to the table.}
\RU{Действительно, после вызова этой функции, в \ac{LR} был записан
адрес после инструкции}\EN{Indeed, after this function calling, \ac{LR} will contain address after} \\ 
\TT{``BL \_\_ARM\_common\_switch8\_thumb''}\RU{, а там как раз и начинается таблица.}
\EN{ instruction, and the table is beginning right there.}

\RU{Еще можно отметить что код для этого выделен в отдельную функцию для того, чтобы и в других местах,
в похожих случаях, обрабатывались \IT{switch()}-и, и не нужно было каждый раз генерировать во всех этих
местах такой фрагмент кода.}
\EN{It is also worth noting the code is generated as a separate function in order to reuse it, in similar places,
in similar cases, for \IT{switch()} processing, so compiler will not generate same code at each point.}
% код выделен -> en?

\IDA 
\RU{распознала эту служебную функцию и таблицу автоматически, дописав комментарии к меткам вроде}
\EN{successfully perceived it as a service function and table, automatically, and added commentaries to labels
like}\\
\TT{jumptable 000000FA case 0}.

