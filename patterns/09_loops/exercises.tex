\section{\Exercises}

\subsection{\Exercise \#1}

\index{x86!\Instructions!LOOP}
\RU{Почему инструкция}\EN{Why isn't the} \LOOP \RU{больше не используется современными 
компиляторами}\EN{instruction used by modern compilers anymore}?

\subsection{\Exercise \#2}

\RU{Возьмите пример рассмотренный в этой секции}\EN{Take a loop example from this section} 
(\ref{loops_src}), 
\RU{скомпилируйте его в вашей любимой}\EN{compile it in your favorite} \ac{OS}
\RU{и компиляторе, и модифицируйте исполняемый файл так, чтобы цикл был в пределах}\EN{and compiler 
and modify (patch) the executable file so the loop range will be} [6..20].

\subsection{\Exercise \#3}
\label{exercise_loops_3}

\WhatThisCodeDoes

\begin{lstlisting}[caption=\Optimizing MSVC 2010]
$SG2795	DB	'%d', 0aH, 00H

_main	PROC
	push	esi
	push	edi
	mov	edi, DWORD PTR __imp__printf
	mov	esi, 100
	npad	3 ; align next label
$LL3@main:
	push	esi
	push	OFFSET $SG2795 ; '%d'
	call	edi
	dec	esi
	add	esp, 8
	test	esi, esi
	jg	SHORT $LL3@main
	pop	edi
	xor	eax, eax
	pop	esi
	ret	0
_main	ENDP
\end{lstlisting}

\begin{lstlisting}[caption=\NonOptimizingKeilVI (\ARMMode)]
main PROC
        PUSH     {r4,lr}
        MOV      r4,#0x64
|L0.8|
        MOV      r1,r4
        ADR      r0,|L0.40|
        BL       __2printf
        SUB      r4,r4,#1
        CMP      r4,#0
        MOVLE    r0,#0
        BGT      |L0.8|
        POP      {r4,pc}
        ENDP

|L0.40|
        DCB      "%d\n",0
\end{lstlisting}

\begin{lstlisting}[caption=\NonOptimizingKeilVI (\ThumbMode)]
main PROC
        PUSH     {r4,lr}
        MOVS     r4,#0x64
|L0.4|
        MOVS     r1,r4
        ADR      r0,|L0.24|
        BL       __2printf
        SUBS     r4,r4,#1
        CMP      r4,#0
        BGT      |L0.4|
        MOVS     r0,#0
        POP      {r4,pc}
        ENDP

        DCW      0x0000
|L0.24|
        DCB      "%d\n",0
\end{lstlisting}

\begin{lstlisting}[caption=\Optimizing GCC 4.9 (ARM64)]
main:
	stp	x29, x30, [sp, -32]!
	add	x29, sp, 0
	stp	x19, x20, [sp,16]
	adrp	x20, .LC0
	mov	w19, 100
	add	x20, x20, :lo12:.LC0
.L2:
	mov	w1, w19
	mov	x0, x20
	bl	printf
	subs	w19, w19, #1
	bne	.L2
	ldp	x19, x20, [sp,16]
	ldp	x29, x30, [sp], 32
	ret
.LC0:
	.string	"%d\n"
\end{lstlisting}

\lstinputlisting[caption=\Optimizing GCC 4.4.5 (MIPS) (IDA)]{patterns/09_loops/ex3_MIPS_O3_IDA.lst}

\RU{Ответ}\EN{Answer}: \ref{exercise_solutions_loops_3}.

\subsection{\Exercise \#4}
\label{exercise_loops_4}

\WhatThisCodeDoes

\begin{lstlisting}[caption=\Optimizing MSVC 2010]
$SG2795	DB	'%d', 0aH, 00H

_main	PROC
	push	esi
	push	edi
	mov	edi, DWORD PTR __imp__printf
	mov	esi, 1
	npad	3 ; align next label
$LL3@main:
	push	esi
	push	OFFSET $SG2795 ; '%d'
	call	edi
	add	esi, 3
	add	esp, 8
	cmp	esi, 100
	jl	SHORT $LL3@main
	pop	edi
	xor	eax, eax
	pop	esi
	ret	0
_main	ENDP
\end{lstlisting}

\begin{lstlisting}[caption=\NonOptimizingKeilVI (\ARMMode)]
main PROC
        PUSH     {r4,lr}
        MOV      r4,#1
|L0.8|
        MOV      r1,r4
        ADR      r0,|L0.40|
        BL       __2printf
        ADD      r4,r4,#3
        CMP      r4,#0x64
        MOVGE    r0,#0
        BLT      |L0.8|
        POP      {r4,pc}
        ENDP

|L0.40|
        DCB      "%d\n",0
\end{lstlisting}

\begin{lstlisting}[caption=\NonOptimizingKeilVI (\ThumbMode)]
main PROC
        PUSH     {r4,lr}
        MOVS     r4,#1
|L0.4|
        MOVS     r1,r4
        ADR      r0,|L0.24|
        BL       __2printf
        ADDS     r4,r4,#3
        CMP      r4,#0x64
        BLT      |L0.4|
        MOVS     r0,#0
        POP      {r4,pc}
        ENDP

        DCW      0x0000
|L0.24|
        DCB      "%d\n",0
\end{lstlisting}

\begin{lstlisting}[caption=\Optimizing GCC 4.9 (ARM64)]
main:
	stp	x29, x30, [sp, -32]!
	add	x29, sp, 0
	stp	x19, x20, [sp,16]
	adrp	x20, .LC0
	mov	w19, 1
	add	x20, x20, :lo12:.LC0
.L2:
	mov	w1, w19
	mov	x0, x20
	add	w19, w19, 3
	bl	printf
	cmp	w19, 100
	bne	.L2
	ldp	x19, x20, [sp,16]
	ldp	x29, x30, [sp], 32
	ret
.LC0:
	.string	"%d\n"
\end{lstlisting}

\lstinputlisting[caption=\Optimizing GCC 4.4.5 (MIPS) (IDA)]{patterns/09_loops/ex4_MIPS_O3_IDA.lst}

\RU{Ответ}\EN{Answer}: \ref{exercise_solutions_loops_4}.
