\section{\RU{Несколько итераторов}\EN{Several iterators}}
\label{loop_iterators}

\RU{Часто, у цикла только один итератор, но в итоговом коде их может быть несколько.}
\EN{Common loop has only one iterator, but there could be several in resulting code.}

\RU{Вот очень простой пример}\EN{Here is very simple example}:

\lstinputlisting{patterns/09_loops/iterators/iterators.c.\LANG}

\RU{Здесь два умножения на каждой итерации, а это дорогая операция.}
\EN{There are two multiplications at each iteration and this is costly operation.}
\RU{Сможем ли мы соптимизировать это как-то}\EN{Can we optimize it somehow}?
\RU{Да, если мы заметим что индексы обоих массивов перескакивают на места, расчитать которые мы
можем легко и без умножения.}
\EN{Yes, if we will notice that both array indices are leaping on places we can calculate easily without 
multiplication.}

\subsection{\RU{Три итератора}\EN{Three iterators}}

\lstinputlisting[caption=\Optimizing MSVC 2013 x64]{patterns/09_loops/iterators/MSVC_2013_x64_Ox.asm.\LANG}

\RU{Теперь здесь три итератора: переменная \IT{cnt} и два индекса, они увеличиваются на 12 и 28 на каждой
итерации, указывая на новые элементы массивов.}
\EN{Now there are 3 iterators: \IT{cnt} variable and two indices, they are increased by 12 and 28 at 
each iteration,
pointing to new array elements.}
\RU{Мы можем переписать этот код на}\EN{We can rewrite this code in} \CCpp:

\lstinputlisting{patterns/09_loops/iterators/iterators3.c.\LANG}

\RU{Так что, ценой модификации трех итераторов на каждой итерации вместо одного, 
мы избавлены от двух операций умножения.}
\EN{So, at the cost of updating 3 iterators on each iteration instead of one, 
two multiplication operations are dropped.}

\subsection{\RU{Два итератора}\EN{Two iterators}}

\RU{GCC 4.9 сделал еще больше, оставив только 2 итератора:}
\EN{GCC 4.9 did even more, leaving only 2 iterators:}

\lstinputlisting[caption=\Optimizing GCC 4.9 x64]{patterns/09_loops/iterators/GCC491_x64_O3.asm.\LANG}

\RU{Здесь больше нет переменной-\IT{счетчика}: GCC рассудил, что она не нужна.}
\EN{There are no \IT{counter} variable: GCC concluded it not needed.}
\RU{Последний элемент массива \IT{a2} вычисляется перед началом цикла (а это просто: $cnt*7$),
и при помощи этого цикл останавливается: просто исполняйте его пока второй индекс не сравняется
с предвычисленным значением.}
\EN{Last element of \IT{a2} array is calculated before loop begin (which is easy: $cnt*7$) 
and that's how loop will be stopped: just iterate until second index not reached this precalculated value.}

\RU{Об умножении используя сдвиги/сложения/вычитания, читайте здесь:}
\EN{About multiplication using shifts/additions/subtractions read here:} 
\ref{multiplication_using_shifts_adds_subs}.

\RU{Этот код можно переписать на \CCpp вот так:}
\EN{This code can be rewritten into \CCpp like that:}

\lstinputlisting{patterns/09_loops/iterators/iterators2.c.\LANG}

\ifdefined\IncludeARM
GCC (Linaro) 4.9 \ForENRU ARM64 \RU{делает тоже самое, только предвычисляет последний индекс массива \IT{a1} вместо
\IT{a2}, а это, конечно, имеет тот же эффект}\EN{do the same, but precalculates last index of \IT{a1} 
array instead of \IT{a2}, and this has the same effect, of course}:

\lstinputlisting[caption=\Optimizing GCC (Linaro) 4.9 ARM64,label=GCC_no_number_sign]{patterns/09_loops/iterators/ARM64_GCC_49_O3.s.\LANG}

% FIXME -> post-increment
\fi

\ifdefined\IncludeMIPS
GCC 4.4.5 \ForENRU MIPS \RU{делает то же самое}\EN{do the same}:

\lstinputlisting[caption=\Optimizing GCC 4.4.5 \ForENRU MIPS (IDA)]{patterns/09_loops/iterators/MIPS_O3_IDA.lst.\LANG}
\fi

\ifx\LITE\undefined
\subsection{\RU{Случай }Intel C++ 2011\EN{ case}}
\index{\CompilerAnomaly}
\label{loops_iterators_loop_anomaly}

\RU{Оптимизации компилятора могут быть очень странными, но, тем не менее, корректными.}
\EN{Compiler optimizations can also be weird, but nevertheless, still correct.}
\RU{Вот что делает Intel C++ 2011}\EN{Here is what Intel C++ compiler 2011 do}:

\lstinputlisting[caption=\Optimizing Intel C++ 2011 x64]{patterns/09_loops/iterators/intel_2011_x64_Ox.asm}

\RU{В начале, принимаются какие-то решения, затем исполняется одна из процедур.}
\EN{First, there are some decisions taken, than one of the routines is executed.}
\RU{Вероятно, это проверка, не пересекаются ли массивы.}
\EN{Supposedly, it is checked, if arrays are intersected.}
\RU{Это хорошо известный способ оптимизации процедур копирования блоков в памяти.}
\EN{This is very well known way of optimizing memory blocks copy routines.}
\RU{Но копирующие процедуры одинаковые}\EN{But copy routines are the same}!
\RU{Видимо, это ошибка оптимизатора Intel C++, который, тем не менее, генерирует работоспособный код.}
\EN{This is probably error of Intel C++ optimizer, which still produce workable code, though.}

\RU{Я добавляю примеры такого кода в мою книгу чтобы читатель мог понимать, что результаты работы
компилятором иногда бывают крайне странными, но корректными, потому что когда компилятор тестировали, 
тесты прошли нормально.}
\EN{I'm adding such example code to my book so reader would understand that compilers output is weird at times,
but still correct, because when compiler was tested, tests were passed.}
\fi
