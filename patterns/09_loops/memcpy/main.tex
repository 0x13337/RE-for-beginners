\section{\RU{Функция копирования блоков памяти}\EN{Memory blocks copying routine}}
\label{loop_memcpy}

\RU{Настоящие функции копирования памяти могут копировать по 4 или 8 байт на каждой итерации, использовать \ac{SIMD},
векторизацию, \etc{}.}
\EN{Real-world memory copy routines may copy 4 or 8 bytes at each iteration, use \ac{SIMD}, 
vectorization, \etc{}.}
\RU{Но ради простоты, этот пример настолько прост, насколько это возможно.}
\EN{But for the sake of simplicity, this example is the simplest possible.}

\lstinputlisting{memcpy.c}

\subsection{\RU{Простейшая реализация}\EN{Straight-forward implementation}}

\lstinputlisting[caption=GCC 4.9 x64 \RU{оптимизация по размеру}\EN{optimized for size} (-Os)]{patterns/09_loops/memcpy/memcpy_GCC49_x64_Os.s.\LANG}

\ifdefined\IncludeARM

\lstinputlisting[caption=GCC 4.9 ARM64 \RU{оптимизация по размеру}\EN{optimized for size} (-Os)]{patterns/09_loops/memcpy/memcpy_GCC49_ARM64_Os.s.\LANG}

\lstinputlisting[caption=\OptimizingKeilVI (\ThumbMode)]{patterns/09_loops/memcpy/memcpy_Keil_Thumb_O3.s.\LANG}

\subsection{ARM \RU{в режиме ARM}\EN{in ARM mode}}

\RU{Keil в режиме ARM пользуется условными суффиксами:}
\EN{Keil in ARM mode takes full advantage of conditional suffixes:}

\lstinputlisting[caption=\OptimizingKeilVI (\ARMMode)]{patterns/09_loops/memcpy/memcpy_Keil_ARM_O3.s.\LANG}

\RU{Вот почему здесь только одна инструкция перехода вместо двух.}
\EN{That's why there is only one branch instruction instead of 2.}

\fi

\ifdefined\IncludeMIPS
\subsection{MIPS}

\lstinputlisting[caption=GCC 4.4.5 \RU{оптимизация по размеру}\EN{optimized for size} (-Os) (IDA)]{patterns/09_loops/memcpy/memcpy_MIPS_Os_IDA.lst.\LANG}

\index{MIPS!\Instructions!LBU}
\index{MIPS!\Instructions!SB}
\RU{Здесь две новых для нас инструкций:}
\EN{Here we have two new instructions:} LBU (\q{Load Byte Unsigned}) \AndENRU SB (\q{Store Byte}).
\RU{Так же как и в ARM, все регистры в MIPS имеют длину в 32 бита. Здесь нет частей регистров равных байту,
как в x86.}
\EN{Just like in ARM, all MIPS registers are 32-bit wide, there are no byte-wide parts like in x86.}
\RU{Так что когда нужно работать с байтами, приходится выделять целый 32-битный регистр для этого.}
\EN{So when dealing with single bytes, we have to allocate whole 32-bit registers for them.}
\RU{LBU загружает байт и сбрасывает все остальные биты (\q{Unsigned}).}
\EN{LBU loads a byte and clears all other bits (\q{Unsigned}).}
\index{MIPS!\Instructions!LB}
\RU{И напротив, инструкция LB (\q{Load Byte}) расширяет байт до 32-битного значения учитывая знак.}
\EN{On the other hand, LB (\q{Load Byte}) instruction sign-extends the loaded byte to a 32-bit value.}
\RU{SB просто записывает байт из младших 8 бит регистра в память.}
\EN{SB just writes a byte from lowest 8 bits of register to memory.}

\fi

\ifx\LITE\undefined
\subsection{\RU{Векторизация}\EN{Vectorization}}

\Optimizing GCC \RU{может из этого примера сделать намного больше}\EN{can do much more on this example}: 
\myref{vec_memcpy}.
\fi
