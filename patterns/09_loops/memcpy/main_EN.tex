\section{Memory blocks copying routine}
\label{loop_memcpy}

Real-world memory copy routines may copy 4 or 8 bytes at each iteration, use \ac{SIMD}, 
vectorization, \etc{}.
But for the sake of simplicity, this example is the simplest possible.

\lstinputlisting{memcpy.c}

\subsection{Straight-forward implementation}

\lstinputlisting[caption=GCC 4.9 x64 optimized for size (-Os)]{patterns/09_loops/memcpy/memcpy_GCC49_x64_Os.s.\LANG}

\ifdefined\IncludeARM

\lstinputlisting[caption=GCC 4.9 ARM64 optimized for size (-Os)]{patterns/09_loops/memcpy/memcpy_GCC49_ARM64_Os.s.\LANG}

\lstinputlisting[caption=\OptimizingKeilVI (\ThumbMode)]{patterns/09_loops/memcpy/memcpy_Keil_Thumb_O3.s.\LANG}

\subsection{ARM in ARM mode}

Keil in ARM mode takes full advantage of conditional suffixes:

\lstinputlisting[caption=\OptimizingKeilVI (\ARMMode)]{patterns/09_loops/memcpy/memcpy_Keil_ARM_O3.s.\LANG}

That's why there is only one branch instruction instead of 2.

\fi

\ifdefined\IncludeMIPS
\subsection{MIPS}

\lstinputlisting[caption=GCC 4.4.5 optimized for size (-Os) (IDA)]{patterns/09_loops/memcpy/memcpy_MIPS_Os_IDA.lst.\LANG}

\index{MIPS!\Instructions!LBU}
\index{MIPS!\Instructions!SB}

Here we have two new instructions: LBU (\q{Load Byte Unsigned}) and SB (\q{Store Byte}).

Just like in ARM, all MIPS registers are 32-bit wide, there are no byte-wide parts like in x86.

So when dealing with single bytes, we have to allocate whole 32-bit registers for them.

LBU loads a byte and clears all other bits (\q{Unsigned}).
\index{MIPS!\Instructions!LB}

On the other hand, LB (\q{Load Byte}) instruction sign-extends the loaded byte to a 32-bit value.

SB just writes a byte from lowest 8 bits of register to memory.

\fi

\subsection{Vectorization}

\Optimizing GCC can do much more on this example: 
\myref{vec_memcpy}.
