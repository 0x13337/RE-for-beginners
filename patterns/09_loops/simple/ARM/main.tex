\subsection{ARM}

\subsubsection{\NonOptimizingKeilVI (\ARMMode)}

\lstinputlisting[label=Keil_number_sign]{patterns/09_loops/simple/ARM/Keil_ARM_O0.asm}

\RU{Счетчик итераций $i$ будет храниться в регистре \Reg{4}.}
\EN{Iteration counter $i$ is to be stored in the \Reg{4} register.}

\EN{The}\RU{Инструкция} \TT{\q{MOV R4, \#2}} \RU{просто инициализирует}\EN{instruction just initializes} $i$.

\EN{The}\RU{Инструкции} \TT{\q{MOV R0, R4}} \AndENRU \TT{\q{BL printing\_function}} \RU{составляют тело цикла.}\EN{instructions
compose the body of the loop}, 
\RU{Первая инструкция готовит аргумент для функции, \ttf а вторая вызывает её.}
\EN{the first instruction preparing the argument for \ttf function and the second calling the function.}

\index{ARM!\Instructions!ADD}
\EN{The}\RU{Инструкция} \TT{\q{ADD R4, R4, \#1}} \RU{прибавляет единицу к $i$ при каждой итерации.}
\EN{instruction just adds 1 to the $i$ variable at each iteration.}

\index{ARM!\Instructions!CMP}
\index{ARM!\Instructions!BLT}
\TT{\q{CMP R4, \#0xA}} \RU{сравнивает}\EN{compares} $i$ \RU{с}\EN{with} \TT{0xA} (10). 
\RU{Следующая за ней инструкция \TT{BLT} (\IT{Branch Less Than}) совершит переход, 
если $i$ меньше чем 10.}
\EN{The next instruction \TT{BLT} (\IT{Branch Less Than}) 
jumps if $i$ is less than 10.}

\RU{В противном случае в \Reg{0} запишется 0 (потому что наша функция возвращает 0)
и произойдет выход из функции.}
\EN{Otherwise, 0 is to be written into \Reg{0} (since our function returns 0)
and function execution finishes.}

\subsubsection{\OptimizingKeilVI (\ThumbMode)}

\lstinputlisting{patterns/09_loops/simple/ARM/Keil_thumb_O3.asm}

\RU{Практически всё то же самое.}\EN{Practically the same.}

\subsubsection{\OptimizingXcodeIV (\ThumbTwoMode)}
\label{ARM_unrolled_loops}

\lstinputlisting{patterns/09_loops/simple/ARM/xcode_thumb_O3.asm}

\RU{На самом деле, в моей функции \ttf было такое:}\EN{In fact, this was in my \ttf function:}

\begin{lstlisting}
void printing_function(int i)
{
    printf ("%d\n", i);
};
\end{lstlisting}

\index{Unrolled loop}
\index{Inline code}
\RU{Так что}\EN{So,} LLVM \RU{не только \IT{развернул} цикл}\EN{not just \IT{unrolled} the loop}, 
\RU{но также и представил мою очень простую функцию \ttf как \IT{inline-функцию}}\EN{but also \IT{inlined} my 
very simple function \ttf},
\RU{и вставил её тело вместо цикла 8 раз}\EN{and inserted its body 8 times instead of calling it}. 
\RU{Это возможно, когда функция очень простая (как та что у меня) и когда
она вызывается не очень много раз, как здесь.}
\EN{This is possible when the function is so simple (like mine) and when it is not called too much (like here).}

\subsubsection{ARM64: \Optimizing GCC 4.9.1}

\lstinputlisting[caption=\Optimizing GCC 4.9.1]{patterns/09_loops/simple/ARM/ARM64_GCC491_O3.s.\LANG}

\subsubsection{ARM64: \NonOptimizing GCC 4.9.1}

\lstinputlisting[caption=\NonOptimizing GCC 4.9.1 -fno-inline]{patterns/09_loops/simple/ARM/ARM64_GCC491_O3.s.\LANG}
