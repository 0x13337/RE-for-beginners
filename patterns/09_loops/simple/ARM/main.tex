\subsection{ARM}

\subsubsection{\NonOptimizingKeilVI (\ARMMode)}

\lstinputlisting[label=Keil_number_sign]{patterns/09_loops/simple/ARM/Keil_ARM_O0.asm}

\RU{Счетчик итераций $i$ будет храниться в регистре \Reg{4}.}
\EN{Iteration counter $i$ is to be stored in the \Reg{4} register.}

\RU{Инструкция }\TT{``MOV R4, \#2''} \RU{просто инициализирует}\EN{instruction just initializing} $i$.

\RU{Инструкции }\TT{``MOV R0, R4''} \AndENRU \TT{``BL printing\_function''} \RU{составляют тело цикла}\EN{instructions are
compose loop body}, 
\RU{первая инструкция готовит аргумент для функции \ttf и вторая собственно вызывает её.}
\EN{the first instruction preparing argument for \ttf function and the second is calling it.}

\index{ARM!\Instructions!ADD}
\RU{Инструкция }\TT{``ADD R4, R4, \#1''} \RU{прибавляет единицу к $i$ при каждой итерации.}
\EN{instruction is just adding $1$ to the $i$ variable during each iteration.}

\index{ARM!\Instructions!CMP}
\index{ARM!\Instructions!BLT}
\TT{``CMP R4, \#0xA''} \RU{сравнивает}\EN{comparing} $i$ \RU{с}\EN{with} \TT{0xA} ($10$). 
\RU{Следующая за ней инструкция}\EN{Next instruction} \TT{BLT} (\IT{Branch Less Than}) 
\RU{совершит переход, если}\EN{will jump if} $i$ \RU{меньше чем}\EN{is less than} $10$.

\RU{В противном случае}\EN{Otherwise}, \RU{в \Reg{0} запишется $0$}\EN{$0$ will be written into \Reg{0}} 
(\RU{потому что наша функция возвращает}\EN{since our function returns} $0$) 
\RU{и произойдет выход из функции}\EN{and function execution ended}.

\subsubsection{\OptimizingKeilVI (\ThumbMode)}

\lstinputlisting{patterns/09_loops/simple/ARM/Keil_thumb_O3.asm}

\RU{Практически, всё то же самое.}\EN{Practically, the same.}

\subsubsection{\OptimizingXcodeIV (\ThumbTwoMode)}
\label{ARM_unrolled_loops}

\lstinputlisting{patterns/09_loops/simple/ARM/xcode_thumb_O3.asm}

\RU{На самом деле, в моей функции \ttf было такое:}\EN{In fact, this was in my \ttf function:}

\begin{lstlisting}
void printing_function(int i)
{
    printf ("%d\n", i);
};
\end{lstlisting}

\index{Unrolled loop}
\index{Inline code}
\RU{Так что}\EN{So}, LLVM \RU{не только \IT{развернул} цикл}\EN{not just \IT{unrolled} the loop}, 
\RU{но также и представил мою очень простую функцию \ttf как \IT{inline-вую}}\EN{but also represented my 
very simple function \ttf as \IT{inlined}},
\RU{и вставил её тело вместо цикла 8 раз}\EN{and inserted its body 8 times instead of loop}. 
\RU{Это возможно, когда функция очень простая, как та что у меня, и когда
она вызывается не очень много раз, как здесь.}
\EN{This is possible when function is so primitive (like mine) and when it is called not many times (like here).}

\subsubsection{ARM64: \Optimizing GCC 4.9.1}

\lstinputlisting[caption=\Optimizing GCC 4.9.1]{patterns/09_loops/simple/ARM/ARM64_GCC491_O3.s.\LANG}

\subsubsection{ARM64: \NonOptimizing GCC 4.9.1}

\lstinputlisting[caption=\NonOptimizing GCC 4.9.1 -fno-inline]{patterns/09_loops/simple/ARM/ARM64_GCC491_O3.s.\LANG}
