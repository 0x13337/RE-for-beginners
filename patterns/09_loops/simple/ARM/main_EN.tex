\subsectionold{ARM}

\subsubsectionold{\NonOptimizingKeilVI (\ARMMode)}

\lstinputlisting[label=Keil_number_sign]{patterns/09_loops/simple/ARM/Keil_ARM_O0.asm}

Iteration counter $i$ is to be stored in the \Reg{4} register.
The \INS{MOV R4, \#2} instruction just initializes $i$.
The \INS{MOV R0, R4} and \INS{BL printing\_function} instructions
compose the body of the loop, the first instruction preparing the argument for 
\ttf function and the second calling the function.
\myindex{ARM!\Instructions!ADD}
The \INS{ADD R4, R4, \#1} instruction just adds 1 to the $i$ variable at each iteration.
\myindex{ARM!\Instructions!CMP}
\myindex{ARM!\Instructions!BLT}
\INS{CMP R4, \#0xA} compares $i$ with \TT{0xA} (10). 
The next instruction \INS{BLT} (\IT{Branch Less Than}) jumps if $i$ is less than 10.
Otherwise, 0 is to be written into \Reg{0} (since our function returns 0)
and function execution finishes.

\subsubsectionold{\OptimizingKeilVI (\ThumbMode)}

\lstinputlisting{patterns/09_loops/simple/ARM/Keil_thumb_O3.asm}

Practically the same.

\subsubsectionold{\OptimizingXcodeIV (\ThumbTwoMode)}
\label{ARM_unrolled_loops}

\lstinputlisting{patterns/09_loops/simple/ARM/xcode_thumb_O3.asm}

In fact, this was in my \ttf function:

\begin{lstlisting}
void printing_function(int i)
{
    printf ("%d\n", i);
};
\end{lstlisting}

\myindex{Unrolled loop}
\myindex{Inline code}
So, LLVM not just \IT{unrolled} the loop, 
but also \IT{inlined} my 
very simple function \ttf,
and inserted its body 8 times instead of calling it. 

This is possible when the function is so simple (like mine) and when it is not called too much (like here).

\subsubsectionold{ARM64: \Optimizing GCC 4.9.1}

\lstinputlisting[caption=\Optimizing GCC 4.9.1]{patterns/09_loops/simple/ARM/ARM64_GCC491_O3_EN.s}

\subsubsectionold{ARM64: \NonOptimizing GCC 4.9.1}

\lstinputlisting[caption=\NonOptimizing GCC 4.9.1 -fno-inline]{patterns/09_loops/simple/ARM/ARM64_GCC491_O3_EN.s}
