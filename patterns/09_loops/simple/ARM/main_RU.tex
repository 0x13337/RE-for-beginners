\subsection{ARM}

\subsubsection{\NonOptimizingKeilVI (\ARMMode)}

\lstinputlisting[label=Keil_number_sign]{patterns/09_loops/simple/ARM/Keil_ARM_O0.asm}

Счетчик итераций $i$ будет храниться в регистре \Reg{4}.
Инструкция \INS{MOV R4, \#2} просто инициализирует $i$.
Инструкции \INS{MOV R0, R4} и \INS{BL printing\_function} составляют тело цикла., 
Первая инструкция готовит аргумент для функции, \ttf а вторая вызывает её.
\myindex{ARM!\Instructions!ADD}
Инструкция \INS{ADD R4, R4, \#1} прибавляет единицу к $i$ при каждой итерации.
\myindex{ARM!\Instructions!CMP}
\myindex{ARM!\Instructions!BLT}
\INS{CMP R4, \#0xA} сравнивает $i$ с \TT{0xA} (10). 
Следующая за ней инструкция \INS{BLT} (\IT{Branch Less Than}) совершит переход, если $i$ меньше чем 10.
В противном случае в \Reg{0} запишется 0 (потому что наша функция возвращает 0) 
и произойдет выход из функции.

\subsubsection{\OptimizingKeilVI (\ThumbMode)}

\lstinputlisting{patterns/09_loops/simple/ARM/Keil_thumb_O3.asm}

Практически всё то же самое.

\subsubsection{\OptimizingXcodeIV (\ThumbTwoMode)}
\label{ARM_unrolled_loops}

\lstinputlisting{patterns/09_loops/simple/ARM/xcode_thumb_O3.asm}

На самом деле, в моей функции \ttf было такое:

\begin{lstlisting}
void printing_function(int i)
{
    printf ("%d\n", i);
};
\end{lstlisting}

\myindex{Unrolled loop}
\myindex{Inline code}
Так что LLVM не только \IT{развернул} цикл, 
но также и представил мою очень простую функцию \ttf как \IT{inline-функцию},
и вставил её тело вместо цикла 8 раз. 
Это возможно, когда функция очень простая (как та что у меня) и когда
она вызывается не очень много раз, как здесь.

\subsubsection{ARM64: \Optimizing GCC 4.9.1}

\lstinputlisting[caption=\Optimizing GCC 4.9.1]{patterns/09_loops/simple/ARM/ARM64_GCC491_O3_RU.s}

\subsubsection{ARM64: \NonOptimizing GCC 4.9.1}

\lstinputlisting[caption=\NonOptimizing GCC 4.9.1 -fno-inline]{patterns/09_loops/simple/ARM/ARM64_GCC491_O3_RU.s}
