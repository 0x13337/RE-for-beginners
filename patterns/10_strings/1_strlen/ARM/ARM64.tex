\subsubsection{ARM64}

\myparagraph{\Optimizing GCC (Linaro) 4.9}

\lstinputlisting{patterns/10_strings/1_strlen/ARM/ARM64_GCC_O3.lst.\LANG}

\RU{Алгоритм такой же как и в}\EN{The algorithm is the same as in} \myref{strlen_MSVC_Ox}: 
\RU{найти нулевой байт, затем вычислить разницу между указателями, затем отнять 1 от результата}\EN{find a zero 
byte, calculate the difference between the pointers and decrement the result by 1}.
\RU{Комментарии добавлены автором книги}\EN{Some comments were added by the author of this book}.

\RU{Стоит добавить, что наш пример имеет ошибку: \TT{my\_strlen()}
возвращает 32-битный \Tint, тогда как должна возвращать \TT{size\_t} или иной 64-битный тип.}%
\EN{The only thing worth noting is that our example is somewhat wrong: \TT{my\_strlen()}
returns 32-bit \Tint, while it has to return \TT{size\_t} or another 64-bit type.}

\RU{Причина в том, что теоретически, \TT{strlen()} можно вызывать для огромных блоков в памяти,
превышающих 4GB, так что она должна иметь возможность вернуть 64-битное значение на 64-битной платформе.}
\EN{The reason is that, theoretically, \TT{strlen()} can be called for a huge blocks in memory that exceeds
4GB, so it must able to return a 64-bit value on 64-bit platforms.}
\RU{Так что из-за моей ошибки, последняя инструкция \SUB работает над 32-битной частью регистра, тогда
как предпоследняя \SUB работает с полными 64-битными частями (она вычисляет разницу между указателями).}
\EN{Because of my mistake, the last \SUB instruction operates on a 32-bit part of register, while the penultimate
\SUB instruction works on full the 64-bit register (it calculates the difference between the pointers).}
\RU{Это моя ошибка, но лучше оставить это как есть, как пример кода, который возможен в таком случае.}
\EN{It's my mistake, it is better to leave it as is, as an example of how the code could look like in such case.}

\myparagraph{\NonOptimizing GCC (Linaro) 4.9}

\lstinputlisting{patterns/10_strings/1_strlen/ARM/ARM64_GCC_O0.lst.\LANG}

\RU{Более многословно}\EN{It's more verbose}.
\RU{Переменные часто сохраняются в память и загружаются назад (локальный стек)}\EN{The variables are 
often tossed here to and from memory (local stack)}.
\RU{Здесь та же ошибка: операция декремента происходит над 32-битной частью регистра}\EN{The same mistake here: 
the decrement operation happens on a 32-bit register part}.
