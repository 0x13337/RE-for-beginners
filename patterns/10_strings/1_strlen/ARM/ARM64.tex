\subsubsection{ARM64}

\myparagraph{\Optimizing GCC (Linaro) 4.9}

\lstinputlisting{patterns/10_strings/1_strlen/ARM/ARM64_GCC_O3.lst.\LANG}

\RU{Алгоритм такой же как и в}\EN{The algorithm is the same as in} \ref{strlen_MSVC_Ox}: 
\RU{найти нулевой байт, затем вычислить разницу между указателями, затем отнять 1 от результата}\EN{find a zero 
byte, then calculate difference between pointers, then decrement result}.
\RU{Я добавил комментариев}\EN{I added some comments}.
\RU{Вот что еще стоит добавить, так это то что мой пример с ошибкой}\EN{The only thing worth noting is that 
my example is somewhat broken}: \TT{my\_strlen()} \RU{возвращает 32-битный}\EN{function returns
32-bit} \Tint, \RU{тогда как должна возвращать}\EN{while it should return} \TT{size\_t} \RU{или иной 64-битный
тип}\EN{or other 64-bit type}.
\RU{Причина в том, что, теоретически, \TT{strlen()} можно вызывать для огромных блоков в памяти,
превышающих 4GB, так что она должна иметь возможность вернуть 64-битное значение на 64-битной платформе.}
\EN{The reason for that is that, theoretically, \TT{strlen()} can be called for huge blocks in memory, exceeding
4GB, so it must able to return 64-bit value on 64-bit platform.}
\RU{Так что из-за моей ошибки, последняя инструкция \SUB работает над 32-битной частью регистра, тогда
как предпоследняя \SUB работает с полными 64-битными частями (она вычисляет разницу между указателями).}
\EN{So because of my mistake, the last \SUB instruction operates on 32-bit part of register, while penultimate
\SUB instruction works on full 64-bit parts (it calculates pointer difference).}
\RU{Это моя ошибка, но я решил оставить это как есть, как пример кода, который возможен в таком случае.}
\EN{It's my mistake, but I decided to leave it as is, as an example of what code could be in such case.}

\myparagraph{\NonOptimizing GCC (Linaro) 4.9}

\lstinputlisting{patterns/10_strings/1_strlen/ARM/ARM64_GCC_O0.lst.\LANG}

\RU{Более многословно}\EN{It's more verbose}.
\RU{Переменные часто сохраняются в память и загружаются назад (локальный стек)}\EN{Variables are 
often tossed here to and from memory (local stack)}.
\RU{Здесь та же ошибка: операция декремента происходит над 32-битной частью регистра}\EN{The same mistake here: 
decrement operation is happen on 32-bit register part}.
