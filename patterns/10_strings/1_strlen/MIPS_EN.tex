\subsection{MIPS}

\lstinputlisting[caption=\Optimizing GCC 4.4.5 (IDA)]{patterns/10_strings/1_strlen/MIPS_O3_IDA_EN.lst}

\myindex{MIPS!\Instructions!NOR}
\myindex{MIPS!\Pseudoinstructions!NOT}

MIPS lacks a \NOT instruction, but has \NOR which is \TT{OR~+~NOT} operation.

This operation is widely used in digital electronics\footnote{NOR is called \q{universal gate}.
\index{Apollo Guidance Computer}
For example, the Apollo Guidance Computer used in the Apollo program, 
was built by only using 5600 NOR gates:
[Jens Eickhoff, \IT{Onboard Computers, Onboard Software and Satellite Operations: An Introduction}, (2011)].
But NOR isn't very popular in computer programming.

So, the NOT operation is implemented here as \TT{NOR~DST,~\$ZERO,~SRC}.

From fundamentals \myref{sec:signednumbers} we know that bitwise inverting a signed number is the same 
as changing its sign and subtracting 1 from the result.

So what \NOT does here is to take the value of $str$ and transform it into $-str-1$.
The addition operation that follows prepares result.

