\section{\RU{Обрезка строк}\EN{Strings trimming}}
\newcommand{\CRLF}{\ac{CR}/\ac{LF}}

\RU{Еще одна весьма востребованная операция это удаление некоторых символов в начале и/или конце
строки.}
\EN{Another very common task is to remove some characters on begin and/or end.}

\RU{В этом примере, мы будем работать с ф-цией, удаляющей все символы перевода строки 
(\CRLF{}) в конце входной строки:}
\EN{In this example, we will work with a function which removes all newline characters 
(\CRLF{}) at the input string end:}

\lstinputlisting{patterns/10_strings/4_trim/strtrim.c}

\RU{Входной аргумент всегда возвращается на выходе, это удобно, когда вам нужно объеденять
ф-ции обработки строк в цепочки, как это сделано здесь в ф-ции \main.}
\EN{Input argument is always returned on exit, this is convenient when you need to chain 
string processing functions, like it was done here in the \main function.}

\RU{Вторая часть for() (\TT{str\_len>0 \&\& (c=s[str\_len-1])}) называется в \CCpp ``short-circuit'' 
(короткое замыкание) и это очень удобно: \cite[1.3.8]{CBook}.}
\EN{The second part of for() (\TT{str\_len>0 \&\& (c=s[str\_len-1])}) is so called ``short-circuit'' 
in \CCpp and is very convenient \cite[1.3.8]{CBook}.}
\RU{Компиляторы \CCpp гарантируют последовательное вычисление слева направо.}
\EN{\CCpp compilers guarantee evaluation sequence from left to right.}
\RU{Так что если первое условие не истинно после вычисления, второе никогда не будет
вычисляться.}
\EN{So if the first clause is false after evaluation, second will never be evaluated.}

% subsections
\subsection{x64: \Optimizing MSVC 2013}

\lstinputlisting[caption=\Optimizing MSVC 2013 x64]{patterns/10_strings/4_trim/MSVC2013_x64_Ox.asm}

\RU{В начале, MSVC вставил тело ф-ции \strlen{} прямо в код, потому что решил, что так будет
быстрее чем обычная работа \strlen{} + время на вызов её и возврат из нее.}
\EN{First, MSVC inlined \strlen{} function code right in the code, because it concludes this 
will be faster then usual \strlen{} work + cost of calling it and returning from it.}
\RU{Это также называется \IT{inlining}}\EN{This is called inlining}: \ref{inline_code}.

\RU{Первая инструкия ф-ции \strlen{} вставленная здесь, это}\EN{First instruction of 
inlined \strlen{} is} \TT{OR RAX, 0xFFFFFFFFFFFFFFFF}. 
\RU{Я не знаю, почему MSVC использовала \TT{OR} вместо \TT{MOV RAX, 0xFFFFFFFFFFFFFFFF}, 
но он делает это часто.}
\EN{I don't know, why MSVC uses \TT{OR} instead of \TT{MOV RAX, 0xFFFFFFFFFFFFFFFF}, 
but it does this often.}
\RU{И конечно, это эквивалентно друг другу: все биты просто выставляются, а все выставленные
биты это -1 в дополнительном коде (two's complement): \ref{sec:signednumbers}.}
\EN{And of course, it is equivalent: all bits are just set, and all bits set is -1 
in two's complement arithmetics: \ref{sec:signednumbers}.}

\RU{Кто-то мог бы спросить, зачем вообще нужно использовать число -1 в ф-ции \strlen{}?}
\EN{Why would -1 number be used in \strlen{}, one might ask?}
\RU{Вследствии оптимизации, конечно}\EN{Due to optimizations, of course}.
\RU{Вот что сделал MSVC}\EN{Here is the code MSVC did}:

\lstinputlisting[caption=\RU{Вставленная \strlen{} сгенерированная}\EN{Inlined \strlen{} by} MSVC 2013 x64]{patterns/10_strings/4_trim/strlen_MSVC.asm}

\RU{Попробуйте написать короче, если хотите инициализировать счетчик нулем!}
\EN{Try to write shorter if you want to initialize counter at 0!}
\RU{Вот моя попытка}\EN{Here is my attempt}:

\lstinputlisting[caption=\RU{Моя версия}\EN{My version of} \strlen{}]{patterns/10_strings/4_trim/my_strlen.asm}

\RU{У меня не получилось. Нам придется вводить дополнительную инструкцию JMP!}
\EN{I failed. We ought to use additional JMP instruction anyway!}

\RU{Что сделал MSVC 2013, так это передвинул инструкцию \TT{INC} в место перед загрузкой символа.}
\EN{So what MSVC 2013 compiler did is moved \TT{INC} instruction to the place before 
actual character load.}
\RU{Если самый первый символ --- нулевой, всё нормально, \RAX содержит 0 в этот момент, так что
итоговая длина строки будет 0.}
\EN{If the very first character is 0, that's OK, \RAX is 0 at this moment, 
so the resulting string length is 0.}

\RU{Остальную часть ф-ции проще понять.}
\EN{All the rest in the function is seems easy to understand.}
\RU{Еще один трюк в конце}\EN{Another trick is at the end}.
\RU{Если не учитывать вставленный код \strlen{}, здесь только 3 условных перехода в ф-ции.}
\EN{If not to count \strlen{} inlined code, there are only 3 conditional jumps in function.}
\RU{Должно было быть 4: четвертый должен был быть в конце ф-ции, проверяющий, является
ли символ нулевым.}
\EN{There should be 4: 4th is to be located at the function end, checking, 
if the character is zero.}
\RU{Но там безусловный переход на метку}\EN{But there are unconditional jump to the} 
``\$LN18@str\_trim''\RU{, где мы видим \TT{JE}, которая в начале служила
для проверки пустоты входящей строки, прямо после завершения \strlen{}.}\EN{ label, where 
we see \TT{JE}, which was first used to check if the input string is empty, right after \strlen{} 
finish.}
\RU{Так что код использует инструкцию}\EN{So the code uses} \TT{JE} \RU{в этом месте
для двух целей}\EN{instruction at this place for two purposes}!
\RU{Это может быть излишним, но так или иначе, MSVC так сделал.}
\EN{This may be overkill, but nevertheless, MSVC did it.}

\RU{О том, почему так важно обходиться без условных переходов, если это возможно,
читайте здесь:}
\EN{Read more, why it's important to do the job without conditional jumps, 
if possible:} \ref{branch_predictors}.

\subsection{x64: \NonOptimizing GCC 4.9.1}

\lstinputlisting{patterns/10_strings/4_trim/GCC491_x64_O0.asm}

\RU{Я добавил комментариев}\EN{I added my comments}.
\RU{После исполнения \strlen{}, управление передается на метку L2,
и там проверяются два выражения, одно после другого.}
\EN{After \strlen{} execution, control is passed to L2 label, 
and there are two clauses are checked, one after one.}
\RU{Второе никогда не будет проверяться, если первое выражение не истинно (\IT{str\_len==0})
(это ``short-circuit'').}
\EN{Second will never be checked, if first (\IT{str\_len==0}) is false (this is ``short-circuit'').}

\RU{Теперь посмотрим на эту ф-цию в коротком виде:}
\EN{Now let's see this function in short form:}

\begin{itemize}
\item \RU{Первая часть for() (вызов \strlen{})}\EN{First for() part (call to \strlen{})}
\item goto L2
\item L5:
\item \RU{Тело for(). переход на выход, если нужно}\EN{for() body. goto exit, if needed}
\item \RU{Третья часть for() (декремент str\_len)}\EN{for() third part (decrement of str\_len)}
\item L2:
\item \RU{Вторая часть for(): проверить первое выжание, затем второе. 
переход на начало тела цикла, или выход.}
\EN{for() second part: check first clause, then second. goto loop body begin or exit.}
\item L4:
\item // \RU{выход}\EN{exit}
\item return s
\end{itemize}

\subsection{x64: \Optimizing GCC 4.9.1}
\label{string_trim_GCC_x64_O3}

\lstinputlisting{patterns/10_strings/4_trim/GCC491_x64_O3.asm}

\RU{Тут более сложный результат}\EN{Now this is more complex}.
\RU{Код перед циклом исполняется только один раз, но также содержит проверку символов
\CRLF{}!}
\EN{Code before loop body begin executed only once, but it has \CRLF{} 
characters check too!}
\RU{Зачем нужна это дублирование кода}\EN{What this code duplication is for}?

\RU{Обычная реализация главного цикла это, наверное, такая:}
\EN{Common way to implement main loop is probably this:}

\begin{itemize}
\item (\RU{начало цикла}\EN{loop begin}) \RU{проверить символы}\EN{check for} 
\CRLF{}\RU{, принять решения}\EN{ characters, make decisions}
\item \RU{записать нулевой символ}\EN{store zero character}
\end{itemize}

\RU{Но GCC решил поменять местами эти два шага}\EN{But GCC decided to reverse these two steps}. 
\RU{Конечно, шаг \IT{записать нулевой символ} не может быть первым, так что нужна еще одна
проверка:}
\EN{Of course, \IT{store zero character} cannot be first step, so another check is needed:}

\begin{itemize}
\item 
\RU{обработать первый символ. сравнить его с \CRLF{}, выйти если символ не равен \CRLF{}}
\EN{workout first character. match it to \CRLF{}, exit if character is not \CRLF{}}
\item (\RU{начало цикла}\EN{loop begin}) \RU{записать нулевой символ}\EN{store zero character}
\item 
\RU{проверить символы \CRLF{}, принять решения}
\EN{check for \CRLF{} characters, make decisions}
\end{itemize}

\RU{Теперь основной цикл очень короткий, а это очень хорошо для современных процессоров.}
\EN{Now the main loop is very short, which is very good for modern \ac{CPU}s.}

\RU{Код не использует переменную str\_len, но str\_len-1.}
\EN{The code doesn't use str\_len variable, but str\_len-1.}
\RU{Так что это больше похоже на индекс в буфере}\EN{So this is more like index in buffer}.
\RU{Должно быть, GCC заметил что выражение str\_len-1 используется дважды.}
\EN{Apparently, GCC notices that str\_len-1 statement is used twice.}
\RU{Так что будет лучше выделить переменную, которая всегда содержит значение равное 
текущей длине строки минус 1, и уменьшать его на 1 (это тот же эффект, что и уменьшать
переменную str\_len).}
\EN{So it's better to allocate a variable which is always holds a value lesser of 
current string length by one, 
and decrement it (this is the same effect as decrementing str\_len variable).}

\ifdefined\IncludeARM
\subsection{ARM64: \NonOptimizing GCC (Linaro) 4.9}

\RU{Реализация простая и прямолинейная}\EN{This implementation is straightforward}:

\lstinputlisting[caption=\NonOptimizing GCC (Linaro) 4.9]{patterns/10_strings/4_trim/GCC49_ARM64_O0.s}

\subsection{ARM64: \Optimizing GCC (Linaro) 4.9}

\RU{Это более продвинутая оптимизация}\EN{This is more advanced optimization}.
\RU{Первый символ загружается в самом начале и сравнивается с 10 (символ \ac{LF}).}
\EN{First character is loaded at the beginning, and compared against 10 (\ac{LF} character).}
\RU{Символы также загружаются и в главном цикле, для символов после первого.}
\EN{Characters also loaded in the main loop, for characters after first.}
\RU{Это в каком смысле похоже на этот пример: \ref{string_trim_GCC_x64_O3}.}
\EN{This is somewhat similar to \ref{string_trim_GCC_x64_O3} example.}

\lstinputlisting[caption=\Optimizing GCC (Linaro) 4.9]{patterns/10_strings/4_trim/GCC49_ARM64_O3.s}

\subsection{ARM: \OptimizingKeilVI (\ARMMode)}

\RU{И снова, компилятор пользуется условными инструкциями в режиме ARM, поэтому код более 
компактный.}
\EN{And again, compiler took advantage of ARM mode conditional instructions, 
so the code is much more compact.}

\lstinputlisting[caption=\OptimizingKeilVI (\ARMMode)]{patterns/10_strings/4_trim/Keil_ARM_O3.s}

\subsection{ARM: \OptimizingKeilVI (\ThumbMode)}
\index{\CompilerAnomaly}
\label{Keil_anomaly}

\RU{В режиме Thumb куда меньше условных инструкций, так что код более простой.}
\EN{There are less number of conditional instructions in Thumb mode, so the code is more ordinary.}
\RU{Но здесь есть одна странность со сдвигами на 0x20 и 0x19.}
\EN{But there are one really weird thing with 0x20 and 0x19 offsets.}
\RU{Почему компилятор Keil сделал так}\EN{Why Keil compiler did so}?
\RU{Честно говоря, я не знаю}\EN{Honestly, I have no idea}.
\RU{Возможно, это выверт процесса оптимизации компилятора.}
\EN{Probably, this is a quirk of Keil optimization process.}
\RU{Тем не менее, код будет работать корректно}\EN{Nevertheless, the code will work correctly}.

\lstinputlisting[caption=\OptimizingKeilVI (\ThumbMode)]{patterns/10_strings/4_trim/Keil_thumb_O3.s}

\fi
\ifdefined\IncludeMIPS
\subsection{MIPS}

\lstinputlisting[caption=\Optimizing GCC 4.4.5 (IDA)]{patterns/10_strings/4_trim/MIPS_O3_IDA.lst}

\RU{�������� � ��������� S- ���������� ``saved temporaries'', ��� ���, �������� \$S0 ����������� 
� ��������� ����� � ����������������� �� ����� ������.}
\EN{Registers prefixed with S- are also called ``saved temporaries'', 
so \$S0 value is saved in the local stack
and restored upon finish.}

\fi

