\section{\Exercises}

\subsection{\Exercise \#1}
\label{exercise_strlen_1}

\WhatThisCodeDoes\

\begin{lstlisting}[caption=MSVC 2010 /Ox]
_s$ = 8			
_f	PROC
	mov	edx, DWORD PTR _s$[esp-4]
	mov	cl, BYTE PTR [edx]
	xor	eax, eax
	test	cl, cl
	je	SHORT $LN2@f
	npad	4
$LL4@f:
	cmp	cl, 32	
	jne	SHORT $LN3@f
	inc	eax
$LN3@f:
	mov	cl, BYTE PTR [edx+1]
	inc	edx
	test	cl, cl
	jne	SHORT $LL4@f
$LN2@f:
	ret	0
_f	ENDP
\end{lstlisting}

\begin{lstlisting}[caption=GCC 4.8.1 -O3]
f:
.LFB24:
	push	ebx
	mov	ecx, DWORD PTR [esp+8]
	xor	eax, eax
	movzx	edx, BYTE PTR [ecx]
	test	dl, dl
	je	.L2
.L3:
	cmp	dl, 32
	lea	ebx, [eax+1]
	cmove	eax, ebx
	add	ecx, 1
	movzx	edx, BYTE PTR [ecx]
	test	dl, dl
	jne	.L3
.L2:
	pop	ebx
	ret
\end{lstlisting}

\begin{lstlisting}[caption=Keil 5.03 (\ARMMode) -O3]
f PROC
        MOV      r1,#0
|L0.4|
        LDRB     r2,[r0,#0]
        CMP      r2,#0
        MOVEQ    r0,r1
        BXEQ     lr
        CMP      r2,#0x20
        ADDEQ    r1,r1,#1
        ADD      r0,r0,#1
        B        |L0.4|
        ENDP
\end{lstlisting}

\begin{lstlisting}[caption=Keil 5.03 (\ThumbMode) -O3]
f PROC
        MOVS     r1,#0
        B        |L0.12|
|L0.4|
        CMP      r2,#0x20
        BNE      |L0.10|
        ADDS     r1,r1,#1
|L0.10|
        ADDS     r0,r0,#1
|L0.12|
        LDRB     r2,[r0,#0]
        CMP      r2,#0
        BNE      |L0.4|
        MOVS     r0,r1
        BX       lr
        ENDP
\end{lstlisting}

\Answer\: \ref{exercise_solutions_strlen_1}.

\subsection{\Exercise \#2}
\label{exercise_strlen_2}
% toupper()

\index{OpenWatcom}
\RU{Это стандартная функция из библиотек Си. Исходник взят из OpenWatcom}
\EN{This is standard C library function. Source code taken from OpenWatcom}.

\WhatThisCodeDoes\

\lstinputlisting[caption=MSVC 2010 /O3]{patterns/10_strlen/exercises/toupper_msvc.asm}

\lstinputlisting[caption=GCC 4.4.1 -O3]{patterns/10_strlen/exercises/toupper_gcc.asm}

\lstinputlisting[caption=Keil 5.03 (\ARMMode) -O3]{patterns/10_strlen/exercises/toupper_ARM.s}

\lstinputlisting[caption=Keil 5.03 (\ThumbMode) -O3]{patterns/10_strlen/exercises/toupper_thumb.s}

\Answer\: \ref{exercise_solutions_strlen_2}.
