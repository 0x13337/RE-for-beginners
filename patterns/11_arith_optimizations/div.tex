\section{\RU{Деление}\EN{Division}}

\subsection{\RU{Деление используя сдвиги}\EN{Division using shifts}}
\label{division_by_shifting}

\RU{Например}\EN{Example}:

\begin{lstlisting}
unsigned int f(unsigned int a)
{
	return a/4;
};
\end{lstlisting}

\RU{Имеем в итоге}\EN{We get} (MSVC 2010):

\begin{lstlisting}[caption=MSVC 2010]
_a$ = 8							; size = 4
_f	PROC
	mov	eax, DWORD PTR _a$[esp-4]
	shr	eax, 2
	ret	0
_f	ENDP
\end{lstlisting}

\label{SHR}
\index{x86!\Instructions!SHR}
\RU{Инструкция \SHR (\IT{SHift Right}) в данном примере сдвигает число на 2 бита вправо. 
При этом, освободившиеся два бита слева (т.е. самые 
старшие разряды), выставляются в нули. А самые младшие 2 бита выкидываются. 
Фактически, эти два выкинутых бита ~--- остаток от деления.}
\EN{The \SHR (\IT{SHift Right}) instruction in this example is shifting a number by 2 bits to the right.
The two freed bits at left (e.g., two most significant bits) are set to zero.
The two least significant bits are dropped.
In fact, these two dropped bits are the division operation remainder.}

\index{x86!\Instructions!SHR}
\RU{Инструкция \SHR работает так же, как и \SHL, только в другую сторону.}
\EN{The \SHR instruction works just like \SHL, but in the other direction.}

\begin{center}
	\begin{tikzpicture}[scale=0.7, every node/.style={scale=0.7}]
	\edef\bitsize{1cm}
	\tikzstyle{byte}=[draw,minimum size=\bitsize]	
	\tikzstyle{every path}=[thick]

	\node [draw,rectangle,minimum size=\bitsize] (a1) {7};
	\node [draw,rectangle,minimum size=\bitsize] (a2) [right of=a1] {6};
	\node [draw,rectangle,minimum size=\bitsize] (a3) [right of=a2] {5};
	\node [draw,rectangle,minimum size=\bitsize] (a4) [right of=a3] {4};
	\node [draw,rectangle,minimum size=\bitsize] (a5) [right of=a4] {3};
	\node [draw,rectangle,minimum size=\bitsize] (a6) [right of=a5] {2};
	\node [draw,rectangle,minimum size=\bitsize] (a7) [right of=a6] {1};
	\node [draw,rectangle,minimum size=\bitsize] (a8) [right of=a7] {0};

	\node (empty) [below of=a1] {};

	\node [draw,rectangle,minimum size=\bitsize] (b1) [below of=empty] {7};
	\node [draw,rectangle,minimum size=\bitsize] (b2) [right of=b1] {6};
	\node [draw,rectangle,minimum size=\bitsize] (b3) [right of=b2] {5};
	\node [draw,rectangle,minimum size=\bitsize] (b4) [right of=b3] {4};
	\node [draw,rectangle,minimum size=\bitsize] (b5) [right of=b4] {3};
	\node [draw,rectangle,minimum size=\bitsize] (b6) [right of=b5] {2};
	\node [draw,rectangle,minimum size=\bitsize] (b7) [right of=b6] {1};
	\node [draw,rectangle,minimum size=\bitsize] (b8) [right of=b7] {0};
	
	\node [shape=rectangle,draw,minimum size=\bitsize] (c) [left=of b1] {0};
	\node [shape=rectangle,draw,minimum size=\bitsize] (d) [right=of b8] {CF};
	
	\draw [->] (c.east) -- (b1.west);

	\draw [->] (a1.south) -- (b2.north);
	\draw [->] (a2.south) -- (b3.north);
	\draw [->] (a3.south) -- (b4.north);
	\draw [->] (a4.south) -- (b5.north);
	\draw [->] (a5.south) -- (b6.north);
	\draw [->] (a6.south) -- (b7.north);
	\draw [->] (a7.south) -- (b8.north);
	
	\draw [->] (a8.south) -- (d.north);

	\end{tikzpicture}
\end{center}



\RU{Для того, чтобы это проще понять, представьте себе десятичную систему счисления и число $23$. 
$23$ можно разделить на $10$ просто откинув последний разряд ($3$ ~--- это остаток от деления). 
После этой операции останется $2$ как \glslink{quotient}{частное}.}
\EN{It is easy to understand if you imagine the number $23$ in the decimal numeral system.
$23$ can be easily divided by $10$ just by dropping last digit ($3$~---is division remainder). 
$2$ is left after the operation as a \gls{quotient}.}\\
\\
\RU{Так что остаток выбрасывается, но это нормально, мы все-таки работаем с целочисленными
значениями, а это не числа с плавающей точкой!}
\EN{So the remainder is dropped, but that's OK, we work on integer values anyway, 
these are not floating point numbers!}\\
\\
\ifdefined\IncludeARM
\RU{Деление на 4 в}\EN{Division by 4 in} ARM:

\begin{lstlisting}[caption=\NonOptimizingKeilVI (\ARMMode)]
f PROC
        LSR      r0,r0,#2
        BX       lr
        ENDP
\end{lstlisting}
\fi

\ifdefined\IncludeMIPS
\RU{Деление на 4 в}\EN{Division by 4 in} MIPS:

\begin{lstlisting}[caption=\Optimizing GCC 4.4.5 (IDA)]
                 jr      $ra
                 srl     $v0, $a0, 2 ; branch delay slot
\end{lstlisting}

\index{MIPS!\Instructions!SRL}
\RU{Инструкция SRL это}\EN{The SRL instruction is} \q{shift-right logical}.
\fi
