\section{\RU{Умножение}\EN{Multiplication}}

\subsection{\RU{Умножение при помощи сложения}\EN{Multiplication using addition}}

\RU{Вот простой пример}\EN{Here is a simple example}:

\begin{lstlisting}[caption=\Optimizing MSVC 2010]
unsigned int f(unsigned int a)
{
	return a*8;
};
\end{lstlisting}

\RU{Умножение на 8 заменяется на три инструкции сложения, делающих то же самое.}
\EN{Multiplication by 8 is replaced by 3 addition instructions, which do the same.}
\RU{Должно быть, оптимизатор в MSVC решил, что этот код может быть быстрее.}
\EN{Apparently, MSVC's optimizer decided that this code can be faster.}

\begin{lstlisting}
_TEXT	SEGMENT
_a$ = 8							; size = 4
_f	PROC
; File c:\polygon\c\2.c
	mov	eax, DWORD PTR _a$[esp-4]
	add	eax, eax
	add	eax, eax
	add	eax, eax
	ret	0
_f	ENDP
_TEXT	ENDS
END
\end{lstlisting}

\subsection{\RU{Умножение при помощи сдвигов}\EN{Multiplication using shifting}}
\label{subsec:mult_using_shifts}

\RU{Ещё очень часто умножения и деления на числа вида $2^{n}$ заменяются на инструкции сдвигов.}
\EN{Multiplication and division instructions by a numbers that's a power of 2 are often replaced
by shift instructions.}

\begin{lstlisting}
unsigned int f(unsigned int a)
{
	return a*4;
};
\end{lstlisting}

\begin{lstlisting}[caption=\NonOptimizing MSVC 2010]
_a$ = 8		; size = 4
_f	PROC
	push	ebp
	mov	ebp, esp
	mov	eax, DWORD PTR _a$[ebp]
	shl	eax, 2
	pop	ebp
	ret	0
_f	ENDP
\end{lstlisting}

\RU{Умножить на 4 это просто сдвинуть число на 2 бита влево, 
вставив 2 нулевых бита справа (как два самых младших бита). 
Это как умножить 3 на 100~--- нужно просто дописать два нуля справа.}
\EN{Multiplication by 4 is just shifting the number to the left by 2 bits
and inserting 2 zero bits at the right (as the last two bits).
It is just like multiplying 3 by 100~---we need to just add two zeroes at the right.}

\RU{Вот как работает инструкция сдвига влево}\EN{That's how the shift left instruction works}:

\index{x86!\Instructions!SHL}
\begin{center}
	\begin{tikzpicture}[scale=0.7, every node/.style={scale=0.7}]
	\edef\bitsize{1cm}
	\tikzstyle{byte}=[draw,minimum size=\bitsize]	
	\tikzstyle{every path}=[thick]

	\node [draw,rectangle,minimum size=\bitsize] (a1) {7};
	\node [draw,rectangle,minimum size=\bitsize] (a2) [right of=a1] {6};
	\node [draw,rectangle,minimum size=\bitsize] (a3) [right of=a2] {5};
	\node [draw,rectangle,minimum size=\bitsize] (a4) [right of=a3] {4};
	\node [draw,rectangle,minimum size=\bitsize] (a5) [right of=a4] {3};
	\node [draw,rectangle,minimum size=\bitsize] (a6) [right of=a5] {2};
	\node [draw,rectangle,minimum size=\bitsize] (a7) [right of=a6] {1};
	\node [draw,rectangle,minimum size=\bitsize] (a8) [right of=a7] {0};

	\node (empty) [below of=a1] {};

	\node [draw,rectangle,minimum size=\bitsize] (b1) [below of=empty] {7};
	\node [draw,rectangle,minimum size=\bitsize] (b2) [right of=b1] {6};
	\node [draw,rectangle,minimum size=\bitsize] (b3) [right of=b2] {5};
	\node [draw,rectangle,minimum size=\bitsize] (b4) [right of=b3] {4};
	\node [draw,rectangle,minimum size=\bitsize] (b5) [right of=b4] {3};
	\node [draw,rectangle,minimum size=\bitsize] (b6) [right of=b5] {2};
	\node [draw,rectangle,minimum size=\bitsize] (b7) [right of=b6] {1};
	\node [draw,rectangle,minimum size=\bitsize] (b8) [right of=b7] {0};
	
	\node [shape=rectangle,draw,minimum size=\bitsize] (d) [left=of b1] {nowhere};
	\node [shape=rectangle,draw,minimum size=\bitsize] (c) [right=of b8] {0};
	
	\draw [->] (c.west) -- (b8.east);

	\draw [->] (a2.south) -- (b1.north);
	\draw [->] (a3.south) -- (b2.north);
	\draw [->] (a4.south) -- (b3.north);
	\draw [->] (a5.south) -- (b4.north);
	\draw [->] (a6.south) -- (b5.north);
	\draw [->] (a7.south) -- (b6.north);
	\draw [->] (a8.south) -- (b7.north);
	
	\draw [->] (a1.south) -- (d.north);

	\end{tikzpicture}
\end{center}


\RU{Добавленные биты справа~--- всегда нули}\EN{The added bits at right are always zeroes}.

\ifdefined\IncludeARM
\RU{Умножение на 4 в}\EN{Multiplication by 4 in} ARM:

\begin{lstlisting}[caption=\NonOptimizingKeilVI (\ARMMode)]
f PROC
        LSL      r0,r0,#2
        BX       lr
        ENDP
\end{lstlisting}
\fi

\ifdefined\IncludeMIPS
\RU{Умножение на 4 в}\EN{Multiplication by 4 in} MIPS:

\lstinputlisting[caption=\Optimizing GCC 4.4.5 (IDA)]{patterns/11_arith_optimizations/MIPS_SLL.lst}

\index{MIPS!\Instructions!SLL}
SLL \RU{это}\EN{is} \q{Shift Left Logical}.
\fi

\subsection{\RU{Умножение при помощи сдвигов, сложений и вычитаний}
\EN{Multiplication using shifting, subtracting, and adding}}
\label{multiplication_using_shifts_adds_subs}

\RU{Можно избавиться от операции умножения, если вы умножаете на числа вроде 7 или 17,
и использовать сдвиги.}
\EN{It's still possible to get rid of the multiplication operation when you multiply by numbers like
7 or 17 again by using shifting.}
\RU{Здесь используется относительно простая математика}\EN{The mathematics used here is relatively easy}.

\subsubsection{32-\EN{bit}\RU{бита}}

\lstinputlisting{patterns/11_arith_optimizations/mult_shifts.c}

\myparagraph{x86}

\lstinputlisting[caption=\Optimizing MSVC 2012]{patterns/11_arith_optimizations/mult_shifts_MSVC_2012_Ox.asm}

\ifdefined\IncludeARM
\myparagraph{ARM}

\RU{Keil, генерируя код для режима ARM, использует модификаторы инструкции, в которых можно задавать
сдвиг для второго операнда:}
\EN{Keil for ARM mode takes advantage of the second operand's shift modifiers:}

\lstinputlisting[caption=\OptimizingKeilVI (\ARMMode)]{patterns/11_arith_optimizations/mult_shifts_Keil_ARM_O3.s}

\RU{Но таких модификаторов в режиме Thumb нет.}
\EN{But there are no such modifiers in Thumb mode.}
\RU{И он также не смог оптимизировать функцию \TT{f2()}}\EN{It also can't optimize \TT{f2()}}:

\lstinputlisting[caption=\OptimizingKeilVI (\ThumbMode)]{patterns/11_arith_optimizations/mult_shifts_Keil_thumb_O3.s}
\fi

\ifdefined\IncludeMIPS
\myparagraph{MIPS}

\lstinputlisting[caption=\Optimizing GCC 4.4.5 (IDA)]{patterns/11_arith_optimizations/mult_shifts_MIPS_O3_IDA.lst}
\fi

\subsubsection{64-\EN{bit}\RU{бита}}

\lstinputlisting{patterns/11_arith_optimizations/mult_shifts_64.c}

\myparagraph{x64}

\lstinputlisting[caption=\Optimizing MSVC 2012]{patterns/11_arith_optimizations/mult_shifts_64_GCC49_x64_O3.s}

\ifdefined\IncludeARM
\myparagraph{ARM64}

\ifdefined\IncludeGCC
\RU{GCC 4.9 для ARM64 также очень лаконичен благодаря модификаторам сдвига:}
\EN{GCC 4.9 for ARM64 is also terse, thanks to the shift modifiers:}

\lstinputlisting[caption=\Optimizing GCC (Linaro) 4.9 ARM64]{patterns/11_arith_optimizations/mult_shifts_64_GCC49_ARM64.s}
\fi
\fi
