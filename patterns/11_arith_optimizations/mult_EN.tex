\subsection{Multiplication}

\subsubsection{Multiplication using addition}

Here is a simple example:

\begin{lstlisting}[caption=\Optimizing MSVC 2010]
unsigned int f(unsigned int a)
{
	return a*8;
};
\end{lstlisting}

Multiplication by 8 is replaced by 3 addition instructions, which do the same.
Apparently, MSVC's optimizer decided that this code can be faster.

\begin{lstlisting}
_TEXT	SEGMENT
_a$ = 8		; size = 4
_f	PROC
; File c:\polygon\c\2.c
	mov	eax, DWORD PTR _a$[esp-4]
	add	eax, eax
	add	eax, eax
	add	eax, eax
	ret	0
_f	ENDP
_TEXT	ENDS
END
\end{lstlisting}

\subsubsection{Multiplication using shifting}
\label{subsec:mult_using_shifts}

Multiplication and division instructions by a numbers that's a power of 2 are often replaced by shift instructions.

\begin{lstlisting}
unsigned int f(unsigned int a)
{
	return a*4;
};
\end{lstlisting}

\begin{lstlisting}[caption=\NonOptimizing MSVC 2010]
_a$ = 8		; size = 4
_f	PROC
	push	ebp
	mov	ebp, esp
	mov	eax, DWORD PTR _a$[ebp]
	shl	eax, 2
	pop	ebp
	ret	0
_f	ENDP
\end{lstlisting}


Multiplication by 4 is just shifting the number to the left by 2 bits
and inserting 2 zero bits at the right (as the last two bits).
It is just like multiplying 3 by 100~---we need to just add two zeroes at the right.

That's how the shift left instruction works:

\myindex{x86!\Instructions!SHL}
\begin{center}
	\begin{tikzpicture}[scale=0.7, every node/.style={scale=0.7}]
	\edef\bitsize{1cm}
	\tikzstyle{byte}=[draw,minimum size=\bitsize]	
	\tikzstyle{every path}=[thick]

	\node [draw,rectangle,minimum size=\bitsize] (a1) {7};
	\node [draw,rectangle,minimum size=\bitsize] (a2) [right of=a1] {6};
	\node [draw,rectangle,minimum size=\bitsize] (a3) [right of=a2] {5};
	\node [draw,rectangle,minimum size=\bitsize] (a4) [right of=a3] {4};
	\node [draw,rectangle,minimum size=\bitsize] (a5) [right of=a4] {3};
	\node [draw,rectangle,minimum size=\bitsize] (a6) [right of=a5] {2};
	\node [draw,rectangle,minimum size=\bitsize] (a7) [right of=a6] {1};
	\node [draw,rectangle,minimum size=\bitsize] (a8) [right of=a7] {0};

	\node (empty) [below of=a1] {};

	\node [draw,rectangle,minimum size=\bitsize] (b1) [below of=empty] {7};
	\node [draw,rectangle,minimum size=\bitsize] (b2) [right of=b1] {6};
	\node [draw,rectangle,minimum size=\bitsize] (b3) [right of=b2] {5};
	\node [draw,rectangle,minimum size=\bitsize] (b4) [right of=b3] {4};
	\node [draw,rectangle,minimum size=\bitsize] (b5) [right of=b4] {3};
	\node [draw,rectangle,minimum size=\bitsize] (b6) [right of=b5] {2};
	\node [draw,rectangle,minimum size=\bitsize] (b7) [right of=b6] {1};
	\node [draw,rectangle,minimum size=\bitsize] (b8) [right of=b7] {0};
	
	\node [shape=rectangle,draw,minimum size=\bitsize] (d) [left=of b1] {nowhere};
	\node [shape=rectangle,draw,minimum size=\bitsize] (c) [right=of b8] {0};
	
	\draw [->] (c.west) -- (b8.east);

	\draw [->] (a2.south) -- (b1.north);
	\draw [->] (a3.south) -- (b2.north);
	\draw [->] (a4.south) -- (b3.north);
	\draw [->] (a5.south) -- (b4.north);
	\draw [->] (a6.south) -- (b5.north);
	\draw [->] (a7.south) -- (b6.north);
	\draw [->] (a8.south) -- (b7.north);
	
	\draw [->] (a1.south) -- (d.north);

	\end{tikzpicture}
\end{center}


The added bits at right are always zeroes.

Multiplication by 4 in ARM:

\begin{lstlisting}[caption=\NonOptimizingKeilVI (\ARMMode)]
f PROC
        LSL      r0,r0,#2
        BX       lr
        ENDP
\end{lstlisting}

Multiplication by 4 in MIPS:

\lstinputlisting[caption=\Optimizing GCC 4.4.5 (IDA)]{patterns/11_arith_optimizations/MIPS_SLL.lst}

\myindex{MIPS!\Instructions!SLL}
\INS{SLL} is \q{Shift Left Logical}.

\subsubsection{Multiplication using shifting, subtracting, and adding}
\label{multiplication_using_shifts_adds_subs}

It's still possible to get rid of the multiplication operation when you multiply by numbers like
7 or 17 again by using shifting.
The mathematics used here is relatively easy.

\myparagraph{32-bit}

\lstinputlisting{patterns/11_arith_optimizations/mult_shifts.c}

\mysubparagraph{x86}

\lstinputlisting[caption=\Optimizing MSVC 2012]{patterns/11_arith_optimizations/mult_shifts_MSVC_2012_Ox.asm}

\mysubparagraph{ARM}

Keil for ARM mode takes advantage of the second operand's shift modifiers:

\lstinputlisting[caption=\OptimizingKeilVI (\ARMMode)]{patterns/11_arith_optimizations/mult_shifts_Keil_ARM_O3.s}

But there are no such modifiers in Thumb mode.
It also can't optimize \TT{f2()}:

\lstinputlisting[caption=\OptimizingKeilVI (\ThumbMode)]{patterns/11_arith_optimizations/mult_shifts_Keil_thumb_O3.s}

\mysubparagraph{MIPS}

\lstinputlisting[caption=\Optimizing GCC 4.4.5 (IDA)]{patterns/11_arith_optimizations/mult_shifts_MIPS_O3_IDA.lst}

\myparagraph{64-bit}

\lstinputlisting{patterns/11_arith_optimizations/mult_shifts_64.c}

\mysubparagraph{x64}

\lstinputlisting[caption=\Optimizing MSVC 2012]{patterns/11_arith_optimizations/mult_shifts_64_GCC49_x64_O3.s}

\mysubparagraph{ARM64}

GCC 4.9 for ARM64 is also terse, thanks to the shift modifiers:

\lstinputlisting[caption=\Optimizing GCC (Linaro) 4.9 ARM64]{patterns/11_arith_optimizations/mult_shifts_64_GCC49_ARM64.s}

\myparagraph{Booth's multiplication algorithm}

\myindex{Data general Nova}
\myindex{Booth's multiplication algorithm}
There was a time when computers were big and that expensive, that some of them lacks hardware support of multiplication
operation in \ac{CPU}, like Data General Nova.
And when one need multiplication operation, it can be provided at software level, for example, using Booth's multiplication
algorithm.
This is an multiplication algorithm which uses only addition operation and shifts.

What modern optimizing compilers do, isn't the same,
but the goal (multiplication) and resources (faster operations) are the same.

