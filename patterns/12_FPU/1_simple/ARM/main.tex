\subsection{ARM: \OptimizingXcodeIV (\ARMMode)}

\RU{Пока в ARM не было стандартного набора инструкций для работы с плавающей точкой}
\EN{Until ARM got standardized floating point support}, \RU{разные производители процессоров
могли добавлять свои расширения для работы с ними}\EN{several processor manufacturers added their own 
instructions extensions}.
\RU{Позже, был принят стандарт}\EN{Then, } VFP (\IT{Vector Floating Point})\EN{ was standardized}.

\RU{Важное отличие от x86 в том, что там вы работаете с FPU-стеком, а здесь стека нет, 
здесь вы работаете просто с регистрами.}
\EN{One important difference from x86 is that in ARM, there
is no stack, you work just with registers.}

\lstinputlisting[label=ARM_leaf_example10,caption=\OptimizingXcodeIV (\ARMMode)]{patterns/12_FPU/1_simple/ARM/Xcode_ARM_O3.asm.\LANG}

\index{ARM!D-\registers{}}
\index{ARM!S-\registers{}}
\RU{Итак, здесь мы видим использование новых регистров, с префиксом D.}
\EN{So, we see here new some registers used, with D prefix.}
\RU{Это 64-битные регистры, их 32, и их можно
использовать и для чисел с плавающей точкой двойной точности (double) и для 
SIMD (в ARM это называется NEON).}
\EN{These are 64-bit registers, there are 32 of them, and they can be used both for floating-point numbers 
(double) but also for SIMD (it is called NEON here in ARM).}
\RU{Имеются также 32 32-битных S-регистра, они применяются для работы с числами 
с плавающей точкой одинарной точности (float).}
\EN{There are also 32 32-bit S-registers, intended to be used for single precision 
floating pointer numbers (float).}
\RU{Запомнить легко: D-регистры предназначены для чисел double-точности, 
а S-регистры ~--- для чисел single-точности.}
\EN{It is easy to remember: D-registers are for double precision numbers, while
S-registers~---for single precision numbers.}
\RU{Больше об этом}\EN{More about it}: \ref{ARM_VFP_registers}.

\RU{Обе константы}\EN{Both} ($3.14$ \AndENRU $4.1$) \RU{хранятся в памяти в формате IEEE 754.}
\EN{constants are stored in memory in IEEE 754 form.}

\index{ARM!\Instructions!VLDR}
\index{ARM!\Instructions!VMOV}
\RU{Инструкции }\TT{VLDR} \AndENRU \TT{VMOV}
\RU{, как можно догадаться, это аналоги обычных \TT{LDR} и \MOV, но они работают с D-регистрами.}
\EN{, as it can be easily deduced, are analogous to the \TT{LDR} and \MOV instructions,
but they work with D-registers.}
\RU{Важно отметить, что эти инструкции, как и D-регистры, предназначены не только для работы 
с числами с плавающей точкой, но пригодны также и для работы с SIMD (NEON), и позже это также будет видно.}
\EN{It should be noted that these instructions, just like the D-registers, are intended not only for
floating point numbers, but can be also used for SIMD (NEON) operations and this will also be shown soon.}

\RU{Аргументы передаются в функцию обычным путем, через R-регистры, однако, 
каждое число, имеющее двойную точность, занимает 64 бита, так что для передачи каждого нужны два R-регистра.}
\EN{The arguments are passed to the function in a common way, via the R-registers, however
each number that has double precision has a size of 64 bits, so two R-registers are needed to pass each one.}

\TT{``VMOV D17, R0, R1''} \RU{в самом начале составляет два 32-битных значения из \Reg{0} и \Reg{1} 
в одно 64-битное и сохраняет в}
\EN{at the start, composes two 32-bit values from \Reg{0} and \Reg{1} into one 64-bit value
and saves it to} \TT{D17}.

\TT{``VMOV R0, R1, D16''} \RU{в конце это обратная процедура}\EN{is the inverse operation}, 
\RU{то что было в}\EN{what was in} \TT{D16} 
\RU{остается в двух регистрах}\EN{is split in two registers,} \Reg{0} \AndENRU \Reg{1},
\RU{потому что,}\EN{becase} \RU{число с двойной точностью,}\EN{a double-precision number} 
\RU{занимающее 64 бита}\EN{that needs 64 bits for storage}, \RU{возвращается в паре регистров \Reg{0} и \Reg{1}}
\EN{is returned in \Reg{0} and \Reg{1}}.

\index{ARM!\Instructions!VDIV}
\index{ARM!\Instructions!VMUL}
\index{ARM!\Instructions!VADD}
\TT{VDIV}, \TT{VMUL} \AndENRU \TT{VADD}, \RU{это, собственно, инструкции для работы с числами 
с плавающей точкой, вычисляющие, соответственно, \glslink{quotient}{частное}, \glslink{product}{произведение} и сумму.}
\EN{are instruction for processing floating point numbers that compute \gls{quotient}, 
\gls{product} and sum, respectively.}

\RU{Код для thumb-2 такой же.}\EN{The code for thumb-2 is same.}

\subsection{ARM: \OptimizingKeilVI (\ThumbMode)}

\lstinputlisting{patterns/12_FPU/1_simple/ARM/Keil_O3_thumb.asm.\LANG}

\RU{Keil компилировал для процессора, в котором может и не быть поддержки FPU или NEON.}
\EN{Keil generated code for a processor without FPU or NEON support.}
\RU{Так что числа с двойной точностью передаются в парах обычных R-регистров}
\EN{The double-precision floating-point numbers are passed via generic R-registers},
\RU{а вместо FPU-инструкций вызываются сервисные библиотечные функции}
\EN{and instead of FPU-instructions, service library functions are called (like}
\TT{\_\_aeabi\_dmul}, \TT{\_\_aeabi\_ddiv}, \TT{\_\_aeabi\_dadd}
\RU{, эмулирующие умножение, деление и сложение чисел с плавающей точкой.}
\EN{) which emulate multiplication, division and addition for floating-point numbers.}
\RU{Конечно, это медленнее чем FPU-сопроцессор, но лучше, чем ничего.}
\EN{Of course, that is slower than FPU-coprocessor, but still better than nothing.}

\RU{Кстати, похожие библиотеки для эмуляции сопроцессорных инструкций были очень распространены в x86, 
когда сопроцессор был редким и дорогим, и стоял далеко не на всех компьютерах.}
\EN{By the way, similar FPU-emulating libraries were very popular in the x86 world when coprocessors were rare
and expensive, and were installed only on expensive computers.}

\index{ARM!soft float}
\index{ARM!armel}
\index{ARM!armhf}
\index{ARM!hard float}
\RU{Эмуляция FPU-сопроцессора в ARM называется \IT{soft float} или \IT{armel}, 
а использование FPU-инструкций сопроцессора ~--- \IT{hard float} или \IT{armhf}.}
\EN{The FPU-coprocessor emulation is called \IT{soft float} or \IT{armel} in the ARM world, 
while using the coprocessor's FPU-instructions is called \IT{hard float} or \IT{armhf}.}

\iffalse
% TODO разобраться...
\index{Raspberry Pi}
\RU{Ядро Linux, например, для Raspberry Pi может поставляться в двух вариантах.}
\EN{For example, the Linux kernel for Raspberry Pi is compiled in two variants.}
\RU{В случае \IT{soft float}, аргументы будут передаваться через R-регистры, 
а в случае \IT{hard float}, через D-регистры.}
\EN{In the \IT{soft float} case, arguments are passed via R-registers, and in the \IT{hard float} 
case~---via D-registers.}

\RU{И это то, что помешает использовать, например, armhf-библиотеки
из armel-кода или наоборот, поэтому, весь код в дистрибутиве Linux должен быть скомпилирован
в соответствии с выбранным соглашением о вызовах.}
\EN{And that is what stops you from using armhf-libraries from armel-code or vice versa,
and that is
why all the code in Linux distributions must be compiled according to a single convention.}
\fi

\subsection{ARM64: \Optimizing GCC (Linaro) 4.9}

\RU{Очень компактный код}\EN{Very compact code}:

\lstinputlisting[caption=\Optimizing GCC (Linaro) 4.9]{patterns/12_FPU/1_simple/ARM/ARM64_GCC_O3.s.\LANG}

\subsection{ARM64: \NonOptimizing GCC (Linaro) 4.9}

\lstinputlisting[caption=\NonOptimizing GCC (Linaro) 4.9]{patterns/12_FPU/1_simple/ARM/ARM64_GCC_O0.s.\LANG}

\NonOptimizing GCC \RU{более многословный}\EN{is more verbose}.
\RU{Здесь много ненужных перетасовок значений, включая явно избыточный код 
(последние две инструкции \TT{GMOV}).}
\EN{There is a lot of unnecessary value shuffling, including some clearly redundant code 
(the last two \TT{FMOV} instructions).}
\RU{Должно быть}\EN{Probably}, GCC 4.9 \RU{пока еще не очень хорош для генерации кода под ARM64}\EN{is not 
yet good in generating ARM64 code}.
\RU{Вот что интересно заметить, это то что у ARM64 64-битные регистры, и D-регистры так же 64-битные.}
\EN{What is worth noting is that ARM64 has 64-bit registers, and the D-registers are 64-bit ones as well.}
\RU{Так что компилятор может сохранять значения типа \Tdouble в \ac{GPR} вместо локального стека.}
\EN{So the compiler is free to save values of type \Tdouble in \ac{GPR}s instead of the local stack.}
\RU{Это не было возможно на 32-битных CPU}\EN{This isn't possible on 32-bit CPUs}.

\RU{И снова, как упражнение, вы можете попробовать соптимизировать эту ф-цию вручную, без добавления
новых инструкций вроде \TT{FMADD}.}
\EN{And again, as an exercise, you can try to optimize this function manually, without introducing
new instructions like \TT{FMADD}.}
