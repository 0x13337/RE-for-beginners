\subsubsection{GCC}

\RU{GCC 4.4.1 (с опцией \Othree) генерирует похожий код, хотя и с некоторой разницей:}
\EN{GCC 4.4.1 (with \Othree option) emits the same code, however, slightly different:}

\lstinputlisting[caption=\Optimizing GCC 4.4.1]{patterns/12_FPU/1_simple/GCC_\LANG.asm}

\RU{Разница в том, что в стек сначала заталкивается $3.14$ (в \ST{0}), а затем значение 
из \TT{arg\_0} делится на то что лежит в регистре \ST{0}.}
\EN{The difference is that, first of all, $3.14$ is pushed to stack (into \ST{0}), and then value 
in \TT{arg\_0} is divided by value in the \ST{0} register.}

\index{x86!\Instructions!FDIVR}
\RU{\FDIVR означает \IT{Reverse Divide} ~--- делить поменяв делитель и делимое местами. 
Точно такой же инструкции для умножения нет, потому что она была бы бессмыслена (ведь умножение ~--- 
операция коммутативная), так что остается только \FMUL без соответствующей ей \TT{-R} инструкции.}
\EN{\FDIVR meaning \IT{Reverse Divide}~---to divide with divisor and dividend swapped with each other. 
There is no likewise instruction for multiplication since multiplication is 
commutative operation, so we have just \FMUL without its \TT{-R} counterpart.}

\index{x86!\Instructions!FADDP}
\RU{\FADDP не только складывает два значения, но также и выталкивает из стека одно значение. 
После этого, в \ST{0} остается только результат сложения.}
\EN{\FADDP adding two values but also popping one value from stack. 
After that operation, \ST{0} holds the sum.}

\RU{Этот фрагмент кода получен при помощи \IDA, которая регистр \ST{0} называет для краткости просто \TT{ST}.}
\EN{This fragment of disassembled code was produced using \IDA which named the \ST{0} register as \TT{ST} for short.}

