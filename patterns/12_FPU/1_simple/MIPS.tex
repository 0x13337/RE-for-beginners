\subsection{MIPS}

\RU{MIPS может поддерживать несколько сопроцессоров (вплоть до 4), нулевой из которых это специальный
управляющий сопроцессор, а первый~--- это FPU.}
\EN{MIPS can support several coprocessors (up to 4), 
the zeroth of which is a special control coprocessor,
and first coprocessor is the FPU.}

\RU{Как и в ARM, сопроцессор в MIPS это не стековая машина. Он имеет 32 32-битных регистра (\$F0-\$F31):}
\EN{As in ARM, the MIPS coprocessor is not a stack machine, it has 32 32-bit registers (\$F0-\$F31):}
\myref{MIPS_FPU_registers}.
\RU{Когда нужно работать с 64-битными значениями типа \Tdouble, используется пара 32-битных F-регистров.}
\EN{When one needs to work with 64-bit \Tdouble values, a pair of 32-bit F-registers is used.}

\lstinputlisting[caption=\Optimizing GCC 4.4.5 (IDA)]{patterns/12_FPU/1_simple/MIPS_O3_IDA.lst.\LANG}

\RU{Новые инструкции}\EN{The new instructions here are}:

\begin{itemize}

\index{MIPS!\Instructions!LWC1}
\item LWC1 \RU{загружает 32-битное слово в регистр первого сопроцессора (отсюда \q{1} в названии инструкции).}
\EN{loads a 32-bit word into a register of the first coprocessor (hence \q{1} in instruction name).}
\index{MIPS!\Pseudoinstructions!L.D}
\RU{Пара инструкций LWC1 может быть объединена в одну псевдоинструкцию L.D.}
\EN{A pair of LWC1 instructions may be combined into a L.D pseudoinstruction.}

\index{MIPS!\Instructions!DIV.D}
\index{MIPS!\Instructions!MUL.D}
\index{MIPS!\Instructions!ADD.D}
\item DIV.D, MUL.D, ADD.D \RU{производят деление, умножение и сложение соответственно}\EN{do division, multiplication, and addition respectively} 
(\q{.D} \RU{в суффиксе означает двойную точность}\EN{in the suffix stands for double precision}, 
\q{.S}\RU{~--- одинарную точность}\EN{ stands for single precision})

\end{itemize}

\index{MIPS!\Instructions!LUI}
\index{\CompilerAnomaly}
\label{MIPS_FPU_LUI}
\RU{Здесь также имеется странная аномалия компилятора: инструкция LUI помеченная мною вопросительным знаком.}
\EN{There is also a weird compiler anomaly: the LUI instructions that I've marked with a question mark.}
\RU{Мне трудно понять, зачем загружать часть 64-битной константы типа \Tdouble в регистр \$V0.}
\EN{It's hard for me to understand why load a part of a 64-bit constant of \Tdouble type into the \$V0 register.}
\RU{От этих инструкций нет толка}\EN{These instruction have no effect}.
% TODO did you try checking out compiler source code?
\RU{Если кто-то об этом что-то знает, пожалуйста, напишите мне емейл}%
\EN{If someone knows more about it, please drop me an email}\footnote{\EMAIL}.
