\subsectionold{MIPS}

MIPS can support several coprocessors (up to 4), 
the zeroth of which is a special control coprocessor,
and first coprocessor is the FPU.

As in ARM, the MIPS coprocessor is not a stack machine, it has 32 32-bit registers (\$F0-\$F31):
\myref{MIPS_FPU_registers}.

When one needs to work with 64-bit \Tdouble values, a pair of 32-bit F-registers is used.

\lstinputlisting[caption=\Optimizing GCC 4.4.5 (IDA)]{patterns/12_FPU/1_simple/MIPS_O3_IDA_EN.lst}

The new instructions here are:

\myindex{MIPS!\Instructions!LWC1}
\myindex{MIPS!\Instructions!DIV.D}
\myindex{MIPS!\Instructions!MUL.D}
\myindex{MIPS!\Instructions!ADD.D}
\begin{itemize}

\item \INS{LWC1} loads a 32-bit word into a register of the first coprocessor (hence \q{1} in instruction name).
\myindex{MIPS!\Pseudoinstructions!L.D}

A pair of \INS{LWC1} instructions may be combined into a \INS{L.D} pseudoinstruction.

\item \INS{DIV.D}, \INS{MUL.D}, \INS{ADD.D} do division, multiplication, and addition respectively 
(\q{.D} in the suffix stands for double precision, \q{.S} stands for single precision)

\end{itemize}

\myindex{MIPS!\Instructions!LUI}
\myindex{\CompilerAnomaly}
\label{MIPS_FPU_LUI}

There is also a weird compiler anomaly: the \INS{LUI} instructions that we've marked with a question mark.
It's hard for me to understand why load a part of a 64-bit constant of \Tdouble type into the \$V0 register.
These instructions has no effect.
% TODO did you try checking out compiler source code?
If someone knows more about it, please drop an email to author\footnote{\EMAIL}.

