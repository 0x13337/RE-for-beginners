\subsection{ARM + \NonOptimizingXcode + \ThumbTwoMode}
\label{FPU_passing_floats_ARM}

\lstinputlisting{patterns/12_FPU/2_passing_floats/Xcode_thumb_O0.asm}

\RU{Как я уже писал, 64-битные числа с плавающей точкой передаются в парах R-регистров.}
\EN{As I wrote before, 64-bit floating pointer numbers passing in R-registers pairs.}
\RU{Этот код слегка избыточен (наверное, потому что не включена оптимизация), ведь, можно было бы 
загружать значения напрямую в R-регистры минуя загрузку в D-регистры.}
\EN{This is code is redundant for a little (certainly because optimization is turned off), because,
it is actually possible to load values into R-registers straightforwardly without touching D-registers.}

\RU{Итак, видно, что функция}\EN{So, as we see,} \TT{\_pow} \RU{получает первый аргумент в}
\EN{function receiving first argument in} \Reg{0} \AndENRU \Reg{1}, \RU{а второй в}\EN{and the second one in} 
\Reg{2} \AndENRU \Reg{3}. 
\RU{Функция оставляет результат в}\EN{Function leaves result in} \Reg{0} \AndENRU \Reg{1}.
\RU{Результат работы}\EN{Result of} \TT{\_pow} \RU{перекладывается в}\EN{is moved into} \TT{D16}, 
\RU{затем в пару}\EN{then in} \Reg{1} \AndENRU \Reg{2}\EN{ pair}, \RU{откуда}\EN{from where} 
\printf \RU{будет читать это число}\EN{will take this number}.

\subsection{ARM + \NonOptimizingKeil + \ARMMode}

\lstinputlisting{patterns/12_FPU/2_passing_floats/Keil_ARM_O0.asm}

\RU{Здесь не используются D-регистры, используются только пары R-регистров.}
\EN{D-registers are not used here, only R-register pairs are used.}

