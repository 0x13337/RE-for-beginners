\subsection{ARM + \NonOptimizingXcodeIV (\ThumbTwoMode)}
\label{FPU_passing_floats_ARM}

\lstinputlisting{patterns/12_FPU/2_passing_floats/Xcode_thumb_O0.asm}

\RU{Как уже было указано, 64-битные числа с плавающей точкой передаются в парах R-регистров.}%
\EN{As it was mentioned before, 64-bit floating pointer numbers are passed in R-registers pairs.}
\RU{Этот код слегка избыточен (наверное, потому что не включена оптимизация), ведь можно было бы 
загружать значения напрямую в R-регистры минуя загрузку в D-регистры.}
\EN{This code is a bit redundant (certainly because optimization is turned off), 
since it is possible to load values into the R-registers directly without touching the D-registers.}

\RU{Итак, видно, что функция}\EN{So, as we see, the} \TT{\_pow} \RU{получает первый аргумент в}
\EN{function receives its first argument in} \Reg{0} \AndENRU \Reg{1}, \RU{а второй в}\EN{and its second one in} 
\Reg{2} \AndENRU \Reg{3}. 
\RU{Функция оставляет результат в}\EN{The function leaves its result in} \Reg{0} \AndENRU \Reg{1}.
\RU{Результат работы}\EN{The result of} \TT{\_pow} \RU{перекладывается в}\EN{is moved into} \TT{D16}, 
\RU{затем в пару}\EN{then in the} \Reg{1} \AndENRU \Reg{2}\EN{ pair}, \RU{откуда}\EN{from where} 
\printf \RU{берет это число-результат.}
\EN{takes the resulting number.}

\subsection{ARM + \NonOptimizingKeilVI (\ARMMode)}

\lstinputlisting{patterns/12_FPU/2_passing_floats/Keil_ARM_O0.asm}

\RU{Здесь не используются D-регистры, используются только пары R-регистров.}
\EN{D-registers are not used here, just R-register pairs.}

\subsection{ARM64 + \Optimizing GCC (Linaro) 4.9}

\lstinputlisting[caption=\Optimizing GCC (Linaro) 4.9]{patterns/12_FPU/2_passing_floats/ARM64.s.\LANG}

\RU{Константы загружаются в}\EN{The constants are loaded into} \RegD{0} \AndENRU \RegD{1}: 
\RU{функция }pow() \RU{берет их оттуда}\EN{takes them from there}.
\RU{Результат в}\EN{The result will be in} \RegD{0} \RU{после исполнения}\EN{after the execution of} pow().
\RU{Он пропускается в}\EN{It is to be passed to} \printf \RU{без всякой модификации и перемещений}\EN{without 
any modification and moving}, 
\RU{потому что}\EN{because} \printf \RU{берет аргументы \glslink{integral type}{интегральных типов} и указатели 
из X-регистров,
а аргументы типа плавающей точки из D-регистров}\EN{takes arguments of \glslink{integral type}{integral types} 
and pointers from X-registers, and floating point arguments from D-registers}.
