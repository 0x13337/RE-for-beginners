\subsection{ARM + \NonOptimizingXcodeIV (\ThumbTwoMode)}
\label{FPU_passing_floats_ARM}

\lstinputlisting{patterns/12_FPU/2_passing_floats/Xcode_thumb_O0.asm}

Как уже было указано, 64-битные числа с плавающей точкой передаются в парах R-регистров.

Этот код слегка избыточен (наверное, потому что не включена оптимизация), ведь можно было бы 
загружать значения напрямую в R-регистры минуя загрузку в D-регистры.

Итак, видно, что функция \GTT{\_pow} получает первый аргумент в \Reg{0} и \Reg{1}, а второй в \Reg{2} и \Reg{3}. 
Функция оставляет результат в \Reg{0} и \Reg{1}.
Результат работы \GTT{\_pow} перекладывается в \GTT{D16}, 
затем в пару \Reg{1} и \Reg{2}, откуда 
\printf берет это число-результат.

\subsection{ARM + \NonOptimizingKeilVI (\ARMMode)}

\lstinputlisting{patterns/12_FPU/2_passing_floats/Keil_ARM_O0.asm}

Здесь не используются D-регистры, используются только пары R-регистров.

\subsection{ARM64 + \Optimizing GCC (Linaro) 4.9}

\lstinputlisting[caption=\Optimizing GCC (Linaro) 4.9]{patterns/12_FPU/2_passing_floats/ARM64_RU.s}

Константы загружаются в \RegD{0} и \RegD{1}: 
функция \TT{pow()} берет их оттуда.
Результат в \RegD{0} после исполнения \TT{pow()}.
Он пропускается в \printf без всякой модификации и перемещений, 
потому что \printf берет аргументы \glslink{integral type}{интегральных типов} и указатели 
из X-регистров, а аргументы типа плавающей точки из D-регистров.

