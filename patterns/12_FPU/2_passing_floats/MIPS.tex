\subsection{MIPS}

\lstinputlisting[caption=\Optimizing GCC 4.4.5 (IDA)]{patterns/12_FPU/2_passing_floats/MIPS_O3_IDA.lst.\LANG}

\RU{И снова мы здесь видим, как LUI загружает 32-битную часть числа типа \Tdouble в \$V0.}
\EN{And again, we see here LUI loading a 32-bit part of a \Tdouble number into \$V0.}
\RU{И снова трудно понять почему}\EN{And again, it's hard to comprehend why}.

\myindex{MIPS!\Instructions!MFC1}
\RU{Новая для нас инструкция это}
\EN{The new instruction for us here is} \INS{MFC1} (\q{Move From Coprocessor 1})\RU{ (копировать из первого сопроцессора)}.
\RU{FPU это сопроцессор под номером 1, вот откуда \q{1} в имени инструкции.}
\EN{The FPU is coprocessor number 1, hence \q{1} in the instruction name.}
\RU{Эта инструкция переносит значения из регистров сопроцессора в регистры основного CPU (\ac{GPR}).}
\EN{This instruction transfers values from the coprocessor's registers to the registers of the CPU (\ac{GPR}).}
\RU{Так что результат исполнения pow() в итоге копируется в регистры \$A3 и \$A2
и из этой пары регистров \printf берет его как 64-битное значение типа \Tdouble.}
\EN{So in the end the result from pow() is moved to registers \$A3 and \$A2, 
and \printf takes a 64-bit double value from this register pair.}
