\subsection{x86}

\RU{Посмотрим, что у нас вышло}\EN{Let's see what we got in} (MSVC 2010):

\lstinputlisting[caption=MSVC 2010]{patterns/12_FPU/2_passing_floats/MSVC_\LANG.asm}

\index{x86!\Instructions!FLD}
\index{x86!\Instructions!FSTP}
\RU{\FLD и \FSTP перемещают переменные из/в сегмента данных в FPU-стек. 
\TT{pow()}\footnote{стандартная функция Си, возводящая число в степень} достает оба значения из FPU-стека и 
возвращает результат в \ST{0}. 
\printf берет 8 байт из стека и трактует их как переменную типа \Tdouble.}
\EN{\FLD and \FSTP are moving variables from/to data segment to FPU stack. 
\TT{pow()}\footnote{standard C function, raises a number to the given power (exponentiation)}
taking both values from FPU-stack and 
returns result in the \ST{0} register.
\printf takes 8 bytes from local stack and interpret them as \Tdouble type variable.}

\ifdefined\IncludeARM
\RU{Кстати, с тем же успехом, можно было бы перекладывать эти два числа из памяти в стек
при помощи пары \MOV}
\EN{By the way, pair of \MOV instructions could be used here for moving values from memory
into stack}: 
\RU{ведь в памяти числа в формате IEEE 754, и pow() также принимает их в том же
формате, и никакая конверсия не требуется.}
\EN{because values in memory are stored in IEEE 754 format, and pow() also takes them in this
format, so, no conversion is necessary.}
\RU{Собственно, так и происходит в следующем примере с ARM}
\EN{That's how it's done in the next ARM example}:
\ref{FPU_passing_floats_ARM}.
\fi
