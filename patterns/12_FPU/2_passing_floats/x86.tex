\subsection{x86}

\RU{Посмотрим, что у нас вышло}\EN{Let's see what we get in} (MSVC 2010):

\lstinputlisting[caption=MSVC 2010]{patterns/12_FPU/2_passing_floats/MSVC.asm.\LANG}

\index{x86!\Instructions!FLD}
\index{x86!\Instructions!FSTP}
\RU{\FLD и \FSTP перемещают переменные из сегмента данных в FPU-стек или обратно. 
\TT{pow()}\footnote{стандартная функция Си, возводящая число в степень} достает оба значения из FPU-стека и 
возвращает результат в \ST{0}. 
\printf берет 8 байт из стека и трактует их как переменную типа \Tdouble.}
\EN{\FLD and \FSTP move variables between the data segment and the FPU stack. 
\TT{pow()}\footnote{a standard C function, raises a number to the given power (exponentiation)}
takes both values from the stack of the FPU and 
returns its result in the \ST{0} register.
\printf takes 8 bytes from the local stack and interprets them as \Tdouble type variable.}

\ifdefined\IncludeARM
\RU{Кстати, с тем же успехом можно было бы перекладывать эти два числа из памяти в стек
при помощи пары \MOV:}
\EN{By the way, a pair of \MOV instructions could be used here for moving values from the memory
into the stack,} 
\RU{ведь в памяти числа в формате IEEE 754, pow() также принимает их в том же
формате, и никакая конверсия не требуется.}
\EN{because the values in memory are stored in IEEE 754 format, and pow() also takes them in this
format, so no conversion is necessary.}
\RU{Собственно, так и происходит в следующем примере с ARM}%
\EN{That's how it's done in the next example, for ARM}:
\myref{FPU_passing_floats_ARM}.
\fi
