\subsection{ARM}

\subsubsection{\OptimizingXcodeIV (\ARMMode)}

\lstinputlisting[caption=\OptimizingXcodeIV (\ARMMode)]{patterns/12_FPU/3_comparison/ARM/Xcode_ARM.lst.\LANG}

\index{ARM!\Registers!APSR}
\index{ARM!\Registers!FPSCR}
\RU{Очень простой случай.}\EN{A very simple case.}
\RU{Входные величины помещаются в}\EN{The input values are placed into the} \TT{D17} \AndENRU \TT{D16} 
\RU{и сравниваются при помощи инструкции}\EN{registers and then compared using the} 
\TT{VCMPE}\EN{ instruction}.
\RU{Как и в сопроцессорах x86, сопроцессор в ARM имеет свой собственный регистр статуса и флагов}%
\EN{Just like in the x86 coprocessor, the ARM coprocessor has its own status and flags register} (\ac{FPSCR}),
\RU{потому что есть необходимость хранить специфичные для его работы флаги.}
\EN{since there is a need to store coprocessor-specific flags.}
% TODO -> расписать регистр по битам
\index{ARM!\Instructions!VMRS}
\RU{И так же, как и в x86}\EN{And just like in x86}, 
\RU{в ARM нет инструкций условного перехода}%
\EN{there are no conditional jump instruction in ARM}, 
\RU{проверяющих биты в регистре статуса сопроцессора}\EN{that can check bits in the status register of the coprocessor}. 
\RU{Поэтому имеется инструкция}\EN{So there is} \TT{VMRS}%
\RU{, копирующая 4 бита}\EN{, which copies 4 bits} (N, Z, C, V) 
\RU{из статуса сопроцессора в биты \IT{общего} статуса (регистр \ac{APSR}).}
\EN{from the coprocessor status word into bits of the \IT{general} status register (\ac{APSR}).}

\index{ARM!\Instructions!VMOVGT}
\TT{VMOVGT} \RU{это аналог}\EN{is the analog of the} \TT{MOVGT}, 
\RU{инструкция для D-регистров, срабатывающая, если при сравнении один операнд был больше чем второй}
\EN{instruction for D-registers, it executes if one operand is greater than the other while comparing} 
(\IT{GT\EMDASH{}Greater Than}). 

\RU{Если она сработает}\EN{If it gets executed}, 
\RU{в \TT{D16} запишется значение $b$}\EN{the value of $b$ is to be written into \TT{D16}}%
\RU{, лежащее в тот момент в}\EN{(that is currently stored in in} \TT{D17}\EN{)}.

\RU{В обратном случае}\EN{Otherwise} 
\RU{в \TT{D16} остается значение $a$.}
\EN{the value of $a$ stays in the \TT{D16} register.}

\index{ARM!\Instructions!VMOV}
\RU{Предпоследняя инструкция \TT{VMOV} готовит то, что было в \TT{D16}, для возврата через 
пару регистров \Reg{0} и \Reg{1}.}
\EN{The penultimate instruction \TT{VMOV} prepares the value in the \TT{D16} register for returning it via the \Reg{0} and \Reg{1}
register pair.}

\subsubsection{\OptimizingXcodeIV (\ThumbTwoMode)}

\begin{lstlisting}[caption=\OptimizingXcodeIV (\ThumbTwoMode)]
VMOV            D16, R2, R3 ; b
VMOV            D17, R0, R1 ; a
VCMPE.F64       D17, D16
VMRS            APSR_nzcv, FPSCR
IT GT 
VMOVGT.F64      D16, D17
VMOV            R0, R1, D16
BX              LR
\end{lstlisting}

\RU{Почти то же самое, что и в предыдущем примере, за парой отличий.}
\EN{Almost the same as in the previous example, however slightly different.}
\RU{Как мы уже знаем, многие инструкции в режиме ARM можно дополнять условием.}
\EN{As we already know, many instructions in ARM mode can be supplemented by condition predicate.}

\RU{Но в режиме Thumb такого нет.}
\EN{But there is no such thing in Thumb mode.} 
\RU{В 16-битных инструкций просто нет места для лишних 4 битов, при помощи
которых можно было бы закодировать условие выполнения.}
\EN{There is no space in the 16-bit instructions for 4 more bits in which conditions can be encoded.}

\index{ARM!\ThumbTwoMode}
\RU{Поэтому в Thumb-2 добавили возможность дополнять Thumb-инструкции условиями.}
\EN{However, Thumb-2 was extended to make it possible to specify predicates to old Thumb instructions.}

\RU{В листинге, сгенерированном при помощи \IDA, мы видим инструкцию \TT{VMOVGT}, 
такую же как и в предыдущем примере.}
\EN{Here, in the \IDA-generated listing, we see the \TT{VMOVGT} instruction, as in previous example.}

\RU{В реальности}\EN{In fact,} 
\RU{там закодирована обычная инструкция \TT{VMOV}}%
\EN{the usual \TT{VMOV} is encoded there}, 
\RU{просто \IDA добавила суффикс \TT{-GT} к ней}%
\EN{but \IDA adds the \TT{-GT} suffix to it}, 
\RU{потому что перед этой инструкцией стоит \TT{IT GT}.}
\EN{since there is a \TT{\q{IT GT}} instruction placed right before it.}

\label{ARM_Thumb_IT}
\index{ARM!\Instructions!IT}
\index{ARM!if-then block}
\EN{The}\RU{Инструкция} \TT{IT} \RU{определяет так называемый}\EN{instruction defines a so-called} \IT{if-then block}. 
\RU{После этой инструкции можно указывать до четырех инструкций, 
к каждой из которых будет добавлен суффикс условия.}
\EN{After the instruction it is possible to place up to 4 instructions, 
each of them has a predicate suffix.}
\RU{В нашем примере}\EN{In our example,} \TT{IT GT} \RU{означает,}\EN{implies}
\RU{что следующая за ней инструкция будет исполнена, если условие}
\EN{that the next instruction is to be executed, if the}
\IT{GT} (\IT{Greater Than}) \RU{справедливо}\EN{condition is true}.

\index{Angry Birds}
\RU{Теперь более сложный пример. Кстати, из}\EN{Here is a more complex code fragment, by the way, from} 
Angry Birds (\RU{для}\EN{for} iOS):

% FIXME russian listing:
\begin{lstlisting}[caption=Angry Birds Classic]
...
ITE NE
VMOVNE          R2, R3, D16
VMOVEQ          R2, R3, D17
BLX             _objc_msgSend ; not prefixed
...
\end{lstlisting}

\TT{ITE} \RU{означает}\EN{stands for} \IT{if-then-else} 
\RU{и кодирует суффиксы для двух следующих за ней инструкций.}
\EN{and it encodes suffixes for the next two instructions.}
\RU{Первая из них исполнится, если условие, закодированное в}
\EN{The first instruction executes if the condition encoded in} \TT{ITE} (\IT{NE, not equal}) 
\RU{будет в тот момент справедливо}\EN{is true at},
\RU{а вторая~--- если это условие не сработает}\EN{and the 
second---if the condition is not true}.
(\RU{Обратное условие от}\EN{The inverse condition of} \TT{NE} \RU{это}\EN{is} \TT{EQ} (\IT{equal})).

\EN{The instruction followed after the second VMOV (or VMOVEQ) is a normal one, not prefixed (BLX).}
\RU{Инструкция следующая за второй VMOV (или VMOEQ) нормальная, без префикса (BLX).}

\index{Angry Birds}
\RU{Ещё чуть сложнее}\EN{One more that's slightly harder}, 
\RU{и снова этот фрагмент из}\EN{which is also from} Angry Birds:

% FIXME russian listing:
\begin{lstlisting}[caption=Angry Birds Classic]
...
ITTTT EQ
MOVEQ           R0, R4
ADDEQ           SP, SP, #0x20
POPEQ.W         {R8,R10}
POPEQ           {R4-R7,PC}
BLX             ___stack_chk_fail ; not prefixed
...
\end{lstlisting}

\RU{Четыре символа \q{T} в инструкции означают, что четыре последующие инструкции будут исполнены если условие соблюдается.}
\EN{Four \q{T} symbols in the instruction mnemonic mean 
that the four subsequent instructions are to be executed if the condition is true.}
\RU{Поэтому \IDA добавила ко всем четырем инструкциям суффикс}
\EN{That's why \IDA adds the} \TT{-EQ}\EN{ suffix
to each one of them}. 

\RU{А если бы здесь было, например,}\EN{And if there was be, for example,}
\TT{ITEEE EQ} (\IT{if-then-else-else-else}), 
\RU{тогда суффиксы для следующих четырех инструкций были бы расставлены так:}
\EN{then the suffixes would have been set as follows:}

\begin{lstlisting}
-EQ
-NE
-NE
-NE
\end{lstlisting}

\index{Angry Birds}
\RU{Ещё фрагмент из}\EN{Another fragment from} Angry Birds:

% FIXME russian listing:
\begin{lstlisting}[caption=Angry Birds Classic]
...
CMP.W           R0, #0xFFFFFFFF
ITTE LE
SUBLE.W         R10, R0, #1
NEGLE           R0, R0
MOVGT           R10, R0
MOVS            R6, #0         ; not prefixed
CBZ             R0, loc_1E7E32 ; not prefixed
...
\end{lstlisting}

\TT{ITTE} (\IT{if-then-then-else}) 
\RU{означает, что первая и вторая инструкции исполнятся, если условие \TT{LE} (\IT{Less or Equal})
справедливо, а третья~--- если справедливо обратное условие (\TT{GT}\EMDASH\IT{Greater Than}).}
\EN{implies that the 1st and 2nd instructions are to be executed if the \TT{LE} (\IT{Less or Equal})
condition is true, and the 3rd---if the inverse condition (\TT{GT}\EMDASH\IT{Greater Than}) 
is true.}

\RU{Компиляторы способны генерировать далеко не все варианты.}
\EN{Compilers usually don't generate all possible combinations.}
\index{Angry Birds}
\RU{Например, в вышеупомянутой игре Angry Birds (версия \IT{classic} для iOS)}
\EN{For example, in the mentioned Angry Birds game (\IT{classic} version for iOS)}
\RU{встречаются только такие варианты инструкции \TT{IT}}\EN{only these variants of the \TT{IT} instruction are used}: 
\TT{IT}, \TT{ITE}, \TT{ITT}, \TT{ITTE}, \TT{ITTT}, \TT{ITTTT}.
\index{\GrepUsage}
\RU{Как я это узнал?}\EN{How did I learn this?}
\RU{В \IDA можно сгенерировать листинг (так я и сделал), только в опциях я установил
показ 4 байтов для каждого опкода.}
\EN{In \IDA It is possible to produce listing files, so I created them with an option 
to show 4 bytes for each opcode .}
\RU{Затем, зная что старшая часть 16-битного опкода (\TT{IT} это \TT{0xBF}), я сделал при помощи \TT{grep} это:}
\EN{Then, knowing the high part of the 16-bit opcode (\TT{IT} is \TT{0xBF}),
I did the following using \TT{grep}:}

\begin{lstlisting}
cat AngryBirdsClassic.lst | grep " BF" | grep "IT" > results.lst
\end{lstlisting}

\index{ARM!\ThumbTwoMode}
\RU{Кстати, если писать на ассемблере для режима Thumb-2 вручную, и дополнять инструкции суффиксами
условия, то ассемблер автоматически будет добавлять инструкцию \TT{IT} с соответствующими флагами там,
где надо.}
\EN{By the way, if you program in ARM assembly language manually for Thumb-2 mode, 
and you add conditional suffixes,
the assembler will add the \TT{IT} instructions automatically with the required flags where it is necessary.}

\subsubsection{\NonOptimizingXcodeIV (\ARMMode)}

\begin{lstlisting}[caption=\NonOptimizingXcodeIV (\ARMMode)]
b               = -0x20
a               = -0x18
val_to_return   = -0x10
saved_R7        = -4

                STR             R7, [SP,#saved_R7]!
                MOV             R7, SP
                SUB             SP, SP, #0x1C
                BIC             SP, SP, #7
                VMOV            D16, R2, R3
                VMOV            D17, R0, R1
                VSTR            D17, [SP,#0x20+a]
                VSTR            D16, [SP,#0x20+b]
                VLDR            D16, [SP,#0x20+a]
                VLDR            D17, [SP,#0x20+b]
                VCMPE.F64       D16, D17
                VMRS            APSR_nzcv, FPSCR
                BLE             loc_2E08
                VLDR            D16, [SP,#0x20+a]
                VSTR            D16, [SP,#0x20+val_to_return]
                B               loc_2E10

loc_2E08
                VLDR            D16, [SP,#0x20+b]
                VSTR            D16, [SP,#0x20+val_to_return]

loc_2E10
                VLDR            D16, [SP,#0x20+val_to_return]
                VMOV            R0, R1, D16
                MOV             SP, R7
                LDR             R7, [SP+0x20+b],#4
                BX              LR
\end{lstlisting}

\RU{Почти то же самое, что мы уже видели}\EN{Almost the same as we already saw}, 
\RU{но много избыточного кода из-за хранения $a$ и $b$, 
а также выходного значения, в локальном стеке.}
\EN{but there is too much redundant code because the $a$ and $b$ variables are stored in the local stack, as well
as the return value.}

\subsubsection{\OptimizingKeilVI (\ThumbMode)}

\begin{lstlisting}[caption=\OptimizingKeilVI (\ThumbMode)]
                PUSH    {R3-R7,LR}
                MOVS    R4, R2
                MOVS    R5, R3
                MOVS    R6, R0
                MOVS    R7, R1
                BL      __aeabi_cdrcmple
                BCS     loc_1C0
                MOVS    R0, R6
                MOVS    R1, R7
                POP     {R3-R7,PC}

loc_1C0
                MOVS    R0, R4
                MOVS    R1, R5
                POP     {R3-R7,PC}
\end{lstlisting}

\RU{Keil не генерирует FPU-инструкции, потому что не 
рассчитывает на то, что они будет поддерживаться, а простым сравнением побитово здесь не обойтись.}
\EN{Keil doesn't generate FPU-instructions since it cannot rely on them being
supported on the target CPU, and it cannot be done by straightforward bitwise comparing.}
%TODO1: why?
\RU{Для сравнения вызывается библиотечная функция}\EN{So it calls an external library
function to do the comparison:} \TT{\_\_aeabi\_cdrcmple}. 
\index{ARM!\Instructions!BCS}\\
\\
N.B. \RU{Результат
сравнения эта функция оставляет в флагах, чтобы следующая за вызовом инструкция}
\EN{The result of the comparison is to be left in the flags by this function, so the following}
\TT{BCS} (\IT{Carry set\RU{~}---\RU{ }Greater than or equal})
\RU{могла работать без дополнительного кода.}\EN{instruction can work without any additional code.}
