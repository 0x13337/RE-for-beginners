\subsection{ARM64}

\subsubsection{\Optimizing GCC (Linaro) 4.9}

\lstinputlisting{patterns/12_FPU/3_comparison/ARM/ARM64_GCC_O3.lst.\LANG}

\EN{The}\RU{В} ARM64 \ac{ISA} \RU{теперь есть FPU-инструкции, устанавливающие флаги CPU}\EN{has FPU-instructions 
which set} \ac{APSR} \RU{вместо}\EN{the CPU flags instead of} \ac{FPSCR} \EN{for convenience}\RU{для удобства}.
\EN{The}\ac{FPU} \RU{больше не отдельное устройство (по крайней мере логически)}\EN{is not a separate device here 
anymore (at least, logically)}.
\index{ARM!\Instructions!FCMPE}
\RU{Это}\EN{Here we see} \TT{FCMPE}. \RU{Она сравнивает два значения, переданных в}\EN{It compares the two values 
passed in} \RegD{0} \AndENRU \RegD{1} 
(\RU{а это первый и второй аргументы функции}\EN{which are the first and second arguments of the function})
\RU{и выставляет флаги в}\EN{and sets} \ac{APSR}\EN{ flags} (N, Z, C, V).

\index{ARM!\Instructions!FCSEL}
\TT{FCSEL} (\IT{Floating Conditional Select}) \RU{копирует значение}\EN{copies the value of} \RegD{0} \OrENRU 
\RegD{1} \RU{в}\EN{into} \RegD{0} \RU{в зависимости от условия}\EN{depending on the condition} 
(\TT{GT}\EMDASH{}Greater Than\RU{\EMDASH{}больше чем}),
\RU{и снова, она использует флаги в регистре}\EN{and again, it uses flags in} \ac{APSR} \RU{вместо}\EN{register
instead of} \ac{FPSCR}.
\RU{Это куда удобнее, если сравнивать с тем набором инструкций, что был в процессорах раньше.}
\EN{This is much more convenient, compared to the instruction set in older CPUs.}

\RU{Если условие верно}\EN{If the condition is true} (\TT{GT}), \RU{тогда значение из}\EN{then the value of} \RegD{0} 
\RU{копируется в}\EN{is copied into} \RegD{0} (\RU{т.е. ничего не происходит}\EN{i.e., nothing happens}).
\RU{Если условие не верно, то значение}\EN{If the condition is not true, the value of} \RegD{1} 
\RU{копируется в}\EN{is copied into} \RegD{0}.

\subsubsection{\NonOptimizing GCC (Linaro) 4.9}

\lstinputlisting{patterns/12_FPU/3_comparison/ARM/ARM64_GCC.lst.\LANG}

\RU{Неоптимизирующий GCC более многословен}\EN{Non-optimizing GCC is more verbose}.
\RU{В начале функция сохраняет значения входных аргументов в локальном стеке}
\EN{First, the function saves its input argument values in the local stack} (\IT{Register Save Area}).
\RU{Затем код перезагружает значения в регистры}\EN{Then the code reloads these values into registers}
\RegX{0}/\RegX{1} \RU{и наконец копирует их в}\EN{and finally copies them to} 
\RegD{0}/\RegD{1} \RU{для сравнения инструкцией}\EN{to be compared using} \TT{FCMPE}. 
\RU{Много избыточного кода, но так работают неоптимизирующие компиляторы}\EN{A lot of redundant code, 
but that is how non-optimizing compilers work}.
\TT{FCMPE} \RU{сравнивает значения и устанавливает флаги в}\EN{compares the values and sets the} \ac{APSR}\EN{ flags}.
\RU{В этот момент компилятор ещё не думает о более удобной инструкции}\EN{At this moment, 
the compiler is not thinking yet about the more convenient} \TT{FCSEL}\RU{, так что он работает старым 
методом}\EN{ instruction, so it proceed using old methods}: 
\RU{использует инструкцию}\EN{using the} \TT{BLE}\EN{ instruction} (\IT{Branch if Less than or Equal}\RU{ (переход
если меньше или равно)}).
\RU{В одном случае}\EN{In the first case} ($a>b$)\EN{, the value of}\RU{ значение} $a$ \RU{перезагружается в}\EN{gets loaded 
into} \RegX{0}.
\RU{В другом случае}\EN{In the other case} ($a<=b$)\EN{, the value of}\RU{ значение} $b$ \RU{загружается в}\EN{gets loaded into} 
\RegX{0}.
\RU{Наконец, значение из}\EN{Finally, the value from} \RegX{0} \RU{копируется в}\EN{gets copied into} \RegD{0}, 
\RU{потому что возвращаемое значение оставляется в этом регистре}\EN{because the return value needs to be in this 
register}.

\myparagraph{\Exercise}

\RU{Для упражнения вы можете попробовать оптимизировать этот фрагмент кода вручную, удалив избыточные инструкции,
но не добавляя новых (включая \TT{FCSEL})}\EN{As an exercise, you can try optimizing this piece of code 
manually by removing redundant instructions and not introducing new ones (including \TT{FCSEL})}.

\subsubsection{\Optimizing GCC (Linaro) 4.9\EMDASH{}float}

\RU{Я ещё переписал пример. Теперь здесь \Tfloat вместо \Tdouble}\EN{I also rewrote this example 
to use \Tfloat instead of \Tdouble}.

\begin{lstlisting}
float f_max (float a, float b)
{
	if (a>b)
		return a;

	return b;
};
\end{lstlisting}

\lstinputlisting{patterns/12_FPU/3_comparison/ARM/ARM64_GCC_O3_float.lst.\LANG}

\RU{Всё то же самое, только используются S-регистры вместо D-.}
\EN{It is the same code, but the S-registers are used instead of D- ones.}
\RU{Так что числа типа \Tfloat передаются в 32-битных S-регистрах (а это младшие части 64-битных D-регистров).}
\EN{It's because numbers of type \Tfloat are passed in 32-bit S-registers (which are in fact the lower parts of the 64-bit D-registers).}

