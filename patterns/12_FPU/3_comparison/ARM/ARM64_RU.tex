\subsection{ARM64}

\subsubsection{\Optimizing GCC (Linaro) 4.9}

\lstinputlisting{patterns/12_FPU/3_comparison/ARM/ARM64_GCC_O3_RU.lst}

В ARM64 \ac{ISA} теперь есть FPU-инструкции, устанавливающие флаги CPU \ac{APSR} вместо \ac{FPSCR} для удобства.
\ac{FPU} больше не отдельное устройство (по крайней мере логически).
\myindex{ARM!\Instructions!FCMPE}
Это \INS{FCMPE}. Она сравнивает два значения, переданных в \RegD{0} и \RegD{1} 
(а это первый и второй аргументы функции) и выставляет флаги в \ac{APSR} (N, Z, C, V).

\myindex{ARM!\Instructions!FCSEL}
\INS{FCSEL} (\IT{Floating Conditional Select}) копирует значение \RegD{0} или
\RegD{1} в \RegD{0} в зависимости от условия 
(\GTT{GT} --- Greater Than --- больше чем),
и снова, она использует флаги в регистре \ac{APSR} вместо \ac{FPSCR}.
Это куда удобнее, если сравнивать с тем набором инструкций, что был в процессорах раньше.

Если условие верно (\GTT{GT}), тогда значение из \RegD{0} копируется в \RegD{0} (т.е. ничего не происходит).
Если условие не верно, то значение \RegD{1} копируется в \RegD{0}.

\subsubsection{\NonOptimizing GCC (Linaro) 4.9}

\lstinputlisting{patterns/12_FPU/3_comparison/ARM/ARM64_GCC_RU.lst}

Неоптимизирующий GCC более многословен.
В начале функция сохраняет значения входных аргументов в локальном стеке (\IT{Register Save Area}).
Затем код перезагружает значения в регистры
\RegX{0}/\RegX{1} и наконец копирует их в 
\RegD{0}/\RegD{1} для сравнения инструкцией \INS{FCMPE}. 
Много избыточного кода, но так работают неоптимизирующие компиляторы.
\INS{FCMPE} сравнивает значения и устанавливает флаги в \ac{APSR}.
В этот момент компилятор ещё не думает о более удобной инструкции \INS{FCSEL}, так что он работает старым 
методом: 
использует инструкцию \INS{BLE} (\IT{Branch if Less than or Equal} (переход если меньше или равно)).
В одном случае ($a>b$) значение $a$ перезагружается в \RegX{0}.
В другом случае ($a<=b$) значение $b$ загружается в \RegX{0}.
Наконец, значение из \RegX{0} копируется в \RegD{0}, 
потому что возвращаемое значение оставляется в этом регистре.

\myparagraph{\Exercise}

Для упражнения вы можете попробовать оптимизировать этот фрагмент кода вручную, удалив избыточные инструкции,
но не добавляя новых (включая \INS{FCSEL}).

\subsubsection{\Optimizing GCC (Linaro) 4.9\EMDASH{}float}

Перепишем пример. Теперь здесь \Tfloat вместо \Tdouble.

\begin{lstlisting}
float f_max (float a, float b)
{
	if (a>b)
		return a;

	return b;
};
\end{lstlisting}

\lstinputlisting{patterns/12_FPU/3_comparison/ARM/ARM64_GCC_O3_float_RU.lst}

Всё то же самое, только используются S-регистры вместо D-.
Так что числа типа \Tfloat передаются в 32-битных S-регистрах (а это младшие части 64-битных D-регистров).

