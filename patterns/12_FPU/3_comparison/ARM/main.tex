\subsection{ARM}

\subsubsection{\OptimizingXcode + \ARMMode}

\begin{lstlisting}[caption=\OptimizingXcode + \ARMMode]
VMOV            D16, R2, R3 ; b
VMOV            D17, R0, R1 ; a
VCMPE.F64       D17, D16
VMRS            APSR_nzcv, FPSCR
VMOVGT.F64      D16, D17 ; copy b to D16
VMOV            R0, R1, D16
BX              LR
\end{lstlisting}

\index{ARM!\Registers!APSR}
\index{ARM!\Registers!FPSCR}
\RU{Очень простой случай.}\EN{A very simple case.}
\RU{Входные величины помещаются в}\EN{Input values are placed into the} \TT{D17} \AndENRU \TT{D16} 
\RU{и сравниваются при помощи инструкции}\EN{registers and then compared with the help of} 
\TT{VCMPE}\EN{ instruction}.
\RU{Как и в сопроцессорах x86, сопроцессор в ARM имеет свой собственный регистр статуса и флагов}
\EN{Just like in x86 coprocessor, ARM coprocessor has its own status and flags register}, (\TT{FPSCR}),
\RU{потому как есть необходимость хранить специфичные для его работы флаги.}
\EN{since there is a need to store coprocessor-specific flags.}

\index{ARM!\Instructions!VMRS}
\RU{И так же, как и в x86}\EN{And just like in x86}, 
\RU{в ARM нет инструкций условного перехода}
\EN{there are no conditional jump instruction in ARM}, 
\RU{проверяющих биты в регистре статуса сопроцессора}\EN{checking bits in coprocessor status register}, 
\RU{так что имеется инструкция}\EN{so there is} \TT{VMRS}
\RU{, копирующая 4 бита}\EN{ instruction, copying 4 bits} (N, Z, C, V) 
\RU{из статуса сопроцессора в биты \IT{общего} статуса (регистр \TT{APSR}).}
\EN{from the coprocessor status word into bits of \IT{general} status (\TT{APSR} register).}

\index{ARM!\Instructions!VMOVGT}
\TT{VMOVGT} \RU{это аналог}\EN{is analogue of} \TT{MOVGT}, 
\RU{инструкция, сработающая если при сравнении один операнд был больше чем второй}
\EN{instruction, to be executed if one operand is greater than other while comparing} 
(\IT{GT\EMDASH{}Greater Than}). 

\RU{Если она сработает}\EN{If it will be executed}, 
\RU{в \TT{D16} запишется значение $b$}\EN{$b$ value will be written into \TT{D16}}, 
\RU{лежащее в тот момент в}\EN{stored at the moment in} \TT{D17}.

\RU{А если не сработает}\EN{And if it will not be triggered}, 
\RU{то в \TT{D16} останется лежать значение $a$.}
\EN{then $a$ value will stay in the \TT{D16} register.}

\index{ARM!\Instructions!VMOV}
\RU{Предпоследняя инструкция \TT{VMOV} подготовит то что было в \TT{D16} для возврата через 
пару регистров \Reg{0} и \Reg{1}.}
\EN{Penultimate instruction \TT{VMOV} will prepare value in the \TT{D16} register for returning via \Reg{0} and \Reg{1}
registers pair.}

\subsubsection{\OptimizingXcode + \ThumbTwoMode}

\begin{lstlisting}[caption=\OptimizingXcode + \ThumbTwoMode]
VMOV            D16, R2, R3 ; b
VMOV            D17, R0, R1 ; a
VCMPE.F64       D17, D16
VMRS            APSR_nzcv, FPSCR
IT GT 
VMOVGT.F64      D16, D17
VMOV            R0, R1, D16
BX              LR
\end{lstlisting}

\RU{Почти то же самое что и в предыдущем примере, за парой отличий.}
\EN{Almost the same as in previous example, howeverm slightly different.}
\RU{Дело в том, многие инструкции в режиме ARM
можно дополнять условием, которое если справедливо, то инструкция выполнится.}
\EN{As a matter of fact, many instructions in ARM mode can be supplied by condition predicate,
and the instruction is to be executed if condition is true.}

\RU{Но в режиме thumb такого нет}
\EN{But there is no such thing in thumb mode}. 
\RU{В 16-битных инструкций просто нет места для лишних 4 битов, при помощи
которых можно было бы закодировать условие выполнения.}
\EN{There is no place in 16-bit instructions for spare 4 bits where condition can be encoded.}

\index{ARM!\ThumbTwoMode}
\RU{Поэтому в thumb-2 добавили возможность дополнять thumb-инструкции условиями.}
\EN{However, thumb-2 was extended to make possible to specify predicates to old thumb instructions.}

\RU{Здесь, в листинге сгенерированном при помощи \IDA, мы видим инструкцию \TT{VMOVGT}, 
такую же как и в предыдущем примере.}
\EN{Here, is the \IDA-generated listing, we see \TT{VMOVGT} instruction, the same as in previous example.}

\RU{Но в реальности}\EN{But in fact}, 
\RU{там закодирована обычная инструкция \TT{VMOV}}
\EN{usual \TT{VMOV} is encoded there}, 
\RU{просто \IDA добавила суффикс \TT{-GT} к ней}
\EN{but \IDA added \TT{-GT} suffix to it}, 
\RU{потому что перед этой инструкцией стоит \TT{``IT GT''}}
\EN{since there is \TT{``IT GT''} instruction placed right before}.

\index{ARM!\Instructions!IT}
\index{ARM!if-then block}
\RU{Инструкция }\TT{IT} \RU{определяет так называемый}\EN{instruction defines so-called} \IT{if-then block}. 
\RU{После этой инструкции, можно указывать до четырех инструкций, к которым будет добавлен суффикс условия.}
\EN{After the instruction, it is possible to place up to 4 instructions, to which predicate suffix will be added.}
\RU{В нашем примере}\EN{In our example}, \TT{``IT GT''} \RU{означает}\EN{meaning},
\RU{что следующая за ней инструкция будет исполнена}\EN{the next instruction will be executed}, 
\RU{если условие}\EN{if} \IT{GT} (\IT{Greater Than}) \RU{справедливо}\EN{condition is true}.

\index{Angry Birds}
\RU{Теперь более сложный пример, кстати, из}\EN{Now more complex code fragment, by the way, from} 
``Angry Birds'' (\RU{для}\EN{for} iOS):

\begin{lstlisting}[caption=Angry Birds Classic]
ITE NE
VMOVNE          R2, R3, D16
VMOVEQ          R2, R3, D17
\end{lstlisting}

\TT{ITE} \RU{означает}\EN{meaning} \IT{if-then-else} 
\RU{и кодирует суффиксы для двух следующих за ней инструкций.}
\EN{and it encode suffixes for two next instructions.}
\RU{Первая из них исполнится, если условие, закодированное в}
\EN{First instruction will execute if condition encoded in} \TT{ITE} (\IT{NE, not equal}) 
\RU{будет в тот момент справедливо}\EN{will be true at the moment},
\RU{а вторая ~--- если это условие не сработает}\EN{and the 
second~---if the condition will not be true}.
(\RU{Обратное условие от}\EN{Inverse condition of} \TT{NE} \RU{это}\EN{is} \TT{EQ} (\IT{equal})).

\index{Angry Birds}
\RU{Еще чуть сложнее}\EN{Slightly harder}, \RU{и снова этот фрагмент из}\EN{and this fragment from} 
``Angry Birds''\EN{ as well}:

\begin{lstlisting}[caption=Angry Birds Classic]
ITTTT EQ
MOVEQ           R0, R4
ADDEQ           SP, SP, #0x20
POPEQ.W         {R8,R10}
POPEQ           {R4-R7,PC}
\end{lstlisting}

\RU{4 символа ``T'' в инструкции означают что 4 следующие инструкции будут исполнены если условие соблюдается.}
\EN{4 ``T'' symbols in instruction mnemonic means 
the 4 next instructions will be executed if condition is true.}
\RU{Поэтому \IDA добавила ко всем четырем инструкциям суффикс}
\EN{That's why \IDA added} \TT{-EQ}\EN{ suffix
to each 4 instructions}. 

\RU{А если бы здесь было, например,}\EN{And if there will be e.g.}
\TT{ITEEE EQ} (\IT{if-then-else-else-else}), 
\RU{тогда суффиксы для следующих четырех инструкций были бы расставлены так:}
\EN{then suffixes will be set as follows:}

\begin{lstlisting}
-EQ
-NE
-NE
-NE
\end{lstlisting}

\index{Angry Birds}
\RU{Еще фрагмент из}\EN{Another fragment from} ``Angry Birds'':

\begin{lstlisting}[caption=Angry Birds Classic]
CMP.W           R0, #0xFFFFFFFF
ITTE LE
SUBLE.W         R10, R0, #1
NEGLE           R0, R0
MOVGT           R10, R0
\end{lstlisting}

\TT{ITTE} (\IT{if-then-then-else}) \RU{означает что первая и вторая инструкции исполнятся}
\EN{means the 1st and 2nd instructions will be executed}, 
\RU{если условие}\EN{if} \TT{LE} (\IT{Less or Equal}) \RU{справедливо}\EN{condition is true},
\RU{а третья}\EN{and 3rd}\EMDASH\RU{если справедливо обратное условие}\EN{if inverse condition} 
(\TT{GT}\EMDASH\IT{Greater Than})\EN{ is true}.

\RU{Компиляторы способны генерировать далеко не все варианты.}
\EN{Compilers usually are not generating all possible combinations.}
\index{Angry Birds}
\RU{Например, в вышеупомянутой игре ``Angry Birds'' (версия \IT{classic} для iOS)}
\EN{For example, it mentioned ``Angry Birds'' game (\IT{classic} version for iOS)}
\RU{встречаются только такие варианты инструкции \TT{IT}}\EN{only these cases of \TT{IT} instruction are used}: 
\TT{IT}, \TT{ITE}, \TT{ITT}, \TT{ITTE}, \TT{ITTT}, \TT{ITTTT}.
\index{\GrepUsage}
\RU{Как я это узнал?}\EN{How I learnt this?}
\RU{В \IDA можно сгенерировать листинг, так я и сделал, только в опциях я установил так 
чтобы показывались 4 байта для каждого опкода.}
\EN{In \IDA it is possible to produce listing files, so I did it, but I also set in options 
to show 4 bytes of each opcodes .}
\RU{Затем, зная, что старшая часть 16-битного опкода \TT{IT} это \TT{0xBF}, я сделал при помощи \TT{grep} это:}
\EN{Then, knowing the high part of 16-bit opcode \TT{IT} is \TT{0xBF},
I did this with \TT{grep}:}

\begin{lstlisting}
cat AngryBirdsClassic.lst | grep " BF" | grep "IT" > results.lst
\end{lstlisting}

\index{ARM!\ThumbTwoMode}
\RU{Кстати, если писать на ассемблере для режима thumb-2 вручную, и дополнять инструкции суффиксами
условия, то ассемблер автоматически будет добавлять инструкцию \TT{IT} с соответствующими флагами, там,
где надо.}
\EN{By the way, if to program in ARM assembly language manually for thumb-2 mode, 
with adding conditional suffixes,
assembler will add \TT{IT} instructions automatically, with respectable flags, where it is necessary.}

\subsubsection{\NonOptimizingXcode + \ARMMode}

\begin{lstlisting}[caption=\NonOptimizingXcode + \ARMMode]
b               = -0x20
a               = -0x18
val_to_return   = -0x10
saved_R7        = -4

                STR             R7, [SP,#saved_R7]!
                MOV             R7, SP
                SUB             SP, SP, #0x1C
                BIC             SP, SP, #7
                VMOV            D16, R2, R3
                VMOV            D17, R0, R1
                VSTR            D17, [SP,#0x20+a]
                VSTR            D16, [SP,#0x20+b]
                VLDR            D16, [SP,#0x20+a]
                VLDR            D17, [SP,#0x20+b]
                VCMPE.F64       D16, D17
                VMRS            APSR_nzcv, FPSCR
                BLE             loc_2E08
                VLDR            D16, [SP,#0x20+a]
                VSTR            D16, [SP,#0x20+val_to_return]
                B               loc_2E10

loc_2E08
                VLDR            D16, [SP,#0x20+b]
                VSTR            D16, [SP,#0x20+val_to_return]

loc_2E10
                VLDR            D16, [SP,#0x20+val_to_return]
                VMOV            R0, R1, D16
                MOV             SP, R7
                LDR             R7, [SP+0x20+b],#4
                BX              LR
\end{lstlisting}

\RU{Почти то же самое что мы уже видели}\EN{Almost the same we already saw}, 
\RU{но много избыточного кода из-за хранения $a$ и $b$, 
а также выходного значения, в локальном стеке.}
\EN{but too much redundant code because of $a$ and $b$ variables storage in local stack, as well
as returning value.}

\subsubsection{\OptimizingKeil + \ThumbMode}

\begin{lstlisting}[caption=\OptimizingKeil + \ThumbMode]
                PUSH    {R3-R7,LR}
                MOVS    R4, R2
                MOVS    R5, R3
                MOVS    R6, R0
                MOVS    R7, R1
                BL      __aeabi_cdrcmple
                BCS     loc_1C0
                MOVS    R0, R6
                MOVS    R1, R7
                POP     {R3-R7,PC}

loc_1C0
                MOVS    R0, R4
                MOVS    R1, R5
                POP     {R3-R7,PC}
\end{lstlisting}

\RU{Keil не генерирует специальную инструкцию для сравнения чисел с плавающей запятой, потому что не 
рассчитывает на то что она будет поддерживаться, а простым сравнением побитово здесь не обойтись.}
\EN{Keil not generates special instruction for float numbers comparing since it cannot rely it will
be supported on the target CPU, and it cannot be done by straightforward bitwise comparing.}
%TODO: why?
\RU{Для сравнения вызывается библиотечная функция}\EN{So there is called external library
function for comparing:} \TT{\_\_aeabi\_cdrcmple}. 
\index{ARM!\Instructions!BCS}
N.B. \RU{Результат
сравнения эта функция оставляет в флагах, чтобы следующая за вызовом инструкция}
\EN{Comparison result is to be leaved in flags, so the following}
\TT{BCS} (\IT{Carry set - Greater than or equal})
\RU{могла работать без дополнительного кода.}\EN{instruction may work without any additional code.}

 
