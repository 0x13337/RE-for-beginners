\subsection{\NonOptimizing MSVC}

\RU{Вот что выдал MSVC 2010}\EN{MSVC 2010 generated}:

\lstinputlisting[caption=MSVC 2010]{patterns/12_FPU/3_comparison/MSVC_\LANG.asm}

\index{x86!\Instructions!FLD}
\RU{Итак, \FLD загружает \TT{\_b} в регистр \ST{0}.}
\EN{So, \FLD loading \TT{\_b} into the \ST{0} register.}

\label{Czero_etc}
\newcommand{\Czero}{\TT{C0}\xspace}
\newcommand{\Ctwo}{\TT{C2}\xspace}
\newcommand{\Cthree}{\TT{C3}\xspace}
\newcommand{\CThreeBits}{\Cthree/\Ctwo/\Czero}

\index{x86!\Instructions!FCOMP}
\RU{\FCOMP сравнивает содержимое \ST{0} с тем что лежит в \TT{\_a} и выставляет биты \CThreeBits в 
регистре статуса FPU. Это 16-битный регистр отражающий текущее состояние FPU.} 
\EN{\FCOMP compares the value in the \ST{0} register with what is in \TT{\_a} value 
and set \CThreeBits bits in FPU 
status word register. 
This is 16-bit register reflecting current state of FPU.} 

\RU{Итак, биты \CThreeBits выставлены, но, к сожалению, у процессоров до Intel P6 
\footnote{Intel P6 это Pentium Pro, Pentium II, и далее} нет инструкций условного перехода,
проверяющих эти биты. 
Возможно, так сложилось исторически (вспомните о том что FPU когда-то был вообще отдельным чипом). 
А у Intel P6 появились инструкции \FCOMI/\FCOMIP/\FUCOMI/\FUCOMIP ~--- делающие тоже самое, 
только напрямую модифицирующие флаги \ZF/\PF/\CF.}
\EN{For now \CThreeBits bits are set, but unfortunately, CPU before Intel P6
\footnote{Intel P6 is Pentium Pro, Pentium II, etc} has not any conditional 
jumps instructions which are checking these bits. 
Probably, it is a matter of history (remember: FPU was separate chip in past). 
Modern CPU starting at Intel P6 has \FCOMI/\FCOMIP/\FUCOMI/\FUCOMIP 
instructions~---which does the same, but modifies CPU flags \ZF/\PF/\CF.}

\RU{После этого, инструкция \FCOMP выдергивает одно значение из стека. 
Это отличает её от \FCOM, которая просто сравнивает значения, оставляя стек в таком же состоянии.}
\EN{After bits are set, the \FCOMP instruction popping one variable from stack. 
This is what distinguish it from \FCOM, which is just comparing values, leaving the stack at the same state.}

\index{x86!\Instructions!FNSTSW}
\RU{\FNSTSW копирует содержимое регистра статуса в \AX. Биты \CThreeBits занимают позиции, 
соответственно, 14, 10, 8, в этих позициях они и остаются в регистре \AX, 
и все они расположены в старшей части регистра ~--- \AH.}
\EN{\FNSTSW copies FPU status word register to the \AX. Bits \CThreeBits are placed at positions 14/10/8, 
they will be at the same positions in the \AX register and all they are placed in high part of the \AX{}~---\AH{}.}

\begin{itemize}
\item
\RU{Если b>a в нашем случае, то биты \CThreeBits должны быть выставлены так:}
\EN{If b>a in our example, then \CThreeBits bits will be set as following:} 0, 0, 0.
\item
\RU{Если a>b, то биты будут выставлены:}\EN{If a>b, then bits will be set:} 0, 0, 1.
\item
\RU{Если a=b, то биты будут выставлены так:}\EN{If a=b, then bits will be set:} 1, 0, 0.
\end{itemize}
% TODO: table here?

\RU{После исполнения \TT{test ah, 5}, бит \Cthree и \TT{C1} сбросится в ноль, 
на позициях 0 и 2 (внутри регистра \AH) 
останутся соответственно \Czero и \Ctwo.}
\EN{After \TT{test ah, 5} execution, bits \Cthree and \TT{C1} will be set to 0, 
but at positions 0 and 2 (in the \AH registers) 
\Czero and \Ctwo bits will be leaved.}

\label{parity_flag}
\index{x86!\Registers!\RU{Флаг четности}\EN{Parity flag}}
\RU{Теперь немного о \IT{parity flag}\footnote{флаг четности}. Еще один замечательный рудимент}
\EN{Now let's talk about parity flag. Another notable epoch rudiment}:

\begin{framed}
\begin{quotation}
One common reason to test the parity flag actually has nothing to do with parity. The FPU has four condition flags 
(C0 to C3), but they can not be tested directly, and must instead be first copied to the flags register. 
When this happens, C0 is placed in the carry flag, C2 in the parity flag and C3 in the zero flag. 
The C2 flag is set when e.g. incomparable floating point values (NaN or unsupported format) are compared 
with the FUCOM instructions.\footnote{\url{http://en.wikipedia.org/wiki/Parity_flag}}
\end{quotation}
\end{framed}

\RU{Этот флаг выставляется в $1$ если количество единиц в последнем результате ~--- четно. 
И в $0$ если ~--- нечетно.}
\EN{This flag is to be set to $1$ if ones number is even. And to $0$ if odd.}

\index{x86!\Instructions!JP}
\RU{Таким образом, что мы имеем, флаг \PF будет выставлен в $1$, если \Czero и \Ctwo 
оба $1$ или оба $0$. 
И тогда сработает последующий \JP (\IT{jump if PF==1}). 
Если мы вернемся чуть назад и посмотрим значения \CThreeBits 
для разных вариантов, то увидим, что условный переход \JP сработает в двух случаях: если b>a или если a==b 
(ведь бит \Cthree уже \IT{вылетел} после исполнения \TT{test ah, 5}).}
\EN{Thus, \PF flag will be set to 1 if both \Czero and \Ctwo are set to $0$ or both are $1$.
And then following \JP (\IT{jump if PF==1}) will be triggered. 
If we recall values of the \CThreeBits for various cases,
we will see the conditional jump 
\JP will be triggered in two cases: if b>a or a==b 
(\Cthree bit is already not considering here since it was cleared while execution of 
the \TT{test ah, 5} instruction).}

\RU{Дальше все просто. Если условный переход сработал, то \FLD загрузит значение \TT{\_b} в \ST{0}, 
а если не сработал, то загрузится \TT{\_a} и произойдет выход из функции.}
\EN{It is all simple thereafter. If conditional jump was triggered, \FLD will load the \TT{\_b} value 
to the \ST{0} register, and if it is not triggered, the value of the \TT{\_a} variable will be loaded.}

\RU{Но это еще не все!}\EN{But it is not over yet!}
