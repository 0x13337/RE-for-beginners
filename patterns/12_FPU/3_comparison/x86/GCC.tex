\ifdefined\IncludeGCC
\subsubsection{GCC 4.4.1}

\lstinputlisting[caption=GCC 4.4.1]{patterns/12_FPU/3_comparison/x86/GCC.asm.\LANG}

\myindex{x86!\Instructions!FUCOMPP}
\RU{\FUCOMPP~--- это почти то же что и \FCOM, только выкидывает из стека оба значения после сравнения, 
а также несколько иначе реагирует на \q{не-числа}.}
\EN{\FUCOMPP{} is almost like \FCOM, but pops both values from the stack and handles
\q{not-a-numbers} differently.}

\myindex{\RU{Не-числа}\EN{Non-a-numbers} (NaNs)}
\RU{Немного о \IT{не-числах}}\EN{A bit about \IT{not-a-numbers}}.

\newcommand{\NANFN}{\footnote{\RU{\href{http://go.yurichev.com/17129}{ru.wikipedia.org/wiki/NaN}}%
\EN{\href{http://go.yurichev.com/17130}{wikipedia.org/wiki/NaN}}}}

\RU{FPU умеет работать со специальными переменными, которые числами не являются и называются \q{не числа} или 
\gls{NaN}\NANFN. 
Это бесконечность, результат деления на ноль, и так далее. Нечисла бывают \q{тихие} и \q{сигнализирующие}. 
С первыми можно продолжать работать и далее, а вот если вы попытаетесь совершить какую-то операцию 
с сигнализирующим нечислом, то сработает исключение.}
\EN{The FPU is able to deal with special values which are \IT{not-a-numbers} or 
\gls{NaN}s\NANFN. 
These are infinity, result of division by 0, etc. 
Not-a-numbers can be \q{quiet} and \q{signaling}. It is possible to continue to work with \q{quiet} NaNs, 
but if one tries to do any operation with \q{signaling} NaNs, an exception is to be raised.}

\myindex{x86!\Instructions!FCOM}
\myindex{x86!\Instructions!FUCOM}
\RU{Так вот, \FCOM вызовет исключение если любой из операндов какое-либо нечисло.
\FUCOM же вызовет исключение только если один из операндов именно \q{сигнализирующее нечисло}.}
\EN{\FCOM raising an exception if any operand is \gls{NaN}. 
\FUCOM raising an exception only if any operand is a signaling \gls{NaN} (SNaN).}

\myindex{x86!\Instructions!SAHF}
\label{SAHF}
\RU{Далее мы видим \SAHF (\IT{Store AH into Flags})~--- это довольно редкая инструкция в коде, не использующим FPU. 
8 бит из \AH перекладываются в младшие 8 бит регистра статуса процессора в таком порядке:}
\EN{The next instruction is \SAHF (\IT{Store AH into Flags})~---this is a rare 
instruction in code not related to the FPU. 
8 bits from AH are moved into the lower 8 bits of the CPU flags in the following order:}

\begin{center}
\begin{bytefield}[endianness=big,bitwidth=0.03\linewidth]{8}
\bitheader{7,6,4,2,0} \\
\bitbox{1}{SF} & 
\bitbox{1}{ZF} & 
\bitbox{1}{} & 
\bitbox{1}{AF} & 
\bitbox{1}{} & 
\bitbox{1}{PF} & 
\bitbox{1}{} & 
\bitbox{1}{CF}
\end{bytefield}
\end{center}


\myindex{x86!\Instructions!FNSTSW}
\RU{Вспомним, что \FNSTSW перегружает интересующие нас биты \CThreeBits в \AH, 
и соответственно они будут в позициях 6, 2, 0 в регистре \AH:}
\EN{Let's recall that \FNSTSW moves the bits that interest us (\CThreeBits) into \AH 
and they are in positions 6, 2, 0 of the \AH register:}

\begin{center}
\ifdefined\ebook
\begin{bytefield}[endianness=big,bitwidth=0.06\linewidth]{8}
\else
\begin{bytefield}[endianness=big,bitwidth=0.03\linewidth]{8}
\fi
\bitheader{6,2,1,0} \\
\bitbox{1}{} & 
\bitbox{1}{\TT{C3}} & 
\bitbox{3}{} & 
\bitbox{1}{\TT{C2}} & 
\bitbox{1}{\TT{C1}} & 
\bitbox{1}{\TT{C0}}
\end{bytefield}
\end{center}


\RU{Иными словами, пара инструкций \INS{fnstsw  ax / sahf} перекладывает биты \CThreeBits в флаги \ZF, \PF, \CF.}
\EN{In other words, the \INS{fnstsw  ax / sahf} instruction pair moves \CThreeBits into \ZF, \PF and \CF.}

\RU{Теперь снова вспомним, какие значения бит \CThreeBits будут при каких результатах сравнения:}
\EN{Now let's also recall the values of \CThreeBits in different conditions:}

\begin{itemize}
\item
\RU{Если $a$ больше $b$ в нашем случае, то биты \CThreeBits должны быть выставлены так:}
\EN{If $a$ is greater than $b$ in our example, then \CThreeBits are to be set to:} 0, 0, 0.
\item
\RU{Если $a$ меньше $b$, то биты будут выставлены так:}
\EN{if $a$ is less than $b$, then the bits are to be set to:} 0, 0, 1.
\item
\RU{Если $a=b$, то так:}\EN{If $a=b$, then:} 1, 0, 0.
\end{itemize}
% TODO: table?

\RU{Иными словами, после трех инструкций \FUCOMPP/\FNSTSW/\SAHF 
возможны такие состояния флагов:}
\EN{In other words, these states of the CPU flags are possible
after three \FUCOMPP/\FNSTSW/\SAHF instructions:}

\begin{itemize}
\item
\RU{Если $a>b$ в нашем случае, то флаги будут выставлены так:}
\EN{If $a>b$, the CPU flags are to be set as:} \GTT{ZF=0, PF=0, CF=0}.
\item
\RU{Если $a<b$, то флаги будут выставлены так:}
\EN{If $a<b$, then the flags are to be set as:} \GTT{ZF=0, PF=0, CF=1}.
\item
\RU{Если $a=b$, то так:}\EN{And if $a=b$, then:} \GTT{ZF=1, PF=0, CF=0}.
\end{itemize}
% TODO: table?

\myindex{x86!\Instructions!SETcc}
\myindex{x86!\Instructions!JNBE}
\RU{Инструкция \SETNBE выставит в \AL единицу или ноль в зависимости от флагов и условий. 
Это почти аналог \JNBE, за тем лишь исключением, что \SETcc
\footnote{\IT{cc} это \IT{condition code}}
выставляет 1 или 0 в \AL, а \Jcc делает переход или нет. 
\SETNBE запишет 1 только если \GTT{CF=0} и \GTT{ZF=0}. Если это не так, то запишет 0 в \AL.}
\EN{Depending on the CPU flags and conditions, \SETNBE stores 1 or 0 to AL. 
It is almost the counterpart of \JNBE, with the exception that \SETcc 
\footnote{\IT{cc} is \IT{condition code}} stores 1 or 0 in \AL, 
but \Jcc does actually jump or not. 
\SETNBE stores 1 only if \GTT{CF=0} and \GTT{ZF=0}. 
If it is not true, 0 is to be stored into \AL.}

\RU{\CF будет 0 и \ZF будет 0 одновременно только в одном случае: если $a>b$.}
\EN{Only in one case both \CF and \ZF are 0: if $a>b$.}

\RU{Тогда в \AL будет записана 1, последующий условный переход \JZ выполнен не будет 
и функция вернет~\GTT{\_a}. 
В остальных случаях, функция вернет~\GTT{\_b}.}
\EN{Then 1 is to be stored to \AL, 
the subsequent \JZ is not to be triggered and the function will return {\_a}. 
In all other cases, {\_b} is to be returned.}
\fi
