\ifdefined\IncludeGCC
\subsubsection{GCC 4.8.1 \RU{с оптимизацией \Othree}\EN{with \Othree optimization turned on}}
\label{gcc481_o3}

\EN{Some new FPU instructions were added in the P6 Intel family}\RU{В линейке процессоров P6 от Intel 
появились новые FPU-инструкции}\footnote{\EN{Starting at}\RU{Начиная с} Pentium Pro, 
Pentium-II, \etc.}.
\index{x86!\Instructions!FUCOMI}
\RU{Это}\EN{These are} \TT{FUCOMI} (\RU{сравнить операнды и выставить флаги основного CPU}\EN{compare 
operands and set flags of the main CPU}) \AndENRU 
\index{x86!\Instructions!FCMOVcc}
\TT{FCMOVcc} (\RU{работает как}\EN{works like} \TT{CMOVcc}, \RU{но на регистрах FPU}\EN{but on FPU registers}).
\RU{Очевидно, разработчики GCC решили отказаться от поддержки процессоров до линейки P6 (ранние Pentium, 80486, etc{}.)}
\EN{Apparently, the maintainers of GCC decided to drop support of pre-P6 Intel CPUs (early Pentiums, 80486, etc{})}.

\RU{И кстати, FPU уже давно не отдельная часть процессора в линейке P6, так что флаги основного CPU можно
модифицировать из FPU.}
\EN{And also, the FPU is no longer separate unit in P6 Intel family, so now it is possible to modify/check flags 
of the main CPU from the FPU.}

\RU{Вот что имеем}\EN{So what we get is}:

\lstinputlisting[caption=\Optimizing GCC 4.8.1]{patterns/12_FPU/3_comparison/x86/GCC481_O3.s.\LANG}

\RU{Не совсем понимаю, зачем здесь \TT{FXCH} (поменять местами операнды).}
\EN{I'm not sure why \TT{FXCH} (swap operands) is here.}
\RU{От нее легко избавиться поменяв местами инструкции \FLD либо заменив 
\TT{FCMOVBE} (\IT{below or equal}~--- меньше или равно) на 
\TT{FCMOVA} (\IT{above}~--- больше).}
\EN{It's possible to get rid of it easily by swapping the first two \FLD instructions or by replacing 
\TT{FCMOVBE} (\IT{below or equal}) by \TT{FCMOVA} (\IT{above}).}
\RU{Должно быть, неаккуратность компилятора}\EN{Probably it's a compiler inaccuracy}.

\RU{Так что}\EN{So} \TT{FUCOMI} \EN{compares}\RU{сравнивает} \ST{0} ($a$) \AndENRU \ST{1} ($b$) 
\RU{и затем устанавливает флаги основного CPU}\EN{and then sets some flags in the main CPU}.
\TT{FCMOVBE} \RU{проверяет флаги и копирует}\EN{checks the flags and copies} \ST{1} 
(\RU{в тот момент там находится $b$}\EN{$b$ here at the moment}) \RU{в}\EN{to} 
\ST{0} (\RU{там $a$}\EN{$a$ here}) \RU{если}\EN{if} $ST0 (a) <= ST1 (b)$.
\RU{В противном случае}\EN{Otherwise} ($a>b$), \RU{она оставляет}\EN{it leaves} $a$ \InENRU \ST{0}.

\RU{Последняя}\EN{The last} \FSTP \RU{оставляет содержимое}\EN{leaves} \ST{0} 
\RU{на вершине стека, выбрасывая содержимое}\EN{on top of the stack, discarding the contents of} \ST{1}.

\ifdefined\IncludeGDB
\RU{Попробуем оттрасировать функцию в}\EN{Let's trace this function in} GDB:

\lstinputlisting[caption=\Optimizing GCC 4.8.1 and GDB,numbers=left]{patterns/12_FPU/3_comparison/x86/gdb.txt}

\RU{Используя}\EN{Using} \q{ni}, \RU{дадим первым двум инструкциям \FLD исполниться.}
\EN{let's execute the first two \FLD instructions.}

\RU{Посмотрим регистры FPU}\EN{Let's examine the FPU registers} (\LineENRU 33).

\RU{Как я писал ранее, регистры FPU это скорее кольцевой буфер, нежели стек}
\EN{As I wrote before, the FPU registers set is a circular buffer rather than a stack} (\myref{FPU_is_rather_circular_buffer}).
\RU{И}\EN{And} GDB \RU{показывает не регистры}\EN{doesn't show} \TT{STx}\RU{, а внутренние регистры FPU}
\EN{registers, but internal the FPU registers} (\TT{Rx}). 
\RU{Стрелка}\EN{The arrow} (\RU{на строке}\EN{at line} 35) 
\RU{указывает на текущую вершину стека.}
\EN{points to the current top of the stack.}
\RU{Вы можете также увидеть содержимое регистра \TT{TOP} в \q{Status Word} (строка 44). Там сейчас 6, так что
вершина стека сейчас указывает на внутренний регистр 6.}
\EN{You can also see the \TT{TOP} register contents in \IT{Status Word} (line 44)---it is 6 now, 
so the stack top is now pointing to internal register 6.}

\RU{Значения $a$ и $b$ меняются местами после исполнения \TT{FXCH}}\EN{The values of $a$ and $b$ are swapped 
after \TT{FXCH} is executed} (\LineENRU 54).

\TT{FUCOMI} \RU{исполнилась}\EN{is executed} (\LineENRU 83). 
\RU{Посмотрим флаги}\EN{Let's see the flags}: \CF \RU{выставлен}\EN{is set} (\LineENRU 95).

\TT{FCMOVBE} \RU{действительно скопировал значение $b$ (см. строку 104).}
\EN{has copied the value of $b$ (see line 104).}

\FSTP \RU{оставляет одно значение на вершине стека}\EN{leaves one value at the top of stack} (\LineENRU 136). 
\RU{Значение \TT{TOP} теперь 7, так что вершина FPU-стека указывает на внутренний регистр 7}\EN{The value of \TT{TOP} is 
now 7, so the FPU stack top is pointing to internal register 7}.
\fi
\fi
