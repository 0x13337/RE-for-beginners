\subsubsection{\Optimizing GCC 4.4.1}

\lstinputlisting[caption=\Optimizing GCC 4.4.1]{patterns/12_FPU/3_comparison/x86/GCC_O3.asm.\LANG}

\index{x86!\Instructions!JA}
\RU{Почти все что здесь есть уже описано мною, кроме одного: использование \JA после \SAHF. 
Действительно, инструкции условных переходов ``больше'', ``меньше'', ``равно'' для сравнения беззнаковых чисел 
(а это \JA, \JAE, \JB, \JBE, \JE/\JZ, \JNA, \JNAE, \JNB, \JNBE, \JNE/\JNZ) проверяют только флаги \CF и \ZF.}
\EN{It is almost the same except that \JA is used after \SAHF. 
Actually, conditional jump instructions that check ``larger'', ``lesser'' or ``equal'' for unsigned number comparison 
(these are \JA, \JAE, \JB, \JBE, \JE/\JZ, \JNA, \JNAE, \JNB, \JNBE, \JNE/\JNZ) check only flags \CF and \ZF.}\\
\\
\EN{Let's recall where bits \CThreeBits are located in the \TT{AH} register after the execution of \TT{FSTSW}/\FNSTSW:}
\RU{Вспомним, как биты \CThreeBits располагаются в регистре \TT{AH} после исполнения \TT{FSTSW}/\FNSTSW:}

\begin{center}
\ifdefined\ebook
\begin{bytefield}[endianness=big,bitwidth=0.06\linewidth]{8}
\else
\begin{bytefield}[endianness=big,bitwidth=0.03\linewidth]{8}
\fi
\bitheader{6,2,1,0} \\
\bitbox{1}{} & 
\bitbox{1}{\TT{C3}} & 
\bitbox{3}{} & 
\bitbox{1}{\TT{C2}} & 
\bitbox{1}{\TT{C1}} & 
\bitbox{1}{\TT{C0}}
\end{bytefield}
\end{center}


\RU{Вспомним также, как располагаются биты из \TT{AH} во флагах CPU после исполнения \SAHF:}
\EN{Let's also recall, how the bits from \TT{AH} are stored into the CPU flags the execution of \SAHF:}

\begin{center}
\begin{bytefield}[endianness=big,bitwidth=0.03\linewidth]{8}
\bitheader{7,6,4,2,0} \\
\bitbox{1}{SF} & 
\bitbox{1}{ZF} & 
\bitbox{1}{} & 
\bitbox{1}{AF} & 
\bitbox{1}{} & 
\bitbox{1}{PF} & 
\bitbox{1}{} & 
\bitbox{1}{CF}
\end{bytefield}
\end{center}


\RU{Биты \Cthree и \Czero после сравнения перекладываются в флаги \ZF и \CF так, 
что перечисленные инструкции переходов могут работать. 
\JA сработает если \CF и \ZF обнулены.}
\EN{After the comparison, the \Cthree and \Czero bits are moved into \ZF and \CF,
so the conditional jumps will able to work here. 
\JA will work if both \CF are \ZF zero.}

\RU{Таким образом, перечисленные инструкции условного перехода можно использовать после инструкций \FNSTSW/\SAHF.}
\EN{Thereby, the conditional jumps instructions listed here can be used after a \FNSTSW/\SAHF instruction pair.}

\RU{Вполне возможно, что биты статуса FPU \CThreeBits преднамеренно были размещены таким образом, 
чтобы переноситься на базовые флаги процессора без перестановок. Хотя я не уверен.}
\EN{Apparently, the FPU \CThreeBits status bits were placed there intentionally, 
to easily map them to base CPU flags without additional permutations. But I'm not sure.}
