\subsubsection{\Optimizing GCC 4.4.1}

\lstinputlisting[caption=\Optimizing GCC 4.4.1]{patterns/12_FPU/3_comparison/x86/GCC_O3_\LANG.asm}

\index{x86!\Instructions!JA}
\RU{Почти все что здесь есть уже описано мною, кроме одного: использование \JA после \SAHF. 
Действительно, инструкции условных переходов ``больше'', ``меньше'', ``равно'' для сравнения беззнаковых чисел 
(а это \JA, \JAE, \JB, \JBE, \JE/\JZ, \JNA, \JNAE, \JNB, \JNBE, \JNE/\JNZ) проверяют только флаги \CF и \ZF.}
\EN{It is almost the same except one: \JA is used after \SAHF. 
Actually, conditional jump instructions checking ``larger'', ``lesser'' or ``equal'' for unsigned number comparison 
(these are \JA, \JAE, \JB, \JBE, \JE/\JZ, \JNA, \JNAE, \JNB, \JNBE, \JNE/\JNZ) are checking only \CF and \ZF flags.}\\
\\
\EN{Let's recall how \CThreeBits bits are located in the \TT{AH} register after \TT{FSTSW}/\FNSTSW execution:}
\RU{Вспомним, как биты \CThreeBits располагаются в регистре \TT{AH} после исполнения \TT{FSTSW}/\FNSTSW:}

\begin{center}
\ifdefined\ebook
\begin{bytefield}[endianness=big,bitwidth=0.06\linewidth]{8}
\else
\begin{bytefield}[endianness=big,bitwidth=0.03\linewidth]{8}
\fi
\bitheader{6,2,1,0} \\
\bitbox{1}{} & 
\bitbox{1}{\TT{C3}} & 
\bitbox{3}{} & 
\bitbox{1}{\TT{C2}} & 
\bitbox{1}{\TT{C1}} & 
\bitbox{1}{\TT{C0}}
\end{bytefield}
\end{center}


\RU{Вспомним также, как располагаются биты из AH во флагах CPU после исполнения \SAHF:}
\EN{Let's also recall, how bits from AH are stored into CPU flags after \SAHF execution:}

\begin{center}
\begin{bytefield}[endianness=big,bitwidth=0.03\linewidth]{8}
\bitheader{7,6,4,2,0} \\
\bitbox{1}{SF} & 
\bitbox{1}{ZF} & 
\bitbox{1}{} & 
\bitbox{1}{AF} & 
\bitbox{1}{} & 
\bitbox{1}{PF} & 
\bitbox{1}{} & 
\bitbox{1}{CF}
\end{bytefield}
\end{center}


\RU{Биты \Cthree и \Czero после сравнения перекладываются в флаги \ZF и \CF так, 
что перечисленные инструкции переходов могут работать. 
\JA сработает если \CF и \ZF обнулены.}
\EN{\Cthree and \Czero bits after comparison are moving into \ZF and \CF flags exactly,
so conditional jumps will able to work here. 
\JA will work if both \CF are \ZF zero.}

\RU{Таким образом, перечисленные инструкции условного перехода можно использовать после инструкций \FNSTSW/\SAHF.}
\EN{Thereby, conditional jumps instructions listed here can be used after \FNSTSW/\SAHF instructions pair.}

\RU{Вполне возможно, что биты статуса FPU \CThreeBits преднамеренно были размещены таким образом, 
чтобы переноситься на базовые флаги процессора без перестановок. Хотя я не уверен.}
\EN{Apparently, FPU \CThreeBits status bits was placed there intentionally, 
so to easily map them to base CPU flags without additional permutations. But I'm not sure.}
