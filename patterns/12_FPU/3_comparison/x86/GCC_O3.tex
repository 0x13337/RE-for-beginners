\subsubsection{GCC 4.4.1 \RU{с оптимизацией \Othree}\EN{with \Othree optimization turned on}}

\lstinputlisting[caption=\Optimizing GCC 4.4.1]{patterns/12_FPU/3_comparison/x86/GCC_O3_\LANG.asm}

\RU{Почти все что здесь есть уже описано мною, кроме одного: использование \JA после \SAHF. 
Действительно, инструкции условных переходов ``больше'', ``меньше'', ``равно'' для сравнения беззнаковых чисел 
(\JA, \JAE, \JB, \JBE, \JE/\JZ, \JNA, \JNAE, \JNB, \JNBE, \JNE/\JNZ) проверяют только флаги \CF и \ZF. 
И биты \CThreeBits после сравнения перекладываются в эти флаги аккурат так, 
чтобы перечисленные инструкции переходов могли работать. \JA сработает если \CF и \ZF обнулены.}
\EN{It is almost the same except one: \JA usage instead of \SAHF. 
Actually, conditional jump instructions checking ``larger'', ``lesser'' or ``equal'' for unsigned number comparison 
(\JA, \JAE, \JB, \JBE, \JE/\JZ, \JNA, \JNAE, \JNB, \JNBE, \JNE/\JNZ) are checking only \CF and \ZF flags. 
And \CThreeBits bits after comparison are moving into these flags exactly in the same fashion 
so conditional jumps will work here. \JA will work if both \CF are \ZF zero.}

\RU{Таким образом, перечисленные инструкции условного перехода можно использовать после инструкций \FNSTSW/\SAHF.}
\EN{Thereby, conditional jumps instructions listed here can be used after \FNSTSW/\SAHF instructions pair.}

\RU{Вполне возможно, что биты статуса FPU \CThreeBits преднамеренно были размещены таким образом, 
чтобы переноситься на базовые флаги процессора без перестановок.}
\EN{Apparently, FPU \CThreeBits status bits was placed there intentionally, 
so to easily map them to base CPU flags without additional permutations.}
