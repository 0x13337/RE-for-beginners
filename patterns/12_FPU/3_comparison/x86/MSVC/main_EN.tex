\subsubsection{\NonOptimizing MSVC}

MSVC 2010 generates the following:

\lstinputlisting[caption=\NonOptimizing MSVC 2010]{patterns/12_FPU/3_comparison/x86/MSVC/MSVC_EN.asm}

\myindex{x86!\Instructions!FLD}

So, \FLD loads \GTT{\_b} into \ST{0}.

\label{Czero_etc}
\newcommand{\Czero}{\GTT{C0}\xspace}
\newcommand{\Ctwo}{\GTT{C2}\xspace}
\newcommand{\Cthree}{\GTT{C3}\xspace}
\newcommand{\CThreeBits}{\Cthree/\Ctwo/\Czero}

\myindex{x86!\Instructions!FCOMP}

\FCOMP compares the value in \ST{0} with what is in \GTT{\_a} 
and sets \CThreeBits bits in FPU status word register, accordingly. 
This is a 16-bit register that reflects the current state of the FPU.

After the bits are set, the \FCOMP instruction also pops one variable from the stack. 
This is what distinguishes it from \FCOM, which is just compares values, leaving the stack in the same state.

Unfortunately, CPUs before Intel P6
\footnote{Intel P6 is Pentium Pro, Pentium II, etc} don't have any conditional 
jumps instructions which check the \CThreeBits bits. 
Probably, it is a matter of history (remember: FPU was separate chip in past).
Modern CPU starting at Intel P6 have \FCOMI/\FCOMIP/\FUCOMI/\FUCOMIP 
instructions~---which do the same, but modify the \ZF/\PF/\CF CPU flags.

\myindex{x86!\Instructions!FNSTSW}

The \FNSTSW instruction copies FPU the status word register to \AX. 
\CThreeBits bits are placed at positions 14/10/8, 
they are at the same positions in the \AX register and all they are placed in the high part of \AX{}~---\AH{}.

\begin{itemize}
\item If $b>a$ in our example, then \CThreeBits bits are to be set as following: 0, 0, 0.
\item If $a>b$, then the bits are: 0, 0, 1.
\item If $a=b$, then the bits are: 1, 0, 0.
\item

If the result is unordered (in case of error), then the set bits are: 1, 1, 1.
\end{itemize}
% TODO: table here?

This is how \CThreeBits bits are located in the \AX register:

\begin{center}
\begin{bytefield}[endianness=big,bitwidth=0.03\linewidth]{16}
\bitheader{14,10,9,8} \\
\bitbox{1}{} & 
\bitbox{1}{\TT{C3}} & 
\bitbox{3}{} & 
\bitbox{1}{\TT{C2}} & 
\bitbox{1}{\TT{C1}} & 
\bitbox{1}{\TT{C0}} &
\bitbox{8}{}
\end{bytefield}
\end{center}


This is how \CThreeBits bits are located in the \AH register:

\begin{center}
\ifdefined\ebook
\begin{bytefield}[endianness=big,bitwidth=0.06\linewidth]{8}
\else
\begin{bytefield}[endianness=big,bitwidth=0.03\linewidth]{8}
\fi
\bitheader{6,2,1,0} \\
\bitbox{1}{} & 
\bitbox{1}{\TT{C3}} & 
\bitbox{3}{} & 
\bitbox{1}{\TT{C2}} & 
\bitbox{1}{\TT{C1}} & 
\bitbox{1}{\TT{C0}}
\end{bytefield}
\end{center}


After the execution of \INS{test ah, 5}\footnote{5=101b}, 
only \Czero and \Ctwo bits (on 0 and 2 position) are considered, all other bits are just
ignored.

\label{parity_flag}
\myindex{x86!\Registers!\Flags!Parity flag}

Now let's talk about the \IT{parity flag}, another notable historical rudiment.

This flag is set to 1 if the number of ones in the result of the last calculation is even, and to 0 if it is odd.

Let's look into Wikipedia\footnote{\href{http://go.yurichev.com/17131}{wikipedia.org/wiki/Parity\_flag}}:

\begin{framed}
\begin{quotation}
One common reason to test the parity flag actually has nothing to do with parity. The FPU has four condition flags 
(C0 to C3), but they can not be tested directly, and must instead be first copied to the flags register. 
When this happens, C0 is placed in the carry flag, C2 in the parity flag and C3 in the zero flag. 
The C2 flag is set when e.g. incomparable floating point values (NaN or unsupported format) are compared 
with the FUCOM instructions.
\end{quotation}
\end{framed}

As noted in Wikipedia, the parity flag used sometimes in FPU code, let's see how.

\myindex{x86!\Instructions!JP}

The \PF flag is to be set to 1 if both \Czero and \Ctwo are set to 0 or both are 1, in which case
the subsequent \JP (\IT{jump if PF==1}) is triggering. 
If we recall the values of \CThreeBits for various cases,
we can see that the conditional jump 
\JP is triggering in two cases: if $b>a$ or $a=b$ 
(\Cthree bit is not considered here, since it was cleared by 
the \INS{test ah, 5} instruction).

It is all simple after that. 
If the conditional jump was triggered, 
\FLD loads the value of \GTT{\_b} 
in \ST{0}, and if it was not triggered, the value of \GTT{\_a} is loaded there.

\myparagraph{And what about checking \Ctwo?}

The \Ctwo flag is set in case of error (\gls{NaN}, \etc{}), but our code doesn't check it.

If the programmer cares about FPU errors, he/she must add additional checks.

\clearpage
\myparagraphold{First \olly example: a=1.2 and b=3.4}
\myindex{\olly}

Let's load the example into \olly:

\begin{figure}[H]
\centering
\myincludegraphics{patterns/12_FPU/3_comparison/x86/MSVC/olly1_1.png}
\caption{\olly: first \FLD is executed}
\label{fig:FPU_comparison_case1_olly1}
\end{figure}

Current arguments of the function: $a=1.2$ and $b=3.4$ (We can see them in the stack: two pairs of 32-bit values).
$b$ (3.4) is already loaded in \ST{0}.
Now \FCOMP is being executed. 
\olly shows the second \FCOMP argument, which is in stack right now.

\clearpage
\FCOMP is executed:

\begin{figure}[H]
\centering
\myincludegraphics{patterns/12_FPU/3_comparison/x86/MSVC/olly1_2.png}
\caption{\olly: \FCOMP is executed}
\label{fig:FPU_comparison_case1_olly2}
\end{figure}

We see the state of the \ac{FPU}'s condition flags: all zeroes.
The popped value is reflected as \ST{7}, it was written earlier about reason for this: 
\myref{FPU_is_rather_circular_buffer}.

\clearpage
\FNSTSW is executed:
\begin{figure}[H]
\centering
\myincludegraphics{patterns/12_FPU/3_comparison/x86/MSVC/olly1_3.png}
\caption{\olly: \FNSTSW is executed}
\label{fig:FPU_comparison_case1_olly3}
\end{figure}

We see that the \GTT{AX} register contain zeroes: indeed, all condition flags are zero.
(\olly disassembles the \FNSTSW instruction as \INS{FSTSW}---they are synonyms).

\clearpage
\TEST is executed:

\begin{figure}[H]
\centering
\myincludegraphics{patterns/12_FPU/3_comparison/x86/MSVC/olly1_4.png}
\caption{\olly: \TEST is executed}
\label{fig:FPU_comparison_case1_olly4}
\end{figure}

The \GTT{PF} flag is set to 1.

Indeed: the number of bits set in 0 is 0 and 0 is an even number.
\olly disassembles \INS{JP} as \ac{JPE}---they are synonyms.
And it is about to trigger now.

\clearpage
\ac{JPE} triggered, \FLD loads the value of $b$ (3.4) in \ST{0}:

\begin{figure}[H]
\centering
\myincludegraphics{patterns/12_FPU/3_comparison/x86/MSVC/olly1_5.png}
\caption{\olly: second \FLD is executed}
\label{fig:FPU_comparison_case1_olly5}
\end{figure}

The function finishes its work.

\clearpage
\myparagraphold{Second \olly example: a=5.6 and b=-4}

Let's load example into \olly:

\begin{figure}[H]
\centering
\myincludegraphics{patterns/12_FPU/3_comparison/x86/MSVC/olly2_1.png}
\caption{\olly: first \FLD executed}
\label{fig:FPU_comparison_case2_olly1}
\end{figure}

Current function arguments: $a=5.6$ and $b=-4$.
$b$ (-4) is already loaded in \ST{0}.
\FCOMP about to execute now. 
\olly shows the second \FCOMP argument, which is in stack right now.

\clearpage
\FCOMP executed:

\begin{figure}[H]
\centering
\myincludegraphics{patterns/12_FPU/3_comparison/x86/MSVC/olly2_2.png}
\caption{\olly: \FCOMP executed}
\label{fig:FPU_comparison_case2_olly2}
\end{figure}

We see the state of the \ac{FPU}'s condition flags: all zeroes except \Czero.

\clearpage
\FNSTSW executed:

\begin{figure}[H]
\centering
\myincludegraphics{patterns/12_FPU/3_comparison/x86/MSVC/olly2_3.png}
\caption{\olly: \FNSTSW executed}
\label{fig:FPU_comparison_case2_olly3}
\end{figure}

We see that the \GTT{AX} register contains \GTT{0x100}: the \Czero flag is at the 16th bit.

\clearpage
\TEST executed:

\begin{figure}[H]
\centering
\myincludegraphics{patterns/12_FPU/3_comparison/x86/MSVC/olly2_4.png}
\caption{\olly: \TEST executed}
\label{fig:FPU_comparison_case2_olly4}
\end{figure}

The \GTT{PF}  flag is cleared.
Indeed: 

the count of bits set in \GTT{0x100} is 1 and 1 is an odd number.
\ac{JPE} is being skipped now.

\clearpage
\ac{JPE} wasn't triggered, so \FLD loads the value of $a$ (5.6) in \ST{0}:

\begin{figure}[H]
\centering
\myincludegraphics{patterns/12_FPU/3_comparison/x86/MSVC/olly2_5.png}
\caption{\olly: second \FLD executed}
\label{fig:FPU_comparison_case2_olly5}
\end{figure}

The function finishes its work.

