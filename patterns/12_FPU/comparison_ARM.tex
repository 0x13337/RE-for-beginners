\subsubsection{ARM + \OptimizingXcode + \ARMMode}

\begin{lstlisting}[caption=\OptimizingXcode + \ARMMode]
VMOV            D16, R2, R3 ; b
VMOV            D17, R0, R1 ; a
VCMPE.F64       D17, D16
VMRS            APSR_nzcv, FPSCR
VMOVGT.F64      D16, D17 ; copy b to D16
VMOV            R0, R1, D16
BX              LR
\end{lstlisting}

\index{ARM!\Registers!APSR}
\index{ARM!\Registers!FPSCR}
\IFRU{Очень простой случай.}{A very simple case.}
\IFRU{Входные величины помещаются в}{Input values are placed into the} \TT{D17} \AndENRU \TT{D16} 
\IFRU{и сравниваются при помощи инструкции}{registers and then compared with the help of} 
\TT{VCMPE}\IFRU{}{ instruction}.
\IFRU{Как и в сопроцессорах x86, сопроцессор в ARM имеет свой собственный регистр статуса и флагов}
{Just like in x86 coprocessor, ARM coprocessor has its own status and flags register}, (\TT{FPSCR}),
\IFRU{потому как есть необходимость хранить специфичные для его работы флаги.}
{since there is a need to store coprocessor-specific flags.}

\index{ARM!\Instructions!VMRS}
\IFRU{И так же как и в x86}{And just like in x86}, 
\IFRU{в ARM нет инструкций условного перехода}
{there are no conditional jump instruction in ARM}, 
\IFRU{проверяющих биты в регистре статуса сопроцессора}{checking bits in coprocessor status register}, 
\IFRU{так что имеется инструкция}{so there is} \TT{VMRS}
\IFRU{, копирующая 4 бита}{ instruction, copying 4 bits} (N, Z, C, V) 
\IFRU{из статуса сопроцессора в биты \IT{общего} статуса (регистр \TT{APSR}).}
{from the coprocessor status word into bits of \IT{general} status (\TT{APSR} register).}

\index{ARM!\Instructions!VMOVGT}
\TT{VMOVGT} \IFRU{это аналог}{is analogue of} \TT{MOVGT}, 
\IFRU{инструкция, сработающая если при сравнении один операнд был больше чем второй}
{instruction, to be executed if one operand is greater than other while comparing} 
(\IT{GT\IFRU{ ~--- }{~---}Greater Than}). 

\IFRU{Если она сработает}{If it will be executed}, 
\IFRU{в \TT{D16} запишется значение $b$}{$b$ value will be written into \TT{D16}}, 
\IFRU{лежащее в тот момент в}{stored at the moment in} \TT{D17}.

\IFRU{А если не сработает}{And if it will not be triggered}, 
\IFRU{то в \TT{D16} останется лежать значение $a$.}
{then $a$ value will stay in the \TT{D16} register.}

\index{ARM!\Instructions!VMOV}
\IFRU{Предпоследняя инструкция \TT{VMOV} подготовит то что было в \TT{D16} для возврата через 
пару регистров \Rzero и \Rone.}
{Penultimate instruction \TT{VMOV} will prepare value in the \TT{D16} register for returning via \Rzero and \Rone
registers pair.}

\subsubsection{ARM + \OptimizingXcode + \ThumbTwoMode}

\begin{lstlisting}[caption=\OptimizingXcode + \ThumbTwoMode]
VMOV            D16, R2, R3 ; b
VMOV            D17, R0, R1 ; a
VCMPE.F64       D17, D16
VMRS            APSR_nzcv, FPSCR
IT GT 
VMOVGT.F64      D16, D17
VMOV            R0, R1, D16
BX              LR
\end{lstlisting}

\IFRU{Почти то же самое что и в предыдущем примере, за парой отличий.}
{Almost the same as in previous example, howeverm slightly different.}
\IFRU{Дело в том, многие инструкции в режиме ARM
можно дополнять условием, которое если справедливо, то инструкция выполнится.}
{As a matter of fact, many instructions in ARM mode can be supplied by condition predicate,
and the instruction is to be executed if condition is true.}

\IFRU{Но в режиме thumb такого нет}
{But there is no such thing in thumb mode}. 
\IFRU{В 16-битных инструкций просто нет места для лишних 4 битов, при помощи
которых можно было бы закодировать условие выполнения.}
{There is no place in 16-bit instructions for spare 4 bits where condition can be encoded.}

\index{ARM!\ThumbTwoMode}
\IFRU{Поэтому в thumb-2 добавили возможность дополнять thumb-инструкции условиями.}
{However, thumb-2 was extended to make possible to specify predicates to old thumb instructions.}

\IFRU{Здесь, в листинге сгенерированном при помощи \IDA, мы видим инструкцию \TT{VMOVGT}, 
такую же как и в предыдущем примере.}
{Here, is the \IDA-generated listing, we see \TT{VMOVGT} instruction, the same as in previous example.}

\IFRU{Но в реальности}{But in fact}, 
\IFRU{там закодирована обычная инструкция \TT{VMOV}}
{usual \TT{VMOV} is encoded there}, 
\IFRU{просто \IDA добавила суффикс \TT{-GT} к ней}
{but \IDA added \TT{-GT} suffix to it}, 
\IFRU{потому что перед этой инструкцией стоит \TT{``IT GT''}}
{since there is \TT{``IT GT''} instruction placed right before}.

\index{ARM!\Instructions!IT}
\index{ARM!if-then block}
\IFRU{Инструкция }\TT{IT} \IFRU{определяет так называемый}{instruction defines so-called} \IT{if-then block}. 
\IFRU{После этой инструкции, можно указывать до четырех инструкций, к которым будет добавлен суффикс условия.}
{After the instruction, it is possible to place up to 4 instructions, to which predicate suffix will be added.}
\IFRU{В нашем примере}{In our example}, \TT{``IT GT''} \IFRU{означает}{meaning},
\IFRU{что следующая за ней инструкция будет исполнена}{the next instruction will be executed}, 
\IFRU{если условие}{if} \IT{GT} (\IT{Greater Than}) \IFRU{справедливо}{condition is true}.

\index{Angry Birds}
\IFRU{Теперь более сложный пример, кстати, из}{Now more complex code fragment, by the way, from} 
``Angry Birds'' (\IFRU{для}{for} iOS):

\begin{lstlisting}[caption=Angry Birds Classic]
ITE NE
VMOVNE          R2, R3, D16
VMOVEQ          R2, R3, D17
\end{lstlisting}

\TT{ITE} \IFRU{означает}{meaning} \IT{if-then-else} 
\IFRU{и кодирует суффиксы для двух следующих за ней инструкций.}
{and it encode suffixes for two next instructions.}
\IFRU{Первая из них исполнится, если условие закодированное в}
{First instruction will execute if condition encoded in} \TT{ITE} (\IT{NE, not equal}) 
\IFRU{будет в тот момент справедливо}{will be true at the moment},
\IFRU{а вторая ~--- если это условие не сработает}{and the 
second~---if the condition will not be true}.
(\IFRU{Обратное условие от}{Inverse condition of} \TT{NE} \IFRU{это}{is} \TT{EQ} (\IT{equal})).

\index{Angry Birds}
\IFRU{Еще чуть сложнее}{Slightly harder}, \IFRU{и снова этот фрагмент из}{and this fragment from} 
``Angry Birds''\IFRU{}{ as well}:

\begin{lstlisting}[caption=Angry Birds Classic]
ITTTT EQ
MOVEQ           R0, R4
ADDEQ           SP, SP, #0x20
POPEQ.W         {R8,R10}
POPEQ           {R4-R7,PC}
\end{lstlisting}

\IFRU{4 символа ``T'' в инструкции означают что 4 следующие инструкции будут исполнены если условие соблюдается.}
{4 ``T'' symbols in instruction mnemonic means 
the 4 next instructions will be executed if condition is true.}
\IFRU{Поэтому \IDA добавила ко всем четырем инструкциям суффикс}
{That's why \IDA added} \TT{-EQ}\IFRU{}{ suffix
to each 4 instructions}. 

\IFRU{А если бы здесь было, например,}{And if there will be e.g.}
\TT{ITEEE EQ} (\IT{if-then-else-else-else}), 
\IFRU{тогда суффиксы для следующих четырех инструкций были бы расставлены так:}
{then suffixes will be set as follows:}

\begin{lstlisting}
-EQ
-NE
-NE
-NE
\end{lstlisting}

\index{Angry Birds}
\IFRU{Еще фрагмент из}{Another fragment from} ``Angry Birds'':

\begin{lstlisting}[caption=Angry Birds Classic]
CMP.W           R0, #0xFFFFFFFF
ITTE LE
SUBLE.W         R10, R0, #1
NEGLE           R0, R0
MOVGT           R10, R0
\end{lstlisting}

\TT{ITTE} (\IT{if-then-then-else}) \IFRU{означает что первая и вторая инструкции исполнятся}
{means the 1st and 2nd instructions will be executed}, 
\IFRU{если условие}{if} \TT{LE} (\IT{Less or Equal}) \IFRU{справедливо}{condition is true},
\IFRU{а третья}{and 3rd}\IFRU{ ~--- }{~---}\IFRU{если справедливо обратное условие}{if inverse condition} 
(\TT{GT}\IFRU{ ~--- }{~---}\IT{Greater Than})\IFRU{}{ is true}.

\IFRU{Компиляторы способны генерировать далеко не все варианты.}
{Compilers usually are not generating all possible combinations.}
\index{Angry Birds}
\IFRU{Например, в вышеупомянутой игре ``Angry Birds'' (версия \IT{classic} для iOS)}
{For example, it mentioned ``Angry Birds'' game (\IT{classic} version for iOS)}
\IFRU{попадаются только такие варианты инструкции \TT{IT}}{only these cases of \TT{IT} instruction are used}: 
\TT{IT}, \TT{ITE}, \TT{ITT}, \TT{ITTE}, \TT{ITTT}, \TT{ITTTT}.
\index{\GrepUsage}
\IFRU{Как я это узнал?}{How I learnt this?}
\IFRU{В \IDA можно сгенерировать листинг, так я и сделал, только в опциях я установил так 
чтобы показывались 4 байта для каждого опкода.}
{In \IDA it is possible to produce listing files, so I did it, but I also set in options 
to show 4 bytes of each opcodes .}
\IFRU{Затем, зная что старшая часть 16-битного опкода \TT{IT} это \TT{0xBF}, я сделал при помощи \TT{grep} это:}
{Then, knowing the high part of 16-bit opcode \TT{IT} is \TT{0xBF},
I did this with \TT{grep}:}

\begin{lstlisting}
cat AngryBirdsClassic.lst | grep " BF" | grep "IT" > results.lst
\end{lstlisting}

\index{ARM!\ThumbTwoMode}
\IFRU{Кстати, если писать на ассемблере для режима thumb-2 вручную, и дополнять инструкции суффиксами
условия, то ассемблер автоматически будет добавлять инструкцию \TT{IT} с соответствующими флагами, там
где надо.}
{By the way, if to program in ARM assembly language manually for thumb-2 mode, 
with adding conditional suffixes,
assembler will add \TT{IT} instructions automatically, with respectable flags, where it is necessary.}

\subsubsection{ARM + \NonOptimizingXcode + \ARMMode}

\begin{lstlisting}[caption=\NonOptimizingXcode + \ARMMode]
b               = -0x20
a               = -0x18
val_to_return   = -0x10
saved_R7        = -4

                STR             R7, [SP,#saved_R7]!
                MOV             R7, SP
                SUB             SP, SP, #0x1C
                BIC             SP, SP, #7
                VMOV            D16, R2, R3
                VMOV            D17, R0, R1
                VSTR            D17, [SP,#0x20+a]
                VSTR            D16, [SP,#0x20+b]
                VLDR            D16, [SP,#0x20+a]
                VLDR            D17, [SP,#0x20+b]
                VCMPE.F64       D16, D17
                VMRS            APSR_nzcv, FPSCR
                BLE             loc_2E08
                VLDR            D16, [SP,#0x20+a]
                VSTR            D16, [SP,#0x20+val_to_return]
                B               loc_2E10

loc_2E08
                VLDR            D16, [SP,#0x20+b]
                VSTR            D16, [SP,#0x20+val_to_return]

loc_2E10
                VLDR            D16, [SP,#0x20+val_to_return]
                VMOV            R0, R1, D16
                MOV             SP, R7
                LDR             R7, [SP+0x20+b],#4
                BX              LR
\end{lstlisting}

\IFRU{Почти то же самое что мы уже видели}{Almost the same we already saw}, 
\IFRU{но много избыточного кода из-за хранения $a$ и $b$, 
а также выходного значения, в локальном стеке.}
{but too much redundant code because of $a$ and $b$ variables storage in local stack, as well
as returning value.}

\subsubsection{ARM + \OptimizingKeil + \ThumbMode}

\begin{lstlisting}[caption=\OptimizingKeil + \ThumbMode]
                PUSH    {R3-R7,LR}
                MOVS    R4, R2
                MOVS    R5, R3
                MOVS    R6, R0
                MOVS    R7, R1
                BL      __aeabi_cdrcmple
                BCS     loc_1C0
                MOVS    R0, R6
                MOVS    R1, R7
                POP     {R3-R7,PC}

loc_1C0
                MOVS    R0, R4
                MOVS    R1, R5
                POP     {R3-R7,PC}
\end{lstlisting}

\IFRU{Keil не генерирует специальную инструкцию для сравнения чисел с плавающей запятой, потому что не 
расчитывает на то что она будет поддерживаться, а простым сравнением побитово здесь не обойтись.}
{Keil not generates special instruction for float numbers comparing since it cannot rely it will
be supported on the target CPU, and it cannot be done by straightforward bitwise comparing.}
%TODO: why?
\IFRU{Для сравнения вызывается библиотечная функция}{So there is called external library
function for comparing:} \TT{\_\_aeabi\_cdrcmple}. 
\index{ARM!\Instructions!BCS}
N.B. \IFRU{Результат
сравнения эта функция оставляет в флагах, чтобы следующая за вызовом инструкция}
{Comparison result is to be leaved in flags, so the following}
\TT{BCS} (\IT{Carry set - Greater than or equal})
\IFRU{могла работать без дополнительного кода.}{instruction may work without any additional code.}

 
