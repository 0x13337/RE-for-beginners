\subsection{ARM}

\subsubsection{\NonOptimizingKeilVI (\ARMMode)}

\lstinputlisting{patterns/13_arrays/1_simple/simple_Keil_ARM_O0.asm.\LANG}

\RU{Тип \Tint требует 32 бита для хранения (или 4 байта),}
\EN{\Tint type requires 32 bits for storage (or 4 bytes),}
\RU{так что для хранения 20 переменных типа \Tint, нужно $80$ (\TT{0x50}) байт.}
\EN{so to store 20 \Tint variables $80$ (\TT{0x50}) bytes are needed.}
\RU{Поэтому инструкция}\EN{So that is why the} \TT{``SUB SP, SP, \#0x50''} 
\RU{в прологе функции выделяет в локальном стеке под массив именно столько места.}
\EN{instruction in the function's prologie allocates exactly this amount of space in the stack.}

\RU{И в первом и во втором цикле, итератор цикла $i$ будет постоянно находится в регистре \Reg{4}.}
\EN{In both the first and second loops, the loop iterator $i$ is placed in the \Reg{4} register.}

\index{ARM!Optional operators!LSL}
\RU{Число, которое нужно записать в массив, вычисляется так: $i*2$, и это эквивалентно 
сдвигу на 1 бит влево, так что инструкция \TT{``MOV R0, R4,LSL\#1''} делает это.}
\EN{The number that is to be written into the array is calculated as $i*2$, which is effectively equivalent 
to shifting it left by one bit, so \TT{``MOV R0, R4,LSL\#1''} instruction does this.}

\index{ARM!\Instructions!STR}
\TT{``STR R0, [SP,R4,LSL\#2]''} \RU{записывает содержимое \Reg{0} в массив}\EN{writes the contents of \Reg{0} into the array}.
\RU{Указатель на элемент массива вычисляется так: \ac{SP} указывает на начало массива, \Reg{4} это $i$.}
\EN{Here is how a pointer to array element is calculated: \ac{SP} points to the start of the array, \Reg{4} is $i$.}
\RU{Так что сдвигаем $i$ на 2 бита влево, что эквивалентно умножению на $4$ 
(ведь каждый элемент массива занимает 4 байта) и прибавляем это к адресу начала массива.}
\EN{So shifting $i$ left by 2 bits is effectively equivalent to multiplication by $4$
(since each array element has a size of 4 bytes) and then it's added to the address of the start of the array.}

\index{ARM!\Instructions!LDR}
\RU{Во втором цикле используется обратная инструкция \TT{``LDR R2, [SP,R4,LSL\#2]''},
она загружает из массива нужное значение, и указатель на него вычисляется точно так же.}
\EN{The second loop has an inverse \TT{``LDR R2, [SP,R4,LSL\#2]''}
instruction, it loads the value we need from the array, and the pointer to it is calculated likewise.}

\subsubsection{\OptimizingKeilVI (\ThumbMode)}

\lstinputlisting{patterns/13_arrays/1_simple/simple_Keil_thumb_O3.asm.\LANG}

\RU{Код для thumb очень похожий.}\EN{Thumb code is very similar.}
\index{ARM!\Instructions!LSLS}
\RU{В thumb имеются отдельные инструкции для битовых сдвигов (как \TT{LSLS}), 
вычисляющие и число для записи в массив и адрес каждого элемента массива.}
\EN{Thumb mode has special instructions for bit shifting (like \TT{LSLS}),
which calculates the value to be written into the array and the address of each element in the array as well.}

\RU{Компилятор почему-то выделил в локальном стеке немного больше места, 
однако последние 4 байта не используются.}
\EN{The compiler allocates slightly more space in the local stack, however, the last 4 bytes are not used.}

\subsubsection{\NonOptimizing GCC 4.9.1 (ARM64)}

\lstinputlisting[caption=\NonOptimizing GCC 4.9.1 (ARM64)]{patterns/13_arrays/1_simple/ARM64_GCC491_O0.s.\LANG}
