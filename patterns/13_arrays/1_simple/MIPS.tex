\ifx\RUSSIAN\undefined
\subsection{MIPS}
% FIXME better start at non-optimizing version?
The function uses a lot of S- registers which must be preserved, so that's why they are saved in the function
prologue and restored in epilogue.

\lstinputlisting[caption=\Optimizing GCC 4.4.5 (IDA)]{patterns/13_arrays/1_simple/MIPS_O3_IDA.lst}

Interesting thing: there are two loops and the first one doesn't needs $i$ variable, it needs only 
$i*2$ variable (leaping by 2 at each iteration) and also address in memory (leaping by 4 at each iteration).
So here we see two variables, one (in \$V0) increasing by 2 each time, and another (in \$V1) --- by 4.

The second loop is a loop where \printf is called, and it should report $i$ value to user, so there are variable
increasing by 1 each time (in \$S0) and also a memory address (in \$S1) increasing by 4 each time.

That reminds us loop optimizations we considered earlier: \ref{loop_iterators}.
Its goal is to get rid of multiplication operations.

\fi
