\subsection{MIPS}
% FIXME better start at non-optimizing version?
\RU{Ф-ция использует много S-регистров, которые должны быть сохранены, вот почему их значения сохраняются
в прологе ф-ции и восстанавливаются в эпилоге.}
\EN{The function uses a lot of S- registers which must be preserved, so that's why they its 
values are saved in the function prologue and restored in epilogue.}

\lstinputlisting[caption=\Optimizing GCC 4.4.5 (IDA)]{patterns/13_arrays/1_simple/MIPS_O3_IDA.lst.\LANG}

\RU{Интересная вещь: здесь два цикла и в первом не нужна переменная $i$, а нужна только переменная
$i*2$ (скачущая через 2 на каждой итерации) и еще адрес в пмяти (скачущий через 4 на каждой итерации).}
\EN{Interesting thing: there are two loops and the first one doesn't needs $i$ variable, it needs only 
$i*2$ variable (leaping by 2 at each iteration) and also address in memory (leaping by 4 at each iteration).}
\RU{Так что мы видим здесь две переменных, одна (в \$V0) увеличивается на 2 каждый раз, и вторая (в \$V1) --- 
на 4.}
\EN{So here we see two variables, one (in \$V0) increasing by 2 each time, and another (in \$V1) --- by 4.}

\RU{Вторй цикл это цикл где вызывается \printf, и он должен показывать значение $i$ пользователю,
так что здесь есть пеменная увеличивающаяся на 1 каждый раз (в \$S0), а также адрес в памяти (в \$S1) 
увеличивающийся на 4 каждый раз.}
\EN{The second loop is a loop where \printf is called, and it should report $i$ value to user, 
so there are variable
increasing by 1 each time (in \$S0) and also a memory address (in \$S1) increasing by 4 each time.}

\RU{Это напоминает нам оптимизацию циклов, которую мы рассматривали раннее:}
\EN{That reminds us loop optimizations we considered earlier:} \ref{loop_iterators}.
\RU{Цель оптимизации в том, чтобы избавиться от операций умножения.}
\EN{Its goal is riddance of multiplication operations.}
