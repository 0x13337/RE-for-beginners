\subsection{\OptimizingXcodeIV (\ThumbTwoMode)}

\RU{Возвращаясь к нашему простому примеру}
\EN{Let's get back to our simple array example} (\myref{arrays_simple}),
\RU{можно посмотреть, как LLVM добавит проверку \q{канарейки}:}
\EN{again, now we can see how LLVM checks the correctness of the \q{canary}:}

% TODO shorten the listing a bit? is full display of unrolled loop necessary?
\lstinputlisting{patterns/13_arrays/3_BO_protection/simple_Xcode_thumb_O3.asm.\LANG}

\index{Unrolled loop}
\RU{Во-первых, LLVM \q{развернул} цикл и все значения записываются в массив по одному, 
уже вычисленные, 
потому что LLVM посчитал что так будет быстрее.}
\EN{First of all, as we see, LLVM \q{unrolled} the loop and all values were written into an array one-by-one,
pre-calculated, as LLVM concluded it can work faster.}
\RU{Кстати, инструкции режима ARM позволяют сделать это ещё быстрее и это может быть вашим 
домашним заданием.}\EN{By the way, instructions in ARM mode may help to do this even faster, 
and finding this could be your homework.}

\RU{В конце функции мы видим сравнение \q{канареек}~--- той что лежит в локальном стеке и корректной, 
на которую ссылается регистр \Reg{8}.}
\EN{At the function end we see the comparison of the \q{canaries}---the one in the local stack and the correct one,
to which \Reg{8} points.}
\index{ARM!\Instructions!IT}
\RU{Если они равны, срабатывает блок из четырех инструкций при помощи \INS{ITTTT EQ}.
Это запись 0 в \Reg{0}, эпилог функции и выход из нее.}
\EN{If they are equal to each other, a 4-instruction block is triggered by \INS{ITTTT EQ},
which contains writing 0 in \Reg{0}, the function epilogue and exit.}
\RU{Если \q{канарейки} не равны, блок не срабатывает и происходит
переход на функцию}\EN{If the \q{canaries} are not equal, the block being skipped,
and the jump to} \TT{\_\_\_stack\_chk\_fail}\RU{, которая, вероятно, остановит работу программы.}
\EN{ function will occur, which, perhaps, will halt execution.}
% TODO1 illustrate this!
