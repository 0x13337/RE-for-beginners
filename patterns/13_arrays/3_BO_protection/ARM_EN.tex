\subsectionold{\OptimizingXcodeIV (\ThumbTwoMode)}


Let's get back to our simple array example (\myref{arrays_simple}),

again, now we can see how LLVM checks the correctness of the \q{canary}:

% TODO shorten the listing a bit? is full display of unrolled loop necessary?
\lstinputlisting{patterns/13_arrays/3_BO_protection/simple_Xcode_thumb_O3_EN.asm}

\myindex{Unrolled loop}

First of all, as we see, LLVM \q{unrolled} the loop and all values were written into an array one-by-one,
pre-calculated, as LLVM concluded it can work faster.
By the way, instructions in ARM mode may help to do this even faster, 
and finding this could be your homework.


At the function end we see the comparison of the \q{canaries}---the one in the local stack and the correct one,
to which \Reg{8} points.
\myindex{ARM!\Instructions!IT}

If they are equal to each other, a 4-instruction block is triggered by \INS{ITTTT EQ},
which contains writing 0 in \Reg{0}, the function epilogue and exit.
If the \q{canaries} are not equal, the block being skipped,
and the jump to \TT{\_\_\_stack\_chk\_fail}
 function will occur, which, perhaps will halt execution.
% TODO1 illustrate this!
