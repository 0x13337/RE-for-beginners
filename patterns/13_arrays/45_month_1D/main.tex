\section{\RU{Массив указателей на строки}\EN{Array of pointers to strings}}
\label{array_of_pointers_to_strings}

\RU{Вот также пример массива указателей.}\EN{Here is an example for an array of pointers.}

\lstinputlisting[caption=\RU{Получить имя месяца}\EN{Get month name},label=get_month1]{patterns/13_arrays/45_month_1D/month1.c.\LANG}

\subsection{x64}

\lstinputlisting[caption=\Optimizing MSVC 2013 x64]{patterns/13_arrays/45_month_1D/month1_MSVC_2013_x64_Ox.asm}

\RU{Код очень простой}\EN{The code is very simple}:

\begin{itemize}

\item
\index{x86!\Instructions!MOVSXD}
\RU{Первая инструкция MOVSXD копирует 32-битное значение из ECX (где передается аргумент $month$)
в RAX с знаковым расширением (потому что аргумент $month$ имеет тип \Tint).}
\EN{The first MOVSXD instruction copies a 32-bit value from ECX (where $month$ argument is passed) 
to RAX with sign-extension (because the $month$ argument is of type \Tint).}
\RU{Причина расширения в том, что это значение будет использоваться в вычислениях наряду с другими 64-битными
значениями.}
\EN{The reason for the sign extension is that this 32-bit value is to be used in calculations
with other 64-bit values.}
\RU{Таким образом, оно должно быть расширено до 64-битного}
\EN{Hence, it has to be promoted to 64-bit}
\footnote{\RU{Это немного странная вещь, но отрицательный индекс массива может быть передан как $month$ 
\ifx\LITE\undefined
(отрицательные индексы массивов будут рассмотрены позже: \myref{negative_array_indices})
\fi
.
И если так будет, отрицательное значение типа \Tint будет расширено со знаком корректно
и соответствующий элемент перед таблицей будет выбран.
Всё это не будет корректно работать без знакового расширения.}
\EN{It is somewhat weird, but negative array index could be passed here as $month$
\ifx\LITE\undefined
(negative array indices will have been explained later: \myref{negative_array_indices}) 
\fi
.
And if this happens, the negative input value of \Tint type is sign-extended correctly 
and the corresponding element before table is picked. 
It is not going to work correctly without sign-extension.}}.

\item
\RU{Затем адрес таблицы указателей загружается в RCX.}
\EN{Then the address of the pointer table is loaded into RCX.}

\item
\RU{В конце концов, входное значение ($month$) умножается на 8 и прибавляется к адресу.}
\EN{Finally, the input value ($month$) is multiplied by 8 and added to the address.}
\RU{Действительно: мы в 64-битной среде и все адреса (или указатели) 
требуют для хранения именно 64 бита (или 8 байт).}
\EN{Indeed: we are in a 64-bit environment and all address (or pointers) require exactly 64 bits (or 8 bytes) 
for storage.}
\RU{Следовательно, каждый элемент таблицы имеет ширину в 8 байт.}
\EN{Hence, each table element is 8 bytes wide.}
\RU{И вот почему для выбора элемента под нужным номером, нужно пропустить $month*8$ байт от начала.}
\EN{And that's why to pick a specific element, $month*8$ bytes has to be skipped from the start.}
\RU{Это то, что делает MOV}\EN{That's what MOV does}.
\RU{В дополнении, эта инструкция также загружает элемент по этому адресу.}
\EN{In addition, this instruction also loads the element at this address.}
\RU{Для 1, элемент будет указателем на строку, содержащую \q{February}, \etc{}.}
\EN{For 1, an element would be a pointer to a string that contains \q{February}, \etc{}.}

\end{itemize}

\Optimizing GCC 4.9 \RU{может это сделать даже лучше}\EN{can do the job even better}
\footnote{\RU{В листинге осталось \q{0+}, потому что вывод ассемблера GCC не так скрупулёзен, чтобы убрать это}
\EN{\q{0+} was left in the listing because GCC assembler output is not tidy enough to eliminate it}.
\RU{Это \IT{displacement} и он здесь нулевой.}\EN{It's \IT{displacement}, and it's zero here.}}:

\begin{lstlisting}[caption=\Optimizing GCC 4.9 x64]
	movsx	rdi, edi
	mov	rax, QWORD PTR month1[0+rdi*8]
	ret
\end{lstlisting}

\subsubsection{32-bit MSVC}

\RU{Скомпилируем также в 32-битном компиляторе MSVC:}\EN{Let's also compile it in the 32-bit MSVC compiler:}

\lstinputlisting[caption=\Optimizing MSVC 2013 x86]{patterns/13_arrays/45_month_1D/month1_MSVC_2013_x86_Ox.asm}

\RU{Входное значение не нужно расширять до 64-битного значения, так что оно используется как есть.}
\EN{The input value does not need to be extended to 64-bit value, so it is used as is.}
\RU{И оно умножается на 4, потому что элементы таблицы имеют ширину 32 бита или 4 байта.}
\EN{And it's multiplied by 4, because the table elements are 32-bit (or 4 bytes) wide.}

\ifdefined\IncludeARM
% FIXME1 move to another file
\subsection{32-\RU{битный}\EN{bit} ARM}

\subsubsection{ARM \RU{в режиме ARM}\EN{in ARM mode}}

\lstinputlisting[caption=\OptimizingKeilVI (\ARMMode)]{patterns/13_arrays/45_month_1D/month1_Keil_ARM_O3.s}

\RU{Адрес таблицы загружается в R1.}
\EN{The address of the able is loaded in R1.}
\index{ARM!\Instructions!LDR}
\RU{Всё остальное делается, используя только одну инструкцию LDR.}
\EN{All the rest is done using just one LDR instruction.}
\RU{Входное значение $month$ сдвигается влево на 2 (что тоже самое что и умножение на 4), это значение
прибавляется к R1 (где находится адрес таблицы) и затем элемент таблицы загружается по этому адресу.}
\EN{Then input value $month$ is shifted left by 2 (which is the same as multiplying by 4), then added
to R1 (where the address of the table is) and then a table element is loaded from this address.}
\RU{32-битный элемент таблицы загружается в R0 из таблицы.}
\EN{The 32-bit table element is loaded into R0 from the table.}

\subsubsection{ARM \RU{в режиме Thumb}\EN{in Thumb mode}}

\RU{Код почти такой же, только менее плотный, потому что здесь, в инструкции LDR, нельзя задать суффикс LSL:}
\EN{The code is mostly the same, but less dense, because the LSL suffix cannot be specified in the LDR instruction here:}

\begin{lstlisting}
get_month1 PROC
        LSLS     r0,r0,#2
        LDR      r1,|L0.64|
        LDR      r0,[r1,r0]
        BX       lr
        ENDP
\end{lstlisting}

\subsection{ARM64}

\lstinputlisting[caption=\Optimizing GCC 4.9 ARM64]{patterns/13_arrays/45_month_1D/month1_GCC49_ARM64_O3.s}

\index{ARM!\Instructions!ADRP/ADD pair}
\RU{Адрес таблицы загружается в X1 используя пару ADRP/ADD.}
\EN{The address of the table is loaded in X1 using ADRP/ADD pair.}
\RU{Соответствующий элемент выбирается используя одну инструкцию LDR, которая берет W0
(регистр где находится значение входного аргумента $month$), сдвигает его на 3 бита влево
(что то же самое что и умножение на 8),
расширяет его учитывая знак (это то что означает суффикс \q{sxtw}) и прибавляет к X0.}
\EN{Then corresponding element is picked using just one LDR, which takes W0 
(the register where input argument $month$ is), shifts it 3 bits to the left (which is the same as multiplying by 8), 
sign-extends it (this is what \q{sxtw} suffix implies) and adds to X0.}
\RU{Затем 64-битное значение загружается из таблицы в X0.}\EN{Then the 64-bit value is loaded from the table into X0.}
\fi

\ifdefined\IncludeMIPS
\subsection{MIPS}

\lstinputlisting[caption=\Optimizing GCC 4.4.5 (IDA)]{patterns/13_arrays/45_month_1D/MIPS_O3_IDA.lst.\LANG}
\fi

\ifx\LITE\undefined
\subsection{\RU{Переполнение массива}\EN{Array overflow}}

\RU{Наша функция принимает значения в пределах $0..11$, но что будет, если будет передано $12$?}
\EN{Our function accepts values in the range of $0..11$, but what if $12$ is passed?}
\RU{В таблице в этом месте нет элемента.}\EN{There is no element in table at this place.}
\RU{Так что функция загрузит какое-то значение, которое волею случая находится там, и вернет его.}
\EN{So the function will load some value which happens to be there, and return it.}
\RU{Позже, какая-то другая функция попытается прочитать текстовую строку по этому адресу и, возможно, упадет.}
\EN{Soon after, some other function can try to get a text string from this address and may crash.}

\RU{Я скомпилировал этот пример в MSVC для win64 и открыл его в IDA чтобы посмотреть, что линкер расположил
после таблицы:}
\EN{I have compiled the example in MSVC for win64 and opened it in IDA to see what the linker has placed after the table:}

\lstinputlisting[caption=\RU{Исполняемый файл в}\EN{Executable file in} IDA]{patterns/13_arrays/45_month_1D/MSVC2012_win64_1.lst}

\RU{Имена месяцев идут сразу после.}\EN{Month names are came right after.}
\RU{Наша программа все-таки крошечная, так что здесь не так уж много данных для расположения их в сегменте
данных, так что это имена месяцев.}
\EN{Our program is tiny, so there isn't much data to pack in the data segment, 
so it just the month names.}
\RU{Но я должен заметить, что там может быть действительно \IT{что угодно}, что линкер решит там расположить,
случайным образом.}
\EN{But I have to note that there might be really \IT{anything} that linker has decided to put by chance.}

\RU{Так что будет если 12 будет передано в функцию?}\EN{So what if 12 is passed to the function?}
\RU{Вернется 13-й элемент таблицы}\EN{The 13th element will be returned}.
\RU{Посмотрим, как CPU обходится с байтами там как с 64-битным значением:}
\EN{Let's see how the CPU treating the bytes there as a 64-bit value:}

\lstinputlisting[caption=\RU{Исполняемый файл в}\EN{Executable file in} IDA]{patterns/13_arrays/45_month_1D/MSVC2012_win64_2.lst}

\RU{И это}\EN{And this is} 0x797261756E614A.
\RU{После этого, какая-от другая функция (вероятно, работающая со строками) попытается загружать байты
по этому адресу, ожидая найти там Си-строку.}
\EN{Soon after, some other function (presumably, one that processes strings) may try to read bytes at 
this address, expecting a C-string there.}
\RU{И скорее всего упадет, потому что это значение не выглядит как действительный адрес.}
\EN{Most likely it is about to crash, because this value does't look like a valid address.}

\subsubsection{\RU{Защита от переполнения массива}\EN{Array overflow protection}}
\epigraph{\RU{Если какая-нибудь неприятность может случиться, она случается}
\EN{If something can go wrong, it will}}{\RU{Закон Мерфи}\EN{Murphy's Law}}

\RU{Немного наивно ожидать что всякий программист, кто будет использовать вашу функцию или библиотеку,
никогда не передаст аргумент больше 11.}
\EN{It's a bit naïve to expect that every programmer who use your function or library will never pass
an argument larger than 11.}

\RU{Существует также хорошая философия \q{fail early and fail loudly} или \q{fail-fast},
которая учит сообщать об ошибках как можно раньше и останавливаться.}
\EN{There exists the philosophy that says \q{fail early and fail loudly} or \q{fail-fast}, 
which teaches to report problems as early as possible and stop.}

\index{\CStandardLibrary!assert()}
\RU{Один из таких методов в \CCpp это макрос assert().}
\EN{One such method in \CCpp is assertions.}
\RU{Мы можем немного изменить нашу программу, чтобы она падала при передаче неверного значения:}
\EN{We can modify our program to fail if an incorrect value is passed:}

\lstinputlisting[caption=assert() \RU{добавлен}\EN{added}]{patterns/13_arrays/45_month_1D/month1_assert.c}

\RU{Макрос будет проверять на верные значения во время каждого старта функции и падать если не выражение.}
\EN{The assertion macro checks for valid values at every function start and fails if the expression is false.}

\lstinputlisting[caption=\Optimizing MSVC 2013 x64]{patterns/13_arrays/45_month_1D/MSVC2013_x64_Ox_checked.asm}

\RU{На самом деле, assert() это не функция, а макрос. Он проверяет условие и передает также номер строки и название
файла в другую функцию, которая покажет эту информацию пользователю.}
\EN{In fact, assert() is not a function, but macro. It checks for a condition, then passes also the line number and file
name to another function which reports this information to the user.}

\RU{Мы видим, что здесь и имя файла и выражение закодировано в UTF-16.}
\EN{Here we see that both file name and condition are encoded in UTF-16.}
\RU{Номер строки также передается (это 29)}\EN{The line number is also passed (it's 29)}.

\RU{Этот механизм, пожалуй, одинаковый во всех компиляторах.}
\EN{This mechanism is probably the same in all compilers.}
\RU{Вот что делает GCC}\EN{Here is what GCC does}:

\lstinputlisting[caption=\Optimizing GCC 4.9 x64]{patterns/13_arrays/45_month_1D/GCC491_x64_O3_checked.s}

\RU{Так что макрос в GCC также передает и имя функции, для удобства.}
\EN{So the macro in GCC also passes the function name for convenience.}

\RU{Ничего не бывает бесплатным: и проверки на корректность тоже.}
\EN{Nothing is really free, and this is true for the sanitizing checks as well.}
\RU{Это может замедлить работу вашей программы, особенно если макрос assert() используется в маленькой
критичной ко времени функции.}
\EN{They make your program slower, especially if the assert() macros used in small time-critical functions.}
\RU{Так что, например, MSVC оставляет проверки в Debug-билдах, но в Release-билдах они исчезают.}
\EN{So MSVC, for example, leaves the checks in Debug builds, but in Release builds they all disappear.}
 
\RU{Ядра Microsoft \gls{Windows NT} также идут в виде билдов \q{checked} и \q{free}}
\EN{Microsoft \gls{Windows NT} kernels come in \q{checked} and \q{free} builds}
\footnote{\href{http://go.yurichev.com/17259}{MSDN}}.
\RU{Первые имеют проверки на валидность аргументов (отсюда \q{checked}), вторые нет (отсюда \q{free},
т.е. \q{свободные} от проверок).}
\EN{The first has validation checks (hence, \q{checked}), the second one doesn't (hence, \q{free} of checks).}

% FIXME: ARM? MIPS?
\fi
