\section{\RU{Массив указателей на строки}\EN{Array of pointers to strings}}
\label{array_of_pointers_to_strings}

\RU{Вот также пример массива указателей.}\EN{Here is also example of array of pointers.}

\lstinputlisting[caption=\RU{Получить имя месяца}\EN{Get month name},label=get_month1]{patterns/13_arrays/45_month_1D/month1.c}

\subsection{x64}

\lstinputlisting[caption=\Optimizing MSVC 2013 x64]{patterns/13_arrays/45_month_1D/month1_MSVC_2013_x64_Ox.asm}

\RU{Код очень простой}\EN{The code is very simple}:

\begin{itemize}

\item
\index{x86!\Instructions!MOVSXD}
\RU{Первая инструкция MOVSXD копирует 32-битное значение из ECX (где передается аргумент $month$)
в RAX с знаковым расширением (потому что аргумент $month$ имеет тип \Tint).}
\EN{The first MOVSXD instruction copies 32-bit value from ECX (where $month$ argument is passed) 
to RAX with sign-extension (because $month$ argument has \Tint type).}
\RU{Причина расширения в том что это значение будет использоваться в вычислениях наряду с другими 64-битными
значениями.}
\EN{The reason of extension laying in the fact that this 32-bit value will be used in calculation among
with other 64-bit values.}
\RU{Таким образом, оно должно быть расширено до 64-битного}
\EN{Hence, it should be promoted to 64-bit value}
\footnote{\RU{Это немного странная вещь, но отрицательный индекс массива может быть передан как $month$ 
(отрицаительные индексы массивов будут рассмотрены позже: \ref{negative_array_indices}).
И если так будет, отрицательное значение типа \Tint будет расширено со знаком корректно
и соответствующий элемент перед таблицей будет выбран.
Всё это не будет корректно работать без знакового расширения.}
\EN{It is somewhat weird issue, but negative array index could be passed here as $month$
(negative array indices will be explained later: \ref{negative_array_indices}). 
And if it will, negative input \Tint value will be sign-extended correctly 
and the corresponding element before table will be picked. 
It will not work correctly without sign-extension.}}.

\item
\RU{Затем адрес таблицы указателей загружается в RCX.}
\EN{Then the address of the pointers table is loaded into RCX.}

\item
\RU{В конце концов, входное значение ($month$) умножается на 8 и прибавляется к адресу.}
\EN{Finally, the input value ($month$) is multiplied by 8 and added to the address.}
\RU{Действительно: мы в 64-битной среде и все адреса (или указатели) 
требуют для хранения именно 64 бита (или 8 байт).}
\EN{Indeed: we are in 64-bit environment and all address (or pointers) require exactly 64 bits (or 8 bytes) 
for storage.}
\RU{Следовательно, каждый элемент таблицы имеет ширину в 8 байт.}
\EN{Hence, each table element has width of 8 bytes.}
\RU{И вот почему для выбора элемента под нужным номером, нужно пропустить $month*8$ байт от начала.}
\EN{And that's why to pick element of specific number, $month*8$ bytes should be skipped from the start.}
\RU{Это то что делает MOV}\EN{That's what MOV does}.
\RU{В дополнении, эта инструкция также загружает элемент по этому адресу.}
\EN{In addition, this instruction also loads element at this address.}
\RU{Для 1, элемент будет указателем на строку содержащую ``February'', итд.}
\EN{For 1, an element would be pointer to the string containing ``February'' string, etc.}

\end{itemize}

\Optimizing GCC 4.9 \RU{может это сделать даже лучше}\EN{can do the job even better}
\footnote{\RU{В листинге осталось ``0+'', потому что вывод ассемблера GCC не так скурпулезен, чтобы убрать это}
\EN{``0+'' left in listing because GCC assembler output is not tidy enough for eliminating it}.
\RU{Это \IT{displacement} и он здесь нулевой.}\EN{It's \IT{displacement}, and it's zero here.}}:

\begin{lstlisting}[caption=\Optimizing GCC 4.9 x64]
	movsx	rdi, edi
	mov	rax, QWORD PTR month1[0+rdi*8]
	ret
\end{lstlisting}

\subsubsection{32-bit MSVC}

\RU{Скомпилируем также в 32-битном компиляторе MSVC:}\EN{Let's also compile it in 32-bit MSVC compiler:}

\lstinputlisting[caption=\Optimizing MSVC 2013 x86]{patterns/13_arrays/45_month_1D/month1_MSVC_2013_x86_Ox.asm}

\RU{Входное значение не нужно расширять до 64-битного значения, так что оно используется как есть.}
\EN{Input value not needed to be extended to 64-bit value, so it is used as is.}
\RU{И оно умножается на 4, потому что элементы таблицы имеют ширину 32 бита или 4 байта.}
\EN{And it's multiplied by 4, because table elements has width of 32-bit or 4 bytes.}

\ifdefined\IncludeARM
% FIXME move to another file
\subsection{32-\RU{битный}\EN{bit} ARM}

\subsubsection{ARM \RU{в режиме ARM}\EN{in ARM mode}}

\lstinputlisting[caption=\OptimizingKeilVI (\ARMMode)]{patterns/13_arrays/45_month_1D/month1_Keil_ARM_O3.s}

\RU{Адрес таблицы загружается в R1.}
\EN{Table address is loaded into R1.}
\index{ARM!\Instructions!LDR}
\RU{Всё остальное делается используя только одну инструкцию LDR.}
\EN{All the rest is done using only one LDR instruction.}
\RU{Входное значение $month$ сдвигается влево на 2 (что то же самое что и умножение на 4), это значение
прибавляется к R1 (где находится адрес таблицы) и затем элемент таблицы загружается по этому адресу.}
\EN{Then input $month$ value is shifted left by 2 (which is the same as multiplying by 4), this value added
to R1 (where address of table is) and then a table element is loaded at this address.}
\RU{32-битный элемент таблицы загружается в R0 из таблицы.}
\EN{32-bit table element is loaded into R0 from the table.}

\subsubsection{ARM \RU{в режиме Thumb}\EN{in Thumb mode}}

\RU{Код почти такой же, только менее плотный, потому что здесь, в инструкции LDR, нельзя задать суффикс LSL:}
\EN{The code is mostly the same, but less dense, because LSL suffix cannot be specified in LDR instruction here:}

\begin{lstlisting}
get_month1 PROC
        LSLS     r0,r0,#2
        LDR      r1,|L0.64|
        LDR      r0,[r1,r0]
        BX       lr
        ENDP
\end{lstlisting}

\subsection{ARM64}

\lstinputlisting[caption=\Optimizing GCC 4.9 ARM64]{patterns/13_arrays/45_month_1D/month1_GCC49_ARM64_O3.s}

\index{ARM!\Instructions!ADRP/ADD pair}
\RU{Адрес таблицы загружается в X1 используя пару ADRP/ADD.}
\EN{Table address is loaded into X1 using ADRP/ADD pair.}
\RU{Соответствующий элемент выбирается используя одну инструкцию LDR, которая берет W0
(регистр где находится значение входного аргумента $month$), сдвигает его на 3 бита влево
(что то же самое что и умножение на 8),
расширяет его учитывая знак (это то что означает суффикс ``sxtw'') и прибавляет к X0.}
\EN{Then corresponding element is picked using only one instruction LDR, which takes W0 
(register where input $month$ argument is), shifts it 3 bits left (which is the same as multiplying by 8), 
sign-extends it (this is what ``sxtw'' suffix mean) and adds to X0.}
\RU{Затем 64-битное значение загружается из таблицы в X0.}\EN{Then 64-bit value is loaded into X0 from the table.}
\fi

\ifdefined\IncludeMIPS
\subsection{MIPS}

\lstinputlisting[caption=\Optimizing GCC 4.4.5 (IDA)]{patterns/13_arrays/45_month_1D/MIPS_O3_IDA.lst}
\fi

\subsection{\RU{Переполнение массива}\EN{Array overflow}}

\RU{Наша ф-ция принимает значения в пределах $0..11$, но что будет, если будет передано $12$?}
\EN{Our function accepts values in range of $0..11$, but what if $12$ will be passed?}
\RU{В таблице в этом месте нет элемента.}\EN{There are no element in table present at this place.}
\RU{Так что ф-ция загрузит какое-то значение, которое волею случая находится там, и вернет его.}
\EN{So, the function will load some value which happened to be located there, and return it.}
\RU{Позже, какая-то другая ф-ция попытается прочитать текстовую строку по этому адресу и, возможно, упадет.}
\EN{Soon after, some other function will try to get a text string at this address and may crash.}

\RU{Я скомпилировал этот пример в MSVC для win64 и открыл его в IDA чтобы посмотреть, что линкер расположил
после таблицы:}
\EN{I compiled the example in MSVC for win64 and opened it in IDA to see what linker placed after the table:}

\lstinputlisting[caption=\RU{Исполняемый файл в}\EN{Executable file in} IDA]{patterns/13_arrays/45_month_1D/MSVC2012_win64_1.lst}

\RU{Имена месяцев идут сразу после.}\EN{Month names are came right after.}
\RU{Наша программа все-таки крошечная, так что здесь не так уж много данных для расположения их в сегменте
данных, так что это имена месяцев.}
\EN{Our program is tiny after all, so there are no much data to pack in the data segment, 
so these are month names.}
\RU{Но я должен заметить, что там может быть действительно \IT{что угодно}, что линкер решит там расположить,
случайным образом.}
\EN{But I should to note that there might be really \IT{anything} what linker decided to put by chance.}

\RU{Так что будет если 12 будет передано в ф-цию?}\EN{So what if 12 will be passed to the function?}
\RU{Вернется 13-й элемент таблицы}\EN{13th table element will be returned}.
\RU{Посмотрим, как CPU будет обходиться с байтами там как с 64-битным значением:}
\EN{Let's see how CPU will treat the bytes there as 64-bit value:}

\lstinputlisting[caption=\RU{Исполняемый файл в}\EN{Executable file in} IDA]{patterns/13_arrays/45_month_1D/MSVC2012_win64_2.lst}

\RU{И это}\EN{And this is} 0x797261756E614A.
\RU{После этого, какая-от другая ф-ция (вероятно, работающая со строками) попытается загруждать байты
по этому адресу, ожидая найти там Си-строку.}
\EN{Soon after, some other function (presumably, string processing) will try to read bytes at 
this address expecting C-string there.}
\RU{И скорее всего упадет, потому что это значение не выглядит как действительный адрес.}
\EN{Most likely it will crash, because this value is don't look like a valid address.}

\subsubsection{\RU{Защита от переполнения массива}\EN{Array overflow protection}}
\epigraph{\RU{Если какая-нибудь неприятность может случиться, она случается}
\EN{If something can go wrong, it will}}{\RU{Закон Мерфи}\EN{Murphy's Law}}

\RU{Немного наивно ожидать что всякий программист, кто будет использовать вашу ф-цию или библиотеку,
никогда не передаст аргумент больше 11.}
\EN{It's a bit naïve to expect that every programmer who use your function or library will never pass
an argument larger than 11.}

\RU{Существует также хорошая философия ``fail early and fail loudly'' или ``fail-fast'',
которая учит сообщать об ошибках как можно раньше и останавливаться.}
\EN{There are also a good ``fail early and fail loudly'' or ``fail-fast'' philosophy, 
which teaches to report problems as early as possible and stop.}

\index{\CStandardLibrary!assert()}
\RU{Один из таких методов в \CCpp это макрос assert().}
\EN{One of such methods in \CCpp is assertions.}
\RU{Мы можем немного изменить нашу программу, чтобы она падала при передаче неверного значения:}
\EN{We can modify our program to fail if incorrect value is passed:}

\lstinputlisting[caption=assert() \RU{добавлен}\EN{added}]{patterns/13_arrays/45_month_1D/month1_assert.c}

\RU{Макрос будет проверять на верные значения во время каждого старта ф-ции и падать если не выражение.}
\EN{Assertion macro will check for valid values at each function start and fail if expression is false.}

\lstinputlisting[caption=\Optimizing MSVC 2013 x64]{patterns/13_arrays/45_month_1D/MSVC2013_x64_Ox_checked.asm}

\RU{На самом деле, assert() это не ф-ция, а макрос. Он проверяет условие и передает также номер строки и название
файла в другую ф-цию, которая покажет эту инфомарцию пользователю.}
\EN{In fact, assert() is not a function, but macro. It checks for condition, then pass also line number and file
name to another function which will show this information to user.}

\RU{Мы видим что здесь и имя файла и выражение закодировано в UTF-16.}
\EN{Here we see that both file name and condition was encoded in UTF-16.}
\RU{Номер строки также передается (это 29)}\EN{Line number is also passed (it's 29)}.

\RU{Этот механизм, пожалуй, одинаковый во всех компиляторах.}
\EN{This mechanism is same in probably all compilers.}
\RU{Вот что делает GCC}\EN{Here is what GCC does}:

\lstinputlisting[caption=\Optimizing GCC 4.9 x64]{patterns/13_arrays/45_month_1D/GCC491_x64_O3_checked.s}

\RU{Так что макрос в GCC также передает и имя ф-ции, для удобства.}
\EN{So the macro in GCC also pass function name for convenience.}

\RU{Ничего не бывает бесплатным: и проверки на корректность тоже.}
\EN{Nothing came for free: sanitizing check as well.}
\RU{Это может замедлить работу вашей программы, особенно если макрос assert() используется в маленькой
критичной ко времени ф-ции.}
\EN{It makes your program slower, especially if assert() macros used in small time-critical functions.}
\RU{Так что, например, MSVC оставляет проверки в Debug-билдах, но в Release-билдах они исчезают.}
\EN{So MSVC, for example, leaves checks in Debug builds, but in Release builds they all are disappear.}
 
\RU{Ядра Microsoft \gls{Windows NT} также идут в виде билдов ``checked'' и ``free''}
\EN{Microsoft \gls{Windows NT} kernels are also came in ``checked'' and ``free'' builds}
\footnote{\url{http://msdn.microsoft.com/en-us/library/windows/hardware/ff543450(v=vs.85).aspx}}.
\RU{Первые имеют проверки на валидность аргументов (отсюда ``checked''), вторые нет (отсюда ``free'',
т.е., ``свободые'' от проверок).}
\EN{First has validation checks (hence, ``checked''), second doesn't (hence, ``free'' of checks).}

% FIXME: ARM? MIPS?
