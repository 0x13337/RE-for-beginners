% TODO: translate
\ifdefined\RUSSIAN
\else
\section{Array of pointers to strings}
\label{array_of_pointers_to_strings}

Here is also example of array of pointers.

\lstinputlisting[caption=Get month name,label=get_month1]{patterns/13_arrays/45_month_1D/month1.c}

\subsection{x64}

The code is very simple. 

\begin{itemize}

\item
\index{x86!\Instructions!MOVSXD}
The first MOVSXD instruction copies 32-bit value from ECX (where $month$ argument is passed) 
to RAX with sign-extension (because $month$ argument has \Tint type).
The reason of extension laying in the fact that this 32-bit value will be used in calculation among
with other 64-bit values. 
Hence, it should be promoted to 64-bit value
\footnote{It is somewhat weird issue, but negative array index could be passed here 
(negative array indices will be explained later: \ref{negative_array_indices}). 
And if it will, negative input \Tint value will be sign-extended correctly 
and the corresponding element before table will be picked. 
It will not work correctly without sign-extension.}.

\item
Then the address of the whole pointers table is loaded into RCX.

\item
Finally, the input value ($month$) is multiplied by 8 and added to the address.
Indeed: we are in 64-bit environment and all address (or pointers) require exactly 64 bits (or 8 bytes) 
for storage.
Hence, each table element has width of 8 bytes.
And that's why to pick element of specific number, $month*8$ bytes should be skipped from the start.
That's what MOV does.
But it also loads element at this address.
For 1 it would be pointer to the string containing ``February'' string, etc.

\end{itemize}

\lstinputlisting[caption=\Optimizing MSVC 2013 x64]{patterns/13_arrays/45_month_1D/month1_MSVC_2013_x64_Ox.asm}

\Optimizing GCC 4.9 can do the job even better
\footnote{``0+'' left in listing because GCC assembler output is not tidy enough for eliminating it. 
It's displacement, and it's zero here.}:

\begin{lstlisting}[caption=\Optimizing GCC 4.9 x64]
	movsx	rdi, edi
	mov	rax, QWORD PTR month1[0+rdi*8]
	ret
\end{lstlisting}

\subsubsection{32-bit MSVC}

Let's also compile it in 32-bit MSVC compiler:

\lstinputlisting[caption=\Optimizing MSVC 2013 x86]{patterns/13_arrays/45_month_1D/month1_MSVC_2013_x86_Ox.asm}

Input value not needed to be extended to 64-bit value, so it is used as is.
And it's multiplied by 4, because table elements has width of 32-bit or 4 bytes.

\subsection{32-bit ARM}

\subsubsection{ARM in ARM mode}

\lstinputlisting[caption=\OptimizingKeilVI + \ARMMode]{patterns/13_arrays/45_month_1D/month1_Keil_ARM_O3.s}

Table address is loaded into R1.
\index{ARM!\Instructions!LDR}
All the rest is done using only one LDR instruction.
Then input $month$ value is shifted left by 2 (which is the same as multiplying by 4), this value added
to R1 (where address of table is) and then a value is loaded at this address.
Then 32-bit value is loaded into R0 from the table.

\subsubsection{ARM in Thumb mode}

The code is mostly the same, but less dense, because LSL suffix cannot be specified in LDR instruction here:

\begin{lstlisting}
get_month1 PROC
        LSLS     r0,r0,#2
        LDR      r1,|L0.64|
        LDR      r0,[r1,r0]
        BX       lr
        ENDP
\end{lstlisting}

\subsection{ARM64}

\lstinputlisting[caption=\Optimizing GCC 4.9 ARM64]{patterns/13_arrays/45_month_1D/month1_GCC49_ARM64_O3.s}

\index{ARM!\Instructions!ADRP/ADD pair}
Table address is loaded into X1 using ADRP/ADD pair.
Then corresponding element is picked using only one instruction LDR, which takes W0 
(register where input $month$ argument is), shifts it 3 bits left (which is the same as multiplying by 8), 
sign-extends it (this is what ``sxtw'' suffix mean) and adds to X0.
Then 64-bit value is loaded into X0 from the table.

\subsubsection{Array overflow}

Our function accepts values in range of 0..11, but what if 12 will be passed?
There are no element in table present at this place.
So, the function will load some value which happened to be located there, and return it.
Soon after, some other function will try to get a string at this address and may crash.

I compiled the example in MSVC for win64 and opened it in IDA to see what linker placed after the table:

\lstinputlisting[caption=Executable in IDA]{patterns/13_arrays/45_month_1D/MSVC2012_win64_1.lst}

Month names are came right after.
Our program is tiny after all, so there are no much data to pack in the data segment, so these are names.
But I should to note that there might be really \IT{anything} what linker decided to put by chance.

So what if 12 will be passed to the function?
13th table element will be returned.
Let's see how CPU will treat the bytes there as 64-bit value:

\lstinputlisting[caption=Executable in IDA]{patterns/13_arrays/45_month_1D/MSVC2012_win64_2.lst}

And this is 0x797261756E614A.
Soon after, some other function (presumably, string processing) will try to read bytes at 
this address expecting C-string there.
Most likely it will crash, because this value is don't look like a valid address.

\subsubsection{Array overflow protection}
\epigraph{If something can go wrong, it will}{Murphy's Law}

It's a bit naïve to expect that every programmer who use your function or library will never pass
an argument larger than 11.

There are also a good philisophy ``fail early and fail loudly'' or ``fail-fast'', which teaches to report
problems as early as possible and stop.

\index{\CStandardLibrary!assert()}
One of such methods in \CCpp is assertions.
We can modify our program to fail if incorrect value is passed:

\lstinputlisting[caption=assert() added]{patterns/13_arrays/45_month_1D/month1_assert.c}

Assertion macro will check for valid values at each function start and fail if it's not.

\lstinputlisting[caption=\Optimizing MSVC 2013 x64]{patterns/13_arrays/45_month_1D/MSVC2013_x64_Ox_checked.asm}

In fact, assert() is not a function, but macro. It checks for condition, then pass also line number and file
name to another function which will show this information to user.

Here we see that both file name and condition was encoded in UTF-16.
Line number is also passed (it's 29).

This mechanism is same in probably all compilers.
Here is what GCC does:

\lstinputlisting[caption=\Optimizing GCC 4.9 x64]{patterns/13_arrays/45_month_1D/GCC491_x64_O3_checked.s}

So it also pass function name for convenience.

Nothing came for free: sanitizing check as well.
It makes your program slower, especially if assert() macros used in small time-critical functions.
So MSVC, for example, leaves checks in Debug builds, but in Release builds they all are disappear.
 
Microsoft \gls{Windows NT} kernels are also came in ``checked'' and ``free'' builds
\footnote{\url{http://msdn.microsoft.com/en-us/library/windows/hardware/ff543450(v=vs.85).aspx}}.
First has validation checks (hence, ``checked''), second doesn't (hence, ``free'' of checks).
\fi
