\sectionold{\RU{Еще немного о массивах}\EN{One more word about arrays}}

\RU{Теперь понятно, почему нельзя написать в исходном коде на \CCpp что-то вроде:}
\EN{Now we understand why it is impossible to write something like this in \CCpp code:}

\begin{lstlisting}
void f(int size)
{
    int a[size];
...
};
\end{lstlisting}

\RU{Чтобы выделить место под массив в локальном стеке, 
компилятору нужно знать размер массива, чего он на стадии компиляции, 
разумеется, знать не может.}
\EN{That's just because the compiler must know the exact array size to allocate space for 
it in the local stack layout on at the compiling stage.}

\myindex{\CLanguageElements!C99!variable length arrays}
\myindex{\CStandardLibrary!alloca()}
\RU{Если вам нужен массив произвольной длины, то выделите столько, сколько нужно, через \TT{malloc()}, 
а затем обращайтесь к выделенному блоку байт как к массиву того типа, который вам нужен.}
\EN{If you need an array of arbitrary size, allocate it by using \TT{malloc()}, then access the allocated memory block
as an array of variables of the type you need.}

\RU{Либо используйте возможность стандарта C99~ [\CNineNineStd 6.7.5/2],
и внутри это очень похоже на alloca()~(\myref{alloca}).}
\EN{Or use the C99 standard feature [\CNineNineStd 6.7.5/2],
and it works like alloca()~(\myref{alloca}) internally.}

\RU{Для работы в с памятью, можно также воспользоваться библиотекой сборщика мусора в Си.}
\EN{It's also possible to use garbage collecting libraries for C.}
\RU{А для языка Си++ есть библиотеки с поддержкой умных указателей.}
\EN{And there are also libraries supporting smart pointers for C++.}

