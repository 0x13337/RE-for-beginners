\section{\RU{Еще немного о массивах}\EN{One more word about arrays}}

\RU{Теперь понятно, почему нельзя написать в исходном коде на \CCpp что-то вроде
\footnote{Впрочем, по стандарту C99 это возможно\cite[6.7.5/2]{C99TC3}: 
GCC может это сделать выделяя место под массив динамически в стеке (как alloca()~(\ref{alloca}))}}
\EN{Now we understand why it is impossible to write something like this in \CCpp code
\footnote{However, it is possible in C99 standard\cite[6.7.5/2]{C99TC3}: 
GCC actually does this by allocating an array dynamically on the stack (like alloca()~(\ref{alloca}))}}:

\begin{lstlisting}
void f(int size)
{
    int a[size];
...
};
\end{lstlisting}

\RU{Все просто потому, чтобы выделять место под массив в локальном стеке, 
компилятору нужно знать его размер, чего он, на стадии компиляции, 
разумеется, знать не может.}
\EN{That's just because the compiler must know the exact array size to allocate space for 
it in the local stack layout on at the compiling stage.}

\index{\CLanguageElements!C99!variable length arrays}
\index{\CStandardLibrary!alloca()}
\RU{Если вам нужен массив произвольной длины, то выделите столько, сколько нужно, через \TT{malloc()}, 
затем обращайтесь к выделенному блоку байт как к массиву того типа, который вам нужен.
Либо используйте возможность стандарта C99\cite[6.7.5/2]{C99TC3}, 
но внутри это очень похоже на alloca()~(\ref{alloca})}
\EN{If you need an array of arbitrary size, allocate it by using \TT{malloc()}, then access the allocated memory block
as an array of variables of the type you need.
Or use the C99 standard feature\cite[6.7.5/2]{C99TC3}, 
but it looks like alloca()~(\ref{alloca}) internally.}
