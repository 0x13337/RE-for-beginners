\section{\RU{Набор строк как двухмерный массив}\EN{Pack of strings as two-dimensional array}}

\RU{Снова вернемся к примеру, который возвращает название месяца:}
\EN{Let's revisit function which returns name of month:} \lstref{get_month1}.
\RU{Как видно, нужна как минимум одна операция загрузки из памяти для подготовки указателя на строку
состоящую из имени месяца.}
\EN{As you may see, at least one memory load operation is needed to prepare a pointer to the string
consisting of month name.}
\RU{Возможно ли избавиться от операции загрузки из памяти?}
\EN{Will it be possible to get rid of this memory load operation?}
\RU{На самом деле, да, если представить список строк как двухмерный массив:}
\EN{In fact yes, if to represent list of strings as a two-dimensional array:}

\lstinputlisting{patterns/13_arrays/55_month_2D/month2.c}

\RU{Вот что получаем:}\EN{Here is what we've got:}

\lstinputlisting[caption=\Optimizing MSVC 2013 x64]{patterns/13_arrays/55_month_2D/MSVC2013_x64_Ox.asm}

\RU{Здесь нет обращений к памяти вообще}\EN{There are no memory access at all}.
\RU{Всё что делает эта ф-ция это вычисляет место, где находится первый символ названия месяца:}
\EN{All this function do is calculating point at which the first character of month name is:} 
$pointer\_to\_the\_table + month * 10$.
\RU{Там также две инструкции LEA, которые работают как несколько инструкций MUL и MOV.}
\EN{There are also two LEA instructions which are effectively working as several MUL and MOV instructions.}

\RU{Ширина массива --- 10 байт}\EN{Width of the array is 10 bytes}. 
\RU{Действительно, самая длинная строка здесь это ``September'' (9 байт) плюс оконечивающий ноль --- это 10 байт.}
\EN{Indeed, longest string here is ``September'' (9 bytes) plus terminating zero is 10 bytes.}
\RU{Остальные названия месяцев дополнены нулевыми байтами, чтобы они занимали столько же места (10 байт).}
\EN{Other month names are padded by zero bytes, so they all occupy the same space (10 bytes).}
\RU{Таким образом, наша функция и работает быстрее, потому что все строки начинаются с тех адресов, 
которые легко вычислить.}
\EN{Thus, our function works even faster, because all string are started at addresses which can be
calculated easily.}

\Optimizing GCC 4.9 \RU{может еще короче}\EN{can do it even shorter}:

\begin{lstlisting}[caption=\Optimizing GCC 4.9 x64]
	movsx	rdi, edi
	lea	rax, [rdi+rdi*4]
	lea	rax, month2[rax+rax]
	ret
\end{lstlisting}

\RU{LEA здесь также используется для умножения на 10.}
\EN{LEA is also used here for multiplication by 10.}

\RU{Неоптимизирующие компиляторы делают умножение по-разному.}
\EN{Non-optimizing compilers do multiplication differently.}

% FIXME translation
\begin{lstlisting}[caption=\NonOptimizing GCC 4.9 x64]
get_month2:
	push	rbp
	mov	rbp, rsp
	mov	DWORD PTR [rbp-4], edi
	mov	eax, DWORD PTR [rbp-4]
	movsx	rdx, eax
; RDX = sign-extended input value
	mov	rax, rdx
; RAX = month
	sal	rax, 2
; RAX = month<<2 = month*4
	add	rax, rdx
; RAX = RAX+RDX = month*4+month = month*5
	add	rax, rax
; RAX = RAX*2 = month*5*2 = month*10
	add	rax, OFFSET FLAT:month2
; RAX = month*10 + pointer to the table
	pop	rbp
	ret
\end{lstlisting}

\NonOptimizing MSVC \RU{просто использует инструкцию IMUL}\EN{just use IMUL instruction}:
\index{x86!\Instructions!IMUL}

\lstinputlisting[caption=\NonOptimizing MSVC 2013 x64]{patterns/13_arrays/55_month_2D/MSVC2013_x64.asm}

\index{\CompilerAnomaly}
\label{MSVC2013_anomaly}
\dots \RU{но вот что странно: зачем добавлять умножение на ноль и добавлять ноль к конечному результату?}
\EN{but one thing is weird here: why to add multiplication by zero and adding zero to the final result?}
\RU{Я не знаю, это выглядит как выверт кодегенератора компилятора, который не был покрыт тестами
компилятора (так или иначе, итоговый код работает корректно).}
\EN{I don't know, this looks like compiler code generator quirk, which wasn't catched by compiler's tests
(resulting code is working correctly after all).}
\RU{Я сознательно добавляю сюда такие фрагменты кода, чтобы читатель понимал, что иногда не нужно
ломать себе голову над подобными артифактами компиляторов.}
\EN{I intentionally add such pieces of code so the reader would understand, 
that sometimes one shouldn't puzzle over such compiler's artifacts.}

\ifdefined\IncludeARM
\subsection{32-bit ARM}

\Optimizing Keil \RU{для режима Thumb использует инструкцию умножения}
\EN{for Thumb mode use multiplication instruction} MULS:

\begin{lstlisting}[caption=\OptimizingKeilVI (\ThumbMode)]
; R0 = month
        MOVS     r1,#0xa
; R1 = 10
        MULS     r0,r1,r0
; R0 = R1*R0 = 10*month
        LDR      r1,|L0.68|
; R1 = pointer to the table
        ADDS     r0,r0,r1
; R0 = R0+R1 = 10*month + pointer to the table
        BX       lr
\end{lstlisting}

\Optimizing Keil \RU{для режима ARM использует операции сложения и сдвига}\EN{for ARM mode use add and 
shift operations}:

\begin{lstlisting}[caption=\OptimizingKeilVI (\ARMMode)]
; R0 = month
        LDR      r1,|L0.104|
; R1 = pointer to the table
        ADD      r0,r0,r0,LSL #2
; R0 = R0+R0<<2 = R0+R0*4 = month*5
        ADD      r0,r1,r0,LSL #1
; R0 = R1+R0<<2 = pointer to the table + month*5*2 = pointer to the table + month*10
        BX       lr
\end{lstlisting}

\subsection{ARM64}

\begin{lstlisting}[caption=\Optimizing GCC 4.9 ARM64]
; W0 = month
	sxtw	x0, w0
; X0 = sign-extended input value
	adrp	x1, .LANCHOR1
	add	x1, x1, :lo12:.LANCHOR1
; X1 = pointer to the table
	add	x0, x0, x0, lsl 2
; X0 = X0+X0<<2 = X0+X0*4 = X0*5
	add	x0, x1, x0, lsl 1
; X0 = X1+X0<<1 = X1+X0*2 = pointer to the table + X0*10
	ret
\end{lstlisting}

\index{ARM!\Instructions!SXTW}
\index{ARM!\Instructions!ADRP/ADD pair}
\RU{SXTW используется для знакового расширения и расширения входного 32-битного значения в 64-битное и сохранения
его в X0.}
\EN{SXTW is used for sign-extension and promoting input 32-bit value into 64-bit one and storing it in X0.}
\RU{Пара ADRP/ADD используется для загрузки адреса таблицы.}
\EN{ADRP/ADD pair is used for loading address of the table.}
\RU{У инструкции ADD также есть суффикс LSL, что помогает с умножением.}
\EN{ADD instructions also has LSL suffix, which helps with multiplications.}
\fi

\ifdefined\IncludeMIPS
\subsection{MIPS}
\lstinputlisting[caption=\Optimizing GCC 4.4.5 (IDA)]{patterns/13_arrays/55_month_2D/MIPS_O3_IDA.lst}
\fi

\subsection{\Conclusion{}}

\RU{Это немного олд-скульная техника для хранения текстовых строк.}
\EN{This is a bit old-school technique to store text strings.}
\RU{Такого можно много найти в \oracle, например.}
\EN{You may find a lot of it in \oracle, for example.}
\RU{Но я не уверен, стоит ли оно того на современных компьютерах.}
\EN{But I don't really know if it's worth to do it on modern computers.}
\RU{Так или ниаче, это был хороший пример массивов, так что я добавил его в эту книгу.}
\EN{Nevertheless, it was a good example of arrays, so I added it to this book.}
