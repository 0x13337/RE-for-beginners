\section{\RU{Набор строк как двухмерный массив}\EN{Pack of strings as a two-dimensional array}}

\RU{Снова вернемся к примеру, который возвращает название месяца:}
\EN{Let's revisit the function that returns the name of a month:} \lstref{get_month1}.
\RU{Как видно, нужна как минимум одна операция загрузки из памяти для подготовки указателя на строку,
состоящую из имени месяца.}
\EN{As you see, at least one memory load operation is needed to prepare a pointer to the string
that's the month's name.}
\RU{Возможно ли избавиться от операции загрузки из памяти?}
\EN{Is it possible to get rid of this memory load operation?}
\RU{На самом деле, да, если представить список строк как двухмерный массив:}
\EN{In fact yes, if you represent the list of strings as a two-dimensional array:}

\lstinputlisting{patterns/13_arrays/55_month_2D/month2.c.\LANG}

\RU{Вот что получаем:}\EN{Here is what we've get:}

\lstinputlisting[caption=\Optimizing MSVC 2013 x64]{patterns/13_arrays/55_month_2D/MSVC2013_x64_Ox.asm.\LANG}

\RU{Здесь нет обращений к памяти вообще}\EN{There are no memory accesses at all}.
\RU{Всё что делает эта ф-ция это вычисляет место, где находится первый символ названия месяца:}
\EN{All this function does is to calculate a point at which the first character of the name of the month is:} 
$pointer\_to\_the\_table + month * 10$.
\RU{Там также две инструкции LEA, которые работают как несколько инструкций MUL и MOV.}
\EN{There are also two LEA instructions, which effectively work as several MUL and MOV instructions.}

\RU{Ширина массива --- 10 байт}\EN{The width of the array is 10 bytes}. 
\RU{Действительно, самая длинная строка здесь это ``September'' (9 байт) плюс оконечивающий ноль --- это 10 байт.}
\EN{Indeed, the longest string here --- ``September'' --- is 9 bytes, and plus the terminating zero is 10 bytes.}
\RU{Остальные названия месяцев дополнены нулевыми байтами, чтобы они занимали столько же места (10 байт).}
\EN{The rest of the month names are padded by zero bytes, so they all occupy the same space (10 bytes).}
\RU{Таким образом, наша функция и работает быстрее, потому что все строки начинаются с тех адресов, 
которые легко вычислить.}
\EN{Thus, our function works even faster, because all string start at an address which can be
calculated easily.}

\Optimizing GCC 4.9 \RU{может еще короче}\EN{can do it even shorter}:

\begin{lstlisting}[caption=\Optimizing GCC 4.9 x64]
	movsx	rdi, edi
	lea	rax, [rdi+rdi*4]
	lea	rax, month2[rax+rax]
	ret
\end{lstlisting}

\RU{LEA здесь также используется для умножения на 10.}
\EN{LEA is also used here for multiplication by 10.}

\RU{Неоптимизирующие компиляторы делают умножение по-разному.}
\EN{Non-optimizing compilers do multiplication differently.}

\lstinputlisting[caption=\NonOptimizing GCC 4.9 x64]{patterns/13_arrays/55_month_2D/x64_GCC49_O0.asm.\LANG}

\NonOptimizing MSVC \RU{просто использует инструкцию IMUL}\EN{just use IMUL instruction}:
\index{x86!\Instructions!IMUL}

\lstinputlisting[caption=\NonOptimizing MSVC 2013 x64]{patterns/13_arrays/55_month_2D/MSVC2013_x64.asm.\LANG}

\index{\CompilerAnomaly}
\label{MSVC2013_anomaly}
\dots \RU{но вот что странно: зачем добавлять умножение на ноль и добавлять ноль к конечному результату?}
\EN{but one thing is weird here: why add multiplication by zero and adding zero to the final result?}
\RU{Я не знаю, это выглядит как выверт кодегенератора компилятора, который не был покрыт тестами
компилятора (так или иначе, итоговый код работает корректно).}
\EN{I don't know, this looks like a compiler code generator quirk, which wasn't caught by the compiler's tests
(the resulting code works correctly, after all).}
\RU{Я сознательно добавляю сюда такие фрагменты кода, чтобы читатель понимал, что иногда не нужно
ломать себе голову над подобными артефактами компиляторов.}
\EN{I intentionally add such pieces of code so the reader would understand, 
that sometimes one shouldn't puzzle over such compiler artifacts.}

\ifdefined\IncludeARM
\subsection{32-bit ARM}

\Optimizing Keil \RU{для режима Thumb использует инструкцию умножения}
\EN{for Thumb mode uses the multiplication instruction} MULS:

\lstinputlisting[caption=\OptimizingKeilVI (\ThumbMode)]{patterns/13_arrays/55_month_2D/Keil_O3_thumb.asm.\LANG}

\Optimizing Keil \RU{для режима ARM использует операции сложения и сдвига}\EN{for ARM mode uses add and 
shift operations}:

\lstinputlisting[caption=\OptimizingKeilVI (\ARMMode)]{patterns/13_arrays/55_month_2D/Keil_O3_ARM.asm.\LANG}

\subsection{ARM64}

\lstinputlisting[caption=\Optimizing GCC 4.9 ARM64]{patterns/13_arrays/55_month_2D/GCC49_ARM64.asm.\LANG}

\index{ARM!\Instructions!SXTW}
\index{ARM!\Instructions!ADRP/ADD pair}
\RU{SXTW используется для знакового расширения и расширения входного 32-битного значения в 64-битное и сохранения
его в X0.}
\EN{SXTW is used for sign-extension and promoting input 32-bit value into a 64-bit one and storing it in X0.}
\RU{Пара ADRP/ADD используется для загрузки адреса таблицы.}
\EN{ADRP/ADD pair is used for loading the address of the table.}
\RU{У инструкции ADD также есть суффикс LSL, что помогает с умножением.}
\EN{The ADD instructions also has a LSL suffix, which helps with multiplications.}
\fi

\ifdefined\IncludeMIPS
\subsection{MIPS}
\lstinputlisting[caption=\Optimizing GCC 4.4.5 (IDA)]{patterns/13_arrays/55_month_2D/MIPS_O3_IDA.lst.\LANG}
\fi

\subsection{\Conclusion{}}

\RU{Это немного олд-скульная техника для хранения текстовых строк.}
\EN{This is a bit old-school technique to store text strings.}
\RU{Такого можно много найти в \oracle, например.}
\EN{You may find a lot of it in \oracle, for example.}
\RU{Но я не уверен, стоит ли оно того на современных компьютерах.}
\EN{But I don't really know if it's worth doing on modern computers.}
\RU{Так или иначе, это был хороший пример массивов, так что я добавил его в эту книгу.}
\EN{Nevertheless, it was a good example of arrays, so I added it to this book.}
