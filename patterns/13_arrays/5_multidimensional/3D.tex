\subsection{\RU{Пример с трехмерным массивом}\EN{Three-dimensional array example}}

\RU{То же самое и для многомерных массивов.}\EN{Same thing about multidimensional arrays.}

\RU{На этот раз будем работать с массивом типа \Tint: каждый элемент требует 4 байта в памяти.}
\EN{Now we will work with array of \Tint type: each element require 4 bytes in memory.}

\RU{Попробуем}\EN{Let's see}:

\lstinputlisting[caption=\RU{простой пример}\EN{simple example}]{patterns/13_arrays/5_multidimensional/multi.c}

\subsubsection{x86}

\RU{В итоге}\EN{We got} (MSVC 2010):

\lstinputlisting[caption=MSVC 2010]{patterns/13_arrays/5_multidimensional/multi_msvc.asm}

\RU{В принципе, ничего удивительного. В \TT{insert()} для вычисления адреса нужного элемента массива, 
три входных аргумента перемножаются по формуле $address=600 \cdot 4 \cdot x + 30 \cdot 4 \cdot y + 4z$, 
чтобы представить массив трехмерным.
Не забывайте также, что тип \Tint 32-битный (4 байта), поэтому все коэффициенты нужно умножить на 4.}
\EN{Nothing special. For index calculation, three input arguments are multiplying 
by formula $address=600 \cdot 4 \cdot x + 30 \cdot 4 \cdot y + 4z$ to represent array as multidimensional.
Do not forget the \Tint type is 32-bit (4 bytes),
so all coefficients must be multiplied by 4.}

\lstinputlisting[caption=GCC 4.4.1]{patterns/13_arrays/5_multidimensional/multi_gcc.asm}

\RU{Компилятор GCC решил всё сделать немного иначе}\EN{GCC compiler does it differently}.
\RU{Для вычисления одной из операций ($30y$), GCC создал код, где нет самой операции умножения.}
\EN{For one of operations calculating ($30y$), GCC produced a code without multiplication instruction.}
\RU{Происходит это так}\EN{This is how it done}: 
$(y+y) \ll 4 - (y+y) = (2y) \ll 4 - 2y = 2 \cdot 16 \cdot y - 2y = 32y - 2y = 30y$. 
\RU{Таким образом, для вычисления $30y$ используется только операция сложения, 
операция битового сдвига и операция вычитания.}\EN{Thus, for $30y$ calculation, only one addition operation
used, one bitwise shift operation and one subtraction operation.}
\RU{Это работает быстрее}\EN{That works faster}.

\ifdefined\IncludeARM
\subsubsection{ARM + \NonOptimizingXcodeIV (\ThumbMode)}

\lstinputlisting[caption=\NonOptimizingXcodeIV (\ThumbMode)]{patterns/13_arrays/5_multidimensional/multi_Xcode_thumb_O0_\LANG.asm}

\NonOptimizing LLVM \RU{сохраняет все переменные в локальном стеке, хотя это и избыточно.}
\EN{saves all variables in local stack, however, it is redundant.}
\RU{Адрес элемента массива вычисляется по уже рассмотренной формуле.}
\EN{Address of array element is calculated by formula we already figured out.}

\subsubsection{ARM + \OptimizingXcodeIV (\ThumbMode)}

\lstinputlisting[caption=\OptimizingXcodeIV (\ThumbMode)]{patterns/13_arrays/5_multidimensional/multi_Xcode_thumb_O3_\LANG.asm}

\RU{Тут используются уже описанные трюки для замены умножения на операции сдвига, сложения и вычитания.}
\EN{Tricks for replacing multiplication by shift, addition and subtraction which we already considered
are also present here.}

\index{ARM!\Instructions!RSB}
\index{ARM!\Instructions!SUB}
\RU{Также мы видим новую для себя инструкцию}\EN{Here we also see new instruction for us:} 
\TT{RSB} (\IT{Reverse Subtract}).
\RU{Она работает так же, как и \SUB, только меняет операнды местами.}
\EN{It works just as \SUB, but it swaps its operands with each other before.}
\RU{Зачем?}\EN{Why?}
\index{ARM!Optional operators!LSL}
\SUB \AndENRU \TT{RSB}\RU{, это те инструкции, ко второму операнду которых 
можно применить коэффициент сдвига, как мы видим и здесь}
\EN{ are those instructions, to the second operand of which shift coefficient may be applied}: 
(\TT{LSL\#4}). 
\RU{Но этот коэффициент можно применить только ко второму операнду.}
\EN{But this coefficient may be applied only to second operand.}
\RU{Для коммутативных операций, таких как сложение или умножение, 
(там операнды можно менять местами и это не влияет на результат).}
\EN{That's fine for commutative operations like addition or multiplication 
(operands may be swapped there without result affecting).}
\RU{Но вычитание ~--- операция некоммутативная, так что, для этих случаев существует инструкция \TT{RSB}.}
\EN{But subtraction is non-commutative operation, so \TT{RSB} exist for these cases.}

\index{ARM!\Instructions!LDR.W}
\RU{Инструкция }\TT{``LDR.W R9, [R9]''} \RU{работает как}\EN{instruction works like} \LEA~(\ref{sec:LEA})
\RU{в x86, и здесь она ничего не делает, она избыточна.}
\EN{in x86, but it does nothing here, it is redundant.}
\RU{Вероятно, компилятор несоптимизировал её.}\EN{Apparently, compiler not optimized it.}
\fi
