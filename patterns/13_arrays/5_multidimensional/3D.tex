\subsection{\RU{Пример с трехмерным массивом}\EN{Three-dimensional array example}}

\RU{То же самое и для многомерных массивов.}\EN{It's thing in multidimensional arrays.}
\RU{На этот раз будем работать с массивом типа \Tint: каждый элемент требует 4 байта в памяти.}
\EN{Now we are going to work with an array of type \Tint: each element requires 4 bytes in memory.}

\RU{Попробуем}\EN{Let's see}:

\lstinputlisting[caption=\RU{простой пример}\EN{simple example}]{patterns/13_arrays/5_multidimensional/multi.c}

\subsubsection{x86}

\RU{В итоге}\EN{We get} (MSVC 2010):

\lstinputlisting[caption=MSVC 2010]{patterns/13_arrays/5_multidimensional/multi_msvc.asm.\LANG}

\RU{В принципе, ничего удивительного. В \TT{insert()} для вычисления адреса нужного элемента массива 
три входных аргумента перемножаются по формуле $address=600 \cdot 4 \cdot x + 30 \cdot 4 \cdot y + 4z$, 
чтобы представить массив трехмерным.
Не забывайте также, что тип \Tint 32-битный (4 байта), поэтому все коэффициенты нужно умножить на 4.}
\EN{Nothing special. For index calculation, three input arguments are used 
in the formula $address=600 \cdot 4 \cdot x + 30 \cdot 4 \cdot y + 4z$, to represent the array as multidimensional.
Do not forget that the \Tint type is 32-bit (4 bytes),
so all coefficients must be multiplied by 4.}

\lstinputlisting[caption=GCC 4.4.1]{patterns/13_arrays/5_multidimensional/multi_gcc.asm.\LANG}

\RU{Компилятор GCC решил всё сделать немного иначе}\EN{The GCC compiler does it differently}.
\RU{Для вычисления одной из операций ($30y$), GCC создал код, где нет самой операции умножения.}
\EN{For one of the operations in the calculation ($30y$), GCC produces code without multiplication instructions.}
\RU{Происходит это так}\EN{This is how it done}: 
$(y+y) \ll 4 - (y+y) = (2y) \ll 4 - 2y = 2 \cdot 16 \cdot y - 2y = 32y - 2y = 30y$. 
\RU{Таким образом, для вычисления $30y$ используется только операция сложения, 
операция битового сдвига и операция вычитания.}\EN{Thus, for the $30y$ calculation, only one addition operation,
one bitwise shift operation and one subtraction operation are used.}
\RU{Это работает быстрее}\EN{This works faster}.

\ifdefined\IncludeARM
\subsubsection{ARM + \NonOptimizingXcodeIV (\ThumbMode)}

\lstinputlisting[caption=\NonOptimizingXcodeIV (\ThumbMode)]{patterns/13_arrays/5_multidimensional/multi_Xcode_thumb_O0.asm.\LANG}

\NonOptimizing LLVM \RU{сохраняет все переменные в локальном стеке, хотя это и избыточно.}
\EN{saves all variables in local stack, which is redundant.}
\RU{Адрес элемента массива вычисляется по уже рассмотренной формуле.}
\EN{The address of the array element is calculated by the formula we already saw.}

\subsubsection{ARM + \OptimizingXcodeIV (\ThumbMode)}

\lstinputlisting[caption=\OptimizingXcodeIV (\ThumbMode)]{patterns/13_arrays/5_multidimensional/multi_Xcode_thumb_O3.asm.\LANG}

\RU{Тут используются уже описанные трюки для замены умножения на операции сдвига, сложения и вычитания.}
\EN{The tricks for replacing multiplication by shift, addition and subtraction which we already saw
are also present here.}

\myindex{ARM!\Instructions!RSB}
\myindex{ARM!\Instructions!SUB}
\RU{Также мы видим новую для себя инструкцию}\EN{Here we also see a new instruction for us:} 
\RSB (\IT{Reverse Subtract}).
\RU{Она работает так же, как и \SUB, только меняет операнды местами.}
\EN{It works just as \SUB, but it swaps its operands with each other before execution.}
\RU{Зачем?}\EN{Why?}
\myindex{ARM!Optional operators!LSL}
\SUB \AndENRU \RSB\RU{ это те инструкции, ко второму операнду которых 
можно применить коэффициент сдвига, как мы видим и здесь}%
\EN{ are instructions, to the second operand of which shift coefficient may be applied}: 
(\INS{LSL\#4}). 
\RU{Но этот коэффициент можно применить только ко второму операнду.}
\EN{But this coefficient can be applied only to second operand.}
\RU{Для коммутативных операций, таких как сложение или умножение, 
операнды можно менять местами и это не влияет на результат.}
\EN{That's fine for commutative operations like addition or multiplication 
(operands may be swapped there without changing the result).}
\RU{Но вычитание~--- операция некоммутативная, так что для этих случаев существует инструкция \RSB.}
\EN{But subtraction is a non-commutative operation, so \RSB exist for these cases.}

\myindex{ARM!\Instructions!LDR.W}
\EN{The}\RU{Инструкция} \INS{LDR.W R9, [R9]} \RU{работает как}\EN{instruction works like} \LEA~(\myref{sec:LEA})
\RU{в x86, и здесь она ничего не делает, она избыточна.}
\EN{in x86, but it does nothing here, it is redundant.}
\RU{Вероятно, компилятор не оптимизировал её.}\EN{Apparently, the compiler did not optimize it out.}
\fi

\ifdefined\IncludeMIPS
\subsubsection{MIPS}

\myindex{MIPS!Global Pointer}
\EN{My example is tiny, so the GCC compiler decided to put the $a$ array into the 64KiB area 
addressable by the Global Pointer.}
\RU{Мой пример такой крошечный, что компилятор GCC решил разместить массив $a$ в 64KiB-области,
адресуемой при помощи Global Pointer.}

\lstinputlisting[caption=\Optimizing GCC 4.4.5 (IDA)]{patterns/13_arrays/5_multidimensional/multi_MIPS_O3_IDA.lst.\LANG}

\fi
