\subsubsection{Three-dimensional array example}

It's the same for multidimensional arrays.

Now we are going to work with an array of type \Tint: each element requires 4 bytes in memory.

Let's see:

\lstinputlisting[caption=simple example]{patterns/13_arrays/5_multidimensional/multi.c}

\myparagraph{x86}

We get (MSVC 2010):

\lstinputlisting[caption=MSVC 2010]{patterns/13_arrays/5_multidimensional/multi_msvc_EN.asm}

Nothing special. For index calculation, three input arguments are used 
in the formula $address=600 \cdot 4 \cdot x + 30 \cdot 4 \cdot y + 4z$, to represent the array as multidimensional.
Do not forget that the \Tint type is 32-bit (4 bytes),
so all coefficients must be multiplied by 4.

\lstinputlisting[caption=GCC 4.4.1]{patterns/13_arrays/5_multidimensional/multi_gcc_EN.asm}

The GCC compiler does it differently.

For one of the operations in the calculation ($30y$), GCC produces code without multiplication instructions.
This is how it done: 
$(y+y) \ll 4 - (y+y) = (2y) \ll 4 - 2y = 2 \cdot 16 \cdot y - 2y = 32y - 2y = 30y$. 
Thus, for the $30y$ calculation, only one addition operation,
one bitwise shift operation and one subtraction operation are used.
This works faster.

\myparagraph{ARM + \NonOptimizingXcodeIV (\ThumbMode)}

\lstinputlisting[caption=\NonOptimizingXcodeIV (\ThumbMode)]{patterns/13_arrays/5_multidimensional/multi_Xcode_thumb_O0_EN.asm}

\NonOptimizing LLVM saves all variables in local stack, which is redundant.

The address of the array element is calculated by the formula we already saw.

\myparagraph{ARM + \OptimizingXcodeIV (\ThumbMode)}

\lstinputlisting[caption=\OptimizingXcodeIV (\ThumbMode)]{patterns/13_arrays/5_multidimensional/multi_Xcode_thumb_O3_EN.asm}

The tricks for replacing multiplication by shift, addition and subtraction which we already saw
are also present here.

\myindex{ARM!\Instructions!RSB}
\myindex{ARM!\Instructions!SUB}
Here we also see a new instruction for us: \RSB (\IT{Reverse Subtract}).

It works just as \SUB, but it swaps its operands with each other before execution.
Why?
\myindex{ARM!Optional operators!LSL}
\SUB and \RSB  are instructions, to the second operand of which shift coefficient may be applied: (\INS{LSL\#4}). 

But this coefficient can be applied only to second operand.

That's fine for commutative operations like addition or multiplication 
(operands may be swapped there without changing the result).

But subtraction is a non-commutative operation, so \RSB exist for these cases.

\myparagraph{MIPS}

\myindex{MIPS!Global Pointer}
My example is tiny, so the GCC compiler decided to put the $a$ array into the 64KiB area 
addressable by the Global Pointer.

\lstinputlisting[caption=\Optimizing GCC 4.4.5 (IDA)]{patterns/13_arrays/5_multidimensional/multi_MIPS_O3_IDA_EN.lst}

