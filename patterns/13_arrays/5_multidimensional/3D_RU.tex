\subsubsection{Пример с трехмерным массивом}

То же самое и для многомерных массивов.
На этот раз будем работать с массивом типа \Tint: каждый элемент требует 4 байта в памяти.

Попробуем:

\lstinputlisting[caption=простой пример]{patterns/13_arrays/5_multidimensional/multi.c}

\myparagraph{x86}

В итоге (MSVC 2010):

\lstinputlisting[caption=MSVC 2010]{patterns/13_arrays/5_multidimensional/multi_msvc_RU.asm}

В принципе, ничего удивительного. В \TT{insert()} для вычисления адреса нужного элемента массива 
три входных аргумента перемножаются по формуле $address=600 \cdot 4 \cdot x + 30 \cdot 4 \cdot y + 4z$, 
чтобы представить массив трехмерным.
Не забывайте также, что тип \Tint 32-битный (4 байта), поэтому все коэффициенты нужно умножить на 4.

\lstinputlisting[caption=GCC 4.4.1]{patterns/13_arrays/5_multidimensional/multi_gcc_RU.asm}

Компилятор GCC решил всё сделать немного иначе.
Для вычисления одной из операций ($30y$), GCC создал код, где нет самой операции умножения.

Происходит это так: 
$(y+y) \ll 4 - (y+y) = (2y) \ll 4 - 2y = 2 \cdot 16 \cdot y - 2y = 32y - 2y = 30y$. 
Таким образом, для вычисления $30y$ используется только операция сложения, 
операция битового сдвига и операция вычитания.
Это работает быстрее.

\myparagraph{ARM + \NonOptimizingXcodeIV (\ThumbMode)}

\lstinputlisting[caption=\NonOptimizingXcodeIV (\ThumbMode)]{patterns/13_arrays/5_multidimensional/multi_Xcode_thumb_O0_RU.asm}

\NonOptimizing LLVM сохраняет все переменные в локальном стеке, хотя это и избыточно.

Адрес элемента массива вычисляется по уже рассмотренной формуле.

\myparagraph{ARM + \OptimizingXcodeIV (\ThumbMode)}

\lstinputlisting[caption=\OptimizingXcodeIV (\ThumbMode)]{patterns/13_arrays/5_multidimensional/multi_Xcode_thumb_O3_RU.asm}

Тут используются уже описанные трюки для замены умножения на операции сдвига, сложения и вычитания.

\myindex{ARM!\Instructions!RSB}
\myindex{ARM!\Instructions!SUB}
Также мы видим новую для себя инструкцию \RSB (\IT{Reverse Subtract}).
Она работает так же, как и \SUB, только меняет операнды местами.

Зачем?
\myindex{ARM!Optional operators!LSL}
\SUB и \RSB это те инструкции, ко второму операнду которых можно применить коэффициент сдвига, как мы видим и здесь: (\INS{LSL\#4}). 
Но этот коэффициент можно применить только ко второму операнду.

Для коммутативных операций, таких как сложение или умножение, 
операнды можно менять местами и это не влияет на результат.

Но вычитание~--- операция некоммутативная, так что для этих случаев существует инструкция \RSB.

\myindex{ARM!\Instructions!LDR.W}
Инструкция \INS{LDR.W R9, [R9]} работает как \LEA~(\myref{sec:LEA}) в x86, и здесь она ничего не делает, она избыточна.

Вероятно, компилятор не оптимизировал её.

\myparagraph{MIPS}

\myindex{MIPS!Global Pointer}

Мой пример такой крошечный, что компилятор GCC решил разместить массив $a$ в 64KiB-области,
адресуемой при помощи Global Pointer.

\lstinputlisting[caption=\Optimizing GCC 4.4.5 (IDA)]{patterns/13_arrays/5_multidimensional/multi_MIPS_O3_IDA_RU.lst}

