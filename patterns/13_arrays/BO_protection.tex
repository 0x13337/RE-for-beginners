\section{\IFRU{Защита от переполнения буфера}{Buffer overflow protection methods}}
\label{subsec:BO_protection}

\newcommand{\URLWPB}{\href{http://en.wikipedia.org/wiki/Buffer_overflow_protection}
{Wikipedia: \IFRU{описания защит, которые компилятор может вставлять в код}
{compiler-side buffer overflow protection methods}}}

\IFRU{В наше время пытаются бороться с этой напастью, не взирая на халатность программистов на \CCpp. 
В MSVC есть опции вроде\footnote{\URLWPB}:}
{There are several methods to protect against it, regardless of \CCpp programmers' negligence.
MSVC has options like\footnote{\URLWPB}:}

\begin{lstlisting}
 /RTCs Stack Frame runtime checking
 /GZ Enable stack checks (/RTCs)
\end{lstlisting}

\index{x86!\Instructions!RET}
\index{Function prologue}
\index{Security cookie}
\IFRU{Один из методов, это в прологе функции вставлять в область локальных переменных 
некоторое случайное значение 
и в эпилоге функции, перед выходом, это число проверять. 
И если проверка не прошла, то не выполнять инструкцию \RET а остановиться (или зависнуть). 
Процесс зависнет, но это лучше, чем удаленная атака на ваш хост.}
{One of the methods is to write random value among local variables to stack at function prologue 
and to check it in function epilogue before function exiting.
And if value is not the same, do not execute last instruction \RET, but halt (or hang).
Process will hang, but that is much better then remote attack to your host.}
    
\newcommand{\CANARYURL}
{
    \IFRU
    {
        \href{http://miningwiki.ru/wiki/\%D0\%9A\%D0\%B0\%D0\%BD\%D0\%B0\%D1\%80\%D0\%B5\%D0\%B9\%D0\%BA\%D0\%B0_\%D0\%B2_\%D1\%88\%D0\%B0\%D1\%85\%D1\%82\%D0\%B5}{Шахтерская энциклопедия: Канарейка в шахте}
    }
    {
        \href{http://en.wikipedia.org/wiki/Domestic\_Canary\#Miner.27s\_canary}{Wikipedia: Miner's canary}
    }
}

\index{Canary}
\IFRU{Это случайное значение иногда называют ``канарейкой''\footnote{``canary'' в англоязычной литературе}, 
по аналогии с шахтной канарейкой\footnote{\CANARYURL},
их использовали шахтеры в свое время, чтобы определять, есть ли в шахте опасный газ.
}
{This random value is called ``canary'' sometimes, it is related to miner's canary\footnote{\CANARYURL},
they were used by miners in these days, in order to detect poisonous gases quickly.}
\IFRU{Канарейки очень к нему чувствительны и либо проявляли сильное беспокойство, либо гибли от газа.}
{Canaries are very sensetive to mine gases, they become very agitated in case of danger, or even dead.}

\IFRU{Если скомпилировать наш простейший пример работы с массивом}
{If to compile our very simple array example}~(\ref{arrays_simple}) \InENRU \ac{MSVC} 
\IFRU{с опцией RTC1 или RTCs}{with RTC1 and RTCs option}, \IFRU{в конце функции будет вызов 
функции}{you will see call to} \TT{@\_RTC\_CheckStackVars@8}\IFRU{, проверяющей корректность ``канарейки''.}
{ function at the function end, checking ``canary'' correctness.}

\IFRU{Посмотрим, как дела обстоят в GCC}{Let's see how GCC handles this}. 
\IFRU{Возьмем пример из секции про}{Let's take} \TT{alloca()}~(\ref{alloca})\EN{ example}:

\lstinputlisting{patterns/02_stack/2_1.c}

\IFRU{По умолчанию, без дополнительных ключей, GCC 4.7.3 вставит в код проверку ``канарейки''}
{By default, without any additional options, GCC 4.7.3 will insert ``canary'' check into code}:

\lstinputlisting[caption=GCC 4.7.3]{patterns/13_arrays/gcc_canary.asm}

\IFRU{Случайное значение находится в}{Random value is located in} \TT{gs:20}. 
\IFRU{Оно записывается в стек, затем, в конце функции, значение в стеке
сравнивается с корректной ``канарейкой'' в}{It is to be written on the stack and then, at the function end,
value in the stack is compared with correct ``canary'' in} \TT{gs:20}. 
\IFRU{Если значения не равны, будет вызвана функция}{If values are not equal to each other, } 
\TT{\_\_stack\_chk\_fail}\IFRU{ и в консоли мы увидим что-то вроде такого}{ function will be called and we will see
something like that in console} (Ubuntu 13.04 x86):

\begin{lstlisting}
*** buffer overflow detected ***: ./2_1 terminated
======= Backtrace: =========
/lib/i386-linux-gnu/libc.so.6(__fortify_fail+0x63)[0xb7699bc3]
/lib/i386-linux-gnu/libc.so.6(+0x10593a)[0xb769893a]
/lib/i386-linux-gnu/libc.so.6(+0x105008)[0xb7698008]
/lib/i386-linux-gnu/libc.so.6(_IO_default_xsputn+0x8c)[0xb7606e5c]
/lib/i386-linux-gnu/libc.so.6(_IO_vfprintf+0x165)[0xb75d7a45]
/lib/i386-linux-gnu/libc.so.6(__vsprintf_chk+0xc9)[0xb76980d9]
/lib/i386-linux-gnu/libc.so.6(__sprintf_chk+0x2f)[0xb7697fef]
./2_1[0x8048404]
/lib/i386-linux-gnu/libc.so.6(__libc_start_main+0xf5)[0xb75ac935]
======= Memory map: ========
08048000-08049000 r-xp 00000000 08:01 2097586    /home/dennis/2_1
08049000-0804a000 r--p 00000000 08:01 2097586    /home/dennis/2_1
0804a000-0804b000 rw-p 00001000 08:01 2097586    /home/dennis/2_1
094d1000-094f2000 rw-p 00000000 00:00 0          [heap]
b7560000-b757b000 r-xp 00000000 08:01 1048602    /lib/i386-linux-gnu/libgcc_s.so.1
b757b000-b757c000 r--p 0001a000 08:01 1048602    /lib/i386-linux-gnu/libgcc_s.so.1
b757c000-b757d000 rw-p 0001b000 08:01 1048602    /lib/i386-linux-gnu/libgcc_s.so.1
b7592000-b7593000 rw-p 00000000 00:00 0
b7593000-b7740000 r-xp 00000000 08:01 1050781    /lib/i386-linux-gnu/libc-2.17.so
b7740000-b7742000 r--p 001ad000 08:01 1050781    /lib/i386-linux-gnu/libc-2.17.so
b7742000-b7743000 rw-p 001af000 08:01 1050781    /lib/i386-linux-gnu/libc-2.17.so
b7743000-b7746000 rw-p 00000000 00:00 0
b775a000-b775d000 rw-p 00000000 00:00 0
b775d000-b775e000 r-xp 00000000 00:00 0          [vdso]
b775e000-b777e000 r-xp 00000000 08:01 1050794    /lib/i386-linux-gnu/ld-2.17.so
b777e000-b777f000 r--p 0001f000 08:01 1050794    /lib/i386-linux-gnu/ld-2.17.so
b777f000-b7780000 rw-p 00020000 08:01 1050794    /lib/i386-linux-gnu/ld-2.17.so
bff35000-bff56000 rw-p 00000000 00:00 0          [stack]
Aborted (core dumped)
\end{lstlisting}

\index{MS-DOS}
gs\EMDASH\IFRU{это так называемый сегментный регистр, эти регистры широко использовались во времена MS-DOS 
и DOS-экстендеров.}{is so-called segment register, these registers were used widely in MS-DOS and DOS-extenders
times.}
\IFRU{Сейчас их функция немного изменилась.}{Today, its function is different.}
\index{TLS}
\index{Windows!TIB}
\IFRU{Если говорить коротко, в Linux \TT{gs} всегда указывает на \ac{TLS}~(\ref{TLS}) ~--- там находится различная 
информация, специфичная для выполняющегося треда}
{If to say briefly, the \TT{gs} register in Linux is always pointing to the
\ac{TLS}~(\ref{TLS})~---various information specific to thread is stored there}
(\IFRU{кстати, в win32 эту же роль играет сегментный регистр \TT{fs},
он всегда указывает на}{by the way, in win32 environment,
the \TT{fs} register plays the same role, it pointing to}
\ac{TIB} \footnote{\url{https://en.wikipedia.org/wiki/Win32_Thread_Information_Block}}). 

\IFRU{Больше информации можно почерпнуть из исходных кодов Linux (по крайней мере, в версии 3.11): 
в файле}{More information can be found in Linux source codes (at least in 3.11 version), in}
\IT{arch/x86/include/asm/stackprotector.h}\IFRU{ в комментариях описывается эта переменная}
{ file this variable is described in comments}.

\subsubsection{\OptimizingXcode + \ThumbTwoMode}

\IFRU{Возвращаясь к нашему простому примеру}{Let's back to our simple array example}~(\ref{arrays_simple}),
\IFRU{можно посмотреть, как LLVM добавит проверку ``канарейки''}
{again, now we can see how LLVM will check ``canary'' correctness}:

\lstinputlisting{patterns/13_arrays/simple_Xcode_thumb_O3_en.asm}

\index{Unrolled loop}
\IFRU{Во-первых, как видно, LLVM ``развернул'' цикл и все значения записываются в массив по одному, 
уже вычисленные, 
потому что LLVM посчитал что так будет быстрее.}
{First of all, as we see, LLVM made loop ``unrolled'' and all values are written into array one-by-one,
already calculated since LLVM concluded it will be faster.}
\IFRU{Кстати, инструкции режима ARM позволяют сделать это еще быстрее и это может быть вашим 
домашним заданием.}{By the way, ARM mode instructions may help to do this even faster, 
and finding this way could be your homework.}

\IFRU{В конце функции мы видим сравнение ``канареек'' ~--- той что лежит в локальном стеке и корректной, 
на которую ссылается регистр \TT{R8}.}
{At the function end wee see ``canaries'' comparison~---that laying in local stack and correct one,
to which the \TT{R8} register pointing.}
\index{ARM!\Instructions!IT}
\IFRU{Если они равны, срабатывает блок из четырех инструкций при помощи \TT{``ITTTT EQ''}, это запись
$0$ в \Rzero, эпилог функции и выход из нее.}
{If they are equal to each other, 4-instruction block is triggered by \TT{``ITTTT EQ''}, it is
writing $0$ into \Rzero, function epilogue and exit.}
\IFRU{Если ``канарейки'' не равны, блок не срабатывает и происходит
переход на функцию}{If ``canaries'' are not equal, block will not be triggered,
and jump to} \TT{\_\_\_stack\_chk\_fail}\IFRU{, которая, вероятно, остановит работу программы.}
{ function will be occurred, which, as I suppose, will halt execution.}
% TODO illustrate this!


