\section{\RU{Многомерные массивы}\EN{Multidimensional arrays}}

\RU{Внутри, многомерный массив выглядит так же, как и линейный.}
\EN{Internally, multidimensional array is essentially the same thing as linear array.}

\RU{Ведь память компьютера линейная, это одномерный массив.
Но для удобства, этот одномерный массив легко представить как многомерный.}
\EN{Since computer memory in linear, it is one-dimensional array.
But this one-dimensional array can be easily represented as multidimensional for convenience.}

\RU{К примеру, элементы массива $a[3][4]$ будут так расположены в одномерном массиве из 12-и ячеек:}
\EN{For example, that is how $a[3][4]$ array elements will be placed in one-dimensional array of 12 cells:}

\begin{center}
\begin{tabular}{ | l | }
\hline
[0][0] \\
\hline
[0][1] \\
\hline
[0][2] \\
\hline
[0][3] \\
\hline
[1][0] \\
\hline
[1][1] \\
\hline
[1][2] \\
\hline
[1][3] \\
\hline
[2][0] \\
\hline
[2][1] \\
\hline
[2][2] \\
\hline
[2][3] \\
\hline
\end{tabular}
\end{center}

\RU{Вот как можно представить двухмерный массив с порядковыми номерами элементов в одномерном 
линейном массиве в памяти}\EN{That is how two-dimensional array with one-dimensional (memory) array index numbers 
can be represented}:

\begin{center}
\begin{tabular}{ | l | l | l | l | }
\hline                        
0 & 1 & 2 & 3 \\
\hline  
4 & 5 & 6 & 7 \\
\hline  
8 & 9 & 10 & 11 \\
\hline  
\end{tabular}
\end{center}

\index{row-major order}
\RU{То есть, чтобы адресовать нужный элемент, в начале умножаем первый индекс на 4 (ширину матрицы), 
затем прибавляем второй индекс.}\EN{So, in order to address elements we need, first multiply first index by
4 (matrix width) and then add second index.}
\RU{Это называется}\EN{That's called} \IT{row-major order}, 
\RU{и такой способ представления массивов и матриц используется по крайней мере в}
\EN{and this method of arrays and matrices representation is used in at least in} \CCpp, Python. 
\RU{Термин }\IT{row-major order} \RU{означает по-русски
примерно следующее: ``в начале записываем элементы первой строки, затем второй \dots и элементы последней 
строки в самом конце''.}
\EN{term in plain English language means: ``first, write elements of first row, then second row \dots 
and finally elements of last row''.}

\index{column-major order}
\index{FORTRAN}
\RU{Другой способ представления называется}\EN{Another method of representation called} 
\IT{column-major order} (\RU{индексы массива используются в обратном порядке}
\EN{array indices used in reverse order}) \RU{и это используется по крайней мере в}\EN{and it is used at least in} 
FORTRAN, MATLAB, R. 
\RU{Термин }\IT{column-major order} \RU{означает по-русски
следующее: ``в начале записываем элементы первого столбца, затем второго \dots и элементы последнего столбца
в самом конце''.}
\EN{term in plain English language means: ``first, write elements of first column, then second column \dots
and finally elements of last column''.}

\RU{То же самое и для многомерных массивов.}\EN{Same thing about multidimensional arrays.}

\RU{Попробуем}\EN{Let's see}:

\lstinputlisting[caption=\RU{простой пример}\EN{simple example}]{patterns/13_arrays/multi.c}

\subsection{x86}

\RU{В итоге}\EN{We got} (MSVC 2010):

\lstinputlisting[caption=MSVC 2010]{patterns/13_arrays/multi_msvc.asm}

\RU{В принципе, ничего удивительного. В \TT{insert()} для вычисления адреса нужного элемента массива, 
три входных аргумента перемножаются по формуле $address=600 \cdot 4 \cdot x + 30 \cdot 4 \cdot y + 4z$, 
чтобы представить массив трехмерным.
Не забывайте также, что тип \Tint 32-битный (4 байта), поэтому все коэффициенты нужно умножить на 4.}
\EN{Nothing special. For index calculation, three input arguments are multiplying 
by formula $address=600 \cdot 4 \cdot x + 30 \cdot 4 \cdot y + 4z$ to represent array as multidimensional.
Do not forget the \Tint type is 32-bit (4 bytes),
so all coefficients must be multiplied by 4.}

\lstinputlisting[caption=GCC 4.4.1]{patterns/13_arrays/multi_gcc.asm}

\RU{Компилятор GCC решил всё сделать немного иначе}\EN{GCC compiler does it differently}.
\RU{Для вычисления одной из операций ($30y$), GCC создал код, где нет самой операции умножения.}
\EN{For one of operations calculating ($30y$), GCC produced a code without multiplication instruction.}
\RU{Происходит это так}\EN{This is how it done}: 
$(y+y) \ll 4 - (y+y) = (2y) \ll 4 - 2y = 2 \cdot 16 \cdot y - 2y = 32y - 2y = 30y$. 
\RU{Таким образом, для вычисления $30y$ используется только операция сложения, 
операция битового сдвига и операция вычитания.}\EN{Thus, for $30y$ calculation, only one addition operation
used, one bitwise shift operation and one subtraction operation.}
\RU{Это работает быстрее}\EN{That works faster}.

\subsection{ARM + \NonOptimizingXcode + \ThumbMode}

\lstinputlisting[caption=\NonOptimizingXcode + \ThumbMode]{patterns/13_arrays/multi_Xcode_thumb_O0_\LANG.asm}

\NonOptimizing LLVM \RU{сохраняет все переменные в локальном стеке, хотя это и избыточно.}
\EN{saves all variables in local stack, however, it is redundant.}
\RU{Адрес элемента массива вычисляется по уже рассмотренной формуле.}
\EN{Address of array element is calculated by formula we already figured out.}

\subsection{ARM + \OptimizingXcode + \ThumbMode}

\lstinputlisting[caption=\OptimizingXcode + \ThumbMode]{patterns/13_arrays/multi_Xcode_thumb_O3_\LANG.asm}

\RU{Тут используются уже описанные трюки для замены умножения на операции сдвига, сложения и вычитания.}
\EN{Here is used tricks for replacing multiplication by shift, addition and subtraction we already considered.}

\index{ARM!\Instructions!RSB}
\index{ARM!\Instructions!SUB}
\RU{Также мы видим новую для себя инструкцию}\EN{Here we also see new instruction for us:} 
\TT{RSB} (\IT{Reverse Subtract}).
\RU{Она работает так же, как и \SUB, только меняет операнды местами.}
\EN{It works just as \SUB, but swapping operands with each other.}
\RU{Зачем?}\EN{Why?}
\index{ARM!Optional operators!LSL}
\SUB, \TT{RSB}, \RU{это те инструкции, ко второму операнду которых можно применить коэффициент сдвига, 
как мы видим и здесь}
\EN{are those instructions, to the second operand of which shift coefficient may be applied}: (\TT{LSL\#4}). 
\RU{Но этот коэффициент можно применить только ко второму операнду.}
\EN{But this coefficient may be applied only to second operand.}
\RU{Для коммутативных операций, таких как сложение или умножение, 
там операнды можно менять местами и это не влияет на результат.}
\EN{That's fine for commutative operations like addition or multiplication, operands may be swapped there
without result affecting.}
\RU{Но вычитание ~--- операция некоммутативная, так что, для этих случаев существует инструкция \TT{RSB}.}
\EN{But subtraction is non-commutative operation, so, for these cases, \TT{RSB} exist.}

\index{ARM!\Instructions!LDR.W}
\RU{Инструкция }\TT{``LDR.W R9, [R9]''} \RU{работает как}\EN{instruction works like} \LEA~(\ref{sec:LEA})
\RU{в x86, и здесь она ничего не делает, она избыточна.}\EN{in x86, but here it does nothing, it is redundant.}
\RU{Вероятно, компилятор несоптимизировал её.}\EN{Apparently, compiler not optimized it.}

