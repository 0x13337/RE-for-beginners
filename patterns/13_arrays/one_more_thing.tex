\subsection{\IFRU{Еще немного о массивах}{One more word about arrays}}

\IFRU{Теперь понятно, почему нельзя написать в исходном коде на \CCpp что-то вроде
\footnote{Впрочем, по стандарту C99 это возможно\cite[6.7.5/2]{C99TC3}: 
GCC может это сделать выделяя место под массив динамически в стеке (как alloca()~(\ref{alloca}))}}
{Now we understand, why it is not possible to write something like that in \CCpp code
\footnote{However, it is possible in C99 standard\cite[6.7.5/2]{C99TC3}: 
GCC is actually do this by allocating array dynammically on the stack (like alloca()~(\ref{alloca}))}}:

\begin{lstlisting}
void f(int size)
{
    int a[size];
...
};
\end{lstlisting}

\IFRU{Все просто потому, чтобы выделять место под массив в локальном стеке или же сегменте данных 
(если массив глобальный), компилятору нужно знать его размер, чего он, на стадии компиляции, 
разумеется знать не может.}
{That's just because compiler must know exact array size to allocate space for it in local stack layout or
in data segment (in case of global variable) on compiling stage.}

\index{\CLanguageElements!C99!variable length arrays}
\index{\CStandardLibrary!alloca()}
\IFRU{Если вам нужен массив произвольной длины, то выделите столько, сколько нужно, через \TT{malloc()}, 
затем обращайтесь к выделенному блоку байт как к массиву того типа, который вам нужен.
Либо используйте возможность стандарта C99\cite[6.7.5/2]{C99TC3}, 
но внутри это будет похоже на alloca()~(\ref{alloca})}
{If you need array of arbitrary size, allocate it by \TT{malloc()}, then access allocated memory block
as array of variables of type you need.
Or use C99 standard feature\cite[6.7.5/2]{C99TC3}, 
but it will be looks like alloca()~(\ref{alloca}) internally.}

