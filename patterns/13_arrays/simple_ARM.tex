\subsection{ARM + \NonOptimizingKeil + \ARMMode}

\lstinputlisting{patterns/13_arrays/simple_Keil_ARM_O0_\LANG.asm}

\RU{Тип \Tint требует 32 бита для хранения, или 4 байта}\EN{\Tint type requires 32 bits for storage,
or 4 bytes}, 
\RU{так что для хранения 20 переменных типа \Tint, нужно $80$ (\TT{0x50}) байт}
\EN{so for storage of 20 \Tint variables, $80$ (\TT{0x50}) bytes are needed},
\RU{поэтому инструкция}\EN{so that is why} \TT{``SUB SP, SP, \#0x50''} 
\RU{в эпилоге функции выделяет в локальном стеке под массив
именно столько места.}
\EN{instruction in function epilogue allocates exactly this amount of space in local stack.}

\RU{И в первом и во втором цикле, итератор цикла $i$ будет постоянно находится в регистре \Reg{4}.}
\EN{In both first and second loops, $i$ loop iterator will be placed in the \Reg{4} register.}

\RU{Число, которое нужно записать в массив}\EN{A number to be written into array},
\RU{вычисляется так $i*2$, и это эквивалентно сдвигу на 1 бит влево}
\EN{is calculating as $i*2$ which is effectively equivalent to shifting left by one bit},
\index{ARM!Optional operators!LSL}
\RU{инструкция}\EN{so} \TT{``MOV R0, R4,LSL\#1''} \RU{делает это}\EN{instruction do this}.

\index{ARM!\Instructions!STR}
\TT{``STR R0, [SP,R4,LSL\#2]''} \RU{записывает содержимое \Reg{0} в массив}\EN{writes \Reg{0} contents into array}.
\RU{Указатель на элемент массива вычисляется так: \ac{SP} указывает на начало массива, \Reg{4} это $i$.}
\EN{Here is how a pointer to array element is to be calculated: \ac{SP} pointing to array begin, \Reg{4} is $i$.}
\RU{Так что сдвигаем $i$ на 2 бита влево, что эквивалентно умножению на $4$ 
(ведь каждый элемент массива занимает 4 байта) и прибавляем это к адресу начала массива.}
\EN{So shift $i$ left by 2 bits, that is effectively equivalent to multiplication by $4$
(since each array element has size of 4 bytes) and add it to address of array begin.}

\index{ARM!\Instructions!LDR}
\RU{Во втором цикле используется обратная инструкция}\EN{The second loop has inverse} 
\TT{``LDR R2, [SP,R4,LSL\#2]''}, 
\RU{она загружает из массива нужное значение, и указатель на него вычисляется точно так же.}
\EN{instruction, it loads from array value we need, and the pointer to it is calculated likewise.}

\subsection{ARM + \OptimizingKeil + \ThumbMode}

\lstinputlisting{patterns/13_arrays/simple_Keil_thumb_O3_\LANG.asm}

\RU{Код для thumb очень похожий.}\EN{Thumb code is very similar.}
\index{ARM!\Instructions!LSLS}
\RU{В thumb имеются отдельные инструкции для битовых сдвигов (как \TT{LSLS}), 
вычисляющие и число для записи в массив и адрес каждого элемента массива.}
\EN{Thumb mode has special instructions for bit shifting (like \TT{LSLS}),
which calculates value to be written into array and address of each element in array as well.}

\RU{Компилятор почему-то выделил в локальном стеке немного больше места, 
однако последние 4 байта не используются.}
\EN{Compiler allocates slightly more space in local stack, however, last 4 bytes are not used.}

