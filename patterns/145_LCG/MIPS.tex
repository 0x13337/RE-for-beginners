\section{MIPS}

\lstinputlisting[caption=\Optimizing GCC 4.4.5 (IDA)]{patterns/145_LCG/MIPS_O3_IDA.lst.\LANG}

\RU{Ух, мы видим здесь только одну константу}
\EN{Wow, we see here only one constant} (0x3C6EF35F \OrENRU 1013904223).
\RU{Где же вторая}\EN{Where is another one} (1664525)?

\RU{Похоже, умножение на 1664525 сделано только при помощи сдвигов и прибавлений!}
\EN{It seems, multiplication by 1664525 is done using just shifts and additions!}
\RU{Проверим эту версию}\EN{Let's check this assumption}:

\lstinputlisting{patterns/145_LCG/test.c}

\lstinputlisting[caption=\Optimizing GCC 4.4.5 (IDA)]{patterns/145_LCG/test_O3_MIPS.lst}

\RU{Действительно}\EN{Indeed}!

\subsection{\RU{Релоки в MIPS}\EN{MIPS relocations}}

\RU{Еще поговорим о том, как на самом деле происходят операции загрузки из памяти и запись в память.}
\EN{I would also focus on how such operations as load from memory and store to memory are actually works.}
\RU{Листинги здесь были сделаны в IDA, которая убирает немного деталей.}
\EN{The listings here are produced by IDA, which hiding some details.}

\RU{Я запущу objdump дважды, чтобы получить дизассемблированный листинг и еще список релоков:}
\EN{I'll run objdump twice to get disassembled listing and also relocations list:}

\lstinputlisting[caption=\Optimizing GCC 4.4.5 (objdump)]{patterns/145_LCG/MIPS_O3_objdump.txt}

\RU{Рассмотрим два релока для ф-ции \TT{my\_srand()}.}
\EN{Let's consider two relocations for the \TT{my\_srand()} function.}
\RU{Первый, для адреса 0, имеет тип \TT{R\_MIPS\_HI16}, и второй, для адреса 8, имеет тип \TT{R\_MIPS\_LO16}.}
\EN{First for address 0 has type of \TT{R\_MIPS\_HI16} and second for address 8 has types \TT{R\_MIPS\_LO16}.}
\RU{Это значит, что адрес начала сегмента .bss будет записан в инструкцию по адресу 0 (старшая часть адреса)
и по адресу 8 (младшая асть адреса).}
\EN{That means that address of beginning of .bss segment will be written into instructions at
address of 0 (high part of address) and 8 (low part of address).}

\RU{Ведь переменная \TT{rand\_state} находится в самом начале сегмента .bss.}
\EN{\TT{rand\_state} variable is at the very start of .bss segment.}

\RU{Так что мы видим нули в операндах инструкций LUI и SW потому что там пока ничего нет --- 
компилятор не знает, что туда записать.}
\EN{So we see zeroes in the operands of instructions LUI and SW, because nothing is there yet --- 
compiler don't know what to write there.}
\RU{Линкер это исправит и старшая часть адреса будет записана в операнд инструкции LUI и младшая часть адреса ---
в операнд инструкции SW.}
\EN{Linker will fix this and high part of address will be written into LUI instruction operand and
low part of address --- to operand of SW instruction.}
\RU{SW просуммирует младшую асть адреса и то что находится в регистре \$V0 (там старшая часть).}
\EN{SW will sum up low part of address and what is in \$V0 register (high part is there).}

\RU{Та же история и с ф-цией my\_rand(): релок R\_MIPS\_HI16 указывает линкеру записать старшую часть
адреса сегмента .bss в инструкцию LUI.}
\EN{The same story about my\_rand() function: R\_MIPS\_HI16 relocation instructs linker to write high part
of .bss segment address into LUI instruction.}
\RU{Так что старшая часть адреса переменной rand\_state будет находится в регистре \$V1.}
\EN{So, high part of rand\_state variable address will reside in \$V1 register.}
\RU{Инструкция LW по адресу 0x10 просуммирует старшую и младшую часть и загрузит значение переменной 
rand\_state в \$V1.}
\EN{LW instruction at address 0x10 will sum up high and low part and load value of rand\_state 
variable into \$V1.}
\RU{Инструкция SW по адресу 0x54 также просуммирует и затем запишет новое значение в глобальную переменную
rand\_state.}
\EN{SW instruction at address 0x54 will also do the summing and then will store new value 
to rand\_state global variable.}

\RU{Так что IDA обрабатывает релоки при загрузке, и таким образом эти детали скрываются.}
\EN{So, IDA processes relocations while loading, thus hiding these details.}
\RU{Но мы должны о них помнить.}\EN{But we should remember about them.}

% TODO add example of compiled binary, GDB example, etc...
