\section{MIPS}

\lstinputlisting[caption=\Optimizing GCC 4.4.5 (IDA)]{patterns/145_LCG/MIPS_O3_IDA.lst.\LANG}

\RU{Ух, мы видим здесь только одну константу}
\EN{Wow, here we see only one constant} (0x3C6EF35F \OrENRU 1013904223).
\RU{Где же вторая}\EN{Where is the other one} (1664525)?

\RU{Похоже, умножение на 1664525 сделано только при помощи сдвигов и прибавлений!}
\EN{It seems that multiplication by 1664525 is done by just using shifts and additions!}
\RU{Проверим эту версию}\EN{Let's check this assumption}:

\lstinputlisting{patterns/145_LCG/test.c}

\lstinputlisting[caption=\Optimizing GCC 4.4.5 (IDA)]{patterns/145_LCG/test_O3_MIPS.lst}

\RU{Действительно}\EN{Indeed}!

\subsection{\RU{Перемещения в MIPS}\EN{MIPS relocations}}

\RU{Ещё поговорим о том, как на самом деле происходят операции загрузки из памяти и запись в память.}
\EN{I want also to focus on how such operations as load from memory and store to memory actually work.}
\RU{Листинги здесь были сделаны в IDA, которая убирает немного деталей.}
\EN{The listings here are produced by IDA, which hides some details.}

\RU{Я запущу objdump дважды: чтобы получить дизассемблированный листинг и список перемещений:}
\EN{I'll run objdump twice to get a disassembled listing and also relocations list:}

\lstinputlisting[caption=\Optimizing GCC 4.4.5 (objdump)]{patterns/145_LCG/MIPS_O3_objdump.txt}

\RU{Рассмотрим два перемещения для функции \TT{my\_srand()}.}
\EN{Let's consider the two relocations for the \TT{my\_srand()} function.}
\RU{Первое, для адреса 0, имеет тип \TT{R\_MIPS\_HI16}, и второе, для адреса 8, имеет тип \TT{R\_MIPS\_LO16}.}
\EN{The first one, for address 0 has a type of \TT{R\_MIPS\_HI16} and the second one for address 8 has a type of \TT{R\_MIPS\_LO16}.}
\RU{Это значит, что адрес начала сегмента .bss будет записан в инструкцию по адресу 0 (старшая часть адреса)
и по адресу 8 (младшая асть адреса).}
\EN{That implies that address of the beginning of the .bss segment is to be written into the instructions at
address of 0 (high part of address) and 8 (low part of address).}
\RU{Ведь переменная \TT{rand\_state} находится в самом начале сегмента .bss.}
\EN{The \TT{rand\_state} variable is at the very start of the .bss segment.}
\RU{Так что мы видим нули в операндах инструкций \LUI и \SW потому что там пока ничего нет~--- 
компилятор не знает, что туда записать.}
\EN{So we see zeroes in the operands of instructions \LUI and \SW, because nothing is there yet---%
the compiler don't know what to write there.}
\RU{Линкер это исправит и старшая часть адреса будет записана в операнд инструкции \LUI и младшая часть адреса~---
в операнд инструкции \SW.}
\EN{The linker will fix this, and the high part of the address will be written into the operand of \LUI and
the low part of the address---to the operand of \SW.}
\RU{\SW просуммирует младшую асть адреса и то что находится в регистре \$V0 (там старшая часть).}
\EN{\SW will sum up the low part of the address and what is in register \$V0 (the high part is there).}

\RU{Та же история и с функцией my\_rand(): перемещение R\_MIPS\_HI16 указывает линкеру записать старшую часть
адреса сегмента .bss в инструкцию \LUI.}
\EN{It's the same story with the my\_rand() function: R\_MIPS\_HI16 relocation instructs the linker to write the high part
of the .bss segment address into instruction \LUI.}
\RU{Так что старшая часть адреса переменной rand\_state находится в регистре \$V1.}
\EN{So the high part of the rand\_state variable address is residing in register \$V1.}
\RU{Инструкция \LW по адресу 0x10 просуммирует старшую и младшую часть и загрузит значение переменной 
rand\_state в \$V1.}
\EN{The \LW instruction at address 0x10 sums up the high and low parts and loads the value of the rand\_state 
variable into \$V1.}
\RU{Инструкция \SW по адресу 0x54 также просуммирует и затем запишет новое значение в глобальную переменную
rand\_state.}
\EN{The \SW instruction at address 0x54 do the summing again and then stores the new value 
to the rand\_state global variable.}

\RU{IDA обрабатывает перемещения при загрузке, и таким образом эти детали скрываются.}
\EN{IDA processes relocations while loading, thus hiding these details,}
\RU{Но мы должны о них помнить.}\EN{but we ought to remember them.}

% TODO add example of compiled binary, GDB example, etc...
