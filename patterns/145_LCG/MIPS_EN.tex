\sectionold{MIPS}

\lstinputlisting[caption=\Optimizing GCC 4.4.5 (IDA)]{patterns/145_LCG/MIPS_O3_IDA_EN.lst}

Wow, here we see only one constant (0x3C6EF35F or 1013904223).
Where is the other one (1664525)?

It seems that multiplication by 1664525 is done by just using shifts and additions!
Let's check this assumption:

\lstinputlisting{patterns/145_LCG/test.c}

\lstinputlisting[caption=\Optimizing GCC 4.4.5 (IDA)]{patterns/145_LCG/test_O3_MIPS.lst}

Indeed!

\subsectionold{MIPS relocations}

We will also focus on how such operations as load from memory and store to memory actually work.

The listings here are produced by IDA, which hides some details.

We'll run objdump twice: to get a disassembled listing and also relocations list:

\lstinputlisting[caption=\Optimizing GCC 4.4.5 (objdump)]{patterns/145_LCG/MIPS_O3_objdump.txt}

Let's consider the two relocations for the \TT{my\_srand()} function.

The first one, for address 0 has a type of \TT{R\_MIPS\_HI16}
and the second one for address 8 has a type of \TT{R\_MIPS\_LO16}.

That implies that address of the beginning of the .bss segment is to be written into the instructions at
address of 0 (high part of address) and 8 (low part of address).

The \TT{rand\_state} variable is at the very start of the .bss segment.

So we see zeroes in the operands of instructions \LUI and \SW, because nothing is there yet---
the compiler don't know what to write there.

The linker will fix this, and the high part of the address will be written into the operand of \LUI and
the low part of the address---to the operand of \SW.

\SW will sum up the low part of the address and what is in register \$V0 (the high part is there).

It's the same story with the my\_rand() function: R\_MIPS\_HI16 relocation instructs the linker to write the high part
of the .bss segment address into instruction \LUI.

So the high part of the rand\_state variable address is residing in register \$V1.

The \LW instruction at address 0x10 sums up the high and low parts and loads the value of the rand\_state 
variable into \$V1.

The \SW instruction at address 0x54 do the summing again and then stores the new value 
to the rand\_state global variable.

IDA processes relocations while loading, thus hiding these details,
but we ought to remember them.

% TODO add example of compiled binary, GDB example, etc...
