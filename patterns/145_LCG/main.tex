\ifx\RUSSIAN\undefined
\chapter{Linear congruential generator as random number generator}
\index{\CStandardLibrary!rand()}

Linear congruential generator is probably simplest possible way to generate random numbers.
It's not in favour in modern times\footnote{Mersenne twister is better}, but it's so simple 
(just one multiplication, one addition and one AND operation), 
so we can use it as an example.

\lstinputlisting{patterns/145_LCG/rand.c}

So there are two functions: one is used for internal state initialization, and the second one is called
for random numbers generating.

We see two constants here used in algorithm.
I defined it using \TT{\#define} \CCpp statement. It's a macro.
The difference between \CCpp macro and constant is that all macros are replaced by its value by \CCpp preprocessor,
and it's not holding any value in the memory.
By contrast, constant is read-only variable.
It's possible to take a pointer (or address) of constant variable, but impossible to do so with macro.

The last AND operation is needed because by C-standard, \TT{rand()} should return value in $0..32767$ range.
If you want to get 32-bit random variables, just the last omit AND operation.

\section{x86}

\lstinputlisting[caption=\Optimizing MSVC 2013]{patterns/145_LCG/rand_MSVC_2013_x86_Ox.asm}

Here we see it: both constants are embedded into the code. There are no memory allocated for them.
\TT{srand()} function just copies input value into the internal \TT{rand\_state} variable.

\TT{rand()} takes it, calculate next \TT{rand\_state}, cuts it and leaves it in the EAX register.

Non-Optimized version is more verbose:

\lstinputlisting[caption=\NonOptimizing MSVC 2013]{patterns/145_LCG/rand_MSVC_2013_x86.asm}

\section{x64}

x64 version is mostly the same and uses 32-bit registers instead of 64-bit ones 
(because we work with \Tint values here).
But srand() function takes input argument from the ECX register rather than from stack:

\lstinputlisting[caption=\Optimizing MSVC 2013 x64]{patterns/145_LCG/rand_MSVC_2013_x64_Ox.asm}

GCC compiler do mostly the same code.

\section{32-bit ARM}

\lstinputlisting[caption=\OptimizingKeilVI (\ARMMode)]{patterns/145_LCG/rand.s_Keil_ARM_O3.s}

It's not possible to embed 32-bit constants into ARM instructions, so Keil is ought to place them externally
and load them additionally.

One interesting thing is that it's not possible to embed 0x7FFF constant as well.
So what Keil does is shifting rand\_state left by 17 bits and then shifting it right by 17 bits.
This is analogous to $(rand\_state \ll 17) \gg 17$ statement. 
Seems to be useless operation, but nevertheless,
what it does is clearing high 17 bits, leaving low 15 bits intact, and that's our goal after all.\\
\\
\Optimizing Keil for thumb mode do mostly the same code.

\fi
