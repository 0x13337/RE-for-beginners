\subsection{ARM}

\index{Linux}
\RU{В ядре Linux 3.8.0 бит \TT{O\_CREAT} проверяется немного иначе.}
\EN{The \TT{O\_CREAT} bit is checked differently in Linux kernel 3.8.0.}

\begin{lstlisting}[caption=linux kernel 3.8.0]
struct file *do_filp_open(int dfd, struct filename *pathname,
		const struct open_flags *op)
{
...
	filp = path_openat(dfd, pathname, &nd, op, flags | LOOKUP_RCU);
...
}

static struct file *path_openat(int dfd, struct filename *pathname,
		struct nameidata *nd, const struct open_flags *op, int flags)
{
...
	error = do_last(nd, &path, file, op, &opened, pathname);
...
}


static int do_last(struct nameidata *nd, struct path *path,
		   struct file *file, const struct open_flags *op,
		   int *opened, struct filename *name)
{
...
	if (!(open_flag & O_CREAT)) {
    ...
		error = lookup_fast(nd, path, &inode);
    ...
	} else {
    ...
		error = complete_walk(nd);
        }
...
}
\end{lstlisting}

\RU{Вот как это выглядит в \IDA, ядро скомпилированное для режима ARM:}
\EN{Here is how the kernel compiled for ARM mode looks in \IDA:}

\begin{lstlisting}[caption=do\_last() (vmlinux)]
...
.text:C0169EA8                 MOV             R9, R3  ; R3 - (4th argument) open_flag
...
.text:C0169ED4                 LDR             R6, [R9] ; R6 - open_flag
...
.text:C0169F68                 TST             R6, #0x40 ; jumptable C0169F00 default case
.text:C0169F6C                 BNE             loc_C016A128
.text:C0169F70                 LDR             R2, [R4,#0x10]
.text:C0169F74                 ADD             R12, R4, #8
.text:C0169F78                 LDR             R3, [R4,#0xC]
.text:C0169F7C                 MOV             R0, R4
.text:C0169F80                 STR             R12, [R11,#var_50]
.text:C0169F84                 LDRB            R3, [R2,R3]
.text:C0169F88                 MOV             R2, R8
.text:C0169F8C                 CMP             R3, #0
.text:C0169F90                 ORRNE           R1, R1, #3
.text:C0169F94                 STRNE           R1, [R4,#0x24]
.text:C0169F98                 ANDS            R3, R6, #0x200000
.text:C0169F9C                 MOV             R1, R12
.text:C0169FA0                 LDRNE           R3, [R4,#0x24]
.text:C0169FA4                 ANDNE           R3, R3, #1
.text:C0169FA8                 EORNE           R3, R3, #1
.text:C0169FAC                 STR             R3, [R11,#var_54]
.text:C0169FB0                 SUB             R3, R11, #-var_38
.text:C0169FB4                 BL              lookup_fast
...
.text:C016A128 loc_C016A128                            ; CODE XREF: do_last.isra.14+DC
.text:C016A128                 MOV             R0, R4
.text:C016A12C                 BL              complete_walk
...
\end{lstlisting}

\index{ARM!\Instructions!TST}
\index{x86!\Instructions!TEST}
\TT{TST} \RU{это аналог инструкции \TEST в x86.}\EN{is analogous to the \TEST instruction in x86.}

\RU{Мы можем \q{узнать} визуально этот фрагмент кода по тому что в одном случае исполнится 
функция \TT{lookup\_fast()}, а в другом \TT{complete\_walk()}.}
\EN{We can \q{spot} visually this code fragment by the fact the 
\TT{lookup\_fast()} is to be executed
in one case and \TT{complete\_walk()} in the other.}
\RU{Это соответствует исходному коду функции \TT{do\_last()}.}
\EN{This corresponds to the source code of the \TT{do\_last()} function.}

\EN{The}\RU{Макрос} \TT{O\_CREAT} \RU{здесь так же равен \TT{0x40}.}\EN{macro equals to \TT{0x40} here too.}

