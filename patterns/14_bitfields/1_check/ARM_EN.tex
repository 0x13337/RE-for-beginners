\subsubsection{ARM}

\myindex{Linux}

The \TT{O\_CREAT} bit is checked differently in Linux kernel 3.8.0.

\begin{lstlisting}[caption=linux kernel 3.8.0]
struct file *do_filp_open(int dfd, struct filename *pathname,
		const struct open_flags *op)
{
...
	filp = path_openat(dfd, pathname, &nd, op, flags | LOOKUP_RCU);
...
}

static struct file *path_openat(int dfd, struct filename *pathname,
		struct nameidata *nd, const struct open_flags *op, int flags)
{
...
	error = do_last(nd, &path, file, op, &opened, pathname);
...
}


static int do_last(struct nameidata *nd, struct path *path,
		   struct file *file, const struct open_flags *op,
		   int *opened, struct filename *name)
{
...
	if (!(open_flag & O_CREAT)) {
    ...
		error = lookup_fast(nd, path, &inode);
    ...
	} else {
    ...
		error = complete_walk(nd);
        }
...
}
\end{lstlisting}

Here is how the kernel compiled for ARM mode looks in \IDA:

\lstinputlisting[caption=do\_last() from vmlinux (IDA)]{patterns/14_bitfields/1_check/IDA.lst}

\myindex{ARM!\Instructions!TST}
\myindex{x86!\Instructions!TEST}
\TT{TST} is analogous to the \TEST instruction in x86.
We can \q{spot} visually this code fragment by the fact the 
\TT{lookup\_fast()} is to be executed in one case and \TT{complete\_walk()} in the other.
This corresponds to the source code of the \TT{do\_last()} function.
The \TT{O\_CREAT} macro equals to \TT{0x40} here too.

