\subsection{x86}
\myindex{Windows!Win32}

\RU{Например в Win32 API:}\EN{Win32 API example:}

\begin{lstlisting}
	HANDLE fh;

	fh=CreateFile ("file", GENERIC_WRITE | GENERIC_READ, FILE_SHARE_READ, NULL, OPEN_ALWAYS, FILE_ATTRIBUTE_NORMAL, NULL);
\end{lstlisting}

\RU{Получаем}\EN{We get} (MSVC 2010):

\begin{lstlisting}[caption=MSVC 2010]
	push	0
	push	128					; 00000080H
	push	4
	push	0
	push	1
	push	-1073741824				; c0000000H
	push	OFFSET $SG78813
	call	DWORD PTR __imp__CreateFileA@28
	mov	DWORD PTR _fh$[ebp], eax
\end{lstlisting}

\RU{Заглянем в файл}\EN{Let's take a look in} WinNT.h:

\begin{lstlisting}[caption=WinNT.h]
#define GENERIC_READ                     (0x80000000L)
#define GENERIC_WRITE                    (0x40000000L)
#define GENERIC_EXECUTE                  (0x20000000L)
#define GENERIC_ALL                      (0x10000000L)
\end{lstlisting}

\RU{Всё ясно}\EN{Everything is clear}, 
\TT{GENERIC\_READ | GENERIC\_WRITE = 0x80000000 | 0x40000000 = 0xC0000000}, 
\RU{и это значение используется как второй аргумент для функции}
\EN{and that value is used as the second argument for the} \TT{CreateFile()}\footnote{\href{http://go.yurichev.com/17065}{msdn.microsoft.com/en-us/library/aa363858(VS.85).aspx}}\EN{function}.

\RU{Как \TT{CreateFile()} будет проверять флаги?}\EN{How would \TT{CreateFile()} check these flags?}
\myindex{Windows!KERNEL32.DLL}
\RU{Заглянем в KERNEL32.DLL от Windows XP SP3 x86 и найдем в функции \TT{CreateFileW()} в том числе и 
такой фрагмент кода:}
\EN{If we look in KERNEL32.DLL in Windows XP SP3 x86, we'll find
this fragment of code in \TT{CreateFileW}:}

\begin{lstlisting}[caption=KERNEL32.DLL (Windows XP SP3 x86)]
.text:7C83D429                 test    byte ptr [ebp+dwDesiredAccess+3], 40h
.text:7C83D42D                 mov     [ebp+var_8], 1
.text:7C83D434                 jz      short loc_7C83D417
.text:7C83D436                 jmp     loc_7C810817
\end{lstlisting}

\myindex{x86!\Instructions!TEST}
\RU{Здесь мы видим инструкцию \TEST. Впрочем, она берет не весь второй аргумент функции, 
а только его самый старший байт (\TT{ebp+dwDesiredAccess+3}) и проверяет его на флаг \TT{0x40}
(имеется ввиду флаг \TT{GENERIC\_WRITE}).}
\EN{Here we see the \TEST instruction, however it doesn't take the whole second argument,
but only the most significant byte (\TT{ebp+dwDesiredAccess+3}) and checks it for flag \TT{0x40}
(which implies the \TT{GENERIC\_WRITE} flag here)}
\myindex{x86!\Instructions!AND}
\RU{\TEST это то же что и \AND, только без сохранения результата 
(вспомните что \CMP это то же что и \SUB, только без сохранения результатов}
\EN{\TEST is basically the same instruction as \AND, but without saving the result
(recall the fact \CMP is merely the same as \SUB, but without saving the result}~(\myref{CMPandSUB})).

\RU{Логика данного фрагмента кода примерно такая:}\EN{The logic of this code fragment is as follows:}

\begin{lstlisting}
if ((dwDesiredAccess&0x40000000) == 0) goto loc_7C83D417
\end{lstlisting}

\myindex{x86!\Instructions!AND}
\myindex{x86!\Registers!ZF}
\RU{Если после операции \AND останется этот бит, то флаг \ZF не будет поднят и условный переход 
\JZ не сработает. 
Переход возможен, только если в переменной \TT{dwDesiredAccess} отсутствует бит \TT{0x40000000}~--- 
тогда результат \AND будет 0, флаг \ZF будет поднят и переход сработает.}
\EN{If \AND instruction leaves this bit, the \ZF flag is to be cleared and the 
\JZ conditional jump is not to be triggered.
The conditional jump is triggered only if the \TT{0x40000000} bit is absent in \TT{dwDesiredAccess} variable
~---then the result of \AND is 0, \ZF is to be set and the conditional jump is to be triggered.}

\ifdefined\IncludeGCC
\RU{Попробуем GCC 4.4.1 и Linux:}\EN{Let's try GCC 4.4.1 and Linux:}

\begin{lstlisting}
#include <stdio.h>
#include <fcntl.h>

void main()
{
	int handle;

	handle=open ("file", O_RDWR | O_CREAT);
};
\end{lstlisting}

\RU{Получим}\EN{We get}:

\lstinputlisting[caption=GCC 4.4.1]{patterns/14_bitfields/1_check/check.asm}

\myindex{Linux!libc.so.6}
\myindex{syscall}
\RU{Заглянем в реализацию функции \TT{open()} в библиотеке \TT{libc.so.6}, но обнаружим что там 
только системный вызов:}
\EN{If we take a look in the \TT{open()} function in the \TT{libc.so.6} library, it is only a syscall:}

\begin{lstlisting}[caption=open() (libc.so.6)]
.text:000BE69B                 mov     edx, [esp+4+mode] ; mode
.text:000BE69F                 mov     ecx, [esp+4+flags] ; flags
.text:000BE6A3                 mov     ebx, [esp+4+filename] ; filename
.text:000BE6A7                 mov     eax, 5
.text:000BE6AC                 int     80h             ; LINUX - sys_open
\end{lstlisting}

\RU{Значит, битовые поля флагов \TT{open()} проверяются где-то в ядре Linux.}
\EN{So, the bit fields for \TT{open()} are apparently checked somewhere in the Linux kernel.}

\RU{Разумеется, и стандартные библиотеки Linux и ядро Linux можно получить в виде исходников, 
но нам интересно попробовать разобраться без них.}
\EN{Of course, it is easy to download both Glibc and the Linux kernel source code, 
but we are interested in understanding the matter without it.}

\RU{При системном вызове \TT{sys\_open} управление в конечном итоге передается в \TT{do\_sys\_open} в ядре Linux 2.6. 
Оттуда~--- в \TT{do\_filp\_open()} (эта функция находится в исходниках ядра в файле \TT{fs/namei.c}).}
\EN{So, as of Linux 2.6, when the \TT{sys\_open} syscall is called, control eventually passes to \TT{do\_sys\_open},
and from there---to the \TT{do\_filp\_open()} function (it's located in the kernel source tree in \TT{fs/namei.c}).}

\newcommand{\URLREGPARM}{\href{http://go.yurichev.com/17040}{ohse.de/uwe/articles/gcc-attributes.html\#func-regparm}}

\myindex{fastcall}
\label{regparm}
N.B. \RU{Помимо передачи параметров функции через стек, существует также возможность передавать 
некоторые из них через регистры. Такое соглашение о вызове называется fastcall~(\myref{fastcall}).
Оно работает немного быстрее, так как для чтения аргументов процессору не нужно обращаться к стеку, лежащему в памяти. 
В GCC есть опция \IT{regparm}\footnote{\URLREGPARM}, 
и с её помощью можно задать, сколько аргументов можно передать через регистры.}
\EN{Aside from passing arguments via the stack,
there is also a method of passing some of them
via registers. This is also called fastcall~(\myref{fastcall}).
This works faster since CPU does not need to access the stack in memory to read argument values.
GCC has the option \IT{regparm}\footnote{\URLREGPARM},
through which it's possible to set the number of arguments that can be passed via registers.}

\newcommand{\URLKERNELNEWB}{\href{http://go.yurichev.com/17066}{kernelnewbies.org/Linux\_2\_6\_20\#head-042c62f290834eb1fe0a1942bbf5bb9a4accbc8f}}
\newcommand{\CALLINGHFILE}{arch/x86/include/asm/calling.h}

\RU{Ядро Linux 2.6 собирается с опцией \TT{-mregparm=3}\footnote{\URLKERNELNEWB}
\footnote{См. также файл \TT{\CALLINGHFILE} в исходниках ядра}.}
\EN{The Linux 2.6 kernel is compiled with \TT{-mregparm=3} option~\footnote{\URLKERNELNEWB}
\footnote{See also \TT{\CALLINGHFILE} file in kernel tree}.}

\RU{Для нас это означает, что первые три аргумента функции будут передаваться через регистры \EAX, 
\EDX и \ECX, 
а остальные через стек. Разумеется, если аргументов у функции меньше трех, то будет задействована 
только часть этих регистров.}
\EN{What this means to us is that the first 3 arguments are to be passed via registers \EAX, \EDX and \ECX, 
and the rest via the stack. 
Of course, if the number of arguments is less than 3, only part of registers set is to be used.}

\RU{Итак, качаем ядро 2.6.31, собираем его в Ubuntu, открываем в \IDA, 
находим функцию \TT{do\_filp\_open()}. В начале мы увидим что-то такое (комментарии мои):}
\EN{So, let's download Linux Kernel 2.6.31, compile it in Ubuntu: \TT{make vmlinux}, open it in \IDA, 
and find the \TT{do\_filp\_open()} function. At the beginning, we see (the comments are mine):}

\lstinputlisting[caption=do\_filp\_open() (linux kernel 2.6.31)]{patterns/14_bitfields/1_check/check2.asm.\LANG}

\RU{GCC сохраняет значения первых трех аргументов в локальном стеке. Иначе, если эти три регистра 
не трогать вообще, то функции компилятора, распределяющей переменные по регистрам (так называемый 
\gls{register allocator}), 
будет очень тесно.}
\EN{GCC saves the values of the first 3 arguments in the local stack. 
If that wasn't done, the compiler would not touch these registers, 
and that would be too tight environment for the compiler's \gls{register allocator}.}

\RU{Далее находим примерно такой фрагмент кода}\EN{Let's find this fragment of code}:

\lstinputlisting[caption=do\_filp\_open() (linux kernel 2.6.31)]{patterns/14_bitfields/1_check/check3.asm}

\RU{\TT{0x40}~--- это значение макроса \TT{O\_CREAT}. 
\TT{open\_flag} проверяется на наличие бита \TT{0x40} и если бит равен 1, то выполняется следующие 
за \JNZ инструкции.}
\EN{\TT{0x40}---is what the \TT{O\_CREAT} macro equals to.
\TT{open\_flag} gets checked for the presence of the \TT{0x40} bit, and if this bit is 1, 
the next \JNZ instruction is triggered.}
\fi
