\subsectionold{x86}
\myindex{Windows!Win32}

Например в Win32 API:

\begin{lstlisting}
	HANDLE fh;

	fh=CreateFile ("file", GENERIC_WRITE | GENERIC_READ, FILE_SHARE_READ, NULL, OPEN_ALWAYS, FILE_ATTRIBUTE_NORMAL, NULL);
\end{lstlisting}

Получаем (MSVC 2010):

\begin{lstlisting}[caption=MSVC 2010]
	push	0
	push	128					; 00000080H
	push	4
	push	0
	push	1
	push	-1073741824				; c0000000H
	push	OFFSET $SG78813
	call	DWORD PTR __imp__CreateFileA@28
	mov	DWORD PTR _fh$[ebp], eax
\end{lstlisting}

Заглянем в файл WinNT.h:

\begin{lstlisting}[caption=WinNT.h]
#define GENERIC_READ                     (0x80000000L)
#define GENERIC_WRITE                    (0x40000000L)
#define GENERIC_EXECUTE                  (0x20000000L)
#define GENERIC_ALL                      (0x10000000L)
\end{lstlisting}

Всё ясно, 
\TT{GENERIC\_READ | GENERIC\_WRITE = 0x80000000 | 0x40000000 = 0xC0000000}, 
и это значение используется как второй аргумент для функции
 \TT{CreateFile()}\footnote{\href{http://go.yurichev.com/17065}{msdn.microsoft.com/en-us/library/aa363858(VS.85).aspx}}.

Как \TT{CreateFile()} будет проверять флаги?
\myindex{Windows!KERNEL32.DLL}
Заглянем в KERNEL32.DLL от Windows XP SP3 x86 и найдем в функции \TT{CreateFileW()} в том числе и 
такой фрагмент кода:

\begin{lstlisting}[caption=KERNEL32.DLL (Windows XP SP3 x86)]
.text:7C83D429                 test    byte ptr [ebp+dwDesiredAccess+3], 40h
.text:7C83D42D                 mov     [ebp+var_8], 1
.text:7C83D434                 jz      short loc_7C83D417
.text:7C83D436                 jmp     loc_7C810817
\end{lstlisting}

\myindex{x86!\Instructions!TEST}
Здесь мы видим инструкцию \TEST. Впрочем, она берет не весь второй аргумент функции, 
а только его самый старший байт (\TT{ebp+dwDesiredAccess+3}) и проверяет его на флаг \TT{0x40}
(имеется ввиду флаг \TT{GENERIC\_WRITE}).

\myindex{x86!\Instructions!AND}
\TEST это то же что и \AND, только без сохранения результата 
(вспомните что \CMP это то же что и \SUB, только без сохранения результатов~(\myref{CMPandSUB})).

Логика данного фрагмента кода примерно такая:

\begin{lstlisting}
if ((dwDesiredAccess&0x40000000) == 0) goto loc_7C83D417
\end{lstlisting}

\myindex{x86!\Instructions!AND}
\myindex{x86!\Registers!ZF}
Если после операции \AND останется этот бит, то флаг \ZF не будет поднят и условный переход 
\JZ не сработает. 
Переход возможен, только если в переменной \TT{dwDesiredAccess} отсутствует бит \TT{0x40000000}~--- 
тогда результат \AND будет 0, флаг \ZF будет поднят и переход сработает.

Попробуем GCC 4.4.1 и Linux:

\begin{lstlisting}
#include <stdio.h>
#include <fcntl.h>

void main()
{
	int handle;

	handle=open ("file", O_RDWR | O_CREAT);
};
\end{lstlisting}

Получим:

\lstinputlisting[caption=GCC 4.4.1]{patterns/14_bitfields/1_check/check.asm}

\myindex{Linux!libc.so.6}
\myindex{syscall}
Заглянем в реализацию функции \TT{open()} в библиотеке \TT{libc.so.6}, но обнаружим что там 
только системный вызов:

\begin{lstlisting}[caption=open() (libc.so.6)]
.text:000BE69B                 mov     edx, [esp+4+mode] ; mode
.text:000BE69F                 mov     ecx, [esp+4+flags] ; flags
.text:000BE6A3                 mov     ebx, [esp+4+filename] ; filename
.text:000BE6A7                 mov     eax, 5
.text:000BE6AC                 int     80h             ; LINUX - sys_open
\end{lstlisting}

Значит, битовые поля флагов \TT{open()} проверяются где-то в ядре Linux.

Разумеется, и стандартные библиотеки Linux и ядро Linux можно получить в виде исходников, 
но нам интересно попробовать разобраться без них.

При системном вызове \TT{sys\_open} управление в конечном итоге передается в \TT{do\_sys\_open} в ядре Linux 2.6. 
Оттуда~--- в \TT{do\_filp\_open()} (эта функция находится в исходниках ядра в файле \TT{fs/namei.c}).

\newcommand{\URLREGPARM}{\href{http://go.yurichev.com/17040}{ohse.de/uwe/articles/gcc-attributes.html\#func-regparm}}

\myindex{fastcall}
\label{regparm}
N.B. Помимо передачи параметров функции через стек, существует также возможность передавать 
некоторые из них через регистры. Такое соглашение о вызове называется fastcall~(\myref{fastcall}).
Оно работает немного быстрее, так как для чтения аргументов процессору не нужно обращаться к стеку, лежащему в памяти. 
В GCC есть опция \IT{regparm}\footnote{\URLREGPARM}, 
и с её помощью можно задать, сколько аргументов можно передать через регистры.

\newcommand{\URLKERNELNEWB}{\href{http://go.yurichev.com/17066}{kernelnewbies.org/Linux\_2\_6\_20\#head-042c62f290834eb1fe0a1942bbf5bb9a4accbc8f}}
\newcommand{\CALLINGHFILE}{arch/x86/include/asm/calling.h}

Ядро Linux 2.6 собирается с опцией \TT{-mregparm=3}\footnote{\URLKERNELNEWB}
\footnote{См. также файл \TT{\CALLINGHFILE} в исходниках ядра}.

Для нас это означает, что первые три аргумента функции будут передаваться через регистры \EAX, \EDX и \ECX, 
а остальные через стек. Разумеется, если аргументов у функции меньше трех, то будет задействована 
только часть этих регистров.

Итак, качаем ядро 2.6.31, собираем его в Ubuntu, открываем в \IDA, 
находим функцию \TT{do\_filp\_open()}. В начале мы увидим что-то такое (комментарии мои):

\lstinputlisting[caption=do\_filp\_open() (linux kernel 2.6.31)]{patterns/14_bitfields/1_check/check2_RU.asm}

GCC сохраняет значения первых трех аргументов в локальном стеке. Иначе, если эти три регистра 
не трогать вообще, то функции компилятора, распределяющей переменные по регистрам (так называемый 
\gls{register allocator}), будет очень тесно.

Далее находим примерно такой фрагмент кода:

\lstinputlisting[caption=do\_filp\_open() (linux kernel 2.6.31)]{patterns/14_bitfields/1_check/check3.asm}

\TT{0x40}~--- это значение макроса \TT{O\_CREAT}. 
\TT{open\_flag} проверяется на наличие бита \TT{0x40} и если бит равен 1, то выполняется следующие 
за \JNZ инструкции.
