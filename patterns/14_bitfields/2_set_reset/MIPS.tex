\subsection{MIPS}

\lstinputlisting[caption=\Optimizing GCC 4.4.5 (IDA)]{patterns/14_bitfields/2_set_reset/MIPS_O3_IDA.lst.\LANG}

\index{MIPS!\Instructions!ORI}
\RU{\INS{ORI} это, конечно, операция \q{ИЛИ}, \q{I} в имени инструкции означает что значение встроено в машинный код.}
\EN{\INS{ORI} is,of course, the OR operation. \q{I} in the instruction name mean that the value is embedded in the machine code.}

\index{MIPS!\Instructions!AND}
\RU{И напротив, есть \AND. Здесь нет возможности использовать \INS{ANDI}, потому что невозможно встроить число 
0xFFFFFDFF в одну инструкцию, так что компилятору приходится в начале загружать значение 0xFFFFFDFF в регистр \$V0,
а затем генерировать \AND, которая возьмет все значения из регистров.}
\EN{But after that we have \AND. There was no way to use \INS{ANDI} because it's not possible to embed the 0xFFFFFDFF number
in a single instruction, so the compiler has to load 0xFFFFFDFF into register \$V0 first and then generates
\AND which takes all its values from registers.}
