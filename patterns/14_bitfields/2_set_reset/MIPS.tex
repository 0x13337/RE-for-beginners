\subsection{MIPS}

\lstinputlisting[caption=\Optimizing GCC 4.4.5 (IDA)]{patterns/14_bitfields/2_set_reset/MIPS_O3_IDA.lst}

\index{MIPS!\Instructions!ORI}
\RU{ORI это, конечно, операция ``ИЛИ'', ``I'' в имени инструкции означает что значение встроено в машинный код.}
\EN{ORI, of course, OR operation. ``I'' in instruction name mean that value is embedded into machine code.}

\index{MIPS!\Instructions!AND}
\RU{И напротив, есть AND. Здесь нет возможности использовать ANDI, потому что невозможно встроить число 
0xFFFFFDFF в одну инструкцию, так что компилятору приходится в начале загружать значение 0xFFFFFDFF в регистр \$V0,
а затем генерировать AND, которая возьмет все значения из регистров.}
\EN{On contrast, here is AND. There was no way to use ANDI because it's not possible to embed 0xFFFFFDFF number
into one single instruction, so compiler ought to load 0xFFFFFDFF value into \$V0 register first and then generate
AND which take all the values from registers.}
