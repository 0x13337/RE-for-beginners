\section{\RU{Установка/сброс отдельного бита: пример с \ac{FPU}}\EN{Specific bit setting/clearing: \ac{FPU} example}}

\index{IEEE 754}
\RU{Как мы уже можем знать, вот как биты расположены в значении типа \Tfloat в формате IEEE 754:}
\EN{As we may know, here is how bits are located in the \Tfloat type value in IEEE 754 form:}

\bigskip
% a hack used here! http://tex.stackexchange.com/questions/73524/bytefield-package
\begin{center}
\begin{bytefield}{32}
	\bitheader[endianness=big]{0,22,23,30,31} \\
	\bitbox{1}{S} & 
	\bitbox{8}{%
		\RU{экспонента}%
		\EN{exponent}%
		\ES{exponente}%
		\PTBRph{}%
		\DEph{}\PLph{}%
		\ITAph{}%
		\FR{exposant}
	} & 
	\bitbox{23}{%
		\RU{мантисса}%
		\EN{mantissa or fraction}%
		\ES{mantisa o fracci\'on}%
		\PTBRph{}%
		\DEph{}\PLph{}%
		\ITAph{}%
		\FR{mantisse ou fraction}
	}
\end{bytefield}
\end{center}

\begin{center}
( S\EMDASH{}%
	\RU{знак}%
	\EN{sign}%
	\ES{signo}%
	\PTBRph{}%
	\DEph{}\PLph{}%
	\ITAph{}%
	\FR{signe}
)
\end{center}


\RU{Знак числа это}\EN{The sign of number is} \ac{MSB}. 
\RU{Возможно ли работать со знаком числа с плавающей точкой не используя FPU-инструкций?}
\EN{Will it be possible to affect sign of floating point number while not using any FPU instructions?}

\lstinputlisting{patterns/14_bitfields/35_set_reset_FPU/abs.c}

\RU{Придется использовать эти трюки в \CCpp с типами данных чтобы копировать из/в значение типа \Tfloat
без конверсии.}
\EN{We need this data type trickery in \CCpp to copy to/from \Tfloat value without actual conversion.}
\RU{Так что здесь три ф-ции: my\_abs() сбрасывает \ac{MSB}; set\_sign() устанавливает \ac{MSB} и 
negate() меняет его на противоположный.}
\EN{So there are three functions: my\_abs() resets \ac{MSB}; set\_sign() sets \ac{MSB} and negate() flips it.}

\subsection{\RU{Кое-что об операции XOR}\EN{A word about XOR operation}}

\RU{XOR (исключающее ИЛИ) часто используется для того чтобы поменять какой-то бит(ы) на противоположный.}
\EN{XOR is widely used when one need just to flip specific bit(s).}
\RU{Действительно, операция XOR с 1 на самом деле просто инвертирует бит:}
\EN{Indeed, XOR operation applied with 1 is effectively inverting bit:}

\begin{center}
\begin{tabular}{ | l | l | l | }
\hline
\cellcolor{blue!25} \RU{вход А}\EN{input A} & 
\cellcolor{blue!25} \RU{вход Б}\EN{input B} & 
\cellcolor{blue!25} \RU{выход}\EN{output} \\
\hline
0 & 0 & 0 \\
\hline
{\color{red} 0} & {\color{red} 1} & {\color{red} 1} \\
\hline
{\color{red} 1} & {\color{red} 0} & {\color{red} 1} \\
\hline
1 & 1 & 0 \\
\hline
\end{tabular}
\end{center}

\RU{И наоборот, операция XOR с 0 ничего не делает, т.е., это холостая операция.}
\EN{On contrary, XOR operation applied with 0 does nothing, i.e., it's idle operation.}
\RU{Это очень важное свойство операции XOR и очень важно помнить его.}
\EN{This is very important property of XOR operation and it's highly recommended to memorize it.}

\subsection{x86}

\RU{Код прямолинеен}\EN{The code is pretty straightforward}:

\lstinputlisting[caption=\Optimizing MSVC 2012]{patterns/14_bitfields/35_set_reset_FPU/abs_MSVC2012_Ox.asm}

\RU{Входное значение типа \Tfloat берется из стека, но мы обходимся с ним как с целочисленным значением.}
\EN{Input value of \Tfloat type is taken from stack but treated as an integer value.}

\ANDIns \AndENRU \OR \RU{сбрасывают и устанавливают нужный бит}\EN{resets and sets desired bit}.
\XOR \RU{переворачивает его}\EN{flips it}.

\RU{В конце, измененное значение загружает в ST0, потоум что числа с плавающей точкой возвращаются в этом 
регистре.}
\EN{Finally, modified value is loaded into ST0, because float-point numbers are returned in this register.}

\RU{Попробуем оптимизирующий MSVC 2012 для x64}\EN{Now let's try optimizing MSVC 2012 for x64}:

\lstinputlisting[caption=\Optimizing MSVC 2012 x64]{patterns/14_bitfields/35_set_reset_FPU/abs_MSVC2012_x64_Ox.asm}

\index{x86!\Instructions!BTR}
\index{x86!\Instructions!BTS}
\index{x86!\Instructions!BTC}
\RU{Во-первых, входное значение передается в XMM0, затем оно копируется в локальный стек и затем мы видим
новые для нас инструкции: BTR, BTS, BTC.}
\EN{First of all, input value is passed in XMM0, then it is copied into local stack and then we see 
new instructions to us: BTR, BTS, BTC.}

\RU{Эти инструкции используются для сброса определенного бита (BTR: ``reset''), 
установки (BTS: ``set'') и инвертирования (BTC: ``complementing'').}
\EN{These instructions are used for resetting (BTR), setting (BTS) and inverting (or complementing: BTC) 
specific bits.}
\RU{31-й бит это \ac{MSB}, если считать с нуля}\EN{31th bit is \ac{MSB} if to count starting at 0}.

\RU{И наконец, результат копируется в регистр XMM0, потому что значения с плавающей точной возвращаются
в регистре XMM0 в среде Win64.}
\EN{Finally, result is copied into XMM0 register, because floating point values are returned in XMM0 in Win64
environment.}

\ifdefined\IncludeMIPS
\subsection{MIPS}

GCC 4.4.5 \ForENRU MIPS \RU{делает почти то же самое}\EN{does mostly the same}:

\lstinputlisting[caption=\Optimizing GCC 4.4.5 (IDA)]{patterns/14_bitfields/35_set_reset_FPU/MIPS_O3_IDA.lst.\LANG}

\index{MIPS!\Instructions!LUI}
\RU{Для загрузки константы 0x80000000 в регистр используется только одна инструкция LUI, потому что LUI сбрасывает
младшие 16 бит и это нули в константе, так что одной LUI без ORI достаточно.}
\EN{One single LUI instruction is used to load 0x80000000 constant into register, because, 
LUI is clearing low 16 bits and these are zeroes in the constant, so one LUI without subsequent ORI is enough.}

\fi

\ifdefined\IncludeARM
\subsection{ARM}

\subsubsection{\OptimizingKeilVI (\ARMMode)}

\lstinputlisting[caption=\OptimizingKeilVI (\ARMMode)]{patterns/14_bitfields/35_set_reset_FPU/abs_Keil_ARM_O3.s.\LANG}

\RU{Пока всё поняно}\EN{So far so good}.
\index{ARM!\Instructions!BIC}
\index{ARM!\Instructions!EOR}
\RU{В ARM есть инструкция BIC для сброса заданных бит.}
\EN{ARM has BIC instruction which explicitly clears specific bit(s).}
\RU{EOR это инструкция в ARM которая делает то же что и XOR}\EN{EOR is ARM instruction name for XOR} 
(``Exclusive OR'').

\subsubsection{\OptimizingKeilVI (\ThumbMode)}

\lstinputlisting[caption=\OptimizingKeilVI (\ThumbMode)]{patterns/14_bitfields/35_set_reset_FPU/abs_Keil_thumb_O3.s}

\RU{В режиме Thumb 16-битные инструкции, в которых много данных нельзя задать, так что здесь
применяется пара инструкций MOVS/LSLS для формирования константы 0x80000000.}
\EN{Thumb mode in ARM offers 16-bit instructions and not much data can be encoded in them, so here is a 
MOVS/LSLS instruction pair which used for forming 0x80000000 constant.}
\RU{Это работает как выражение}\EN{It works like this expression}: $1<<31 = 0x80000000$.

\index{ARM!\Instructions!LSLS}
\index{ARM!\Instructions!LSRS}
\RU{Код my\_abs выглядит странно и работает как выражение}
\EN{my\_abs code is weird and it's effectiely works like this expression}: $(i<<1)>>1$.
\RU{Это выражение выглядит бессмысленным}\EN{This statement looks senseless}.
\RU{Но тем не менее, когда исполняется $input<<1$, \ac{MSB} (бит знака) просто выбрасывается.}
\EN{But nevertheless, when $input<<1$ is executed, \ac{MSB} (sign bit) is just dropped.}
\RU{Когда исполняется следующее выражение $result>>1$, все биты становятся на свои места,
а \ac{MSB} ноль, потому что все ``новые'' биты появляющиеся во время операций сдвига это всегда нули.}
\EN{When subsequent $result>>1$ statement is executed, all bits are now at their own places,
but \ac{MSB} is zero, because all ``new'' bits apperaed in shift operations are always zeroes.}
\RU{Таким образом, пара инструкций LSLS/LSRS сбрасывают \ac{MSB}.}
\EN{That is the way how LSLS/LSRS instructions pair clears \ac{MSB}.}

\subsubsection{\Optimizing GCC 4.6.3 (Raspberry Pi, \ARMMode)}

\lstinputlisting[caption=\Optimizing GCC 4.6.3 \ForENRU Raspberry Pi (\ARMMode)]{patterns/14_bitfields/35_set_reset_FPU/raspberry_GCC_O3_ARM_mode.lst.\LANG}

\RU{Я запускаю Raspberry Pi Linux в QEMU и он эмулирует FPU в ARM, так что здесь используются S-регистры
для передачи значений с плавающей точкой, вместо R-регистров.}
\EN{I run Raspberry Pi Linux in QEMU and it emulates ARM FPU, so S-registers are used here for floating point
numbers instead of R-registers.}

\index{ARM!\Instructions!FMRS}
\RU{Инструкция FMRS копирует данные из \ac{GPR} в FPU и назад.}
\EN{FMRS instruction copies data from \ac{GPR} to FPU and back.}

my\_abs() \AndENRU set\_sign() \RU{выглядят предсказуемо, но}\EN{looks predictably, but} negate()?
\RU{Почему там ADD вместо XOR}\EN{Why there are ADD instead of XOR}?

\index{ARM!\Instructions!XOR}
\index{ARM!\Instructions!ADD}
\RU{Трудно поверить, но инструкция}\EN{It's hard to believe but instruction} 
``ADD register, 0x80000000'' \RU{работает так же как и}\EN{works just like} ``XOR register, 0x80000000''.
\RU{Прежде всего, какая наша цель}\EN{First of all, what's our goal}?
\RU{Цель в том чтобы поменять \ac{MSB} на противоположный, и давайте забудем пока об операции XOR.}
\EN{The goal is to flip \ac{MSB}, and let's forget about XOR operation at all.}
\RU{Из школьной математики мы можем помнить что прибавляя числа вроде 1000 к другим никогда не затрагивают
последние 3 цифры.}
\EN{From school-level mathematics we may know that adding values like 1000 to other values is never affecting
last 3 digits.}
\RU{Например}\EN{For example}: $1234567 + 10000 = 1244567$ (\RU{последние 4 цифры никогда не меняются}
\EN{last 4 digits are never affected}).
\RU{Но мы работаем с двоичной системой исчисления, и 0x80000000 это 100000000000000000000000000000000
в двоичной системе, т.е., только старший бит установлен.}
\EN{But we operate here in binary base and 0x80000000 is 100000000000000000000000000000000 in binary form, i.e.,
only highest bit is set.}
\RU{Прибавление 0x80000000 к любому значению никогда не затронет младших 31 бит, но только \ac{MSB}.}
\EN{Adding 0x80000000 to any value will never affect lowest 31 bits, but will affect only \ac{MSB}. }
\RU{Прибавление 1 к 0 в итоге даст 1}\EN{Adding 1 to 0 will result in 1}.
\RU{Прибавление 1 к 1 даст 10 в двоичном виде, но 32-й бит (считая с нуля) выброшен: 
потому что наши регистры имеют ширину в 32 бита в конце концов, так что результат --- 0.}
\EN{Adding 1 to 1 will result 10 in binary form, but 32th bit (counting from zero) is dropped: 
because our registers has width of 32 bit after all, so result is 0.}
\RU{Вот почему XOR здесь можно заменить на ADD}\EN{That's why XOR can be replaced by ADD here}.
\RU{Я не уверен, почему GCC решил сделать так, но это работает корректно.}
\EN{I'm not sure why GCC decided to do this, but it works correctly.}

\fi
