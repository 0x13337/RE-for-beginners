\section{\RU{Установка/сброс отдельного бита: пример с \ac{FPU}}\EN{Specific bit setting/clearing: \ac{FPU} example}}

\index{IEEE 754}
\RU{Как мы уже можем знать, вот как биты расположены в значении типа \Tfloat в формате IEEE 754:}
\EN{Here is how bits are located in the \Tfloat type in IEEE 754 form:}

\bigskip
% a hack used here! http://tex.stackexchange.com/questions/73524/bytefield-package
\begin{center}
\begin{bytefield}{32}
	\bitheader[endianness=big]{0,22,23,30,31} \\
	\bitbox{1}{S} & 
	\bitbox{8}{%
		\RU{экспонента}%
		\EN{exponent}%
		\ES{exponente}%
		\PTBRph{}%
		\DEph{}\PLph{}%
		\ITAph{}%
		\FR{exposant}
	} & 
	\bitbox{23}{%
		\RU{мантисса}%
		\EN{mantissa or fraction}%
		\ES{mantisa o fracci\'on}%
		\PTBRph{}%
		\DEph{}\PLph{}%
		\ITAph{}%
		\FR{mantisse ou fraction}
	}
\end{bytefield}
\end{center}

\begin{center}
( S\EMDASH{}%
	\RU{знак}%
	\EN{sign}%
	\ES{signo}%
	\PTBRph{}%
	\DEph{}\PLph{}%
	\ITAph{}%
	\FR{signe}
)
\end{center}


\RU{Знак числа --- это}\EN{The sign of number is in the} \ac{MSB}. 
\RU{Возможно ли работать со знаком числа с плавающей точкой, не используя FPU-инструкций?}
\EN{Will it be possible to change the sign of a floating point number without any FPU instructions?}

\lstinputlisting{patterns/14_bitfields/35_set_reset_FPU/abs.c}

\RU{Придется использовать эти трюки в \CCpp с типами данных чтобы копировать из/в значение типа \Tfloat
без конверсии.}
\EN{We need this trickery in \CCpp to copy to/from \Tfloat value without actual conversion.}
\RU{Так что здесь три ф-ции: my\_abs() сбрасывает \ac{MSB}; set\_sign() устанавливает \ac{MSB} и 
negate() меняет его на противоположный.}
\EN{So there are three functions: my\_abs() resets \ac{MSB}; set\_sign() sets \ac{MSB} and negate() flips it.}

\subsection{\RU{Кое-что об операции XOR}\EN{A word about the XOR operation}}

\RU{XOR (исключающее ИЛИ) часто используется для того чтобы поменять какой-то бит(ы) на противоположный.}
\EN{XOR is widely used when one need just to flip specific bit(s).}
\RU{Действительно, операция XOR с 1 на самом деле просто инвертирует бит:}
\EN{Indeed, the XOR operation applied with 1 is effectively inverting a bit:}

\begin{center}
\begin{tabular}{ | l | l | l | }
\hline
\cellcolor{blue!25} \RU{вход А}\EN{input A} & 
\cellcolor{blue!25} \RU{вход Б}\EN{input B} & 
\cellcolor{blue!25} \RU{выход}\EN{output} \\
\hline
0 & 0 & 0 \\
\hline
{\color{red} 0} & {\color{red} 1} & {\color{red} 1} \\
\hline
{\color{red} 1} & {\color{red} 0} & {\color{red} 1} \\
\hline
1 & 1 & 0 \\
\hline
\end{tabular}
\end{center}

\RU{И наоборот, операция XOR с 0 ничего не делает, т.е., это холостая операция.}
\EN{And on the contrary, the XOR operation applied with 0 does nothing, i.e., it's an idle operation.}
\RU{Это очень важное свойство операции XOR и очень важно помнить его.}
\EN{This is a very important property of the XOR operation and it's highly recommended to memorize it.}

\subsection{x86}

\RU{Код прямолинеен}\EN{The code is pretty straightforward}:

\lstinputlisting[caption=\Optimizing MSVC 2012]{patterns/14_bitfields/35_set_reset_FPU/abs_MSVC2012_Ox.asm}

\RU{Входное значение типа \Tfloat берется из стека, но мы обходимся с ним как с целочисленным значением.}
\EN{An input value of type \Tfloat is taken from the stack, but treated as an integer value.}

\ANDIns \AndENRU \OR \RU{сбрасывают и устанавливают нужный бит}\EN{reset and set the desired bit}.
\XOR \RU{переворачивает его}\EN{flips it}.

\RU{В конце, измененное значение загружает в ST0, потому что числа с плавающей точкой возвращаются в этом 
регистре.}
\EN{Finally, the modified value is loaded into ST0, because floating-point numbers are returned in this register.}

\RU{Попробуем оптимизирующий MSVC 2012 для x64}\EN{Now let's try optimizing MSVC 2012 for x64}:

\lstinputlisting[caption=\Optimizing MSVC 2012 x64]{patterns/14_bitfields/35_set_reset_FPU/abs_MSVC2012_x64_Ox.asm}

\index{x86!\Instructions!BTR}
\index{x86!\Instructions!BTS}
\index{x86!\Instructions!BTC}
\RU{Во-первых, входное значение передается в XMM0, затем оно копируется в локальный стек и затем мы видим
новые для нас инструкции: BTR, BTS, BTC.}
\EN{The input value is passed in XMM0, then it is copied into the local stack and then we see 
some instructions that are new to us: BTR, BTS, BTC.}

\RU{Эти инструкции используются для сброса определенного бита (BTR: ``reset''), 
установки (BTS: ``set'') и инвертирования (BTC: ``complementing'').}
\EN{These instructions are used for resetting (BTR), setting (BTS) and inverting (or complementing: BTC) 
specific bits.}
\RU{31-й бит это \ac{MSB}, если считать с нуля}\EN{The 31st bit is \ac{MSB}, counting from 0}.

\RU{И наконец, результат копируется в регистр XMM0, потому что значения с плавающей точной возвращаются
в регистре XMM0 в среде Win64.}
\EN{Finally, the result is copied into XMM0r, because floating point values are returned through XMM0 in Win64
environment.}

\ifdefined\IncludeMIPS
\subsection{MIPS}

GCC 4.4.5 \ForENRU MIPS \RU{делает почти то же самое}\EN{does mostly the same}:

\lstinputlisting[caption=\Optimizing GCC 4.4.5 (IDA)]{patterns/14_bitfields/35_set_reset_FPU/MIPS_O3_IDA.lst.\LANG}

\index{MIPS!\Instructions!LUI}
\RU{Для загрузки константы 0x80000000 в регистр используется только одна инструкция LUI, потому что LUI сбрасывает
младшие 16 бит и это нули в константе, так что одной LUI без ORI достаточно.}
\EN{One single LUI instruction is used to load 0x80000000 into a register, because 
LUI is clearing the low 16 bits and these are zeroes in the constant, so one LUI without subsequent ORI is enough.}

\fi

\ifdefined\IncludeARM
\subsection{ARM}

\subsubsection{\OptimizingKeilVI (\ARMMode)}

\lstinputlisting[caption=\OptimizingKeilVI (\ARMMode)]{patterns/14_bitfields/35_set_reset_FPU/abs_Keil_ARM_O3.s.\LANG}

\RU{Пока всё понятно}\EN{So far so good}.
\index{ARM!\Instructions!BIC}
\index{ARM!\Instructions!EOR}
\RU{В ARM есть инструкция BIC для сброса заданных бит.}
\EN{ARM has the BIC instruction, which explicitly clears specific bit(s).}
\RU{EOR это инструкция в ARM которая делает то же что и XOR}\EN{EOR is the ARM instruction name for XOR} 
(``Exclusive OR'').

\subsubsection{\OptimizingKeilVI (\ThumbMode)}

\lstinputlisting[caption=\OptimizingKeilVI (\ThumbMode)]{patterns/14_bitfields/35_set_reset_FPU/abs_Keil_thumb_O3.s}

\RU{В режиме Thumb 16-битные инструкции, в которых много данных нельзя задать, так что здесь
применяется пара инструкций MOVS/LSLS для формирования константы 0x80000000.}
\EN{Thumb mode in ARM offers 16-bit instructions and not much data can be encoded in them, so here a 
MOVS/LSLS instruction pair is used for forming the 0x80000000 constant.}
\RU{Это работает как выражение}\EN{It works like this}: $1<<31 = 0x80000000$.

\index{ARM!\Instructions!LSLS}
\index{ARM!\Instructions!LSRS}
\RU{Код my\_abs выглядит странно и работает как выражение}
\EN{The code of my\_abs is weird and it effectively works like this expression}: $(i<<1)>>1$.
\RU{Это выражение выглядит бессмысленным}\EN{This statement looks meaningless}.
\RU{Но тем не менее, когда исполняется $input<<1$, \ac{MSB} (бит знака) просто выбрасывается.}
\EN{But nevertheless, when $input<<1$ is executed, the \ac{MSB} (sign bit) is just dropped.}
\RU{Когда исполняется следующее выражение $result>>1$, все биты становятся на свои места,
а \ac{MSB} ноль, потому что все ``новые'' биты появляющиеся во время операций сдвига это всегда нули.}
\EN{When the subsequent $result>>1$ statement is executed, all bits are now in their own places,
but \ac{MSB} is zero, because all ``new'' bits appearing from the shift operations are always zeroes.}
\RU{Таким образом, пара инструкций LSLS/LSRS сбрасывают \ac{MSB}.}
\EN{That is how the LSLS/LSRS instruction pair clears \ac{MSB}.}

\subsubsection{\Optimizing GCC 4.6.3 (Raspberry Pi, \ARMMode)}

\lstinputlisting[caption=\Optimizing GCC 4.6.3 \ForENRU Raspberry Pi (\ARMMode)]{patterns/14_bitfields/35_set_reset_FPU/raspberry_GCC_O3_ARM_mode.lst.\LANG}

\RU{Я запускаю Raspberry Pi Linux в QEMU и он эмулирует FPU в ARM, так что здесь используются S-регистры
для передачи значений с плавающей точкой, вместо R-регистров.}
\EN{I run Raspberry Pi Linux in QEMU and it emulates an ARM FPU, so S-registers are used here for floating point
numbers instead of R-registers.}

\index{ARM!\Instructions!FMRS}
\RU{Инструкция FMRS копирует данные из \ac{GPR} в FPU и назад.}
\EN{The FMRS instruction copies data from \ac{GPR} to the FPU and back.}

my\_abs() \AndENRU set\_sign() \RU{выглядят предсказуемо, но}\EN{looks as expected, but} negate()?
\RU{Почему там ADD вместо XOR}\EN{Why is there ADD instead of XOR}?

\index{ARM!\Instructions!XOR}
\index{ARM!\Instructions!ADD}
\RU{Трудно поверить, но инструкция}\EN{It's hard to believe, but the instruction} 
``ADD register, 0x80000000'' \RU{работает так же как и}\EN{works just like} ``XOR register, 0x80000000''.
\RU{Прежде всего, какая наша цель}\EN{First of all, what's our goal}?
\RU{Цель в том, чтобы поменять \ac{MSB} на противоположный, и давайте забудем пока об операции XOR.}
\EN{The goal is to flip the \ac{MSB}, so let's forget about the XOR operation.}
\RU{Из школьной математики мы можем помнить что прибавляя числа вроде 1000 к другим никогда не затрагивают
последние 3 цифры.}
\EN{From school-level mathematics we may remember that adding values like 1000 to other values never affects
the last 3 digits.}
\RU{Например}\EN{For example}: $1234567 + 10000 = 1244567$ (\RU{последние 4 цифры никогда не меняются}
\EN{last 4 digits are never affected}).
\RU{Но мы работаем с двоичной системой исчисления, и 0x80000000 это 100000000000000000000000000000000
в двоичной системе, т.е., только старший бит установлен.}
\EN{But here we operate in binary base and 0x80000000 is 100000000000000000000000000000000, i.e.,
only the highest bit is set.}
\RU{Прибавление 0x80000000 к любому значению никогда не затронет младших 31 бит, но только \ac{MSB}.}
\EN{Adding 0x80000000 to any value will never affect the lowest 31 bits, but will affect only the \ac{MSB}. }
\RU{Прибавление 1 к 0 в итоге даст 1}\EN{Adding 1 to 0 will result in 1}.
\RU{Прибавление 1 к 1 даст 10 в двоичном виде, но 32-й бит (считая с нуля) выброшен: 
потому что наши регистры имеют ширину в 32 бита в конце концов, так что результат --- 0.}
\EN{Adding 1 to 1 will result 10 in binary form, but the 32th bit (counting from zero) gets dropped, 
because our registers are 32 bit wide, so the result is 0.}
\RU{Вот почему XOR здесь можно заменить на ADD}\EN{That's why XOR can be replaced by ADD here}.
\RU{Я не уверен, почему GCC решил сделать так, но это работает корректно.}
\EN{I'm not sure why GCC decided to do this, but it works correctly.}

\fi
