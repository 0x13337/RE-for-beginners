\subsection{Установка и сброс отдельного бита: пример с \ac{FPU}}

\myindex{IEEE 754}
Как мы уже можем знать, вот как биты расположены в значении типа \Tfloat в формате IEEE 754:

\bigskip
% a hack used here! http://tex.stackexchange.com/questions/73524/bytefield-package
\begin{center}
\begin{bytefield}{32}
	\bitheader[endianness=big]{0,22,23,30,31} \\
	\bitbox{1}{S} & 
	\bitbox{8}{%
		\RU{экспонента}%
		\EN{exponent}%
		\ES{exponente}%
		\PTBRph{}%
		\DEph{}\PLph{}%
		\ITAph{}%
		\FR{exposant}
	} & 
	\bitbox{23}{%
		\RU{мантисса}%
		\EN{mantissa or fraction}%
		\ES{mantisa o fracci\'on}%
		\PTBRph{}%
		\DEph{}\PLph{}%
		\ITAph{}%
		\FR{mantisse ou fraction}
	}
\end{bytefield}
\end{center}

\begin{center}
( S\EMDASH{}%
	\RU{знак}%
	\EN{sign}%
	\ES{signo}%
	\PTBRph{}%
	\DEph{}\PLph{}%
	\ITAph{}%
	\FR{signe}
)
\end{center}


Знак числа~--- это \ac{MSB}. 
Возможно ли работать со знаком числа с плавающей точкой, не используя FPU-инструкций?

\lstinputlisting{patterns/14_bitfields/35_set_reset_FPU/abs.c}

Придется использовать эти трюки в \CCpp с типами данных чтобы копировать из значения типа \Tfloat и обратно
без конверсии.
Так что здесь три функции: \TT{my\_abs()} сбрасывает \ac{MSB}; \TT{set\_sign()} устанавливает \ac{MSB} и 
\TT{negate()} меняет его на противоположный.

\XOR может использоваться для смены бита: \myref{XOR_property}.

\subsubsection{x86}

Код прямолинеен:

\lstinputlisting[caption=\Optimizing MSVC 2012]{patterns/14_bitfields/35_set_reset_FPU/abs_MSVC2012_Ox.asm}

Входное значение типа \Tfloat берется из стека, но мы обходимся с ним как с целочисленным значением.

\AND и \OR сбрасывают и устанавливают нужный бит.
\XOR переворачивает его.

В конце измененное значение загружается в \TT{ST0}, потому что числа с плавающей точкой возвращаются в этом регистре.

Попробуем оптимизирующий MSVC 2012 для x64:

\lstinputlisting[caption=\Optimizing MSVC 2012 x64]{patterns/14_bitfields/35_set_reset_FPU/abs_MSVC2012_x64_Ox.asm}

\myindex{x86!\Instructions!BTR}
\myindex{x86!\Instructions!BTS}
\myindex{x86!\Instructions!BTC}
Во-первых, входное значение передается в \TT{XMM0}, затем оно копируется в локальный стек и затем мы видим
новые для нас инструкции: \BTR, \BTS, \BTC.
Эти инструкции используются для сброса определенного бита (\BTR: \q{reset}), 
установки (\BTS: \q{set}) и инвертирования (\BTC: \q{complement}).
31-й бит это \ac{MSB}, если считать начиная с нуля.
И наконец, результат копируется в регистр \TT{XMM0}, потому что значения с плавающей точной возвращаются
в регистре \TT{XMM0} в среде Win64.

\subsubsection{MIPS}

GCC 4.4.5 для MIPS делает почти то же самое:

\lstinputlisting[caption=\Optimizing GCC 4.4.5 (IDA)]{patterns/14_bitfields/35_set_reset_FPU/MIPS_O3_IDA_RU.lst}

\myindex{MIPS!\Instructions!LUI}
Для загрузки константы 0x80000000 в регистр используется только одна инструкция \LUI, потому что \LUI сбрасывает
младшие 16 бит и это нули в константе, так что одной \LUI без \ORI достаточно.

\subsubsection{ARM}

\myparagraph{\OptimizingKeilVI (\ARMMode)}

\lstinputlisting[caption=\OptimizingKeilVI (\ARMMode)]{patterns/14_bitfields/35_set_reset_FPU/abs_Keil_ARM_O3_RU.s}

Пока всё понятно.
\myindex{ARM!\Instructions!BIC}
\myindex{ARM!\Instructions!EOR}
В ARM есть инструкция \BIC для сброса заданных бит.

\EOR это инструкция в ARM которая делает то же что и \XOR (\q{Exclusive OR}).

\myparagraph{\OptimizingKeilVI (\ThumbMode)}

\lstinputlisting[caption=\OptimizingKeilVI (\ThumbMode)]{patterns/14_bitfields/35_set_reset_FPU/abs_Keil_thumb_O3.s}

В режиме Thumb 16-битные инструкции, в которых нельзя задать много данных, поэтому здесь
применяется пара инструкций \MOVS/\LSLS для формирования константы 0x80000000.

Это работает как выражение: $1<<31 = 0x80000000$.

\myindex{ARM!\Instructions!LSLS}
\myindex{ARM!\Instructions!LSRS}
Код my\_abs выглядит странно и работает как выражение: $(i<<1)>>1$.
Это выражение выглядит бессмысленным.
Но тем не менее, когда исполняется $input<<1$, \ac{MSB} (бит знака) просто выбрасывается.
Когда исполняется следующее выражение $result>>1$, все биты становятся на свои места,
а \ac{MSB} ноль, потому что все \q{новые} биты появляющиеся во время операций сдвига это всегда нули.
Таким образом, пара инструкций \LSLS/\LSRS сбрасывают \ac{MSB}.

\myparagraph{\Optimizing GCC 4.6.3 (Raspberry Pi, \ARMMode)}

\lstinputlisting[caption=\Optimizing GCC 4.6.3 для Raspberry Pi (\ARMMode)]{patterns/14_bitfields/35_set_reset_FPU/raspberry_GCC_O3_ARM_mode_RU.lst}

Запустим Raspberry Pi Linux в QEMU и он эмулирует FPU в ARM, так что здесь используются S-регистры
для передачи значений с плавающей точкой вместо R-регистров.

\myindex{ARM!\Instructions!FMRS}
Инструкция \FMRS копирует данные из \ac{GPR} в FPU и назад.
\TT{my\_abs()} и \TT{set\_sign()} выглядят предсказуемо, но \TT{negate()}?
Почему там \ADD вместо \XOR?

\myindex{ARM!\Instructions!XOR}
\myindex{ARM!\Instructions!ADD}
Трудно поверить, но инструкция 
\INS{ADD register, 0x80000000} работает так же как и \INS{XOR register, 0x80000000}.
Прежде всего, какая наша цель?
Цель в том, чтобы поменять \ac{MSB} на противоположный, и давайте забудем пока об операции \XOR.

Из школьной математики мы можем помнить, что прибавление числа вроде 1000 к другому никогда не затрагивает последние 3 цифры.

Например: $1234567 + 10000 = 1244567$ (последние 4 цифры никогда не меняются).
Но мы работаем с двоичной системой исчисления, и 0x80000000 это 0b100000000000000000000000000000000
в двоичной системе, т.е. только старший бит установлен.

Прибавление 0x80000000 к любому значению никогда не затронет младших 31 бит, а только \ac{MSB}.

Прибавление 1 к 0 в итоге даст 1.
Прибавление 1 к 1 даст 0b10 в двоичном виде, но 32-й бит (считая с нуля) выброшен, 
потому что наши регистры имеют ширину в 32 бита. Так что результат~--- 0.

Вот почему \XOR здесь можно заменить на \ADD.
Трудно сказать, почему GCC решил сделать так, но это работает корректно.%

