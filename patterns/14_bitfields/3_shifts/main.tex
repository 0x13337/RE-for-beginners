\section{\ShiftsSectionName}

\RU{Битовые сдвиги в \CCpp реализованы при помощи операторов $\ll$ и $\gg$.}
\EN{Bit shifts in \CCpp are implemented via $\ll$ and $\gg$ operators.}

\RU{В x86 есть инструкции}\EN{x86 \ac{ISA} has} SHL (SHift Left) \AndENRU SHR (SHift Right) 
\RU{для этого}\EN{instructions for this}.

\subsection{\RU{Деление и умножение при помощи сдвигов}\EN{Division and multiplication using shifts}}
\label{subsec:mult_div_shifts}

\RU{Инструкции сдвига также активно применяются при делении или умножении 
на числа-степени двойки: $2^{n}$ (т.е., $1$, $2$, $4$, $8$, и т.д.).}
\EN{Shift instructions are often used in division and multiplications by power of two numbers:
$2^{n}$ (e.g., $1$, $2$, $4$, $8$, etc).}

\subsubsection{\RU{Умножение}\EN{Multiplication}}

\begin{lstlisting}
unsigned int f(unsigned int a)
{
	return a*4;
};
\end{lstlisting}

\begin{lstlisting}[caption=\NonOptimizing MSVC 2010]
_a$ = 8		; size = 4
_f	PROC
	push	ebp
	mov	ebp, esp
	mov	eax, DWORD PTR _a$[ebp]
	shl	eax, 2
	pop	ebp
	ret	0
_f	ENDP
\end{lstlisting}

\RU{Умножить на $4$ это просто сдвинуть число на 2 бита влево, 
вставив 2 нулевых бита справа (как два самых младших бита). 
Это как умножить $3$ на $100$ ~--- нужно просто дописать два нуля справа.}
\EN{Multiplication by $4$ is just shifting the number to the left by 2 bits,
while inserting 2 zero bits at right (as the last two bits).
It is just like to multiply $3$ by $100$~---we need just to add two zeroes at the right.}

\RU{Вот как работает инструкция сдвига влево}\EN{That's how shift left instruction works}:

\index{x86!\Instructions!SHL}
\begin{center}
	\begin{tikzpicture}[scale=0.7, every node/.style={scale=0.7}]
	\edef\bitsize{1cm}
	\tikzstyle{byte}=[draw,minimum size=\bitsize]	
	\tikzstyle{every path}=[thick]

	\node [draw,rectangle,minimum size=\bitsize] (a1) {7};
	\node [draw,rectangle,minimum size=\bitsize] (a2) [right of=a1] {6};
	\node [draw,rectangle,minimum size=\bitsize] (a3) [right of=a2] {5};
	\node [draw,rectangle,minimum size=\bitsize] (a4) [right of=a3] {4};
	\node [draw,rectangle,minimum size=\bitsize] (a5) [right of=a4] {3};
	\node [draw,rectangle,minimum size=\bitsize] (a6) [right of=a5] {2};
	\node [draw,rectangle,minimum size=\bitsize] (a7) [right of=a6] {1};
	\node [draw,rectangle,minimum size=\bitsize] (a8) [right of=a7] {0};

	\node (empty) [below of=a1] {};

	\node [draw,rectangle,minimum size=\bitsize] (b1) [below of=empty] {7};
	\node [draw,rectangle,minimum size=\bitsize] (b2) [right of=b1] {6};
	\node [draw,rectangle,minimum size=\bitsize] (b3) [right of=b2] {5};
	\node [draw,rectangle,minimum size=\bitsize] (b4) [right of=b3] {4};
	\node [draw,rectangle,minimum size=\bitsize] (b5) [right of=b4] {3};
	\node [draw,rectangle,minimum size=\bitsize] (b6) [right of=b5] {2};
	\node [draw,rectangle,minimum size=\bitsize] (b7) [right of=b6] {1};
	\node [draw,rectangle,minimum size=\bitsize] (b8) [right of=b7] {0};
	
	\node [shape=rectangle,draw,minimum size=\bitsize] (d) [left=of b1] {nowhere};
	\node [shape=rectangle,draw,minimum size=\bitsize] (c) [right=of b8] {0};
	
	\draw [->] (c.west) -- (b8.east);

	\draw [->] (a2.south) -- (b1.north);
	\draw [->] (a3.south) -- (b2.north);
	\draw [->] (a4.south) -- (b3.north);
	\draw [->] (a5.south) -- (b4.north);
	\draw [->] (a6.south) -- (b5.north);
	\draw [->] (a7.south) -- (b6.north);
	\draw [->] (a8.south) -- (b7.north);
	
	\draw [->] (a1.south) -- (d.north);

	\end{tikzpicture}
\end{center}


\RU{Добавленные биты справа --- всегда нули}\EN{Added bits at right---always zeroes}.

\RU{Умножение на 4 в}\EN{Multiplication by 4 in} ARM:

\begin{lstlisting}[caption=\NonOptimizingKeilVI + \ARMMode]
f PROC
        LSL      r0,r0,#2
        BX       lr
        ENDP
\end{lstlisting}

\subsubsection{\RU{Деление}\EN{Division}}

\RU{Например}\EN{For example}:

\begin{lstlisting}
unsigned int f(unsigned int a)
{
	return a/4;
};
\end{lstlisting}

\RU{Имеем в итоге}\EN{We got} (MSVC 2010):

\begin{lstlisting}[caption=MSVC 2010]
_a$ = 8							; size = 4
_f	PROC
	mov	eax, DWORD PTR _a$[esp-4]
	shr	eax, 2
	ret	0
_f	ENDP
\end{lstlisting}

\label{SHR}
\index{x86!\Instructions!SHR}
\RU{Инструкция \SHR (\IT{SHift Right}) в данном примере сдвигает число на 2 бита вправо. 
При этом, освободившиеся два бита слева (т.е., самые 
старшие разряды), выставляются в нули. А самые младшие 2 бита выкидываются. 
Фактически, эти два выкинутых бита ~--- остаток от деления.}
\EN{\SHR (\IT{SHift Right}) instruction in this example is shifting a number by 2 bits right.
Two freed bits at left (e.g., two most significant bits) are set to zero.
Two least significant bits are dropped.
In fact, these two dropped bits~---division operation remainder.}

\index{x86!\Instructions!SHR}
\RU{Инструкция \SHR работает так же, как и \SHL, только в другую сторону.}
\EN{\SHR instruction works just like as \SHL but in other direction.}

\begin{center}
	\begin{tikzpicture}[scale=0.7, every node/.style={scale=0.7}]
	\edef\bitsize{1cm}
	\tikzstyle{byte}=[draw,minimum size=\bitsize]	
	\tikzstyle{every path}=[thick]

	\node [draw,rectangle,minimum size=\bitsize] (a1) {7};
	\node [draw,rectangle,minimum size=\bitsize] (a2) [right of=a1] {6};
	\node [draw,rectangle,minimum size=\bitsize] (a3) [right of=a2] {5};
	\node [draw,rectangle,minimum size=\bitsize] (a4) [right of=a3] {4};
	\node [draw,rectangle,minimum size=\bitsize] (a5) [right of=a4] {3};
	\node [draw,rectangle,minimum size=\bitsize] (a6) [right of=a5] {2};
	\node [draw,rectangle,minimum size=\bitsize] (a7) [right of=a6] {1};
	\node [draw,rectangle,minimum size=\bitsize] (a8) [right of=a7] {0};

	\node (empty) [below of=a1] {};

	\node [draw,rectangle,minimum size=\bitsize] (b1) [below of=empty] {7};
	\node [draw,rectangle,minimum size=\bitsize] (b2) [right of=b1] {6};
	\node [draw,rectangle,minimum size=\bitsize] (b3) [right of=b2] {5};
	\node [draw,rectangle,minimum size=\bitsize] (b4) [right of=b3] {4};
	\node [draw,rectangle,minimum size=\bitsize] (b5) [right of=b4] {3};
	\node [draw,rectangle,minimum size=\bitsize] (b6) [right of=b5] {2};
	\node [draw,rectangle,minimum size=\bitsize] (b7) [right of=b6] {1};
	\node [draw,rectangle,minimum size=\bitsize] (b8) [right of=b7] {0};
	
	\node [shape=rectangle,draw,minimum size=\bitsize] (c) [left=of b1] {0};
	\node [shape=rectangle,draw,minimum size=\bitsize] (d) [right=of b8] {CF};
	
	\draw [->] (c.east) -- (b1.west);

	\draw [->] (a1.south) -- (b2.north);
	\draw [->] (a2.south) -- (b3.north);
	\draw [->] (a3.south) -- (b4.north);
	\draw [->] (a4.south) -- (b5.north);
	\draw [->] (a5.south) -- (b6.north);
	\draw [->] (a6.south) -- (b7.north);
	\draw [->] (a7.south) -- (b8.north);
	
	\draw [->] (a8.south) -- (d.north);

	\end{tikzpicture}
\end{center}



\label{division_by_shifting}
\RU{Для того, чтобы это проще понять, представьте себе десятичную систему счисления и число $23$. 
$23$ можно разделить на $10$ просто откинув последний разряд ($3$ ~--- это остаток от деления). 
После этой операции останется $2$ как \glslink{quotient}{частное}.}
\EN{It can be easily understood if to imagine decimal numeral system and number $23$.
$23$ can be easily divided by $10$ just by dropping last digit ($3$~---is division remainder). 
$2$ is leaving after operation as a \gls{quotient}.}

\RU{Деление на 4 в}\EN{Division by 4 in} ARM:

\begin{lstlisting}[caption=\NonOptimizingKeilVI + \ARMMode]
f PROC
        LSR      r0,r0,#2
        BX       lr
        ENDP
\end{lstlisting}
