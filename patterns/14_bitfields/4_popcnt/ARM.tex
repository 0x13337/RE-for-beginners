\subsection{ARM + \OptimizingXcodeIV (\ARMMode)}

\lstinputlisting[caption=\OptimizingXcodeIV (\ARMMode),label=ARM_leaf_example4]{patterns/14_bitfields/4_popcnt/ARM_Xcode_O3.lst.\LANG}

\myindex{ARM!\Instructions!TST}
\TST \RU{это то же что и}\EN{is the same things as} \TEST \InENRU x86.

\myindex{ARM!Optional operators!LSL}
\myindex{ARM!Optional operators!LSR}
\myindex{ARM!Optional operators!ASR}
\myindex{ARM!Optional operators!ROR}
\myindex{ARM!Optional operators!RRX}
\myindex{ARM!\Instructions!MOV}
\myindex{ARM!\Instructions!TST}
\myindex{ARM!\Instructions!CMP}
\myindex{ARM!\Instructions!ADD}
\myindex{ARM!\Instructions!SUB}
\myindex{ARM!\Instructions!RSB}
\RU{Как уже было указано}\EN{As was noted before}~(\myref{shifts_in_ARM_mode}),
\RU{в режиме ARM нет отдельной инструкции для сдвигов.}
\EN{there are no separate shifting instructions in ARM mode.}
\RU{Однако, модификаторами}\EN{However, there are modifiers} 
LSL (\IT{Logical Shift Left}), 
LSR (\IT{Logical Shift Right}), 
ASR (\IT{Arithmetic Shift Right}), 
ROR (\IT{Rotate Right}) \AndENRU 
RRX (\IT{Rotate Right with Extend}) \RU{можно дополнять некоторые инструкции, такие как}
\EN{, which may be added to such instructions as} \MOV, \TST,
\CMP, \ADD, \SUB, \RSB\footnote{\DataProcessingInstructionsFootNote}.

\RU{Эти модификаторы указывают, как сдвигать второй операнд, и на сколько.}
\EN{These modificators define how to shift the second operand and by how many bits.}

\myindex{ARM!\Instructions!TST}
\myindex{ARM!Optional operators!LSL}
\RU{Таким образом, инструкция }\EN{Thus the} \TT{\q{TST R1, R2,LSL R3}} 
\RU{здесь работает как}\EN{instruction works here as} $R1 \land (R2 \ll R3)$.

\subsection{ARM + \OptimizingXcodeIV (\ThumbTwoMode)}

\myindex{ARM!\Instructions!LSL.W}
\myindex{ARM!\Instructions!LSL}
\RU{Почти такое же}\EN{Almost the same}, 
\RU{только здесь применяется пара инструкций}\EN{but here are two} 
\INS{LSL.W}/\TST 
\RU{вместо одной}\EN{instructions are used instead of a single} 
\TST,
\RU{ведь в режиме Thumb нельзя добавлять модификатор}\EN{because in Thumb mode it is not
possible to define} \LSL \RU{прямо в}\EN{modifier directly in} \TST.

\begin{lstlisting}[label=ARM_leaf_example5]
                MOV             R1, R0
                MOVS            R0, #0
                MOV.W           R9, #1
                MOVS            R3, #0
loc_2F7A
                LSL.W           R2, R9, R3
                TST             R2, R1
                ADD.W           R3, R3, #1
                IT NE
                ADDNE           R0, #1
                CMP             R3, #32
                BNE             loc_2F7A
                BX              LR
\end{lstlisting}

\subsection{ARM64 + \Optimizing GCC 4.9}

\RU{Возьмем 64-битный пример, который уже был здесь использован}\EN{Let's take the 64-bit example which has been already used}: 
\myref{popcnt_x64_example}.

\lstinputlisting[caption=\Optimizing GCC (Linaro) 4.8]{patterns/14_bitfields/4_popcnt/ARM64_GCC_O3.s.\LANG}

\RU{Результат очень похож на тот, что GCC сгенерировал для x64}\EN{The result is very similar to what GCC 
generates for x64}: \myref{shifts64_GCC_O3}.

\myindex{ARM!\Instructions!CSEL}
\EN{The}\RU{Инструкция} \CSEL \RU{это}\EN{instruction is} \q{Conditional SELect}\RU{ (выбор при условии)}. 
\RU{Она просто выбирает одну из переменных, в зависимости от флагов выставленных}\EN{It just chooses one 
variable of two depending on the flags set by} \TST \RU{и копирует значение в регистр}\EN{and copies the value 
into} \RegW{2}\RU{, содержащий переменную \q{rt}}\EN{, which holds the \q{rt} variable}.

\subsection{ARM64 + \NonOptimizing GCC 4.9}

\RU{И снова будем использовать 64-битный пример, который мы использовали ранее}
\EN{And again, we'll work on the 64-bit example which was already used}: \myref{popcnt_x64_example}.

\RU{Код более многословный, как обычно}\EN{The code is more verbose, as usual}.

\lstinputlisting[caption=\NonOptimizing GCC (Linaro) 4.8]{patterns/14_bitfields/4_popcnt/ARM64_GCC_O0.s.\LANG}
