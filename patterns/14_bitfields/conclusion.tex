%FIXME ROL/ROR
\section{\Conclusion{}}

\RU{Инструкции сдвига, аналогичные операторам \CCpp $\ll$ и $\gg$, в x86 это SHR/SHL (для беззнаковых значений),
SAR/SHL (для знаковых значений).}
\EN{Analogous to the \CCpp shifting operators $\ll$ and $\gg$, 
shift instructions in x86 are SHR/SHL (for unsigned values), SAR/SHL (for signed values).}

\RU{Инструкции сдвига в ARM это LSR/LSL (для беззнаковых значений), ASR/LSL (для знаковых значений).}
\EN{ARM shift instructions are LSR/LSL (for unsigned values), ASR/LSL (for signed values).}
\RU{Можно также добавлять суффикс сдвига для некоторых инструкций.}
\EN{It's also possible to add shifting suffix to some instructions.}
% FIXME: which instructions?

\subsection{\RU{Проверка определенного бита (известного на стадии компиляции)}
\EN{Check for specific bit (known at compiling stage)}}

\RU{Проверить, присутствует ли бит 1000000 (0x40) в значении в регистре:}
\EN{Test if the 1000000 bit (0x40) is present in register's value:}

\lstinputlisting[caption=\CCpp]{patterns/14_bitfields/c_snippet0.c}

% FIXME: translate to Russian...
\begin{lstlisting}[caption=x86]
TEST REG, 40h
JNZ is_set
; bit is not set
\end{lstlisting}

\begin{lstlisting}[caption=x86]
TEST REG, 40h
JZ is_cleared
; bit is set
\end{lstlisting}

\begin{lstlisting}[caption=ARM (\ARMMode)]
TST REG, #0x40
BNE is_set
; bit is not set
\end{lstlisting}

\RU{Иногда AND используется вместо TEST, но флаги выставляются точно также.}
\EN{Sometimes, AND is used instead of TEST, but flags are set just as the same.}

\subsection{\RU{Проверка определенного бита (заданного во время исполнения)}
\EN{Check for specific bit (specified at runtime)}}

\RU{Это обычно происходит при помощи вот такого фрагмента на \CCpp (сдвинуть значение на $n$ бит вправо,
затем отрезать самый младший бит):}
\EN{This is usually done by this \CCpp code snippet (shift value by $n$ bits right, then cut off lowest bit):}

\lstinputlisting[caption=\CCpp]{patterns/14_bitfields/c_snippet1.c}

\RU{Это обычно реализуется в x86-коде так:}
\EN{This is usually implemented in x86 code as:}

\begin{lstlisting}[caption=x86]
; REG=input_value
; CL=n
SHR REG, CL
AND REG, 1
\end{lstlisting}

\RU{Или (сдвинуть $1$ $n$ раз влево, изолировать этот же бит во входном значении и проверить, не ноль ли он):}
\EN{Or (shift $1$ bit $n$ times left, isolate this bit in input value and check, if it's not zero):}

\lstinputlisting[caption=\CCpp]{patterns/14_bitfields/c_snippet2.c}

\RU{Это обычно так реализуется в x86-коде:}
\EN{This is usually implemented in x86 code as:}

\begin{lstlisting}[caption=x86]
; CL=n
MOV REG, 1
SHL REG, CL
AND input_value, REG
\end{lstlisting}

\subsection{\RU{Установка определенного бита (известного во время компиляции)}
\EN{Set specific bit (known at compiling stage)}}

\begin{lstlisting}[caption=\CCpp]
value=value|0x40;
\end{lstlisting}

\begin{lstlisting}[caption=x86]
OR REG, 40h
\end{lstlisting}

\begin{lstlisting}[caption=ARM (\ARMMode) \AndENRU ARM64]
ORR R0, R0, #0x40
\end{lstlisting}

\subsection{\RU{Установка определенного бита (заданного во время исполнения)}
\EN{Set specific bit (specified at runtime)}}

\lstinputlisting[caption=\CCpp]{patterns/14_bitfields/c_snippet3.c}

\RU{Это обычно так реализуется в x86-коде:}
\EN{This is usually implemented in x86 code as:}

\begin{lstlisting}[caption=x86]
; CL=n
MOV REG, 1
SHL REG, CL
OR input_value, REG
\end{lstlisting}

\subsection{\RU{Сброс определенного бита (известного во время компиляции)}
\EN{Clear specific bit (known at compiling stage)}}

\RU{Просто исполните операцию логического ``И'' (AND) с инвертированным значением:}
\EN{Just apply AND operation with inverted value:}

\begin{lstlisting}[caption=\CCpp]
value=value&(~0x40);
\end{lstlisting}

\begin{lstlisting}[caption=x86]
AND REG, 0FFFFFFBFh
\end{lstlisting}

\begin{lstlisting}[caption=x64]
AND REG, 0FFFFFFFFFFFFFFBFh
\end{lstlisting}

\RU{Это на самом деле сохранение всех бит кроме одного.}
\EN{This is actually leaving all bits set but one.}

\RU{В ARM в режиме ARM есть инструкция BIC, работающая как две инструкции NOT+AND:}
\EN{ARM in ARM mode has BIC instruction, which works like two NOT+AND instructions:}

\begin{lstlisting}[caption=ARM (\ARMMode)]
BIC R0, R0, #0x40
\end{lstlisting}

\subsection{\RU{Сброс определенного бита (заданного во время исполнения)}
\EN{Clear specific bit (specified at runtime)}}

\lstinputlisting[caption=\CCpp]{patterns/14_bitfields/c_snippet4.c}

\begin{lstlisting}[caption=x86]
; CL=n
MOV REG, 1
SHL REG, CL
NOT REG
AND input_value, REG
\end{lstlisting}
