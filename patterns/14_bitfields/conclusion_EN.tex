% TODO: add ROL/ROR
\subsection{\Conclusion{}}

\myindex{x86!\Instructions!SHR}
\myindex{x86!\Instructions!SHL}
\myindex{x86!\Instructions!SAR}

Analogous to the \CCpp shifting operators \TT{$\ll$} and \TT{$\gg$},
the shift instructions in x86 are \SHR/\SHL (for unsigned values) and \SAR/\SHL (for signed values).

\myindex{ARM!\Instructions!LSR}
\myindex{ARM!\Instructions!LSL}
\myindex{ARM!\Instructions!ASR}

The shift instructions in ARM are \LSR/\LSL (for unsigned values) and \ASR/\LSL (for signed values).

It's also possible to add shift suffix to some instructions 
(which are called \q{data processing instructions}).
% FIXME: which instructions?

\subsubsection{Check for specific bit (known at compile stage)}

Test if the 1000000 bit (0x40) is present in the register's value:

\lstinputlisting[caption=\CCpp]{patterns/14_bitfields/c_snippet0.c}

\lstinputlisting[caption=x86]{patterns/14_bitfields/TEST_JNZ_EN.lst}

\lstinputlisting[caption=x86]{patterns/14_bitfields/TEST_JZ_EN.lst}

\lstinputlisting[caption=ARM (\ARMMode)]{patterns/14_bitfields/TST_BNE_EN.lst}

\myindex{x86!\Instructions!AND}
\myindex{x86!\Instructions!TEST}

Sometimes, \AND is used instead of \TEST, but the flags that are set are the same.

\subsubsection{Check for specific bit (specified at runtime)}

This is usually done by this \CCpp code snippet (shift value by $n$ bits right, then cut off lowest bit):

\lstinputlisting[caption=\CCpp]{patterns/14_bitfields/c_snippet1.c}

This is usually implemented in x86 code as:

\begin{lstlisting}[caption=x86]
; REG=input_value
; CL=n
SHR REG, CL
AND REG, 1
\end{lstlisting}

Or (shift 1 bit $n$ times left, isolate this bit in input value and check if it's not zero):

\lstinputlisting[caption=\CCpp]{patterns/14_bitfields/c_snippet2.c}

This is usually implemented in x86 code as:

\begin{lstlisting}[caption=x86]
; CL=n
MOV REG, 1
SHL REG, CL
AND input_value, REG
\end{lstlisting}

\subsubsection{Set specific bit (known at compile stage)}

\begin{lstlisting}[caption=\CCpp]
value=value|0x40;
\end{lstlisting}

\begin{lstlisting}[caption=x86]
OR REG, 40h
\end{lstlisting}

\begin{lstlisting}[caption=ARM (\ARMMode) and ARM64]
ORR R0, R0, #0x40
\end{lstlisting}

\subsubsection{Set specific bit (specified at runtime)}

\lstinputlisting[caption=\CCpp]{patterns/14_bitfields/c_snippet3.c}

This is usually implemented in x86 code as:

\begin{lstlisting}[caption=x86]
; CL=n
MOV REG, 1
SHL REG, CL
OR input_value, REG
\end{lstlisting}

\subsubsection{Clear specific bit (known at compile stage)}

Just apply \AND operation with the inverted value:

\begin{lstlisting}[caption=\CCpp]
value=value&(~0x40);
\end{lstlisting}

\begin{lstlisting}[caption=x86]
AND REG, 0FFFFFFBFh
\end{lstlisting}

\begin{lstlisting}[caption=x64]
AND REG, 0FFFFFFFFFFFFFFBFh
\end{lstlisting}

This is actually leaving all bits set except one.

\myindex{ARM!\Instructions!BIC}

ARM in ARM mode has \BIC instruction, which works like the \NOT+\AND instruction pair:

\begin{lstlisting}[caption=ARM (\ARMMode)]
BIC R0, R0, #0x40
\end{lstlisting}

\subsubsection{
Clear specific bit (specified at runtime)}

\lstinputlisting[caption=\CCpp]{patterns/14_bitfields/c_snippet4.c}

\begin{lstlisting}[caption=x86]
; CL=n
MOV REG, 1
SHL REG, CL
NOT REG
AND input_value, REG
\end{lstlisting}
