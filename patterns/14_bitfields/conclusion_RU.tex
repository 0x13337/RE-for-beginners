% TODO: add ROL/ROR
\section{\Conclusion{}}

\myindex{x86!\Instructions!SHR}
\myindex{x86!\Instructions!SHL}
\myindex{x86!\Instructions!SAR}
Инструкции сдвига, аналогичные операторам \CCpp $\ll$ и $\gg$, в x86 это \SHR/\SHL (для беззнаковых значений), \SAR/\SHL (для знаковых значений).

\myindex{ARM!\Instructions!LSR}
\myindex{ARM!\Instructions!LSL}
\myindex{ARM!\Instructions!ASR}
Инструкции сдвига в ARM это \LSR/\LSL (для беззнаковых значений), \ASR/\LSL (для знаковых значений).

Можно также добавлять суффикс сдвига для некоторых инструкций 
(которые называются \q{data processing instructions}).

% FIXME: which instructions?

\subsection{Проверка определенного бита (известного на стадии компиляции)}

Проверить, присутствует ли бит 1000000 (0x40) в значении в регистре:

\lstinputlisting[caption=\CCpp]{patterns/14_bitfields/c_snippet0.c}

\lstinputlisting[caption=x86]{patterns/14_bitfields/TEST_JNZ_RU.lst}

\lstinputlisting[caption=x86]{patterns/14_bitfields/TEST_JZ_RU.lst}

\lstinputlisting[caption=ARM (\ARMMode)]{patterns/14_bitfields/TST_BNE_RU.lst}

\myindex{x86!\Instructions!AND}
\myindex{x86!\Instructions!TEST}
Иногда \AND используется вместо \TEST, но флаги выставляются точно также.

\subsection{Проверка определенного бита (заданного во время исполнения)}

Это обычно происходит при помощи вот такого фрагмента на \CCpp (сдвинуть значение на $n$ бит вправо,
затем отрезать самый младший бит):

\lstinputlisting[caption=\CCpp]{patterns/14_bitfields/c_snippet1.c}

Это обычно реализуется в x86-коде так:

\begin{lstlisting}[caption=x86]
; REG=input_value
; CL=n
SHR REG, CL
AND REG, 1
\end{lstlisting}

Или (сдвинуть 1 $n$ раз влево, изолировать этот же бит во входном значении и проверить, не ноль ли он):

\lstinputlisting[caption=\CCpp]{patterns/14_bitfields/c_snippet2.c}

Это обычно так реализуется в x86-коде:

\begin{lstlisting}[caption=x86]
; CL=n
MOV REG, 1
SHL REG, CL
AND input_value, REG
\end{lstlisting}

\subsection{Установка определенного бита (известного во время компиляции)}

\begin{lstlisting}[caption=\CCpp]
value=value|0x40;
\end{lstlisting}

\begin{lstlisting}[caption=x86]
OR REG, 40h
\end{lstlisting}

\begin{lstlisting}[caption=ARM (\ARMMode) и ARM64]
ORR R0, R0, #0x40
\end{lstlisting}

\subsection{Установка определенного бита (заданного во время исполнения)}

\lstinputlisting[caption=\CCpp]{patterns/14_bitfields/c_snippet3.c}

Это обычно так реализуется в x86-коде:

\begin{lstlisting}[caption=x86]
; CL=n
MOV REG, 1
SHL REG, CL
OR input_value, REG
\end{lstlisting}

\subsection{Сброс определенного бита (известного во время компиляции)}

Просто исполните операцию логического \q{И} (\AND) с инвертированным значением:

\begin{lstlisting}[caption=\CCpp]
value=value&(~0x40);
\end{lstlisting}

\begin{lstlisting}[caption=x86]
AND REG, 0FFFFFFBFh
\end{lstlisting}

\begin{lstlisting}[caption=x64]
AND REG, 0FFFFFFFFFFFFFFBFh
\end{lstlisting}

Это на самом деле сохранение всех бит кроме одного.

\myindex{ARM!\Instructions!BIC}
В ARM в режиме ARM есть инструкция \BIC, работающая как две инструкции \NOT+\AND:

\begin{lstlisting}[caption=ARM (\ARMMode)]
BIC R0, R0, #0x40
\end{lstlisting}

\subsection{Сброс определенного бита (заданного во время исполнения)}

\lstinputlisting[caption=\CCpp]{patterns/14_bitfields/c_snippet4.c}

\begin{lstlisting}[caption=x86]
; CL=n
MOV REG, 1
SHL REG, CL
NOT REG
AND input_value, REG
\end{lstlisting}
