\section{\RU{Выделяем место для структуры через malloc()}\EN{Let's allocate space for a structure using malloc()}}
\label{struct_malloc_example}

\RU{Однако, бывает и так, что проще хранить структуры не в стеке, а в \glslink{heap}{куче}:}
\EN{Sometimes it is simpler to place structures not the in local stack, but in the \gls{heap}:}

\lstinputlisting{patterns/15_structs/2_using_malloc/systemtime_malloc.c}

\RU{Скомпилируем на этот раз с оптимизацией (\Ox) чтобы было проще увидеть то, что нам нужно.}
\EN{Let's compile it now with optimization (\Ox) so it would be easy see what we need.}

\lstinputlisting[caption=\Optimizing MSVC]{patterns/15_structs/2_using_malloc/systemtime_malloc.asm}

\index{\CStandardLibrary!malloc()}
\RU{Итак, \TT{sizeof(SYSTEMTIME) = 16}, именно столько байт выделяется при помощи \TT{malloc()}. 
Она возвращает указатель на только что выделенный блок памяти в \EAX, который копируется в \ESI. 
Win32 функция \TT{GetSystemTime()} обязуется сохранить состояние \ESI, 
поэтому здесь оно нигде не сохраняется и продолжает использоваться после вызова \TT{GetSystemTime()}.}
\EN{So, \TT{sizeof(SYSTEMTIME) = 16} and that is exact number of bytes to be allocated by \TT{malloc()}.
It returns a pointer to a freshly allocated memory block in the \EAX register,
which is then moved into the \ESI register.
\TT{GetSystemTime()} win32 function takes care of saving value in \ESI,
and that is why it is not saved here and continues to be used after the \TT{GetSystemTime()} call.}

\index{x86!\Instructions!MOVZX}
\RU{
Новая инструкция ~--- \MOVZX (\IT{Move with Zero eXtend}). 
Она нужна почти там же где и \MOVSX, 
только всегда очищает остальные биты в 0. Дело в том, что \printf требует 32-битный тип \Tint, 
а в структуре лежит WORD ~--- это 16-битный беззнаковый тип. Поэтому копируя значение из WORD в \Tint, 
нужно очистить биты от 16 до 31, иначе там будет просто случайный мусор, оставшийся от предыдущих действий 
с регистрами.}
\EN{New instruction~---\MOVZX (\IT{Move with Zero eXtend}).
It may be used in most cases as \MOVSX, but it sets the remaining bits to 0.
That's because \printf requires a 32-bit \Tint, but we got a WORD in the structure~---that
is 16-bit unsigned type.
That's why by copying the value from a WORD into \Tint{}, bits from 16 to 31 must be cleared, 
because a random noise may be there, which is left from the previous operations on the register(s).}

\RU{В этом примере я тоже могу представить структуру как массив 8-и WORD-ов:}
\EN{In this example, I again can represent the structure as an array of 8 WORDs:}

\lstinputlisting{patterns/15_structs/2_using_malloc/systemtime_malloc2.c}

\RU{Получим такое}\EN{We get}:

\lstinputlisting[caption=\Optimizing MSVC]{patterns/15_structs/2_using_malloc/systemtime_malloc2.asm}

\RU{И снова мы получаем идентичный код, неотличимый от предыдущего.}
\EN{Again, we got the code cannot be distinguished from the previous one.}
\RU{Но и снова я должен отметить, что в реальности так лучше не делать, 
если только вы не знаете точно, что вы делаете.}
\EN{And again I have to note, you haven't to do this in practice, unless you really know what you are doing.}

