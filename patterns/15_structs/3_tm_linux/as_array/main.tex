\ifdefined\IncludeGCC
\subsection{\RU{Структура как набор переменных}\EN{Structure as a set of values}}

\RU{Чтобы проиллюстрировать то что структура ~--- это просто набор переменных, лежащих в одном месте, 
переделаем немного пример, еще раз заглянув в описание структуры \IT{tm}}
\EN{In order to illustrate that the structure is just variables laying side-by-side in one place, 
let's rework our example while looking at the \IT{tm} structure definition again}: \lstref{struct_tm}.

\lstinputlisting{patterns/15_structs/3_tm_linux/as_array/GCC_tm2.c}

\index{\CStandardLibrary!localtime\_r()}
N.B. \RU{В \TT{localtime\_r} передается указатель именно на \TT{tm\_sec}, 
т.е., на первый элемент ``структуры''.}
\EN{The pointer to the \TT{tm\_sec} field is passed into \TT{localtime\_r}, i.e., 
to the first element of the ``structure''.}

\RU{В итоге, и этот компилятор поворчит}\EN{The compiler will warn us}:

\begin{lstlisting}[caption=GCC 4.7.3]
GCC_tm2.c: In function 'main':
GCC_tm2.c:11:5: warning: passing argument 2 of 'localtime_r' from incompatible pointer type [enabled by default]
In file included from GCC_tm2.c:2:0:
/usr/include/time.h:59:12: note: expected 'struct tm *' but argument is of type 'int *'
\end{lstlisting}

\RU{Тем не менее, сгенерирует такое}\EN{But nevertheless, it will generate this}:

\lstinputlisting[caption=GCC 4.7.3]{patterns/15_structs/3_tm_linux/as_array/GCC_tm2.asm}

\RU{Этот код почти идентичен уже рассмотренному, и нельзя сказать, была ли структура
в оригинальном исходном коде либо набор переменных.}\EN{This code is identical to what we saw previously and it is
not possible to say, was it a structure in original source code or just a pack of variables.}

\RU{И это работает}\EN{And this works}. 
\RU{Однако, в реальности так лучше не делать}\EN{However, it is not recommended to do this in practice}. 
\RU{Обычно, неоптимизируюий компилятор располагает переменные в локальном
стеке в том же порядке, в котором они объявляются в функции.}
\EN{Usually, non-optimizing compilers allocates variables in the local stack in the 
same order as they were declared in the function.}
\RU{Тем не менее, никакой гарантии нет.}
\EN{Nevertheless, there is no guarantee.}

\RU{Кстати, какой-нибудь другой компилятор может предупредить, что переменные}
\EN{By the way, some other compiler may warn about the} \TT{tm\_year}, \TT{tm\_mon}, \TT{tm\_mday},
\TT{tm\_hour}, \TT{tm\_min}\RU{, но не}\EN{ variables, but not} \TT{tm\_sec}\RU{, используются без инициализации}
\EN{ are used without being initialized}.
\RU{Действительно, ведь компилятор не знает что они будут заполнены при вызове функции}
\EN{Indeed, the compiler is not aware that these will be filled by} \TT{localtime\_r()}.

\EN{I chose this example, since all structure fields are of type \Tint.}
\RU{Я выбрал именно этот пример для иллюстрации, потому что все члены структуры имеют тип \Tint.}
\RU{Это не сработает, если поля структуры будут иметь размер 16 бит (\TT{WORD}), как в случае
со структурой \TT{SYSTEMTIME}}\EN{This would not work if structure fields are 16-bit (\TT{WORD}), 
like in the case of the \TT{SYSTEMTIME} structure}\EMDASH{}\TT{GetSystemTime()} 
\RU{заполнит их неверно}\EN{will fill them incorrectly} 
(\RU{потому что локальные переменные будут выровнены по 32-битной границе}
\EN{because the local variables will be aligned on a 32-bit boundary}).
\RU{Читайте об этом в следующей секции}\EN{Read more about it in next section}: 
``\StructurePackingSectionName'' (\myref{structure_packing}).

\index{\SyntacticSugar}
\RU{Так что, структура ~--- это просто набор переменных лежащих в одном месте, рядом.}
\EN{So, a structure is just a pack of variables laying on one place, side-by-side.}
\RU{Я мог бы сказать, что структура ~--- это такой синтаксический сахар, 
заставляющий компилятор удерживать их в одном месте.}
\EN{I could say that the structure is syntactic sugar, directing
the compiler to hold them in one place.}
\RU{Впрочем, я не специалист по языкам программирования, так что, скорее всего, ошибаюсь с этим термином.}
\EN{However, I'm not an expert on programming languages, so most likely I'm wrong with this term.}
\RU{Кстати, когда-то, в очень ранних версиях Си (перед 1972) структур не было вовсе}
\EN{By the way, in some very early C versions (before 1972), 
there were no structures at all} \cite{Ritchie:1993:DCL:155360.155580}.

\RU{Здесь я не добавляю пример с отладчиком: потому что он будет полностью идентичным тому, что вы уже
видели.}\EN{I'm not adding a debugger example here: it will be just the same as you already
saw.}

\subsection{\RU{Структура как массив 32-битных слов}\EN{Structure as an array of 32-bit words}}

\lstinputlisting{patterns/15_structs/3_tm_linux/as_array/GCC_tm3.c}

\RU{Я просто привожу (cast) указатель на структуру к массиву \Tint{}-ов}\EN{I just cast a pointer to structure 
to an array of \Tint{}'s}.
\RU{И это работает}\EN{And that works}!
\RU{Я запустил пример}\EN{I ran the example at} 23:51:45 26-July-2014.

\begin{lstlisting}[label=GCC_tm3_output]
0x0000002D (45)
0x00000033 (51)
0x00000017 (23)
0x0000001A (26)
0x00000006 (6)
0x00000072 (114)
0x00000006 (6)
0x000000CE (206)
0x00000001 (1)
\end{lstlisting}

\RU{Переменные здесь в том же порядке, в котором они перечислены в определении структуры}\EN{The variables here 
are in the same order as they are enumerated in the definition of the structure}: \myref{struct_tm}.

\RU{Вот как это компилируется}\EN{Here is how it gets compiled}:

\lstinputlisting[caption=\Optimizing GCC 4.8.1]{patterns/15_structs/3_tm_linux/as_array/GCC_tm3.lst.\LANG}

\RU{И действительно: место в локальном стеке в начале используется как структура, затем как массив.}
\EN{Indeed: the space in the local stack is first treated as a structure, and then it's treated as an array.}

\RU{Возможно даже модифицировать поля структуры через указатель.}
\EN{It's even possible to modify the fields of the structure through this pointer.}

\RU{И снова, это сомнительный хакерский способ, который не рекомендуется использовать в настоящем коде.}
\EN{And again, it's dubiously hackish way to do things, not recommended for use in production code.}

\ifdefined\IncludeExercises
\myparagraph{\Exercise}

\RU{В качестве упражнения, попробуйте модифицировать (увеличить на 1) 
текущий номер месяца обращаясь со структурой как с массивом.}
\EN{As an exercise, try to modify (increase by 1) the current month number, treating the structure as 
an array.}
\fi

\subsection{\RU{Структура как массив байт}\EN{Structure as an array of bytes}}

\RU{Я могу даже больше. Можно привести (cast) указатель к массиву байт и вывести его:}
\EN{I can do even more. Let's cast the pointer to an array of bytes and dump it:}

\lstinputlisting{patterns/15_structs/3_tm_linux/as_array/GCC_tm4.c}

\begin{lstlisting}
0x2D 0x00 0x00 0x00 
0x33 0x00 0x00 0x00 
0x17 0x00 0x00 0x00 
0x1A 0x00 0x00 0x00 
0x06 0x00 0x00 0x00 
0x72 0x00 0x00 0x00 
0x06 0x00 0x00 0x00 
0xCE 0x00 0x00 0x00 
0x01 0x00 0x00 0x00 
\end{lstlisting}

\RU{Я также запустил этот пример}\EN{I ran this example also at} 23:51:45 26-July-2014.
\RU{Переменные точно такие же, как и в предыдущем выводе}\EN{The values are just the same as in the previous dump} 
(\myref{GCC_tm3_output}), \RU{и конечно, младший байт идет в самом начале, потому что это архитектура 
little-endian}\EN{and of course, the lowest byte goes first, because this is a little-endian architecture} 
(\myref{sec:endianness}).

\lstinputlisting[caption=\Optimizing GCC 4.8.1]{patterns/15_structs/3_tm_linux/as_array/GCC_tm4.lst.\LANG}
\fi
