\subsection{Linux}

\RU{В Линуксе, для примера, возьмем структуру \TT{tm} из \TT{time.h}:}
\EN{As of Linux, let's take \TT{tm} structure from \TT{time.h} for example:}

\lstinputlisting{patterns/15_structs/3_tm_linux/GCC_tm.c}

\RU{Компилируем при помощи}\EN{Let's compile it in} GCC 4.4.1:

\lstinputlisting[caption=GCC 4.4.1]{patterns/15_structs/3_tm_linux/GCC_tm_\LANG.asm}

\RU{К сожалению, по какой-то причине, \IDA не сформировала названия локальных переменных в стеке. 
Но так как мы уже опытные реверсеры :-) то можем обойтись и без этого в таком простом примере.}
\EN{Somehow, \IDA did not created local variables names in local stack.
But since we already experienced reverse engineers :-) we may do it without this information in 
this simple example.}

\index{x86!\Instructions!LEA}
\RU{Обратите внимание на \TT{lea edx, [eax+76Ch]} ~--- эта инструкция прибавляет \TT{0x76C} (1900) к \EAX, 
но не модифицирует флаги. См. также соответствующий раздел об инструкции \LEA{}~(\ref{sec:LEA}).}
\EN{Please also pay attention to the \TT{lea edx, [eax+76Ch]}~---this instruction just adding \TT{0x76C} (1900) to value in the \EAX,
but not modifies any flags. See also relevant section about \LEA{}~(\ref{sec:LEA}).}

\ifdefined\IncludeGDB
\subsubsection{GDB}

\RU{Попробуем загрузить пример в GDB}\EN{Let's try to load the exapmle into GDB}
\footnote{\RU{Я немного подкорректировал результат работы \IT{date} в целях демонстрации.}
\EN{I corrected the \IT{date} result slightly for demonstration purposes.}
\RU{Конечно же, в реальности, у меня бы не получилось так быстро запустить GDB, чтобы значение секунд
осталось бы таким же.}
\EN{Of course, I wasn't able to run GDB that quickly in the same second.}}:

\lstinputlisting[caption=GDB]{patterns/15_structs/3_tm_linux/GCC_tm_GDB.txt}

\RU{Мы легко находим нашу структуру в стеке}\EN{We can easily find our structure in the stack}.
\RU{Для начала, посмотрим, как она объявлена в}\EN{First, let's see how it's defined in} \IT{time.h}:

\begin{lstlisting}[caption=time.h, label=struct_tm]
struct tm
{
  int	tm_sec;
  int	tm_min;
  int	tm_hour;
  int	tm_mday;
  int	tm_mon;
  int	tm_year;
  int	tm_wday;
  int	tm_yday;
  int	tm_isdst;
};
\end{lstlisting}

\RU{Обратите внимание что здесь 32-битные \Tint вместо WORD в SYSTEMTIME}\EN{Take a notice that
32-bit \Tint here instead of WORD in SYSTEMTIME}.
\RU{Так что, каждое поле занимает 32-битное слово}\EN{So, each field occupies 32-bit word}.

\RU{Вот поля нашей структуры в стеке}\EN{Here is a fields of our structure in the stack}:

\begin{lstlisting}
0xbffff0dc:	0x080484c3	0x080485c0	0x000007de	0x00000000
0xbffff0ec:	0x08048301	0x538c93ed	0x00000025 sec	0x0000000a min
0xbffff0fc:	0x00000012 hour	0x00000002 mday	0x00000005 mon 	0x00000072 year
0xbffff10c:	0x00000001 wday	0x00000098 yday	0x00000001 isdst0x00002a30
0xbffff11c:	0x0804b090	0x08048530	0x00000000	0x00000000
\end{lstlisting}

\RU{Либо же, в виде таблицы}\EN{Or as a table}:

\begin{center}
\begin{tabular}{ | l | l | l | }
\hline
\headercolor{} \RU{Шестнадцатиричное число}\EN{Hexadecimal number} & 
\headercolor{} \RU{десятичное число}\EN{decimal number} & 
\headercolor{} \RU{имя поля}\EN{field name} \\
\hline
0x00000025 & 37 	& tm\_sec \\
\hline
0x0000000a & 10 	& tm\_min \\
\hline
0x00000012 & 18 	& tm\_hour \\	
\hline
0x00000002 & 2 		& tm\_mday \\	
\hline
0x00000005 & 5 		& tm\_mon \\	
\hline
0x00000072 & 114 	& tm\_year \\
\hline
0x00000001 & 1 		& tm\_wday \\	
\hline
0x00000098 & 152 	& tm\_yday \\	
\hline
0x00000001 & 1 		& tm\_isdst \\
\hline
\end{tabular}
\end{center}

\RU{Как и в примере с}\EN{Just like in case of} SYSTEMTIME (\ref{sec:SYSTEMTIME}), 
\RU{здесь есть и другие поля, готовые для использования, 
но в нашем примере они не используются, например}
\EN{there are also other fields available, but not used, like} tm\_wday, tm\_yday, tm\_isdst.
\fi
