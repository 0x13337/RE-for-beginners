\subsectionold{ARM}

\subsubsectionold{\OptimizingKeilVI (\ThumbMode)}

\lstinputlisting[caption=\OptimizingKeilVI (\ThumbMode)]{patterns/15_structs/4_packing/packing_Keil_thumb.asm}

As we may recall, here a structure is passed instead of pointer to one,
and since the first 4 function arguments in ARM are passed via registers,
the structure's fields are passed via \TT{R0-R3}.

\myindex{ARM!\Instructions!LDRB}
\myindex{x86!\Instructions!MOVSX}
\TT{LDRB} loads one byte from memory and extends it to 32-bit, taking its sign into account.
This is similar to \MOVSX in x86.
Here it is used to load fields $a$ and $c$ from the structure.

\myindex{Function epilogue}

One more thing we spot easily is that instead of function epilogue, there is jump to another function's epilogue!
Indeed, that was quite different function, not related in any way to ours, however, it has exactly
the same epilogue 
(probably because, it hold 5 local variables too 
($5*4=0x14$)).

Also it is located nearby (take a look at the addresses).

Indeed, it doesn't matter which epilogue gets executed,
if it works just as we need.

Apparently, Keil decides to reuse a part of another function to economize.

The epilogue takes 4 bytes while jump~---only 2.

\subsubsectionold{ARM + \OptimizingXcodeIV (\ThumbTwoMode)}

\lstinputlisting[caption=\OptimizingXcodeIV (\ThumbTwoMode)]{patterns/15_structs/4_packing/packing_Xcode_thumb.asm}

\myindex{ARM!\Instructions!SXTB}
\myindex{x86!\Instructions!MOVSX}
\TT{SXTB} (\IT{Signed Extend Byte}) is analogous to \MOVSX in x86.
All the rest~---just the same.

