\subsection{MIPS}
\label{MIPS_structure_big_endian}

\lstinputlisting[caption=\Optimizing GCC 4.4.5 (IDA),numbers=left]{patterns/15_structs/4_packing/MIPS_O3_IDA.lst}

\RU{Поля структуры приходят в регистрах \$A0..\$A3 и затем перетасовываются в регистры \$A1..\$A4 для \printf.}
\EN{Structure fields come in registers \$A0..\$A3 and then get reshuffled into \$A1..\$A4 for \printf.}
\RU{Но здесь есть две инструкции SRA (\q{Shift Word Right Arithmetic}), которые готовят поля типа \Tchar.}
\EN{But there are two SRA (\q{Shift Word Right Arithmetic}) instructions, which prepare \Tchar fields.}
\RU{Почему}\EN{Why}?
\RU{По умолчанию, MIPS это big-endian архитектура \myref{sec:endianness}, 
и Debian Linux в котором я работаю, также big-endian.}
\EN{MIPS is a big-endian architecture by default \myref{sec:endianness}, 
and the Debian Linux I work in is big-endian as well.}
\RU{Так что когда один байт расположен в 32-битном элементе структуры, он занимает биты 31..24.}
\EN{So when byte variables are stored in 32-bit structure slots, they occupy the high 31..24 bits.}
\RU{И когда переменную типа \Tchar нужно расширить до 32-битного значения, она должна быть сдвинута вправо
на 24 бита.}
\EN{And when a \Tchar variable needs to be extended into a 32-bit value, it must be shifted right by 24 bits.}
\RU{\Tchar это знаковый тип, так что здесь нужно использовать арифметический сдвиг вместо логического.}
\EN{\Tchar is a signed type, so an arithmetical shift is used here instead of logical.}
