\section{\StructurePackingSectionName}

\RU{Достаточно немаловажный момент, это упаковка полей в структурах\footnote{См. также: \URLWPDA}.}
\EN{One important thing is fields packing in structures\footnote{See also: \URLWPDA}.}

\RU{Возьмем простой пример:}\EN{Let's take a simple example:}

\lstinputlisting{patterns/15_structs/4_packing/packing.c}

\RU{Как видно, мы имеем два поля \Tchar (занимающий один байт) и еще два ~--- \Tint (по 4 байта).}
\EN{As we see, we have two \Tchar fields (each is exactly one byte) and two more~---\Tint (each - 4 bytes).}

\subsection{x86}

\RU{Компилируется это все в:}\EN{That's all compiling into:}

\lstinputlisting{patterns/15_structs/4_packing/packing.asm}

\RU{Мы видим здесь что адрес каждого поля в структуре выравнивается по 4-байтной границе. 
Так что каждый \Tchar здесь занимает те же 4 байта что и \Tint. Зачем? 
Затем что процессору удобнее обращаться по таким адресам и кэшировать данные из памяти.}
\EN{As we can see, each field's address is aligned on a 4-bytes border.
That's why each \Tchar occupies 4 bytes here (like \Tint). Why?
Thus it is easier for CPU to access memory at aligned addresses and to cache data from it.}

\RU{Но это не экономично по размеру данных.}\EN{However, it is not very economical in size sense.}

\RU{Попробуем скомпилировать тот же исходник с опцией}\EN{Let's try to compile it with option} (\TT{/Zp1}) 
(\IT{/Zp[n] pack structures on n-byte boundary}).

\lstinputlisting[caption=MSVC /Zp1]{patterns/15_structs/4_packing/packing_msvc_Zp1.asm}

\RU{Теперь структура занимает 10 байт и все \Tchar занимают по байту. Что это дает? 
Экономию места. Недостаток ~--- процессор будет обращаться к этим полям не так эффективно 
по скорости, как мог бы.}
\EN{Now the structure takes only 10 bytes and each \Tchar value takes 1 byte. What it give to us?
Size economy. And as drawback~---CPU will access these fields without maximal performance it can.}

\RU{Как нетрудно догадаться, если структура используется много в каких исходниках и объектных файлах, 
все они должны быть откомпилированы с одним и тем же соглашением об упаковке структур.}
\EN{As it can be easily guessed, if the structure is used in many source and object files,
all these must be compiled with the same convention about structures packing.}

\newcommand{\FNURLMSDNZP}{\footnote{\href{http://msdn.microsoft.com/en-us/library/ms253935.aspx}
{MSDN: Working with Packing Structures}}}
\newcommand{\FNURLGCCPC}{\footnote{\href{http://gcc.gnu.org/onlinedocs/gcc/Structure_002dPacking-Pragmas.html}
{Structure-Packing Pragmas}}}

\RU{Помимо ключа MSVC \TT{/Zp}, указывающего, по какой границе упаковывать поля структур, есть также 
опция компилятора \TT{\#pragma pack}, её можно указывать прямо в исходнике. 
Это справедливо и для MSVC\FNURLMSDNZP и GCC\FNURLGCCPC{}.}
\EN{Aside from MSVC \TT{/Zp} option which set how to align each structure field, here is also
\TT{\#pragma pack} compiler option, it can be defined right in source code.
It is available in both MSVC\FNURLMSDNZP and GCC\FNURLGCCPC{}.}

\RU{Давайте теперь вернемся к \TT{SYSTEMTIME}, которая состоит из 16-битных полей. 
Откуда наш компилятор знает что их надо паковать по однобайтной границе?}
\EN{Let's back to the \TT{SYSTEMTIME} structure consisting in 16-bit fields.
How our compiler know to pack them on 1-byte alignment boundary?}

\RU{В файле \TT{WinNT.h} попадается такое:}\EN{\TT{WinNT.h} file has this:}

\begin{lstlisting}[caption=WinNT.h]
#include "pshpack1.h"
\end{lstlisting}

\RU{И такое:}\EN{And this:}

\begin{lstlisting}[caption=WinNT.h]
#include "pshpack4.h"                   // 4 byte packing is the default
\end{lstlisting}

\RU{Сам файл PshPack1.h выглядит так:}\EN{The file PshPack1.h looks like:}

\begin{lstlisting}[caption=PshPack1.h]
#if ! (defined(lint) || defined(RC_INVOKED))
#if ( _MSC_VER >= 800 && !defined(_M_I86)) || defined(_PUSHPOP_SUPPORTED)
#pragma warning(disable:4103)
#if !(defined( MIDL_PASS )) || defined( __midl )
#pragma pack(push,1)
#else
#pragma pack(1)
#endif
#else
#pragma pack(1)
#endif
#endif /* ! (defined(lint) || defined(RC_INVOKED)) */
\end{lstlisting}

\RU{Собственно, так и задается компилятору, как паковать объявленные после \TT{\#pragma pack} структуры.}
\EN{That's how compiler will pack structures defined after \TT{\#pragma pack}.}

\subsection{ARM + \OptimizingKeil + \ThumbMode}

\lstinputlisting[caption=\OptimizingKeil + \ThumbMode]{patterns/15_structs/4_packing/packing_Keil_thumb.asm}

\RU{Как мы помним, здесь передается не указатель на структуру, а сама структура, а так как в ARM первые 4 аргумента
функции передаются через регистры, то поля структуры передаются через}
\EN{As we may recall, here a structure passed instead of pointer to structure,
and since first 4 function arguments in ARM are passed via registers,
so then structure fields are passed via} \TT{R0-R3}.

\index{ARM!\Instructions!LDRB}
\index{x86!\Instructions!MOVSX}
\RU{Инструкция }\TT{LDRB} \RU{загружает один байт из памяти и расширяет до 32-бит учитывая знак.}
\EN{loads one byte from memory and extending it to 32-bit, taking into account its sign.}
\RU{Это то же что и инструкция}\EN{This is akin to} \MOVSX \RU{в}\EN{instruction in} x86.
\RU{Она здесь применяется для загрузки полей}\EN{Here it is used for loading fields} $a$ \AndENRU $c$ 
\RU{из структуры}\EN{from structure}.

\index{Function epilogue}
\RU{Еще что бросается в глаза, так это то что вместо эпилога функции, переход на эпилог другой функции!}
\EN{One more thing we spot easily, instead of function epilogue, here is jump to another function's epilogue!}
\RU{Действительно, то была совсем другая, не относящаяся к этой, функция, однако, она имела точно такой же
эпилог}\EN{Indeed, that was quite different function, not related in any way to our function, however, it has exactly
the same epilogue} 
(\RU{видимо, тоже хранила в стеке 5 локальных переменных}\EN{probably because, it hold 5 local variables too} 
($5*4=0x14$)).
\RU{К тому же, она находится рядом (обратите внимание на адреса).}
\EN{Also it is located nearly (take a look on addresses).}
\RU{Действительно, нет никакой разницы, какой эпилог исполнять, если он работает так же, как нам нужно.}
\EN{Indeed, there is no difference, which epilogue to execute,
if it works just as we need.}
\RU{Keil решил использовать часть другой ф-ции, вероятно, из-за экономии.}
\EN{Apparently, Keil decides to reuse a part of another function by a reason of economy.}
\RU{Эпилог занимает 4 байта, а переход ~--- только 2.}
\EN{Epilogue takes 4 bytes while jump~---only 2.}

\subsection{ARM + \OptimizingXcode + \ThumbTwoMode}

\lstinputlisting[caption=\OptimizingXcode + \ThumbTwoMode]{patterns/15_structs/4_packing/packing_Xcode_thumb.asm}

\index{ARM!\Instructions!SXTB}
\index{x86!\Instructions!MOVSX}
\TT{SXTB} (\IT{Signed Extend Byte}) \RU{это также аналог}\EN{is analogous to} \MOVSX \InENRU 
x86\RU{, только работает не с памятью, а с регистром.}\EN{ as well, but works not with memory, but with register.}
\RU{Всё остальное ~--- так же.}\EN{All the rest~---just the same.}

