\subsubsection{Пример CPUID}

Язык \CCpp позволяет указывать, сколько именно бит отвести для каждого поля структуры. 
Это удобно если нужно экономить место в памяти. К примеру, для переменной типа \Tbool достаточно одного бита.
Но, это не очень удобно, если нужна скорость.

% FIXME!
% another use of this is to parse binary protocols/packets, for example
% the definition of struct iphdr in include/linux/ip.h

\newcommand{\FNCPUID}{\footnote{\href{http://go.yurichev.com/17069}{wikipedia}}}

\myindex{x86!\Instructions!CPUID}
\label{cpuid}
Рассмотрим пример с инструкцией \CPUID\FNCPUID. 
Эта инструкция возвращает информацию о том, какой процессор имеется в наличии и какие возможности он имеет.

Если перед исполнением инструкции в \EAX будет 1, 
то \CPUID вернет упакованную в \EAX такую информацию о процессоре:

\begin{center}
\begin{tabular}{ | l | l | }
\hline
3:0 (4 бита)& Stepping \\
7:4 (4 бита) & Model \\
11:8 (4 бита) & Family \\
13:12 (2 бита) & Processor Type \\
19:16 (4 бита) & Extended Model \\
27:20 (8 бит) & Extended Family \\
\hline
\end{tabular}
\end{center}

\newcommand{\FNGCCAS}{\footnote{\href{http://go.yurichev.com/17070}
{Подробнее о встроенном ассемблере GCC}}}

MSVC 2010 имеет макрос для \CPUID, а GCC 4.4.1 ~--- нет. 
Поэтому для GCC сделаем эту функцию сами, используя его встроенный ассемблер\FNGCCAS.

\lstinputlisting{patterns/15_structs/6_bitfields/cpuid/CPUID.c}

После того как \CPUID заполнит \EAX/\EBX/\ECX/\EDX, у нас они отразятся в массиве \TT{b[]}. 
Затем, мы имеем указатель на структуру \TT{CPUID\_1\_EAX}, и мы указываем его на значение 
\EAX из массива \TT{b[]}.

Иными словами, мы трактуем 32-битный \Tint как структуру.

Затем мы читаем отдельные биты из структуры.

\myparagraph{MSVC}

Компилируем в MSVC 2008 с опцией \Ox:

\lstinputlisting[caption=\Optimizing MSVC 2008]{patterns/15_structs/6_bitfields/cpuid/CPUID_msvc_Ox.asm}

\myindex{x86!\Instructions!SHR}
Инструкция \TT{SHR} сдвигает значение из \EAX на то количество бит, 
которое нужно \IT{пропустить}, то есть, мы игнорируем некоторые биты \IT{справа}.

\myindex{x86!\Instructions!AND}
А инструкция \AND очищает биты \IT{слева} которые нам не нужны, или же, говоря иначе, 
она оставляет по маске только те биты в \EAX, которые нам сейчас нужны.

\clearpage
\subsubsectionold{MSVC + \olly}
\myindex{\olly}

Загрузим пример в \olly и увидим, какие значения были установлены в EAX/EBX/ECX/EDX после исполнения CPUID: 

\begin{figure}[H]
\centering
\myincludegraphics{patterns/15_structs/6_bitfields/cpuid/olly.png}
\caption{\olly: После исполнения CPUID}
\label{fig:cpuid_olly_1}
\end{figure}

В EAX установлено \TT{0x000206A7} (мой \ac{CPU} --- Intel Xeon E3-1220).\\
В двоичном виде это $0000 0000 0000 0010 0000 0110 1010 0111$.

Вот как распределяются биты по полям в моем случае:

\begin{center}
\begin{tabular}{ | l | l | l | }
\hline
\headercolor{} поле &
\headercolor{} в двоичном виде &
\headercolor{} в десятичном виде \\
\hline
reserved2		& 0000 & 0 \\
\hline
extended\_family\_id	& 00000000 & 0 \\
\hline
extended\_model\_id	& 0010 & 2 \\
\hline
reserved1		& 00 & 0 \\
\hline
processor\_id		& 00 & 0 \\
\hline
family\_id		& 0110 & 6 \\
\hline
model			& 1010 & 10 \\
\hline
stepping		& 0111 & 7 \\
\hline
\end{tabular}
\end{center}

\begin{figure}[H]
\centering
\myincludegraphics{patterns/15_structs/6_bitfields/cpuid/result.png}
\caption{\olly: Результат работы}
\label{fig:cpuid_olly_2}
\end{figure}


\myparagraph{GCC}

Попробуем GCC 4.4.1 с опцией \Othree.

\lstinputlisting[caption=\Optimizing GCC 4.4.1]{patterns/15_structs/6_bitfields/cpuid/CPUID_gcc_O3.asm}

Практически, то же самое. Единственное что стоит отметить это то, что GCC решил зачем-то объединить 
вычисление \\
\TT{extended\_model\_id} и \TT{extended\_family\_id} в один блок, 
вместо того чтобы вычислять их перед соответствующим вызовом \printf.

