\subsection{\WorkingWithFloatAsWithStructSubSubSectionName}
\label{sec:floatasstruct}

\RU{Как уже раннее указывалось в секции о FPU~(\ref{sec:FPU}), 
и \Tfloat и \Tdouble содержат в себе \IT{знак}, \IT{мантиссу} и \IT{экспоненту}. 
Однако, можем ли мы работать с этими полями напрямую? Попробуем с \Tfloat.}
\EN{As we already noted in the section about FPU~(\ref{sec:FPU}), 
both \Tfloat and \Tdouble types consist of a \IT{sign}, 
a \IT{significand} (or \IT{fraction}) and an \IT{exponent}.
But will we be able to work with these fields directly? Let's try this with \Tfloat.}

\bigskip
% a hack used here! http://tex.stackexchange.com/questions/73524/bytefield-package
\begin{center}
\begin{bytefield}{32}
	\bitheader[endianness=big]{0,22,23,30,31} \\
	\bitbox{1}{S} & 
	\bitbox{8}{%
		\RU{экспонента}%
		\EN{exponent}%
		\ES{exponente}%
		\PTBRph{}%
		\DEph{}\PLph{}%
		\ITAph{}%
		\FR{exposant}
	} & 
	\bitbox{23}{%
		\RU{мантисса}%
		\EN{mantissa or fraction}%
		\ES{mantisa o fracci\'on}%
		\PTBRph{}%
		\DEph{}\PLph{}%
		\ITAph{}%
		\FR{mantisse ou fraction}
	}
\end{bytefield}
\end{center}

\begin{center}
( S\EMDASH{}%
	\RU{знак}%
	\EN{sign}%
	\ES{signo}%
	\PTBRph{}%
	\DEph{}\PLph{}%
	\ITAph{}%
	\FR{signe}
)
\end{center}


\lstinputlisting{patterns/15_structs/6_bitfields/float/float.c.\LANG}

\RU{Структура \TT{float\_as\_struct} занимает в памяти столько же места сколько и \Tfloat, 
то есть 4 байта или 32 бита.}
\EN{The \TT{float\_as\_struct} structure occupies the same amount of  memory as \Tfloat, i.e., 4 bytes or 32 bits.}

\RU{Далее мы выставляем во входящем значении отрицательный знак, 
а также прибавляя двойку к экспоненте, мы тем 
самым умножаем всё значение на \TT{$2^2$}, то есть на 4.}
\EN{Now we are setting the negative sign in the input value and also, by adding 2 to the exponent, 
we thereby multiply the whole number by \TT{$2^2$}, i.e., by 4.}

\RU{Компилируем в MSVC 2008 без включенной оптимизации:}
\EN{Let's compile in MSVC 2008 without optimization turned on:}

\lstinputlisting[caption=\NonOptimizing MSVC 2008]{patterns/15_structs/6_bitfields/float/float_msvc.asm.\LANG}

\RU{Слегка избыточно. В версии скомпилированной с флагом \Ox нет вызовов \TT{memcpy()}, 
там работа происходит сразу с переменной f. Но по неоптимизированной версии будет проще понять.}
\EN{A but redundant.
If it was compiled with \Ox flag there would be no \TT{memcpy()} call,
the \ttf variable will be used directly.
But it is easier to understand by looking at the unoptimized version.}

\RU{А что сделает GCC 4.4.1 с опцией \Othree?}\EN{What would GCC 4.4.1 with \Othree do?}

\lstinputlisting[caption=\Optimizing GCC 4.4.1]{patterns/15_structs/6_bitfields/float/float_gcc_O3.asm.\LANG}

\RU{Да, функция \ttf в целом понятна. Однако, что интересно, еще при компиляции, 
не взирая на мешанину с полями структуры, GCC умудрился вычислить результат функции \TT{f(1.234)} еще
во время компиляции и сразу подставить его в аргумент для \printf{}!}
\EN{The \ttf function is almost understandable. However, what is interesting is that GCC was able to calculate
the result of \TT{f(1.234)} during compilation despite all this hodge-podge with the structure fields
and prepared this argument to \printf{} as precalculated at compile time!}
