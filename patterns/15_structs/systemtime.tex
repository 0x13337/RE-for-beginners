\section{\RU{Пример SYSTEMTIME}\EN{SYSTEMTIME example}}

\newcommand{\FNSYSTEMTIME}{\footnote{\href{http://msdn.microsoft.com/en-us/library/ms724950(VS.85).aspx}{MSDN: SYSTEMTIME structure}}}

\RU{Возьмем, к примеру, структуру SYSTEMTIME\FNSYSTEMTIME{} из win32 описывающую время.}
\EN{Let's take SYSTEMTIME\FNSYSTEMTIME{} win32 structure describing time.}

\RU{Она объявлена так:}\EN{That's how it is defined:}

\begin{lstlisting}[caption=WinBase.h]
typedef struct _SYSTEMTIME {
  WORD wYear;
  WORD wMonth;
  WORD wDayOfWeek;
  WORD wDay;
  WORD wHour;
  WORD wMinute;
  WORD wSecond;
  WORD wMilliseconds;
} SYSTEMTIME, *PSYSTEMTIME;
\end{lstlisting}

\RU{Пишем на Си функцию для получения текущего системного времени:}
\EN{Let's write a C function to get current time:}

\lstinputlisting{patterns/15_structs/systemtime.c}

\RU{Что в итоге}\EN{We got} (MSVC 2010):

\lstinputlisting[caption=MSVC 2010]{patterns/15_structs/systemtime.asm}

\RU{Под структуру в стеке выделено 16 байт ~--- именно столько будет \TT{sizeof(WORD)*8}
(в структуре 8 переменных с типом WORD).}
\EN{16 bytes are allocated for this structure in local stack~---that is exactly \TT{sizeof(WORD)*8}
(there are 8 WORD variables in the structure).}

\newcommand{\FNMSDNGST}{\footnote{\href{http://msdn.microsoft.com/en-us/library/ms724390(VS.85).aspx}{MSDN: GetSystemTime function}}}

\RU{Обратите внимание на тот факт, что структура начинается с поля \TT{wYear}. 
Можно сказать, что в качестве аргумента для \TT{GetSystemTime()}\FNMSDNGST передается указатель на структуру 
SYSTEMTIME, но можно также сказать, что передается указатель на поле \TT{wYear}, 
что одно и тоже! 
\TT{GetSystemTime()} пишет текущий год в тот WORD на который указывает переданный указатель, 
затем сдвигается на 2 байта вправо, пишет текущий месяц, и т.д., и т.д.}
\EN{Pay attention to the fact the structure beginning with \TT{wYear} field.
It can be said, an pointer to SYSTEMTIME structure is passed to the \TT{GetSystemTime()}\FNSYSTEMTIME,
but it is also can be said, pointer to the \TT{wYear} field is passed, and that is the same!
\TT{GetSystemTime()} writes current year to the WORD pointer pointing to, then shifts 2 bytes
ahead, then writes current month, etc, etc.}

\RU{Тот факт, что поля структуры это просто переменные расположенные рядом, 
я могу проиллюстрировать следующим образом.}
\EN{The fact the structure fields are just variables located side-by-side, 
I can demonstrate by the following technique.}
\RU{Глядя на описание структуры}\EN{Keeping in ming} \TT{SYSTEMTIME}\RU{, я могу переписать этот простой пример так:}
\EN{ structure description, I can rewrite this simple example like this:}

\lstinputlisting{patterns/15_structs/systemtime2.c}

\RU{Компилятор немного поворчит:}\EN{Compiler will grumble for a little:}

\begin{lstlisting}
systemtime2.c(7) : warning C4133: 'function' : incompatible types - from 'WORD [8]' to 'LPSYSTEMTIME'
\end{lstlisting}

\RU{Тем не менее, выдаст такой код}\EN{But nevertheless, it will produce this code}:

\lstinputlisting[caption=MSVC 2010]{patterns/15_structs/systemtime2.asm}

\RU{И это работает так же}\EN{And it works just as the same}!

\RU{Любопытно что результат на ассемблере неотличим от предыдущего}
\EN{It is very interesting fact the
result in assembly form cannot be distinguished from the result of previous compilation}.
\RU{Таким образом, глядя на этот код, 
никогда нельзя сказать с уверенностью, была ли там объявлена структура, либо просто набор переменных.}
\EN{So by looking at this code, one cannot say for sure, was there structure declared, or just pack of variables.} 

\RU{Тем не менее, никто в здравом уме делать так не будет}\EN{Nevertheless, no one will do it in sane state of mind}.
\RU{Потому что это неудобно}\EN{Since it is not convenient}. 
\RU{К тому же, иногда, поля в структуре могут меняться разработчиками, 
переставляться местами, и т.д}\EN{Also structure fields may be changed by developers, swapped, etc}.

