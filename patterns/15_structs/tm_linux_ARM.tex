\section{struct tm}

\subsection{Linux}

\RU{В Линуксе, для примера, возьмем структуру \TT{tm} из \TT{time.h}:}
\EN{As of Linux, let's take \TT{tm} structure from \TT{time.h} for example:}

\lstinputlisting{patterns/15_structs/GCC_tm.c}

\RU{Компилируем при помощи}\EN{Let's compile it in} GCC 4.4.1:

\lstinputlisting[caption=GCC 4.4.1]{patterns/15_structs/GCC_tm_\LANG.asm}

\RU{К сожалению, по какой-то причине, \IDA не сформировала названия локальных переменных в стеке. 
Но так как мы уже опытные реверсеры :-) то можем обойтись и без этого в таком простом примере.}
\EN{Somehow, \IDA did not created local variables names in local stack.
But since we already experienced reverse engineers :-) we may do it without this information in 
this simple example.}

\index{x86!\Instructions!LEA}
\RU{Обратите внимание на \TT{lea edx, [eax+76Ch]} ~--- эта инструкция прибавляет \TT{0x76C} к \EAX, 
но не модифицирует флаги. См. также соответствующий раздел об инструкции \LEA{}~(\ref{sec:LEA}).}
\EN{Please also pay attention to the \TT{lea edx, [eax+76Ch]}~---this instruction just adding \TT{0x76C} to value in the \EAX,
but not modifies any flags. See also relevant section about \LEA{}~(\ref{sec:LEA}).}

\RU{Чтобы проиллюстрировать то что структура ~--- это просто набор переменных лежащих в одном месте, переделаем немного
пример, заглянув предварительно в файл}
\EN{In order to illustrate the structure is just variables laying side-by-side in one place, let's rework
example, while looking at the file} \IT{time.h}:

\begin{lstlisting}[caption=time.h]
struct tm
{
  int	tm_sec;
  int	tm_min;
  int	tm_hour;
  int	tm_mday;
  int	tm_mon;
  int	tm_year;
  int	tm_wday;
  int	tm_yday;
  int	tm_isdst;
};
\end{lstlisting}

\lstinputlisting{patterns/15_structs/GCC_tm2.c}

N.B. \RU{В \TT{localtime\_r} передается указатель именно на \TT{tm\_sec}, 
т.е., на первый элемент ``структуры''.}
\EN{The pointer to the exactly \TT{tm\_sec} field is passed into \TT{localtime\_r}, i.e., 
to the first ``structure'' element.}

\RU{В итоге, и этот компилятор поворчит}\EN{Compiler will warn us}:

\begin{lstlisting}[caption=GCC 4.7.3]
GCC_tm2.c: In function 'main':
GCC_tm2.c:11:5: warning: passing argument 2 of 'localtime_r' from incompatible pointer type [enabled by default]
In file included from GCC_tm2.c:2:0:
/usr/include/time.h:59:12: note: expected 'struct tm *' but argument is of type 'int *'
\end{lstlisting}

\RU{Тем не менее, сгенерирует такое}\EN{But nevertheless, will generate this}:

\lstinputlisting[caption=GCC 4.7.3]{patterns/15_structs/GCC_tm2.asm}

\RU{Этот код почти идентичен уже рассмотренному, и нельзя сказать, была ли структура
в оригинальном исходном коде либо набор переменных.}\EN{This code is identical to what we saw previously and it is
not possible to say, was it structure in original source code or just pack of variables.}

\RU{И это работает}\EN{And this works}. 
\RU{Однако, в реальности так лучше не делать}\EN{However, it is not recommended to do this in practice}. 
\RU{Обычно, компилятор располагает переменные в локальном
стеке в том же порядке, в котором они объявляются в функции}
\EN{Usually, compiler allocated variables in local stack in the same order as they were declared in function}.
\RU{Тем не менее, никакой гарантии нет}
\EN{Nevertheless, there is no any guarantee}.

\RU{Кстати, какой-нибудь другой компилятор может предупредить, что переменные}
\EN{By the way, some other compiler may warn the} \TT{tm\_year}, \TT{tm\_mon}, \TT{tm\_mday},
\TT{tm\_hour}, \TT{tm\_min}\RU{, но не}\EN{ variables, but not} \TT{tm\_sec}\RU{, используются без инициализации}
\EN{ are used without being initialized}.
\RU{Действительно, ведь компилятор не знает что они будут заполнены при вызове функции}
\EN{Indeed, compiler do not know these will be filled when calling to} \TT{localtime\_r()}.

\RU{Я выбрал именно этот пример для иллюстрации, потому что все члены структуры имеют тип \Tint, а члены структуры
\TT{SYSTEMTIME} ~--- 16-битные \TT{WORD}, и если их объявлять так же, как локальные переменные, 
то они будут выровнены по 32-битной границе 
и ничего не выйдет (потому что \TT{GetSystemTime()} заполнит их неверно).}
\EN{I chose exactly this example for illustration, since all structure fields has \Tint type, 
and \TT{SYSTEMTIME} structure
fields~---16-bit \TT{WORD}, and if to declare them as a local variables, they will be aligned on a 32-bit border,
and nothing will work (because \TT{GetSystemTime()} will fill them incorrectly).}
\RU{Читайте об этом в следующей секции}\EN{Read more about it in next section}: ``\StructurePackingSectionName''.

\index{\SyntacticSugar}
\RU{Так что, структура ~--- это просто набор переменных лежащих в одном месте, рядом.}
\EN{So, structure is just variables pack laying on one place, side-by-side.}
\RU{Я мог бы сказать что структура ~--- это такой синтаксический сахар, 
заставляющий компилятор удерживать их в одном месте.}
\EN{I could say the structure is a syntactic sugar, directing
compiler to hold them in one place.}
\RU{Впрочем, я не специалист по языкам программирования, так что, скорее всего, ошибаюсь с этим термином.}
\EN{However, I'm not programming languages expert, so, most likely, I'm wrong with this term.}
\RU{Кстати, когда-то, в очень ранних версиях Си (перед 1972) структур не было вовсе}
\EN{By the way, there were a times, in very early C versions (before 1972), 
in which there were no structures at all}\cite{Ritchie:1993:DCL:155360.155580}.

\subsection{ARM + \OptimizingKeil + \ThumbMode}

\RU{Этот же пример}\EN{Same example}:

\lstinputlisting[caption=\OptimizingKeil + \ThumbMode]{patterns/15_structs/tm_ARM_keil_thumb.asm}

\subsection{ARM + \OptimizingXcode + \ThumbTwoMode}

\IDA ``\RU{узнала}\EN{get to know}'' \RU{структуру }\TT{tm}\EN{ structure} 
(\RU{потому что}\EN{because} \IDA ``\RU{знает}\EN{knows}'' \RU{типы аргументов библиотечных функций, 
таких как}\EN{argument types of library functions like} \TT{localtime\_r()}), 
\RU{поэтому показала здесь обращения к отдельным элементам структуры и присвоила им имена}
\EN{so it shows here structure elements accesses and also names are assigned to them}.

\lstinputlisting[caption=\OptimizingXcode + \ThumbTwoMode]{patterns/15_structs/tm_ARM_xcode_thumb.asm}

