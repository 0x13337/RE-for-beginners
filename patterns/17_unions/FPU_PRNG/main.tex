\section{\RU{Пример генератора случайных чисел}\EN{Pseudo-random number generator example}}
\label{FPU_PRNG}

\RU{Если нам нужны случайные значения с плавающей запятой в интервале от 0 до 1, самое простое это взять
\ac{PRNG} вроде Mersenne twister.
Он выдает случайные 32-битные числа в виде DWORD.
Затем мы можем преобразовать это число в \Tfloat и затем разделить на \TT{RAND\_MAX} (\TT{0xFFFFFFFF} в данном случае) ~--- 
полученное число будет в интервале от 0 до 1.}
\EN{If we need float random numbers from 0 to 1, the most simplest thing is to use \ac{PRNG} like
Mersenne twister. 
It produces random 32-bit values in DWORD form. 
Then we can transform this value to \Tfloat and then
dividing it by \TT{RAND\_MAX} (\TT{0xFFFFFFFF} in our case)~---value we 
got will be in 0..1 interval.}

\RU{Но как известно, операция деления ~--- это медленная операция. 
Да и вообще хочется избежать лишних операций с FPU.
Сможем ли мы избежать деления?}
\EN{But as we know, division operation is slow.
Also, we would like to issue as small number of FPU operations as possible.
Will it be possible to get rid of division?}

\index{IEEE 754}
\RU{Вспомним состав числа с плавающей запятой: это бит знака, биты мантиссы и биты экспоненты. 
Для получения случайного числа, нам нужно просто заполнить случайными битами все биты мантиссы!}
\EN{Let's recall what float number consisted of: sign bit, significand bits and exponent bits.
We need just to store random bits to all significand bits for getting random float number!}

\RU{Экспонента не может быть нулевой (иначе число будет денормализованным), 
так что в эти биты мы запишем \TT{01111111} ~--- 
это будет означать что экспонента равна единице. Далее заполняем мантиссу случайными битами, 
знак оставляем в виде 0 (что значит наше число положительное), и вуаля. 
Генерируемые числа будут в интервале от 1 до 2, так что нам еще нужно будет отнять единицу.}
\EN{Exponent cannot be zero (number is denormalized in this case), so we will store \TT{01111111} 
to exponent~---this means exponent is 1. Then fill significand with random bits, set sign bit to
0 (which means positive number) and voilà.
Generated numbers will be in 1 to 2 interval, so we also must subtract 1 from it.}

\newcommand{\URLXOR}{\url{http://go.yurichev.com/17308}}

\RU{В моем примере\footnote{идея взята здесь: \URLXOR} 
применяется очень простой линейный конгруэнтный генератор случайных чисел, выдающий 32-битные числа.
Генератор инициализируется текущим временем в стиле UNIX.}
\EN{Very simple linear congruential random numbers generator is used in my 
example\footnote{idea was taken from: \URLXOR}, produces 32-bit numbers. 
The \ac{PRNG} initializing by current time in UNIX-style.}

\RU{Далее, тип \Tfloat представляется в виде \IT{union} ~--- это конструкция \CCpp позволяющая 
интерпретировать часть памяти по-разному. В нашем случае, мы можем создать переменную типа \TT{union} 
и затем обращаться к ней как к \Tfloat или как к \IT{uint32\_t}. Можно сказать, что это хак, причем грязный.}
\EN{Then, \Tfloat type represented as \IT{union}~---it is the \CCpp construction enabling us
to interpret piece of memory as differently typed.
In our case, we are able to create a variable
of \TT{union} type and then access to it as it is \Tfloat or as it is \IT{uint32\_t}. 
It can be said, it is just a hack. A dirty one.}

\RU{Код целочисленного \ac{PRNG} точно такой же, как мы уже рассматривали раннее:}
\EN{The integer \ac{PRNG} code is the same as we already considered:} \ref{LCG_simple}.
\RU{Так что и в скомпилированном виде этот код будет опущен.}
\EN{So this code will be omitted in compiled form.}

\lstinputlisting{patterns/17_unions/FPU_PRNG/FPU_PRNG.cpp.\LANG}

\subsection{x86}

\lstinputlisting[caption=\Optimizing MSVC 2010]{patterns/17_unions/FPU_PRNG/MSVC2010_Ox_Ob0.asm.\LANG}

\EN{Function names are so strange here because I compiled this example as C++ and this is name mangling in C++,
we will talk about it later:}
\RU{Имена ф-ций такие странные, потому что я компилировал этот пример как Си++, и это манглинг имен в Си++, 
мы будем рассматривать это позже:} \ref{namemangling}.

\RU{Если скомпилировать это в MSVC 2012, компилятор будет использовать SIMD-инструкции для FPU, читайте об этом
здесь:}
\EN{If to compile this in MSVC 2012, it will use SIMD instructions for FPU, read more about it here:}
\ref{FPU_PRNG_SIMD}.

\subsection{MIPS}

\lstinputlisting[caption=\Optimizing GCC 4.4.5]{patterns/17_unions/FPU_PRNG/MIPS_O3_IDA.lst.\LANG}

\EN{There are also added useless LUI instruction for some weird reason.}
\RU{Здесь снова зачем-то добавлена инструкция LUI, которая ничего не делает.}
\EN{We considered this artifact earlier:}
\RU{Мы уже рассматривали этот артефакт раннее:} \ref{MIPS_FPU_LUI}.

\subsection{ARM (\ARMMode)}

\lstinputlisting[caption=\Optimizing GCC 4.6.3 (IDA)]{patterns/17_unions/FPU_PRNG/raspberry_GCC_O3_IDA.lst.\LANG}

\index{objdump}
\index{binutils}
\index{IDA}
\RU{Я также сделал дамп в objdump и увидел что FPU-инструкции имеют немного другие имена чем в IDA.}
\EN{I also made a dump in objdump and I saw that FPU-instructions has other names than in IDA.}
\EN{Apparently, IDA and binutils developers used different manuals?}
\RU{Наверное, разработчики IDA и binutils пользовались разной документацией?}
\EN{I suppose, it would be good to know both instruction name variants.}
\RU{Полагаю, будет полезно знать оба варианта названий инструкций.}

\lstinputlisting[caption=\Optimizing GCC 4.6.3 (objdump)]{patterns/17_unions/FPU_PRNG/raspberry_GCC_O3_objdump.lst}

\EN{Instructions at 5c in float\_rand() and at 38 in main() is random noise.}
\RU{Инструкции по адресам 5c в float\_rand() и 38 в main() это случайный мусор.}
