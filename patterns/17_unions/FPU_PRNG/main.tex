\section{\RU{Пример генератора случайных чисел}\EN{Pseudo-random number generator example}}
\label{FPU_PRNG}

\RU{Если нам нужны случайные значения с плавающей запятой в интервале от 0 до 1, самое простое это взять
\ac{PRNG} вроде Mersenne twister.
Он выдает случайные 32-битные числа в виде DWORD.
Затем мы можем преобразовать это число в \Tfloat и затем разделить на \GTT{RAND\_MAX} (\GTT{0xFFFFFFFF} в данном случае)\EMDASH{}
полученное число будет в интервале от 0 до 1.}
\EN{If we need float random numbers between 0 and 1, the simplest thing is to use a \ac{PRNG} like
the Mersenne twister. 
It produces random 32-bit values in DWORD form. 
Then we can transform this value to \Tfloat and then
divide it by \GTT{RAND\_MAX} (\GTT{0xFFFFFFFF} in our case)\EMDASH{}
we getting a value in the 0..1 interval.}

\RU{Но как известно, операция деления\EMDASH{}это медленная операция. 
Да и вообще хочется избежать лишних операций с FPU.
Сможем ли мы избежать деления?}
\EN{But as we know, division is slow.
Also, we would like to issue as few FPU operations as possible.
Can we get rid of the division?}

\index{IEEE 754}
\RU{Вспомним состав числа с плавающей запятой: это бит знака, биты мантиссы и биты экспоненты. 
Для получения случайного числа, нам нужно просто заполнить случайными битами все биты мантиссы!}
\EN{Let's recall what a floating point number consists of: sign bit, significand bits and exponent bits.
We just need to store random bits in all significand bits to get a random float number!}

\RU{Экспонента не может быть нулевой (иначе число с плавающей точкой будет денормализованным), 
так что в эти биты мы запишем \GTT{01111111}\EMDASH{}
это будет означать что экспонента равна единице. Далее заполняем мантиссу случайными битами, 
знак оставляем в виде 0 (что значит наше число положительное), и вуаля. 
Генерируемые числа будут в интервале от 1 до 2, так что нам еще нужно будет отнять единицу.}
\EN{The exponent cannot be zero (the floating number is denormalized in this case), so we are storing \GTT{01111111} 
to exponent\EMDASH{}this means that the exponent is 1. 
Then we filling the significand with random bits, set the sign bit to
0 (which means a positive number) and voilà.
The generated numbers is to be between 1 and 2, so we must also subtract 1.}

\newcommand{\URLXOR}{\url{http://go.yurichev.com/17308}}

\RU{В моем примере\footnote{идея взята здесь: \URLXOR} 
применяется очень простой линейный конгруэнтный генератор случайных чисел, выдающий 32-битные числа.
Генератор инициализируется текущим временем в стиле UNIX.}
\EN{A very simple linear congruential random numbers generator is used in my 
example\footnote{the idea was taken from: \URLXOR}, it produces 32-bit numbers. 
The \ac{PRNG} is initialized with the current time in UNIX timestamp format.}

\RU{Далее, тип \Tfloat представляется в виде \IT{union}\EMDASH{}это конструкция \CCpp позволяющая 
интерпретировать часть памяти по-разному. В нашем случае, мы можем создать переменную типа \IT{union} 
и затем обращаться к ней как к \Tfloat или как к \IT{uint32\_t}. Можно сказать, что это хак, причем грязный.}
\EN{Here we represent the \Tfloat type as an \IT{union}\EMDASH{}it is the \CCpp construction that enables us
to interpret a piece of memory as different types.
In our case, we are able to create a variable
of type \IT{union} and then access to it as it is \Tfloat or as it is \IT{uint32\_t}. 
It can be said, it is just a hack. A dirty one.}

% WTF?
\RU{Код целочисленного \ac{PRNG} точно такой же, как мы уже рассматривали ранее:}
\EN{The integer \ac{PRNG} code is the same as we already considered:} \myref{LCG_simple}.
\RU{Так что и в скомпилированном виде этот код будет опущен.}
\EN{So this code in compiled form is omitted.}

\lstinputlisting{patterns/17_unions/FPU_PRNG/FPU_PRNG.cpp.\LANG}

\subsection{x86}

\lstinputlisting[caption=\Optimizing MSVC 2010]{patterns/17_unions/FPU_PRNG/MSVC2010_Ox_Ob0.asm.\LANG}

\EN{Function names are so strange here because this example was compiled as C++ and this is name mangling in C++,
we will talk about it later:}%
\RU{Имена функций такие странные, потому что этот пример был скомпилирован как Си++, и это манглинг имен в Си++, 
мы будем рассматривать это позже:} \myref{namemangling}.

\RU{Если скомпилировать это в MSVC 2012, компилятор будет использовать SIMD-инструкции для FPU, читайте об этом
здесь:}
\EN{If we compile this in MSVC 2012, it uses the SIMD instructions for the FPU, read more about it here:}
\myref{FPU_PRNG_SIMD}.

\subsection{MIPS}

\lstinputlisting[caption=\Optimizing GCC 4.4.5]{patterns/17_unions/FPU_PRNG/MIPS_O3_IDA.lst.\LANG}

\EN{There is also an useless LUI instruction added for some weird reason.}
\RU{Здесь снова зачем-то добавлена инструкция LUI, которая ничего не делает.}
\EN{We considered this artifact earlier:}
\RU{Мы уже рассматривали этот артефакт ранее:} \myref{MIPS_FPU_LUI}.

\subsection{ARM (\ARMMode)}

\lstinputlisting[caption=\Optimizing GCC 4.6.3 (IDA)]{patterns/17_unions/FPU_PRNG/raspberry_GCC_O3_IDA.lst.\LANG}

\index{objdump}
\index{binutils}
\index{IDA}
\RU{Мы также сделаем дамп в objdump и увидим что FPU-инструкции имеют немного другие имена чем в \IDA.}%
\EN{We'll also make a dump in objdump and we'll see that the FPU instructions have different names than in \IDA.}
\EN{Apparently, IDA and binutils developers used different manuals?}
\RU{Наверное, разработчики IDA и binutils пользовались разной документацией?}
\EN{Perhaps it would be good to know both instruction name variants.}
\RU{Должно быть, будет полезно знать оба варианта названий инструкций.}

\lstinputlisting[caption=\Optimizing GCC 4.6.3 (objdump)]{patterns/17_unions/FPU_PRNG/raspberry_GCC_O3_objdump.lst}

\EN{The instructions at 0x5c in float\_rand() and at 0x38 in main() are random noise.}
\RU{Инструкции по адресам 0x5c в float\_rand() и 0x38 в main() это случайный мусор.}
