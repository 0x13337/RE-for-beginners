\section{\RU{Пример генератора случайных чисел}\EN{Pseudo-random number generator example}}

\RU{Если нам нужны случайные значения с плавающей запятой в интервале от 0 до 1, самое простое это взять
\ac{PRNG} вроде Mersenne twister.
Он выдает случайные 32-битные числа в виде DWORD.
Затем мы можем преобразовать это число в \Tfloat и затем разделить на \TT{RAND\_MAX} (\TT{0xFFFFFFFF} в данном случае) ~--- 
полученное число будет в интервале от 0 до 1.}
\EN{If we need float random numbers from 0 to 1, the most simplest thing is to use \ac{PRNG} like
Mersenne twister. 
It produces random 32-bit values in DWORD form. 
Then we can transform this value to \Tfloat and then
dividing it by \TT{RAND\_MAX} (\TT{0xFFFFFFFF} in our case)~---value we 
got will be in 0..1 interval.}

\RU{Но как известно, операция деления ~--- это медленная операция. 
Да и вообще хочется избежать лишних операций с FPU.
Сможем ли мы избежать деления?}
\EN{But as we know, division operation is slow.
Also, we would like to issue as small number of FPU operations as possible.
Will it be possible to get rid of division?}

\index{IEEE 754}
\RU{Вспомним состав числа с плавающей запятой: это бит знака, биты мантиссы и биты экспоненты. 
Для получения случайного числа, нам нужно просто заполнить случайными битами все биты мантиссы!}
\EN{Let's recall what float number consisted of: sign bit, significand bits and exponent bits.
We need just to store random bits to all significand bits for getting random float number!}

\RU{Экспонента не может быть нулевой (иначе число будет денормализованным), 
так что в эти биты мы запишем \TT{01111111} ~--- 
это будет означать что экспонента равна единице. Далее заполняем мантиссу случайными битами, 
знак оставляем в виде 0 (что значит наше число положительное), и вуаля. 
Генерируемые числа будут в интервале от 1 до 2, так что нам еще нужно будет отнять единицу.}
\EN{Exponent cannot be zero (number is denormalized in this case), so we will store \TT{01111111} 
to exponent~---this means exponent is 1. Then fill significand with random bits, set sign bit to
0 (which means positive number) and voilà.
Generated numbers will be in 1 to 2 interval, so we also must subtract 1 from it.}

\newcommand{\URLXOR}{\url{http://goo.gl/KiVQjZ}}

\RU{В моем примере\footnote{идея взята здесь: \URLXOR} 
применяется очень простой линейный конгруэнтный генератор случайных чисел, выдающий 32-битные числа.
Генератор инициализируется текущим временем в стиле UNIX.}
\EN{Very simple linear congruential random numbers generator is used in my 
example\footnote{idea was taken from: \URLXOR}, produces 32-bit numbers. 
The PRNG initializing by current time in UNIX-style.}

\RU{Далее, тип \Tfloat представляется в виде \IT{union} ~--- это конструкция \CCpp позволяющая 
интерпретировать часть памти по-разному. В нашем случае, мы можем создать переменную типа \TT{union} 
и затем обращаться к ней как к \Tfloat или как к \IT{uint32\_t}. Можно сказать, что это хак, причем грязный.}
\EN{Then, \Tfloat type represented as \IT{union}~---it is the \CCpp construction enabling us
to interpret piece of memory as differently typed.
In our case, we are able to create a variable
of \TT{union} type and then access to it as it is \Tfloat or as it is \IT{uint32\_t}. 
It can be said, it is just a hack. A dirty one.}

\lstinputlisting{patterns/17_unions/FPU_PRNG.cpp}

\lstinputlisting[caption=\Optimizing MSVC 2010]{patterns/17_unions/FPU_PRNG_msvc_2010_Ox_\LANG.asm}

\RU{А результат GCC будет почти таким же.}\EN{GCC produces very similar code.}

