\section{\RU{Вычисление машинного эпсилона}\EN{Calculating machine epsilon}}

\subsection{x86}

\RU{Машинный эпсилон --- это самая маленькая гранула, с которой может работать \ac{FPU} 
\footnote{В русскоязычной литературе встречается также термин ``машинный ноль''.}.}
\EN{Machine epsilon is a smallest possible granule \ac{FPU} can work with.}
\RU{Чем больше бит выделено для числа с плавающей точкой, тем меньше машинный эпсилон.}
\EN{The more bits allocated for floating point number, the smaller machine epsilon.}
\RU{Это}\EN{It is} $2^{-23} = 1.19e-07$ \ForENRU \Tfloat \AndENRU $2^{-52} = 2.22e-16$ \ForENRU \Tdouble.

\RU{Любопытно, что вычислить машинный эпсилон очень легко:}
\EN{It's interesting, how easy it's possible to calculate machine epsilon:}

\lstinputlisting{patterns/17_unions/epsilon/float.c}

\RU{Что мы здесь делаем это обходимся с мантиссой числа в формате IEEE 754 как с целочисленным числом и прибавляем
единицу к нему.}
\EN{What we do here is just treating fraction part of IEEE 754 number as integer and adding 1 to it.}
\RU{Итоговое число с плавающей точкой будет равно $starting\_value+machine\_epsilon$, так что нам
нужно просто вычесть изначальное значение (используя арифметику с плавающей точкой) чтобы измерить, 
какое число отражает один бит в одинарной точности (\Tfloat).}
\EN{Resulting floating number will be equal to $starting\_value+machine\_epsilon$, so we just need to subtract
starting value (using floating point arithmetics) to measure, what number one bit reflects
in the single precision (\Tfloat).}

\RU{\IT{union} здесь нужен чтобы мы могли обращаться к числу в формате IEEE 754 как к обычному целочисленному.}
\EN{\IT{union} serves here as a way to access IEEE 754 number as a regular integer.}
\RU{Прибавление 1 к нему на самом деле прибавляет 1 к \IT{мантиссе} числа, хотя, нужно сказать,
переполнение также возможно, что приведет к прибавлению единицы к экспоненте.}
\EN{Adding 1 to it is in fact adds 1 to \IT{fraction} part of number, however, needless to say,
overflow is possible, which will add another 1 to exponent part.}

\lstinputlisting[caption=\Optimizing MSVC 2010]{patterns/17_unions/epsilon/float_MSVC_2010_Ox.asm.\LANG}

\RU{Вторая инструкция FST избыточная: нет необходимости сохранять входное значение в этом же месте
(компилятор решил выделить переменную $v$ в том же месте локального стека, где находится и 
входной аргумент).}
\EN{Second FST instruction is redundant: there are no need to store input value to the same
place (compiler decided to allocate $v$ variable at the same point of local stack as input 
argument).}

\RU{Далее оно инкрементируется при помощи INC, как если это обычная целочисленная переменная.}
\EN{Then it is incremented with INC, as it is usual integer variable.}
\RU{Затем оно загружается в FPU как если это 32-битное число в формате IEEE 754, FSUBR делает остальную
часть работы и результат в ST0.}
\EN{Then it is loaded into FPU as it is 32-bit IEEE 754 number, FSUBR do the rest of job and resulting
value is in the ST0.}

\RU{Последняя пара инструкций FSTP/FLD избыточна, но компилятор не соптимизировал её.}
\EN{Last FSTP/FLD instruction pair is redundant, but compiler didn't optimized it.}

\ifdefined\IncludeARM
\subsection{ARM64}

\RU{Расширим этот пример до 64-бит:}\EN{Let's extend our example to 64-bit:}

\lstinputlisting[label=machine_epsilon_double_c]{patterns/17_unions/epsilon/double.c}

\RU{В ARM64 нет инструкции для добавления числа к D-регистру в FPU, так что входное значение
(пришедшее в D0) в начале копируется в \ac{GPR},
инкрементируется, копируется в регистр FPU D1, затем происходит вычитание.}
\EN{ARM64 has no instruction which can add a number to FPU D-register, 
so input value (came in D0) is first copied into \ac{GPR},
incremented, copied to FPU register D1, then subtraction occurred.}

\lstinputlisting[caption=\Optimizing GCC 4.9 ARM64]{patterns/17_unions/epsilon/double_GCC49_ARM64_O3.s.\LANG}

\RU{Смотрите также этот пример скомпилированный под x64 с SIMD-инструкциями}
\EN{See also this example compiled for x64 with SIMD instructions}: \ref{machine_epsilon_x64_and_SIMD}.
\fi

\ifdefined\IncludeMIPS
\subsection{MIPS}

\index{MIPS!\Instructions!MTC1}
\RU{Новая для нас здесь инструкция это MTC1 (``Move To Coprocessor 1''), она просто переносит данные
из \ac{GPR} в регистры FPU.}
\EN{New instruction here is MTC1 (``Move To Coprocessor 1''), it just transfers data from \ac{GPR}
to FPU registers.}

\lstinputlisting[caption=\Optimizing GCC 4.4.5 (IDA)]{patterns/17_unions/epsilon/MIPS_O3_IDA.lst}

\fi

\subsection{\Conclusion}

\RU{Трудно сказать, понадобится ли кому-то такая эквилибристика в реальном коде,
но как я уже упоминал много раз в этой книге, этот пример хорошо подходит для объяснения формата
IEEE 754 и \IT{union} в \CCpp.}
\EN{It's hard to say, whether someone will need this trickery in real-world code, 
but as I write many times in this book, this example is serving well 
for explaining IEEE 754 format and \IT{union} in \CCpp.}
