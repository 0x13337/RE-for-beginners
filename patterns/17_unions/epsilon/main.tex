\section{\RU{Вычисление машинного эпсилона}\EN{Calculating machine epsilon}}

\subsection{x86}

\RU{Машинный эпсилон --- это самая маленькая гранула, с которой может работать \ac{FPU} 
\footnote{В русскоязычной литературе встречается также термин ``машинный ноль''.}.}
\EN{The machine epsilon is the smallest possible value the \ac{FPU} can work with.}
\RU{Чем больше бит выделено для числа с плавающей точкой, тем меньше машинный эпсилон.}
\EN{The more bits allocated for floating point number, the smaller the machine epsilon.}
\RU{Это}\EN{It is} $2^{-23} = 1.19e-07$ \ForENRU \Tfloat \AndENRU $2^{-52} = 2.22e-16$ \ForENRU \Tdouble.

\RU{Любопытно, что вычислить машинный эпсилон очень легко:}
\EN{It's interesting, how easy it's to calculate the machine epsilon:}

\lstinputlisting{patterns/17_unions/epsilon/float.c}

\RU{Что мы здесь делаем это обходимся с мантиссой числа в формате IEEE 754 как с целочисленным числом и прибавляем
единицу к нему.}
\EN{What we do here is just treat the fraction part of the IEEE 754 number as integer and add 1 to it.}
\RU{Итоговое число с плавающей точкой будет равно $starting\_value+machine\_epsilon$, так что нам
нужно просто вычесть изначальное значение (используя арифметику с плавающей точкой) чтобы измерить, 
какое число отражает один бит в одинарной точности (\Tfloat).}
\EN{The resulting floating number will be equal to $starting\_value+machine\_epsilon$, so we just need to subtract
the starting value (using floating point arithmetic) to measure, what difference one bit reflects
in the single precision (\Tfloat).}

\RU{\IT{union} здесь нужен чтобы мы могли обращаться к числу в формате IEEE 754 как к обычному целочисленному.}
\EN{The \IT{union} serves here as a way to access IEEE 754 number as a regular integer.}
\RU{Прибавление 1 к нему на самом деле прибавляет 1 к \IT{мантиссе} числа, хотя, нужно сказать,
переполнение также возможно, что приведет к прибавлению единицы к экспоненте.}
\EN{Adding 1 to it in fact adds 1 to the \IT{fraction} part of the number, however, needless to say,
overflow is possible, which will add another 1 to the exponent part.}

\lstinputlisting[caption=\Optimizing MSVC 2010]{patterns/17_unions/epsilon/float_MSVC_2010_Ox.asm.\LANG}

\RU{Вторая инструкция FST избыточная: нет необходимости сохранять входное значение в этом же месте
(компилятор решил выделить переменную $v$ в том же месте локального стека, где находится и 
входной аргумент).}
\EN{The second FST instruction is redundant: there is no need to store the input value in the same
place (the compiler decided to allocate the $v$ variable at the same point in the local stack as the input 
argument).}

\RU{Далее оно инкрементируется при помощи INC, как если это обычная целочисленная переменная.}
\EN{Then it is incremented with INC, as it is a normal integer variable.}
\RU{Затем оно загружается в FPU как если это 32-битное число в формате IEEE 754, FSUBR делает остальную
часть работы и результат в ST0.}
\EN{Then it is loaded into the FPU as a 32-bit IEEE 754 number, FSUBR does the rest of job and the resulting
value is stored in ST0.}

\RU{Последняя пара инструкций FSTP/FLD избыточна, но компилятор не соптимизировал её.}
\EN{The last FSTP/FLD instruction pair is redundant, but the compiler didn't optimize it out.}

\ifdefined\IncludeARM
\subsection{ARM64}

\RU{Расширим этот пример до 64-бит:}\EN{Let's extend our example to 64-bit:}

\lstinputlisting[label=machine_epsilon_double_c]{patterns/17_unions/epsilon/double.c}

\RU{В ARM64 нет инструкции для добавления числа к D-регистру в FPU, так что входное значение
(пришедшее в D0) в начале копируется в \ac{GPR},
инкрементируется, копируется в регистр FPU D1, затем происходит вычитание.}
\EN{ARM64 has no instruction that can add a number to a FPU D-register, 
so the input value (that came in D0) is first copied into \ac{GPR},
incremented, copied to FPU register D1, and then subtraction occurs.}

\lstinputlisting[caption=\Optimizing GCC 4.9 ARM64]{patterns/17_unions/epsilon/double_GCC49_ARM64_O3.s.\LANG}

\RU{Смотрите также этот пример скомпилированный под x64 с SIMD-инструкциями}
\EN{See also this example compiled for x64 with SIMD instructions}: \ref{machine_epsilon_x64_and_SIMD}.
\fi

\ifdefined\IncludeMIPS
\subsection{MIPS}

\index{MIPS!\Instructions!MTC1}
\RU{Новая для нас здесь инструкция это MTC1 (``Move To Coprocessor 1''), она просто переносит данные
из \ac{GPR} в регистры FPU.}
\EN{The new instruction here is MTC1 (``Move To Coprocessor 1''), it just transfers data from \ac{GPR}
to the FPU's registers.}

\lstinputlisting[caption=\Optimizing GCC 4.4.5 (IDA)]{patterns/17_unions/epsilon/MIPS_O3_IDA.lst}

\fi

\subsection{\Conclusion}

\RU{Трудно сказать, понадобится ли кому-то такая эквилибристика в реальном коде,
но как я уже упоминал много раз в этой книге, этот пример хорошо подходит для объяснения формата
IEEE 754 и \IT{union} в \CCpp.}
\EN{It's hard to say whether someone will need this trickery in real-world code, 
but as I write many times in this book, this example serves well 
for explaining the IEEE 754 format and \IT{union}s in \CCpp.}
