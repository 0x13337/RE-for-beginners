\chapter{\RU{Объединения (union)}\EN{Unions}}

\EN{\CCpp \IT{union} is mostly used for interpreting a variable (or memory block) of one data type as a variable of another data type.}
\RU{\IT{union} в \CCpp используется в основном для интерпертации переменной (или блока памяти) одного типа как переменной другого типа.}

% sections
\section{\RU{Пример генератора случайных чисел}\EN{Pseudo-random number generator example}}
\label{FPU_PRNG}

\RU{Если нам нужны случайные значения с плавающей запятой в интервале от 0 до 1, самое простое это взять
\ac{PRNG} вроде Mersenne twister.
Он выдает случайные 32-битные числа в виде DWORD.
Затем мы можем преобразовать это число в \Tfloat и затем разделить на \TT{RAND\_MAX} (\TT{0xFFFFFFFF} в данном случае) ~--- 
полученное число будет в интервале от 0 до 1.}
\EN{If we need float random numbers between 0 and 1, the simplest thing is to use a \ac{PRNG} like
the Mersenne twister. 
It produces random 32-bit values in DWORD form. 
Then we can transform this value to \Tfloat and then
divide it by \TT{RAND\_MAX} (\TT{0xFFFFFFFF} in our case)~---
we getting a value in the 0..1 interval.}

\RU{Но как известно, операция деления ~--- это медленная операция. 
Да и вообще хочется избежать лишних операций с FPU.
Сможем ли мы избежать деления?}
\EN{But as we know, division is slow.
Also, we would like to issue as few FPU operations as possible.
Can we get rid of the division?}

\index{IEEE 754}
\RU{Вспомним состав числа с плавающей запятой: это бит знака, биты мантиссы и биты экспоненты. 
Для получения случайного числа, нам нужно просто заполнить случайными битами все биты мантиссы!}
\EN{Let's recall what a floating point number consists of: sign bit, significand bits and exponent bits.
We just need to store random bits in all significand bits to get a random float number!}

\RU{Экспонента не может быть нулевой (иначе число будет денормализованным), 
так что в эти биты мы запишем \TT{01111111} ~--- 
это будет означать что экспонента равна единице. Далее заполняем мантиссу случайными битами, 
знак оставляем в виде 0 (что значит наше число положительное), и вуаля. 
Генерируемые числа будут в интервале от 1 до 2, так что нам еще нужно будет отнять единицу.}
\EN{The exponent cannot be zero (the number is denormalized in this case), so we are storing \TT{01111111} 
to exponent~---this means that the exponent is 1. 
Then we filling the significand with random bits, set the sign bit to
0 (which means a positive number) and voilà.
The generated numbers is to be between 1 and 2, so we must also subtract 1.}

\newcommand{\URLXOR}{\url{http://go.yurichev.com/17308}}

\RU{В моем примере\footnote{идея взята здесь: \URLXOR} 
применяется очень простой линейный конгруэнтный генератор случайных чисел, выдающий 32-битные числа.
Генератор инициализируется текущим временем в стиле UNIX.}
\EN{A very simple linear congruential random numbers generator is used in my 
example\footnote{the idea was taken from: \URLXOR}, it produces 32-bit numbers. 
The \ac{PRNG} is initialized with the current time in UNIX timestamp format.}

\RU{Далее, тип \Tfloat представляется в виде \IT{union} ~--- это конструкция \CCpp позволяющая 
интерпретировать часть памяти по-разному. В нашем случае, мы можем создать переменную типа \TT{union} 
и затем обращаться к ней как к \Tfloat или как к \IT{uint32\_t}. Можно сказать, что это хак, причем грязный.}
\EN{Here we represent the \Tfloat type as an \IT{union}~---it is the \CCpp construction that enables us
to interpret a piece of memory as different types.
In our case, we are able to create a variable
of type \TT{union} and then access to it as it is \Tfloat or as it is \IT{uint32\_t}. 
It can be said, it is just a hack. A dirty one.}

% WTF?
\RU{Код целочисленного \ac{PRNG} точно такой же, как мы уже рассматривали раннее:}
\EN{The integer \ac{PRNG} code is the same as we already considered:} \myref{LCG_simple}.
\RU{Так что и в скомпилированном виде этот код будет опущен.}
\EN{So this code in compiled form is omitted.}

\lstinputlisting{patterns/17_unions/FPU_PRNG/FPU_PRNG.cpp.\LANG}

\subsection{x86}

\lstinputlisting[caption=\Optimizing MSVC 2010]{patterns/17_unions/FPU_PRNG/MSVC2010_Ox_Ob0.asm.\LANG}

\EN{Function names are so strange here because I compiled this example as C++ and this is name mangling in C++,
we will talk about it later:}
\RU{Имена функций такие странные, потому что я компилировал этот пример как Си++, и это манглинг имен в Си++, 
мы будем рассматривать это позже:} \myref{namemangling}.

\RU{Если скомпилировать это в MSVC 2012, компилятор будет использовать SIMD-инструкции для FPU, читайте об этом
здесь:}
\EN{If we compile this in MSVC 2012, it uses the SIMD instructions for the FPU, read more about it here:}
\myref{FPU_PRNG_SIMD}.

\subsection{MIPS}

\lstinputlisting[caption=\Optimizing GCC 4.4.5]{patterns/17_unions/FPU_PRNG/MIPS_O3_IDA.lst.\LANG}

\EN{There is also an useless LUI instruction added for some weird reason.}
\RU{Здесь снова зачем-то добавлена инструкция LUI, которая ничего не делает.}
\EN{We considered this artifact earlier:}
\RU{Мы уже рассматривали этот артефакт раннее:} \myref{MIPS_FPU_LUI}.

\subsection{ARM (\ARMMode)}

\lstinputlisting[caption=\Optimizing GCC 4.6.3 (IDA)]{patterns/17_unions/FPU_PRNG/raspberry_GCC_O3_IDA.lst.\LANG}

\index{objdump}
\index{binutils}
\index{IDA}
\RU{Я также сделал дамп в objdump и увидел что FPU-инструкции имеют немного другие имена чем в IDA.}
\EN{I also made a dump in objdump and I saw that the FPU instructions have different names than in IDA.}
\EN{Apparently, IDA and binutils developers used different manuals?}
\RU{Наверное, разработчики IDA и binutils пользовались разной документацией?}
\EN{I suppose, it would be good to know both instruction name variants.}
\RU{Полагаю, будет полезно знать оба варианта названий инструкций.}

\lstinputlisting[caption=\Optimizing GCC 4.6.3 (objdump)]{patterns/17_unions/FPU_PRNG/raspberry_GCC_O3_objdump.lst}

\EN{The instructions at 5c in float\_rand() and at 38 in main() are random noise.}
\RU{Инструкции по адресам 5c в float\_rand() и 38 в main() это случайный мусор.}

\section{\RU{Вычисление машинного эпсилона}\EN{Calculating machine epsilon}}

\RU{Машинный эпсилон --- это самая маленькая гранула, с которой может работать \ac{FPU} 
\footnote{В русскоязычной литературе встречается также термин ``машинный ноль''.}.}
\EN{The machine epsilon is the smallest possible value the \ac{FPU} can work with.}
\RU{Чем больше бит выделено для числа с плавающей точкой, тем меньше машинный эпсилон.}
\EN{The more bits allocated for floating point number, the smaller the machine epsilon.}
\RU{Это}\EN{It is} $2^{-23} = 1.19e-07$ \ForENRU \Tfloat \AndENRU $2^{-52} = 2.22e-16$ \ForENRU \Tdouble.

\RU{Любопытно, что вычислить машинный эпсилон очень легко:}
\EN{It's interesting, how easy it's to calculate the machine epsilon:}

\lstinputlisting{patterns/17_unions/epsilon/float.c}

\RU{Что мы здесь делаем это обходимся с мантиссой числа в формате IEEE 754 как с целочисленным числом и прибавляем
единицу к нему.}
\EN{What we do here is just treat the fraction part of the IEEE 754 number as integer and add 1 to it.}
\RU{Итоговое число с плавающей точкой будет равно $starting\_value+machine\_epsilon$, так что нам
нужно просто вычесть изначальное значение (используя арифметику с плавающей точкой) чтобы измерить, 
какое число отражает один бит в одинарной точности (\Tfloat).}
\EN{The resulting floating number is equal to $starting\_value+machine\_epsilon$, so we just need to subtract
the starting value (using floating point arithmetic) to measure, what difference one bit reflects
in the single precision (\Tfloat).}

\RU{\IT{union} здесь нужен чтобы мы могли обращаться к числу в формате IEEE 754 как к обычному целочисленному.}
\EN{The \IT{union} serves here as a way to access IEEE 754 number as a regular integer.}
\RU{Прибавление 1 к нему на самом деле прибавляет 1 к \IT{мантиссе} числа, хотя, нужно сказать,
переполнение также возможно, что приведет к прибавлению единицы к экспоненте.}
\EN{Adding 1 to it in fact adds 1 to the \IT{fraction} part of the number, however, needless to say,
overflow is possible, which will add another 1 to the exponent part.}

\subsection{x86}

\lstinputlisting[caption=\Optimizing MSVC 2010]{patterns/17_unions/epsilon/float_MSVC_2010_Ox.asm.\LANG}

\RU{Вторая инструкция FST избыточная: нет необходимости сохранять входное значение в этом же месте
(компилятор решил выделить переменную $v$ в том же месте локального стека, где находится и 
входной аргумент).}
\EN{The second FST instruction is redundant: there is no need to store the input value in the same
place (the compiler decided to allocate the $v$ variable at the same point in the local stack as the input 
argument).}

\RU{Далее оно инкрементируется при помощи INC, как если это обычная целочисленная переменная.}
\EN{Then it is incremented with INC, as it is a normal integer variable.}
\RU{Затем оно загружается в FPU как если это 32-битное число в формате IEEE 754, FSUBR делает остальную
часть работы и результат в ST0.}
\EN{Then it is loaded into the FPU as a 32-bit IEEE 754 number, FSUBR does the rest of job and the resulting
value is stored in ST0.}

\RU{Последняя пара инструкций FSTP/FLD избыточна, но компилятор не соптимизировал её.}
\EN{The last FSTP/FLD instruction pair is redundant, but the compiler didn't optimize it out.}

\ifdefined\IncludeARM
\subsection{ARM64}

\RU{Расширим этот пример до 64-бит:}\EN{Let's extend our example to 64-bit:}

\lstinputlisting[label=machine_epsilon_double_c]{patterns/17_unions/epsilon/double.c}

\RU{В ARM64 нет инструкции для добавления числа к D-регистру в FPU, так что входное значение
(пришедшее в D0) в начале копируется в \ac{GPR},
инкрементируется, копируется в регистр FPU D1, затем происходит вычитание.}
\EN{ARM64 has no instruction that can add a number to a FPU D-register, 
so the input value (that came in D0) is first copied into \ac{GPR},
incremented, copied to FPU register D1, and then subtraction occurs.}

\lstinputlisting[caption=\Optimizing GCC 4.9 ARM64]{patterns/17_unions/epsilon/double_GCC49_ARM64_O3.s.\LANG}

\RU{Смотрите также этот пример скомпилированный под x64 с SIMD-инструкциями}
\EN{See also this example compiled for x64 with SIMD instructions}: \myref{machine_epsilon_x64_and_SIMD}.
\fi

\ifdefined\IncludeMIPS
\subsection{MIPS}

\index{MIPS!\Instructions!MTC1}
\RU{Новая для нас здесь инструкция это MTC1 (``Move To Coprocessor 1''), она просто переносит данные
из \ac{GPR} в регистры FPU.}
\EN{The new instruction here is MTC1 (``Move To Coprocessor 1''), it just transfers data from \ac{GPR}
to the FPU's registers.}

\lstinputlisting[caption=\Optimizing GCC 4.4.5 (IDA)]{patterns/17_unions/epsilon/MIPS_O3_IDA.lst}

\fi

\subsection{\Conclusion}

\RU{Трудно сказать, понадобится ли кому-то такая эквилибристика в реальном коде,
но как я уже упоминал много раз в этой книге, этот пример хорошо подходит для объяснения формата
IEEE 754 и \IT{union} в \CCpp.}
\EN{It's hard to say whether someone may need this trickery in real-world code, 
but as I write many times in this book, this example serves well 
for explaining the IEEE 754 format and \IT{union}s in \CCpp.}


\section{\RU{Быстрое вычисление обратного квадратного корня}\EN{Fast inverse square root calculation}}

\RU{Вот где еще можно на практике применить трактовку типа \Tfloat как целочисленного, это быстрое вычисление}
\EN{Another well-known algorithm where \Tfloat is interpreted as integer is fast calculation of} $\frac{1}{\sqrt{x}}$.
\index{Quake III Arena}
\RU{Алгоритм стал известным, вероятно потому, что был применен в Quake III Arena.}
\EN{Algorithm became popular, supposedly, because it was used in Quake III Arena.}

\RU{Фрагмент кода взят из}\EN{This piece of code was taken from} Wikipedia
\footnote{\EN{\url{http://go.yurichev.com/17360}}\RU{\url{http://go.yurichev.com/17361}}}:

\begin{lstlisting}
float Q_rsqrt( float number )
{
	long i;
	float x2, y;
	const float threehalfs = 1.5F;
 
	x2 = number * 0.5F;
	y  = number;
	i  = * ( long * ) &y;                       // evil floating point bit level hacking
	i  = 0x5f3759df - ( i >> 1 );               // what the f*ck? 
	y  = * ( float * ) &i;
	y  = y * ( threehalfs - ( x2 * y * y ) );   // 1st iteration
//      y  = y * ( threehalfs - ( x2 * y * y ) );   // 2nd iteration, this can be removed
 
	return y;
}
\end{lstlisting}

\RU{В качестве упражнения, вы можете попробовать скомпилировать эту функцию и разобраться, как она работает.}
\EN{As an exercise, you can try to compile this function and to understand, how it works.}

