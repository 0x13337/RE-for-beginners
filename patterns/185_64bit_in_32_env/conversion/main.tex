\section{\RU{Конвертирование 32-битного значения в 64-битное}\EN{Converting 32-bit value into 64-bit one}}
\label{subsec:sign_extending_32_to_64}

\lstinputlisting{patterns/185_64bit_in_32_env/conversion/4.c}

\subsection{x86}

\lstinputlisting[caption=\Optimizing MSVC 2012]{patterns/185_64bit_in_32_env/conversion/MSVC2012_Ox.asm}

\RU{Здесь появляется необходимость расширить 32-битное знаковое значение в 64-битное знаковое.}
\EN{Here we also run into necessity to extend a 32-bit signed value into a 64-bit signed one.}
\RU{Конвертировать беззнаковые значения очень просто: нужно просто выставить в 0 все биты в старшей части}
\EN{Unsigned values are converted straightforwardly: all bits in the higher part must be set to 0}.
\RU{Но для знаковых типов это не подходит: знак числа должен быть скопирован в старшую часть числа-результата}
\EN{But this is not appropriate for signed data types: the sign has to be copied into the higher part of the resulting number}.
\index{x86!\Instructions!CDQ}
\RU{Здесь это делает инструкция \TT{CDQ}, она берет входное значение в \EAX{}, расширяет его до 64-битного,
и оставляет его в паре регистров \EDX{}:\EAX{}}
\EN{The \TT{CDQ} instruction does that here, it takes its input value in \EAX{}, extends it to 64-bit and leaves it
in the \EDX{}:\EAX{} register pair}.
\RU{Иными словами, инструкция \TT{CDQ} узнает знак числа в \EAX{} (просто берет самый старший бит в \EAX{}) и в зависимости от этого,
выставляет все 32 бита в \EDX{} в 0 или в 1}\EN{In other words, \TT{CDQ} gets the number sign from \EAX{} (by getting the
most significant bit in \EAX{}), and depending of it, sets all 32 bits in \EDX{} to 0 or 1}.
\RU{Её работа в каком-то смысле напоминает работу инструкции \MOVSX{}}\EN{Its operation is somewhat
similar to the \MOVSX{} instruction}.

\ifdefined\IncludeARM
\subsection{ARM}

\lstinputlisting[caption=\OptimizingKeilVI (\ARMMode)]{patterns/185_64bit_in_32_env/conversion/Keil_ARM_O3.s}

\RU{Keil для ARM работает иначе: он просто сдвигает (арифметически) входное значение на 31 бит вправо.}
\EN{Keil for ARM is different: it just arithmetically shifts right the input value by 31 bits.}
\RU{Как мы знаем, бит знака это \ac{MSB}, и арифметический сдвиг копирует бит знака в ``появляющихся'' битах.}
\EN{As we know, the sign bit is \ac{MSB}, and the arithmetical shift copies the sign bit into the ``emerged'' bits.}
\RU{Так что после инструкции ``ASR r1,r0,\#31'', R1 будет содержать 0xFFFFFFFF если входное значение
было отрицательным, или 0 в противном случае.}
\EN{So after ``ASR r1,r0,\#31'', R1 containing 0xFFFFFFFF if the input value was negative
and 0 otherwise.}
\RU{R1 содержит старшую часть возвращаемого 64-битного значения.}
\EN{R1 contains the high part of the resulting 64-bit value.}

\RU{Другими словами, этот код просто копирует \ac{MSB} (бит знака) из входного значения в R0 во все
биты старшей 32-битной части итогового 64-битного значения.}
\EN{In other words, this code just copies the \ac{MSB} (sign bit) from the input value in R0 to all bits
of the high 32-bit part of the resulting 64-bit value.}

\fi

\ifdefined\IncludeMIPS
\subsection{MIPS}

\ifdefined\IncludeGCC
\EN{GCC for MIPS does the same as Keil did for ARM mode:}
\RU{GCC для MIPS делает то же, что сделал Keil для режима ARM:}

\lstinputlisting[caption=\Optimizing GCC 4.4.5 (IDA)]{patterns/185_64bit_in_32_env/conversion/MIPS_O3_IDA.lst}
\fi

\fi
