\chapter{\RU{64-битные значения в 32-битной среде}\EN{64-bit values in 32-bit environment}}
\label{sec:64bit_in_32_env}

\RU{В среде, где \ac{GPR}-ы 32-битные, 64-битные значения хранятся и передаются как пары 32-битных значений}
\EN{In the 32-bit environment \ac{GPR}'s are 32-bit, so 64-bit values are stored and passed as 32-bit value pairs}
\footnote{\RU{Кстати, в 16-битной среде, 32-битные значения передаются 16-битными парами точно так же}
\EN{By the way, 32-bit values are passed as pairs in 16-bit environment just as the same}: 
\ref{win16_32bit_values}}.

\section{\RU{Возврат 64-битного значения}\EN{Returning of 64-bit value}}

\lstinputlisting{patterns/185_64bit_in_32_env/0.c}

\subsection{x86}

\RU{64-битные значения в 32-битной среде возвращаются из ф-ций в паре регистров \EDX{}:\EAX{}.}
\EN{In a 32-bit environment, 64-bit values are returned from a functions in \EDX{}:\EAX{} registers pair.}

\lstinputlisting[caption=\Optimizing MSVC 2010]{patterns/185_64bit_in_32_env/0_MSVC_2010_Ox.asm}

\ifx\RUSSIAN\undefined
\ifdefined\IncludeMIPS
\subsection{MIPS}

64-bit value is returned in V0-V1 (\$2-\$3) registers pair (V0 (\$2) is for high part and V1 (\$3) for low part):

\lstinputlisting[caption=\Optimizing GCC 4.4.5 (assembly listing)]{patterns/185_64bit_in_32_env/0_MIPS.s}

\lstinputlisting[caption=\Optimizing GCC 4.4.5 (IDA)]{patterns/185_64bit_in_32_env/0_MIPS_IDA.lst}
\fi
\fi

\section{\RU{Передача аргументов, сложение, вычитание}\EN{Arguments passing, addition, subtraction}}

\lstinputlisting{patterns/185_64bit_in_32_env/1.c}

\lstinputlisting[caption=\Optimizing MSVC 2012 /Ob1]{patterns/185_64bit_in_32_env/1_MSVC.asm}

\RU{В}\EN{We may see in the} \TT{f1\_test()} \RU{видно как каждое 64-битное число передается двумя 32-битными значениями,
сначала старшая часть, затем младшая}\EN{function as each 64-bit value is passed by two 32-bit values,
high part first, then low part}. \\
\\
\RU{Сложение и вычитание происходит также парами}\EN{Addition and subtraction occurring by pairs as well}. \\
\\
\index{x86!\Instructions!ADC}
\RU{При сложении, в начале складываются младшие 32 бита}\EN{While addition, low 32-bit part are added first}.
\RU{Если при сложении был перенос, выставляется флаг CF}\EN{If carry was occurred while addition, CF flag is set}.
\RU{Следующая инструкция \TT{ADC} складывает старшие части чисел, но также прибавляет единицу если $CF=1$}
\EN{The next \TT{ADC} instruction adds high parts of values, but also adds 1 if $CF=1$}. \\
\\
\index{x86!\Instructions!SBB}
\RU{Вычитание также происходит парами}\EN{Subtraction is also occurred by pairs}.
\RU{Первый}\EN{The very first} \SUB \RU{может также включить флаг переноса CF, который затем будет проверяться в \TT{SBB}}
\EN{may also turn CF flag on, which will be checked in the subsequent \TT{SBB} instruction}:
\RU{если флаг переноса включен, то от результата отнимется единица}\EN{if carry flag is on, then 1 will also be subtracted
from the result}. \\
\\
\RU{Легко увидеть, как результат работы \TT{f1()} затем передается в \printf{}}\EN{It is easily can be seen how
\TT{f1()} function is then passed to \printf{}}.

\lstinputlisting[caption=GCC 4.8.1 -O1 -fno-inline]{patterns/185_64bit_in_32_env/1_GCC.asm}

\RU{Код GCC почти такой же}\EN{GCC code is the same}.

\section{\RU{Умножение, деление}\EN{Multiplication, division}}

\lstinputlisting{patterns/185_64bit_in_32_env/2.c}

\lstinputlisting[caption=\Optimizing MSVC 2012 /Ob1]{patterns/185_64bit_in_32_env/2_MSVC.asm}

\RU{Умножение и деление это более сложная операция, так что обычно, компилятор встраивает вызовы библиотечных ф-ций,
делающих это}\EN{Multiplication and division is more complex operation, so usually, the compiler embedds calls to
the library functions doing that}. \\
\\
\RU{Значение этих библиотечных ф-ций, здесь}\EN{These functions meaning are here}: \ref{sec:MSVC_library_func}.

\lstinputlisting[caption=\Optimizing GCC 4.8.1 -fno-inline]{patterns/185_64bit_in_32_env/2_GCC.asm}

\RU{GCC делает почти то же самое, тем не менее,
встраивает код умножения прямо в ф-цию, посчитав что так будет эффективнее}\EN{GCC doing almost the same, but multiplication
code is inlined right in the function, thinking it could be more efficient}.
\RU{У GCC другие имена библиотечных ф-ций}\EN{GCC has different library function names}: \ref{sec:GCC_library_func}.

\section{\RU{Сдвиг вправо}\EN{Shifting right}}

\lstinputlisting{patterns/185_64bit_in_32_env/3.c}

\lstinputlisting[caption=\Optimizing MSVC 2012 /Ob1]{patterns/185_64bit_in_32_env/3_MSVC.asm}

\lstinputlisting[caption=\Optimizing GCC 4.8.1 -fno-inline]{patterns/185_64bit_in_32_env/3_GCC.asm}

\index{x86!\Instructions!SHRD}
\RU{Сдвиг происходит также в две операции: в начале сдвигается младшая часть, затем старшая}
\EN{Shifting also occurring in two passes: first lower part is shifting, then higher part}.
\RU{Но младшая часть сдвигается
при помощи инструкции \TT{SHRD}, она сдвигает значение в \EDX{} на 7 бит, но подтягивает новые биты из \EAX{}, т.е., из старшей части}
\EN{But the lower part is shifting with the help of \TT{SHRD} instruction, it shifting \EDX{} value by 7 bits, but pulling new bits
from \EAX{}, i.e., from the higher part}.
\RU{Старшая часть сдвигается более известной инструкцией \SHR{}: действительно, ведь освободившиеся биты в старшей части нужно
просто заполнить нулями}\EN{Higher part is shifting using more popular \SHR{} instruction: indeed, freed bits in the higher part
should be just filled with zeroes}.

\section{\RU{Конвертирование 32-битного значения в 64-битное}\EN{Converting 32-bit value into 64-bit one}}
\label{subsec:sign_extending_32_to_64}

\lstinputlisting{patterns/185_64bit_in_32_env/4.c}

\lstinputlisting[caption=\Optimizing MSVC 2012 /Ob1]{patterns/185_64bit_in_32_env/4_MSVC.asm}

\RU{Здесь появляется необходимость расширить 32-битное знаковое значение из $c$ в 64-битное знаковое}
\EN{Here we also run into necessity to extend 32-bit signed value from $c$ into 64-bit signed}.
\RU{Конвертировать беззнаковые значения очень просто: нужно просто выставить в 0 все биты в старшей части}
\EN{Unsigned values are converted straightforwardly: all bits in higher part must be set to 0}.
\RU{Но для знаковых типов это не подходит: знак числа должен быть скопирован в старшую часть числа-результата}
\EN{But it is not appropriate for signed data types: sign should be copied into higher part of resulting number}.
\index{x86!\Instructions!CDQ}
\RU{Здесь это делает инструкция \TT{CDQ}, она берет входное значение в \EAX{}, расширяет его до 64-битного,
и оставляет его в паре регистров \EDX{}:\EAX{}}
\EN{\TT{CDQ} instruction doing that here, it takes input value in \EAX{}, extending it to 64-bit and leaving it
in the \EDX{}:\EAX{} registers pair}.
\RU{Иными словами, инструкция \TT{CDQ} узнает знак числа в \EAX{} (просто берет самый старший бит в \EAX{}) и в зависимости от этого,
выставляет все 32 бита в \EDX{} в 0 или в 1}\EN{In other words, \TT{CDQ} instruction gets number sign in \EAX{} (by getting just
most significant bit in \EAX{}), and depending of it, setting all 32-bits in \EDX{} to 0 or 1}.
\RU{Её работа в каком-то смысле напоминает работу инструкции \MOVSX{}}\EN{Its operation is somewhat
similar to the \MOVSX{} instruction}.

\lstinputlisting[caption=\Optimizing GCC 4.8.1 -fno-inline]{patterns/185_64bit_in_32_env/4_GCC.asm}

\RU{GCC генерирует такой же код как и MSVC, но обходится без вызова библиотечной ф-ции для перемножения значений.}
\EN{GCC generates just the same code as MSVC, but inlines multiplication code right in the function.}
