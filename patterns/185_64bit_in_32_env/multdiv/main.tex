\section{\RU{Умножение, деление}\EN{Multiplication, division}}

\lstinputlisting{patterns/185_64bit_in_32_env/multdiv/2.c}

\subsection{x86}

\lstinputlisting[caption=\Optimizing MSVC 2013 /Ob1]{patterns/185_64bit_in_32_env/multdiv/2_MSVC.asm.\LANG}

\RU{Умножение и деление --- это более сложная операция, так что обычно, компилятор встраивает вызовы библиотечных функций,
делающих это}\EN{Multiplication and division are more complex operations, so usually the compiler embeds calls to
a library functions doing that}.

\ifx\LITE\undefined
\RU{Значение этих библиотечных функций, здесь}\EN{These functions are described here}: \myref{sec:MSVC_library_func}.
\fi

\ifdefined\IncludeGCC
\lstinputlisting[caption=\Optimizing GCC 4.8.1 -fno-inline]{patterns/185_64bit_in_32_env/multdiv/2_GCC.asm.\LANG}

\RU{GCC делает почти то же самое, тем не менее,
встраивает код умножения прямо в функцию, посчитав что так будет эффективнее}\EN{GCC does the expected, but the multiplication
code is inlined right in the function, thinking it could be more efficient}.
\RU{У GCC другие имена библиотечных функций}\EN{GCC has different library function names}: \myref{sec:GCC_library_func}.
\fi

\ifdefined\IncludeARM
\subsection{ARM}

\RU{Keil для режима Thumb вставляет вызовы библиотечных функций:}
\EN{Keil for Thumb mode inserts library subroutine calls:}

\lstinputlisting[caption=\OptimizingKeilVI (\ThumbMode)]{patterns/185_64bit_in_32_env/multdiv/Keil_thumb_O3.s}

\RU{Keil для режима ARM, тем не менее, может сгенерировать код для умножения 64-битных чисел:}
\EN{Keil for ARM mode, on the other hand, is able to produce 64-bit multiplication code:}

\lstinputlisting[caption=\OptimizingKeilVI (\ARMMode)]{patterns/185_64bit_in_32_env/multdiv/Keil_ARM_O3.s}
% TODO add explanation
\fi

\ifdefined\IncludeMIPS
\subsection{MIPS}

\Optimizing GCC \ForENRU MIPS 
\EN{can generate 64-bit multiplication code, but has to call a library routine for 64-bit division:}
\RU{может генерировать код для 64-битного умножения, но для 64-битного деления приходится вызывать библиотечную функцию:}

\lstinputlisting[caption=\Optimizing GCC 4.4.5 (IDA)]{patterns/185_64bit_in_32_env/multdiv/MIPS_O3_IDA.lst}

\RU{Тут также много \ac{NOP}-ов, это возможно заполнение delay slot-ов после инструкции умножения (она ведь работает
медленнее прочих инструкций).}
\EN{There are a lot of \ac{NOP}s, probably delay slots filled after the multiplication instruction (it's slower
than other instructions, after all).}

% TODO add explanation
\fi
