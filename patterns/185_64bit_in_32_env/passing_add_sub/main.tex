\section{\RU{Передача аргументов, сложение, вычитание}\EN{Arguments passing, addition, subtraction}}

\lstinputlisting{patterns/185_64bit_in_32_env/passing_add_sub/1.c}

\subsection{x86}

\lstinputlisting[caption=\Optimizing MSVC 2012 /Ob1]{patterns/185_64bit_in_32_env/passing_add_sub/1_MSVC.asm}

\RU{В}\EN{We can see in the} \TT{f\_add\_test()} \RU{видно, как каждое 64-битное число передается двумя 32-битными значениями,
сначала старшая часть, затем младшая}\EN{function that each 64-bit value is passed using two 32-bit values,
high part first, then low part}. \\
\\
\RU{Сложение и вычитание происходит также парами}\EN{Addition and subtraction occur in pairs as well}. \\
\\
\index{x86!\Instructions!ADC}
\RU{При сложении, в начале складываются младшие 32 бита}\EN{In addition, the low 32-bit part are added first}.
\RU{Если при сложении был перенос, выставляется флаг CF}\EN{If carry was occurred while adding, the CF flag is set}.
\RU{Следующая инструкция \TT{ADC} складывает старшие части чисел, но также прибавляет единицу если $CF=1$}
\EN{The following \TT{ADC} instruction adds the high parts of the values, and also adds 1 if $CF=1$}. \\
\\
\index{x86!\Instructions!SBB}
\RU{Вычитание также происходит парами}\EN{Subtraction also occurs in pairs}.
\RU{Первый}\EN{The first} \SUB \RU{может также включить флаг переноса CF, который затем будет проверяться в \TT{SBB}}
\EN{may also turn on the CF flag, which is to be checked in the subsequent \TT{SBB} instruction}:
\RU{если флаг переноса включен, то от результата отнимется единица}
\EN{if the carry flag is on, then 1 is also to be subtracted from the result}. \\
\\
\RU{Легко увидеть, как результат работы \TT{f\_add()} затем передается в \printf{}}\EN{It is easy to see how
the \TT{f\_add()} function result is then passed to \printf{}}.

\ifdefined\IncludeGCC
\lstinputlisting[caption=GCC 4.8.1 -O1 -fno-inline]{patterns/185_64bit_in_32_env/passing_add_sub/1_GCC.asm}

\RU{Код GCC почти такой же}\EN{GCC code is the same}.
\fi

\ifdefined\IncludeARM
\subsection{ARM}

\lstinputlisting[caption=\OptimizingKeilVI (\ARMMode)]{patterns/185_64bit_in_32_env/passing_add_sub/Keil_ARM_O3.s}

\index{ARM!\Instructions!ADDS}
\index{ARM!\Instructions!SUBS}
\index{ARM!\Instructions!ADC}
\index{ARM!\Instructions!SBC}
\RU{Первое 64-битное значение передается в паре регистров R0 и R1, второе --- в паре R2 и R3.}
\EN{The first 64-bit value is passed in R0 and R1 register pair, the second in R2 and R3 register pair.}
\RU{В ARM также есть инструкция ADC (учитывающая флаг переноса) и SBC (\q{subtract with carry} --- вычесть
с переносом).}
\EN{ARM has the ADC instruction as well (which counts carry flag) and SBC (\q{subtract with carry}).}

\RU{Важная вещь: когда младшие части слагаются/вычитаются, используются инструкции ADDS и SUBS с суффиксом
-S.}
\EN{Important thing: when the low parts are added/subtracted, ADDS and SUBS instructions with -S suffix are used.}
\RU{Суффикс -S означает \q{set flags} (установить флаги), а флаги (особенно флаг переноса) это то что однозначно
нужно последующим инструкциями ADC/SBC.}
\EN{The -S suffix stands for \q{set flags}, and flags (esp. carry flag) is what consequent ADC/SBC instructions
definitely need.}

\RU{А иначе инструкции без суффикса -S здесь вполне бы подошли (ADD и SUB).}
\EN{Otherwise, instructions without the -S suffix would do the job (ADD and SUB).}

\fi

\ifdefined\IncludeMIPS
\subsection{MIPS}

\lstinputlisting[caption=\Optimizing GCC 4.4.5 (IDA)]{patterns/185_64bit_in_32_env/passing_add_sub/MIPS_O3_IDA.lst.\LANG}

\RU{В MIPS нет регистра флагов, так что эта информация не присутствует после исполнения арифметических операций.}
\EN{MIPS has no flags register, so there is no such information present after the execution of arithmetic operations.}
\EN{So there are no instructions like x86's ADC and SBB.}
\RU{Так что здесь нет инструкций как ADC или SBB в x86.}
\RU{Чтобы получить информацию о том, был бы выставлен флаг переноса, происходит сравнение (используя инструкцию
\q{SLTU}), которая выставляет целевой регистр в 1 или 0.}
\EN{To know if the carry flag would be set, a comparison (using \q{SLTU} instruction) also occurs, 
which sets the destination register to 1 or 0.}
\RU{Эта 1 или 0 затем прибавляется к итоговому результату, или вычитается.}
\EN{This 1 or 0 is then added or subtracted to/from the final result.}

\fi
