\section{\RU{Передача аргументов, сложение, вычитание}\EN{Arguments passing, addition, subtraction}}

\lstinputlisting{patterns/185_64bit_in_32_env/passing_add_sub/1.c}

\subsection{x86}

\lstinputlisting[caption=\Optimizing MSVC 2012 /Ob1]{patterns/185_64bit_in_32_env/passing_add_sub/1_MSVC.asm}

\RU{В}\EN{We may see in the} \TT{f1\_test()} \RU{видно как каждое 64-битное число передается двумя 32-битными значениями,
сначала старшая часть, затем младшая}\EN{function as each 64-bit value is passed by two 32-bit values,
high part first, then low part}. \\
\\
\RU{Сложение и вычитание происходит также парами}\EN{Addition and subtraction occurring by pairs as well}. \\
\\
\index{x86!\Instructions!ADC}
\RU{При сложении, в начале складываются младшие 32 бита}\EN{While addition, low 32-bit part are added first}.
\RU{Если при сложении был перенос, выставляется флаг CF}\EN{If carry was occurred while addition, CF flag is set}.
\RU{Следующая инструкция \TT{ADC} складывает старшие части чисел, но также прибавляет единицу если $CF=1$}
\EN{The next \TT{ADC} instruction adds high parts of values, but also adds 1 if $CF=1$}. \\
\\
\index{x86!\Instructions!SBB}
\RU{Вычитание также происходит парами}\EN{Subtraction is also occurred by pairs}.
\RU{Первый}\EN{The very first} \SUB \RU{может также включить флаг переноса CF, который затем будет проверяться в \TT{SBB}}
\EN{may also turn CF flag on, which will be checked in the subsequent \TT{SBB} instruction}:
\RU{если флаг переноса включен, то от результата отнимется единица}\EN{if carry flag is on, then 1 will also be subtracted
from the result}. \\
\\
\RU{Легко увидеть, как результат работы \TT{f1()} затем передается в \printf{}}\EN{It is easily can be seen how
\TT{f1()} function is then passed to \printf{}}.

\lstinputlisting[caption=GCC 4.8.1 -O1 -fno-inline]{patterns/185_64bit_in_32_env/passing_add_sub/1_GCC.asm}

\RU{Код GCC почти такой же}\EN{GCC code is the same}.

\ifdefined\IncludeARM
\subsection{ARM}

\lstinputlisting[caption=\OptimizingKeilVI (\ARMMode)]{patterns/185_64bit_in_32_env/passing_add_sub/Keil_ARM_O3.s}

\index{ARM!\Instructions!ADDS}
\index{ARM!\Instructions!SUBS}
\index{ARM!\Instructions!ADC}
\index{ARM!\Instructions!SBC}
\RU{Первое 64-битное значение передается в паре регистров R0 и R1, второе --- в паре R2 и R3.}
\EN{First 64-bit value is passed in R0 and R1 registers pair, second in R2 and R3 register pair.}
\RU{В ARM также есть инструкция ADC (учитывающая флаг переноса) и SBC (``subtract with carry'' --- вычесть
с переносом).}
\EN{ARM has ADC instruction as well (which counts carry flag) and SBC (``subtract with carry'').}

\RU{Важная вещь: когда младшие части слагаются/вычитаются, используются инструкции ADDS и SUBS с суффиксом
-S.}
\EN{Important thing: when low parts are added/subtracting, ADDS and SUBS instructions with -S suffix are used.}
\RU{Суффикс -S означает ``set flags'' (установить флаги), а флаги (особенно флаг переноса) это то что однозначно
нужно последующим инструцкиями ADC/SBC.}
\EN{-S suffix mean ``set flags'', and flags (esp. carry flag) is what consequent ADC/SBC instruction
definitely need.}

\RU{А иначе инструкции без суффикса -S здесь вполне бы подошли (ADD и SUB).}
\EN{Otherwise, instructions without -S suffix would do the job (ADD and SUB).}

\fi
