\section{\RU{Сдвиг вправо}\EN{Shifting right}}

\lstinputlisting{patterns/185_64bit_in_32_env/shifting/3.c}

\subsection{x86}

\lstinputlisting[caption=\Optimizing MSVC 2012 /Ob1]{patterns/185_64bit_in_32_env/shifting/3_MSVC.asm}

\ifdefined\IncludeGCC
\lstinputlisting[caption=\Optimizing GCC 4.8.1 -fno-inline]{patterns/185_64bit_in_32_env/shifting/3_GCC.asm}
\fi

\index{x86!\Instructions!SHRD}
\RU{Сдвиг происходит также в две операции: в начале сдвигается младшая часть, затем старшая}
\EN{Shifting also occurs in two passes: first the lower part is shifted, then the higher part}.
\RU{Но младшая часть сдвигается
при помощи инструкции \TT{SHRD}, она сдвигает значение в \EDX{} на 7 бит, но подтягивает новые биты из \EAX{}, т.е. из старшей части.}
\EN{But the lower part is shifted with the help of the \TT{SHRD} instruction, it shifts the value of \EDX{} by 7 bits, but pulls new bits
from \EAX{}, i.e., from the higher part.}
\RU{Старшая часть сдвигается более известной инструкцией \SHR{}: действительно, ведь освободившиеся биты в старшей части нужно
просто заполнить нулями}\EN{The higher part is shifted using the more popular \SHR{} instruction: indeed, the freed bits in the higher part
must be filled with zeroes}.

\ifdefined\IncludeARM
\subsection{ARM}

\RU{В ARM нет такой инструкции как SHRD в x86, так что компилятору Keil приходится всё это делать,
используя простые сдвиги и операции \q{ИЛИ}:}
\EN{ARM doesn't have such instruction as SHRD in x86, so the Keil compiler ought to do this using simple shifts
and OR operations:}

\lstinputlisting[caption=\OptimizingKeilVI (\ARMMode)]{patterns/185_64bit_in_32_env/shifting/Keil_ARM_O3.s}

\lstinputlisting[caption=\OptimizingKeilVI (\ThumbMode)]{patterns/185_64bit_in_32_env/shifting/Keil_thumb_O3.s}
% TODO add explanation
\fi

\ifdefined\IncludeMIPS
\subsection{MIPS}

\ifdefined\IncludeGCC
\EN{GCC for MIPS follows the same algorithm as Keil does for Thumb mode:}
\RU{GCC для MIPS реализует тот же алгоритм, что сделал Keil для режима Thumb:}

\lstinputlisting[caption=\Optimizing GCC 4.4.5 (IDA)]{patterns/185_64bit_in_32_env/shifting/MIPS_O3_IDA.lst}
\fi

% TODO add explanation
\fi
