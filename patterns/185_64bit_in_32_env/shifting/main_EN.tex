\section{Shifting right}

\lstinputlisting{patterns/185_64bit_in_32_env/shifting/3.c}

\subsection{x86}

\lstinputlisting[caption=\Optimizing MSVC 2012 /Ob1]{patterns/185_64bit_in_32_env/shifting/3_MSVC.asm}

\ifdefined\IncludeGCC
\lstinputlisting[caption=\Optimizing GCC 4.8.1 -fno-inline]{patterns/185_64bit_in_32_env/shifting/3_GCC.asm}
\fi

\myindex{x86!\Instructions!SHRD}

Shifting also occurs in two passes: first the lower part is shifted, then the higher part.
But the lower part is shifted with the help of the \INS{SHRD} instruction, it shifts the value of \EDX{} by 7 bits, but pulls new bits
from \EAX{}, i.e., from the higher part.
The higher part is shifted using the more popular \SHR{} instruction: indeed, the freed bits in the higher part
must be filled with zeroes.

\ifdefined\IncludeARM
\subsection{ARM}

ARM doesn't have such instruction as \INS{SHRD} in x86, so the Keil compiler ought to do this using simple shifts and \INS{OR} operations:

\lstinputlisting[caption=\OptimizingKeilVI (\ARMMode)]{patterns/185_64bit_in_32_env/shifting/Keil_ARM_O3.s}

\lstinputlisting[caption=\OptimizingKeilVI (\ThumbMode)]{patterns/185_64bit_in_32_env/shifting/Keil_thumb_O3.s}
% TODO add explanation
\fi

\ifdefined\IncludeMIPS
\subsection{MIPS}

\ifdefined\IncludeGCC
GCC for MIPS follows the same algorithm as Keil does for Thumb mode:

\lstinputlisting[caption=\Optimizing GCC 4.4.5 (IDA)]{patterns/185_64bit_in_32_env/shifting/MIPS_O3_IDA.lst}
\fi

% TODO add explanation
\fi

