\sectionold{Сдвиг вправо}

\lstinputlisting{patterns/185_64bit_in_32_env/shifting/3.c}

\subsectionold{x86}

\lstinputlisting[caption=\Optimizing MSVC 2012 /Ob1]{patterns/185_64bit_in_32_env/shifting/3_MSVC.asm}

\lstinputlisting[caption=\Optimizing GCC 4.8.1 -fno-inline]{patterns/185_64bit_in_32_env/shifting/3_GCC.asm}

\myindex{x86!\Instructions!SHRD}
Сдвиг происходит также в две операции: в начале сдвигается младшая часть, затем старшая.
Но младшая часть сдвигается при помощи инструкции \INS{SHRD}, она сдвигает значение в \EDX{} на 7 бит, но подтягивает новые биты из \EAX{}, т.е. из старшей части.
Старшая часть сдвигается более известной инструкцией \SHR{}: действительно, ведь освободившиеся биты в старшей части нужно
просто заполнить нулями.

\subsectionold{ARM}

В ARM нет такой инструкции как \INS{SHRD} в x86, так что компилятору Keil приходится всё это делать,
используя простые сдвиги и операции \q{ИЛИ}:

\lstinputlisting[caption=\OptimizingKeilVI (\ARMMode)]{patterns/185_64bit_in_32_env/shifting/Keil_ARM_O3.s}

\lstinputlisting[caption=\OptimizingKeilVI (\ThumbMode)]{patterns/185_64bit_in_32_env/shifting/Keil_thumb_O3.s}
% TODO add explanation

\subsectionold{MIPS}

GCC для MIPS реализует тот же алгоритм, что сделал Keil для режима Thumb:

\lstinputlisting[caption=\Optimizing GCC 4.4.5 (IDA)]{patterns/185_64bit_in_32_env/shifting/MIPS_O3_IDA.lst}

% TODO add explanation

