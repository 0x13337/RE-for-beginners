\chapter{SIMD}

\label{SIMD_x86}
\ac{SIMD} \RU{это акроним}\EN{is an acronym}: \IT{Single Instruction, Multiple Data}.

\RU{Как можно судить по названию, это обработка множества данных исполняя только одну инструкцию.}
\EN{As its name implies, it processes multiple data using only one instruction.}

\RU{Как и \ac{FPU}, эта подсистема процессора выглядит так же отдельным процессором внутри x86.}
\EN{Like the \ac{FPU}, that \ac{CPU} subsystem looks like a separate processor inside x86.}

\index{x86!MMX}
\RU{SIMD в x86 начался с MMX. Появилось 8 64-битных регистров MM0-MM7.}
\EN{SIMD began as MMX in x86. 8 new 64-bit registers appeared: MM0-MM7.}

\RU{Каждый MMX-регистр может содержать 2 32-битных значения, 4 16-битных или же 8 байт. 
Например, складывая значения двух MMX-регистров, можно складывать одновременно 8 8-битных значений.}
\EN{Each MMX register can hold 2 32-bit values, 4 16-bit values or 8 bytes.
For example, it is possible to add 8 8-bit values (bytes) simultaneously by adding two values in MMX registers.}

\RU{Простой пример, это некий графический редактор, который хранит открытое изображение как двумерный массив. 
Когда пользователь меняет яркость изображения, редактору нужно, например, прибавить некий коэффициент 
ко всем пикселям, или отнять. 
Для простоты можно представить, что изображение у нас бело-серо-черное и каждый пиксель занимает один байт, 
то с помощью MMX можно менять яркость сразу у восьми пикселей.}
\EN{One simple example is a graphics editor that represents an image as a two dimensional array.
When the user changes the brightness of the image, the editor must add or subtract a coefficient to/from each pixel value.
For the sake of brevity if we say that the image is grayscale and each pixel is defined by one 8-bit byte, then it is possible
to change the brightness of 8 pixels simultaneously.}
\RU{Кстати, вот причина почему в SIMD присутствуют инструкции с \IT{насыщением} (\IT{saturation}).}
\EN{By the way, this is the reason why the \IT{saturation} instructions are present in SIMD.}
\RU{Когда пользователь в графическом редакторе изменяет яркость, переполнение и антипереполнение (\IT{underflow})
не нужны, так что в SIMD имеются, например, инструкции сложения, которые ничего не будут прибавлять
если максимальное значение уже достигнуто, и т.д.}
\EN{When the user changes the brightness in the graphics editor, overflow and underflow is not desirable, 
so there are addition instructions in SIMD which will not add anything if the maximum value is reached, etc.}

\RU{Когда MMX только появилось, эти регистры на самом деле располагались в FPU-регистрах. 
Можно было использовать 
либо FPU либо MMX в одно и то же время. Можно подумать, что Intel решило немного сэкономить на транзисторах, 
но на самом деле причина такого симбиоза проще ~--- более старая \ac{OS} не знающая о дополнительных 
регистрах процессора не будет сохранять их во время переключения задач, а вот регистры FPU сохранять будет. 
Таким образом, процессор с MMX + старая \ac{OS} + задача, использующая возможности MMX = все 
это может работать вместе.}
\EN{When MMX appeared, these registers were actually located in the FPU's registers. 
It was possible to use either FPU or MMX at the same time. One might think that Intel saved on transistors,
but in fact the reason of such symbiosis was simpler~---older \ac{OS}es that are not aware 
of the additional CPU registers would not save them at the context switch, but will save the FPU registers.
Thus, MMX-enabled CPU + old \ac{OS} + process utilizing MMX features will still work.}

\index{x86!SSE}
\index{x86!SSE2}
SSE\EMDASH\RU{это расширение регистров до 128 бит, теперь уже отдельно от FPU.}\EN{is extension of the SIMD registers to 128 bits, now separate from the FPU.}

\index{x86!AVX}
AVX\EMDASH\RU{расширение регистров до 256 бит.}\EN{another extension, to 256 bits.}

\RU{Немного о практическом применении.}\EN{Now about practical usage.}

\RU{Конечно же, копирование блоков в памяти (\TT{memcpy}), сравнение (\TT{memcmp}), и подобное.}
\EN{Of course, memory copy routines (\TT{memcpy}), memory comparing (\TT{memcmp}) and so on.}

\index{DES}
\RU{Еще пример: имеется алгоритм шифрования DES, который берет 64-битный блок, 56-битный ключ, 
шифрует блок с ключом и образуется 64-битный результат.
Алгоритм DES можно легко представить в виде очень большой электронной цифровой схемы, 
с проводами, элементами И, ИЛИ, НЕ.}
\EN{One more example: the DES encryption algorithm takes a 64-bit block and a 56-bit key, encrypt the block and produces a 64-bit result.
The DES algorithm may be considered as a very large electronic circuit, with wires and AND/OR/NOT gates.}

\label{bitslicedes}
\newcommand{\URLBS}{\url{http://go.yurichev.com/17329}}

\RU{Идея bitslice DES\footnote{\URLBS} ~--- это обработка сразу группы блоков и ключей одновременно. 
Скажем, на x86 переменная типа \IT{unsigned int} вмещает в себе 32 бита, так что там можно хранить 
промежуточные результаты сразу для 32-х блоков-ключей, используя 64+56 переменных типа \IT{unsigned int}.}
\EN{Bitslice DES\footnote{\URLBS}~---is the idea of processing groups of blocks and keys simultaneously.
Let's say, variable of type \IT{unsigned int} on x86 can hold up to 32 bits, so it is possible to store there
intermediate results for 32 block-key pairs simultaneously, using 64+56 variables of type \IT{unsigned int}.}

\index{\oracle}
\RU{Я написал утилиту для перебора паролей/хешей \oracle (которые основаны на алгоритме DES), 
переделав алгоритм bitslice DES для SSE2 и AVX ~--- и теперь возможно шифровать одновременно 
128 или 256 блоков-ключей:}
\EN{I wrote an utility to brute-force \oracle passwords/hashes (ones based on DES),
using slightly modified bitslice DES algorithm for SSE2 and AVX~---now it is possible to encrypt 128 
or 256 block-keys pairs simultaneously.}

\url{http://go.yurichev.com/17313}

% sections
\section{\RU{Векторизация}\EN{Vectorization}}

\newcommand{\URLVEC}{\href{http://en.wikipedia.org/wiki/Vectorization_(computer_science)}{Wikipedia: vectorization}}

\RU{Векторизация\footnote{\URLVEC} это когда у вас есть цикл, который берет на вход несколько массивов и выдает, 
например, один массив данных. 
Тело цикла берет некоторые элементы из входных массивов, что-то делает с ними и помещает в выходной. 
Важно, что операция применяемая ко всем элементам одна и та же. 
Векторизация ~--- это обрабатывать несколько элементов одновременно.}
\EN{Vectorization\footnote{\URLVEC}, for example, is when you have a loop taking couple of arrays at input and produces one array.
Loop body takes values from input arrays, do something and put result into output array.
It is important that there is only one single operation applied to each element.
Vectorization~---is to process several elements simultaneously.}

\RU{Векторизация ~--- это не самая новая технология: автор сих строк видел её по крайней мере на 
линейке суперкомпьютеров Cray Y-MP от 1988, когда работал на его версии-``лайт'' Cray Y-MP EL
\footnote{Удаленно. Он находится в музее суперкомпьютеров: \url{http://www.cray-cyber.org}}}
\EN{Vectorization is not very fresh technology: author of this textbook saw it at least on Cray Y-MP 
supercomputer line from 1988 when played with its ``lite'' version Cray Y-MP EL
\footnote{Remotely. It is installed in the museum of supercomputers: \url{http://www.cray-cyber.org}}}.

\RU{Например:}\EN{For example:}

\begin{lstlisting}
for (i = 0; i < 1024; i++)
{
    C[i] = A[i]*B[i];
}
\end{lstlisting}

\RU{Этот фрагмент кода берет элементы из A и B, перемножает и сохраняет результат в C.}
\EN{This fragment of code takes elements from A and B, multiplies them and save result into C.}

\index{x86!\Instructions!PLMULLD}
\index{x86!\Instructions!PLMULHW}
\newcommand{\PMULLD}{\IT{PMULLD} (\IT{\RU{Перемножить запакованные знаковые DWORD и сохранить младшую часть результата}
\EN{Multiply Packed Signed Dword Integers and Store Low Result}})}
\newcommand{\PMULHW}{\TT{PMULHW} (\IT{\RU{Перемножить запакованные знаковые DWORD и сохранить старшую часть результата}
\EN{Multiply Packed Signed Integers and Store High Result}})}

\RU{Если представить, что каждый элемент массива ~--- это 32-битный \Tint, то их можно загружать сразу 
по 4 из А в 128-битный XMM-регистр, 
из B в другой XMM-регистр и выполнив инструкцию \PMULLD{} и \PMULHW{}, можно получить 4 64-битных 
\glslink{product}{произведения} сразу.}
\EN{If each array element we have is 32-bit \Tint, then it is possible to load 4 elements from A into 128-bit 
XMM-register, from B to another XMM-registers, and by executing \PMULLD{} and \PMULHW{}, 
it is possible to get 4 64-bit \glspl{product} at once.}

\RU{Таким образом, тело цикла исполняется $1024/4$ раза вместо 1024, что в 4 раза меньше, и, конечно, быстрее.}
\EN{Thus, loop body count is $1024/4$ instead of $1024$, that is 4 times less and, of course, faster.}

\newcommand{\URLINTELVEC}{\href{http://www.intel.com/intelpress/sum_vmmx.htm}{Excerpt: Effective Automatic Vectorization}}

\subsection{\RU{Пример сложения}\EN{Addition example}}

\index{Intel C++}
\RU{Некоторые компиляторы умеют делать автоматическую векторизацию в простых случаях, 
например Intel C++\footnote{Еще о том, как Intel C++ умеет автоматически векторизовать циклы: \URLINTELVEC}.}
\EN{Some compilers can do vectorization automatically in a simple cases, 
e.g., Intel C++\footnote{More about Intel C++ automatic vectorization: \URLINTELVEC}.}

\RU{Я написал очень простую функцию:}\EN{I wrote tiny function:}

\begin{lstlisting}
int f (int sz, int *ar1, int *ar2, int *ar3)
{
	for (int i=0; i<sz; i++)
		ar3[i]=ar1[i]+ar2[i];

	return 0;
};
\end{lstlisting}

\subsubsection{Intel C++}

\RU{Компилирую при помощи}\EN{Let's compile it with} Intel C++ 11.1.051 win32:

\begin{verbatim}
icl intel.cpp /QaxSSE2 /Faintel.asm /Ox
\end{verbatim}

\RU{Имеем такое (в \IDA):}\EN{We got (in \IDA):}

\lstinputlisting{patterns/19_SIMD/18_1_en.asm}

\RU{Инструкции, имеющие отношение к SSE2 это:}\EN{SSE2-related instructions are:}
\index{x86!\Instructions!MOVDQA}
\index{x86!\Instructions!MOVDQU}
\index{x86!\Instructions!PADDD}
\begin{itemize}
\item
\MOVDQU (\IT{Move Unaligned Double Quadword})\EMDASH\RU{она просто загружает 16 байт из памяти в XMM-регистр}
\EN{it just load 16 bytes from memory into a XMM-register}.

\item
\PADDD (\IT{Add Packed Integers})\EMDASH\RU{складывает сразу 4 пары 32-битных чисел и оставляет в первом операнде результат. 
Кстати, если произойдет переполнение, то исключения не произойдет и никакие флаги не установятся, 
запишутся просто младшие 32 бита результата. 
Если один из операндов \PADDD ~--- адрес значения в памяти, 
то требуется чтобы адрес был выровнен по 16-байтной границе. Если он не выровнен, произойдет исключение
\footnote{О выравнивании данных см. также: \URLWPDA}.}
\EN{adding 4 pairs of 32-bit numbers and leaving result in first operand.
By the way, no exception raised in case of overflow and no flags will be set, just low 32-bit of result will
be stored.
If one of \PADDD operands is address of value in memory,
then address must be aligned on a 16-byte boundary. If it is not aligned, exception will be occurred
\footnote{More about data aligning: \URLWPDA}.}

\item
\MOVDQA (\IT{Move Aligned Double Quadword})\EMDASH\RU{тоже что и \MOVDQU, только подразумевает 
что адрес в памяти выровнен по 16-байтной границе. 
Если он не выровнен, произойдет исключение. 
\MOVDQA работает быстрее чем \MOVDQU, но требует вышеозначенного.}
\EN{the same as \MOVDQU, but requires address of value in memory to be aligned on a 16-bit border.
If it is not aligned, exception will be raised.
\MOVDQA works faster than \MOVDQU, but requires aforesaid.}

\end{itemize}

\RU{Итак, эти SSE2-инструкции исполнятся только в том случае если еще осталось просуммировать 
4 пары переменных типа \Tint плюс если указатель \TT{ar3} выровнен по 16-байтной границе.}
\EN{So, these SSE2-instructions will be executed only in case if there are more 4 pairs to work on
plus pointer \TT{ar3} is aligned on a 16-byte boundary.}

\RU{Более того, если еще и \TT{ar2} выровнен по 16-байтной границе, то будет выполняться этот фрагмент кода:}
\EN{More than that, if \TT{ar2} is aligned on a 16-byte boundary as well, this fragment of code will be executed:}

\begin{lstlisting}
movdqu  xmm0, xmmword ptr [ebx+edi*4] ; ar1+i*4
paddd   xmm0, xmmword ptr [esi+edi*4] ; ar2+i*4
movdqa  xmmword ptr [eax+edi*4], xmm0 ; ar3+i*4
\end{lstlisting}

\RU{А иначе, значение из \TT{ar2} загрузится в \XMM{0} используя инструкцию \MOVDQU, 
которая не требует выровненного указателя, зато может работать чуть медленнее:}
\EN{Otherwise, value from \TT{ar2} will be loaded into \XMM{0} using \MOVDQU,
it does not require aligned pointer, but may work slower:}

\begin{lstlisting}
movdqu  xmm1, xmmword ptr [ebx+edi*4] ; ar1+i*4
movdqu  xmm0, xmmword ptr [esi+edi*4] ; ar2+i*4 is not 16-byte aligned, so load it to xmm0
paddd   xmm1, xmm0
movdqa  xmmword ptr [eax+edi*4], xmm1 ; ar3+i*4
\end{lstlisting}

\RU{А во всех остальных случаях, будет исполняться код, который был бы, как если бы не была 
включена поддержка SSE2.}
\EN{In all other cases, non-SSE2 code will be executed.}

\subsubsection{GCC}

\newcommand{\URLGCCVEC}{\url{http://gcc.gnu.org/projects/tree-ssa/vectorization.html}}

\RU{Но и GCC умеет кое-что векторизировать\footnote{Подробнее о векторизации в GCC: \URLGCCVEC}, 
если компилировать с опциями \Othree и включить поддержку SSE2: \TT{-msse2}.}
\EN{GCC may also vectorize in a simple cases\footnote{More about GCC vectorization support: \URLGCCVEC},
if to use \Othree option and to turn on SSE2 support: \TT{-msse2}.}

\RU{Вот что вышло}\EN{What we got} (GCC 4.4.1):

\lstinputlisting{patterns/19_SIMD/18_2_gcc_O3.asm}

\RU{Почти то же самое, хотя и не так дотошно как Intel C++.}
\EN{Almost the same, however, not as meticulously as Intel C++ doing it.}

\ifx\RUSSIAN\undefined
\subsection{\RU{Пример копирования блоков}\EN{Memory copy example}}
\label{vec_memcpy}

Let's revisit simple memcpy() example (\ref{loop_memcpy}):

\lstinputlisting{memcpy.c}

And that's what optimizing GCC 4.9.1 did:

\lstinputlisting[caption=\Optimizing GCC 4.9.1 x64]{patterns/19_SIMD/memcpy_GCC49_x64_O3.s}
\fi

\section{\RU{Реализация \strlen при помощи SIMD}\EN{SIMD \strlen implementation}}

\newcommand{\URLMSDNSSE}{\href{http://go.yurichev.com/17262}{MSDN: MMX, SSE, and SSE2 Intrinsics}}

\RU{Прежде всего, следует заметить, что SIMD-инструкции можно вставлять в \CCpp код при помощи специальных 
макросов\footnote{\URLMSDNSSE}. В MSVC, часть находится в файле \TT{intrin.h}.}
\EN{It should be noted that the \ac{SIMD} instructions can be inserted in \CCpp code via 
special macros\footnote{\URLMSDNSSE}.
For MSVC, some of them are located in the \TT{intrin.h} file.}

\newcommand{\URLSTRLEN}{http://go.yurichev.com/17330}

\index{\CStandardLibrary!strlen()}
\RU{Имеется возможность реализовать функцию \strlen\footnote{strlen() ~--- стандартная функция Си 
для подсчета длины строки} при помощи SIMD-инструкций, работающий в 2-2.5 раза быстрее обычной реализации. 
Эта функция будет загружать в XMM-регистр сразу 16 байт и проверять каждый на ноль}
\EN{It is possible to implement the \strlen function\footnote{strlen()~---standard C library function for calculating
string length} using SIMD instructions that works 2-2.5 times faster than the common implementation.
This function will load 16 characters into a XMM-register and check each against zero}
\footnote{\RU{Пример базируется на исходнике отсюда: \url{\URLSTRLEN}.}
\EN{The example is based on source code from: \url{\URLSTRLEN}.}}.

\lstinputlisting{patterns/19_SIMD/18_3.c}

\RU{Компилируем в MSVC 2010 с опцией \Ox:}\EN{Let's compile it in MSVC 2010 with \Ox option:}

\lstinputlisting[caption=\Optimizing MSVC 2010]{patterns/19_SIMD/18_4_msvc_Ox.asm.\LANG}

\RU{Итак, прежде всего, мы проверяем указатель \TT{str}, выровнен ли он по 16-байтной границе. 
Если нет, то мы вызовем обычную реализацию \strlen.}
\EN{First, we check if the \TT{str} pointer is aligned on a 16-byte boundary.
If not, we call the generic \strlen implementation.}

\RU{Далее мы загружаем по 16 байт в регистр \XMM{1} при помощи команды \MOVDQA.}
\EN{Then, we load the next 16 bytes into the \XMM{1} register using \MOVDQA.}

\RU{Наблюдательный читатель может спросить, почему в этом месте мы не можем использовать \MOVDQU, 
которая может загружать откуда угодно не взирая на факт, выровнен ли указатель?}
\EN{An observant reader might ask, why can't \MOVDQU be used here since it can load data from the memory
regardless pointer alignment?}

\RU{Да, можно было бы сделать вот как: если указатель выровнен, загружаем используя \MOVDQA, 
иначе используем работающую чуть медленнее \MOVDQU.}
\EN{Yes, it might be done in this way: if the pointer is aligned, load data using \MOVDQA,
if not~---use the slower \MOVDQU.}

\RU{Однако здесь кроется не сразу заметная проблема, которая проявляется вот в чем:}
\EN{But here we are may hit another caveat:}

\index{Page (memory)}
\newcommand{\URLPAGE}{\href{http://go.yurichev.com/17136}{wikipedia}}

\RU{В \ac{OS} линии \gls{Windows NT} (и не только), память выделяется страницами по 4 KiB (4096 байт). 
Каждый win32-процесс якобы имеет в наличии 4 GiB, но на самом деле, 
только некоторые части этого адресного пространства присоединены к реальной физической памяти. 
Если процесс обратится к блоку памяти, которого не существует, сработает исключение. 
Так работает \ac{VM}\footnote{\URLPAGE}.}
\EN{In the \gls{Windows NT} line of \ac{OS} (but not limited to it), memory is allocated by pages of 4 KiB (4096 bytes).
Each win32-process has 4 GiB available, but in fact, only some parts
of the address space are connected to real physical memory.
If the process is accessing an absent memory block, an exception will be raised.
That's how \ac{VM} works\footnote{\URLPAGE}.}

\RU{Так вот, функция, читающая сразу по 16 байт, имеет возможность нечаянно вылезти за границу 
выделенного блока памяти. 
Предположим, \ac{OS} выделила программе 8192 (0x2000) байт по адресу 0x008c0000. 
Таким образом, блок занимает байты с адреса 0x008c0000 по 0x008c1fff включительно.}
\EN{So, a function loading 16 bytes at once  may step over the border of an allocated memory block.
Let's say that the \ac{OS} has allocated 8192 (0x2000) bytes at address 0x008c0000.
Thus, the block is the bytes starting from address 0x008c0000 to 0x008c1fff inclusive.}

\RU{За этим блоком, то есть начиная с адреса 0x008c2000 нет вообще ничего, т.е., \ac{OS} не выделяла там память. 
Обращение к памяти начиная с этого адреса вызовет исключение.}
\EN{After the block, that is, starting from address 0x008c2000 there is nothing at all, e.g. the \ac{OS} not allocated
any memory there.
Any attempt to access memory starting from that address will raise an exception.}

\RU{И предположим, что программа хранит некую строку из, скажем, пяти символов почти в самом конце блока, 
что не является преступлением:}
\EN{And let's consider the example in which the program is holding a string that contains 5 characters almost at the end of a block,
and that is not a crime.}

\begin{center}
  \begin{tabular}{ | l | l | }
    \hline
        0x008c1ff8 & 'h' \\
        0x008c1ff9 & 'e' \\
        0x008c1ffa & 'l' \\
        0x008c1ffb & 'l' \\
        0x008c1ffc & 'o' \\
        0x008c1ffd & '\textbackslash{}x00' \\
        0x008c1ffe & \RU{здесь случайный мусор}\EN{random noise} \\
        0x008c1fff & \RU{здесь случайный мусор}\EN{random noise} \\
    \hline
  \end{tabular}
\end{center}

\RU{В обычных условиях, программа вызывает \strlen передав ей указатель на строку \TT{'hello'} 
лежащую по адресу 0x008c1ff8. 
\strlen будет читать по одному байту до 0x008c1ffd, где ноль, и здесь она закончит работу.}
\EN{So, in normal conditions the program calls \strlen, passing it a pointer to the string \TT{'hello'} 
placed in memory at address 0x008c1ff8.
\strlen will read one byte at a time until 0x008c1ffd, where there's a zero byte, and it will stop working.}

\RU{Теперь, если мы напишем свою реализацию \strlen читающую сразу по 16 байт, с любого адреса, 
будь он выровнен по 16-байтной границе или нет, 
\MOVDQU попытается загрузить 16 байт с адреса 0x008c1ff8 по 0x008c2008, и произойдет исключение. 
Это ситуация которой, конечно, хочется избежать.}
\EN{Now if we implement our own \strlen reading 16 byte at once, starting at any address, aligned or not,
\MOVDQU may attempt to load 16 bytes at once at address 0x008c1ff8 up to 0x008c2008, 
and then an exception will be raised.
That situation is to be avoided, of course.}

\RU{Поэтому мы будем работать только с адресами, выровненными по 16 байт, что в сочетании со знанием 
что размер страницы \ac{OS} также, как правило, выровнен по 16 байт, 
даст некоторую гарантию что наша функция не будет пытаться читать из мест в невыделенной памяти.}
\EN{So then we'll work only with the addresses aligned on a 16 byte boundary, which in combination with the knowledge
that the \ac{OS}' page size is usually aligned on a 16-byte boundary gives us some warranty that our function will not
read from unallocated memory.}

\RU{Вернемся к нашей функции}\EN{Let's get back to our function}.

\index{x86!\Instructions!PXOR}
\verb|_mm_setzero_si128()|\EMDASH\RU{это макрос, генерирующий \TT{pxor xmm0, xmm0} ~--- инструкция просто обнуляет регистр \XMM{0}.}
\EN{is a macro generating \TT{pxor xmm0, xmm0}~---it just clears the \XMM{0} register}.

\verb|_mm_load_si128()|\EMDASH\RU{это макрос для \MOVDQA, он просто загружает 16 байт по адресу из указателя в \XMM{1}.}
\EN{is a macro for \MOVDQA, it just loads 16 bytes from the address into the \XMM{1} register.}

\index{x86!\Instructions!PCMPEQB}
\verb|_mm_cmpeq_epi8()|\EMDASH\RU{это макрос для \PCMPEQB, это инструкция, которая 
побайтово сравнивает значения из двух XMM регистров.} 
\EN{is a macro for \PCMPEQB, an instruction that compares two XMM-registers bytewise.}

\RU{И если какой-то из байт равен другому, то в результирующем значении будет выставлено на месте этого 
байта \TT{0xff}, либо 0, если байты не были равны.}
\EN{And if some byte was equals to the one in the other register, there will be \TT{0xff} at this point in the result or 0 if otherwise.}

\RU{Например.}\EN{For example.}

\begin{verbatim}
XMM1: 11223344556677880000000000000000
XMM0: 11ab3444007877881111111111111111
\end{verbatim}

\RU{После исполнения \TT{pcmpeqb xmm1, xmm0}, регистр \XMM{1} будет содержать:}
\EN{After the execution of \TT{pcmpeqb xmm1, xmm0}, the \XMM{1} register will contain:}

\begin{verbatim}
XMM1: ff0000ff0000ffff0000000000000000
\end{verbatim}

\RU{Эта инструкция в нашем случае, сравнивает каждый 16-байтный блок с блоком состоящим из 16-и нулевых байт, 
выставленным в \XMM{0} при помощи \TT{pxor xmm0, xmm0}.}
\EN{In our case, this instruction compares each 16-byte block with a block of 16 zero-bytes,
which was set in the \XMM{0} register by \TT{pxor xmm0, xmm0}.}

\index{x86!\Instructions!PMOVMSKB}
\RU{Следующий макрос \TT{\_mm\_movemask\_epi8()} ~--- это инструкция \TT{PMOVMSKB}.}
\EN{The next macro is \TT{\_mm\_movemask\_epi8()}~---that is the \TT{PMOVMSKB} instruction.}

\RU{Она очень удобна как раз для использования в паре с \PCMPEQB.}
\EN{It is very useful with \PCMPEQB.}

\TT{pmovmskb eax, xmm1}

\RU{Эта инструкция выставит самый первый бит \EAX в единицу, если старший бит первого байта в 
регистре \XMM{1} является единицей. 
Иными словами, если первый байт в регистре \XMM{1} является \TT{0xff}, то первый бит в \EAX будет также единицей, 
иначе нулем.}
\EN{This instruction will set first \EAX bit to 1 if the most significant bit of the first byte in \XMM{1} is $1$.
In other words, if the first byte of the \XMM{1} register is \TT{0xff}, the first bit of \EAX will be set to 1, too.}

\RU{Если второй байт в регистре \XMM{1} является \TT{0xff}, то второй бит в \EAX также будет единицей. 
Иными словами, инструкция отвечает на вопрос, ``какие из байт в \XMM{1} являются \TT{0xff}?''
В результате приготовит 16 бит и запишет в \EAX. Остальные биты в \EAX обнулятся.}
\EN{If the second byte in the \XMM{1} register is \TT{0xff}, then the second bit in \EAX will be set to 1.
In other words, the instruction is answering the question ``which bytes in \XMM{1} are \TT{0xff}?''
and will return 16 bits in the \EAX register. The other bits in the \EAX register will be cleared.}

\RU{Кстати, не забывайте также вот о какой особенности нашего алгоритма.}
\EN{By the way, do not forget about this quirk of our algorithm.}
\RU{На вход может прийти 16 байт вроде}\EN{There might be 16 bytes in the input like}:

\begin{center}
\begin{bytefield}[endianness=big,bitwidth=0.05\linewidth]{16}
\bitheader{15,14,13,12,11,10,9,3,2,1,0} \\
\bitbox{1}{``h''} & 
\bitbox{1}{``e''} & 
\bitbox{1}{``l''} & 
\bitbox{1}{``l''} & 
\bitbox{1}{``o''} & 
\bitbox{1}{0} & 
\bitbox{7}{\garbage{}} & 
\bitbox{1}{0} &
\bitbox{2}{\garbage{}} & 
\end{bytefield}
\end{center}

\RU{Это строка \TT{'hello'}, после нее терминирующий ноль, затем немного мусора в памяти.}
\EN{It is the \TT{'hello'} string, terminating zero, and some random noise in memory.}
\RU{Если мы загрузим эти 16 байт в \XMM{1} и сравним с нулевым \XMM{0}, то в итоге получим такое 
\footnote{Я использую здесь порядок с \ac{MSB} до \ac{LSB}}:}
\EN{If we load these 16 bytes into \XMM{1} and compare them with the zeroed \XMM{0}, we will get something like
\footnote{I use here order from \ac{MSB} to \ac{LSB}}:}

\begin{verbatim}
XMM1: 0000ff00000000000000ff0000000000
\end{verbatim}

\RU{Это означает что инструкция сравнения обнаружила два нулевых байта, что и не удивительно.}
\EN{This means that the instruction found two zero bytes, and it is not surprising.}

\RU{\TT{PMOVMSKB} в нашем случае подготовит \EAX вот так (в двоичном представлении):} 
\EN{\TT{PMOVMSKB} in our case will set \EAX to (in binary representation):} \IT{0010000000100000b}.

\RU{Совершенно очевидно, что далее наша функция должна учитывать только первый встретившийся
нулевой бит и игнорировать все остальное.}
\EN{Obviously, our function must take only the first zero bit and ignore the rest.}

\index{x86!\Instructions!BSF}
\label{instruction_BSF}
\RU{Следующая инструкция\EMDASH}\EN{The next instruction is }\TT{BSF} (\IT{Bit Scan Forward}). 
\RU{Это инструкция находит самый младший бит во втором операнде и записывает его позицию в первый операнд.}
\EN{This instruction finds the first bit set to 1 and stores its position into the first operand.}

\begin{verbatim}
EAX=0010000000100000b
\end{verbatim}

\RU{После исполнения этой инструкции \TT{bsf eax, eax}, в \EAX будет 5, что означает, 
что единица найдена в пятой позиции (считая с нуля).}
\EN{After the execution of \TT{bsf eax, eax}, \EAX will contain 5, meaning 
1 was found at the 5th bit position (starting from zero).}

\RU{Для использования этой инструкции, в MSVC также имеется макрос}
\EN{MSVC has a macro for this instruction:} \TT{\_BitScanForward}.

\RU{А дальше все просто. Если нулевой байт найден, его позиция прибавляется к тому что 
мы уже насчитали и возвращается результат.}
\EN{Now it is simple. If a zero byte was found, its position is added to what we have already counted and now we have 
the return result.}

\RU{Почти всё.}\EN{Almost all.}

\RU{Кстати, следует также отметить, что компилятор MSVC сгенерировал два тела цикла сразу, для оптимизации.}
\EN{By the way, it is also should be noted that the MSVC compiler emitted two loop bodies side by side, for optimization.}

\RU{Кстати, в SSE 4.2 (который появился в Intel Core i7) все эти манипуляции со строками могут быть еще проще:}
\EN{By the way, SSE 4.2 (that appeared in Intel Core i7) offers more instructions where these string manipulations might be
even easier:} \url{http://go.yurichev.com/17331}

