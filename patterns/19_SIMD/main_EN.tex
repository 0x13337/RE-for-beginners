\chapterold{SIMD}

\label{SIMD_x86}
\ac{SIMD} is an acronym: \IT{Single Instruction, Multiple Data}.

As its name implies, it processes multiple data using only one instruction.

Like the \ac{FPU}, that \ac{CPU} subsystem looks like a separate processor inside x86.

\myindex{x86!MMX}

SIMD began as MMX in x86. 8 new 64-bit registers appeared: MM0-MM7.

Each MMX register can hold 2 32-bit values, 4 16-bit values or 8 bytes.
For example, it is possible to add 8 8-bit values (bytes) simultaneously by adding two values in MMX registers.

One simple example is a graphics editor that represents an image as a two dimensional array.
When the user changes the brightness of the image, the editor must add or subtract a coefficient to/from each pixel value.
For the sake of brevity if we say that the image is grayscale and each pixel is defined by one 8-bit byte, then it is possible
to change the brightness of 8 pixels simultaneously.

By the way, this is the reason why the \IT{saturation} instructions are present in SIMD.

When the user changes the brightness in the graphics editor, overflow and underflow are not desirable, 
so there are addition instructions in SIMD which are not adding anything if the maximum value is reached, etc.

When MMX appeared, these registers were actually located in the FPU's registers. 
It was possible to use either FPU or MMX at the same time. One might think that Intel saved on transistors,
but in fact the reason of such symbiosis was simpler~---older \ac{OS}es that are not aware 
of the additional CPU registers would not save them at the context switch, 
but saving the FPU registers.
Thus, MMX-enabled CPU + old \ac{OS} + process utilizing MMX features will still work.

\myindex{x86!SSE}
\myindex{x86!SSE2}
SSE---is extension of the SIMD registers to 128 bits, now separate from the FPU.

\myindex{x86!AVX}
AVX---another extension, to 256 bits.

Now about practical usage.

Of course, this is memory copy routines (\TT{memcpy}), memory comparing (\TT{memcmp}) and so on.

\myindex{DES}

One more example: the DES encryption algorithm takes a 64-bit block and a 56-bit key, encrypt the block and produces a 64-bit result.
The DES algorithm may be considered as a very large electronic circuit, with wires and AND/OR/NOT gates.

\label{bitslicedes}
\newcommand{\URLBS}{\url{http://go.yurichev.com/17329}}

Bitslice DES\footnote{\URLBS}~---is the idea of processing groups of blocks and keys simultaneously.
Let's say, variable of type \IT{unsigned int} on x86 can hold up to 32 bits, so it is possible to store there
intermediate results for 32 block-key pairs simultaneously, using 64+56 variables of type \IT{unsigned int}.

\myindex{\oracle}
There is an utility to brute-force \oracle passwords/hashes (ones based on DES),
using slightly modified bitslice DES algorithm for SSE2 and AVX---now it is possible to encrypt 128 
or 256 block-keys pairs simultaneously.

\url{http://go.yurichev.com/17313}

% sections
\sectionold{Vectorization}

\newcommand{\URLVEC}{\href{http://go.yurichev.com/17080}{Wikipedia: vectorization}}

Vectorization\footnote{\URLVEC} is when, for example, you have a loop taking couple of arrays for input and producing one array.
The loop body takes values from the input arrays, does something and puts the result into the output array.
%It is important that there is only a single operation applied to each element.
Vectorization is to process several elements simultaneously.

Vectorization is not very fresh technology: the author of this textbook saw it at least on the Cray Y-MP 
supercomputer line from 1988 when he played with its \q{lite} version Cray Y-MP EL
\footnote{Remotely. It is installed in the museum of supercomputers: \url{http://go.yurichev.com/17081}}.

% FIXME! add assembly listing!
For example:

\begin{lstlisting}
for (i = 0; i < 1024; i++)
{
    C[i] = A[i]*B[i];
}
\end{lstlisting}

This fragment of code takes elements from A and B, multiplies them and saves the result into C.

\myindex{x86!\Instructions!PLMULLD}
\myindex{x86!\Instructions!PLMULHW}
\newcommand{\PMULLD}{\IT{PMULLD} (\IT{Multiply Packed Signed Dword Integers and Store Low Result})}
\newcommand{\PMULHW}{\TT{PMULHW} (\IT{Multiply Packed Signed Integers and Store High Result})}

If each array element we have is 32-bit \Tint, then it is possible to load 4 elements from A into a 128-bit 
XMM-register, from B to another XMM-registers, and by executing \PMULLD{} and \PMULHW{}, 
it is possible to get 4 64-bit \glspl{product} at once.

Thus, loop body execution count is $1024/4$ instead of 1024, that is 4 times less and, of course, faster.

\newcommand{\URLINTELVEC}{\href{http://go.yurichev.com/17082}{Excerpt: Effective Automatic Vectorization}}

\subsectionold{Addition example}

\myindex{Intel C++}

Some compilers can do vectorization automatically in simple cases, 
e.g., Intel C++\footnote{More about Intel C++ automatic vectorization: \URLINTELVEC}.

Here is tiny function:

\begin{lstlisting}
int f (int sz, int *ar1, int *ar2, int *ar3)
{
	for (int i=0; i<sz; i++)
		ar3[i]=ar1[i]+ar2[i];

	return 0;
};
\end{lstlisting}

\subsubsectionold{Intel C++}

Let's compile it with Intel C++ 11.1.051 win32:

\begin{verbatim}
icl intel.cpp /QaxSSE2 /Faintel.asm /Ox
\end{verbatim}

We got (in \IDA):

\lstinputlisting{patterns/19_SIMD/18_1_EN.asm}

The SSE2-related instructions are:
\myindex{x86!\Instructions!MOVDQA}
\myindex{x86!\Instructions!MOVDQU}
\myindex{x86!\Instructions!PADDD}
\begin{itemize}
\item
\MOVDQU (\IT{Move Unaligned Double Quadword})---just loads 16 bytes from memory into a XMM-register.

\item
\PADDD (\IT{Add Packed Integers})---adds 4 pairs of 32-bit numbers and leaves the result in the first operand.
By the way, no exception is raised in case of overflow and no flags are to be set, 
just the low 32 bits of the result are to be stored.
If one of \PADDD's operands is the address of a value in memory,
then the address must be aligned on a 16-byte boundary. 
If it is not aligned, an exception will be triggered
\footnote{More about data alignment: \URLWPDA}.

\item
\MOVDQA (\IT{Move Aligned Double Quadword})
is the same as \MOVDQU, but requires the address of the value in memory to be aligned on a 16-bit boundary.
If it is not aligned, exception will be raised.
\MOVDQA works faster than \MOVDQU, but requires aforesaid.

\end{itemize}

So, these SSE2-instructions are to be executed only in case there are more than 4 pairs to work on
and the pointer \TT{ar3} is aligned on a 16-byte boundary.

Also, if \TT{ar2} is aligned on a 16-byte boundary as well, 
this fragment of code is to be executed:

\begin{lstlisting}
movdqu  xmm0, xmmword ptr [ebx+edi*4] ; ar1+i*4
paddd   xmm0, xmmword ptr [esi+edi*4] ; ar2+i*4
movdqa  xmmword ptr [eax+edi*4], xmm0 ; ar3+i*4
\end{lstlisting}

Otherwise, the value from \TT{ar2} is to be loaded into \XMM{0} using \MOVDQU,
which does not require aligned pointer, but may work slower:

\lstinputlisting{patterns/19_SIMD/18_1_excerpt_EN.asm}

In all other cases, non-SSE2 code is to be executed.

\subsubsectionold{GCC}

\newcommand{\URLGCCVEC}{\url{http://go.yurichev.com/17083}}

GCC may also vectorize in simple cases\footnote{More about GCC vectorization support: \URLGCCVEC},
if the \Othree option is used and SSE2 support is turned on: \TT{-msse2}.

What we get (GCC 4.4.1):

\lstinputlisting{patterns/19_SIMD/18_2_gcc_O3.asm}

Almost the same, however, not as meticulously as Intel C++.

\subsectionold{Memory copy example}
\label{vec_memcpy}

Let's revisit the simple memcpy() example
(\myref{loop_memcpy}):

\lstinputlisting{memcpy.c}

And that's what optimizations GCC 4.9.1 did:

\lstinputlisting[caption=\Optimizing GCC 4.9.1 x64]{patterns/19_SIMD/memcpy_GCC49_x64_O3_EN.s}

\subsection{SIMD \strlen implementation}
\label{SIMD_strlen}

\newcommand{\URLMSDNSSE}{\href{http://go.yurichev.com/17262}{MSDN: MMX, SSE, and SSE2 Intrinsics}}

It has to be noted that the \ac{SIMD} instructions can be inserted in \CCpp code via special macros\footnote{\URLMSDNSSE}.
For MSVC, some of them are located in the \TT{intrin.h} file.

\newcommand{\URLSTRLEN}{http://go.yurichev.com/17330}

\myindex{\CStandardLibrary!strlen()}

It is possible to implement the \strlen function\footnote{strlen()~---standard C library function for calculating
string length} using SIMD instructions that works 2-2.5 times faster than the common implementation.
This function loads 16 characters into a XMM-register and check each against zero
\footnote{
The example is based on source code from: \url{\URLSTRLEN}.}.

\lstinputlisting{patterns/19_SIMD/18_3.c}

Let's compile it in MSVC 2010 with \Ox option:

\lstinputlisting[caption=\Optimizing MSVC 2010]{patterns/19_SIMD/18_4_msvc_Ox_EN.asm}

How it works?
First of all, we need to understand goal of the function.
It calculates C-string length, but we can use different terms: it's task is searching for zero byte, and then calculating its position relatively to string start.

First, we check if the \TT{str} pointer is aligned on a 16-byte boundary.
If not, we call the generic \strlen implementation.

Then, we load the next 16 bytes into the \XMM{1} register using \MOVDQA.

An observant reader might ask, why can't \MOVDQU be used here since it can load data from the memory
regardless pointer alignment?

Yes, it might be done in this way: if the pointer is aligned, load data using \MOVDQA,
if not~---use the slower \MOVDQU.

But here we are may hit another caveat:

\myindex{Page (memory)}
\newcommand{\URLPAGE}{\href{http://go.yurichev.com/17136}{wikipedia}}

In the \gls{Windows NT} line of \ac{OS} (but not limited to it), memory is allocated by pages of 4 KiB (4096 bytes).
Each win32-process has 4 GiB available, but in fact, only some parts
of the address space are connected to real physical memory.
If the process is accessing an absent memory block, an exception is to be raised.
That's how \ac{VM} works\footnote{\URLPAGE}.

So, a function loading 16 bytes at once  may step over the border of an allocated memory block.
Let's say that the \ac{OS} has allocated 8192 (0x2000) bytes at address 0x008c0000.
Thus, the block is the bytes starting from address 0x008c0000 to 0x008c1fff inclusive.

After the block, that is, starting from address 0x008c2000 there is nothing at all, e.g. the \ac{OS} not allocated
any memory there.
Any attempt to access memory starting from that address will raise an exception.

And let's consider the example in which the program is holding a string that contains 5 characters almost
at the end of a block, and that is not a crime.

\begin{center}
  \begin{tabular}{ | l | l | }
    \hline
        0x008c1ff8 & 'h' \\
        0x008c1ff9 & 'e' \\
        0x008c1ffa & 'l' \\
        0x008c1ffb & 'l' \\
        0x008c1ffc & 'o' \\
        0x008c1ffd & '\textbackslash{}x00' \\
        0x008c1ffe & random noise \\
        0x008c1fff & random noise \\
    \hline
  \end{tabular}
\end{center}

So, in normal conditions the program calls \strlen, passing it a pointer to the string \TT{'hello'} 
placed in memory at address 0x008c1ff8.
\strlen reads one byte at a time until 0x008c1ffd, where there's a zero byte, and then it stops.

Now if we implement our own \strlen reading 16 byte at once, starting at any address, aligned or not,
\MOVDQU may attempt to load 16 bytes at once at address 0x008c1ff8 up to 0x008c2008, 
and then an exception will be raised.
That situation is to be avoided, of course.

So then we'll work only with the addresses aligned on a 16 byte boundary, which in combination with the knowledge
that the \ac{OS}' page size is usually aligned on a 16-byte boundary gives us some warranty that our function will not
read from unallocated memory.

Let's get back to our function.

\myindex{x86!\Instructions!PXOR}
\verb|_mm_setzero_si128()|---is a macro generating \TT{pxor xmm0, xmm0}~---it just clears the \XMM{0} register.

\verb|_mm_load_si128()|---is a macro for \MOVDQA, it just loads 16 bytes from the address into the \XMM{1} register.

\myindex{x86!\Instructions!PCMPEQB}
\verb|_mm_cmpeq_epi8()|---is a macro for \PCMPEQB, an instruction that compares two XMM-registers bytewise.

And if some byte was equals to the one in the other register, 
there will be \TT{0xff} at this point in the result or 0 if otherwise.

For example:

\begin{verbatim}
XMM1: 0x11223344556677880000000000000000
XMM0: 0x11ab3444007877881111111111111111
\end{verbatim}

After the execution of \TT{pcmpeqb xmm1, xmm0}, the \XMM{1} register contains:

\begin{verbatim}
XMM1: 0xff0000ff0000ffff0000000000000000
\end{verbatim}

In our case, this instruction compares each 16-byte block with a block of 16 zero-bytes,
which was set in the \XMM{0} register by \TT{pxor xmm0, xmm0}.

\myindex{x86!\Instructions!PMOVMSKB}

The next macro is \TT{\_mm\_movemask\_epi8()}~---that is the \TT{PMOVMSKB} instruction.

It is very useful with \PCMPEQB.

\TT{pmovmskb eax, xmm1}

This instruction sets first \EAX bit to 1 if the most significant bit of the first byte in \XMM{1} is 1.
In other words, if the first byte of the \XMM{1} register is \TT{0xff}, then the first bit of \EAX is to be 1, too.

If the second byte in the \XMM{1} register is \TT{0xff}, then the second bit in \EAX is to be set to 1.
In other words, the instruction is answering the question \q{which bytes in \XMM{1} are \TT{0xff}?}
and returns 16 bits in the \EAX register. 
The other bits in the \EAX register are to be cleared.

By the way, do not forget about this quirk of our algorithm.
There might be 16 bytes in the input like:

\begin{center}
\ifdefined\ebook
\begin{bytefield}[endianness=big,bitwidth=0.062\linewidth]{16}
\else
\begin{bytefield}[endianness=big,bitwidth=0.05\linewidth]{16}
\fi
\bitheader{15,14,13,12,11,10,9,3,2,1,0} \\
\bitbox{1}{\q{h}} & 
\bitbox{1}{\q{e}} & 
\bitbox{1}{\q{l}} & 
\bitbox{1}{\q{l}} & 
\bitbox{1}{\q{o}} & 
\bitbox{1}{0} & 
\bitbox{7}{\garbage{}} & 
\bitbox{1}{0} &
\bitbox{2}{\garbage{}} & 
\end{bytefield}
\end{center}


It is the \TT{'hello'} string, terminating zero, and some random noise in memory.

If we load these 16 bytes into \XMM{1} and compare them with the zeroed \XMM{0}, 
we are getting something like
\footnote{An order from \ac{MSB} to \ac{LSB} is used here.}:

\begin{verbatim}
XMM1: 0x0000ff00000000000000ff0000000000
\end{verbatim}

This means that the instruction found two zero bytes, and it is not surprising.
 
\TT{PMOVMSKB} in our case will set \EAX to\\
\IT{0b0010000000100000}.

Obviously, our function must take only the first zero bit and ignore the rest.

\myindex{x86!\Instructions!BSF}
\label{instruction_BSF}
The next instruction is \TT{BSF} (\IT{Bit Scan Forward}). 

This instruction finds the first bit set to 1 and stores its position into the first operand.

\begin{verbatim}
EAX=0b0010000000100000
\end{verbatim}

After the execution of \TT{bsf eax, eax}, \EAX contains 5, meaning 
1 was found at the 5th bit position (starting from zero).

MSVC has a macro for this instruction: \TT{\_BitScanForward}.

Now it is simple. If a zero byte was found, its position is added to what we have already counted and now we have 
the return result.

Almost all.

By the way, it is also has to be noted that the MSVC compiler emitted two loop bodies side by side, for optimization.

By the way, SSE 4.2 (that appeared in Intel Core i7) offers more instructions where these string manipulations might be
even easier: \url{http://go.yurichev.com/17331}


