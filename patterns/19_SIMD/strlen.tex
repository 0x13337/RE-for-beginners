\section{\RU{Реализация \strlen при помощи SIMD}\EN{SIMD \strlen implementation}}
\label{SIMD_strlen}

\newcommand{\URLMSDNSSE}{\href{http://go.yurichev.com/17262}{MSDN: MMX, SSE, and SSE2 Intrinsics}}

\RU{Прежде всего, следует заметить, что SIMD-инструкции можно вставлять в \CCpp код при помощи специальных 
макросов\footnote{\URLMSDNSSE}. В MSVC, часть находится в файле \TT{intrin.h}.}
\EN{It has to be noted that the \ac{SIMD} instructions can be inserted in \CCpp code via 
special macros\footnote{\URLMSDNSSE}.
For MSVC, some of them are located in the \TT{intrin.h} file.}

\newcommand{\URLSTRLEN}{http://go.yurichev.com/17330}

\index{\CStandardLibrary!strlen()}
\RU{Имеется возможность реализовать функцию \strlen\footnote{strlen() ~--- стандартная функция Си 
для подсчета длины строки} при помощи SIMD-инструкций, работающий в 2-2.5 раза быстрее обычной реализации. 
Эта функция будет загружать в XMM-регистр сразу 16 байт и проверять каждый на ноль}
\EN{It is possible to implement the \strlen function\footnote{strlen()~---standard C library function for calculating
string length} using SIMD instructions that works 2-2.5 times faster than the common implementation.
This function loads 16 characters into a XMM-register and check each against zero}
\footnote{\RU{Пример базируется на исходнике отсюда: \url{\URLSTRLEN}.}
\EN{The example is based on source code from: \url{\URLSTRLEN}.}}.

\lstinputlisting{patterns/19_SIMD/18_3.c}

\RU{Компилируем в MSVC 2010 с опцией \Ox:}\EN{Let's compile it in MSVC 2010 with \Ox option:}

\lstinputlisting[caption=\Optimizing MSVC 2010]{patterns/19_SIMD/18_4_msvc_Ox.asm.\LANG}

\RU{Итак, прежде всего, мы проверяем указатель \TT{str}, выровнен ли он по 16-байтной границе. 
Если нет, то мы вызовем обычную реализацию \strlen.}
\EN{First, we check if the \TT{str} pointer is aligned on a 16-byte boundary.
If not, we call the generic \strlen implementation.}

\RU{Далее мы загружаем по 16 байт в регистр \XMM{1} при помощи команды \MOVDQA.}
\EN{Then, we load the next 16 bytes into the \XMM{1} register using \MOVDQA.}

\RU{Наблюдательный читатель может спросить, почему в этом месте мы не можем использовать \MOVDQU, 
которая может загружать откуда угодно не взирая на факт, выровнен ли указатель?}
\EN{An observant reader might ask, why can't \MOVDQU be used here since it can load data from the memory
regardless pointer alignment?}

\RU{Да, можно было бы сделать вот как: если указатель выровнен, загружаем используя \MOVDQA, 
иначе используем работающую чуть медленнее \MOVDQU.}
\EN{Yes, it might be done in this way: if the pointer is aligned, load data using \MOVDQA,
if not~---use the slower \MOVDQU.}

\RU{Однако здесь кроется не сразу заметная проблема, которая проявляется вот в чем:}
\EN{But here we are may hit another caveat:}

\index{Page (memory)}
\newcommand{\URLPAGE}{\href{http://go.yurichev.com/17136}{wikipedia}}

\RU{В \ac{OS} линии \gls{Windows NT} (и не только), память выделяется страницами по 4 KiB (4096 байт). 
Каждый win32-процесс якобы имеет в наличии 4 GiB, но на самом деле, 
только некоторые части этого адресного пространства присоединены к реальной физической памяти. 
Если процесс обратится к блоку памяти, которого не существует, сработает исключение. 
Так работает \ac{VM}\footnote{\URLPAGE}.}
\EN{In the \gls{Windows NT} line of \ac{OS} (but not limited to it), memory is allocated by pages of 4 KiB (4096 bytes).
Each win32-process has 4 GiB available, but in fact, only some parts
of the address space are connected to real physical memory.
If the process is accessing an absent memory block, an exception is to be raised.
That's how \ac{VM} works\footnote{\URLPAGE}.}

\RU{Так вот, функция, читающая сразу по 16 байт, имеет возможность нечаянно вылезти за границу 
выделенного блока памяти. 
Предположим, \ac{OS} выделила программе 8192 (0x2000) байт по адресу 0x008c0000. 
Таким образом, блок занимает байты с адреса 0x008c0000 по 0x008c1fff включительно.}
\EN{So, a function loading 16 bytes at once  may step over the border of an allocated memory block.
Let's say that the \ac{OS} has allocated 8192 (0x2000) bytes at address 0x008c0000.
Thus, the block is the bytes starting from address 0x008c0000 to 0x008c1fff inclusive.}

\RU{За этим блоком, то есть начиная с адреса 0x008c2000 нет вообще ничего, т.е. \ac{OS} не выделяла там память. 
Обращение к памяти начиная с этого адреса вызовет исключение.}
\EN{After the block, that is, starting from address 0x008c2000 there is nothing at all, e.g. the \ac{OS} not allocated
any memory there.
Any attempt to access memory starting from that address will raise an exception.}

\RU{И предположим, что программа хранит некую строку из, скажем, пяти символов почти в самом конце блока, 
что не является преступлением:}
\EN{And let's consider the example in which the program is holding a string that contains 5 characters almost at the end of a block,
and that is not a crime.}

\begin{center}
  \begin{tabular}{ | l | l | }
    \hline
        0x008c1ff8 & 'h' \\
        0x008c1ff9 & 'e' \\
        0x008c1ffa & 'l' \\
        0x008c1ffb & 'l' \\
        0x008c1ffc & 'o' \\
        0x008c1ffd & '\textbackslash{}x00' \\
        0x008c1ffe & \RU{здесь случайный мусор}\EN{random noise} \\
        0x008c1fff & \RU{здесь случайный мусор}\EN{random noise} \\
    \hline
  \end{tabular}
\end{center}

\RU{В обычных условиях, программа вызывает \strlen передав ей указатель на строку \TT{'hello'} 
лежащую по адресу 0x008c1ff8. 
\strlen будет читать по одному байту до 0x008c1ffd, где ноль, и здесь она закончит работу.}
\EN{So, in normal conditions the program calls \strlen, passing it a pointer to the string \TT{'hello'} 
placed in memory at address 0x008c1ff8.
\strlen reads one byte at a time until 0x008c1ffd, where there's a zero byte, and then it stops.}

\RU{Теперь, если мы напишем свою реализацию \strlen читающую сразу по 16 байт, с любого адреса, 
будь он выровнен по 16-байтной границе или нет, 
\MOVDQU попытается загрузить 16 байт с адреса 0x008c1ff8 по 0x008c2008, и произойдет исключение. 
Это ситуация которой, конечно, хочется избежать.}
\EN{Now if we implement our own \strlen reading 16 byte at once, starting at any address, aligned or not,
\MOVDQU may attempt to load 16 bytes at once at address 0x008c1ff8 up to 0x008c2008, 
and then an exception will be raised.
That situation is to be avoided, of course.}

\RU{Поэтому мы будем работать только с адресами, выровненными по 16 байт, что в сочетании со знанием 
что размер страницы \ac{OS} также, как правило, выровнен по 16 байт, 
даст некоторую гарантию что наша функция не будет пытаться читать из мест в невыделенной памяти.}
\EN{So then we'll work only with the addresses aligned on a 16 byte boundary, which in combination with the knowledge
that the \ac{OS}' page size is usually aligned on a 16-byte boundary gives us some warranty that our function will not
read from unallocated memory.}

\RU{Вернемся к нашей функции}\EN{Let's get back to our function}.

\index{x86!\Instructions!PXOR}
\verb|_mm_setzero_si128()|\EMDASH\RU{это макрос, генерирующий \TT{pxor xmm0, xmm0} ~--- инструкция просто обнуляет регистр \XMM{0}.}
\EN{is a macro generating \TT{pxor xmm0, xmm0}~---it just clears the \XMM{0} register}.

\verb|_mm_load_si128()|\EMDASH\RU{это макрос для \MOVDQA, он просто загружает 16 байт по адресу из указателя в \XMM{1}.}
\EN{is a macro for \MOVDQA, it just loads 16 bytes from the address into the \XMM{1} register.}

\index{x86!\Instructions!PCMPEQB}
\verb|_mm_cmpeq_epi8()|\EMDASH\RU{это макрос для \PCMPEQB, это инструкция, которая 
побайтово сравнивает значения из двух XMM регистров.} 
\EN{is a macro for \PCMPEQB, an instruction that compares two XMM-registers bytewise.}

\RU{И если какой-то из байт равен другому, 
то в результирующем значении будет выставлено на месте этого 
байта \TT{0xff}, либо 0, если байты не были равны.}
\EN{And if some byte was equals to the one in the other register, 
there will be \TT{0xff} at this point in the result or 0 if otherwise.}

\RU{Например.}\EN{For example.}

\begin{verbatim}
XMM1: 11223344556677880000000000000000
XMM0: 11ab3444007877881111111111111111
\end{verbatim}

\RU{После исполнения \TT{pcmpeqb xmm1, xmm0}, регистр \XMM{1} содержит:}
\EN{After the execution of \TT{pcmpeqb xmm1, xmm0}, the \XMM{1} register contains:}

\begin{verbatim}
XMM1: ff0000ff0000ffff0000000000000000
\end{verbatim}

\RU{Эта инструкция в нашем случае, сравнивает каждый 16-байтный блок с блоком состоящим из 16-и нулевых байт, 
выставленным в \XMM{0} при помощи \TT{pxor xmm0, xmm0}.}
\EN{In our case, this instruction compares each 16-byte block with a block of 16 zero-bytes,
which was set in the \XMM{0} register by \TT{pxor xmm0, xmm0}.}

\index{x86!\Instructions!PMOVMSKB}
\RU{Следующий макрос \TT{\_mm\_movemask\_epi8()} ~--- это инструкция \TT{PMOVMSKB}.}
\EN{The next macro is \TT{\_mm\_movemask\_epi8()}~---that is the \TT{PMOVMSKB} instruction.}

\RU{Она очень удобна как раз для использования в паре с \PCMPEQB.}
\EN{It is very useful with \PCMPEQB.}

\TT{pmovmskb eax, xmm1}

\RU{Эта инструкция выставит самый первый бит \EAX в единицу, если старший бит первого байта в 
регистре \XMM{1} является единицей. 
Иными словами, если первый байт в регистре \XMM{1} является \TT{0xff}, то первый бит в \EAX будет также единицей, 
иначе нулем.}
\EN{This instruction sets first \EAX bit to 1 if the most significant bit of the first byte in \XMM{1} is $1$.
In other words, if the first byte of the \XMM{1} register is \TT{0xff}, then the first bit of \EAX is to be 1, too.}

\RU{Если второй байт в регистре \XMM{1} является \TT{0xff}, то второй бит в \EAX также будет единицей. 
Иными словами, инструкция отвечает на вопрос, ``какие из байт в \XMM{1} являются \TT{0xff}?''
В результате приготовит 16 бит и запишет в \EAX. Остальные биты в \EAX обнулятся.}
\EN{If the second byte in the \XMM{1} register is \TT{0xff}, then the second bit in \EAX is to be set to 1.
In other words, the instruction is answering the question ``which bytes in \XMM{1} are \TT{0xff}?''
and returns 16 bits in the \EAX register. 
The other bits in the \EAX register are to be cleared.}

\RU{Кстати, не забывайте также вот о какой особенности нашего алгоритма.}
\EN{By the way, do not forget about this quirk of our algorithm.}
\RU{На вход может прийти 16 байт вроде}\EN{There might be 16 bytes in the input like}:

\begin{center}
\begin{bytefield}[endianness=big,bitwidth=0.05\linewidth]{16}
\bitheader{15,14,13,12,11,10,9,3,2,1,0} \\
\bitbox{1}{``h''} & 
\bitbox{1}{``e''} & 
\bitbox{1}{``l''} & 
\bitbox{1}{``l''} & 
\bitbox{1}{``o''} & 
\bitbox{1}{0} & 
\bitbox{7}{\garbage{}} & 
\bitbox{1}{0} &
\bitbox{2}{\garbage{}} & 
\end{bytefield}
\end{center}

\RU{Это строка \TT{'hello'}, после нее терминирующий ноль, затем немного мусора в памяти.}
\EN{It is the \TT{'hello'} string, terminating zero, and some random noise in memory.}
\RU{Если мы загрузим эти 16 байт в \XMM{1} и сравним с нулевым \XMM{0}, то в итоге получим такое 
\footnote{Я использую здесь порядок с \ac{MSB} до \ac{LSB}}:}
\EN{If we load these 16 bytes into \XMM{1} and compare them with the zeroed \XMM{0}, 
we are getting something like
\footnote{I use here order from \ac{MSB} to \ac{LSB}}:}

\begin{verbatim}
XMM1: 0000ff00000000000000ff0000000000
\end{verbatim}

\RU{Это означает что инструкция сравнения обнаружила два нулевых байта, что и не удивительно.}
\EN{This means that the instruction found two zero bytes, and it is not surprising.}

\RU{\TT{PMOVMSKB} в нашем случае подготовит \EAX вот так (в двоичном представлении):} 
\EN{\TT{PMOVMSKB} in our case will set \EAX to (in binary representation):} \IT{0010000000100000b}.

\RU{Совершенно очевидно, что далее наша функция должна учитывать только первый встретившийся
нулевой бит и игнорировать все остальное.}
\EN{Obviously, our function must take only the first zero bit and ignore the rest.}

\index{x86!\Instructions!BSF}
\label{instruction_BSF}
\RU{Следующая инструкция \EMDASH}\EN{The next instruction is} \TT{BSF} (\IT{Bit Scan Forward}). 
\RU{Это инструкция находит самый младший бит во втором операнде и записывает его позицию в первый операнд.}
\EN{This instruction finds the first bit set to 1 and stores its position into the first operand.}

\begin{verbatim}
EAX=0010000000100000b
\end{verbatim}

\RU{После исполнения этой инструкции \TT{bsf eax, eax}, в \EAX будет 5, что означает, 
что единица найдена в пятой позиции (считая с нуля).}
\EN{After the execution of \TT{bsf eax, eax}, \EAX contains 5, meaning 
1 was found at the 5th bit position (starting from zero).}

\RU{Для использования этой инструкции, в MSVC также имеется макрос}
\EN{MSVC has a macro for this instruction:} \TT{\_BitScanForward}.

\RU{А дальше все просто. Если нулевой байт найден, его позиция прибавляется к тому что 
мы уже насчитали и возвращается результат.}
\EN{Now it is simple. If a zero byte was found, its position is added to what we have already counted and now we have 
the return result.}

\RU{Почти всё.}\EN{Almost all.}

\RU{Кстати, следует также отметить, что компилятор MSVC сгенерировал два тела цикла сразу, для оптимизации.}
\EN{By the way, it is also has to be noted that the MSVC compiler emitted two loop bodies side by side, for optimization.}

\RU{Кстати, в SSE 4.2 (который появился в Intel Core i7) все эти манипуляции со строками могут быть еще проще:}
\EN{By the way, SSE 4.2 (that appeared in Intel Core i7) offers more instructions where these string manipulations might be
even easier:} \url{http://go.yurichev.com/17331}
