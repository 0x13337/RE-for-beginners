\section{\RU{Реализация \strlen при помощи SIMD}\EN{SIMD \strlen implementation}}

\newcommand{\URLMSDNSSE}{\href{http://go.yurichev.com/17262}{MSDN: MMX, SSE, and SSE2 Intrinsics}}

\RU{Прежде всего, следует заметить, что SIMD-инструкции можно вставлять в \CCpp код при помощи специальных 
макросов\footnote{\URLMSDNSSE}. В MSVC, часть находится в файле \TT{intrin.h}.}
\EN{It should be noted the \ac{SIMD}-instructions may be inserted into \CCpp code via 
special macros\footnote{\URLMSDNSSE}.
As of MSVC, some of them are located in the \TT{intrin.h} file.}

\newcommand{\URLSTRLEN}{http://go.yurichev.com/17330}

\index{\CStandardLibrary!strlen()}
\RU{Имеется возможность реализовать функцию \strlen\footnote{strlen() ~--- стандартная функция Си 
для подсчета длины строки} при помощи SIMD-инструкций, работающий в 2-2.5 раза быстрее обычной реализации. 
Эта функция будет загружать в XMM-регистр сразу 16 байт и проверять каждый на ноль}
\EN{It is possible to implement \strlen function\footnote{strlen()~---standard C library function for calculating
string length} using SIMD-instructions, working 2-2.5 times faster than common implementation.
This function will load 16 characters into a XMM-register and check each against zero}
\footnote{\RU{Пример базируется на исходнике отсюда: \url{\URLSTRLEN}.}
\EN{The example is based on source code from: \url{\URLSTRLEN}.}}.

\lstinputlisting{patterns/19_SIMD/18_3.c}

\RU{Компилируем в MSVC 2010 с опцией \Ox:}\EN{Let's compile in MSVC 2010 with \Ox option:}

\lstinputlisting[caption=\Optimizing MSVC 2010]{patterns/19_SIMD/18_4_msvc_Ox.asm.\LANG}

\RU{Итак, прежде всего, мы проверяем указатель \TT{str}, выровнен ли он по 16-байтной границе. 
Если нет, то мы вызовем обычную реализацию \strlen.}
\EN{First of all, we check \TT{str} pointer, if it is aligned on a 16-byte boundary.
If not, let's call generic \strlen implementation.}

\RU{Далее мы загружаем по 16 байт в регистр \XMM{1} при помощи команды \MOVDQA.}
\EN{Then, load next 16 bytes into the \XMM{1} register using \MOVDQA instruction.}

\RU{Наблюдательный читатель может спросить, почему в этом месте мы не можем использовать \MOVDQU, 
которая может загружать откуда угодно не взирая на факт, выровнен ли указатель?}
\EN{Observant reader might ask, why \MOVDQU cannot be used here since it can load data from the memory
regardless pointer alignment?}

\RU{Да, можно было бы сделать вот как: если указатель выровнен, загружаем используя \MOVDQA, 
иначе используем работающую чуть медленнее \MOVDQU.}
\EN{Yes, it might be done in this way: if pointer is aligned, load data using \MOVDQA,
if not~---use slower \MOVDQU.}

\RU{Однако здесь кроется не сразу заметная проблема, которая проявляется вот в чем:}
\EN{But here we are may stick into hard to notice caveat:}

\index{Page (memory)}
\newcommand{\URLPAGE}{\href{http://go.yurichev.com/17136}{wikipedia}}

\RU{В \ac{OS} линии \gls{Windows NT} (и не только), память выделяется страницами по 4 KiB (4096 байт). 
Каждый win32-процесс якобы имеет в наличии 4 GiB, но на самом деле, 
только некоторые части этого адресного пространства присоединены к реальной физической памяти. 
Если процесс обратится к блоку памяти, которого не существует, сработает исключение. 
Так работает \ac{VM}\footnote{\URLPAGE}.}
\EN{In \gls{Windows NT} line of \ac{OS} (but not limited to it), memory allocated by pages of 4 KiB (4096 bytes).
Each win32-process has ostensibly 4 GiB, but in fact, only some parts
of address space are connected to real physical memory.
If the process accessing to the absent memory block, exception will be raised.
That's how \ac{VM} works\footnote{\URLPAGE}.}

\RU{Так вот, функция, читающая сразу по 16 байт, имеет возможность нечаянно вылезти за границу 
выделенного блока памяти. 
Предположим, \ac{OS} выделила программе 8192 (0x2000) байт по адресу 0x008c0000. 
Таким образом, блок занимает байты с адреса 0x008c0000 по 0x008c1fff включительно.}
\EN{So, a function loading 16 bytes at once, may step over a border of allocated memory block.
Let's consider, \ac{OS} allocated 8192 (0x2000) bytes at the address 0x008c0000.
Thus, the block is the bytes starting from address 0x008c0000 to 0x008c1fff inclusive.}

\RU{За этим блоком, то есть начиная с адреса 0x008c2000 нет вообще ничего, т.е., \ac{OS} не выделяла там память. 
Обращение к памяти начиная с этого адреса вызовет исключение.}
\EN{After the block, that is, starting from address 0x008c2000 there is nothing at all, e.g., \ac{OS} not allocated
any memory there.
Attempt to access a memory starting from the address will raise exception.}

\RU{И предположим, что программа хранит некую строку из, скажем, пяти символов почти в самом конце блока, 
что не является преступлением:}
\EN{And let's consider, the program holding a string containing 5 characters almost at the end of block,
and that is not a crime.}

\begin{center}
  \begin{tabular}{ | l | l | }
    \hline
        0x008c1ff8 & 'h' \\
        0x008c1ff9 & 'e' \\
        0x008c1ffa & 'l' \\
        0x008c1ffb & 'l' \\
        0x008c1ffc & 'o' \\
        0x008c1ffd & '\textbackslash{}x00' \\
        0x008c1ffe & \RU{здесь случайный мусор}\EN{random noise} \\
        0x008c1fff & \RU{здесь случайный мусор}\EN{random noise} \\
    \hline
  \end{tabular}
\end{center}

\RU{В обычных условиях, программа вызывает \strlen передав ей указатель на строку \TT{'hello'} 
лежащую по адресу 0x008c1ff8. 
\strlen будет читать по одному байту до 0x008c1ffd, где ноль, и здесь она закончит работу.}
\EN{So, in common conditions the program calling \strlen passing it a pointer to string \TT{'hello'} 
lying in memory at address 0x008c1ff8.
\strlen will read one byte at a time until 0x008c1ffd, where zero-byte, and so here it will stop working.}

\RU{Теперь, если мы напишем свою реализацию \strlen читающую сразу по 16 байт, с любого адреса, 
будь он выровнен по 16-байтной границе или нет, 
\MOVDQU попытается загрузить 16 байт с адреса 0x008c1ff8 по 0x008c2008, и произойдет исключение. 
Это ситуация которой, конечно, хочется избежать.}
\EN{Now if we implement our own \strlen reading 16 byte at once, starting at any address, will it be aligned or not,
\MOVDQU may attempt to load 16 bytes at once at address 0x008c1ff8 up to 0x008c2008, 
and then exception will be raised.
That's the situation to be avoided, of course.}

\RU{Поэтому мы будем работать только с адресами, выровненными по 16 байт, что в сочетании со знанием 
что размер страницы \ac{OS} также, как правило, выровнен по 16 байт, 
даст некоторую гарантию что наша функция не будет пытаться читать из мест в невыделенной памяти.}
\EN{So then we'll work only with the addresses aligned on a 16 byte boundary, what in combination with a knowledge
of \ac{OS} page size is usually aligned on a 16-byte boundary too, give us some warranty our function will not
read from unallocated memory.}

\RU{Вернемся к нашей функции}\EN{Let's back to our function}.

\index{x86!\Instructions!PXOR}
\verb|_mm_setzero_si128()|\EMDASH\RU{это макрос, генерирующий \TT{pxor xmm0, xmm0} ~--- инструкция просто обнуляет регистр \XMM{0}.}
\EN{is a macro generating \TT{pxor xmm0, xmm0}~---instruction just clears the \XMM{0} register}.

\verb|_mm_load_si128()|\EMDASH\RU{это макрос для \MOVDQA, он просто загружает 16 байт по адресу из указателя в \XMM{1}.}
\EN{is a macro for \MOVDQA, it just loading 16 bytes from the address into the \XMM{1} register.}

\index{x86!\Instructions!PCMPEQB}
\verb|_mm_cmpeq_epi8()|\EMDASH\RU{это макрос для \PCMPEQB, это инструкция которая 
побайтово сравнивает значения из двух XMM регистров.} 
\EN{is a macro for \PCMPEQB, is an instruction comparing two XMM-registers bytewise.}

\RU{И если какой-то из байт равен другому, то в результирующем значении будет выставлено на месте этого 
байта \TT{0xff}, либо 0, если байты не были равны.}
\EN{And if some byte was equals to other, there will be \TT{0xff} at this point in the result or 0 if otherwise.}

\RU{Например.}\EN{For example.}

\begin{verbatim}
XMM1: 11223344556677880000000000000000
XMM0: 11ab3444007877881111111111111111
\end{verbatim}

\RU{После исполнения \TT{pcmpeqb xmm1, xmm0}, регистр \XMM{1} будет содержать:}
\EN{After \TT{pcmpeqb xmm1, xmm0} execution, the \XMM{1} register shall contain:}

\begin{verbatim}
XMM1: ff0000ff0000ffff0000000000000000
\end{verbatim}

\RU{Эта инструкция в нашем случае, сравнивает каждый 16-байтный блок с блоком состоящим из 16-и нулевых байт, 
выставленным в \XMM{0} при помощи \TT{pxor xmm0, xmm0}.}
\EN{In our case, this instruction comparing each 16-byte block with the block of 16 zero-bytes,
was set in the \XMM{0} register by \TT{pxor xmm0, xmm0}.}

\index{x86!\Instructions!PMOVMSKB}
\RU{Следующий макрос \TT{\_mm\_movemask\_epi8()} ~--- это инструкция \TT{PMOVMSKB}.}
\EN{The next macro is \TT{\_mm\_movemask\_epi8()}~---that is \TT{PMOVMSKB} instruction.}

\RU{Она очень удобна как раз для использования в паре с \PCMPEQB.}
\EN{It is very useful if to use it with \PCMPEQB.}

\TT{pmovmskb eax, xmm1}

\RU{Эта инструкция выставит самый первый бит \EAX в единицу, если старший бит первого байта в 
регистре \XMM{1} является единицей. 
Иными словами, если первый байт в регистре \XMM{1} является \TT{0xff}, то первый бит в \EAX будет также единицей, 
иначе нулем.}
\EN{This instruction will set first \EAX bit into 1 if most significant bit of the first byte in the \XMM{1} is $1$.
In other words, if first byte of the \XMM{1} register is \TT{0xff}, first \EAX bit will be set to 1 too.}

\RU{Если второй байт в регистре \XMM{1} является \TT{0xff}, то второй бит в \EAX также будет единицей. 
Иными словами, инструкция отвечает на вопрос, ``какие из байт в \XMM{1} являются \TT{0xff}?''
В результате приготовит 16 бит и запишет в \EAX. Остальные биты в \EAX обнулятся.}
\EN{If second byte in the \XMM{1} register is \TT{0xff}, then second \EAX bit will be set to 1 too.
In other words, the instruction is answering to the question ``which bytes in the \XMM{1} are \TT{0xff}?''
And will prepare 16 bits in the \EAX register. Other bits in the \EAX register are to be cleared.}

\RU{Кстати, не забывайте также вот о какой особенности нашего алгоритма.}
\EN{By the way, do not forget about this feature of our algorithm.}
\RU{На вход может прийти 16 байт вроде}\EN{There might be 16 bytes on input like}:

\begin{center}
\begin{bytefield}[endianness=big,bitwidth=0.05\linewidth]{16}
\bitheader{15,14,13,12,11,10,9,3,2,1,0} \\
\bitbox{1}{``h''} & 
\bitbox{1}{``e''} & 
\bitbox{1}{``l''} & 
\bitbox{1}{``l''} & 
\bitbox{1}{``o''} & 
\bitbox{1}{0} & 
\bitbox{7}{\garbage{}} & 
\bitbox{1}{0} &
\bitbox{2}{\garbage{}} & 
\end{bytefield}
\end{center}

\RU{Это строка \TT{'hello'}, после нее терминирующий ноль, затем немного мусора в памяти.}
\EN{It is a \TT{'hello'} string, terminating zero, and also a random noise in memory.}
\RU{Если мы загрузим эти 16 байт в \XMM{1} и сравним с нулевым \XMM{0}, то в итоге получим такое 
\footnote{Я использую здесь порядок с \ac{MSB} до \ac{LSB}}:}
\EN{If we load these 16 bytes into \XMM{1} and compare them with zeroed \XMM{0}, we will get something like
\footnote{I use here order from \ac{MSB} to \ac{LSB}}:}

\begin{verbatim}
XMM1: 0000ff00000000000000ff0000000000
\end{verbatim}

\RU{Это означает что инструкция сравнения обнаружила два нулевых байта, что и не удивительно.}
\EN{This means, the instruction found two zero bytes, and that is not surprising.}

\RU{\TT{PMOVMSKB} в нашем случае подготовит \EAX вот так (в двоичном представлении):} 
\EN{\TT{PMOVMSKB} in our case will prepare \EAX like (in binary representation):} \IT{0010000000100000b}.

\RU{Совершенно очевидно, что далее наша функция должна учитывать только первый встретившийся
нулевой бит и игнорировать все остальное.}
\EN{Obviously, our function must take only first zero bit and ignore the rest ones.}

\index{x86!\Instructions!BSF}
\label{instruction_BSF}
\RU{Следующая инструкция}\EN{The next instruction}\EMDASH\TT{BSF} (\IT{Bit Scan Forward}). 
\RU{Это инструкция находит самый младший бит во втором операнде и записывает его позицию в первый операнд.}
\EN{This instruction find first bit set to 1 and stores its position into first operand.}

\begin{verbatim}
EAX=0010000000100000b
\end{verbatim}

\RU{После исполнения этой инструкции \TT{bsf eax, eax}, в \EAX будет 5, что означает, 
что единица найдена в пятой позиции (считая с нуля).}
\EN{After \TT{bsf eax, eax} instruction execution, \EAX will contain 5, this means, 
1 found at 5th bit position (starting from zero).}

\RU{Для использования этой инструкции, в MSVC также имеется макрос}
\EN{MSVC has a macro for this instruction:} \TT{\_BitScanForward}.

\RU{А дальше все просто. Если нулевой байт найден, его позиция прибавляется к тому что 
мы уже насчитали и возвращается результат.}
\EN{Now it is simple. If zero byte found, its position added to what we already counted and now we have 
ready to return result.}

\RU{Почти всё.}\EN{Almost all.}

\RU{Кстати, следует также отметить, что компилятор MSVC сгенерировал два тела цикла сразу, для оптимизации.}
\EN{By the way, it is also should be noted, MSVC compiler emitted two loop bodies side by side, for optimization.}

\RU{Кстати, в SSE 4.2 (который появился в Intel Core i7) все эти манипуляции со строками могут быть еще проще:}
\EN{By the way, SSE 4.2 (appeared in Intel Core i7) offers more instructions where these string manipulations might be
even easier:} \url{http://go.yurichev.com/17331}
