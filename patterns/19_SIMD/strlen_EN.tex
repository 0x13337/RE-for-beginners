\subsection{SIMD \strlen implementation}
\label{SIMD_strlen}

\newcommand{\URLMSDNSSE}{\href{http://go.yurichev.com/17262}{MSDN: MMX, SSE, and SSE2 Intrinsics}}

It has to be noted that the \ac{SIMD} instructions can be inserted in \CCpp code via special macros\footnote{\URLMSDNSSE}.
For MSVC, some of them are located in the \TT{intrin.h} file.

\newcommand{\URLSTRLEN}{http://go.yurichev.com/17330}

\myindex{\CStandardLibrary!strlen()}

It is possible to implement the \strlen function\footnote{strlen()~---standard C library function for calculating
string length} using SIMD instructions that works 2-2.5 times faster than the common implementation.
This function loads 16 characters into a XMM-register and check each against zero
\footnote{
The example is based on source code from: \url{\URLSTRLEN}.}.

\lstinputlisting{patterns/19_SIMD/18_3.c}

Let's compile it in MSVC 2010 with \Ox option:

\lstinputlisting[caption=\Optimizing MSVC 2010]{patterns/19_SIMD/18_4_msvc_Ox_EN.asm}

How it works?
First of all, we must understand goal of the function.
It calculates C-string length, but we can use different terms: it's task is searching for zero byte, and then calculating its position relatively to string start.

First, we check if the \TT{str} pointer is aligned on a 16-byte boundary.
If not, we call the generic \strlen implementation.

Then, we load the next 16 bytes into the \XMM{1} register using \MOVDQA.

An observant reader might ask, why can't \MOVDQU be used here since it can load data from the memory
regardless pointer alignment?

Yes, it might be done in this way: if the pointer is aligned, load data using \MOVDQA,
if not~---use the slower \MOVDQU.

But here we are may hit another caveat:

\myindex{Page (memory)}
\newcommand{\URLPAGE}{\href{http://go.yurichev.com/17136}{wikipedia}}

In the \gls{Windows NT} line of \ac{OS} (but not limited to it), memory is allocated by pages of 4 KiB (4096 bytes).
Each win32-process has 4 GiB available, but in fact, only some parts
of the address space are connected to real physical memory.
If the process is accessing an absent memory block, an exception is to be raised.
That's how \ac{VM} works\footnote{\URLPAGE}.

So, a function loading 16 bytes at once  may step over the border of an allocated memory block.
Let's say that the \ac{OS} has allocated 8192 (0x2000) bytes at address 0x008c0000.
Thus, the block is the bytes starting from address 0x008c0000 to 0x008c1fff inclusive.

After the block, that is, starting from address 0x008c2000 there is nothing at all, e.g. the \ac{OS} not allocated
any memory there.
Any attempt to access memory starting from that address will raise an exception.

And let's consider the example in which the program is holding a string that contains 5 characters almost
at the end of a block, and that is not a crime.

\begin{center}
  \begin{tabular}{ | l | l | }
    \hline
        0x008c1ff8 & 'h' \\
        0x008c1ff9 & 'e' \\
        0x008c1ffa & 'l' \\
        0x008c1ffb & 'l' \\
        0x008c1ffc & 'o' \\
        0x008c1ffd & '\textbackslash{}x00' \\
        0x008c1ffe & random noise \\
        0x008c1fff & random noise \\
    \hline
  \end{tabular}
\end{center}

So, in normal conditions the program calls \strlen, passing it a pointer to the string \TT{'hello'} 
placed in memory at address 0x008c1ff8.
\strlen reads one byte at a time until 0x008c1ffd, where there's a zero byte, and then it stops.

Now if we implement our own \strlen reading 16 byte at once, starting at any address, aligned or not,
\MOVDQU may attempt to load 16 bytes at once at address 0x008c1ff8 up to 0x008c2008, 
and then an exception will be raised.
That situation is to be avoided, of course.

So then we'll work only with the addresses aligned on a 16 byte boundary, which in combination with the knowledge
that the \ac{OS}' page size is usually aligned on a 16-byte boundary gives us some warranty that our function will not
read from unallocated memory.

Let's get back to our function.

\myindex{x86!\Instructions!PXOR}
\verb|_mm_setzero_si128()|---is a macro generating \TT{pxor xmm0, xmm0}~---it just clears the \XMM{0} register.

\verb|_mm_load_si128()|---is a macro for \MOVDQA, it just loads 16 bytes from the address into the \XMM{1} register.

\myindex{x86!\Instructions!PCMPEQB}
\verb|_mm_cmpeq_epi8()|---is a macro for \PCMPEQB, an instruction that compares two XMM-registers bytewise.

And if some byte was equals to the one in the other register, 
there will be \TT{0xff} at this point in the result or 0 if otherwise.

For example:

\begin{verbatim}
XMM1: 0x11223344556677880000000000000000
XMM0: 0x11ab3444007877881111111111111111
\end{verbatim}

After the execution of \TT{pcmpeqb xmm1, xmm0}, the \XMM{1} register contains:

\begin{verbatim}
XMM1: 0xff0000ff0000ffff0000000000000000
\end{verbatim}

In our case, this instruction compares each 16-byte block with a block of 16 zero-bytes,
which was set in the \XMM{0} register by \TT{pxor xmm0, xmm0}.

\myindex{x86!\Instructions!PMOVMSKB}

The next macro is \TT{\_mm\_movemask\_epi8()}~---that is the \TT{PMOVMSKB} instruction.

It is very useful with \PCMPEQB.

\TT{pmovmskb eax, xmm1}

This instruction sets first \EAX bit to 1 if the most significant bit of the first byte in \XMM{1} is 1.
In other words, if the first byte of the \XMM{1} register is \TT{0xff}, then the first bit of \EAX is to be 1, too.

If the second byte in the \XMM{1} register is \TT{0xff}, then the second bit in \EAX is to be set to 1.
In other words, the instruction is answering the question \q{which bytes in \XMM{1} are \TT{0xff}?}
and returns 16 bits in the \EAX register. 
The other bits in the \EAX register are to be cleared.

By the way, do not forget about this quirk of our algorithm.
There might be 16 bytes in the input like:

\begin{center}
\ifdefined\ebook
\begin{bytefield}[endianness=big,bitwidth=0.062\linewidth]{16}
\else
\begin{bytefield}[endianness=big,bitwidth=0.05\linewidth]{16}
\fi
\bitheader{15,14,13,12,11,10,9,3,2,1,0} \\
\bitbox{1}{\q{h}} & 
\bitbox{1}{\q{e}} & 
\bitbox{1}{\q{l}} & 
\bitbox{1}{\q{l}} & 
\bitbox{1}{\q{o}} & 
\bitbox{1}{0} & 
\bitbox{7}{\garbage{}} & 
\bitbox{1}{0} &
\bitbox{2}{\garbage{}} & 
\end{bytefield}
\end{center}


It is the \TT{'hello'} string, terminating zero, and some random noise in memory.

If we load these 16 bytes into \XMM{1} and compare them with the zeroed \XMM{0}, 
we are getting something like
\footnote{An order from \ac{MSB} to \ac{LSB} is used here.}:

\begin{verbatim}
XMM1: 0x0000ff00000000000000ff0000000000
\end{verbatim}

This means that the instruction found two zero bytes, and it is not surprising.
 
\TT{PMOVMSKB} in our case will set \EAX to\\
\IT{0b0010000000100000}.

Obviously, our function must take only the first zero bit and ignore the rest.

\myindex{x86!\Instructions!BSF}
\label{instruction_BSF}
The next instruction is \TT{BSF} (\IT{Bit Scan Forward}). 

This instruction finds the first bit set to 1 and stores its position into the first operand.

\begin{verbatim}
EAX=0b0010000000100000
\end{verbatim}

After the execution of \TT{bsf eax, eax}, \EAX contains 5, meaning 
1 was found at the 5th bit position (starting from zero).

MSVC has a macro for this instruction: \TT{\_BitScanForward}.

Now it is simple. If a zero byte was found, its position is added to what we have already counted and now we have 
the return result.

Almost all.

By the way, it is also has to be noted that the MSVC compiler emitted two loop bodies side by side, for optimization.

By the way, SSE 4.2 (that appeared in Intel Core i7) offers more instructions where these string manipulations might be
even easier: \url{http://go.yurichev.com/17331}
