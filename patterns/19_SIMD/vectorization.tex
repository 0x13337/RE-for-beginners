\section{\RU{Векторизация}\EN{Vectorization}}

\newcommand{\URLVEC}{\href{http://go.yurichev.com/17080}{Wikipedia: vectorization}}

\RU{Векторизация\footnote{\URLVEC} это когда у вас есть цикл, который берет на вход несколько массивов и выдает, 
например, один массив данных. 
Тело цикла берет некоторые элементы из входных массивов, что-то делает с ними и помещает в выходной. 
Важно, что операция, применяемая ко всем элементам одна и та же. 
Векторизация ~--- это обрабатывать несколько элементов одновременно.}
\EN{Vectorization\footnote{\URLVEC}, for example, is when you have a loop taking couple of arrays at input and produces one array.
Loop body takes values from input arrays, do something and put result into output array.
It is important that there is only one single operation applied to each element.
Vectorization~---is to process several elements simultaneously.}

\RU{Векторизация ~--- это не самая новая технология: автор сих строк видел её по крайней мере на 
линейке суперкомпьютеров Cray Y-MP от 1988, когда работал на его версии-``лайт'' Cray Y-MP EL
\footnote{Удаленно. Он находится в музее суперкомпьютеров: \url{http://go.yurichev.com/17081}}}
\EN{Vectorization is not very fresh technology: author of this textbook saw it at least on Cray Y-MP 
supercomputer line from 1988 when played with its ``lite'' version Cray Y-MP EL
\footnote{Remotely. It is installed in the museum of supercomputers: \url{http://go.yurichev.com/17081}}}.

\RU{Например:}\EN{For example:}

\begin{lstlisting}
for (i = 0; i < 1024; i++)
{
    C[i] = A[i]*B[i];
}
\end{lstlisting}

\RU{Этот фрагмент кода берет элементы из A и B, перемножает и сохраняет результат в C.}
\EN{This fragment of code takes elements from A and B, multiplies them and save result into C.}

\index{x86!\Instructions!PLMULLD}
\index{x86!\Instructions!PLMULHW}
\newcommand{\PMULLD}{\IT{PMULLD} (\IT{\RU{Перемножить запакованные знаковые DWORD и сохранить младшую часть результата}
\EN{Multiply Packed Signed Dword Integers and Store Low Result}})}
\newcommand{\PMULHW}{\TT{PMULHW} (\IT{\RU{Перемножить запакованные знаковые DWORD и сохранить старшую часть результата}
\EN{Multiply Packed Signed Integers and Store High Result}})}

\RU{Если представить, что каждый элемент массива ~--- это 32-битный \Tint, то их можно загружать сразу 
по 4 из А в 128-битный XMM-регистр, 
из B в другой XMM-регистр и выполнив инструкцию \PMULLD{} и \PMULHW{}, можно получить 4 64-битных 
\glslink{product}{произведения} сразу.}
\EN{If each array element we have is 32-bit \Tint, then it is possible to load 4 elements from A into 128-bit 
XMM-register, from B to another XMM-registers, and by executing \PMULLD{} and \PMULHW{}, 
it is possible to get 4 64-bit \glspl{product} at once.}

\RU{Таким образом, тело цикла исполняется $1024/4$ раза вместо 1024, что в 4 раза меньше, и, конечно, быстрее.}
\EN{Thus, loop body count is $1024/4$ instead of $1024$, that is 4 times less and, of course, faster.}

\newcommand{\URLINTELVEC}{\href{http://go.yurichev.com/17082}{Excerpt: Effective Automatic Vectorization}}

\subsection{\RU{Пример сложения}\EN{Addition example}}

\index{Intel C++}
\RU{Некоторые компиляторы умеют делать автоматическую векторизацию в простых случаях, 
например, Intel C++\footnote{Еще о том, как Intel C++ умеет автоматически векторизовать циклы: \URLINTELVEC}.}
\EN{Some compilers can do vectorization automatically in a simple cases, 
e.g., Intel C++\footnote{More about Intel C++ automatic vectorization: \URLINTELVEC}.}

\RU{Я написал очень простую функцию:}\EN{I wrote tiny function:}

\begin{lstlisting}
int f (int sz, int *ar1, int *ar2, int *ar3)
{
	for (int i=0; i<sz; i++)
		ar3[i]=ar1[i]+ar2[i];

	return 0;
};
\end{lstlisting}

\subsubsection{Intel C++}

\RU{Компилирую при помощи}\EN{Let's compile it with} Intel C++ 11.1.051 win32:

\begin{verbatim}
icl intel.cpp /QaxSSE2 /Faintel.asm /Ox
\end{verbatim}

\RU{Имеем такое (в \IDA):}\EN{We got (in \IDA):}

\lstinputlisting{patterns/19_SIMD/18_1.asm.\LANG}

\RU{Инструкции, имеющие отношение к SSE2 это:}\EN{SSE2-related instructions are:}
\index{x86!\Instructions!MOVDQA}
\index{x86!\Instructions!MOVDQU}
\index{x86!\Instructions!PADDD}
\begin{itemize}
\item
\MOVDQU (\IT{Move Unaligned Double Quadword})\EMDASH\RU{она просто загружает 16 байт из памяти в XMM-регистр}
\EN{it just load 16 bytes from memory into a XMM-register}.

\item
\PADDD (\IT{Add Packed Integers})\EMDASH\RU{складывает сразу 4 пары 32-битных чисел и оставляет в первом операнде результат. 
Кстати, если произойдет переполнение, то исключения не произойдет и никакие флаги не установятся, 
запишутся просто младшие 32 бита результата. 
Если один из операндов \PADDD ~--- адрес значения в памяти, 
то требуется чтобы адрес был выровнен по 16-байтной границе. Если он не выровнен, произойдет исключение
\footnote{О выравнивании данных см. также: \URLWPDA}.}
\EN{adding 4 pairs of 32-bit numbers and leaving result in first operand.
By the way, no exception raised in case of overflow and no flags will be set, just low 32-bit of result will
be stored.
If one of \PADDD operands is address of value in memory,
then address must be aligned on a 16-byte boundary. If it is not aligned, exception will be occurred
\footnote{More about data aligning: \URLWPDA}.}

\item
\MOVDQA (\IT{Move Aligned Double Quadword})\EMDASH\RU{тоже что и \MOVDQU, только подразумевает 
что адрес в памяти выровнен по 16-байтной границе. 
Если он не выровнен, произойдет исключение. 
\MOVDQA работает быстрее чем \MOVDQU, но требует вышеозначенного.}
\EN{the same as \MOVDQU, but requires address of value in memory to be aligned on a 16-bit boundary.
If it is not aligned, exception will be raised.
\MOVDQA works faster than \MOVDQU, but requires aforesaid.}

\end{itemize}

\RU{Итак, эти SSE2-инструкции исполнятся только в том случае если еще осталось просуммировать 
4 пары переменных типа \Tint плюс если указатель \TT{ar3} выровнен по 16-байтной границе.}
\EN{So, these SSE2-instructions will be executed only in case if there are more 4 pairs to work on
plus pointer \TT{ar3} is aligned on a 16-byte boundary.}

\RU{Более того, если еще и \TT{ar2} выровнен по 16-байтной границе, то будет выполняться этот фрагмент кода:}
\EN{More than that, if \TT{ar2} is aligned on a 16-byte boundary as well, this fragment of code will be executed:}

\begin{lstlisting}
movdqu  xmm0, xmmword ptr [ebx+edi*4] ; ar1+i*4
paddd   xmm0, xmmword ptr [esi+edi*4] ; ar2+i*4
movdqa  xmmword ptr [eax+edi*4], xmm0 ; ar3+i*4
\end{lstlisting}

\RU{А иначе, значение из \TT{ar2} загрузится в \XMM{0} используя инструкцию \MOVDQU, 
которая не требует выровненного указателя, зато может работать чуть медленнее:}
\EN{Otherwise, value from \TT{ar2} will be loaded into \XMM{0} using \MOVDQU,
it does not require aligned pointer, but may work slower:}

\lstinputlisting{patterns/19_SIMD/18_1_excerpt.asm.\LANG}

\RU{А во всех остальных случаях, будет исполняться код, который был бы, как если бы не была 
включена поддержка SSE2.}
\EN{In all other cases, non-SSE2 code will be executed.}

\subsubsection{GCC}

\newcommand{\URLGCCVEC}{\url{http://go.yurichev.com/17083}}

\RU{Но и GCC умеет кое-что векторизировать\footnote{Подробнее о векторизации в GCC: \URLGCCVEC}, 
если компилировать с опциями \Othree и включить поддержку SSE2: \TT{-msse2}.}
\EN{GCC may also vectorize in a simple cases\footnote{More about GCC vectorization support: \URLGCCVEC},
if to use \Othree option and to turn on SSE2 support: \TT{-msse2}.}

\RU{Вот что вышло}\EN{What we got} (GCC 4.4.1):

\lstinputlisting{patterns/19_SIMD/18_2_gcc_O3.asm}

\RU{Почти то же самое, хотя и не так дотошно, как Intel C++.}
\EN{Almost the same, however, not as meticulously as Intel C++ doing it.}

\subsection{\RU{Пример копирования блоков}\EN{Memory copy example}}
\label{vec_memcpy}

\RU{Вернемся к простому примеру memcpy()}\EN{Let's revisit simple memcpy() example}
(\ref{loop_memcpy}):

\lstinputlisting{memcpy.c}

\RU{И вот что делает оптимизирующий GCC 4.9.1:}
\EN{And that's what optimizing GCC 4.9.1 did:}

\lstinputlisting[caption=\Optimizing GCC 4.9.1 x64]{patterns/19_SIMD/memcpy_GCC49_x64_O3.s.\LANG}
