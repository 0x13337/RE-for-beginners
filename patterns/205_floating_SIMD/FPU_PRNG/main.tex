\section{\RU{И снова пример генератора случайных чисел}\EN{Pseudo-random number generator example revisited}}
\label{FPU_PRNG_SIMD}

\RU{Вернемся к примеру \q{пример генератора случайных чисел} \lstref{FPU_PRNG}.}
\EN{Let's revisit \q{pseudo-random number generator example} example \lstref{FPU_PRNG}.}

\RU{Если скомпилировать это в MSVC 2012, компилятор будет использовать SIMD-инструкции для FPU.}
\EN{If we compile this in MSVC 2012, it will use the SIMD instructions for the FPU.}

\lstinputlisting[caption=\Optimizing MSVC 2012]{patterns/205_floating_SIMD/FPU_PRNG/MSVC2012_Ox_Ob0.asm.\LANG}

% FIXME1 rewrite!
\RU{У всех инструкций суффикс -SS, это означает \q{Scalar Single}.}
\EN{All instructions have the -SS suffix, which stands for \q{Scalar Single}.}
\RU{\q{Scalar} означает что только одно значение хранится в регистре.}
\EN{\q{Scalar} implies that only one value is stored in the register.}
\RU{\q{Single} означает что это тип \Tfloat.}
\EN{\q{Single} stands for \Tfloat data type.}
