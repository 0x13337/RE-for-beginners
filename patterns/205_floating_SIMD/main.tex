\section{\IFRU{Работа с числами с плавающей запятой в x64 используя SIMD}
{Working with float point numbers using SIMD in x64}}

\label{floating_SIMD}
\index{IEEE 754}
\index{SIMD}
\index{SSE}
\index{SSE2}
\IFRU{Разумеется, FPU остался в x86-совместимых процессорах в то время, когда ввели расширение x64}
{Of course, \ac{FPU} remained in x86-compatible processors, when x64 extension was added}.
\IFRU{Но в то время уже везде были \ac{SIMD}-расширения (\ac{SSE}, SSE2, итд), которые, к тому же, тоже умеют работать
с числами с плавающей запятой}{But at the time, \ac{SIMD}-extensions (\ac{SSE}, SSE2, etc) were already present,
which can work with float point numbers as well}. \IFRU{Формат чисел остается тот же}{Number format
remaining the same} (IEEE 754).

\index{x86-64}
\IFRU{Так что, компиляторы для x86-64 обычно используют \ac{SIMD}-инструкции}{So, x86-64 compilers are usually
use \ac{SIMD}-instructions}.
\IFRU{Это, можно сказать, хорошая новость, потому что работать с ними легче}
{It can be said, it's a good news, because it's easier to work with them}.

\IFRU{Примеры будем использовать из секции о FPU}{We will reuse here examples from the FPU section} \ref{sec:FPU}.

\subsection{\IFRU{Простой пример}{Simple example}}

\lstinputlisting{patterns/12_FPU/simple.c}

\lstinputlisting[caption=MSVC 2012 x64 /Ox]{patterns/205_floating_SIMD/simple_MSVC_2012_x64_Ox.asm}

\IFRU{Собственно, входные значения с плавающей запятой передаются через регистры \TT{XMM0}-\TT{XMM3}, 
а остальные --- через стек}{Input floating point values are passed in \TT{XMM0}-\TT{XMM3} registers,
all the rest---via stack}
\footnote{\href{http://msdn.microsoft.com/en-us/library/zthk2dkh.aspx}{MSDN: Parameter Passing}}.

$a$ \IFRU{передается через}{is passed in} \TT{XMM0}, $b$\EMDASH{}\IFRU{через}{via} \TT{XMM1}.
\IFRU{Но XMM-регистры (как мы уже знаем из секции о \ac{SIMD}\ref{SIMD_x86}) 128-битные, 
а значения типа \Tdouble --- 64-битные,
так что используется только младшая половина регистра}
{XMM-registers are 128-bit (as we know from the section about \ac{SIMD}\ref{SIMD_x86}), 
but \Tdouble values---64 bit ones, so only lower register half is used}.

\index{x86!\Instructions!DIVSD}
\TT{DIVSD} \IFRU{это SSE-инструкия, означает}{is SSE-instruction, meaning} 
``Divide Scalar Double-Precision Floating-Point Values'', 
\IFRU{и просто делит значение типа \Tdouble на другое, лежащие в младших половинах операндов}{it just divides
one value of \Tdouble type by another, stored in the lower halves of operands}.

\IFRU{Константы закодированы компилятором в формате IEEE 754}{Constants are encoded by compiler in IEEE 754 format}.

\index{x86!\Instructions!MULSD}
\index{x86!\Instructions!ADDSD}
\TT{MULSD} \AndENRU \TT{ADDSD} \IFRU{работают так же, только производят умножение и сложение}
{works just as the same, but doing multiplication and addition}.

\IFRU{Результат работы ф-ции типа \Tdouble ф-ция оставляет в регистре \TT{XMM0}}
{The result of \Tdouble type the function leaves in \TT{XMM0} register}.\\
\\
\IFRU{Как работает неоптимизирующий MSVC}{That is how non-optimizing MSVC works}:

\lstinputlisting[caption=MSVC 2012 x64]{patterns/205_floating_SIMD/simple_MSVC_2012_x64.asm}

\index{Shadow space}
\IFRU{Чуть более избыточно}{Slightly redundant}. 
\IFRU{Входные аргументы сохраняются в}{Input arguments are saved in} ``shadow space'' (\ref{shadow_space}), 
\IFRU{причем, только младшие половины регистров, т.е., только 64-битные значения типа \Tdouble}
{but only lower register halves, i.e., only 64-bit values of \Tdouble type}.\\
\\
\IFRU{Результат работы компилятора GCC точно такой же}{GCC produces very same code}.

\subsection{\IFRU{Передача чисел с плавающей запятой в аргументах}{Passing floating point number via arguments}}

\lstinputlisting{patterns/12_FPU/pow.c}

\IFRU{Они передаются в младших половинах регистров}{They are passed in lower halves
of the} \TT{XMM0}-\TT{XMM3}\EN{ registers}.

\lstinputlisting[caption=MSVC 2012 x64 /Ox]{patterns/205_floating_SIMD/pow_MSVC_2012_x64_Ox.asm}

\index{x86!\Instructions!MOVSD}
\index{x86!\Instructions!MOVSDX}
\IFRU{Инструкции}{There are no} \TT{MOVSDX} \IFRU{нет в документации от}{instruction in} 
Intel\cite{Intel} \AndENRU AMD\cite{AMD}\EN{ manuals}, 
\IFRU{там она называется просто}{it is called there just} \TT{MOVSD}.
\IFRU{Таким образом, в процессорах x86 две инструкции с одинаковым именем}{So there are two instructions
sharing the same name in x86} (\IFRU{о второй}{about other}: \ref{REP_MOVSx}).
\IFRU{Возможно, в Microsoft решили избежать
путанницы и переименовали инструкцию в}{Apparently, Microsoft developers wanted to get rid of mess,
so they renamed it into} \TT{MOVSDX}.
\IFRU{Она просто загружает значение в младшую половину XMM-регистра}{It just loads a value into
lower half of XMM-register}.

\RU{Ф-ция }\TT{pow()} \IFRU{берет аргументы из}{takes arguments from} \TT{XMM0} \AndENRU \TT{XMM1}, 
\IFRU{и возвращает результат в}{and returning result in} \TT{XMM0}.
\IFRU{Далее он перекладывается в}{It is then moved into} \RDX \ForENRU \printf. 
\IFRU{Почему}{Why}? 
\IFRU{Честно говоря, не знаю, может быть, это потому что}{Honestly speaking, I don't know, maybe because} 
\printf\EMDASH{}\IFRU{ф-ция с переменным количеством аргументов}{is a variable arguments function}?

\lstinputlisting[caption=GCC 4.4.6 x64 -O3]{patterns/205_floating_SIMD/pow_GCC446_x64_O3_\IFRU{ru}{en}.s}

GCC \IFRU{работает понятнее}{making more clear result}. 
\IFRU{Значение для}{Value for} \printf \IFRU{передается в}{is passed in} \TT{XMM0}. 
\IFRU{Кстати, вот тот случай, когда в}{By the way, here is a case when 1 is written into} \EAX
\ForENRU \printf \IFRU{записывается 1 --- это значит, что будет передан один аргумент в векторных регистрах, 
так того требует стандарт}{---this mean that one argument will be passed in vector registers,
just as the standard requires} \cite{SysVABI}.

\subsection{\IFRU{Пример с сравнением}{Comparison example}}

\lstinputlisting{patterns/12_FPU/d_max.c}

\lstinputlisting[caption=MSVC 2012 x64 /Ox]{patterns/205_floating_SIMD/d_max_MSVC_2012_x64_Ox.asm}

\Optimizing MSVC \IFRU{генерирует очень понятный код}{generates very easy code to understand}.

\index{x86!\Instructions!COMISD}
\RU{Инструкция }\TT{COMISD} \IFRU{это}{is} ``Compare Scalar Ordered Double-Precision Floating-Point 
Values and Set EFLAGS''. \IFRU{Собственно, это она и делает}{Essentially, that is what it does}.\\
\\
\NonOptimizing MSVC \IFRU{генерирует более избыточно, но тоже всё понятно}{generates more redundant code,
but it is still not hard to understand}:

\lstinputlisting[caption=MSVC 2012 x64]{patterns/205_floating_SIMD/d_max_MSVC_2012_x64.asm}

\index{x86!\Instructions!MAXSD}
\IFRU{А вот}{However,} GCC 4.4.6 \IFRU{дошел в оптимизации дальше и применил инструкцию}
{did more optimizing and used the} \TT{MAXSD} (``Return Maximum Scalar 
Double-Precision  Floating-Point Value'')\IFRU{, которая просто выбирает максимальное значение}{ instruction,
which just choose maximal value}!

\lstinputlisting[caption=GCC 4.4.6 x64 -O3]{patterns/205_floating_SIMD/d_max_GCC446_x64_O3.s}

\subsection{\IFRU{Краткий итог}{Summary}}

\IFRU{Во всех приведенных примерах, в XMM-регистрах используется только младшая половина регистра, там
хранится значение в формате IEEE 754}{Only lower half of XMM-registers are used in all examples here, 
a number in IEEE 754 format is stored there}.

\IFRU{Собственно, все инструкции с суффиксом}{Essentially, all instructions prefixed by} 
\TT{-SD} (``Scalar Double-Precision'')\EMDASH{}\IFRU{это инструкции для работы с числами с плавающей 
запятой в формате IEEE 754, 
хранящиеся в младшей 64-битной половине XMM-регистра}{are instructions working with float point numbers
in IEEE 754 format stored in the lower 64-bit half of XMM-register}.

\IFRU{Всё удобнее чем это было в FPU, видимо, сказывается тот факт что расширения 
SIMD развивались не так хаотично как FPU в прошлом}
{And it is easier than FPU, apparently because SIMD extensions were evolved not as chaotic as FPU in the past}.
\IFRU{Стековая модель регистров не используется}{Stack register model is not used}.

\index{x86!\Instructions!ADDSS}
\index{x86!\Instructions!MOVSS}
\index{x86!\Instructions!COMISS}
\IFRU{Если вы попробуете заменить в этих примерах}{If you would try to replace} \Tdouble \IFRU{на}{to} \Tfloat
\IFRU{, то инструкции будут использоваться те же,
только с суффиксом}{in these examples, the same instructions will be used, but prefixed with} \TT{-SS} 
(``Scalar Single-Precision''), \IFRU{например}{for example}, \TT{MOVSS}, \TT{COMISS}, \TT{ADDSS}, \IFRU{итд}{etc}.

``Scalar'' \IFRU{означает что SIMD-регистр будет хранить только одно значение, вместо нескольких}{mean that
SIMD-register will contain only one value instead of several}.
\IFRU{Инструкции, работающие с несколькими значениями в регистре одновременно, имеют ``Packed'' в названии}
{Instructions working with several values in a register simultaneously, has ``Packed'' in the name}.

