\chapter{\RU{64 бита}\EN{64 bits}}

\section{x86-64}
\myindex{x86-64}
\label{x86-64}

\RU{Это расширение x86-архитуктуры до 64 бит.}\EN{It is a 64-bit extension to the x86 architecture.}

\RU{С точки зрения начинающего reverse engineer-а, наиболее важные отличия от 32-битного x86 это:}
\EN{From the reverse engineer's perspective, the most important changes are:}

\myindex{\CLanguageElements!\Pointers}
\begin{itemize}

\item
\RU{Почти все регистры (кроме FPU и SIMD) расширены до 64-бит и получили префикс R-. 
И еще 8 регистров добавлено. 
В итоге имеются эти \ac{GPR}-ы:}
\EN{Almost all registers (except FPU and SIMD) were extended to 64 bits and got a R- prefix.
8 additional registers wer added.
Now \ac{GPR}'s are:} \RAX, \RBX, \RCX, \RDX, 
\RBP, \RSP, \RSI, \RDI, \Reg{8}, \Reg{9}, \Reg{10}, 
\Reg{11}, \Reg{12}, \Reg{13}, \Reg{14}, \Reg{15}. 

\RU{К ним также можно обращаться так же, как и прежде. Например, для доступа к младшим 32 битам \RAX 
можно использовать \EAX:}
\EN{It is still possible to access the \IT{older} register parts as usual. 
For example, it is possible to access the lower 32-bit part of the \RAX register using \EAX:}

\RegTableOne{RAX}{EAX}{AX}{AH}{AL}

\RU{У новых регистров \GTT{R8-R15} также имеются их \IT{младшие части}: \GTT{R8D-R15D} 
(младшие 32-битные части), 
\GTT{R8W-R15W} (младшие 16-битные части), \GTT{R8L-R15L} (младшие 8-битные части).}
\EN{The new \GTT{R8-R15} registers also have their \IT{lower parts}: \GTT{R8D-R15D} (lower 32-bit parts),
\GTT{R8W-R15W} (lower 16-bit parts), \GTT{R8L-R15L} (lower 8-bit parts).}

\RegTableFour{R8}{R8D}{R8W}{R8L}

\RU{Удвоено количество SIMD-регистров: с 8 до 16:}
\EN{The number of SIMD registers was doubled from 8 to 16:} \XMM{0}-\XMM{15}.

\item
\RU{В win64 передача всех параметров немного иная, это немного похоже на fastcall 
(\myref{fastcall})
.
Первые 4 аргумента записываются в регистры \RCX, \RDX, \Reg{8}, \Reg{9}, а остальные ~--- в стек. 
Вызывающая функция также должна подготовить место из 32 байт чтобы вызываемая функция могла сохранить 
там первые 4 аргумента и использовать эти регистры по своему усмотрению. 
Короткие функции могут использовать аргументы прямо из регистров, но б\'{о}льшие функции могут сохранять 
их значения на будущее.}
\EN{In Win64, the function calling convention is slightly different, somewhat resembling fastcall
(\myref{fastcall})
.
The first 4 arguments are stored in the \RCX, \RDX, \Reg{8}, \Reg{9} registers, the rest~---in the stack.
The \gls{caller} function must also allocate 32 bytes so the \gls{callee} may save there 4 first arguments and use these 
registers for its own needs.
Short functions may use arguments just from registers, but larger ones may save their values on the stack.}

\RU{Соглашение }System V AMD64 ABI (Linux, *BSD, \MacOSX)\cite{SysVABI} \RU{также напоминает}\EN{also somewhat resembles}
fastcall, \RU{использует 6 регистров}\EN{it uses 6 registers} 
\RDI, \RSI, \RDX, \RCX, \Reg{8}, \Reg{9} \RU{для первых шести аргументов}\EN{for the first 6 arguments}.
\RU{Остальные передаются через стек}\EN{All the rest are passed via the stack}.

\RU{См. также в соответствующем разделе о способах передачи аргументов через стек}
\EN{See also the section on calling conventions}~(\myref{sec:callingconventions}).

\item
\RU{\Tint в \CCpp остается 32-битным для совместимости.}
\EN{The \CCpp \Tint type is still 32-bit for compatibility.}

\item
\RU{Все указатели теперь 64-битные}\EN{All pointers are 64-bit now}.

\RU{На это иногда сетуют: ведь теперь для хранения всех указателей нужно в 2 раза больше места 
в памяти, в т.ч. и в кэш-памяти, не смотря на то что x64-процессоры могут адресовать только 48 бит
внешней \ac{RAM}.}
\EN{This provokes irritation sometimes: now one needs twice as much memory for storing pointers,
including cache memory, despite the fact that x64 \ac{CPU}s can address only 48 bits of external 
\ac{RAM}.}

\end{itemize}

\myindex{Register allocation}
\RU{Из-за того, что регистров общего пользования теперь вдвое больше, у компиляторов теперь больше 
свободного места для маневра, называемого \glslink{register allocator}{register allocation}.
Для нас это означает, что в итоговом коде будет меньше локальных переменных.}
\EN{Since now the number of registers is doubled, the compilers have more space for maneuvering called 
\glslink{register allocator}{register allocation}.
For us this implies that the emitted code containing less number of local variables.}

\myindex{DES}
\RU{Для примера, функция вычисляющая первый S-бокс алгоритма шифрования DES, 
она обрабатывает сразу 32/64/128/256 значений, в зависимости от типа \GTT{DES\_type} (uint32, uint64, SSE2 или AVX), 
методом bitslice DES (больше об этом методе читайте здесь~(\myref{bitslicedes})):}
\EN{For example, the function that calculates the first S-box of the DES encryption algorithm processes
32/64/128/256 values at once (depending on \GTT{DES\_type} type (uint32, uint64, SSE2 or AVX)) 
using the bitslice DES method
(read more about this technique here ~(\myref{bitslicedes})):}

\lstinputlisting{patterns/20_x64/19_1.c}

\RU{Здесь много локальных переменных. Конечно, далеко не все они будут в локальном стеке. 
Компилируем обычным MSVC 2008 с опцией \Ox:}
\EN{There are a lot of local variables. 
Of course, not all those going into the local stack.
Let's compile it with MSVC 2008 with \Ox option:}

\lstinputlisting[caption=\Optimizing MSVC 2008]{patterns/20_x64/19_2_msvc_Ox.asm}

\RU{5 переменных компилятору пришлось разместить в локальном стеке.}
\EN{5 variables were allocated in the local stack by the compiler.}

\RU{Теперь попробуем то же самое только в 64-битной версии MSVC 2008:}
\EN{Now let's try the same thing in the 64-bit version of MSVC 2008:}

\lstinputlisting[caption=\Optimizing MSVC 2008]{patterns/20_x64/19_3_msvc_x64.asm}

\RU{Компилятор ничего не выделил в локальном стеке, а \GTT{x36} это синоним для \GTT{a5}.}
\EN{Nothing was allocated in the local stack by the compiler, \GTT{x36} is synonym for \GTT{a5}.}

\iffalse
% FIXME1 невнятно
\RU{Кстати, видно, что функция сохраняет регистры \RCX, \RDX в отведенных для 
этого вызываемой функцией местах, 
а \Reg{8} и \Reg{9} не сохраняет, а начинает использовать их сразу.}
\EN{By the way, we can see here that the function saved the \RCX and \RDX registers in space allocated by the \gls{caller},
but \Reg{8} and \Reg{9} were not saved but used from the beginning.}
\fi

\RU{Кстати, существуют процессоры с еще большим количеством \ac{GPR}, например, 
Itanium ~--- 128 регистров.}
\EN{By the way, there are CPUs with much more \ac{GPR}'s, e.g. Itanium (128 registers).}

\ifdefined\IncludeARM
\section{ARM}

\RU{64-битные инструкции появились в}\EN{64-bit instructions appeared in} ARMv8.
\fi

\section{\RU{Числа с плавающей запятой}\EN{Float point numbers}}

\RU{О том как происходит работа с числами с плавающей запятой в x86-64, читайте здесь: \myref{floating_SIMD}.}
\EN{How floating point numbers are processed in x86-64 is explained here: \myref{floating_SIMD}.}
