\chapter{\RU{Inline-функции}\EN{Inline functions}}
\index{Inline code}
\label{inline_code}

\RU{Inline-код это когда компилятор, вместо того чтобы генерировать инструкцию вызова небольшой функции,
просто вставляет её тело прямо в это место.}
\EN{Inlined code is when compiler, instead of placing call instruction to small or tiny function,
just placing its body right in-place.}

\lstinputlisting[caption=\RU{Простой пример}\EN{Simple example}]{patterns/22_inline_function/1.c}

\RU{... это компилируется вполне предсказуемо, хотя, если включить оптимизации GCC (\Othree), мы увидим:}
\EN{... is compiled in very predictable way, however, if to turn on GCC optimization (\Othree), we'll see:}

\lstinputlisting[caption=GCC 4.8.1 \Othree]{patterns/22_inline_function/1.s}

(\RU{Здесь деление заменено умножением}\EN{Here division is done by multiplication}(\ref{sec:divisionbynine}).)

\RU{Да, наша маленькая ф-ция \TT{celsius\_to\_fahrenheit()} была помещена прямо перед вызовом \printf.}
\EN{Yes, our small function \TT{celsius\_to\_fahrenheit()} was just placed before \printf call.}
\RU{Почему? Это может быть быстрее чем исполнять код самой ф-ции плюс затраты на вызов и возврат.}
\EN{Why? It may be faster than executing this function's code plus calling/returning overhead.}

\RU{В прошлом, такие ф-ции нужно было маркировать ключевым словом ``inline'' в определении ф-ции, хотя,
в наше время, такие ф-ции выбираются компилятором автоматически.}
\EN{In past, such function must be marked with ``inline'' keyword in function's declaration, however,
in modern times, these functions are chosen automatically by compiler.}

% sections
\section{\RU{Ф-ции работы со строками и памятью}\EN{Strings and memory functions}}

\RU{Другая очень частая оптимизация это вставка кода строковых ф-ций таких как}
\EN{Another very common automatic optimization tactic is inlining of string functions like}
\IT{strcpy()}, \IT{strcmp()}, \IT{strlen()}, \IT{memcmp()}, \IT{memcpy()}, \RU{и т.д}\EN{etc}.

\RU{Иногда это быстрее, чем вызывать отдельную ф-цию.}\EN{Sometimes it's faster then to call separate function.}

\RU{Это очень часто встречающшиеся шаблонные вставки, которые желательно распозновать 
reverse engineer-ам ``на глаз''.}
\EN{These are very frequent patterns, which are highly advisable to reverse engineers 
to learn to detect automatically.}

% subsections
\subsection{strcmp()}
\index{\CStandardLibrary!strcmp()}

\lstinputlisting[caption=\RU{пример с strcmp()}\EN{strcmp() example}]{patterns/22_inline_function/str_mem/strcmp.c}

\lstinputlisting[caption=\Optimizing GCC 4.8.1]{patterns/22_inline_function/str_mem/strcmp_GCC_O3.s}

\lstinputlisting[caption=\Optimizing MSVC 2010]{patterns/22_inline_function/str_mem/strcmp_MSVC_2010_Ox.asm}

\subsection{strlen()}

\lstinputlisting[caption=\RU{пример с strlen()}\EN{strlen() example}]{patterns/22_inline_function/str_mem/strlen.c}

\lstinputlisting[caption=MSVC 2010 /Ox]{patterns/22_inline_function/str_mem/strlen_MSVC_2010_Ox.asm}

\subsection{strcpy()}

\lstinputlisting[caption=\RU{пример с strcpy()}\EN{strcpy() example}]{patterns/22_inline_function/str_mem/strcpy.c}

\lstinputlisting[caption=MSVC 2010 /Ox]{patterns/22_inline_function/str_mem/strcpy_MSVC_2010_Ox.asm}

\subsection{memcpy()}

\subsubsection{\RU{Короткие блоки}\EN{Short blocks}}
\label{copying_short_blocks}

\RU{Если нужно скопировать немного байт, то, нередко, 
\TT{memcpy()} заменяется на несколько инструкций \MOV.}
\EN{Short block copy routine is often implemented as pack of \MOV instructions.}

\lstinputlisting[caption=\RU{пример с memcpy()}\EN{memcpy() example}]{patterns/22_inline_function/str_mem/memcpy_7.c}

\lstinputlisting[caption=MSVC 2010 /Ox]{patterns/22_inline_function/str_mem/memcpy_7_MSVC_2010_Ox.asm}

\lstinputlisting[caption=GCC 4.8.1 \Othree]{patterns/22_inline_function/str_mem/memcpy_7_GCC_O3.s}

\RU{Обынчо это происходит так: в начале копируются 4-байтные блоки, затем 16-битное слово (если нужно), 
затем последний байт (если нужно).}
\EN{That's usually done as follows: 4-byte blocks are copied first, then 16-bit word (if needed), 
then the last byte (if needed).}

\RU{Точно так же при помощи \MOV копируются структуры}\EN{Structures are also copied using
\MOV}: \ref{short_struct_copying_using_MOV}.

\subsubsection{\RU{Длинные блоки}\EN{Long blocks}}

\RU{Здесь компиляторы ведут себя по-разному.}\EN{Compilers behave differently here.}

\lstinputlisting[caption=\RU{пример с memcpy()}\EN{memcpy() example}]{patterns/22_inline_function/str_mem/memcpy.c}

\RU{При копировании 128 байт, MSVC может обойтись одной инструкцией \TT{MOVSD} (ведь 128 кратно 4):}
\EN{While copying 128 bytes, MSVC can do this with single \TT{MOVSD} instruction (because 128 
divides evenly by 4):}

\lstinputlisting[caption=MSVC 2010 /Ox]{patterns/22_inline_function/str_mem/memcpy_128_MSVC_2010_Ox.asm}

\RU{При копировании 123-х байт, в начале копируется 30 32-битных слов при помощи \TT{MOVSD} 
(это 120 байт), 
затем копируется 2 байта при помощи \TT{MOVSW}, 
затем еще один байт при помощи \TT{MOVSB}.}
\EN{When 123 bytes are copying, 30 32-byte words are copied first using instruction \TT{MOVSD}
(that's 120 bytes),
then 2 bytes are copied using \TT{MOVSW}, 
then one more byte using \TT{MOVSB}.}

\lstinputlisting[caption=MSVC 2010 /Ox]{patterns/22_inline_function/str_mem/memcpy_123_MSVC_2010_Ox.asm}

\RU{GCC во всех случаях вставляет большую универсальную ф-цию, работающую для всех размеров блоков:}
\EN{GCC uses one big universal functions, working for any block size:}

\lstinputlisting[caption=GCC 4.8.1 \Othree]{patterns/22_inline_function/str_mem/memcpy_GCC.s}

\RU{Универсальные ф-ции копирования блоков обычно работают по следующей схеме: 
вычислить, сколько 32-битных слов
можно скопировать, затем сделать это при помощи \TT{MOVSD}, затем скопировать остатки.}
\EN{Universal memory copy functions are usually works as follows:
calculate, how many 32-bit words can be copied, then copy then by \TT{MOVSD}, then copy
remaining bytes.}

\RU{Более сложные ф-ции копирования используют \ac{SIMD} и учитывают выравнивание.}
\EN{More complex copy functions uses \ac{SIMD} instructions and take aligning into consideration.}

\subsection{memcmp()}
\index{\CStandardLibrary!memcmp()}

\lstinputlisting[caption=\RU{пример с memcmp()}\EN{memcmp() example}]{patterns/22_inline_function/str_mem/memcmp.c}

\RU{Для блоков разной длины, MSVC 2010 вставляет одну и ту же универсальную ф-цию:}
\EN{For any block size, MSVC 2010 inserts the same universal function:}

\lstinputlisting[caption=\Optimizing MSVC 2010]{patterns/22_inline_function/str_mem/memcmp_MSVC_2010_Ox.asm}


\subsection{\RU{Скрипт для IDA}\EN{IDA script}}

\index{IDA}
\RU{Я написал небольшой скрипт для \IDA для поиска и сворачивания таких очень часто 
попадающихся inline-функций:}
\EN{I wrote small \IDA script for searching and folding such very frequently seen pieces of 
inline code:} \\
\url{\YurichevIDAIDCScripts}.


