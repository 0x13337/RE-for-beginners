\chapter{\RU{Неверно дизассемблированный код}\EN{Incorrectly disassembled code}}
\label{sec:incorrectly_disasmed_code}

\RU{Практикующие reverse engineer-ы часто сталкиваются с неверно дизассемблированным кодом}
\EN{Practicing reverse engineers often dealing with incorrectly disassembled code}.

\section{\RU{Дизассемблирование началось в неверном месте}\EN{Disassembling started incorrectly} (x86)}

\RU{В отличие от ARM и MIPS (где у каждой инструкции длина или 2 или 4 байта), x86-инструкции имеют переменную длину,
так что, любой дизассемблер, начиная работу с середины x86-инструкции, может выдать неверные результаты.}
\EN{Unlike ARM and MIPS (where any instruction has length of 2 or 4 bytes), x86 instructions has variable size,
so, any disassembler, starting at the middle of x86 instruction, may produce incorrect results.}

\RU{Как пример}\EN{As an example}:

\lstinputlisting{patterns/23_incorrect_disassembly/x86_wrong_start.asm}

\RU{В начале мы видим неверно дизассемблированные инструкции, но потом, так или иначе, дизассемблер находит верный след}
\EN{There are incorrectly disassembled instructions at the beginning, but eventually, disassembler finds right 
track}.

\section{\RU{Как выглядят случайные данные в дизассемблированном виде}\EN{How random noise looks disassembled}?}

\RU{Общее, что можно сразу заметить, это}\EN{Common properties which can be easily spotted are}:

\begin{itemize}
\item \RU{Необычно большой разброс инструкций}\EN{Unusually big instruction dispersion}.
\index{x86!\Instructions!PUSH}
\index{x86!\Instructions!MOV}
\index{x86!\Instructions!CALL}
\index{x86!\Instructions!IN}
\index{x86!\Instructions!OUT}
\RU{Самые частые x86-инструкции это}\EN{Most frequent x86 instructions are} \PUSH{}, \MOV{}, \CALL{}, \RU{но здесь мы видим
инструкции из любых групп: \ac{FPU}-инструкции, инструкции \TT{IN}/\TT{OUT}, редкие и системные инструкции, всё друг с другом смешано 
в одном месте}\EN{but here we will see
instructions from any instruction group: \ac{FPU} instructions, \TT{IN}/\TT{OUT} instructions, rare and system instructions,
everything messed up in one single place}.

\item \RU{Большие и случайные значения, смещения,}\EN{Big and random values, offsets and} immediates.

\item \RU{Переходы с неверными смещениями часто имеют адрес перехода в середину другой инструкции}
\EN{Jumps having incorrect offsets often jumping into the middle of another instructions}.
\end{itemize}

\lstinputlisting[caption=\randomNoise{} (x86)]{patterns/23_incorrect_disassembly/x86.asm}

\index{x86-64}
\lstinputlisting[caption=\randomNoise{} (x86-64)]{patterns/23_incorrect_disassembly/x64.asm}

\index{ARM}
\lstinputlisting[caption=\randomNoise{} (ARM (\ARMMode))]{patterns/23_incorrect_disassembly/ARM.asm}

\lstinputlisting[caption=\randomNoise{} (ARM (\ThumbMode))]{patterns/23_incorrect_disassembly/ARM_thumb.asm}

\index{MIPS}
\lstinputlisting[caption=\randomNoise (MIPS little endian)]{patterns/23_incorrect_disassembly/MIPS.asm}

\RU{Также важно помнить, что хитрым образом написанный код для распаковки и дешифровки (включая самомодифицирующийся),
также может выглядеть как случайный шум, тем не менее, он исполняется корректно}\EN{It is also important to keep in mind that 
cleverly constructed unpacking and decrypting code 
(including self-modifying) may looks like noise as well, nevertheless, it executes correctly}.

\section{\RU{Информационная энтропия среднестатистического кода}\EN{Information entropy of average code}}

\index{\RU{Информационная энтропия}\EN{Information entropy}}
\RU{Результаты работы утилиты \IT{ent}}\EN{\IT{ent} utility results}\footnote{\url{http://www.fourmilab.ch/random/}}.

(\RU{Энтропия идеально сжатого (или зашифрованного) файла\EMDASH{}8 бит на байт; файла с нулями любой длины\EMDASH{}0 бит на байт.}
\EN{Entropy of ideally compressed (or encrypted) file is 8 bits per byte; of zero file of arbitrary size if 0 bits per byte.})

\RU{Здесь видно что код для CPU с 4-байтными инструкциями (ARM в режиме ARM и MIPS) наименее экономичны в этом смысле.}
\EN{Here we can see that a code for CPU with 4-byte instructions (ARM in ARM mode and MIPS) is least effective in this sense.}

\subsection{x86}

\RU{Секция \TT{.text} файла \TT{ntoskrnl.exe} из}
\EN{\TT{.text} section of \TT{ntoskrnl.exe} file from} Windows 2003:

\begin{lstlisting}
Entropy = 6.662739 bits per byte.

Optimum compression would reduce the size
of this 593920 byte file by 16 percent.
...
\end{lstlisting}

\RU{Секция \TT{.text} файла}\EN{\TT{.text} section of} \TT{ntoskrnl.exe} \RU{из}\EN{from} Windows 7 x64:

\begin{lstlisting}
Entropy = 6.549586 bits per byte.

Optimum compression would reduce the size
of this 1685504 byte file by 18 percent.
...
\end{lstlisting}

\subsection{ARM (\ThumbMode)}
\index{ARM}

AngryBirds Classic:

\begin{lstlisting}
Entropy = 7.058766 bits per byte.

Optimum compression would reduce the size
of this 3336888 byte file by 11 percent.
...
\end{lstlisting}

\subsection{ARM (\ARMMode)}

Linux Kernel 3.8.0:

\begin{lstlisting}
Entropy = 6.036160 bits per byte.

Optimum compression would reduce the size
of this 6946037 byte file by 24 percent.
...
\end{lstlisting}

\subsection{MIPS (little endian)}
\index{MIPS}

\RU{Секция \TT{.text} файла}\EN{\TT{.text} section of} \TT{user32.dll} \RU{из}\EN{from} Windows NT 4:

\begin{lstlisting}
Entropy = 6.098227 bits per byte.

Optimum compression would reduce the size
of this 433152 byte file by 23 percent.
....
\end{lstlisting}

