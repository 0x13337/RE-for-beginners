\chapter{Duff's device}
\index{Duff's device}

Duff's device 
\EN{\footnote{\href{http://go.yurichev.com/17137}{wikipedia}}}
\RU{\footnote{\href{http://go.yurichev.com/17138}{wikipedia}}}
\RU{это развернутый цикл с возможностью перехода внутри цикла.}
\EN{is an unrolled loop with a feature to jump inside of it.}
\RU{Развернутый цикл реализован используя fallthrough-выражение switch().}
\EN{Unrolled loop is implemented using fallthrough switch() statement.}

\RU{Я буду использовать здесь упрощенную версию кода Тома Даффа.}
\EN{I would use here a slightly simplified version of Tom Duff's original code.}

\RU{Скажем, нам нужно написать ф-цию, очищающую регион в памяти.}
\EN{Let's say, we need to write a function which clears a region in memory.}
\RU{Кто-то может подумать о простом цикле, очищающем байт за байтом.}
\EN{Someone can think about simple loop, clearing byte per byte.}
\RU{Это, очевидно, медленно, так как все современные компьютеры имеют намного более широкую шину памяти.}
\EN{It's obviously slow, since all modern computers has much wider memory bus.}
\RU{Так что более правильный способ это очищать регион в памяти блоками по 4 или 8 байт.}
\EN{So the more right way is to clear memory region by 4 or 8 byte blocks.}
\RU{Так как мы будем работать с 64-битным примером, мы будем очищать память блоками по 8 байт.}
\EN{Since we will work with 64-bit example here, we will clear memory by 8 byte blocks.}
\RU{Пока всё хорошо}\EN{So far so good}.
\RU{Но что насчет хвоста}\EN{But what about tail}? 
\RU{Ф-ция очистки памяти будет также вызываться и для блоков с длиной не кратной 8.}
\EN{Memory clearing routine will also be called for regions of size not multiple by 8.}

\RU{Вот алгоритм}\EN{So here is an algorithm}:

\begin{itemize}
\item \RU{вычислить количество 8-байтных блоков, очистить их используя 8-байтный (64-битный) доступ к памяти;}
\EN{calculate number of 8-byte blocks, clear them using 8-byte (64-bit) memory accesses;}

\item \RU{вычислить размер хвоста, очистить его используя 1-байтный доступ к памяти.}
\EN{calculate size of tale, clear it using 1-byte memory accesses.}
\end{itemize}

\RU{Второй шаг можно реализовать используя простой цикл}\EN{The second step can be implemented using simple loop}.
\RU{Но давайте реализуем его используя развернутый цикл}\EN{But let's implement it as an unrolled loop}:

\lstinputlisting{patterns/24_duff_device/duff.c.\LANG}

\RU{В начале разберемся, как происходят вычисления.}\EN{Let's first understand how calculation is done.}
\RU{Размер региона в памяти приходит в 64-битном значении.}\EN{Memory region size is came in 64-bit value.}
\RU{И это значение можно разделить на две части:}\EN{And this value can be divided by two parts:}

% .... 7 6 5 4 3 2 1 0
%|....|B|B|B|B|B|S|S|S|

\begin{center}
\begin{bytefield}[endianness=big,bitwidth=0.03\linewidth]{11}
\bitheader{7,6,5,4,3,2,1,0} \\
\bitbox{3}{\dots} & 
\bitbox{1}{B} & 
\bitbox{1}{B} & 
\bitbox{1}{B} & 
\bitbox{1}{B} & 
\bitbox{1}{B} & 
\bitbox{1}{S} & 
\bitbox{1}{S} & 
\bitbox{1}{S}
\end{bytefield}
\end{center}

\RU{( ``B'' это количество 8-байтных блоков и ``S'' это размер хвоста в байтах ).}
\EN{( ``B'' is number of 8-byte blocks and ``S'' is size of tale in bytes ).}

\RU{Если разделить размер входного блока в памяти на 8, то значение просто сдвигается на 3 бита влево.}
\EN{If to divide input memory region size by 8, the value is just shifted by 3 bits left.}
\RU{Но для вычисления остатка, нам нужно просто изолировать младшие 3 бита!}
\EN{But to calculate remainder, we just need to isolate lowest 3 bits!}
\RU{Так что количество 8-байтных блоков вычисляется как $count>>3$, а остаток как $count \& 7$.}
\EN{So the number of 8-byte blocks calculated as $count>>3$ and remainder calculated as $count \& 7$.}

\RU{В начале, нужно определить, будем ли мы вообще исполнять 8-байтную процедуру,
так что нам нужно узнать, не больше ли $count$ чем 7.}
\EN{At first, we also need to understand, will we execute 8-byte procedure at all, so we need
to compare $count$ value, if it's greater than 7.}
\RU{Мы делаем это очищая младшие 3 бита и сравнивая результат с нулем, потому что,
всё что нам нужно узнать, это ответ на вопрос, содержит ли старшая часть значения $count$ ненулевые биты.}
\EN{We do this by clearing 3 lowest bits and comparing resulting number with zero, because, 
all we need here is to answer the question, if the high part of $count$ value is non-zero.}

\RU{Конечно, это работает потому что 8 это $2^{3}$, так что деление на числа вида $2^n$ это легко.}
\EN{Of course, this works because 8 is $2^{3}$, so division by $2^n$ numbers is that easy.}
\RU{Это невозможно с другими числами}\EN{It's not possible for other numbers}.

\RU{А на самом деле, трудно сказать, стоит ли пользоваться такими хакерскими трюками, потому что они
приводят к коду, который затем тяжело читать.}
\EN{It's actually hard to say, if these geekery tricks are worth using, because they lead
to hard-to-read code.}
\RU{С другой стороны, эти трюки очень популярны и практикующий программист, хотя может и не использовать
их, всё же должен их понимать.}
\EN{However, these tricks are very popular and practicing programmer, 
although may not use them, should nevertheless understand them.}

\RU{Так что первая часть простая: получить количество 8-байтных блоков и записать 64-битные нулевые значения
в память.}
\EN{So the first part is simple: get number of 8-byte blocks and write 64-bit zero values to memory.}

\RU{Вторая часть это развернутый цикл реализованный как fallthrough-выражение switch().}
\EN{The second part is unrolled loop implemented as fallthrough switch() statement.}
\RU{В начале, выразим на обычном русском языке, что мы хотим сделать.}
\EN{First, let's express in plain English, what we should do here.}
\RU{Мы должны ``записать столько нулевых байт в память, сколько значение $count\&7$ нам говорит''.}
\EN{We should ``write as many zero bytes in memory, as $count\&7$ value tells us''.}
\RU{Если это 0, перейти на конец, больше ничего делать не нужно.}
\EN{If it's 0, jump to the end, there are no work to do.}
\RU{Если это 1, перейти на место внутри выражения switch(), где произойдет только одна операция записи.}
\EN{If it's 1, jump to the place inside switch() statement, where only one storage operation
is to be occurred.}
\RU{Если это 2, перейти на другое место, где две операции записи будут исполнены, итд.}
\EN{If it's 2, jump to another place, where two storage operation to be executed, etc.}
\RU{7 во входном значении приведет к тому что исполнятся все 7 операций.}
\EN{7 as input value will lead to execution of all 7 operations.}
\RU{8 здесь нет, потому что регион памяти размером в 8 байт будет обработан первой частью нашей ф-ции.}
\EN{There are no 8, because memory region of 8 bytes will be processed by the first part of our function.}

\RU{Так что мы сделали развернутый цикл}\EN{So we did it as an unrolled loop}.
\RU{Это однозначно работало быстрее обычных циклов на старых компьютерах
(и наоборот, на современных процессорах короткие циклы работают быстрее развернутых).}
\EN{It was definitely faster on older computers than usual loops (on contrary, 
modern CPUs works better on short loops then on unrolled ones).}
\RU{Может быть, это всё еще может иметь смысл на современных маломощных дешевых \ac{MCU}.}
\EN{Maybe it's still can be good on the modern low-cost embedded \ac{MCU}s.}

\RU{Посмотрим что сделает оптимизирующий MSVC 2012}\EN{Let's see what optimizing MSVC 2012 does}:

\lstinputlisting{patterns/24_duff_device/duff_MSVC2012_x64_Ox.asm.\LANG}

\RU{Первая часть ф-ции выглядит для нас предсказуемо.}
\EN{First part of the function is predictable to us.}
\RU{Вторая часть это просто развернутый цикл и переход передает управление на нужную инструкцию
внутри него.}
\EN{The second part is just unrolled loop and a jump passing flow to the correct instruction
inside of it.}
\RU{Между парами инструкций \TT{MOV}/\TT{INC} никакого другого кода нет, так что исполнение
продолжается до самого конца, исполняются столько пар, сколько нужно.}
\EN{There are no other code between \TT{MOV}/\TT{INC} instruction pairs, 
so execution will fall till the very end, executing as many pairs, as needed.}

\RU{Кстати, мы можем заметить что пара \TT{MOV}/\TT{INC} занимает какое-то фиксированное количество
байт (3+3).}
\EN{By the way, we may observe that \TT{MOV}/\TT{INC} pair consumes some fixed number of bytes (3+3).}
\RU{Так что пара занимает 6 байт}\EN{So the pair consumes 6 bytes}.
\RU{Зная это, мы можем избавиться от таблицы переходов в switch(), мы можем просто умножить входное значение
на 6 и перейти на \TT{текущий\_RIP + входное\_значение * 6}.}
\EN{Knowing that, we can get rid of switch() jumptable, we can just multiple input value by 6
and jump to $current\_RIP + input\_value * 6$.}
\RU{Это будет также быстрее, потому что не нужно будет загружать элемент из таблицы переходов (\IT{jumptable}).}
\EN{This will also be faster because we do not need to fetch a value from the jumptable.}
\RU{Может быть, 6 не самая подходящая константа для быстрого умножения, и может быть оно того и не стоит,
но вы поняли идею\footnote{В качестве упражнения, вы можете попробовать переработать этот код и избавиться
от таблицы переходов.
\index{x86!\Instructions!STOSB}
Пару инструкций тоже можно переписать так что они будет занимать 4 байта или 8.
1 байт тоже возможен (используя инструкцию \TT{STOSB}).}.}
\EN{Well, 6 probably is not very good constant for fast multiplication and maybe it's not worth it,
but you've got the idea\footnote{As an exercise, you can try rework the code to get rid of 
jumptable. 
\index{x86!\Instructions!STOSB}
Instruction pair can be rewritten so it will consume 4 bytes or maybe 8. 
1 byte is also possible (using \TT{STOSB} instruction).}.}
\RU{Так в прошлом делали с развернутыми циклами олд-скульные демомейкеры.}
\EN{That is what old-school demomakers did in past with unrolled loops.}
