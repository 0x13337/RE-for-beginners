\ifdefined\RUSSIAN
\else
\chapter{Duff's device}
\index{Duff's device}

Duff's device\footnote{\url{http://en.wikipedia.org/wiki/Duff's_device}} is an unrolled loop
with a feature to jump inside of it.
Unrolled loop is implemented using fall-through switch() statement.

I would use here a slightly simplified version of Tom Duff's original code.

Let's say, we need to write a function which clears a region in memory.
Someone can think about simple loop, clearing byte per byte.
It's obviously slow, since all modern computer has much wider memory bus.
So the more right way is to clear memory region by 4 or 8 byte blocks.
Since we will work with 64-bit example here, we will clear memory by 8 byte blocks.
So far so good.
But what about tail? Memory clearing routing will also be called for regions of size
not multiple by 8.

So here is an algorithm:

\begin{itemize}
\item calculate number of 8-byte blocks, clear them using 8-byte memory accesses;
\item calculate size of tale, clear it using 1-byte memory accesses.
\end{itemize}

The second step can be implemented using simple loop.
But let's implement it as an unrolled loop:

\lstinputlisting{patterns/24_duff_device/duff.c}

Let's first understand how calculation is done.
Memory region size is came in 64-bit value. 
And it can be divided by two parts:

% .... 7 6 5 4 3 2 1 0
%|....|B|B|B|B|B|S|S|S|

\begin{center}
\begin{bytefield}[endianness=big,bitwidth=0.03\linewidth]{11}
\bitheader{7,6,5,4,3,2,1,0} \\
\bitbox{3}{\dots} & 
\bitbox{1}{B} & 
\bitbox{1}{B} & 
\bitbox{1}{B} & 
\bitbox{1}{B} & 
\bitbox{1}{B} & 
\bitbox{1}{S} & 
\bitbox{1}{S} & 
\bitbox{1}{S}
\end{bytefield}
\end{center}

( B is number of 8-byte blocks and S is size of tale in bytes ).

If to divide input memory region size by 8, the value is just shifted by 3 bits left.
But to calculate remainder, we just need to isolate lowest 3 bits!
So the number of 8-byte blocks calculated as $count>>3$ and remainder calculated as 
$count \& 7$.

At first, we also need to understand, will we execute 8-byte procude at all, so we need
to compare $count$ value, if it's greater than 7.
We do this by clearing 3 lowest bits and comparing resulting number with zero, because, 
all we need here is to answer the question, if the high part of count value is non-zero.

Of course, this works because 8 is $2^{3}$, so division by $2^n$ numbers is that easy. 
It's not possible for other numbers.

It's actually hard to say, if these geekery tricks are worth using, because they lead
to hard-to-read code. 
However, these tricks are very popular and practicing programmer, 
although may not use them, should nevertheless understand them.

So the first part is simple: get number of 8-byte blocks and write 64-bit zero values to memory.

The second part is unrolled loop implemented as fall-through switch() statement.
First, let's express in plain English, what we should do here. 
We should ``write as many zero bytes in memory, as $count\&7$ value tells us''.
If it's 0, jump to the end, there are no work to do.
If it's 1, jump to the place inside switch() statement, where only one storage operation
is to be occurred.
If it's 2, jump to another place, where two storage operation to be executed, etc.
7 as input value will lead to execution of all 7 operations.
There are no 8, because memory region of 8 bytes will be processed by the first part of our
function.

So we did it as an unrolled loop. 
It was definitely faster on older computers than usual loops (on contrary, 
modern CPUs works better on short loops then on unrolled ones). 
Maybe it's still can be good on the modern low-cost embedded \ac{MCU}s.

Let's see what \Optimizing MSVC 2012 does:

\lstinputlisting{patterns/24_duff_device/duff_MSVC2012_x64_Ox.asm}

First part of the function is predictable to us.
The second part is just unrolled loop and a jump passing flow to the correct instruction
inside of it. 
There are no other code between \TT{MOV}/\TT{INC} instruction pairs, 
so execution will fall till the very end, executing as many pairs, as needed.

By the way, we may observe that \TT{MOV}/\TT{INC} pair consumes some fixed number of bytes (3+3).
So the pair consumes 6 bytes.
Knowing that, we can get rid of switch() jumptable, we can just multiple input value by 6
and jump to $current\_RIP + input\_value * 6$.
This will also be faster because we do not need to fetch a value from the jumptable.
Well, 6 probably is not very good constant for fast multiplication and maybe it's not worth it,
but you've got the idea\footnote{As an exercise, you can try rework the code to get rid of 
jumptable. 
\index{x86!\Instructions!STOSB}
Instruction pair can be rewritten so it will consume 4 bytes or maybe 8. 
1 byte is also possible (using \TT{STOSB} instruction).}.
That is what old-school demomakers did in past with unrolled loops.
\fi
