\section{ARM: \OptimizingKeilVI (\ARMMode)}

\RU{И снова, компилятор пользуется условными инструкциями в режиме ARM, поэтому код более 
компактный.}
\EN{And again, compiler took advantage of ARM mode conditional instructions, 
so the code is much more compact.}

\lstinputlisting[caption=\OptimizingKeilVI (\ARMMode)]{patterns/28_string_trim/Keil_ARM_O3.s.\LANG}

\section{ARM: \OptimizingKeilVI (\ThumbMode)}
\index{\CompilerAnomaly}
\label{Keil_anomaly}

\RU{В режиме Thumb куда меньше условных инструкций, так что код более простой.}
\EN{There are less number of conditional instructions in Thumb mode, so the code is more ordinary.}
\RU{Но здесь есть одна странность со сдвигами на 0x20 и 0x19.}
\EN{But there are one really weird thing with 0x20 and 0x19 offsets.}
\RU{Почему компилятор Keil сделал так}\EN{Why Keil compiler did so}?
\RU{Честно говоря, я не знаю}\EN{Honestly, I have no idea}.
\RU{Возможно, это выверт процесса оптимизации компилятора.}
\EN{Probably, this is a quirk of Keil optimization process.}
\RU{Тем не менее, код будет работать корректно}\EN{Nevertheless, the code will work correctly}.

\lstinputlisting[caption=\OptimizingKeilVI (\ThumbMode)]{patterns/28_string_trim/Keil_thumb_O3.s.\LANG}
