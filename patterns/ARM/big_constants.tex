\section{\RU{Загрузка констант в регистр}\EN{Loading a constant into a register}}

\subsection{\RU{32-битный}\EN{32-bit} ARM}
\label{ARM_big_constants_loading}

\RU{Как мы уже знаем, все инструкции имеют длину в 4 байта в режиме ARM и 2 байта в режиме Thumb.}
\EN{Aa we already know, all instructions have a length of 4 bytes in ARM mode and 2 bytes in Thumb mode.}
\RU{Как в таком случае записать в регистр 32-битное число, если его невозможно закодировать
внутри одной инструкции?}
\EN{Then how can we load a 32-bit value into a register, if it's not possible to encode it in one instruction?}

\RU{Попробуем}\EN{Let's try}:

\begin{lstlisting}
unsigned int f()
{
	return 0x12345678;
};
\end{lstlisting}

\begin{lstlisting}[caption=GCC 4.6.3 -O3 \ARMMode]
f:
        ldr     r0, .L2
        bx      lr
.L2:
        .word   305419896 ; 0x12345678
\end{lstlisting}

\RU{Т.е., значение \TT{0x12345678} просто записано в памяти отдельно и загружается, если нужно.}
\EN{So, the \TT{0x12345678} value is just stored aside in memory and loaded if needed.}
\RU{Но можно обойтись и без дополнительного обращения к памяти.}
\EN{But it's possible to get rid of the additional memory access.}

\begin{lstlisting}[caption=GCC 4.6.3 -O3 -march{=}armv7-a (\ARMMode)]
movw    r0, #22136      ; 0x5678
movt    r0, #4660       ; 0x1234
bx      lr
\end{lstlisting}

\RU{Видно, что число загружается в регистр по частям, в начале младшая часть 
(при помощи инструкции MOVW), затем старшая (при помощи MOVT).}
\EN{We see that the value is loaded into the register by parts, the lower part first (using MOVW), 
then the higher (using MOVT).}

\RU{Следовательно, нужно 2 инструкции в режиме ARM, чтобы записать 32-битное число в регистр.}
\EN{This means that 2 instructions are necessary in ARM mode for loading a 32-bit value into a register.}
\RU{Это не так уж и страшно, потому что в реальном коде не так уж и много констант (кроме 0 и 1).}
\EN{It's not a real problem, because in fact there are not many constants in real code (except of 0 and 1).}
\RU{Значит ли это, что это исполняется медленнее чем одна инструкция, как две инструкции?}
\EN{Does it mean that the two-instruction version is slower than one-instruction version?}
\RU{Вряд ли, наверняка современные процессоры ARM наверняка умеют распознавать такие 
последовательности и исполнять их быстро.}
\EN{Doubtfully. Most likely, modern ARM processors are able to detect such sequences and execute
them fast.}

\RU{А \IDA легко распознает подобные паттерны в коде и дизассемблирует эту ф-цию как:}
\EN{On the other hand, \IDA is able to detect such patterns in the code and disassembles this function as:}

\begin{lstlisting}
MOV    R0, 0x12345678
BX     LR
\end{lstlisting}

\subsection{ARM64}

\begin{lstlisting}
uint64_t f()
{
	return 0x12345678ABCDEF01;
};
\end{lstlisting}

\begin{lstlisting}[caption=GCC 4.9.1 -O3]
mov	x0, 61185   ; 0xef01
movk	x0, 0xabcd, lsl 16
movk	x0, 0x5678, lsl 32
movk	x0, 0x1234, lsl 48
ret
\end{lstlisting}

\index{ARM!\Instructions!MOVK}
\TT{MOVK} \RU{означает}\EN{means} ``MOV Keep'', \RU{т.е., она записывает 16-битное значение в регистр, не трогая
при этом остальные биты.}\EN{i.e., it writes a 16-bit value into the register, not touching the rest of the bits.}
\index{ARM!Optional operators!LSL}
\EN{The}\RU{Суффикс} \TT{LSL} \RU{сдвигает значение в каждом случае влево на 16, 32 и 48 бит. Сдвиг происходит
перед загрузкой.}\EN{suffix shifts left the value by 16, 32 and 48 bits at each step. The shifting is done before loading.}
\RU{Таким образом, нужно 4 инструкции, чтобы записать в регистр 64-битное значение.}
\EN{This means that 4 instructions are necessary to load a 64-bit value into a register.}

\subsubsection{\RU{Записать числа с плавающей точкой в регистр}\EN{Storing floating-point number into register}}

\RU{Некоторые числа можно записывать в D-регистр при помощи только одной инструкции.}
\EN{It's possible to store a floating-point number into a D-register using only one instruction.}

\RU{Например}\EN{For example}:

\begin{lstlisting}
double a()
{
	return 1.5;
};
\end{lstlisting}

\begin{lstlisting}[caption=GCC 4.9.1 -O3 + objdump]
0000000000000000 <a>:
   0:   1e6f1000        fmov    d0, #1.500000000000000000e+000
   4:   d65f03c0        ret
\end{lstlisting}

\RU{Число $1.5$ действительно было закодировано в 32-битной инструкции.}
\EN{The number $1.5$ was indeed encoded in a 32-bit instruction.}
\RU{Но как}\EN{But how}?
\index{ARM!\Instructions!FMOV}
\RU{В ARM64, инструкцию \TT{FMOV} есть 8 бит для кодирования некоторых чисел с плавающей запятой.}
\EN{In ARM64, there are 8 bits in the \TT{FMOV} instruction for encoding some floating-point numbers.}
\RU{В \cite{ARM64ref} алгоритм называется \TT{VFPExpandImm()}.}
\EN{The algorithm is called \TT{VFPExpandImm()} in \cite{ARM64ref}.}
\index{minifloat}
\EN{This is also called}\RU{Это также называется} \IT{minifloat}\footnote{\href{http://go.yurichev.com/17139}{wikipedia}}.
\RU{Я попробовал разные: $30.0$ и $31.0$ компилятору удается закодировать, а $32.0$ уже нет, для него
приходится выделять 8 байт в памяти и записать его там в формате IEEE 754:}
\EN{I tried it with different values: the compiler is able to encode $30.0$ and $31.0$, but it couldn't encode $32.0$,
as 8 bytes have to be allocated for this number in the IEEE 754 format:}

\begin{lstlisting}
double a()
{
	return 32;
};
\end{lstlisting}

\begin{lstlisting}[caption=GCC 4.9.1 -O3]
a:
	ldr	d0, .LC0
	ret
.LC0:
	.word	0
	.word	1077936128
\end{lstlisting}
