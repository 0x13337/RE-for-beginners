\chapter{\EN{More about ARM}\RU{Еще немного об ARM}}

\section{\RU{Загрузка констант в регистр}\EN{Loading constants into register}}

\subsection{\RU{32-битный}\EN{32-bit} ARM}
\label{ARM_big_constants_loading}

\RU{Как мы уже знаем, все инструкции имеют длину в 4 байта в режиме ARM и 2 байта в режиме Thumb.}
\EN{Aa we already know, all instructions has length of 4 bytes in ARM mode and 2 bytes in Thumb mode.}
\RU{Как в таком случае записать в регистр 32-битное число, если его невозможно закодировать
внутри одной инструкции?}
\EN{How to load 32-bit value into register, if it's not possible to encode it inside one instruction?}

\RU{Попробуем}\EN{Let's try}:

\begin{lstlisting}
unsigned int f()
{
	return 0x12345678;
};
\end{lstlisting}

\begin{lstlisting}[caption=GCC 4.6.3 -O3 \ARMMode]
f:
        ldr     r0, .L2
        bx      lr
.L2:
        .word   305419896 ; 0x12345678
\end{lstlisting}

\RU{Т.е., значение \TT{0x12345678} просто записано в памяти отдельно и загружается, если нужно.}
\EN{So, the \TT{0x12345678} value just stored aside in memory and loads if it needs.}
\RU{Но можно обойтись и без дополнительного обращения к памяти.}
\EN{But it's possible to get rid of additional memory access.}

\begin{lstlisting}[caption=GCC 4.6.3 -O3 -march=armv7-a \ARMMode]
movw    r0, #22136      ; 0x5678
movt    r0, #4660       ; 0x1234
bx      lr
\end{lstlisting}
% FIXME: what is MOVW? MOVT?

\RU{Видно что число загружается в регистр по частям, в начале младшая часть, затем старшая.}
\EN{We see that value is loaded into register by parts, lower part first, then higher.}

\RU{Следовательно, нужно 2 инструкции в режиме ARM, чтобы записать 32-битное число в регистр.}
\EN{It means, 2 instructions are necessary in ARM mode for loading 32-bit value into register.}
\RU{Это не так уж и страшно, потому что в реальном коде не так уж и много констант (кроме 0 и 1).}
\EN{It's not a real problem, because in fact there are not much constants in the real code (except of 0 and 1).}
\RU{Значит ли это, что это исполняется медленнее чем одна инструкция, как две инструкции?}
\EN{Does it mean it executes slower then one instruction, as two instructions?}
\RU{Врядли, наверняка современные процессоры ARM наверняка умеют распознавать такие 
последовательности и исполнять их быстро.}
\EN{Doubtfully. Most likely, modern ARM processors are able to detect such sequences and execute
them fast.}

\RU{А \IDA легко распознает подобные паттерны в коде и дизассемблирует эту ф-цию как:}
\EN{On the other hand, \IDA is able to detect such patterns in the code and disassembles this function as:}

\begin{lstlisting}
MOV    R0, 0x12345678
BX     LR
\end{lstlisting}

\subsection{ARM64}

\begin{lstlisting}
uint64_t f()
{
	return 0x12345678ABCDEF01;
};
\end{lstlisting}

\begin{lstlisting}[caption=GCC 4.9.1 -O3]
mov	x0, 61185   ; 0xef01
movk	x0, 0xabcd, lsl 16
movk	x0, 0x5678, lsl 32
movk	x0, 0x1234, lsl 48
ret
\end{lstlisting}

\TT{MOVK} \RU{означает}\EN{means} ``MOV Keep'', \RU{т.е., она записывает 16-битное значение в регистр, не трогая
при этом остальные биты.}\EN{i.e., it writes 16-bit value into register, not touching other bits at the same 
time.}
\RU{Суффикс }\TT{LSL} \RU{сдвигает значение в каждом случае влево на 16, 32 и 48 бит. Сдвиг происходит
перед загрузкой.}\EN{suffix shifts value left by 16, 32 and 48 bits at each step. Shifting done before loading.}
\RU{Таким образом, нужно 4 инструкции, чтобы записать в регистр 64-битное значение.}
\EN{This means, 4 instructions are necessary to load 64-bit value into register.}

\subsubsection{\RU{Записать числа с плавающей точкой в регистр}\EN{Storing floating number into register}}

\RU{Некоторые числа можно записывать в D-регистр при помощи только одной инструкции.}
\EN{It's possible to store a floating number into D-register using only one instruction.}

\RU{Например}\EN{For example}:

\begin{lstlisting}
double a()
{
	return 1.5;
};
\end{lstlisting}

\begin{lstlisting}[caption=GCC 4.9.1 -O3 + objdump]
0000000000000000 <a>:
   0:   1e6f1000        fmov    d0, #1.500000000000000000e+000
   4:   d65f03c0        ret
\end{lstlisting}

\RU{Число $1.5$ действительно было закодировано в 32-битной инструкции.}
\EN{$1.5$ number was indeed encoded in 32-bit instruction.}
\RU{Но как}\EN{But how}?
\RU{В ARM64, инструкцию \TT{FMOV} есть 8 бит для кодирования некоторых чисел с плавающей запятой.}
\EN{In ARM64, there are 8 bits in \TT{FMOV} instruction for encoding some float point numbers.}
\RU{В \cite{ARM64ref} алгоритм называется \TT{VFPExpandImm()}.}
\EN{The algorithm is called \TT{VFPExpandImm()} in \cite{ARM64ref}.}
\RU{Я попробовал разные: $30.0$ и $31.0$ компилятору удается закодировать, а $32.0$ уже нет, для него
приходится выделять 8 байт в памяти и записать его там в формате IEEE 754:}
\EN{I tried different: compiler is able to encode $30.0$ and $31.0$, but it couldn't encode $32.0$,
an 8 bytes should be allocated to this number in IEEE 754 format:}

\begin{lstlisting}
double a()
{
	return 32;
};
\end{lstlisting}

\begin{lstlisting}[caption=GCC 4.9.1 -O3]
a:
	ldr	d0, .LC0
	ret
.LC0:
	.word	0
	.word	1077936128
\end{lstlisting}

\section{\RU{Релоки}\EN{Relocs} \InENRU ARM64}
\label{ARM64_relocs}

\RU{Как известно, в ARM64 инструкции 4-байтные, так что записать длинное число в регистр одной инструкцией нельзя.}
\EN{As we know, there are 4-byte instructions in ARM64, so it is impossible to write large number into register
using single instruction.}
\RU{Тем не менее, файл может быть загружен по произвольному адресу в памяти, для этого релоки и нужны.}
\EN{Nevertheless, image may be loaded at random address in memory, so that's why relocs are existing.}
\RU{Больше о них (в связи с Win32 PE)}\EN{Read more about them (in relation to Win32 PE)}: \ref{subsec:relocs}.

\RU{В ARM64 принят следующий метод: адрес формируется при помощи пары инструкций: \TT{ADRP} и \ADD.}
\EN{ARM64 method is to form address using \TT{ADRP} and \ADD instructions pair.}
\RU{Первая загружает в регистр адрес 4Kb-страницы, а вторая прибавляет остаток.}
\EN{The first loads 4Kb-page address and the second adding remainder.}
\RU{Я скомпилировал пример из}\EN{I compiled example from} ``\HelloWorldSectionName'' 
(\lstref{hw_c}) \InENRU GCC (Linaro) 4.9 \RU{под}\EN{under} win32:

\begin{lstlisting}
...>aarch64-linux-gnu-gcc.exe hw.c -c

...>aarch64-linux-gnu-objdump.exe -d hw.o

...

0000000000000000 <main>:
   0:   a9bf7bfd        stp     x29, x30, [sp,#-16]!
   4:   910003fd        mov     x29, sp
   8:   90000000        adrp    x0, 0 <main>
   c:   91000000        add     x0, x0, #0x0
  10:   94000000        bl      0 <printf>
  14:   52800000        mov     w0, #0x0                        // #0
  18:   a8c17bfd        ldp     x29, x30, [sp],#16
  1c:   d65f03c0        ret

...>aarch64-linux-gnu-objdump.exe -r hw.o

...

RELOCATION RECORDS FOR [.text]:
OFFSET           TYPE              VALUE
0000000000000008 R_AARCH64_ADR_PREL_PG_HI21  .rodata
000000000000000c R_AARCH64_ADD_ABS_LO12_NC  .rodata
0000000000000010 R_AARCH64_CALL26  printf
\end{lstlisting}

\RU{Итак, в этом объектом файле три релока.}
\EN{So there are 3 relocs in this object file.}

\begin{itemize}
\item \RU{Самый первый записывает 21-битный адрес страницы в битовые поля инструкции \TT{ADRP}.}
\EN{The very first writes 21-bit page address into \TT{ADRP} instruction bit fields.}

\item \RU{Второй ---- 12 бит адреса, относительного от начала страницы, в поля инструкции \ADD.}
\EN{Second---12 bit of address relative to page start, into \ADD instruction bit fields.}

\item \RU{Последний, 26-битный, накладывается на инструкцию по адресу \TT{0x10}, где переход на ф-цию \printf.}
\EN{Last, 26-bit one, is applied to the instruction at \TT{0x10} address where the 
jump to the \printf function is.}
\RU{Из-за того что в ARM64 (да и в ARM в режиме ARM) невозможен переход по адресу не кратному 4,
так что доступное адресное пространство становится не 26 бит, а 28.}
\EN{Since it's not possible in ARM64 (and in ARM in ARM mode) to jump to the address not multiple of 4,
so the available address space is not 26 bits, but 28.}
\end{itemize}
% FIXME: objdump of resulting image
\RU{Если запустить пример в GDB под Linux и посмотреть опкоды:}
\EN{Here is what we see if to load the example in GDB under Linux and to see opcodes:}

\begin{lstlisting}
(gdb) disas/r main
Dump of assembler code for function main:
   0x0000000000400590 <+0>:     fd 7b bf a9     stp     x29, x30, [sp,#-16]!
   0x0000000000400594 <+4>:     fd 03 00 91     mov     x29, sp
   0x0000000000400598 <+8>:     00 00 00 90     adrp    x0, 0x400000
   0x000000000040059c <+12>:    00 20 19 91     add     x0, x0, #0x648
=> 0x00000000004005a0 <+16>:    a0 ff ff 97     bl      0x400420 <puts@plt>
   0x00000000004005a4 <+20>:    00 00 80 52     mov     w0, #0x0                        // #0
   0x00000000004005a8 <+24>:    fd 7b c1 a8     ldp     x29, x30, [sp],#16
   0x00000000004005ac <+28>:    c0 03 5f d6     ret
End of assembler dump.
\end{lstlisting}

\RU{Видно что}\EN{We see that} \IT{immhi} \AndENRU \IT{immlo} \RU{в инструкции}\EN{in} 
\TT{ADRP}\EN{ instruction}\EMDASH{}
\RU{нули, т.е., строка по нужному адресу находится в той же 4Kb-странице что и код ф-ции \main.}
\EN{are zeroes, i.e., the string is located in the same 4Kb-page as a \main function code.}

\RU{Дизассемблируем инструкцию \ADD вручную используя}\EN{Let's disassemble \ADD instruction manually 
using} \cite{ARM64ref}\footnote{N.B.: \RU{GDB показывает байты опкодов в том же порядке, в котором
они расположены в памяти, но документация от ARM показывает их наоборот}\EN{GDB shows opcode bytes in the same
order as they located in memory, but ARM manuals shows them contrariwise}}, \RU{и получим}\EN{and we will see}:

\begin{lstlisting}
91          19          20       00
10010001    00011001    00100000 00000000
sf=1 0 0 10001  shift=00 imm12=011001001000 (0x648) Rn=00000 (X0) Rd=00000 (X0)
\end{lstlisting}

\RU{Т.е., загрузчик Linux дописал \TT{0x648} на место бит \TT{imm12} в этой инструкции.}
\EN{So, the Linux loader writes \TT{0x648} on the \TT{imm12} bits place in this instruction.}
\TT{0x648} \RU{это адрес строки}\EN{is the address of} ``Hello!'' \RU{относительно начала 4Kb-страницы}\EN{string
relatively to the start of 4Kb-page}:

\lstinputlisting{patterns/ARM/gdb2.txt}

\RU{Действительно}\EN{Indeed}, \main \RU{начинается с}\EN{is beginning at} \TT{0x400590}, \RU{а строка}
\EN{and the} ``Hello!'' \RU{с}\EN{string at} \TT{0x400648}, 
\RU{страница начинается с}\EN{the page beginning at} \TT{0x400000}, \RU{а следующая страница 
уже с}\EN{and the next page at} \TT{0x401000}.

\RU{Больше о релоках связанных с ARM64}\EN{More about ARM64-related relocs}: \cite{ARM64_ELF}.
