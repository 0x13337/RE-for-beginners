\chapter{\EN{ARM-specific details}\RU{Кое-что специфичное для ARM}}

\section{\RU{Знак номера}\EN{Number sign} (\#) \RU{перед числом}\EN{before number}}

\RU{Компилятор Keil, \IDA и objdump предваряет все числа знаком номера (``\#''), например:}
\EN{The Keil compiler, \IDA and objdump precede all numbers with the ``\#'' number sign, for example:}
\lstref{Keil_number_sign}.
\RU{Но когда GCC 4.9 выдает результат на языке ассемблера, он так не делает, например:}
\EN{But when GCC 4.9 generates assembly language output, it doesn't, for example: }
\lstref{GCC_no_number_sign}.

\RU{Так что листинги для ARM в этой книге в каком-то смысле перемешаны.}
\EN{The ARM listings in this book are somewhat mixed.}

\RU{Я не знаю, как правильнее.}\EN{I'm not sure which method is right.}
\RU{Должно быть, всякий должен придерживаться тех правил, которые приняты в той среде,
в которой он работает.}
\EN{Supposedly, one has to obey the rules accepted in environment he/she works in.}

% sections
\section{\RU{Режимы адресации}\EN{Adressing modes}}
\index{ARM!\RU{Режимы адресации}\EN{Adressing modes}}
\label{ARM_postindex_vs_preindex}
\index{\CLanguageElements!\PostIncrement}
\index{\CLanguageElements!\PostDecrement}
\index{\CLanguageElements!\PreIncrement}
\index{\CLanguageElements!\PreDecrement}

\EN{This instruction is possible in ARM64:}
\RU{В ARM64 возможна такая инструкция:}

\index{ARM!\Instructions!LDR}
\begin{lstlisting}
ldr	x0, [x29,24]
\end{lstlisting}

\EN{This mean, add 24 to value in X29 and load value at this address.}
\RU{И это означает прибавить 24 к значению в X29 и загрузить значение по этому адресу.}
\RU{Обратите внимание что 24 внутри скобок.}
\EN{Please note that 24 is inside brackets.}
\EN{Meaning is different if number is outside brackets:}
\RU{А если снаружи скобок, то весь смысл меняется:}

\begin{lstlisting}
ldr	w4, [x1],28
\end{lstlisting}

\RU{Это означает, загрузить значение по адресу в X1, затем прибавить 28 к X1.}
\EN{This mean, load a value at address in X1, then add 28 to X1.}

\index{PDP-11}
\RU{ARM позволяет прибавлять некоторую константу к адресу, с которого происходит загрузка, либо вычитать.}
\EN{ARM allows to add some constant to the address used for loading, or to subtract.}
\RU{Причем, позволяет это делать до загрузки или после.}
\EN{And it's possible both after loading and before.}

\RU{Такого режима адресации в x86 нет, но он есть в некоторых других процессорах, даже на PDP-11.}
\EN{There is no such addressing mode in x86, but it is present in some other processors, even on PDP-11.}
\RU{Существует байка, что режимы пре-инкремента, пост-инкремента, 
пре-декремента и пост-декремента адреса в PDP-11}
\EN{There is a legend the pre-increment, post-increment, pre-decrement and post-decrement modes in PDP-11},
\RU{были ``виновны'' в появлении таких конструкций языка Си (который разрабатывался на PDP-11) как}
\EN{were ``guilty'' in appearance such C language (which developed on PDP-11) constructs as}
*ptr++, *++ptr, *ptr-{}-, *-{}-ptr. 
\RU{Кстати, это является трудно запоминаемой особенностью в Си.}
\EN{By the way, this is one of hard to memorize C feature.}
\RU{Дела обстоят так:}\EN{This is how it is:}

% FIXME: add ARM assembly...
\begin{center}
\begin{tabular}{ | l | l | l | l | }
\hline
\headercolor{} \RU{термин в Си}\EN{C term} & 
\headercolor{} \RU{термин в ARM}\EN{ARM term} & 
\headercolor{} \RU{выражение Си}\EN{C statement} & 
\headercolor{} \RU{как это работает}\EN{how it works} \\
\hline
\PostIncrement & 
post-indexed addressing & 
\TT{*ptr++} & 
\RU{использовать значение \TT{*ptr}}\EN{use \TT{*ptr} value}, \\
& & & \RU{затем инкремент указателя \TT{ptr}}\EN{then \gls{increment} \TT{ptr} pointer} \\
\hline
\PostDecrement & 
post-indexed addressing & 
\TT{*ptr-{}-} & 
\RU{использовать значение \TT{*ptr}}\EN{use \TT{*ptr} value}, \\
& & & \RU{затем \glslink{decrement}{декремент} указателя \TT{ptr}}\EN{then \gls{decrement} \TT{ptr} pointer} \\
\hline
\PreIncrement & 
pre-indexed addressing & 
\TT{*++ptr} & 
\RU{инкремент указателя \TT{ptr}}\EN{\gls{increment} \TT{ptr} pointer}, \\
& & & \RU{затем использовать значение \TT{*ptr}}\EN{then use \TT{*ptr} value} \\
\hline
\PreDecrement & 
pre-indexed addressing & 
\TT{*-{}-ptr} & 
\RU{\glslink{decrement}{декремент} указателя \TT{ptr}}\EN{\gls{decrement} \TT{ptr} pointer}, \\
& & & \RU{затем использовать значение \TT{*ptr}}\EN{then use \TT{*ptr} value} \\
\hline
\end{tabular}
\end{center}

Pre-indexing \EN{marked as exclamation mark in ARM assembly language}\RU{маркируется как 
восклицательный знак в ассемблере ARM}.
\RU{Для примера, смотрите строку 2 в}\EN{For example, see line 2 in} \lstref{hw_ARM64_GCC}.

\RU{Деннис Ритчи (один из создателей ЯП Си) указывал, что, это, вероятно, придумал Кен Томпсон 
(еще один создатель Си),
потому что подобная возможность процессора имелась еще в PDP-7}
\EN{Dennis Ritchie (one of C language creators) mentioned that it is, probably, was invented by Ken Thompson
(another C creator) because this processor feature was present in PDP-7}
\cite{Ritchie:1986}\cite{Ritchie:1993:DCL:155360.155580}.
\RU{Таким образом, компиляторы с ЯП Си на тот процессор, где это есть, могут использовать это.}
\EN{Thus, C language compilers may use it, if it is present on target processor.}

\RU{Всё это очень удобно для работы с массивами.}
\EN{That's very convenient for array processings.}

\section{\RU{Загрузка констант в регистр}\EN{Loading constants into register}}

\subsection{\RU{32-битный}\EN{32-bit} ARM}
\label{ARM_big_constants_loading}

\RU{Как мы уже знаем, все инструкции имеют длину в 4 байта в режиме ARM и 2 байта в режиме Thumb.}
\EN{Aa we already know, all instructions has length of 4 bytes in ARM mode and 2 bytes in Thumb mode.}
\RU{Как в таком случае записать в регистр 32-битное число, если его невозможно закодировать
внутри одной инструкции?}
\EN{How to load 32-bit value into register, if it's not possible to encode it inside one instruction?}

\RU{Попробуем}\EN{Let's try}:

\begin{lstlisting}
unsigned int f()
{
	return 0x12345678;
};
\end{lstlisting}

\begin{lstlisting}[caption=GCC 4.6.3 -O3 \ARMMode]
f:
        ldr     r0, .L2
        bx      lr
.L2:
        .word   305419896 ; 0x12345678
\end{lstlisting}

\RU{Т.е., значение \TT{0x12345678} просто записано в памяти отдельно и загружается, если нужно.}
\EN{So, the \TT{0x12345678} value just stored aside in memory and loads if it needs.}
\RU{Но можно обойтись и без дополнительного обращения к памяти.}
\EN{But it's possible to get rid of additional memory access.}

\begin{lstlisting}[caption=GCC 4.6.3 -O3 -march{=}armv7-a (\ARMMode)]
movw    r0, #22136      ; 0x5678
movt    r0, #4660       ; 0x1234
bx      lr
\end{lstlisting}

\RU{Видно что число загружается в регистр по частям, в начале младшая часть 
(при помощи инструкции MOVW), затем старшая (при помощи MOVT).}
\EN{We see that value is loaded into register by parts, lower part first (using MOVW instruction), 
then higher (using MOVT).}

\RU{Следовательно, нужно 2 инструкции в режиме ARM, чтобы записать 32-битное число в регистр.}
\EN{It means, 2 instructions are necessary in ARM mode for loading 32-bit value into register.}
\RU{Это не так уж и страшно, потому что в реальном коде не так уж и много констант (кроме 0 и 1).}
\EN{It's not a real problem, because in fact there are not much constants in the real code (except of 0 and 1).}
\RU{Значит ли это, что это исполняется медленнее чем одна инструкция, как две инструкции?}
\EN{Does it mean it executes slower then one instruction, as two instructions?}
\RU{Врядли, наверняка современные процессоры ARM наверняка умеют распознавать такие 
последовательности и исполнять их быстро.}
\EN{Doubtfully. Most likely, modern ARM processors are able to detect such sequences and execute
them fast.}

\RU{А \IDA легко распознает подобные паттерны в коде и дизассемблирует эту ф-цию как:}
\EN{On the other hand, \IDA is able to detect such patterns in the code and disassembles this function as:}

\begin{lstlisting}
MOV    R0, 0x12345678
BX     LR
\end{lstlisting}

\subsection{ARM64}

\begin{lstlisting}
uint64_t f()
{
	return 0x12345678ABCDEF01;
};
\end{lstlisting}

\begin{lstlisting}[caption=GCC 4.9.1 -O3]
mov	x0, 61185   ; 0xef01
movk	x0, 0xabcd, lsl 16
movk	x0, 0x5678, lsl 32
movk	x0, 0x1234, lsl 48
ret
\end{lstlisting}

\index{ARM!\Instructions!MOVK}
\TT{MOVK} \RU{означает}\EN{means} ``MOV Keep'', \RU{т.е., она записывает 16-битное значение в регистр, не трогая
при этом остальные биты.}\EN{i.e., it writes 16-bit value into register, not touching other bits at the same 
time.}
\index{ARM!Optional operators!LSL}
\RU{Суффикс }\TT{LSL} \RU{сдвигает значение в каждом случае влево на 16, 32 и 48 бит. Сдвиг происходит
перед загрузкой.}\EN{suffix shifts value left by 16, 32 and 48 bits at each step. Shifting done before loading.}
\RU{Таким образом, нужно 4 инструкции, чтобы записать в регистр 64-битное значение.}
\EN{This means, 4 instructions are necessary to load 64-bit value into register.}

\subsubsection{\RU{Записать числа с плавающей точкой в регистр}\EN{Storing floating number into register}}

\RU{Некоторые числа можно записывать в D-регистр при помощи только одной инструкции.}
\EN{It's possible to store a floating number into D-register using only one instruction.}

\RU{Например}\EN{For example}:

\begin{lstlisting}
double a()
{
	return 1.5;
};
\end{lstlisting}

\begin{lstlisting}[caption=GCC 4.9.1 -O3 + objdump]
0000000000000000 <a>:
   0:   1e6f1000        fmov    d0, #1.500000000000000000e+000
   4:   d65f03c0        ret
\end{lstlisting}

\RU{Число $1.5$ действительно было закодировано в 32-битной инструкции.}
\EN{$1.5$ number was indeed encoded in 32-bit instruction.}
\RU{Но как}\EN{But how}?
\index{ARM!\Instructions!FMOV}
\RU{В ARM64, инструкцию \TT{FMOV} есть 8 бит для кодирования некоторых чисел с плавающей запятой.}
\EN{In ARM64, there are 8 bits in \TT{FMOV} instruction for encoding some float point numbers.}
\RU{В \cite{ARM64ref} алгоритм называется \TT{VFPExpandImm()}.}
\EN{The algorithm is called \TT{VFPExpandImm()} in \cite{ARM64ref}.}
\index{minifloat}
\EN{This is also called}\RU{Это также называется} \IT{minifloat}\footnote{\url{http://go.yurichev.com/17139}}.
\RU{Я попробовал разные: $30.0$ и $31.0$ компилятору удается закодировать, а $32.0$ уже нет, для него
приходится выделять 8 байт в памяти и записать его там в формате IEEE 754:}
\EN{I tried different: compiler is able to encode $30.0$ and $31.0$, but it couldn't encode $32.0$,
an 8 bytes should be allocated to this number in IEEE 754 format:}

\begin{lstlisting}
double a()
{
	return 32;
};
\end{lstlisting}

\begin{lstlisting}[caption=GCC 4.9.1 -O3]
a:
	ldr	d0, .LC0
	ret
.LC0:
	.word	0
	.word	1077936128
\end{lstlisting}

\newcommand{\ARMELF}{[\IT{ELF for the ARM 64-bit Architecture (AArch64)}, (2013)]\footnote{\AlsoAvailableAs \url{http://go.yurichev.com/17288}}}

\section{\RU{Релоки}\EN{Relocs} \InENRU ARM64}
\label{ARM64_relocs}

\RU{Как известно, в ARM64 инструкции 4-байтные, так что записать длинное число в регистр одной инструкцией нельзя.}
\EN{As we know, there are 4-byte instructions in ARM64, so it is impossible to write a large number into a register
using a single instruction.}
\RU{Тем не менее, файл может быть загружен по произвольному адресу в памяти, для этого релоки и нужны.}
\EN{Nevertheless, an executable image can be loaded at any random address in memory, so that's why relocs exists.}
\RU{Больше о них (в связи с Win32 PE)}\EN{Read more about them (in relation to Win32 PE)}: \myref{subsec:relocs}.

\myindex{ARM!\Instructions!ADRP/ADD pair}
\RU{В ARM64 принят следующий метод: адрес формируется при помощи пары инструкций: \TT{ADRP} и \ADD.}
\EN{The address is formed using the \TT{ADRP} and \ADD instruction pair in ARM64.}
\RU{Первая загружает в регистр адрес 4KiB-страницы, а вторая прибавляет остаток.}
\EN{The first loads a 4KiB-page address and the second one adds the remainder.}
\RU{Скомпилируем пример из}\EN{Let's compile the example from} \q{\HelloWorldSectionName} 
(\lstref{hw_c}) \InENRU GCC (Linaro) 4.9 \RU{под}\EN{under} win32:

\begin{lstlisting}[caption=GCC (Linaro) 4.9 \AndENRU objdump \EN{of object file}\RU{объектного файла}]
...>aarch64-linux-gnu-gcc.exe hw.c -c

...>aarch64-linux-gnu-objdump.exe -d hw.o

...

0000000000000000 <main>:
   0:   a9bf7bfd        stp     x29, x30, [sp,#-16]!
   4:   910003fd        mov     x29, sp
   8:   90000000        adrp    x0, 0 <main>
   c:   91000000        add     x0, x0, #0x0
  10:   94000000        bl      0 <printf>
  14:   52800000        mov     w0, #0x0                        // #0
  18:   a8c17bfd        ldp     x29, x30, [sp],#16
  1c:   d65f03c0        ret

...>aarch64-linux-gnu-objdump.exe -r hw.o

...

RELOCATION RECORDS FOR [.text]:
OFFSET           TYPE              VALUE
0000000000000008 R_AARCH64_ADR_PREL_PG_HI21  .rodata
000000000000000c R_AARCH64_ADD_ABS_LO12_NC  .rodata
0000000000000010 R_AARCH64_CALL26  printf
\end{lstlisting}

\RU{Итак, в этом объектом файле три релока.}
\EN{So there are 3 relocs in this object file.}

\begin{itemize}
\item 
\RU{Самый первый берет адрес страницы, отсекает младшие 12 бит и записывает оставшиеся старшие 21
в битовые поля инструкции \TT{ADRP}. Это потому что младшие 12 бит кодировать не нужно,
и в ADRP выделено место только для 21 бит.}
\EN{The first one takes the page address, cuts the lowest 12 bits and writes the remaining high 21 bits
to the \TT{ADRP} instruction's bit fields. This is because we don't need to encode the low 12 bits,
and the ADRP instruction has space only for 21 bits.}

\item \RU{Второй ---- 12 бит адреса, относительного от начала страницы, в поля инструкции \ADD.}
\EN{The second one puts the 12 bits of the address relative to the page start into the \ADD instruction's bit fields.}

\item \RU{Последний, 26-битный, накладывается на инструкцию по адресу \TT{0x10}, где переход на функцию \printf.}
\EN{The last, 26-bit one, is applied to the instruction at address \TT{0x10} where the 
jump to the \printf function is.}
\RU{Все адреса инструкций в ARM64 (да и в ARM в режиме ARM) имеют нули в двух младших битах
(потому что все инструкции имеют размер в 4 байта),
так что нужно кодировать только старшие 26 бит из 28-битного адресного пространства ($\pm 128$MB).}
\EN{All ARM64 (and in ARM in ARM mode) instruction addresses have zeroes in the two lowest bits
(because all instructions have a size of 4 bytes),
so one need to encode only the highest 26 bits of 28-bit address space ($\pm 128$MB).}

\end{itemize}

\RU{В слинкованном исполняемом файле релоков в этих местах нет: потому что там уже точно известно, 
где будет находится строка \q{Hello!}, и в какой странице, а также известен адрес функции \puts.}
\EN{There are no such relocs in the executable file: because it's known where the \q{Hello!} string
is located, in which page, and the address of \puts is also known.}
\RU{И поэтому там, в инструкциях \TT{ADRP}, \ADD и \TT{BL}, уже проставлены нужные значения 
(их проставил линкер во время компоновки):}
\EN{So there are values set already in the \TT{ADRP}, \ADD and \TT{BL} instructions
(the linker has written them while linking):}

\begin{lstlisting}[caption=objdump \EN{of executable file}\RU{исполняемого файла}]
0000000000400590 <main>:
  400590:       a9bf7bfd        stp     x29, x30, [sp,#-16]!
  400594:       910003fd        mov     x29, sp
  400598:       90000000        adrp    x0, 400000 <_init-0x3b8>
  40059c:       91192000        add     x0, x0, #0x648
  4005a0:       97ffffa0        bl      400420 <puts@plt>
  4005a4:       52800000        mov     w0, #0x0                        // #0
  4005a8:       a8c17bfd        ldp     x29, x30, [sp],#16
  4005ac:       d65f03c0        ret

...

Contents of section .rodata:
 400640 01000200 00000000 48656c6c 6f210000  ........Hello!..
\end{lstlisting}

\myindex{ARM!\Instructions!BL}

\ifdefined\ENGLISH{}
As an example, let's try to disassemble the BL instruction manually.\\
\TT{0x97ffffa0} is $10010111111111111111111110100000b$.
According to [\ARMSixFourRef C5.6.26], \IT{imm26} is the last 26 bits: $imm26 = 11111111111111111110100000$.
It is \TT{0x3FFFFA0}, but the \ac{MSB} is 1, 
so the number is negative, and we can convert it manually to convenient form for us.
By the rules of negation (\myref{sec:signednumbers:negation}), just invert all bits: (it is \TT{1011111=0x5F}), and add 1 (\TT{0x5F+1=0x60}).
So the number in signed form is \TT{-0x60}.
Let's multiplicate \TT{-0x60} by 4 (because address stored in opcode is divided by 4): it is \TT{-0x180}.
Now let's calculate destination address: \TT{0x4005a0} + (\TT{-0x180}) = \TT{0x400420} 
(please note: we consider the address of the BL instruction, not the current value of \ac{PC}, which may be different!).
So the destination address is \TT{0x400420}.\\
\\
More about ARM64-related relocs: \ARMELF.
\fi

\ifdefined\RUSSIAN{}
В качестве примера, попробуем дизассемблировать инструкцию BL вручную.\\
\TT{0x97ffffa0} это $10010111111111111111111110100000b$.
В соответствии с [\ARMSixFourRef C5.6.26], \IT{imm26} это последние 26 бит: $imm26 = 11111111111111111110100000$.
Это \TT{0x3FFFFA0}, но \ac{MSB} это 1, 
так что число отрицательное, мы можем вручную его конвертировать в удобный для нас вид.
По правилам изменения знака (\myref{sec:signednumbers:negation}), просто инвертируем все биты: (\TT{1011111=0x5F}) и прибавляем 1 (\TT{0x5F+1=0x60}).
Так что число в знаковом виде: \TT{-0x60}.
Умножим \TT{-0x60} на 4 (потому что адрес записанный в опкоде разделен на 4): это \TT{-0x180}.
Теперь вычисляем адрес назначения: \TT{0x4005a0} + (\TT{-0x180}) = \TT{0x400420} 
(пожалуйста заметьте: мы берем адрес инструкции BL, а не текущее значение \ac{PC}, которое может быть другим!).
Так что адрес в итоге \TT{0x400420}.\\
\\
Больше о релоках связанных с ARM64: \ARMELF.
\fi

