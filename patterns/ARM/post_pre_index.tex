\section{\RU{Режимы адресации}\EN{Addressing modes}}
\myindex{ARM!\RU{Режимы адресации}\EN{Addressing modes}}
\label{ARM_postindex_vs_preindex}
\myindex{\CLanguageElements!\PostIncrement}
\myindex{\CLanguageElements!\PostDecrement}
\myindex{\CLanguageElements!\PreIncrement}
\myindex{\CLanguageElements!\PreDecrement}

\EN{This instruction is possible in ARM64:}
\RU{В ARM64 возможна такая инструкция:}

\myindex{ARM!\Instructions!LDR}
\begin{lstlisting}
ldr	x0, [x29,24]
\end{lstlisting}

\EN{This means add 24 to the value in X29 and load the value from this address.}
\RU{И это означает прибавить 24 к значению в X29 и загрузить значение по этому адресу.}
\RU{Обратите внимание что 24 внутри скобок.}
\EN{Please note that 24 is inside the brackets.}
\EN{The meaning is different if the number is outside the brackets:}
\RU{А если снаружи скобок, то весь смысл меняется:}

\begin{lstlisting}
ldr	w4, [x1],28
\end{lstlisting}

\RU{Это означает, загрузить значение по адресу в X1, затем прибавить 28 к X1.}
\EN{This means load the value at the address in X1, then add 28 to X1.}

\myindex{PDP-11}
\RU{ARM позволяет прибавлять некоторую константу к адресу, с которого происходит загрузка, либо вычитать.}
\EN{ARM allows you to add or subtract a constant to/from the address used for loading.}
\RU{Причем, позволяет это делать до загрузки или после.}
\EN{And it's possible to do that both before and after loading.}

\RU{Такого режима адресации в x86 нет, но он есть в некоторых других процессорах, даже на PDP-11.}
\EN{There is no such addressing mode in x86, but it is present in some other processors, even on PDP-11.}
\RU{Существует байка, что режимы пре-инкремента, пост-инкремента, 
пре-декремента и пост-декремента адреса в PDP-11}
\EN{There is a legend that the pre-increment, post-increment, pre-decrement and post-decrement modes in PDP-11},
\RU{были \q{виновны} в появлении таких конструкций языка Си (который разрабатывался на PDP-11) как}
\EN{were \q{guilty} for the appearance of such C language (which developed on PDP-11) constructs as}
*ptr++, *++ptr, *ptr-{}-, *-{}-ptr. 
\RU{Кстати, это является трудно запоминаемой особенностью в Си.}
\EN{By the way, this is one of the hard to memorize C features.}
\RU{Дела обстоят так:}\EN{This is how it is:}

% FIXME: add ARM assembly...
\begin{center}
\begin{tabular}{ | l | l | l | l | }
\hline
\headercolor{} \RU{термин в Си}\EN{C term} & 
\headercolor{} \RU{термин в ARM}\EN{ARM term} & 
\headercolor{} \RU{выражение Си}\EN{C statement} & 
\headercolor{} \RU{как это работает}\EN{how it works} \\
\hline
\PostIncrement & 
post-indexed addressing & 
\TT{*ptr++} & 
\RU{использовать значение \TT{*ptr}}\EN{use \TT{*ptr} value}, \\
& & & \RU{затем инкремент указателя \TT{ptr}}\EN{then \gls{increment} \TT{ptr} pointer} \\
\hline
\PostDecrement & 
post-indexed addressing & 
\TT{*ptr-{}-} & 
\RU{использовать значение \TT{*ptr}}\EN{use \TT{*ptr} value}, \\
& & & \RU{затем \glslink{decrement}{декремент} указателя \TT{ptr}}\EN{then \gls{decrement} \TT{ptr} pointer} \\
\hline
\PreIncrement & 
pre-indexed addressing & 
\TT{*++ptr} & 
\RU{инкремент указателя \TT{ptr}}\EN{\gls{increment} \TT{ptr} pointer}, \\
& & & \RU{затем использовать значение \TT{*ptr}}\EN{then use \TT{*ptr} value} \\
\hline
\PreDecrement & 
pre-indexed addressing & 
\TT{*-{}-ptr} & 
\RU{\glslink{decrement}{декремент} указателя \TT{ptr}}\EN{\gls{decrement} \TT{ptr} pointer}, \\
& & & \RU{затем использовать значение \TT{*ptr}}\EN{then use \TT{*ptr} value} \\
\hline
\end{tabular}
\end{center}

Pre-indexing \EN{is marked with an exclamation mark in the ARM assembly language}\RU{маркируется как 
восклицательный знак в ассемблере ARM}.
\RU{Для примера, смотрите строку 2 в}\EN{For example, see line 2 in} \lstref{hw_ARM64_GCC}.

\RU{Деннис Ритчи (один из создателей ЯП Си) указывал, что, это, вероятно, придумал Кен Томпсон 
(еще один создатель Си),
потому что подобная возможность процессора имелась еще в PDP-7}
\EN{Dennis Ritchie (one of the creators of the C language) mentioned that it probably was invented by Ken Thompson
(another C creator) because this processor feature was present in PDP-7}
\url{http://yurichev.com/mirrors/C/c_dmr_postincrement.txt}, \RitchieDevC{}.
\RU{Таким образом, компиляторы с ЯП Си на тот процессор, где это есть, могут использовать это.}
\EN{Thus, C language compilers may use it, if it is present on the target processor.}

\RU{Всё это очень удобно для работы с массивами.}
\EN{That's very convenient for array processing.}
