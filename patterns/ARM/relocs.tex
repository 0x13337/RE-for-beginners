\section{\RU{Релоки}\EN{Relocs} \InENRU ARM64}
\label{ARM64_relocs}

\RU{Как известно, в ARM64 инструкции 4-байтные, так что записать длинное число в регистр одной инструкцией нельзя.}
\EN{As we know, there are 4-byte instructions in ARM64, so it is impossible to write large number into register
using single instruction.}
\RU{Тем не менее, файл может быть загружен по произвольному адресу в памяти, для этого релоки и нужны.}
\EN{Nevertheless, image may be loaded at random address in memory, so that's why relocs are existing.}
\RU{Больше о них (в связи с Win32 PE)}\EN{Read more about them (in relation to Win32 PE)}: \ref{subsec:relocs}.

\index{ARM!\Instructions!ADRP/ADD pair}
\RU{В ARM64 принят следующий метод: адрес формируется при помощи пары инструкций: \TT{ADRP} и \ADD.}
\EN{ARM64 method is to form address using \TT{ADRP} and \ADD instructions pair.}
\RU{Первая загружает в регистр адрес 4Kb-страницы, а вторая прибавляет остаток.}
\EN{The first loads 4Kb-page address and the second adding remainder.}
\RU{Я скомпилировал пример из}\EN{I compiled example from} ``\HelloWorldSectionName'' 
(\lstref{hw_c}) \InENRU GCC (Linaro) 4.9 \RU{под}\EN{under} win32:

\begin{lstlisting}[caption=GCC (Linaro) 4.9 \AndENRU objdump \EN{of object file}\RU{объектного файла}]
...>aarch64-linux-gnu-gcc.exe hw.c -c

...>aarch64-linux-gnu-objdump.exe -d hw.o

...

0000000000000000 <main>:
   0:   a9bf7bfd        stp     x29, x30, [sp,#-16]!
   4:   910003fd        mov     x29, sp
   8:   90000000        adrp    x0, 0 <main>
   c:   91000000        add     x0, x0, #0x0
  10:   94000000        bl      0 <printf>
  14:   52800000        mov     w0, #0x0                        // #0
  18:   a8c17bfd        ldp     x29, x30, [sp],#16
  1c:   d65f03c0        ret

...>aarch64-linux-gnu-objdump.exe -r hw.o

...

RELOCATION RECORDS FOR [.text]:
OFFSET           TYPE              VALUE
0000000000000008 R_AARCH64_ADR_PREL_PG_HI21  .rodata
000000000000000c R_AARCH64_ADD_ABS_LO12_NC  .rodata
0000000000000010 R_AARCH64_CALL26  printf
\end{lstlisting}

\RU{Итак, в этом объектом файле три релока.}
\EN{So there are 3 relocs in this object file.}

\begin{itemize}
\item \RU{Самый первый записывает 21-битный адрес страницы в битовые поля инструкции \TT{ADRP}.}
\EN{The very first writes 21-bit page address into \TT{ADRP} instruction bit fields.}

\item \RU{Второй ---- 12 бит адреса, относительного от начала страницы, в поля инструкции \ADD.}
\EN{Second---12 bit of address relative to page start, into \ADD instruction bit fields.}

\item \RU{Последний, 26-битный, накладывается на инструкцию по адресу \TT{0x10}, где переход на ф-цию \printf.}
\EN{Last, 26-bit one, is applied to the instruction at \TT{0x10} address where the 
jump to the \printf function is.}
\RU{Из-за того что в ARM64 (да и в ARM в режиме ARM) невозможен переход по адресу не кратному 4,
так что доступное адресное пространство становится не 26 бит, а 28.}
\EN{Since it's not possible in ARM64 (and in ARM in ARM mode) to jump to the address not multiple of 4,
so the available address space is not 26 bits, but 28.}
\end{itemize}

\RU{В слинкованном исполняемом файле релоков в этих местах нет: потому что там уже точно известно, 
где будет находится строка ``Hello!'', и в какой странице, а также известен адрес ф-ции \puts.}
\EN{There are no such relocs in the executable file: because, it's known, where the ``Hello!'' string
will be located, in which page, and also \puts function address is known.}
\RU{И поэтому там, в инструкциях \TT{ADRP}, \ADD и \TT{BL}, уже проставлены нужные значения 
(их проставил линкер во время компоновки):}
\EN{So there are values already set in the \TT{ADRP}, \ADD and \TT{BL} instructions
(linker set it while linking):}

\begin{lstlisting}[caption=objdump \EN{of executable file}\RU{исполняемого файла}]
0000000000400590 <main>:
  400590:       a9bf7bfd        stp     x29, x30, [sp,#-16]!
  400594:       910003fd        mov     x29, sp
  400598:       90000000        adrp    x0, 400000 <_init-0x3b8>
  40059c:       91192000        add     x0, x0, #0x648
  4005a0:       97ffffa0        bl      400420 <puts@plt>
  4005a4:       52800000        mov     w0, #0x0                        // #0
  4005a8:       a8c17bfd        ldp     x29, x30, [sp],#16
  4005ac:       d65f03c0        ret

...

Contents of section .rodata:
 400640 01000200 00000000 48656c6c 6f210000  ........Hello!..
\end{lstlisting}

\RU{Больше о релоках связанных с ARM64}\EN{More about ARM64-related relocs}: \cite{ARM64_ELF}.
