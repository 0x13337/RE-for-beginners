\newcommand{\ARMELF}{[\IT{ELF for the ARM 64-bit Architecture (AArch64)}, (2013)]\footnote{\AlsoAvailableAs \url{http://go.yurichev.com/17288}}}

\section{Relocs in ARM64}
\label{ARM64_relocs}

As we know, there are 4-byte instructions in ARM64, so it is impossible to write a large number into a register
using a single instruction.

Nevertheless, an executable image can be loaded at any random address in memory, so that's why relocs exists.
Read more about them (in relation to Win32 PE): \myref{subsec:relocs}.

\myindex{ARM!\Instructions!ADRP/ADD pair}

The address is formed using the \TT{ADRP} and \ADD instruction pair in ARM64.

The first loads a 4KiB-page address and the second one adds the remainder.
Let's compile the example from \q{\HelloWorldSectionName} 
(\lstref{hw_c}) in GCC (Linaro) 4.9 under win32:

\begin{lstlisting}[caption=GCC (Linaro) 4.9 and objdump of object file]
...>aarch64-linux-gnu-gcc.exe hw.c -c

...>aarch64-linux-gnu-objdump.exe -d hw.o

...

0000000000000000 <main>:
   0:   a9bf7bfd        stp     x29, x30, [sp,#-16]!
   4:   910003fd        mov     x29, sp
   8:   90000000        adrp    x0, 0 <main>
   c:   91000000        add     x0, x0, #0x0
  10:   94000000        bl      0 <printf>
  14:   52800000        mov     w0, #0x0                        // #0
  18:   a8c17bfd        ldp     x29, x30, [sp],#16
  1c:   d65f03c0        ret

...>aarch64-linux-gnu-objdump.exe -r hw.o

...

RELOCATION RECORDS FOR [.text]:
OFFSET           TYPE              VALUE
0000000000000008 R_AARCH64_ADR_PREL_PG_HI21  .rodata
000000000000000c R_AARCH64_ADD_ABS_LO12_NC  .rodata
0000000000000010 R_AARCH64_CALL26  printf
\end{lstlisting}

So there are 3 relocs in this object file.

\begin{itemize}
\item 
The first one takes the page address, cuts the lowest 12 bits and writes the remaining high 21 bits
to the \TT{ADRP} instruction's bit fields. This is because we don't need to encode the low 12 bits,
and the ADRP instruction has space only for 21 bits.

\item 
The second one puts the 12 bits of the address relative to the page start into the \ADD instruction's bit fields.

\item 
The last, 26-bit one, is applied to the instruction at address \TT{0x10} where the 
jump to the \printf function is.

All ARM64 (and in ARM in ARM mode) instruction addresses have zeroes in the two lowest bits
(because all instructions have a size of 4 bytes),
so one need to encode only the highest 26 bits of 28-bit address space ($\pm 128$MB).

\end{itemize}

There are no such relocs in the executable file: because it's known where the \q{Hello!} string
is located, in which page, and the address of \puts is also known.

So there are values set already in the \TT{ADRP}, \ADD and \TT{BL} instructions
(the linker has written them while linking):

\begin{lstlisting}[caption=objdump of executable file]
0000000000400590 <main>:
  400590:       a9bf7bfd        stp     x29, x30, [sp,#-16]!
  400594:       910003fd        mov     x29, sp
  400598:       90000000        adrp    x0, 400000 <_init-0x3b8>
  40059c:       91192000        add     x0, x0, #0x648
  4005a0:       97ffffa0        bl      400420 <puts@plt>
  4005a4:       52800000        mov     w0, #0x0                        // #0
  4005a8:       a8c17bfd        ldp     x29, x30, [sp],#16
  4005ac:       d65f03c0        ret

...

Contents of section .rodata:
 400640 01000200 00000000 48656c6c 6f210000  ........Hello!..
\end{lstlisting}

\myindex{ARM!\Instructions!BL}

As an example, let's try to disassemble the BL instruction manually.\\
\TT{0x97ffffa0} is $10010111111111111111111110100000b$.
According to [\ARMSixFourRef C5.6.26], \IT{imm26} is the last 26 bits: $imm26 = 11111111111111111110100000$.
It is \TT{0x3FFFFA0}, but the \ac{MSB} is 1, 
so the number is negative, and we can convert it manually to convenient form for us.
By the rules of negation (\myref{sec:signednumbers:negation}), just invert all bits: (it is \TT{1011111=0x5F}), and add 1 (\TT{0x5F+1=0x60}).
So the number in signed form is \TT{-0x60}.
Let's multiplicate \TT{-0x60} by 4 (because address stored in opcode is divided by 4): it is \TT{-0x180}.
Now let's calculate destination address: \TT{0x4005a0} + (\TT{-0x180}) = \TT{0x400420} 
(please note: we consider the address of the BL instruction, not the current value of \ac{PC}, which may be different!).
So the destination address is \TT{0x400420}.\\
\\
More about ARM64-related relocs: \ARMELF.
