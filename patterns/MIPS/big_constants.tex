\section{\RU{Загрузка констант в регистр}\EN{Loading constants into register}}
\label{MIPS_big_constants}

\begin{lstlisting}
unsigned int f()
{
	return 0x12345678;
};
\end{lstlisting}

\RU{В MIPS, так же как и в ARM, все инструкции имеют размер 32 бита, так что невозможно
закодировать 32-битную константу в инструкцию.}
\EN{All instructions in MIPS, just like ARM, have a of 32-bit, so it's not possible to
embed a 32-bit constant into one instruction.}
\myindex{MIPS!\Instructions!LI}
\myindex{MIPS!\Instructions!ORI}
\RU{Так что это транслируется в две инструкции:
первая загружает старшую часть 32-битного числа и вторая применяет операцию \q{ИЛИ},
эффект от которой в том, что она просто выставляет младшие 16 бит целевого регистра:}
\EN{So this translates to at least two instructions: 
the first loads the high part of the 32-bit number and the second
one applies an OR operation, which effectively sets the low 16-bit part of the target register:}

\begin{lstlisting}[caption=GCC 4.4.5 -O3 (\assemblyOutput)]
        li      $2,305397760                    # 0x12340000
        j       $31
        ori     $2,$2,0x5678 ; branch delay slot
\end{lstlisting}

\IDA \RU{знает о таких часто встречающихся последовательностях, так что для удобства, 
она показывает последнюю инструкцию ORI как псевдоинструкцию LI, 
которая якобы загружает полное 32-битное значение в регистр \$V0.}
\EN{is fully aware of such frequently encountered code patterns, 
so, for convenience it shows the last ORI instruction as the LI pseudoinstruction,
which allegedly loads a full 32-bit number into the \$V0 register.}

\myindex{MIPS!\Instructions!LUI}

\begin{lstlisting}[caption=GCC 4.4.5 -O3 (IDA)]
         lui     $v0, 0x1234
         jr      $ra
         li      $v0, 0x12345678 ; branch delay slot
\end{lstlisting}

\RU{В выводе на ассемблере от GCC есть псевдоинструкция LUI, но на самом деле, 
там LUI (\q{Load Upper Immediate}), загружающая 16-битное значение в старшую часть регистра.}
\EN{The GCC assembly output has the LI pseudoinstruction, but in fact, LUI (\q{Load Upper Imeddiate}) is there,
which stores a 16-bit value into the high part of the register.}
