\section{\RU{Загрузка констант в регистр}\EN{Loading constants into register}}

\begin{lstlisting}
unsigned int f()
{
	return 0x12345678;
};
\end{lstlisting}

\RU{В MIPS, так же как и в ARM, все инструкции имеют размер 32 бита, так что невозможно
закодировать 32-битную константу в инструкцию.}
\EN{MIPS, just like ARM, has all instructions with size of 32-bit, so it's not possible to
embed 32-bit constant into one instruction.}
\index{MIPS!\Instructions!LI}
\index{MIPS!\Instructions!ORI}
\RU{Так что это транслируется в две инструкции:
первая загружает старшую часть 32-битного числа и вторая применяет операцию ``ИЛИ'',
эффект от которой в том, что она просто выставляет младшие 16 бит целевого регистра:}
\EN{So this translates to at least two instructions: 
first loads high part of 32-bit number and the second
one apply OR operation which effectively sets low 16-bit part of the target register:}

\begin{lstlisting}[caption=GCC 4.4.5 -O3 (\assemblyOutput)]
        li      $2,305397760                    # 0x12340000
        j       $31
        ori     $2,$2,0x5678 ; branch delay slot
\end{lstlisting}

\IDA \RU{знает о таких часто встречающихся последовательностях, так что для удобства, 
она показывает последнюю инструкцию ORI как псевдоинструкцию LI, 
которая якобы загружает полное 32-битное значение в регистр \$V0.}
\EN{is fully aware of such often encountering code patterns, 
so, for convenience, it shows the last ORI instruction as LI pseudoinstruction,
which allegedly loads full 32-bit number into \$V0 register.}

\index{MIPS!\Instructions!LUI}

\begin{lstlisting}[caption=GCC 4.4.5 -O3 (IDA)]
         lui     $v0, 0x1234
         jr      $ra
         li      $v0, 0x12345678 ; branch delay slot
\end{lstlisting}

\RU{В выводе на ассемблере от GCC есть псевдоинструкция LUI, но на самом деле, 
там LUI (``Load Upper Imeddiate''), загружающая 16-битное значение в старшую часть регистра.}
\EN{GCC assembly output has LI pseudoinstruction, but in fact, LUI (``Load Upper Imeddiate'') there,
which stores 16-bit value into high part of register.}
