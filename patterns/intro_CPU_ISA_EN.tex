\sectionold{A short introduction to the CPU}

The \ac{CPU} is the device that executes the machine code a program consists of.

\textbf{A short glossary:}

\begin{description}
\item[Instruction]: A primitive \ac{CPU} command.
The simplest examples include: moving data between registers, working with memory, primitive arithmetic operations.
As a rule, each \ac{CPU} has its own instruction set architecture (\ac{ISA}).

\item[Machine code]: Code that the \ac{CPU} directly processes. 
Each instruction is usually encoded by several bytes.
\item[Assembly language]: Mnemonic code and some extensions like macros that are intended to make a programmer's life easier.
\item[CPU register]: Each \ac{CPU} has a fixed set of general purpose registers (\ac{GPR}).
$\approx 8$ in x86, $\approx 16$ in x86-64, $\approx 16$ in ARM.
The easiest way to understand a register is to think of it as an untyped temporary variable.
Imagine if you were working with a high-level \ac{PL} and could only use eight 32-bit (or 64-bit) variables.
Yet a lot can be done using just these!
\end{description}

% TODO1 add about linker: "компоновщик" и "редактор связей" в русскоязычной лит-ре

% A note on the experiments in this area (like the LISP machines http://en.wikipedia.org/wiki/Lisp_machine
% might be useful
One might wonder why there needs to be a difference between machine code and a \ac{PL}.  The answer lies in the fact that humans and \ac{CPU}s are not alike---%
it is much easier for humans to use a high-level \ac{PL} like \CCpp, Java, Python, etc., but it is easier for a \ac{CPU} to use a much lower level of abstraction.
Perhaps it would be possible to invent a \ac{CPU} that can execute high-level \ac{PL} code, but it would be many times more complex than the \ac{CPU}s we know of today.
In a similar fashion, it is very inconvenient for humans to write in assembly language, due to it being so low-level and difficult to write in without making a huge number of annoying mistakes.
The program that converts the high-level \ac{PL} code into assembly is called a \IT{compiler}.
\footnote{Old-school Russian literature also use term \q{translator}.}.

\myindex{ARM!\ARMMode}%
\myindex{ARM!\ThumbMode}%
\myindex{ARM!\ThumbTwoMode}%

\subsectionold{A couple of words about different \ac{ISA}s}
The x86 \ac{ISA} has always been one with variable-length opcodes, so when the 64-bit era came, the x64 extensions did not impact the \ac{ISA} very significantly. In fact, the x86 \ac{ISA} still contains a lot of instructions that first appeared in 16-bit 8086 CPU, yet are still found in the CPUs of today.
ARM is a \ac{RISC} \ac{CPU} designed with constant-length opcode in mind, which had some advantages in the past.
In the very beginning, all ARM instructions were encoded in 4 bytes%
\footnote{
By the way, fixed-length instructions are handy because one can calculate the next (or previous) 
instruction address without effort. This feature will be discussed in the switch() operator~(\myref{sec:SwitchARMLot}) section.
}.
This is now referred to as \q{ARM mode}.
Then they thought it wasn't as frugal as they first imagined.
In fact, most used \ac{CPU} instructions\footnote{These are MOV/PUSH/CALL/Jcc} in real world applications can be encoded using less information.
They therefore added another \ac{ISA}, called Thumb, where each instruction was encoded in just 2 bytes.
This is now referred as \q{Thumb mode}.
However, not \IT{all} ARM instructions can be encoded in just 2 bytes, so the Thumb instruction set is somewhat limited.
It is worth noting that code compiled for ARM mode and Thumb mode may of course coexist within one single program.
The ARM creators thought Thumb could be extended, giving rise to Thumb-2, which appeared in ARMv7.
Thumb-2 still uses 2-byte instructions, but has some new instructions which have the size of 4 bytes.
There is a common misconception that Thumb-2 is a mix of ARM and Thumb. This is incorrect. 
Rather, Thumb-2 was extended to fully support all processor features so it could
compete with ARM mode---a goal that was clearly achieved, as the majority of applications for \idevices are compiled for the Thumb-2 instruction set (admittedly, largely due to the fact that Xcode does this by default).
Later the 64-bit ARM came out. This \ac{ISA} has 4-byte opcodes, and lacked the need of any additional Thumb mode.
However, the 64-bit requirements affected the \ac{ISA}, resulting in us now having three ARM instruction sets: ARM mode, Thumb mode (including Thumb-2) and ARM64.
These \ac{ISA}s intersect partially, but it can be said that they are different \ac{ISA}s, rather than variations of the same one.
Therefore, we would try to add fragments of code in all three ARM \ac{ISA}s in this book.
\myindex{PowerPC}%
\myindex{MIPS}%
\myindex{Alpha AXP}%
There are, by the way, many other \ac{RISC} \ac{ISA}s with fixed length 32-bit opcodes, such as MIPS, PowerPC and Alpha AXP.
