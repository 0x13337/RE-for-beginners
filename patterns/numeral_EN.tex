% TODO translate
\section{Numeral systems}

Humans accustomed to decimal numeral system probably because almost all ones has 10 fingers.
Nevertheless, number 10 has no significant meaning in science and mathematics.
Natural numeral system in digital electronics is binary: 0 is for absence of current in wire and 1 for presence.
10 in binary is 2 in decimal, 100 in binary is 4 in binary and so on.

How to convert from one system to another?

Positional notation is used almost everywhere, this means, the digit (number placed in single character) has some weight depending on where it is placed.
If 2 is placed at the leftmost place, it's 2.
If it is placed at the place one digit before leftmost, it's 20.

What 1234 stands for?

$10^3 \cdot 3 + 10^2 \cdot 2 + 10^1 \cdot 3 + 1 \cdot 4$ = 1234 or 
$1000 \cdot 3 + 100 \cdot 2 + 10 \cdot 3 + 4 = 1234$

Same story for binary numbers, but base is 2 instead of 10.
What 101011 stands for?

$2^5 \cdot 1 + 2^4 \cdot 0 + 2^3 \cdot 1 + 2^2 \cdot 0 + 2^1 \cdot 1 + 1 \cdot 1 = 43$ or
$32 \cdot 1 + 16 \cdot 0 + 8 \cdot 1 + 4 \cdot 0 + 2 \cdot 1 + 1 = 43$

Positional notation can be opposed to non-positional notation such as Roman numeric system.
Perhaps, humankind switched to positional notation because it's easier to do basic operations (addition, multiplication, etc) on paper by hand.

Indeed, binary numbers can be added, subtracted and so on in the very same as taught in schools, but only 2 digits are available.

Binary numbers are bulkey when represented in source code and dumps, so that is where hexadecimal numeral system can be used.
Hexadecimal system uses 0..9 numbers and also 6 Latin characters: A..F.
Each hexadecimal digit takes 4 bits or 4 binary digits, so it's very easy to convert from binary number to hexadecimal and back, even manually, in one's mind.

\begin{center}
\begin{longtable}{ | l | l | l | }
\hline
\HeaderColor hexadecimal & \HeaderColor binary & \HeaderColor decimal \\
\hline
0	&0000	&0 \\
1	&0001	&1 \\
2	&0010	&2 \\
3	&0011	&3 \\
4	&0100	&4 \\
5	&0101	&5 \\
6	&0110	&6 \\
7	&0111	&7 \\
8	&1000	&8 \\
9	&1001	&9 \\
A	&1010	&10 \\
B	&1011	&11 \\
C	&1100	&12 \\
D	&1101	&13 \\
E	&1110	&14 \\
F	&1111	&15 \\
\hline
\end{longtable}
\end{center}

How to detect, which system is currently used?

Decimal numbers are usually written as is, i.e., 1234. But some assemblers allows to make emphasis on decimal system and this number can be written with "d" suffix: 1234d.

Binary numbers sometimes prepended with "0b" prefix: 0b100110111 (\ac{GCC} has non-standard language extension for this: \url{https://gcc.gnu.org/onlinedocs/gcc/Binary-constants.html}).

Hexadecimal numbers are preprended with "0x" prefix in \CCpp and other \ac{PL}s: 0x1234ABCD.
Or they are has "h" suffix: 1234ABCDh - this is common way of representing them in assemblers and debuggers.
If the number is started with A..F digit, 0 should be added before: 0ABCDEFh.

Should one learn to convert numbers in mind? A table of 1-digit hexadecimal numbers can easily be memorized.
As of larger numbers, probably, it's not worth to torment yourself.

\subsection{Octal numeral system}

Another numeral system heavily used in past of computer programming is octal: there are 8 digits (0..7) and each is mapped to 3 bits, so it's easy to convert numbers back and forth.
It has been superseded by hexadecimal system almost everywhere, but surprisingly, there is *NIX utility used by many people often which takes octal number as argument: \TT{chmod}.

\myindex{UNIX!chmod}
As many *NIX users know, \TT{chmod} argument can be a number of 3 digits. The first one is rights for owner of file, second is rights for group (to which file belongs), third is for everyone else.
And each digit can be represented in binary form:

\begin{center}
\begin{longtable}{ | l | l | l | }
\hline
\HeaderColor decimal & \HeaderColor binary & \HeaderColor meaning \\
\hline
7	&111	&\textbf{rwx} \\
6	&110	&\textbf{rw-} \\
5	&101	&\textbf{r-x} \\
4	&100	&\textbf{r-{}-} \\
3	&011	&\textbf{-wx} \\
2	&010	&\textbf{-w-} \\
1	&001	&\textbf{-{}-x} \\
0	&000	&\textbf{-{}-{}-} \\
\hline
\end{longtable}
\end{center}

So the each bit is mapped to flag: read/write/execute.

Now the reason why I'm talking about \TT{chmod} here is that the whole number in argument can be represented as octal number.
Let's take for example, 664.
When you run \TT{chmod 664 file}, you set read/write permissions for owner, read permissions for group and again, read permissions for everyone else.
Let's convert 664 octal number to binary, this will be \TT{110100100}, or \TT{110 100 100}.

Now we see that each triplet describe permissions for owner/group/others: first is \TT{rw-}, second is \TT{r--} and third is \TT{r--}.

Octal numeral system was also popular on old computers like PDP-8, because word there could be 12, 24 or 36 bits, and these numbers are divisible by 3, so octal system was natural on that environment.
Nowadays, all popular computers employs word/address size of 16, 32 or 64 bits, and these numbers are divisible by 4, so hexadecimal system is more natural here.

Octal numeral system is supported by all standard \CCpp compilers.
This is source of confusion sometimes, because octal numbers are encoded with zero prepended, for example, 0277 is 255.
And sometimes, you may make a typo and write "09" instead of 9, and the compiler wouldn't allow you.
GCC may report something like that: \TT{error: invalid digit "9" in octal constant}.

\subsection{Divisibility}

When you see a decimal number like 120, you can quickly deduce that it's divisible by 10, because the last digit is zero.
In the same way, 123400 is divisible by 100, because two last digits are zeroes.

Likewise, hexadecimal number 0x1230 is divisible by 0x10 (or 16), 0x123000 is divisible by 0x1000 (or 4096), etc.

Binary number 0b1000101000 is divisible by 0b1000 (8), etc.

This property can be used often to realize quickly if a size of some block in memory is padded to some boundary.
For example, sections in \ac{PE} files are almost always started at addresses ending with 3 hexadecimal zeroes: 0x41000, 0x10001000, etc.
The reason behind this is in the fact that almost all \ac{PE} sections are padded to 0x1000 (4096) boundary.

