% done

\section{\IFRU{Указатели на функции}{Pointers to functions}}
\label{sec:pointerstofunctions}

\IFRU{Указатель на функцию, в целом, как и любой другой указатель, просто адрес указывающий на начало функции 
в сегменте кода.}
{Pointer to function, as any other pointer, is just an address of function beginning in its code segment.}

\IFRU{Это применяется часто в т.н. callback-ах}{It is often used in callbacks}
\footnote{\url{http://en.wikipedia.org/wiki/Callback_(computer_science)}}.

\IFRU{Известные примеры:}{Well-known examples are:}

\begin{itemize}
\item
\qsort\footnote{\url{http://en.wikipedia.org/wiki/Qsort_(C_standard_library)}},
{\TT{atexit()}}\footnote{\url{http://www.opengroup.org/onlinepubs/009695399/functions/atexit.html}} \IFRU{из стандартной библиотеки Си}{from the standard C library}; 
\item
\IFRU{сигналы в *NIX ОС}{signals in *NIX OS}\footnote{\url{http://en.wikipedia.org/wiki/Signal.h}};
\item
\IFRU{запуск тредов}{thread starting}: \TT{CreateThread()} (win32), \TT{pthread\_create()} (POSIX);
\item
\IFRU{множество функций win32, например}{a lot of win32 functions, for example} \TT{EnumChildWindows()}\footnote{\url{http://msdn.microsoft.com/en-us/library/ms633494(VS.85).aspx}}.
\end{itemize}

\IFRU{Итак, функция \qsort это реализация алгоритма "быстрой сортировки". 
Функция может сортировать что угодно, 
любые типы данных, но при условии что вы имеете функцию сравнения двух элементов данных и 
\qsort может вызывать её.}
{So, \qsort function is a \CCpp standard library quicksort implemenation. The functions is able to sort
anything, any types of data, if you have a function for two elements comparison and \qsort is able
to call it.}

\IFRU{Эта функция сравнения может определяться так:}{The comparison function can be defined as:}

\begin{lstlisting}
int (*compare)(const void *, const void *)
\end{lstlisting}

\IFRU{Воспользуемся немного модифицированным примером, который я нашел вот}
{Let's use slightly modified example I found} \href{http://cplus.about.com/od/learningc/ss/pointers2_8.htm}
{\IFRU{здесь}{here}}:

\lstinputlisting{pointers_to_functions/17_1.c}

\IFRU{Компилируем в MSVC 2010 (я убрал некоторые части для краткости) с опцией \Ox}
{Let's compile it in MSVC 2010 (I omitted some parts for the sake of brefity) with \Ox option}:

\lstinputlisting{pointers_to_functions/17_2_msvc_Ox.asm}

\IFRU{Ничего особо удивительного здесь мы не видим. В качестве четвертого аргумента, 
в \qsort просто передается адрес метки \TT{\_comp}, где собственно и располагается функция \TT{comp()}.}
{Nothing surprising so far. As a fourth argument, an address of label \TT{\_comp} is passed, that's just a place
where function \TT{comp()} located.}

\IFRU{Как \qsort вызывает её?}{How \qsort calling it?}

\IFRU{Посмотрим в MSVCR80.DLL (эта DLL куда в MSVC вынесены функции из стандартных библиотек Си):}
{Let's take a look into this function located in MSVCR80.DLL (a MSVC DLL module with C standard library functions):}

\lstinputlisting{pointers_to_functions/17_3_MSVCR.lst}

\IFRU{\TT{comp} ~--- это четвертый аргумент функции. 
Здесь просто передается управление по адресу указанному в \TT{comp}. 
Перед этим подготавливается два аргумента для функции \TT{comp()}. Далее, проверяется результат её выполнения.}
{\TT{comp} ~--- is fourth function argument.
Here the control is just passed to the address in \TT{comp}.
Before it, two arguments prepared for \TT{comp()}. Its result is checked after its execution.}

\IFRU{Вот почему использование указателей на функции ~--- это опасно. 
Во-первых, если вызвать \qsort с неправильным указателем на функцию, 
то \qsort, дойдя до этого вызова, может передать управление неизвестно куда, 
процесс упадет, и эту ошибку можно будет найти не сразу.}
{That's why it's dangerous to use pointers to functions.
First of all, if you call \qsort with incorrect pointer to function, \qsort may pass control
to incorrect place, process may crash and this bug will be hard to find.}

\IFRU{Во-вторых, типизация callback-функции должна строго соблюдаться, 
вызов не той функции с не теми аргументами не того типа, 
может привести к плачевным результатам, 
хотя падение процесса это и не проблема ~--- а проблема это найти ошибку ~--- ведь компилятор 
на стадии компиляции может вас и не предупредить о потенциальных неприятностях.}
{Second reason is that callback function types should comply strictly, calling wrong function
with wrong arguments of wrong types may lead to serious problems, however, process crashing is not a 
big problem ~--- big problem is to determine a reason of crashing ~--- because compiler may be 
silent about potential trouble while compiling.}

\subsection{GCC}

\IFRU{Не слишком большая разница:}{Not a big difference:}

\begin{lstlisting}
                lea     eax, [esp+40h+var_28]
                mov     [esp+40h+var_40], eax
                mov     [esp+40h+var_28], 764h
                mov     [esp+40h+var_24], 2Dh
                mov     [esp+40h+var_20], 0C8h
                mov     [esp+40h+var_1C], 0FFFFFF9Eh
                mov     [esp+40h+var_18], 0FF7h
                mov     [esp+40h+var_14], 5
                mov     [esp+40h+var_10], 0FFFFCFC7h
                mov     [esp+40h+var_C], 43Fh
                mov     [esp+40h+var_8], 58h
                mov     [esp+40h+var_4], 0FFFE7960h
                mov     [esp+40h+var_34], offset comp
                mov     [esp+40h+var_38], 4
                mov     [esp+40h+var_3C], 0Ah
                call    _qsort
\end{lstlisting}

\IFRU{Функция \TT{comp()}}{\TT{comp()} function}:

\begin{lstlisting}
                public comp
comp            proc near

arg_0           = dword ptr  8
arg_4           = dword ptr  0Ch

                push    ebp
                mov     ebp, esp
                mov     eax, [ebp+arg_4]
                mov     ecx, [ebp+arg_0]
                mov     edx, [eax]
                xor     eax, eax
                cmp     [ecx], edx
                jnz     short loc_8048458
                pop     ebp
                retn
loc_8048458:
                setnl   al
                movzx   eax, al
                lea     eax, [eax+eax-1]
                pop     ebp
                retn
comp            endp
\end{lstlisting}

\IFRU{Реализация \qsort находится в \TT{libc.so.6}, и представляет собой просто враппер для \TT{qsort\_r()}.}
{\qsort implementation is located in \TT{libc.so.6} and it is in fact just a wrapper for \TT{qsort\_r()}.}

\IFRU{Она, в свою очередь, вызывает \TT{quicksort()}, где есть вызовы определенной нами функции через 
переданный указатель:}
{It will call then \TT{quicksort()}, where our defined function will be called via passed pointer:}

\IFRU{(файл libc.so.6, версия glibc ~--- 2.10.1)}{(File libc.so.6, glibc version ~--- 2.10.1)}

\begin{lstlisting}
.text:0002DDF6                 mov     edx, [ebp+arg_10]
.text:0002DDF9                 mov     [esp+4], esi
.text:0002DDFD                 mov     [esp], edi
.text:0002DE00                 mov     [esp+8], edx
.text:0002DE04                 call    [ebp+arg_C]
...
\end{lstlisting}
