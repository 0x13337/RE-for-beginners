\chapter{\IFRU{Предисловие}{Preface}}

\IFRU
{Здесь (будет) немного моих заметок о reverse engineering на русском языке для начинающих, 
для тех кто хочет научиться понимать создаваемый \CCpp компиляторами код для x86 (коего, 
практически, больше всего остального) и ARM.}
{Here (will be) some of my notes about reverse engineering in English language for 
those beginners who like to learn to understand x86 (which is a most large mass of 
all executable software in the world) and ARM code created by \CCpp compilers.}

\IFRU{У термина ``reverse engineering'' несколько популярных значений: 
1) исследование скомпилированных
программ; 2) сканирование трехмерной модели для последующего копирования;
3) восстановление структуры СУБД. Настоящий сборник заметок
связан с первым значением}
{There are several popular meaning of ``reverse engineering'' term: 
1) reverse engineering of software: researching of compiled programs;
3D model scanning and reworking in order to make a copy of it;
3) recreating \ac{DBMS} structure.
These notes are related to the first meaning.}

\section{\IFRU{Мини-}{Mini-}\ac{FAQ}}

\newcommand{\FNURLREDDIT}{\footnote{\url{http://www.reddit.com/r/ReverseEngineering/}}}

\begin{itemize}
\item
Q: \IFRU{Нужно ли учится понимать язык ассемблера в наше время?}
{Should one learn to understand assembly language these days?} \\
A: \IFRU{Да: ради того чтобы понимать лучше внутреннее устройство, отлаживать код лучше и быстрее.}
{Yes: for the sake of understanding internals deeper, debug your software better and faster.}

\item
Q: \IFRU{Нужно ли учиться писать на языке ассемблера в наше время?}
{Should one learn to write in assembly language these days?} \\
A: \IFRU{Пожалуй, нет, если только не писать низкоуровневый код для \ac{OS}.}
{Unless one write low-level \ac{OS} code, probably no.}

\item
Q: \IFRU{Но для написания очень оптимизированных процедур?}
{But for writing highly optimized routines?} \\
A: \IFRU{Нет, современные компиляторы \CCpp делают это лучше.}
{No, modern \CCpp compilers do this job better.}

\item
Q: \IFRU{Нужно ли знать внутреннее устройство микропроцессоров}
{Should I learn microprocessor internals}? \\
A: \IFRU{Современные}{Modern} \ac{CPU}\IFRU{ очень сложные}{-s are very complex}.
\IFRU{Если вы не собираетесь писать очень оптимизированный код,
или не работаете над кодегенератором компилятора,
тогда устройство CPU можно изучать только в общих чертах}
{If you do not plan to write highly optimized code, if you do not work on compiler's code generator,
then you may learn internals in bare outlines.}
\footnote{\IFRU{Очень хороший текст на эту тему}{Very good text about it}: \cite{AgnerFog}}.
\IFRU{В то же время, для понимания и анализа кода,
достаточно только знать \ac{ISA}, назначения регистров, т.е., ``внешнюю'' часть \ac{CPU}, доступную
для прикладного программиста}
{At the same time, in order to understand and analyze of compiled code, it is enough to know
only \ac{ISA}, register's descriptions, i.e., ``outside'' part of \ac{CPU}, available to
application programmer}.

\item
Q: \IFRU{И все таки зачем мне учить ассемблер}{So why should I learn assembly language anyway}? \\
A: \IFRU{В основном, для лучшего понимания происходящего во время отладки и для исследования
программ без наличия исходных кодов}
{Mostly for better understanding while debugging what is going on and for reverse engineering without source codes}.

\item
Q: \IFRU{Как можно найти работу reverse engineer-а}{How would I search for reverse engineering job}? \\
A: \IFRU{На reddit посвященному RE\FNURLREDDIT время от времени бывают hiring thread}
{There are hiring threads appears from time to time on reddit devoted to RE\FNURLREDDIT}
(\href{http://www.reddit.com/r/ReverseEngineering/comments/1hywvr/rreverseengineerings_q3_2013_hiring_thread/}{2013 Q3}),
\IFRU{посмотрите там}{try to take a look there}.
\end{itemize}

\section{\IFRU{Об авторе}{About the author}}

\IFRU{Денис Юричев ~--- опытный reverse engineer, свободный для найма как reverse engineer, консультант, (персональный) преподаватель.
С его резюме можно ознакомиться \href{http://yurichev.com/Dennis_Yurichev.pdf}{здесь}.}
{Dennis Yurichev is an experienced reverse engineer and is available for hire as reverse engineer,
consultant or (personal) teacher.
His CV is available \href{http://yurichev.com/Dennis_Yurichev.pdf}{here}.}

\section{\IFRU{Благодарности}{Thanks}}

\IFRU{Андрей ''herm1t'' Баранович, Слава ''Avid'' Казаков, Станислав ''Beaver'' Бобрицкий, Александр Лысенко, 
Александр ''Lstar'' Черненький, Андрей Зубинский}
{Andrey ''herm1t'' Baranovich, Slava ''Avid'' Kazakov, Stanislav ''Beaver'' Bobrytskyy, Alexander Lysenko, 
Alexander ''Lstar'' Chernenkiy, Andrew Zubinski}, \IFRU{Марк}{Mark} ``Logxen'' \IFRU{Купер}{Cooper},
Shell Rocket, Arnaud Patard (rtp \IFRU{на}{on} \#debian-arm IRC), 
\IFRU{и всем тем на github.com кто присылал замечания и коррективы}{and all folks on github.com
for notes and correctives}.

% \section{\IFRU{Целевая аудитория}{Target audience}}

