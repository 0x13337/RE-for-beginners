\chapter{\IFRU{Введение}{Preface}}

\IFRU
{Здесь (будет) немного моих заметок о reverse engineering на русском языке для начинающих, 
для тех кто хочет научиться понимать создаваемый \CCpp компиляторами код для x86 (коего, 
практически, больше всего остального) и ARM.}
{Here (will be) some of my notes about reverse engineering in English language for 
those beginners who like to learn to understand x86 (which is a most large mass of 
all executable software in the world) and ARM code created by \CCpp compilers.}

%\IFRU{Наиболее используемых компилятора два: MSVC и GCC, на них и будем ставить эксперименты.}
%{There are two most used compilers: MSVC and GCC, these we will use for experiments.}

\IFRU
{Имеется два основных синтаксиса ассемблера: Intel (больше распространенный в DOS/Windows) и 
AT\&T (распространен в *NIX)}
{There are two most used assembly language syntax: Intel (most used in DOS/Windows) and AT\&T (used in *NIX)}
\footnote{\url{http://en.wikipedia.org/wiki/X86_assembly_language\#Syntax}}. 
\IFRU
{Здесь принят Intel-овский синтаксис. \IDA также выдает Intel-овский.}
{Here we use Intel syntax. \IDA produce Intel syntax listings too.}
