\chapter{\IFRU{Введение}{Preface}}

\IFRU
{Здесь (будет) немного моих заметок о reverse engineering на русском языке для начинающих, 
для тех кто хочет научиться понимать создаваемый \CCpp компиляторами код для x86 (коего, 
практически, больше всего остального) и ARM.}
{Here (will be) some of my notes about reverse engineering in English language for 
those beginners who like to learn to understand x86 (which is a most large mass of 
all executable software in the world) and ARM code created by \CCpp compilers.}

\IFRU{У термина ``reverse engineering'' несколько популярных значений: 1) исследование скомпилированных
программ; 2) сканирование трехмерной модели для последующего копирования;
3) восстановление структуры СУБД. Настоящий сборник заметок
связан с первым значением}
{There are several popular meaning of ``reverse engineering'' term: 
1) researching of compiled programs;
3D model scanning and reworking in order to make a copy of it;
3) recreating \ac{DBMS} structure.
These notes are related to the first meaning.}

\section{\IFRU{Мини-}{Mini-}\ac{FAQ}}

\begin{itemize}
\item
Q: \IFRU{Нужно ли учится понимать язык ассемблера в наше время?}
{Should one learn to understand assembly language these days?} \\
A: \IFRU{Да: ради того чтобы понимать лучше внутреннее устройство, отлаживать код лучше и быстрее.}
{Yes: for the sake of understanding internals deeper, debug your software better and faster.}

\item
Q: \IFRU{Нужно ли учиться писать на языке ассемблера в наше время?}
{Should one learn to write in assembly language these days?} \\
A: \IFRU{Пожалуй, нет, если только не писать низкоуровневый код для \ac{OS}.}
{Unless one write low-level \ac{OS} code, probably no.}

\item
Q: \IFRU{Но для написания очень оптимизированных процедур?}
{But for writing highly optimized routines?} \\
A: \IFRU{Нет, современные компиляторы \CCpp делают это лучше.}
{No, modern \CCpp compilers do this job better.}
\end{itemize}

% \section{\IFRU{Целевая аудитория}{Target audience}}

