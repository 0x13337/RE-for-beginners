\section*{Vorwort}

Es gibt verschiedene verbreitete Interpretationen des Begriffs Reverse Engineering:
1) Reverse Engineering von Software: Rückgewinnung des Quellcodes bereits kompilierter Programme;
2) Das Erfassen von 3D Strukturen und die digitale Manipulationen die zur Duplizierung notwendig sind;
3) Nachbilden von \ac{DBMS}-Strukturen.
Dieses Buch behandelt die erste Interpretation.

\subsection*{Themen die in Tiefe behandelt werden}

x86/x64, ARM/ARM64, MIPS, Java/JVM.

\subsection*{weitere behandelte Themen}

\oracle (\myref{oracle}),
Itanium (\myref{itanium}),
copy-protection dongles (\myref{dongles}), 
LD\_PRELOAD (\myref{ld_preload}),
stack overflow,
\ac{ELF},
win32 PE file format (\myref{win32_pe}),
x86-64 (\myref{x86-64}),
critical sections (\myref{critical_sections}),
syscalls (\myref{syscalls}), 
\ac{TLS},
position-independent code (\ac{PIC}) (\myref{sec:PIC}), 
profile-guided optimization (\myref{PGO}),
C++ STL (\myref{cpp_STL}),
OpenMP (\myref{openmp}),
SEH (\myref{sec:SEH}).

\subsection*{Übungen und Aufgaben}

\subsection*{mini-FAQ}

\par Q: \DEph{}
\par A: \DEph{}

% etc



\DEph{}

