\subsection{ARM: \IFRU{8 аргументов в \printf}{8 \printf arguments}}

Для того, чтобы посмотреть, как остальные аргументы будут передаваться через стек, изменим пример еще раз, 
увеличив количество передаваемых аргументов до 9 (строка формата \printf и еще 8 переменных типа \Tint):

\begin{lstlisting}
void printf_main2()
{
	printf("a=%d; b=%d; c=%d; d=%d; e=%d; f=%d; g=%d; h=%d\n", 1, 2, 3, 4, 5, 6, 7, 8);
};
\end{lstlisting}

\subsubsection{\OptimizingKeil: режим ARM}

\begin{lstlisting}
.text:00000028             printf_main2
.text:00000028
.text:00000028             var_18          = -0x18
.text:00000028             var_14          = -0x14
.text:00000028             var_4           = -4
.text:00000028
.text:00000028 04 E0 2D E5                 STR     LR, [SP,#var_4]!
.text:0000002C 14 D0 4D E2                 SUB     SP, SP, #0x14
.text:00000030 08 30 A0 E3                 MOV     R3, #8
.text:00000034 07 20 A0 E3                 MOV     R2, #7
.text:00000038 06 10 A0 E3                 MOV     R1, #6
.text:0000003C 05 00 A0 E3                 MOV     R0, #5
.text:00000040 04 C0 8D E2                 ADD     R12, SP, #0x18+var_14
.text:00000044 0F 00 8C E8                 STMIA   R12, {R0-R3}
.text:00000048 04 00 A0 E3                 MOV     R0, #4
.text:0000004C 00 00 8D E5                 STR     R0, [SP,#0x18+var_18]
.text:00000050 03 30 A0 E3                 MOV     R3, #3
.text:00000054 02 20 A0 E3                 MOV     R2, #2
.text:00000058 01 10 A0 E3                 MOV     R1, #1
.text:0000005C 6E 0F 8F E2                 ADR     R0, aADBDCDDDEDFDGD ; "a=%d; b=%d; c=%d; d=%d; e=%d; f=%d; g=%"...
.text:00000060 BC 18 00 EB                 BL      __2printf
.text:00000064 14 D0 8D E2                 ADD     SP, SP, #0x14
.text:00000068 04 F0 9D E4                 LDR     PC, [SP+4+var_4],#4
\end{lstlisting}

Этот код можно условно разделить на несколько частей:

\begin{itemize}
\item Пролог функции:

Самая первая инструкция \TT{``STR LR, [SP,\#var\_4]!''} сохраняет в стеке \LR, ведь, 
нам придется использовать его для вызова \printf.

Вторая инструкция \TT{``SUB SP, SP, \#0x14''} уменьшает указатель стека \SP, но на самом деле, эта процедура нужна для выделения в локальном стеке места размером 0x14 (20) байт. Действительно, нам нужно передать 5 32-битных значений через стек в \printf, каждое значение занимает 4 байта, а $5*4=20$ --- как раз. Остальные 4 32-битных значения будут переданы через регистры.

\item Передача 5, 6, 7 и 8 через стек:

Затем значения 5, 6, 7 и 8 записываются в регистры R0, R1, R2 и R3 соответственно. Затем инструкция \TT{``ADD R12, SP, \#0x18+var\_14''} записывает в регистр R12 место в стеке, куда будут помещены эти 4 значения. 
\IT{var\_14} это макрос ассемблера, равный $-0x14$, такие макросы создает \IDA, чтобы удобнее было показывать, как код обращается к стеку. Переменные \IT{var\_?}, создаваемые в \IDA это локальные переменные в стеке. 
Так что, в \TT{R12} будет записано $SP+4$. 
Следующая инструкция \TT{``STMIA R12, {R0-R3}''} записывает содержимое регистров R0-R3 по адресу в памяти, на который указывает R12. 
Инструкция \TT{STMIA} означает \IT{Store Multiple Increment After}. 
\IT{Increment After} означает что R12 будет увеличиваться на 4 после записи каждого значения регистра.

\item Передача 4 через стек:
4 записывается в R0, затем, это значение, при помощи инструкции \TT{``STR R0, [SP,\#0x18+var\_18]''} попадает
в стек. \IT{var\_18} равен $-0x18$, смещение будет 0, так что, значение из регистра R0 (4) запишется туда, куда
указывает \SP.

\item Передача 1, 2 и 3 через регистры:

Значения для первых трех чисел (a, b, c) (1, 2, 3 соответственно) передаются в регистрах R1, R2 и R3 перед самим
вызововм \printf, а остальные 5 значений передаются через стек, и вот как.

\item Вызов \printf:

\item Эпилог функции:

Инструкция \TT{``ADD SP, SP, \#0x14''} возвращает \SP на прежнее место, аннулируя таким образом, всё что было
записано в стеке. Конечно, то что было записано туда, там пока и остается, но всё это будет многократно 
перезаписано следующими функциями.

Инструкция \TT{``LDR PC, [SP+4+var\_4],\#4''} загружает в \PC сохраненное значение \LR из стека, таким образом,
обеспечивая выход из функции.

\end{itemize}

\subsubsection{\OptimizingKeil: режим thumb}

\begin{lstlisting}
.text:0000001C             printf_main2
.text:0000001C
.text:0000001C             var_18          = -0x18
.text:0000001C             var_14          = -0x14
.text:0000001C             var_8           = -8
.text:0000001C
.text:0000001C 00 B5                       PUSH    {LR}
.text:0000001E 08 23                       MOVS    R3, #8
.text:00000020 85 B0                       SUB     SP, SP, #0x14
.text:00000022 04 93                       STR     R3, [SP,#0x18+var_8]
.text:00000024 07 22                       MOVS    R2, #7
.text:00000026 06 21                       MOVS    R1, #6
.text:00000028 05 20                       MOVS    R0, #5
.text:0000002A 01 AB                       ADD     R3, SP, #0x18+var_14
.text:0000002C 07 C3                       STMIA   R3!, {R0-R2}
.text:0000002E 04 20                       MOVS    R0, #4
.text:00000030 00 90                       STR     R0, [SP,#0x18+var_18]
.text:00000032 03 23                       MOVS    R3, #3
.text:00000034 02 22                       MOVS    R2, #2
.text:00000036 01 21                       MOVS    R1, #1
.text:00000038 A0 A0                       ADR     R0, aADBDCDDDEDFDGD ; "a=%d; b=%d; c=%d; d=%d; e=%d; f=%d; g=%"...
.text:0000003A 06 F0 D9 F8                 BL      __2printf
.text:0000003E
.text:0000003E             loc_3E                                  ; CODE XREF: example13_f+16
.text:0000003E 05 B0                       ADD     SP, SP, #0x14
.text:00000040 00 BD                       POP     {PC}
\end{lstlisting}

Это почти то же самое что и в предыдущем примере, только код для thumb и значения укладываются в 
стек немного иначе: в начале 8 за первый раз, затем 5, 6, 7 за второй раз и 4 за третий раз.

\subsubsection{\OptimizingXcode: режим ARM}

\begin{lstlisting}
__text:0000290C             _printf_main2
__text:0000290C
__text:0000290C             var_1C          = -0x1C
__text:0000290C             var_C           = -0xC
__text:0000290C
__text:0000290C 80 40 2D E9                 STMFD           SP!, {R7,LR}
__text:00002910 0D 70 A0 E1                 MOV             R7, SP
__text:00002914 14 D0 4D E2                 SUB             SP, SP, #0x14
__text:00002918 70 05 01 E3                 MOV             R0, #0x1570
__text:0000291C 07 C0 A0 E3                 MOV             R12, #7
__text:00002920 00 00 40 E3                 MOVT            R0, #0
__text:00002924 04 20 A0 E3                 MOV             R2, #4
__text:00002928 00 00 8F E0                 ADD             R0, PC, R0
__text:0000292C 06 30 A0 E3                 MOV             R3, #6
__text:00002930 05 10 A0 E3                 MOV             R1, #5
__text:00002934 00 20 8D E5                 STR             R2, [SP,#0x1C+var_1C]
__text:00002938 0A 10 8D E9                 STMFA           SP, {R1,R3,R12}
__text:0000293C 08 90 A0 E3                 MOV             R9, #8
__text:00002940 01 10 A0 E3                 MOV             R1, #1
__text:00002944 02 20 A0 E3                 MOV             R2, #2
__text:00002948 03 30 A0 E3                 MOV             R3, #3
__text:0000294C 10 90 8D E5                 STR             R9, [SP,#0x1C+var_C]
__text:00002950 A4 05 00 EB                 BL              _printf
__text:00002954 07 D0 A0 E1                 MOV             SP, R7
__text:00002958 80 80 BD E8                 LDMFD           SP!, {R7,PC}
\end{lstlisting}

Почти то же самое что мы уже видели, за исключением того что \TT{STMFA} (Store Multiple Full Ascending) это синоним
инструкции \TT{STMIB} (Store Multiple Increment Before). 
Эта инструкция увеличивает \SP и только затем записывает в память очередной регистр, но не наоборот.

Второе что бросается в глаза, это то что инструкции как будто бы расположены случайно. Например, регистр R0
подготавливается в трех местах, по адресам 0x2918, 0x2920, 0x2928, когда это можно было бы сделать в одном месте.
Однако, у оптимизирующего компилятора могут быть свои доводы о том, как лучше составлять инструкции друг с другом
для лучшей эффективности исполнения.
Процессор обычно старается исполнять одновременно идущие друг за другом инструкции.
К примеру, инструкции \TT{``MOVT R0, \#0''} и \TT{``ADD R0, PC, R0''} не могут быть исполнены одновременно,
потому что обе инструкции модифицируют R0. 
А вот инструкции \TT{``MOVT R0, \#0''} и \TT{``MOV R2, \#4''} легко можно исполнить одновременно, 
потому что их действия никак не конфликтуют друг с другом. 
Вероятно, компилятор старается генерировать код именно таким образом, конечно, там где это возможно.
 
\subsubsection{\OptimizingXcode: режим thumb}

\begin{lstlisting}
__text:00002BA0                   _printf_main2
__text:00002BA0
__text:00002BA0                   var_1C          = -0x1C
__text:00002BA0                   var_18          = -0x18
__text:00002BA0                   var_C           = -0xC
__text:00002BA0
__text:00002BA0 80 B5                             PUSH            {R7,LR}
__text:00002BA2 6F 46                             MOV             R7, SP
__text:00002BA4 85 B0                             SUB             SP, SP, #0x14
__text:00002BA6 41 F2 D8 20                       MOVW            R0, #0x12D8
__text:00002BAA 4F F0 07 0C                       MOV.W           R12, #7
__text:00002BAE C0 F2 00 00                       MOVT.W          R0, #0
__text:00002BB2 04 22                             MOVS            R2, #4
__text:00002BB4 78 44                             ADD             R0, PC  ; char *
__text:00002BB6 06 23                             MOVS            R3, #6
__text:00002BB8 05 21                             MOVS            R1, #5
__text:00002BBA 0D F1 04 0E                       ADD.W           LR, SP, #0x1C+var_18
__text:00002BBE 00 92                             STR             R2, [SP,#0x1C+var_1C]
__text:00002BC0 4F F0 08 09                       MOV.W           R9, #8
__text:00002BC4 8E E8 0A 10                       STMIA.W         LR, {R1,R3,R12}
__text:00002BC8 01 21                             MOVS            R1, #1
__text:00002BCA 02 22                             MOVS            R2, #2
__text:00002BCC 03 23                             MOVS            R3, #3
__text:00002BCE CD F8 10 90                       STR.W           R9, [SP,#0x1C+var_C]
__text:00002BD2 01 F0 0A EA                       BLX             _printf
__text:00002BD6 05 B0                             ADD             SP, SP, #0x14
__text:00002BD8 80 BD                             POP             {R7,PC}
\end{lstlisting}

Почти то же самое что и в  предыдущем примере, лишь за тем исключением что здесь используются thumb-инструкции.

