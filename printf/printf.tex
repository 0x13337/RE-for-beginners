% done

\section{\IFRU{printf() с несколькими агрументами}{
printf() with several arguments}}

\IFRU{Попробуем теперь немного расширить пример \IT{Hello, world!}~\ref{sec:helloworld}, 
написав в теле функции \main:}
{Now let's extend \IT{Hello, world!}~\ref{sec:helloworld} example, replacing \printf in 
\main function body by this:}

\begin{lstlisting}
printf("a=%d; b=%d; c=%d", 1, 2, 3);
\end{lstlisting}

\IFRU{Компилируем при помощи MSVC 2010 Express, и в итоге получим:}
{Let's compile it by MSVC 2010 and we got:}

\begin{lstlisting}
$SG3830	DB	'a=%d; b=%d; c=%d', 00H

...

	push	3
	push	2
	push	1
	push	OFFSET $SG3830
	call	_printf
	add	esp, 16					; 00000010H
\end{lstlisting}

\IFRU{Все почти то же, за исключением того, что теперь видно, что аргументы для \printf заталкиваются в стек в обратном порядке: самый первый аргумент заталкивается последним.}
{Almost the same, but now we can see that \printf arguments are pushing into stack in reverse order: and the first argument is pushing in as the last one.}

\IFRU
{Всего 4 аргумента. 4*4 = 16 ~--- именно 16 байт занимают в стеке указатель на строку плюс еще 
3 числа типа \Tint.}
{Here 4 arguments. 4*4 = 16 ~--- exactly 16 byte they occupy in stack: 32-bit pointer to string and 
3 number of \Tint type.}

\IFRU{Переменные типа \Tint в 32-битной системе, как известно, имеет ширину 32 бита.}
{Variables of \Tint type in 32-bit environment has 32-bit width.}

\IFRU{Когда при помощи инструкции \TT{ADD ESP, X} корректируется указатель стека \ESP 
после вызова какой-либо функции, зачастую можно сделать вывод о том, сколько аргументов 
у вызываемой функции было, разделив X на 4.}
{When stack pointer (\ESP register) is corrected by \TT{ADD ESP, X} instruction after some function 
call, often, the number of function arguments could be deduced here: just divide X by 4.}

\IFRU{Конечно, это относится только к cdecl-методу передачи аргументов через стек.}
{Of course, this is related only to \IT{cdecl} calling convention.}

\IFRU{См.также в соответствующем разделе о способах передачи аргументов через стек}
{See also section about calling conventions}~\ref{sec:callingconventions}.

\IFRU{Иногда бывает так, что подряд идут несколько вызовов разных функций, 
но стек корректируется только один раз, после последнего вызова:}
{It is also possible for compiler to merge several \TT{ADD ESP, X} instructions into one, after last call:}

\begin{lstlisting}
push a1
push a2
call ...
...
push a1
call ...
...
push a1
push a2
push a3
call ...
add esp, 24
\end{lstlisting}

\IFRU{Скомпилируем то же самое в Linux при помощи GCC 4.4.1 и посмотрим в \IDA что вышло:}
{Now let's compile the same in Linux by GCC 4.4.1 and take a look in \IDA what we got:}

\begin{lstlisting}
main            proc near

var_10          = dword ptr -10h
var_C           = dword ptr -0Ch
var_8           = dword ptr -8
var_4           = dword ptr -4

                push    ebp
                mov     ebp, esp
                and     esp, 0FFFFFFF0h
                sub     esp, 10h
                mov     eax, offset aADBDCD ; "a=%d; b=%d; c=%d"
                mov     [esp+10h+var_4], 3
                mov     [esp+10h+var_8], 2
                mov     [esp+10h+var_C], 1
                mov     [esp+10h+var_10], eax
                call    _printf
                mov     eax, 0
                leave
                retn
main            endp
\end{lstlisting}

\IFRU{Можно сказать, что этот короткий код созданный GCC отличается от кода MSVC только способом помещения 
значений в стек.
Здесь GCC снова работает со стеком напрямую без \PUSH/\POP.}
{It can be said, the difference between code by MSVC and GCC is only in method of placing arguments into stack. 
Here GCC working directly with stack without \PUSH/\POP.}

\IFRU{Кстати, эта разница неплохо иллюстрирует тот важный момент, что процессору, в общем, все равно как будут 
передаваться параметры функций. Можно создать гипотетический компилятор, который будет передавать их при 
помощи указателя на структуру с параметрами, не пользуясь стеком вообще.}
{By the way, this difference is a good illustration that CPU is not aware of how arguments is passed to functions. 
It is also possible to create hypothetical compiler which is able to pass arguments 
via some special structure not using stack at all.}
