\section{\printf \IFRU{с несколькими агрументами}{with several arguments}}

\IFRU{Попробуем теперь немного расширить пример \IT{\HelloWorldSectionName}~\ref{sec:helloworld}, 
написав в теле функции \main:}
{Now let's extend \IT{\HelloWorldSectionName}~\ref{sec:helloworld} example, replacing \printf in 
\main function body by this:}

\begin{lstlisting}
printf("a=%d; b=%d; c=%d", 1, 2, 3);
\end{lstlisting}

\subsection{x86}

\IFRU{Компилируем при помощи MSVC 2010 Express, и в итоге получим:}
{Let's compile it by MSVC 2010 and we got:}

\begin{lstlisting}
$SG3830	DB	'a=%d; b=%d; c=%d', 00H

...

	push	3
	push	2
	push	1
	push	OFFSET $SG3830
	call	_printf
	add	esp, 16					; 00000010H
\end{lstlisting}

\IFRU{Все почти то же, за исключением того, что теперь видно, что аргументы для \printf заталкиваются в стек в обратном порядке: самый первый аргумент заталкивается последним.}
{Almost the same, but now we can see that \printf arguments are pushing into stack in reverse order: and the first argument is pushing in as the last one.}

\IFRU{Кстати, вспомним что переменные типа \Tint в 32-битной системе, как известно, имеет ширину 32 бита, это 4 байта.}
{By the way, variables of \Tint type in 32-bit environment has 32-bit width, that's 4 bytes.}

\IFRU{Итак, у нас всего 4 аргумента. 4*4 = 16 ~--- именно 16 байт занимают в стеке указатель на строку плюс еще 3 числа типа \Tint.}
{So, we got here 4 arguments. 4*4 = 16 ~--- they occupy in stack exactly 16 bytes: 32-bit pointer to string and 3 number of \Tint type.}

\IFRU{Когда при помощи инструкции \TT{``ADD ESP, X''} корректируется указатель стека \ESP 
после вызова какой-либо функции, зачастую можно сделать вывод о том, сколько аргументов 
у вызываемой функции было, разделив X на 4.}
{When stack pointer (\ESP register) is corrected by \TT{``ADD ESP, X''} instruction after some function 
call, often, the number of function arguments could be deduced here: just divide X by 4.}

\IFRU{Конечно, это относится только к cdecl-методу передачи аргументов через стек.}
{Of course, this is related only to \IT{cdecl} calling convention.}

\IFRU{См.также в соответствующем разделе о способах передачи аргументов через стек}
{See also section about calling conventions}~\ref{sec:callingconventions}.

\IFRU{Иногда бывает так, что подряд идут несколько вызовов разных функций, 
но стек корректируется только один раз, после последнего вызова:}
{It is also possible for compiler to merge several \TT{``ADD ESP, X''} instructions into one, after last call:}

\begin{lstlisting}
push a1
push a2
call ...
...
push a1
call ...
...
push a1
push a2
push a3
call ...
add esp, 24
\end{lstlisting}

\IFRU{Скомпилируем то же самое в Linux при помощи GCC 4.4.1 и посмотрим в \IDA что вышло:}
{Now let's compile the same in Linux by GCC 4.4.1 and take a look in \IDA what we got:}

\begin{lstlisting}
main            proc near

var_10          = dword ptr -10h
var_C           = dword ptr -0Ch
var_8           = dword ptr -8
var_4           = dword ptr -4

                push    ebp
                mov     ebp, esp
                and     esp, 0FFFFFFF0h
                sub     esp, 10h
                mov     eax, offset aADBDCD ; "a=%d; b=%d; c=%d"
                mov     [esp+10h+var_4], 3
                mov     [esp+10h+var_8], 2
                mov     [esp+10h+var_C], 1
                mov     [esp+10h+var_10], eax
                call    _printf
                mov     eax, 0
                leave
                retn
main            endp
\end{lstlisting}

\IFRU{Можно сказать, что этот короткий код созданный GCC отличается от кода MSVC только способом помещения 
значений в стек.
Здесь GCC снова работает со стеком напрямую без \PUSH/\POP.}
{It can be said, the difference between code by MSVC and GCC is only in method of placing arguments into stack. 
Here GCC working directly with stack without \PUSH/\POP.}


%NOTTRANSLATED
\subsection{ARM: 3 аргумента в \printf}

В ARM традиционно принята такая схема передачи аргументов в функцию: 4 первых аргумента через регистры R0-R3,
а остальные ~--- через стек. Это немного похоже на то как аргументы передаются в fastcall~\ref{fastcall} или 
win64~\ref{sec:callingconventions_win64}.

\subsubsection{Оптимизирующий Keil: режим ARM}

\begin{lstlisting}
.text:00000014                             EXPORT printf_main1
.text:00000014             printf_main1
.text:00000014 03 30 A0 E3                 MOV     R3, #3
.text:00000018 02 20 A0 E3                 MOV     R2, #2
.text:0000001C 01 10 A0 E3                 MOV     R1, #1
.text:00000020 1E 0E 8F E2                 ADR     R0, aADBDCD     ; "a=%d; b=%d; c=%d\n"
.text:00000024 CB 18 00 EA                 B       __2printf
\end{lstlisting}

\subsubsection{Оптимизирующий Keil: режим thumb}

\begin{lstlisting}
.text:0000000C             printf_main1
.text:0000000C 10 B5                       PUSH    {R4,LR}
.text:0000000E 03 23                       MOVS    R3, #3
.text:00000010 02 22                       MOVS    R2, #2
.text:00000012 01 21                       MOVS    R1, #1
.text:00000014 A4 A0                       ADR     R0, aADBDCD     ; "a=%d; b=%d; c=%d\n"
.text:00000016 06 F0 EB F8                 BL      __2printf
.text:0000001A 10 BD                       POP     {R4,PC}
\end{lstlisting}

\subsection{ARM: 8 аргументов в \printf}

Для того, чтобы посмотреть, как остальные аргументы будут передаваться через стек, изменим пример еще раз, 
увеличив количество передаваемых аргументов до 9 (строка формата \printf и еще 8 переменных типа \Tint):

\begin{lstlisting}
void printf_main2()
{
	printf("a=%d; b=%d; c=%d; d=%d; e=%d; f=%d; g=%d; h=%d\n", 1, 2, 3, 4, 5, 6, 7, 8);
};
\end{lstlisting}

\subsubsection{Оптимизирующий Keil: режим ARM}

\begin{lstlisting}
.text:00000028             printf_main2
.text:00000028
.text:00000028             var_18          = -0x18
.text:00000028             var_14          = -0x14
.text:00000028             var_4           = -4
.text:00000028
.text:00000028 04 E0 2D E5                 STR     LR, [SP,#var_4]!
.text:0000002C 14 D0 4D E2                 SUB     SP, SP, #0x14
.text:00000030 08 30 A0 E3                 MOV     R3, #8
.text:00000034 07 20 A0 E3                 MOV     R2, #7
.text:00000038 06 10 A0 E3                 MOV     R1, #6
.text:0000003C 05 00 A0 E3                 MOV     R0, #5
.text:00000040 04 C0 8D E2                 ADD     R12, SP, #0x18+var_14
.text:00000044 0F 00 8C E8                 STMIA   R12, {R0-R3}
.text:00000048 04 00 A0 E3                 MOV     R0, #4
.text:0000004C 00 00 8D E5                 STR     R0, [SP,#0x18+var_18]
.text:00000050 03 30 A0 E3                 MOV     R3, #3
.text:00000054 02 20 A0 E3                 MOV     R2, #2
.text:00000058 01 10 A0 E3                 MOV     R1, #1
.text:0000005C 6E 0F 8F E2                 ADR     R0, aADBDCDDDEDFDGD ; "a=%d; b=%d; c=%d; d=%d; e=%d; f=%d; g=%"...
.text:00000060 BC 18 00 EB                 BL      __2printf
.text:00000064 14 D0 8D E2                 ADD     SP, SP, #0x14
.text:00000068 04 F0 9D E4                 LDR     PC, [SP+4+var_4],#4
\end{lstlisting}


\subsubsection{Оптимизирующий Keil: режим thumb}

\begin{lstlisting}
.text:0000001C             printf_main2
.text:0000001C
.text:0000001C             var_18          = -0x18
.text:0000001C             var_14          = -0x14
.text:0000001C             var_8           = -8
.text:0000001C
.text:0000001C 00 B5                       PUSH    {LR}
.text:0000001E 08 23                       MOVS    R3, #8
.text:00000020 85 B0                       SUB     SP, SP, #0x14
.text:00000022 04 93                       STR     R3, [SP,#0x18+var_8]
.text:00000024 07 22                       MOVS    R2, #7
.text:00000026 06 21                       MOVS    R1, #6
.text:00000028 05 20                       MOVS    R0, #5
.text:0000002A 01 AB                       ADD     R3, SP, #0x18+var_14
.text:0000002C 07 C3                       STMIA   R3!, {R0-R2}
.text:0000002E 04 20                       MOVS    R0, #4
.text:00000030 00 90                       STR     R0, [SP,#0x18+var_18]
.text:00000032 03 23                       MOVS    R3, #3
.text:00000034 02 22                       MOVS    R2, #2
.text:00000036 01 21                       MOVS    R1, #1
.text:00000038 A0 A0                       ADR     R0, aADBDCDDDEDFDGD ; "a=%d; b=%d; c=%d; d=%d; e=%d; f=%d; g=%"...
.text:0000003A 06 F0 D9 F8                 BL      __2printf
.text:0000003E
.text:0000003E             loc_3E                                  ; CODE XREF: example13_f+16j
.text:0000003E 05 B0                       ADD     SP, SP, #0x14
.text:00000040 00 BD                       POP     {PC}

\end{lstlisting}


\subsubsection{Оптимизирующий Xcode: режим ARM}

\begin{lstlisting}
__text:0000290C             _printf_main2
__text:0000290C
__text:0000290C             var_1C          = -0x1C
__text:0000290C             var_C           = -0xC
__text:0000290C
__text:0000290C 80 40 2D E9                 STMFD           SP!, {R7,LR}
__text:00002910 0D 70 A0 E1                 MOV             R7, SP
__text:00002914 14 D0 4D E2                 SUB             SP, SP, #0x14
__text:00002918 70 05 01 E3                 MOV             R0, #0x1570
__text:0000291C 07 C0 A0 E3                 MOV             R12, #7
__text:00002920 00 00 40 E3                 MOVT            R0, #0
__text:00002924 04 20 A0 E3                 MOV             R2, #4
__text:00002928 00 00 8F E0                 ADD             R0, PC, R0
__text:0000292C 06 30 A0 E3                 MOV             R3, #6
__text:00002930 05 10 A0 E3                 MOV             R1, #5
__text:00002934 00 20 8D E5                 STR             R2, [SP,#0x1C+var_1C]
__text:00002938 0A 10 8D E9                 STMFA           SP, {R1,R3,R12}
__text:0000293C 08 90 A0 E3                 MOV             R9, #8
__text:00002940 01 10 A0 E3                 MOV             R1, #1
__text:00002944 02 20 A0 E3                 MOV             R2, #2
__text:00002948 03 30 A0 E3                 MOV             R3, #3
__text:0000294C 10 90 8D E5                 STR             R9, [SP,#0x1C+var_C]
__text:00002950 A4 05 00 EB                 BL              _printf
__text:00002954 07 D0 A0 E1                 MOV             SP, R7
__text:00002958 80 80 BD E8                 LDMFD           SP!, {R7,PC}
\end{lstlisting}


\subsubsection{Оптимизирующий Xcode: режим thumb}

\begin{lstlisting}
__text:00002BA0                   _printf_main2
__text:00002BA0
__text:00002BA0                   var_1C          = -0x1C
__text:00002BA0                   var_18          = -0x18
__text:00002BA0                   var_C           = -0xC
__text:00002BA0
__text:00002BA0 80 B5                             PUSH            {R7,LR}
__text:00002BA2 6F 46                             MOV             R7, SP
__text:00002BA4 85 B0                             SUB             SP, SP, #0x14
__text:00002BA6 41 F2 D8 20                       MOVW            R0, #0x12D8
__text:00002BAA 4F F0 07 0C                       MOV.W           R12, #7
__text:00002BAE C0 F2 00 00                       MOVT.W          R0, #0
__text:00002BB2 04 22                             MOVS            R2, #4
__text:00002BB4 78 44                             ADD             R0, PC  ; char *
__text:00002BB6 06 23                             MOVS            R3, #6
__text:00002BB8 05 21                             MOVS            R1, #5
__text:00002BBA 0D F1 04 0E                       ADD.W           LR, SP, #0x1C+var_18
__text:00002BBE 00 92                             STR             R2, [SP,#0x1C+var_1C]
__text:00002BC0 4F F0 08 09                       MOV.W           R9, #8
__text:00002BC4 8E E8 0A 10                       STMIA.W         LR, {R1,R3,R12}
__text:00002BC8 01 21                             MOVS            R1, #1
__text:00002BCA 02 22                             MOVS            R2, #2
__text:00002BCC 03 23                             MOVS            R3, #3
__text:00002BCE CD F8 10 90                       STR.W           R9, [SP,#0x1C+var_C]
__text:00002BD2 01 F0 0A EA                       BLX             _printf
__text:00002BD6 05 B0                             ADD             SP, SP, #0x14
__text:00002BD8 80 BD                             POP             {R7,PC}
\end{lstlisting}




\subsection{\IFRU{Кстати}{By the way}}

\IFRU{Кстати, разница между способом передачи параметров принятая в x86 и ARM, неплохо иллюстрирует тот важный момент, что процессору, в общем, все равно как будут 
передаваться параметры функций. Можно создать гипотетический компилятор, который будет передавать их при 
помощи указателя на структуру с параметрами, не пользуясь стеком вообще.}
{By the way, this difference between passing arguments in x86 and ARM is a good illustration that CPU is not aware of how arguments is passed to functions. 
It is also possible to create hypothetical compiler which is able to pass arguments 
via some special structure not using stack at all.}

