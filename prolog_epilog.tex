\section{\IFRU{Пролог и эпилог в функции}{Function prologue and epilogue}}
\label{sec:prologepilog}
\index{Function epilogue}
\index{Function prologue}

\IFRU{Пролог функции это инструкции в самом начале функции. Как правило это что-то вроде такого
фрагмента кода:}
{Function prologue is instructions at function start. It is often something like the following
code fragment:}

\begin{lstlisting}
    push    ebp
    mov     ebp, esp
    sub     esp, X
\end{lstlisting}

\IFRU
{Эти инструкции делают следующее: сохраняют значение регистра \EBP на будущее, выставляют \EBP равным \ESP, 
затем подготавливают место в стеке для хранения локальных переменных.}
{What these instruction do: saves the value in the \EBP register,
set value of the \EBP register to the value of the \ESP and then allocates space on the stack 
for local variables.}

\IFRU{\EBP сохраняет свое значение на протяжении всей функции, он будет использоваться здесь для доступа 
к локальным переменным и аргументам. Можно было бы использовать и \ESP, но он постоянно меняется и 
это не очень удобно.}
{Value in the \EBP is fixed over a period of function execution and it is to be used for local variables and 
arguments access. 
One can use \ESP, but it changing over time and it is not convenient.}

\IFRU{Эпилог функции аннулирует выделенное место в стеке, возвращает значение \EBP на то что было и возвращает 
управление в вызывающую функцию:}
{Function epilogue annulled allocated space in stack, returns value in the \EBP register back to initial state 
and returns flow control to \gls{callee}:}

\begin{lstlisting}
    mov    esp, ebp
    pop    ebp
    ret    0
\end{lstlisting}

\index{\Recursion}
\IFRU{Наличие эпилога и пролога может несколько ухудшить эффективность рекурсии.}
{Epilogue and prologue can make recursion performance worse.}

\IFRU{Например, однажды я написал функцию для поиска нужного узла в двоичном дереве. 
Рекурсивно она выглядела очень красиво, но из-за того что при каждом вызове тратилось время на эпилог и пролог, 
все это работало в несколько раз медленнее чем та же функция но без рекурсии.}
{For example, once upon a time I wrote a function to seek right node in binary tree. 
As a recursive function it would look stylish but since an additional time
is to be spend at each function call
for prologue/epilogue, it was working couple of times slower than iterative (recursion-free)
implementation.}

\index{\Recursion!Tail recursion}
\IFRU{Кстати, поэтому есть такая вещь как \glslink{tail call}{хвостовая рекурсия}}
{By the way, that is the reason of \gls{tail call} existence}.
