\section{\IFRU{Пролог и эпилог в функции}{Function prologue and epilogue}}
\label{sec:prologepilog}
\index{Function epilogue}
\index{Function prologue}

\IFRU{Пролог функции это инструкции в самом начале функции. Как правило это что-то вроде такого
фрагмента кода:}
{Function prologue is instructions at function start. It is often something like the following
code fragment:}

\begin{lstlisting}
    push    ebp
    mov     ebp, esp
    sub     esp, X
\end{lstlisting}

\IFRU
{Эти инструкции делают следующее: сохраняют значение регистра \EBP на будущее, выставляют \EBP равным \ESP, 
затем подготавливают место в стеке для хранения локальных переменных.}
{What these instruction do: save \EBP register value, set \EBP to \ESP and then allocate space in stack 
for local variables.}

\IFRU{\EBP сохраняет свое значение на протяжении всей функции, он будет использоваться здесь для доступа 
к локальным переменным и аргументам. Можно было бы использовать и \ESP, но он постоянно меняется и 
это не очень удобно.}
{\EBP value is fixed over a period of function execution and it will be used for local variables and 
arguments access. 
One can use \ESP, but it changing over time and it is not convenient.}

\IFRU{Эпилог функции аннулирует выделенное место в стеке, возвращает значение \EBP на то что было и возвращает 
управление в вызывающую функцию:}
{Function epilogue annuled allocated space in stack, returning \EBP value to initial state 
and returning control flow to callee:}

\begin{lstlisting}
    mov    esp, ebp
    pop    ebp
    ret    0
\end{lstlisting}

\index{\Recursion}
\IFRU{Наличие эпилога и пролога может несколько ухудшить эффективность рекурсии.

Например, однажды я написал функцию для поиска нужного узла в двоичном дереве. 
Рекурсивно она выглядела очень красиво, но из-за того что при каждом вызове тратилось время на эпилог и пролог, 
все это работало в несколько раз медленнее чем та же функция но без рекурсии.}
{Epilogue and prologue can make recursion performance worse.

For example, once upon a time I wrote a function to seek right node in binary tree. 
As a recursive function it would look stylish but because some time was spent at each function call
for prologue/epilogue, it was working couple of times slower than the implementation 
without recursion.}

\index{\Recursion!Tail recursion}
\newcommand{\URLT}{\url{http://en.wikipedia.org/wiki/Tail_call}}
\IFRU
{Кстати, поэтому есть такая вещь как хвостовая рекурсия\footnote{\URLT}: 
когда компилятор или интерпретатор превращает рекурсию (с которой возможно это проделать: 
\IT{хвостовую}) в итерацию для эффективности.}
{By the way, that is the reason of tail call\footnote{\URLT} existence: when compiler (or interpreter) 
transforms recursion (with which it's possible: \IT{tail recursion}) into iteration for efficiency.}
