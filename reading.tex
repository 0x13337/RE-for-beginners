\part{%
	\RU{Что стоит почитать}%
	\EN{Books/blogs worth reading}%
	\ES{Libros/blogs que merecen lectura}%
	\PTBRph{}%
	\DEph{}\PLph{}%
	\ITAph{}%
}

% TODO названия книг а не ссылки!
\chapter{%
	\RU{Книги}%
	\EN{Books}%
	\ES{Libros}%
	\PTBRph{}%
	\DEph{}\PLph{}%
	\ITAph{}%
}

\section{Windows}

\cite{Russinovich}.

\section{\CCpp}

\cite{CPP11}.

\section{x86 / x86-64}

\cite{Intel}, \cite{AMD}

\section{ARM}

\RU{Документация от ARM:}%
\EN{ARM manuals:}%
\ES{Manuales de ARM:}%
\PTBRph{}%
\DEph{}\PLph{}%
\ITAph{}
\url{http://go.yurichev.com/17024}

\section{%
	\RU{Криптография}%
	\EN{Cryptography}%
	\ES{Criptograf\'ia}%
	\PTBRph{}%
	\DEph{}\PLph{}%
	\ITAph{}%
}

\cite{Schneier}

\chapter{%
	\RU{Блоги}%
	\EN{Blogs}%
	\ES{Blogs}%
	\PTBRph{}%
	\DEph{}\PLph{}%
	\ITAph{}%
}

\section{Windows}

\begin{itemize}
\item
\href{http://go.yurichev.com/17025}{Microsoft: Raymond Chen}
\item
\href{http://go.yurichev.com/17026}{nynaeve.net}
\end{itemize}

\chapter{%
	\RU{Прочее}%
	\EN{Other}%
	\ES{Otros}%
	\PTBRph{}%
	\DEph{}\PLph{}%
	\ITAph{}%
}

\RU{Имеются два отличных субреддита на reddit.com посвященных \ac{RE}:}%
\EN{There are two excellent \ac{RE}-related subreddits on reddit.com:} %
\ES{Existen dos excelentes subreddits relacionados con \ac{RE} en reddit.com:}%
\PTBRph{}%
\DEph{}\PLph{}%
\ITAph{}
\href{http://go.yurichev.com/17027}{reddit.com/r/ReverseEngineering/} \AndENRU 
\href{http://go.yurichev.com/17028}{reddit.com/r/remath}
\RU{(для тем посвященных пересечению \ac{RE} и математики).}%
\EN{(on the topics for the intersection of \ac{RE} and mathematics).}%
\ES{(en los t\'opicos de la intersecci\'on de \ac{RE} y matem\'aticas).}%
\PTBRph{}%
\DEph{}\PLph{}%
\ITAph{}

\RU{Имеется также часть сайта Stack Exchange}%
\EN{There is also a \ac{RE} part of the Stack Exchange}%
\ES{Tambi\'en hay una secci\'on sobre \ac{RE} en el sitio web de Stack Exchange:}%
\PTBRph{}%
\DEph{}\PLph{}%
\ITAph{}%
\THAph{}
\RU{посвященная \ac{RE}:}%
\EN{website:}%
\\
\href{http://go.yurichev.com/17029}{reverseengineering.stackexchange.com}.

\RU{На IRC есть канал}%
\EN{On IRC there's a}%
\ES{En IRC hay un canal}%
\PTBRph{}%
\DEph{}\PLph{}%
\ITAph{}
\#\#re 
\RU{на}%
\EN{channel on}%
\ES{en}%
\PTBRph{}%
\DEph{}\PLph{}%
\ITAph{}
FreeNode\footnote{\href{http://go.yurichev.com/17030}{freenode.net}}.
