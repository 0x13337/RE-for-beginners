\part{\RU{Что стоит почитать}\EN{Books/blogs worth reading}}
% TODO названия книг а не ссылки!
\chapter{\RU{Книги}\EN{Books}}

\section{Windows}

\cite{Russinovich}.

\section{\CCpp}

\cite{CPP11}.

\section{x86 / x86-64}

\cite{Intel}, \cite{AMD}

\section{ARM}

\RU{Документация от ARM}\EN{ARM manuals}: \url{http://infocenter.arm.com/help/index.jsp?topic=/com.arm.doc.subset.architecture.reference/index.html}

\chapter{\RU{Блоги}\EN{Blogs}}

\section{Windows}

\begin{itemize}
\item
\href{http://blogs.msdn.com/oldnewthing/}{Microsoft: Raymond Chen}
\item
\url{http://www.nynaeve.net/}
\end{itemize}

\chapter{\RU{Прочее}\EN{Other}}

\RU{Имеются два отличных субреддита на reddit.com посвященных \ac{RE}}
\EN{There are two excellent \ac{RE}-related subreddits on reddit.com}: 
\href{http://www.reddit.com/r/ReverseEngineering/}{ReverseEngineering} \AndENRU 
\href{http://www.reddit.com/r/remath}{REMath}
(\RU{для тем посвященных пересечению \ac{RE} и математики}
\EN{for the topics on the intersection of \ac{RE} and mathematics}).

\RU{Имеется также часть сайта}\EN{There are also \ac{RE} part of} Stack Exchange 
\RU{посвященная \ac{RE}}\EN{website}:\\
\url{http://reverseengineering.stackexchange.com/}.

\RU{На IRC есть канал}\EN{On IRC there are} \#\#re \RU{на}\EN{channel on} FreeNode\footnote{\url{https://freenode.net/}}.
