\chapter{\IFRU{Что стоит почитать}{Books/blogs worth reading}}

\section{\IFRU{Книги}{Books}}

\subsection{Windows}

\cite{Russinovich}.

\subsection{\CCpp}

\cite{CPP11}.

\subsection{x86 / x86-64}

\cite{Intel}, \cite{AMD}

\subsection{ARM}

\IFRU{Документация от ARM}{ARM manuals}: \url{http://infocenter.arm.com/help/index.jsp?topic=/com.arm.doc.subset.architecture.reference/index.html}

\section{\IFRU{Блоги}{Blogs}}

\subsection{Windows}

\begin{itemize}
\item
\href{http://blogs.msdn.com/oldnewthing/}{Microsoft: Raymond Chen}
\item
\url{http://www.nynaeve.net/}
\end{itemize}

\section{\IFRU{Прочее}{Other}}

\IFRU{Имеются два отличных субреддита на reddit.com посвященных \ac{RE}}
{There are two excellent \ac{RE}-related subreddits on reddit.com}: 
\href{http://www.reddit.com/r/ReverseEngineering/}{ReverseEngineering} \AndENRU 
\href{http://www.reddit.com/r/remath}{REMath}
(\IFRU{для тем посвященных пересечению \ac{RE} и математики}
{for the topics on the intersection of \ac{RE} and mathematics}).

\IFRU{Имеется также часть сайта}{There are also \ac{RE} part of} Stack Exchange 
\IFRU{посвященная \ac{RE}}{website}:\\
\url{http://reverseengineering.stackexchange.com/}.
