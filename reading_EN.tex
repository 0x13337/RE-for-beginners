\part{Books/blogs worth reading}

\chapter{Books and other materials}

\section{Reverse Engineering}

\begin{itemize}
\item Eldad Eilam, \IT{Reversing: Secrets of Reverse Engineering}, (2005)

\item Bruce Dang, Alexandre Gazet, Elias Bachaalany, Sebastien Josse, \IT{Practical Reverse Engineering: x86, x64, ARM, Windows Kernel, Reversing Tools, and Obfuscation}, (2014)

\item Michael Sikorski, Andrew Honig, \IT{Practical Malware Analysis: The Hands-On Guide to Dissecting Malicious Software}, (2012)

\item Chris Eagle, \IT{IDA Pro Book}, (2011)
\end{itemize}


\section{Windows}

\begin{itemize}
\item \Russinovich
\end{itemize}

\EN{Blogs}\ES{Blogs}\RU{Блоги}:

\begin{itemize}
\item \href{http://go.yurichev.com/17025}{Microsoft: Raymond Chen}
\item \href{http://go.yurichev.com/17026}{nynaeve.net}
\end{itemize}



\section{\CCpp}

\begin{itemize}
\item \KRBook

\item C++11 standard\footnote{\AlsoAvailableAs \url{http://www.open-std.org/jtc1/sc22/wg21/docs/papers/2013/n3690.pdf}.}

\item [\AgnerFogCCP]\footnote{\AlsoAvailableAs \url{http://agner.org/optimize/optimizing_cpp.pdf}.}
\end{itemize}

\section{x86 / x86-64}

\label{x86_manuals}
\begin{itemize}
\item Intel manuals\footnote{\AlsoAvailableAs \url{http://www.intel.com/content/www/us/en/processors/architectures-software-developer-manuals.html}}

\item AMD manuals\footnote{\AlsoAvailableAs \url{http://developer.amd.com/resources/developer-guides-manuals/}}

\item \AgnerFog{}\footnote{\AlsoAvailableAs \url{http://agner.org/optimize/microarchitecture.pdf}}

\item \AgnerFogCC{}\footnote{\AlsoAvailableAs \url{http://www.agner.org/optimize/calling_conventions.pdf}}
\end{itemize}

\section{ARM}

\begin{itemize}
\item ARM manuals\footnote{\AlsoAvailableAs \url{http://infocenter.arm.com/help/index.jsp?topic=/com.arm.doc.subset.architecture.reference/index.html}}

\item \ARMSixFourRefURL
\end{itemize}

\section{Java}

\Javabook.

\section{Cryptography}

\sectionold{Cryptography}
\label{crypto_books}

\begin{itemize}
\item \Schneier{}

\item (Free) lvh, \IT{Crypto 101}\footnote{\AlsoAvailableAs \url{https://www.crypto101.io/}}

\item (Free) Dan Boneh, Victor Shoup, \IT{A Graduate Course in Applied Cryptography}\footnote{\AlsoAvailableAs \url{https://crypto.stanford.edu/~dabo/cryptobook/}}.
\end{itemize}



\chapter{Other}

There are two excellent \ac{RE}-related subreddits on reddit.com:
\href{http://go.yurichev.com/17027}{reddit.com/r/ReverseEngineering/} and
\href{http://go.yurichev.com/17028}{reddit.com/r/remath}
(on the topics for the intersection of \ac{RE} and mathematics).

There is also a \ac{RE} part of the Stack Exchange website:

\par \href{http://go.yurichev.com/17029}{reverseengineering.stackexchange.com}.

On IRC there's a \#\#re channel on
FreeNode\footnote{\href{http://go.yurichev.com/17030}{freenode.net}}.

