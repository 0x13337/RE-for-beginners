\subsection{\IFRU{Глобальные переменные}{Global variables}}

\IFRU
{А что если переменная \TT{x} из предыдущего примера будет глобальной переменной а не локальной? 
Тогда к ней смогут обращаться из любого другого места, а не только из тела функции. 
Это снова не очень хорошая практика программирования, но ради примера мы можем себе это позволить.}
{What if \TT{x} variable from previous example will not be local but global variable? 
Then it will be accessible from any place but not only from function body. 
It is not very good programming practice, but for the sake of experiment we could do this.}

\lstinputlisting{scanf/4_2_msvc.asm}

\IFRU
{Ничего особенного, в целом. Теперь \TT{x} объявлена в сегменте \TT{\_DATA}. 
Память для нее в стеке более не выделяется. Все обращения к ней происходит не через стек, а уже напрямую. 
Её значение неопределено. 
Это означает, что память под нее будет выделена, но ни компилятор, ни ОС не будет заботиться о том, 
что там будет лежать на момент старта функции \TT{\_main}.
В качестве домашнего задания, попробуйте объявить большой неопределенный массив и посмотреть 
что там будет лежать после загрузки.}
{Now \TT{x} variable is defined in \TT{\_DATA} segment. 
Memory in local stack is not allocated anymore. 
All accesses to it are not via stack but directly to process memory. 
Its value is not defined. 
This mean that memory will be allocated by operation system, but not compiler, 
neither operation system will not take care about its initial value at the moment of 
\main function start.
As experiment, try to declare large array and see what will it contain after 
program loading.}

\IFRU{Попробуем изменить объявление этой переменной:}{Now let's assign value to variable explicitly:}

\begin{lstlisting}
int x=10; // default value
\end{lstlisting}

\IFRU{Выйдет в итоге:}{We got:}

\begin{lstlisting}
_DATA	SEGMENT
_x	DD	0aH

...
\end{lstlisting}

\IFRU{Здесь уже по месту этой переменной записано \TT{0xA} с типом DD (dword = 32 бита).}
{Here we see value 0xA of DWORD type (DD meaning DWORD = 32 bit).}

\IFRU{Если вы откроете скомпилированный .exe-файл в \IDA, то увидите что \IT{x} 
находится аккурат в начале сегмента \TT{\_DATA}, после этой переменной будут текстовые строки.}
{If you will open compiled .exe in \IDA, you will see \IT{x} placed at the beginning of 
\TT{\_DATA} segment, and after you'll see text strings.}

\IFRU{А вот если вы откроете в \IDA, .exe скомплированный в прошлом примере, 
где значение \IT{x} неопределено, то в IDA вы увидите:}
{If you will open compiled .exe in \IDA from previous example where \IT{x} value is not defined, 
you'll see something like this:}

\begin{lstlisting}
.data:0040FA80 _x              dd ?                    ; DATA XREF: _main+10
.data:0040FA80                                         ; _main+22
.data:0040FA84 dword_40FA84    dd ?                    ; DATA XREF: _memset+1E
.data:0040FA84                                         ; unknown_libname_1+28
.data:0040FA88 dword_40FA88    dd ?                    ; DATA XREF: ___sbh_find_block+5
.data:0040FA88                                         ; ___sbh_free_block+2BC
.data:0040FA8C ; LPVOID lpMem
.data:0040FA8C lpMem           dd ?                    ; DATA XREF: ___sbh_find_block+B
.data:0040FA8C                                         ; ___sbh_free_block+2CA
.data:0040FA90 dword_40FA90    dd ?                    ; DATA XREF: _V6_HeapAlloc+13
.data:0040FA90                                         ; __calloc_impl+72
.data:0040FA94 dword_40FA94    dd ?                    ; DATA XREF: ___sbh_free_block+2FE
\end{lstlisting}

\IFRU{\TT{\_x} обозначен как \TT{?}, наряду с другими переменными не требующими инициализции. 
Это означает, что при загрузке .exe в память, место под все это выделено будет. 
Но в самом .exe ничего этого нет. Неинициализированные переменные не занимают места в исполняемых файлах. Удобно для больших массивов, например.}
{\TT{\_x} marked as \TT{?} among another variables not required to be initialized. 
This mean that after loading .exe to memory, place for all these variables will be 
allocated and some random garbage will be here. 
But in .exe file these not initialized variables are not occupy anything. 
It is suitable for large arrays, for example.}

\IFRU{В Linux все также почти. За исключением того что если значение \TT{x} не определено, 
то эта переменная будет находится в сегменте \TT{\_bss}. В ELF\footnote{Формат исполняемых файлов, использующийся в Linux и некоторых других *NIX} этот сегмент имеет такие аттрибуты:}
{It is almost the same in Linux, except segment names and properties: 
not initialized variables are located in \TT{\_bss} segment. 
In ELF\footnote{Executable file format widely used in *NIX system including Linux} 
file format this segment has such attributes:}

\begin{lstlisting}
; Segment type: Uninitialized
; Segment permissions: Read/Write
\end{lstlisting}

\IFRU{Ну а если сделать присвоение этой переменной значения 10, то она будет находится в сегменте \TT{\_data},
это сегмент с такими аттрибутами:}
{If to assign some value to variable, it will be placed in \TT{\_data} segment, 
this is segment with such attributes:}

\begin{lstlisting}
; Segment type: Pure data
; Segment permissions: Read/Write
\end{lstlisting}
