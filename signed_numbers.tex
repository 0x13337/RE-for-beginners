\section{\SignedNumbersSectionName}
\label{sec:signednumbers}
\index{Signed numbers}

\newcommand{\URLS}{\url{http://en.wikipedia.org/wiki/Signed_number_representations}}

\IFRU
{Методов представления чисел с знаком ``плюс'' или ``минус'' несколько\footnote{\URLS}, 
а в x86 применяется метод ``дополнительный код'' или ``two's complement''.}
{There are several methods of representing signed numbers\footnote{\URLS}, 
but in x86 architecture used ``two's complement''.}

\index{x86!\Instructions!JA}
\index{x86!\Instructions!JB}
\index{x86!\Instructions!JL}
\index{x86!\Instructions!JG}
\IFRU{Разница в подходе к знаковым/беззнаковым числам, собственно, нужна потому что, например, 
если представить \TT{0xFFFFFFFE} и \TT{0x0000002} как беззнаковое, то первое число ($4294967294$) больше второго ($2$). 
Если их оба представить как знаковые, то первое будет $-2$, которое, разумеется, меньше чем второе ($2$).
Вот почему инструкции для условных переходов~\ref{sec:Jcc} представлены в обоих версиях ~--- 
и для знаковых сравнений (например \JG, \JL) и для беззнаковых (\JA, \JB).}
{Difference between signed and unsigned numbers is that if we represent \TT{0xFFFFFFFE} and \TT{0x0000002} 
as unsigned, then first number ($4294967294$) is bigger than second ($2$). 
If to represent them both as signed, first will be $-2$, and it is lesser than second ($2$). 
That is the reason why conditional jumps~\ref{sec:Jcc} are present both for signed (e.g. \JG, \JL) 
and unsigned (\JA, \JB) operations.}

\subsection{\IFRU{Переполнение integer}{Integer overflow}}

\IFRU{Бывает так, что ошибки представления знаковых/беззнаковых могут привести к уязвимости 
\IT{переполнение integer}.}
{It is worth noting that incorrect representation of number can lead integer overflow vulnerability.}

\IFRU{Например, есть некий сервис, который принимает по сети некие пакеты. 
В пакете есть заголовок где указана длина пакета. Это 32-битное значение. 
В процессе приема пакета, 
сервис проверяет это значение и сверяет, больше ли оно чем максимальный размер пакета, скажем, константа
\TT{MAX\_PACKET\_SIZE} (например, 10 килобайт). 
Сравнение знаковое. Злоумышленник подставляет значение \TT{0xFFFFFFFF}. Это число трактуется как знаковое $-1$ 
и оно меньше чем $10000$. Проверка проходит. Продолжаем дальше и копируем этот пакет куда-нибудь себе 
в сегмент данных\dots вызов функции \TT{memcpy (dst, src, 0xFFFFFFFF)} скорее всего, 
затрет много чего внутри процесса.}
{For example, we have some network service, it receives network packets. 
In that packets there are also field where subpacket length is coded. 
It is 32-bit value. 
After network packet received, service checking that field, and if it is larger than, 
for example, some \TT{MAX\_PACKET\_SIZE} (let's say, 10 kilobytes), packet ignored as incorrect. 
Comparison is signed. Intruder set this value to the \TT{0xFFFFFFFF}.
While comparison, this number is considered as signed $-1$ and it's lesser than 10 kilobytes. 
No error here. 
Service would like to copy that subpacket to another place in memory and call 
\TT{memcpy (dst, src, 0xFFFFFFFF)} function: this operation, rapidly garbling a lot of 
inside of process memory.}

% TODO: move to .bib
\IFRU{Немного подробнее}{More about it}: \url{http://www.phrack.org/issues.html?issue=60&id=10}

