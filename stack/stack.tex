% done

\section{\IFRU{Стек}{Stack}}
\label{sec:stack}

\IFRU{Стек в компьютерных науках ~--- это одна из наиболее фундаментальных вещей}
{Stack ~--- is one of the most fundamental things in computer science.}\footnote{\url{http://en.wikipedia.org/wiki/Call_stack}}.

\IFRU{Технически, это просто кусок памяти процесса + регистр \ESP который указывает где-то в пределах этого куска.}{Technically, this is just piece of process memory + \ESP register as a pointer within piece of memory.}

\IFRU
{Часто используемые инструкции для работы со стеком это \PUSH и \POP. 
\PUSH уменьшает \ESP на 4, затем записывает по адресу на который указывает \ESP содержимое своего единственного операнда.}
{Most frequently used stack access instructions are \PUSH and \POP. 
\PUSH subtracting \ESP by 4 and then writing contents of its sole operand to the memory address pointing by \ESP.} 

\IFRU{\POP это обратная операция ~--- сначала достает из \ESP значение и кладет его в операнд 
(который очень часто является регистром) и затем увеличивает \ESP на 4. 
Конечно, это для 32-битной среды. В x64-среде это будет 8 а не 4.}
{\POP is reverse operation: get a data from memory pointing by \ESP and then add 4 to \ESP. Of course, 
this is for 32-bit environment. 8 will be here instead of 4 in x64 environment.}

\IFRU{В самом начале, \ESP указывает на конец стека.}{After stack allocation, \ESP pointing to the end of stack.}
\IFRU{\PUSH уменьшает \ESP, а \POP ~--- увеличивает.}{\PUSH increasing \ESP, and \POP decreasing.}
\IFRU{Конец стека находится в начале блока памяти. Это странно, но это так.}
{The end of stack is actually at the beginning of allocated memory block. It seems strange, but it is so.}

\IFRU{Для чего используется стек?}{What stack is used for?}

\subsection{\IFRU{Сохранение адреса куда должно вернуться управление после вызова функции}
{Save return address where function should return control after execution}}

\IFRU
{При вызове другой функции через \CALL, сначала в стек записывается адрес указывающий на место аккурат после 
инструкции \CALL, затем делается безусловный переход (\TT{JMP}) на адрес указанный в операнде.} 
{While calling another function by \CALL instruction, the address of point exactly after \CALL is saved 
to stack, and then unconditional jump to the address from CALL operand is executed.} 

\IFRU{\CALL это аналог пары инструкций \TT{PUSH address\_after\_call / JMP}.}
{\CALL is \TT{PUSH address\_after\_call / JMP operand} instructions pair equivalent}.

\IFRU{\RET вытаскивает из стека значение и передает управление по этому адресу ~--- 
это аналог пары инструкций \TT{POP tmp / JMP tmp}.}
{\RET is fetching value from stack and jump to it ~--- it is \TT{POP tmp / JMP tmp} instructions pair equivalent.}

\IFRU{Крайне легко устроить переполнение стека запустив бесконечную рекурсию:}
{Stack overflow is simple, just run eternal recursion:}

\begin{lstlisting}
void f()
{
	f();
};
\end{lstlisting}

\IFRU{MSVC 2008 предупреждает о проблеме:}{MSVC 2008 reporting about problem:}

\begin{lstlisting}
c:\tmp6>cl ss.cpp /Fass.asm
Microsoft (R) 32-bit C/C++ Optimizing Compiler Version 15.00.21022.08 for 80x86
Copyright (C) Microsoft Corporation.  All rights reserved.

ss.cpp
c:\tmp6\ss.cpp(4) : warning C4717: 'f' : recursive on all control paths, function will cause runtime stack overflow
\end{lstlisting}

\IFRU{... но тем не менее создает нужный код:}{... but generates right code anyway:}

\begin{lstlisting}
?f@@YAXXZ PROC						; f
; File c:\tmp6\ss.cpp
; Line 2
	push	ebp
	mov	ebp, esp
; Line 3
	call	?f@@YAXXZ				; f
; Line 4
	pop	ebp
	ret	0
?f@@YAXXZ ENDP						; f
\end{lstlisting}

\IFRU
{... причем, если включить оптимизацию (\Ox), то будет даже интереснее, без переполнения стека, 
но работать будет \IT{корректно}\footnote{здесь ирония}:}
{... Also, if we turn on optimization (\Ox option), the optimized code will not overflow stack, 
but will work \IT{correctly}\footnote{irony here}:}

\begin{lstlisting}
?f@@YAXXZ PROC						; f
; File c:\tmp6\ss.cpp
; Line 2
$LL3@f:
; Line 3
	jmp	SHORT $LL3@f
?f@@YAXXZ ENDP						; f
\end{lstlisting}

\IFRU{GCC 4.4.1 генерирует точно такой же код в обоих случаях, хотя и не предупреждает о проблеме.}
{GCC 4.4.1 generating the same code in both cases, although not warning about problem.}

\subsection{\IFRU{Передача параметров для функции}{Function arguments passing}}

\begin{lstlisting}
push arg3
push arg2
push arg1
call f
add esp, 4*3
\end{lstlisting}

\IFRU{Вызываемая функция получает свои параметры также через указатель \ESP.}
{Callee{\footnote{Function being called}} function get its arguments via \ESP ponter.}

\IFRU{См.также в соответствующем разделе о способах передачи аргументов через стек}
{See also section about calling conventions}~\ref{sec:callingconventions}.

\IFRU{Важно отметить, что, в общем, никто не заставляет программистов передавать параметры именно через стек,
это не является требованием к исполняемому коду.}
{It is important to note that no one oblige programmers to pass arguments through stack, it is not prerequisite.}

\IFRU{Вы можете делать это совершенно иначе, не используя стек.}
{One could implement any other method not using stack.}

\IFRU{К примеру, можно выделять в куче\footnote{heap в англоязычной литературе} место для аргументов, 
заполнять их и передавать в функцию указатель на это место через \EAX. И это вполне будет работать.}
{For example, it is possible to allocate a place for arguments in heap, fill it and pass to a function 
via pointer to this pack in \EAX register. And this will work} 

\IFRU{Однако, так традиционно сложилось, что передача аргументов происходит именно через стек.}
{However, it is convenient tradition to use stack for this.}

\subsection{\IFRU{Хранение локальных переменных}{Local variable storage}}

\IFRU{Функция может выделить для себя некоторое место в стеке для локальных переменных просто отодвинув 
\ESP глубже к концу стека.}
{A function could allocate some space in stack for its local variables just shifting 
\ESP pointer deeply enough to stack bottom.}

\IFRU{Это снова не является необходимым требованием. Вы можете хранить локальные переменные где угодно. 
Но по традиции всё сложилось так.}
{It is also not prerequisite. You could store local variables wherever you like. 
But traditionally it is so.}

\subsubsection{\IFRU{Функция alloca()}{alloca() function}}

\IFRU{Интересен случай с функцией \TT{alloca()}}
{It is worth noting \TT{alloca()} function.}\footnote{
\IFRU
{Реализацию функции можно посмотреть в файлах}
{Function implementation can be found in} 
  \TT{alloca16.asm} 
  \IFRU{и}{and} 
  \TT{chkstk.asm} 
  \IFRU{в}{in} 
  \TT{C:\textbackslash{}Program Files (x86)\textbackslash{}Microsoft Visual Studio 10.0\textbackslash{}VC\textbackslash{}crt\textbackslash{}src\textbackslash{}intel}}. 

\IFRU{Эта функция работает как \TT{malloc()}, но выделяет память прямо в стеке.} 
{This function works like \TT{malloc()}, but allocate memory just in stack.}

\IFRU{Память освобождать через \TT{free()} не нужно, так как эпилог функции~\ref{sec:prologepilog} 
вернет \ESP назад в изначальное состояние и выделенная память просто анулируется.}
{Allocated memory chunk is not needed to be freed via \TT{free()} function call because 
function epilogue~\ref{sec:prologepilog} will return \ESP back to initial state and 
allocated memory will be just annuled.} 

\IFRU{Интересна реализация функции \TT{alloca()}.}
{It is worth noting how \TT{alloca()} implemented.}

\IFRU{Эта функция, если упрощенно, просто сдвигает \ESP вглубь стека 
на столько байт сколько вам нужно и возвращает \ESP в качестве указателя на выделенный блок.}
{This function, if to simplify, just shifting \ESP deeply to stack bottom so much bytes you 
need and set \ESP as a pointer to that \IT{allocated} block.}
\IFRU{Попробуем:}{Let's try:}

\lstinputlisting{stack/2_1.c}

\IFRU{Компилируем}{Let's compile} (MSVC 2010):

\lstinputlisting{stack/2_2_msvc.asm}

\IFRU
{Единственный параметр в \TT{alloca()} передается через \EAX, а не как обычно через стек.}
{The sole \TT{alloca()} argument passed via \EAX (but not via stack).}

\IFRU{После вызова \TT{alloca()}, \ESP теперь указывает на блок в 600 байт который 
мы можем использовать под \TT{buf}.}
{After \TT{alloca()} call, \ESP is not pointing to the block of 600 bytes and we can 
use it as memory for \TT{buf} array.}

\IFRU{А GCC 4.4.1 обходится без вызова других функций:}
{GCC 4.4.1 can do the same without calling external functions:}

\lstinputlisting{\IFRU{stack/2_2_gcc_ru.asm}{stack/2_2_gcc_en.asm}}

\subsection{(Windows) SEH}

\IFRU{В стеке хранятся записи SEH (\IT{Structured Exception Handling}) для функции (если имеются)}
{SEH (\IT{Structured Exception Handling}) records are also stored in stack (if needed).}
\footnote{
\IFRU{О SEH: классическая статья Мэтта Питрека}{Classic Matt Pietrek article about SEH}: 
\url{http://www.microsoft.com/msj/0197/Exception/Exception.aspx}}.

\subsection{\IFRU{Защита от переполнений буфера}{Buffer overflow protection}}

More about it here~\ref{subsec:bufferoverflow}.

