\subsubsection{x86}

\IFRU{Это дает в итоге}{Result} (MSVC 2010):

\lstinputlisting{switch/8_2_msvc.asm}

\IFRU{В принципе, ничего особо нового для нас здесь, за исключением того, что компилятор зачем-то 
перекладывает входящую переменную \TT{a} во временную в локальном стеке \TT{v64}.}
{Nothing specially new to us, with the exception that compiler moving input variable 
\TT{a} to temporary local variable \TT{tv64}.}

\IFRU{Если скомпилировать это при помощи GCC 4.4.1, то будет почти то же самое, даже с максимальной оптимизацией 
(ключ \Othree).}
{If to compile the same in GCC 4.4.1, we'll get alsmost the same, even with maximal optimization 
turned on (\Othree option).}

\IFRU{Попробуем, включить оптимизацию кодегенератора}
{Now let's turn on optimization in} MSVC (\Ox): \TT{cl 1.c /Fa1.asm /Ox}

\lstinputlisting{switch/8_3_msvc.asm}

\IFRU{Вот здесь уже все немного по-другому, причем не без грязных хаков.}
{Here we can see even dirty hacks.}

\IFRU
{Первое: \TT{а} помещается в \EAX и от него отнимается 0. Звучит абсурдно, но нужно это для того, чтобы проверить, 
0 ли в \EAX был до этого? Если да, то выставится флаг \ZF (что означает что результат отнимания нуля от числа 
стал нулем) и первый условный переход \JE (\IT{Jump if Equal} или его синоним \JZ ~--- \IT{Jump if Zero}) 
сработает на метку \TT{\$LN4@f}, где выводится сообщение \TT{'zero'}.
Если первый переход не сработал, от значения отнимается по единице, 
и если на какой-то стадии образуется в результате 0, то сработает соответствующий переход.}
{First: \TT{a} is placed into \EAX and 0 subtracted from it. Sounds absurdly, but it may need to check if 
0 was in \EAX before? If yes, flag \ZF will be set (this also mean that subtracting from zero is zero) 
and first conditional jump \JE (\IT{Jump if Equal} or synonym \JZ ~--- \IT{Jump if Zero}) will be triggered 
and control flow passed to \TT{\$LN4@f} label, where \TT{'zero'} message is begin printed. 
If first jump was not triggered, 1 subtracted from input value and if at some stage 0 will be resulted, 
corresponding jump will be triggered.}

\IFRU{И в конце концов, если ни один из условных переходов не сработал, управление передается \printf
с агрументом \TT{'something unknown'}.}
{And if no jump triggered at all, control flow passed to \printf with argument \TT{'something unknown'}.}

\label{jump_to_last_printf}
\IFRU
{Второе: мы видим две, мягко говоря, необычные вещи: указатель на сообщение помещается в переменную \TT{a}, 
и затем \printf вызывается не через \CALL, а через \JMP. Объяснение этому простое. 
Вызывающая функция заталкивает в стек некоторое значение и через \CALL вызывает нашу функцию. 
\CALL в свою очередь затакливает в стек адрес возврата и делает безусловный переход на адрес нашей функции. 
Наша функция в самом начале (да и в любом её месте, потому что в теле функции нет ни одной инструкции, 
которая меняет что-то в стеке или в \ESP) имеет следующую разметку стека:}
{Second: we see unusual thing for us: string pointer is placed into \TT{a} variable, and 
then \printf is called not via \CALL, but via \JMP. This could be explained simply. 
Caller pushing to stack some value and via \CALL calling our function. 
\CALL itself pushing returning address to stack and do unconditional jump to our function address. 
Our function at any place of its execution (since it do not contain any instruction moving stack 
pointer) has the following stack layout:}

\begin{itemize}
\item\ESP ~--- \IFRU{хранится адрес возврата}{pointing to return address} 
\item\TT{ESP+4} ~--- \IFRU{хранится значение \TT{a}}{pointing to \TT{a} variable} 
\end{itemize}

\IFRU{С другой стороны, чтобы вызвать \printf нам нужна почти такая же разметка стека, 
только в первом аргументе нужен указатель на строку. Что, собственно, этот код и делает.}
{On the other side, when we need to call \printf here, we need exactly the same stack 
layout, except of first \printf argument pointing to string. 
And that is what our code does.}

\IFRU{Он заменяет свой первый аргумент на другой и затем передает управление \printf, как если бы вызвали не 
нашу функцию \TT{f()}, а сразу \printf. 
\printf выводит некую строку на \TT{stdout}, затем исполняет инструкцию \RET, 
которая из стека достает адрес возврата и управление передается в ту функцию, 
которая вызывала \TT{f()}, минуя при этом саму \TT{f()}.}
{It replaces function's first argument to different and 
jumping to \printf, as if not our function \TT{f()} was called firstly, but immediately \printf.
\printf printing some string to \TT{stdout} and then execute \RET instruction, which POPping 
return address from stack and control flow is returned not to \TT{f()}, but to \TT{f()}'s callee, 
escaping \TT{f()}.}

\newcommand{\URLSJ}{\url{http://en.wikipedia.org/wiki/Setjmp.h}}
\IFRU{Все это возможно потому что \printf вызывается в \TT{f()} в самом конце. 
Все это чем-то даже похоже на \TT{longjmp()}\footnote{\URLSJ}.
И все это, разумеется, сделано для экономии времени исполнения.}
{All it's possible because \printf is called right at the end of \TT{f()} in any case. 
In some way, it's all similar to \TT{longjmp()}\footnote{\URLSJ}. 
And of course, it's all done for the sake of speed.}

\IFRU{Похожая ситуация с компилятором для ARM описана в секции}
{Similar case with ARM compiler described in} ``\PrintfSeveralArgumentsSectionName'', 
\IFRU{здесь}{section, here}~\ref{ARM_B_to_printf}.
