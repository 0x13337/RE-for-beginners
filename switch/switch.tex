\section{\SwitchCaseDefaultSectionName}

\subsection{\IFRU{Если вариантов мало}{Few number of cases}}

\lstinputlisting{switch/8_1.c}

\IFRU{Это дает в итоге}{Result} (MSVC 2010):

\lstinputlisting{switch/8_2_msvc.asm}

\IFRU{В принципе, ничего особо нового для нас здесь, за исключением того, что компилятор зачем-то 
перекладывает входящую переменную \TT{a} во временную в локальном стеке \TT{v64}.}
{Nothing specially new to us, with the exception that compiler moving input variable 
\TT{a} to temporary local variable \TT{tv64}.}

\IFRU{Если скомпилировать это при помощи GCC 4.4.1, то будет почти то же самое, даже с максимальной оптимизацией 
(ключ \Othree).}
{If to compile the same in GCC 4.4.1, we'll get alsmost the same, even with maximal optimization 
turned on (\Othree option).}

\IFRU{Попробуем, включить оптимизацию кодегенератора}
{Now let's turn on optimization in} MSVC (\Ox): \TT{cl 1.c /Fa1.asm /Ox}

\lstinputlisting{switch/8_3_msvc.asm}

\IFRU{Вот здесь уже все немного по-другому, причем не без грязных хаков.}
{Here we can see even dirty hacks.}

\IFRU
{Первое: \TT{а} помещается в \EAX и от него отнимается 0. Звучит абсурдно, но нужно это для того, чтобы проверить, 
0 ли в \EAX был до этого? Если да, то выставится флаг \ZF (что означает что результат отнимания нуля от числа 
стал нулем) и первый условный переход \JE (\IT{Jump if Equal} или его синоним \JZ ~--- \IT{Jump if Zero}) 
сработает на метку \TT{\$LN4@f}, где выводится сообщение \TT{'zero'}.
Если первый переход не сработал, от значения отнимается по единице, 
и если на какой-то стадии образуется в результате 0, то сработает соответствующий переход.}
{First: \TT{a} is placed into \EAX and 0 subtracted from it. Sounds absurdly, but it may need to check if 
0 was in \EAX before? If yes, flag \ZF will be set (this also mean that subtracting from zero is zero) 
and first conditional jump \JE (\IT{Jump if Equal} or synonym \JZ ~--- \IT{Jump if Zero}) will be triggered 
and control flow passed to \TT{\$LN4@f} label, where \TT{'zero'} message is begin printed. 
If first jump was not triggered, 1 subtracted from input value and if at some stage 0 will be resulted, 
corresponding jump will be triggered.}

\IFRU{И в конце концов, если ни один из условных переходов не сработал, управление передается \printf
с агрументом \TT{'something unknown'}.}
{And if no jump triggered at all, control flow passed to \printf with argument \TT{'something unknown'}.}

\label{jump_to_last_printf}
\IFRU
{Второе: мы видим две, мягко говоря, необычные вещи: указатель на сообщение помещается в переменную \TT{a}, 
и затем \printf вызывается не через \CALL, а через \JMP. Объяснение этому простое. 
Вызывающая функция заталкивает в стек некоторое значение и через \CALL вызывает нашу функцию. 
\CALL в свою очередь затакливает в стек адрес возврата и делает безусловный переход на адрес нашей функции. 
Наша функция в самом начале (да и в любом её месте, потому что в теле функции нет ни одной инструкции, 
которая меняет что-то в стеке или в \ESP) имеет следующую разметку стека:}
{Second: we see unusual thing for us: string pointer is placed into \TT{a} variable, and 
then \printf is called not via \CALL, but via \JMP. This could be explained simply. 
Caller pushing to stack some value and via \CALL calling our function. 
\CALL itself pushing returning address to stack and do unconditional jump to our function address. 
Our function at any place of its execution (since it do not contain any instruction moving stack 
pointer) has the following stack layout:}

\begin{itemize}
\item\ESP ~--- \IFRU{хранится адрес возврата}{pointing to return address} 
\item\TT{ESP+4} ~--- \IFRU{хранится значение \TT{a}}{pointing to \TT{a} variable} 
\end{itemize}

\IFRU{С другой стороны, чтобы вызвать \printf нам нужна почти такая же разметка стека, 
только в первом аргументе нужен указатель на строку. Что, собственно, этот код и делает.}
{On the other side, when we need to call \printf here, we need exactly the same stack 
layout, except of first \printf argument pointing to string. 
And that is what our code does.}

\IFRU{Он заменяет свой первый аргумент на другой и затем передает управление \printf, как если бы вызвали не 
нашу функцию \TT{f()}, а сразу \printf. 
\printf выводит некую строку на \TT{stdout}, затем исполняет инструкцию \RET, 
которая из стека достает адрес возврата и управление передается в ту функцию, 
которая вызывала \TT{f()}, минуя при этом саму \TT{f()}.}
{It replaces function's first argument to different and 
jumping to \printf, as if not our function \TT{f()} was called firstly, but immediately \printf.
\printf printing some string to \TT{stdout} and then execute \RET instruction, which POPping 
return address from stack and control flow is returned not to \TT{f()}, but to \TT{f()}'s callee, 
escaping \TT{f()}.}

\newcommand{\URLSJ}{\url{http://en.wikipedia.org/wiki/Setjmp.h}}
\IFRU{Все это возможно потому что \printf вызывается в \TT{f()} в самом конце. 
Все это чем-то даже похоже на \TT{longjmp()}\footnote{\URLSJ}.
И все это, разумеется, сделано для экономии времени исполнения.}
{All it's possible because \printf is called right at the end of \TT{f()} in any case. 
In some way, it's all similar to \TT{longjmp()}\footnote{\URLSJ}. 
And of course, it's all done for the sake of speed.}

\IFRU{Похожая ситуация с компилятором для ARM описана в секции}
{Similar case with ARM compiler described in} ``\PrintfSeveralArgumentsSectionName'', 
\IFRU{здесь}{section, here}~\ref{ARM_B_to_printf}.

\subsection{\IFRU{И если много}{A lot of cases}}

\IFRU{А если вветвлений слишком много, то конечно генерировать слишком длинный код с многочисленными \JE/\JNE 
уже не так удобно.}
{If \TT{switch()} statement contain a lot of case's, it is not very handy for compiler to emit too large code
with a lot \JE/\JNE instructions.}

\lstinputlisting{switch/8_4.c}

\IFRU{Имеем в итоге}{We got} (MSVC 2010):

\lstinputlisting{switch/8_5_msvc.asm}

\IFRU{Здесь происходит следующее: в теле функции есть набор вызовов \printf с разными аргументами. 
Все они имеют, конечно же, адреса, а также внутренние символические метки, которые придумал им компилятор.
Все эти метки складываются во внутреннюю таблицу \TT{\$LN11@f}.}
{OK, what we see here is: there are a set of \printf calls with various arguments. 
All them has not only addresses in process memory, but also internal symbolic labels generated 
by compiler. 
All these labels are also places into \TT{\$LN11@f} internal table.}

\IFRU{В начале функции, если \TT{a} больше 4, то сразу происходит переход на метку \TT{\$LN1@f}, 
где вызывается \printf с аргументом \TT{'something unknown'}.}
{At the function beginning, if \TT{a} is greater than 4, control flow is passed to label 
\TT{\$LN1@f}, where \printf with argument \TT{'something unknown'} is called.}

\IFRU{А если \TT{a} меньше или равно 4, то это значение умножается на 4 и прибавляется адрес таблицы с переходами. 
Таким образом, получается адрес внутри таблицы, где лежит нужный адрес внутри тела функции. 
Например, возьмем \TT{a} равным 2. $2*4 = 8$ (ведь все элементы таблицы это адреса внутри 32-битного процесса, 
таким образом, каждый элемент занимает 4 байта). 8 прибавить к \TT{\$LN11@f} ~--- это будет элемент таблицы,
где лежит \TT{\$LN4@f}. \JMP вытаскивает из таблицы адрес \TT{\$LN4@f} и делает безусловный переход туда.}
{And if \TT{a} value is less or equals to 4, let's multiply it by 4 and add \TT{\$LN1@f} 
table address. That is how address inside of table is constructed, pointing exactly to the 
element we need. For example, let's say \TT{a} is equal to 2. $2*4 = 8$ (all table elements 
are addresses within 32-bit process, that is why all elements contain 4 bytes). 
Address of \TT{\$LN11@f} table + 8 = it will be table element where \TT{\$LN4@f} label is stored. 
\JMP fetch \TT{\$LN4@f} address from the table and jump to it.}

\IFRU{А там вызывается \printf с агрументом \TT{'two'}. 
Дословно, инструкция \TT{jmp DWORD PTR \$LN11@f[ecx*4]} 
означает \IT{перейти по DWORD, который лежит по адресу} \TT{\$LN11@f + ecx * 4}.}
{Then corresponding \printf is called with argument \TT{'two'}. 
Literally, \TT{jmp DWORD PTR \$LN11@f[ecx*4]} instruction mean 
\IT{jump to DWORD, which is stored at address} \TT{\$LN11@f + ecx * 4}.}

\TT{npad}~\ref{sec:npad} 
\IFRU{это команда ассемблеру немного выровнять начало таблицы, дабы она располагалась по 
адресу кратному 4 (или 16). Это нужно для того чтобы процессор мог эффективнее загружать 32-битное 
значения из памяти, через шину с памятью, кеш-память, итд.}
{is assembler macro, aligning next label so that it will be stored at address aligned by 4 bytes (or 16). 
This is very suitable for processor, because it can fetch 32-bit values from memory through memory bus, 
cache memory, etc, in much effective way if it is aligned.}

\IFRU{Посмотрим что сгенерирует GCC 4.4.1}{Let's see what GCC 4.4.1 generate}:

\lstinputlisting{switch/8_6_gcc.asm}

\IFRU{Практически то же самое, за исключением мелкого нюанса: аргумент из \TT{arg\_0} умножается на 4 
при помощи сдвига влево на 2 бита (это почти то же самое что и умножение на 4)~\ref{SHR}. 
Затем адрес метки внутри функции берется из массива \TT{off\_804855C} и адресуется при помощи 
вычисленного индекса.}
{It is almost the same, except little nuance: argument \TT{arg\_0} is multiplied by 4 with by
shifting it to left by 2 bits (it is almost the same as multiplication by 4)~\ref{SHR}. 
Then label address is taken from \TT{off\_804855C} array, address calculated and stored into 
\EAX, then \TT{``JMP EAX''} do actual jump.}

