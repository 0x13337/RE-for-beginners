\chapter{\IFRU{Задачи}{Tasks}}

\IFRU{Почти для всех задач, если не указано иное, два вопроса:}
{There are two questions almost for every task, if otherwise is not specified:}

1) \IFRU{Что делает эта функция? Ответ должен состоять из одной фразы.}
{What this function does? Answer in one-sentence form.}

2) \IFRU{Перепишите эту функцию на \CCpp}{Rewrite this function into \CCpp}.

\IFRU{Подсказки и ответы собраны в приложении к этой книге.}{Hints and solutions are in the appendix of
this book.}

\section{\IFRU{Легкий уровень}{Easy level}}

\subsection{\Task 1.1}

\IFRU{Это стандартная функция из библиотек Си. Исходник взят из OpenWatcom. Скомпилировано в MSVC 2010.}
{This is standard C library function. Source code taken from OpenWatcom. Compiled in MSVC 2010.}

\lstinputlisting{tasks/tasks_1_1_msvc.asm}

\IFRU{Это он же скомпилирован при помощи GCC 4.4.1 с опцией \Othree (максимальная оптимизация)}
{It is the same code compiled by GCC 4.4.1 with \Othree option (maximum optimization)}:

\lstinputlisting{tasks/tasks_1_1_gcc.asm}

\subsection{\Task 1.2}

\IFRU{Это также стандартная функция из библиотек Си. Исходник взят из OpenWatcom и немного переделан. 
Скомпилировано в MSVC 2010 с флагом (\Ox).}
{This is also standard C library function. Source code is taken from OpenWatcom and modified slightly.
Compiled in MSVC 2010 with \Ox optimization flag.}

\IFRU{Эта функция использует стандартные функции Си:}
{This function also use these standard C functions:} isspace() \AndENRU isdigit().

\lstinputlisting{tasks/tasks_1_2_msvc.asm}

\IFRU{То же скомпилировано в GCC 4.4.1. Задача немного усложняется тем, что GCC представил isspace() и isdigit() 
как inline-функции и вставил их тела прямо в код.}
{Same code compiled in GCC 4.4.1. This task is sligthly harder since GCC compiled isspace() and isdigit()
functions as inline-functions and inserted their bodies right into the code.}

\lstinputlisting{tasks/tasks_1_2_gcc.asm}

\subsection{\Task 1.3}

\IFRU{Это также стандартная функция из библиотек Си, а вернее, две функции, работающие в паре. 
Исходник взят из MSVC 2010 и немного переделан.}
{This is standard C function too, actually, two functions working in pair.
Source code taken from MSVC 2010 and modified sligthly.}

\IFRU{Суть переделки в том, что эта функция может корректно работать в мульти-тредовой среде, 
а я, для упрощения (или запутывания) убрал поддержку этого.}
{The matter of modification is that this function can work properly in multi-threaded environment,
and I removed its support for simplification (or for confusion).}

\IFRU{Скомпилировано в MSVC 2010 с флагом (\Ox)}{Compiled in MSVC 2010 with \Ox flag}.

\lstinputlisting{tasks/tasks_1_3_msvc.asm}

\IFRU{То же скомпилировано при помощи GCC 4.4.1}{Same code compiled in GCC 4.4.1}:

\lstinputlisting{tasks/tasks_1_3_gcc.asm}

\subsection{\Task 1.4}

\IFRU{Это стандартная функция из библиотек Си. Исходник взят из MSVC 2010. Скомпилировано в MSVC 2010 с флагом \Ox.}
{This is standard C library function. Source code taken from MSVC 2010. Compiled in MSVC 2010 with \Ox flag.}

\lstinputlisting{tasks/tasks_1_4_msvc.asm}

\IFRU{То же скомпилировано при помощи GCC 4.4.1}
{Same code compiled in GCC 4.4.1}:

\lstinputlisting{tasks/tasks_1_4_gcc.asm}

\subsection{\Task 1.5}

\IFRU{Задача, скорее, на эрудицию, нежели на чтение кода.}
{This task is rather on knowledge than on reading code.}

\IFRU{Функция взята из OpenWatcom. Скомпилировано в MSVC 2010 с флагом \Ox.}
{The function is taken from OpenWatcom. Compiled in MSVC 2010 with \Ox flag.}

\lstinputlisting{tasks/tasks_1_5_msvc.asm}

\subsection{\Task 1.6}

\IFRU{Скомпилировано в MSVC 2010 с ключом \Ox.}
{Compiled in MSVC 2010 with \Ox option.}

\lstinputlisting{tasks/tasks_1_6_msvc.asm}

\subsection{\Task 1.7}

\IFRU{Это взята функция из ядра Linux 2.6.}{This function is taken from Linux 2.6 kernel.}

\IFRU{Скомпилировано в MSVC 2010 с опцией \Ox:}{Compiled in MSVC 2010 with \Ox option:}

\lstinputlisting{tasks/tasks_1_7_msvc.asm}

\subsection{\Task 1.8}

\IFRU{Скомпилировано в MSVC 2010 с опцией \TT{/O1}\footnote{/O1: оптимизация по размеру кода}:}
{Compiled in MSVC 2010 with \TT{/O1} option\footnote{/O1: minimize space}:}

\lstinputlisting{tasks/tasks_1_8_msvc.asm}

\subsection{\Task 1.9}

\IFRU{Скомпилировано в MSVC 2010 с опцией \TT{/O1}:}
{Compiled in MSVC 2010 with \TT{/O1} option:}

\lstinputlisting{tasks/tasks_1_9_msvc.asm}

\subsection{\Task 1.10}

\IFRU{Если это скомпилировать и запустить, появится некоторое число. Откуда оно берется? 
Откуда оно берется если скомпилировать в MSVC с оптимизациями (\Ox)?}
{If to compile this piece of code and run, some number will be printed. Where it came from?
Where it came from if to compile it in MSVC with optimization (\Ox)?}

\begin{lstlisting}
#include <stdio.h>

int main()
{
	printf ("%d\n");

	return 0;
};
\end{lstlisting}

\section{\IFRU{Средний уровень}{Middle level}}

\subsection{\Task 2.1}

\IFRU{Довольно известный алгоритм, также включен в стандартную библиотеку Си. Исходник взят из glibc 2.11.1. 
Скомпилирован в GCC 4.4.1 с ключом \TT{-Os} (оптимизация по размеру кода). 
Листинг сделан дизассемблером IDA 4.9 из ELF-файла созданным GCC и линкером.}
{Well-known algorithm, also included in standard C library. Source code was taken from glibc 2.11.1.
Compiled in GCC 4.4.1 with \TT{-Os} option (code size optimization).
Listing was done by IDA 4.9 disassembler from ELF-file generated by GCC and linker.}

\IFRU{Для тех кто хочет использовать IDA в процессе изучения, вот здесь лежат .elf и .idb файлы, 
.idb можно открыть при помощи бесплатой IDA 4.9:}
{For those who wants use IDA while learning, here you may find .elf and .idb files,
.idb can be opened with freeware IDA 4.9:}

\url{http://yurichev.com/RE-tasks/middle/1/}

\lstinputlisting{tasks/tasks_2_1_gcc.asm}

\subsection{\Task 2.2}

Имеется небольшой исполняемый файл, внутри которого находится довольно известная криптосистема.
Попробуйте её идентифицировать.

\begin{itemize}
\item
\href{http://yurichev.com/RE-tasks/middle/2/unknown_cryptosystem.exe}{Windows x86}

\item
\href{http://yurichev.com/RE-tasks/middle/2/unknown_encryption_linux86.tar}{Linux x86}

\item
\href{http://yurichev.com/RE-tasks/middle/2/unknown_encryption_MacOSX.tar}{MacOSX (x64)}
\end{itemize}

\subsection{\Task 2.3}

Имеется небольшой исполняемый файл, некая утилита.
Она открывает другой файл, читает его, что-то вычисляет и показывает число с плавающей точкой.
Попробуйте разобраться, что она делает.

\begin{itemize}
\item
\href{http://yurichev.com/RE-tasks/middle/3/unknown_utility_2_3.exe}{Windows x86}

\item
\href{http://yurichev.com/RE-tasks/middle/3/unknown_utility_2_3_Linux86.tar}{Linux x86}

\item
\href{http://yurichev.com/RE-tasks/middle/3/unknown_utility_2_3_MacOSX.tar}{MacOSX (x64)}
\end{itemize}

\section{crackme / keygenme}

\IFRU{Несколько моих keygenme\footnote{программа имитирующая защиту вымышленной программы, 
для которой нужно сделать генератор ключей/лицензий.}:}
{Couple of my keygenmes\footnote{program which imitates fictional software protection, 
for which one needs to make a keys/licenses generator}:}

\url{http://crackmes.de/users/yonkie/}

