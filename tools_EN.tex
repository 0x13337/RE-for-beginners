\part{Tools}

\epigraph{Now that Dennis Yurichev has made this book free (libre), it is a
contribution to the world of free knowledge and free education.
However, for our freedom's sake, we need free (libre) reverse
engineering tools to replace the proprietary tools described in this book.}{Richard M. Stallman}

\chapter{Binary analysis}

Tools you use when you don't run any process.

\myindex{Hiew}
\myindex{UNIX!strings}
\myindex{UNIX!xxd}
\myindex{UNIX!od}

\begin{itemize}
\item
(Free, open-source) \IT{ent}\footnote{\url{http://www.fourmilab.ch/random/}}: entropy analyzing tool.
Read more about entropy: \url{http://yurichev.com/blog/entropy/}.

\item
\label{Hiew}
\IT{Hiew}\footnote{\href{http://go.yurichev.com/17035}{hiew.ru}}:
for small modifications of code in binary files.

\item (Free, open-source) \IT{xxd} and \IT{od}: standard UNIX utilities for dumping.

\item (Free, open-source) \IT{strings}: *NIX tool for searching for ASCII strings in binary files, including executable ones.
Sysinternals has alternative\footnote{\url{https://technet.microsoft.com/en-us/sysinternals/strings}}
supporting wide char strings (UTF-16, widely used in Windows).

\item (Free, open-source) \IT{Binwalk}\footnote{\url{http://binwalk.org/}}: analyzing firmware images.

\item
\myindex{binary grep}
(Free, open-source) \IT{binary grep}:
a small utility for searching any byte sequence in a big pile of files, 
including non-executable ones: \BGREPURL.
\end{itemize}

\section{Disassemblers}

\myindex{IDA}
\myindex{Binary Ninja}
\myindex{BinNavi}
\myindex{objdump}

\begin{itemize}
\item \IT{IDA}. An older freeware version is available for download
\footnote{\href{http://go.yurichev.com/17031}{hex-rays.com/products/ida/support/download\_freeware.shtml}}.
\ShortHotKeyCheatsheet: \myref{sec:IDA_cheatsheet}

\item \IT{Binary Ninja}\footnote{\url{http://binary.ninja/}}

\item (Free, open-source) \IT{zynamics BinNavi}\footnote{\url{https://www.zynamics.com/binnavi.html}}

\item (Free, open-source) \IT{objdump}: simple command-line utility for dumping and diassembling.

\item (Free, open-ssource) \IT{readelf}\footnote{\url{https://sourceware.org/binutils/docs/binutils/readelf.html}}:
dump information about ELF file.
\end{itemize}

\section{Decompilers}

There is only one known, publicly available, high-quality decompiler to C code: \IT{Hex-Rays}:
\href{http://go.yurichev.com/17033}{hex-rays.com/products/decompiler/}

\section{Patch comparison/diffing}

You may want to use it when you compare original version of some executable and patched one, in order to find
what has been patched and why.

\begin{itemize}
\item (Free) \IT{zynamics BinDiff}\footnote{\url{https://www.zynamics.com/software.html}}

\item (Free, open-source) \IT{Diaphora}\footnote{\url{https://github.com/joxeankoret/diaphora}}
\end{itemize}

\chapter{Live analysis}

Tools you use on a live system or during running of a process.

\section{Debuggers}

\myindex{\olly}
\myindex{Radare}
\myindex{GDB}
\myindex{tracer}
\myindex{LLDB}
\myindex{WinDbg}

\begin{itemize}
\item (Free) \IT{OllyDbg}.
Very popular user-mode win32 debugger\footnote{\href{http://go.yurichev.com/17032}{ollydbg.de}}.
\ShortHotKeyCheatsheet: \myref{sec:Olly_cheatsheet}

\item (Free, open-source) \IT{GDB}.
Not quite popular debugger among reverse engineers, becase it's intended mostly for programmers.
Some commands: \myref{sec:GDB_cheatsheet}.
There is a visual interface for GDB, ``GDB dashboard''\footnote{\url{https://github.com/cyrus-and/gdb-dashboard}}.

\item (Free, open-source) \IT{LLDB}\footnote{\url{http://lldb.llvm.org/}}.

\item \IT{WinDbg}\footnote{\url{https://developer.microsoft.com/en-us/windows/hardware/windows-driver-kit}}:
kernel debugger for Windows.

\item (Free, open-source) \IT{Radare} \ac{AKA} rada.re \ac{AKA} r2\footnote{\url{http://rada.re/r/}}.
A GUI also exists: \IT{ragui}\footnote{\url{http://radare.org/ragui/}}.

\item (Free, open-source) \IT{tracer}.
\label{tracer}
The author often uses \IT{tracer}
\footnote{\href{http://go.yurichev.com/17338}{yurichev.com}}
instead of a debugger.

The author of these lines stopped using a debugger eventually, since all he needs from it is to spot function arguments while
executing, or registers state at some point.
Loading a debugger each time is too much, so a small utility called \IT{tracer} was born.
It works from command line, allows intercepting function execution,
setting breakpoints at arbitrary places, reading and changing registers state, etc.

N.B.: the \IT{tracer} isn't evolving, because it was developed as a demonstration tool for this book, not as everyday tool.
\end{itemize}

\section{Library calls tracing}

\IT{ltrace}\footnote{\url{http://www.ltrace.org/}}.

\section{System calls tracing}

\label{strace}
\myindex{strace}
\myindex{dtruss}
\subsection{strace / dtruss}

\myindex{syscall}
It shows which system calls (syscalls(\myref{syscalls})) are called by a process right now.

For example:

\begin{lstlisting}
# strace df -h

...

access("/etc/ld.so.nohwcap", F_OK)      = -1 ENOENT (No such file or directory)
open("/lib/i386-linux-gnu/libc.so.6", O_RDONLY|O_CLOEXEC) = 3
read(3, "\177ELF\1\1\1\0\0\0\0\0\0\0\0\0\3\0\3\0\1\0\0\0\220\232\1\0004\0\0\0"..., 512) = 512
fstat64(3, {st_mode=S_IFREG|0755, st_size=1770984, ...}) = 0
mmap2(NULL, 1780508, PROT_READ|PROT_EXEC, MAP_PRIVATE|MAP_DENYWRITE, 3, 0) = 0xb75b3000
\end{lstlisting}

\myindex{\MacOSX}
\MacOSX has dtruss for doing the same.

\myindex{Cygwin}
Cygwin also has strace, but as far as it's known, it works only for .exe-files
compiled for the cygwin environment itself.

\section{Network sniffing}

\IT{Sniffing} is intercepting some information you may be interested in.

(Free, open-source) \IT{Wireshark}\footnote{\url{https://www.wireshark.org/}} for network sniffing.
It has also capability for USB sniffing\footnote{\url{https://wiki.wireshark.org/CaptureSetup/USB}}.

Wireshark has a younger (or older) brother \IT{tcpdump}\footnote{\url{http://www.tcpdump.org/}}, simpler command-line tool.

\section{Sysinternals}

\myindex{Sysinternals}
(Free) Sysinternals (developed by Mark Russinovich)
\footnote{\url{https://technet.microsoft.com/en-us/sysinternals/bb842062}}.
At least these tools are imporant and worth studying: Process Explorer, Handle, VMMap, TCPView, Process Monitor.

\section{Valgrind}

(Free, open-source) a power tool for detecting memory leaks: \url{http://valgrind.org/}.
Due to its powerful \ac{JIT} mechanism, Valgrind is used as a framework for other tools.

% TODO network fuzzing

\section{Emulators}

\begin{itemize}
\item (Free, open-source) \IT{QEMU}\footnote{\url{http://qemu.org}}: emulator for various CPUs and architectures.

\item (Free, open-source) \IT{DosBox}\footnote{\url{https://www.dosbox.com/}}: MS-DOS emulator, mostly used for retrogaming.

\item (Free, open-source) \IT{SimH}\footnote{\url{http://simh.trailing-edge.com/}}: emulator of ancient computers, mainframes, etc.
\end{itemize}

\chapter{Other tools}

\IT{Microsoft Visual Studio Express}
\footnote{\href{http://go.yurichev.com/17034}{visualstudio.com/en-US/products/visual-studio-express-vs}}:
Stripped-down free version of Visual Studio, convenient for simple experiments.

Some useful options: \myref{sec:MSVC_options}.

\chapter{Something missing here?}

If you know a great tool not listed here, please drop me a note: \TT{\EMAIL}.

